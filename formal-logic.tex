%%%%%%%%%%%%%%%%%%%%%%%%%%%%%%%%%%%%%%%%%%%%%%%%%%%%%%%%%%%%%%%%%%%%%%

\documentclass{article}

%%%%%%%%%%%%%%%%%%%%%%%%%%%%%%%%%%%%%%%%%%%%%%%%%%%%%%%%%%%%%%%%%%%%%%

\begin{document}

\setcounter{secnumdepth}{5}
\setcounter{tocdepth}{5}

% --------------------------------------------------------------------

\title{Mathematical Logic}
\date{}
\maketitle

% --------------------------------------------------------------------

\tableofcontents

% --------------------------------------------------------------------

\part{Formal Systems}

\section{Formal Language Syntax}

The $Syntactic$ structure of a $Language$ as defined by a $Grammar$ is
outlined below.

A $Formal$ $Language$, $L$, is a possibly infinite subset of an
infinite $Vocabulary$, $\Sigma^*$, that is the set of all possible
finite $Symbols$ over a possibly infinite $Alphabet$, $\Sigma$, whose
members will be called $Glyphs$ to differentiate from $Symbols$. This
set $\Sigma^*$ is the Kleene star or Free monoid of $\Sigma$; the
smallest superset of $\Sigma$ that is closed under string
concatenation.

The $Syntax$ is that part of the $Language$ that refers only to the
literal strings of characters or $Symbols$ of the $Language$ with no
regard to their meaning or interpretation; only the condition that
they can be identified and differentiated from one-another is
required.

The entire content of a $Language$ is uniquely determined by the set
of all $Terminal$ strings generated by the $Production$ or rewrite
rules of a $Grammar$. This possibly infinite set of $Terminals$ will
be a subset of the $Vocabulary$ over an $Alphabet$.

\subsection{Lexical Elements}

The definition of a $Grammar$ and hence a $Language$ determined by it
will not involve any units smaller than the $Symbol$ and because of
the finite size of the $Alphabet$ available in most situations,
$Symbols$ are composed of 0 or more of the $Glyphs$ that uniquely
determine the $Alphabet$. An alternative to the finite $Alphabet$ may
be achieved by indexed subscripting. $Alphabets$ of $Glyphs$ will be
termed '$Lexical$' to differentiate from $Syntactic$ $Elements$
($Vocabularies$ of $Symbols$).

    \begin{description}

    \item[Glyph] \hfill \\
    a single character or letter

    \item[Alphabet ($\Sigma$)] \hfill \\
    a possibly infinite set of all $Glyphs$ used to compose the
    $Symbols$ of a $Language$

    \end{description}

\subsection{Syntactic Elements}

Within a $Language$ defined by a $Grammar$ over a given $Vocabulary$,
the $Symbols$ will be divided into two disjoint subsets according to
whether they are $Terminal$ or $Non-terminal$ $Symbols$.

The definition of a $Non-terminal$ $Symbol$ is one for which a
$Production$ rule exists in the $Grammar$ with that $Symbol$ as the
input $Expression$ to the $Production$. Thus a $Grammar$ is specified
by a finite set of $Productions$, $P$, a finite set of $Non-terminal$
$Symbols$, $N$, and a finite set of $Terminal$ $Symbols$,
$T$. Additionally, in certain $Grammars$ it is allowed for multiple
$Non-terminals$ to appear in an $Expression$, that is one or more
$Symbols$ taken together. However, since the $Vocabulary$ is a closed
set under string concatenation, the set of $Symbols$ and $Expressions$
will be identical by members.

    \begin{description}

    \item[Symbol] \hfill \\
    a word consisting of a finite string (ordered set) of $Glyphs$
    over an $Alphabet$

    \item[Expression] \hfill \\
    a finite string of $Symbols$

    \item[Vocabulary ($\Sigma^{*}$)] \hfill \\
    set of all $Symbols$ and $Expressions$ over an $Alphabet$

    \item[Production] \hfill \\
    a rule specifying a $Non-terminal$ $Symbol$ substitution

    \item[Grammar] \hfill \\
    a finite set of $Productions$ over the $Symbols$ of a $Vocabulary$

    \end{description}

\paragraph{Special Symbols}

Two special $Symbols$ are recognized:

    \begin{description}

    \item[Empty Symbol ($\varepsilon$)] \hfill \\
    the $Symbol$ of zero length and a $Terminal$ $Symbol$

    \item[Start Symbol ($S$)] \hfill \\
    a unique $Non-terminal$ $Symbol$

    \end{description}

A $Grammar$ $generates$ a $Language$ by the repeated application
of its $Production$ $Rules$ beginning with the $Start$ $Symbol$. A
sequence of rule applications is a $Derivation$.

Formal definition of a $Grammar$ as a 4-tuple:
\[
    G(N,T,P,S)
\]

The unrestricted form of a $Production$:
\[
    (N \cup T)^*N(N \cup T)^* \rightarrow (N \cup T)^*
\]
That is, a function from one $Expression$ to another, where the left
$Expression$ must contain at least one $Non-terminal$ $Symbol$. By
convention, $Non-terminal$ $Symbols$ will be denoted by capitals
($A,B,C,\cdots$), and $Terminals$ by lowercase ($a,b,c,\cdots$), and
expressions by Greek letters ($\alpha,\beta,\gamma$).

Let:

\[
    \mathcal{A} = \{ Alphabets \},\: \mathcal{V} = \{ Vocabularies \}
\] \[
    \mathcal{G} = \{ Grammars \},\: \mathcal{L} = \{ Languages \}
\]

    \begin{description}

    \item Definition of the Kleene star or Free monoid over an
      $Alphabet$ where $++$ is the operation to $Concatenate$ two
      strings of $Glyphs$:
    \[
        \forall \: \Sigma \in \mathcal{A} \:
        \exists \: \Sigma^* \in \mathcal{V}
        : \Sigma^* = \bigcup_{i=0}^{|\Sigma|} \Sigma_i
        = (\Sigma,++)
    \]

    \item Definition of a $Language$ in terms of a $Vocabulary$:
    \[
        \forall \: L \in \mathcal{L} \:
        \exists \: \Sigma^* \in \mathcal{V}
        : L \subseteq \Sigma^*
    \]

    \item Existence of the $Empty$ $Symbol$, $\varepsilon$:
    \[
        \forall \: \Sigma^* \in \mathcal{V} \:
        \exists ! \: \varepsilon \in \Sigma^*
        : |\varepsilon|=0
    \]

    \end{description}

\subsection{Formal Grammars}

\subsubsection{Chomsky Hierarchy}

Possible $Grammars$ are classified by how restrictive the $Production$
rules are. By convention, They may be organized into a hierarchy of
sets under proper inclusion, where $Type-0$ is an unrestricted
grammar, covering all possible formal grammars.

\[
    Type-0 \supset Type-1 \supset Type-2 \supset Type-3
\]

 These different levels in the hierarchy are also $Recognizable$ by
 different formulations of $Automata$ described in the next section.

\paragraph{Type-0: Unrestricted}

    \subparagraph{Semi-decidable}
    $Production$ $rules$ of an $Unrestricted$ $Grammar$ have the form
    \[
        \alpha \rightarrow \beta
    \]
    where $\alpha$ and $\beta$ are $Expressions$ of $N \cup T$ and
    $\alpha \neq \varepsilon$.

    A completely unrestricted $Grammar$ is called $recursively$
    $enumerable$ or $Semi-decidable$. This means membership of the
    $Language$ can be decided by an algorithm, but non-membership
    cannot, and the class of $Languages$ having this property is
    called $\mathsf{RE}$. Members of this class are also $Diophantine$
    sets and the lattice of $\mathsf{RE}$ sets under inclusion is
    written $\mathcal{E}$.

    The complement of $\mathsf{RE}$ is the class of $Languages$ for
    which an algorithm may decide non-membership only and is termed
    $\mathsf{coRE}$. The class of automata capable of implementing
    these algorithms is the $Turing$ $Machine$.

    \subparagraph{Decidable}
    A $recursive$ (as opposed to $recursively$ $enumerable$) or
    $Decidable$ $Language$ is defined as the intersection of
    $\mathsf{RE}$ and $\mathsf{coRE}$:
    \[
        \mathsf{R} = \mathsf{RE} \cap \mathsf{coRE}
    \]
    That is, it can be decided whether a $Symbol$ is a member or not
    by a $total$ $computable$ $function$ (one which returns $True$ or
    $False$ depending on membership). $Decidable$ $Languages$ are
    recognizable by a $decider$ or $Total$ $Turing$ $Machine$.

\paragraph{Type-1: Context-sensitive}

    \subparagraph{Context-sensitive}
    $Context-sensitive$ $Grammars$ have the restriction that the
    result of a $Production$ is not shorter than the input. Formally
    stated $Productions$ are of the form
    \[
        \alpha A \beta \rightarrow \alpha \gamma \beta
    \]
    where $|A| => |\gamma|$. In this formulation $\alpha$ and $\beta$
    form the $Context$ of $A$.

    Requiring that $S$ does not appear on the right of any
    $Production$ and allowing the rule
    \[
        S \rightarrow \varepsilon
    \]
    makes the $Context-sensitive$ $Languages$ a proper superset of the
    $Context-free$ $Languages$.

    $Context-sensitive$ $Languages$ are equivalent to a $Linear$
    $Bounded$ $Automaton$ (a linear bounded non-deterministic $Turing$
    $Machine$).

    \subparagraph{Indexed}

    An $Indexed$ $Grammar$ has an extra set of $Symbols$ $F$, the
    $Indexed$ $Symbols$. $Productions$ are of three possible forms,
    \[
        A[\sigma] \rightarrow \alpha[\sigma]
    \]\[
        A[\sigma] \rightarrow B[f\sigma]
    \]\[
        A[f\sigma] \rightarrow \alpha[\sigma]
    \]
    where $f \in F$ and $\sigma$ is a string of $Index$ $Symbols$.

    In effect these are used to form a $stack$ by the $Production$
    rules where $Indices$ are either pushed or popped from the stack.

    A $Indexed$ $Language$ can be recognized by a $Nested$ $Stack$
    $Automaton$.

    \subparagraph{Generalized Contex-free}
    A $Generalized$ $Context-free$ $Grammar$ adds to the rewrite rules
    of a $Context-free$ $Grammar$ a set of non-context-free
    $composition$ $functions$ that combine tuples of symbols:
    \[
        f(\langle x_1,\cdots,x_m\rangle,\cdots,\langle
        y_1,\cdots,y_n\rangle)=\gamma
    \]
    where $\gamma$ is a single tuple or another composition function
    that reduces to a single tuple.

    Rules are of the form:
    \[
        A \rightarrow f(X,Y,\cdots)
    \]
    where $X$,$Y$,$\cdots$ are string tuples or $Non-terminal$ $Symbols$.

    There are several weakly equivalent to the composition formulation.

    \begin{description}
    \item[Linear context-free rewriting system]
    Weakly equivalent to $multi-component$ $Tree-adjoining$ $Grammars$
    where composition functions are both $linear$ and $regular$. Can
    be recognized by $Thread$ $Automata$.
    \item[Tree-adjoining]
    Elementary rewriting unit is a tree rather than a $Symbol$. Can be
    recognized by $Embedded$ $pushdown$ $automata$.
    \item[Linear indexed grammar]
    A modified $Indexed$ $Grammar$ where only one symbol receives the stack.
    \item[Combinatory Categorical Grammar]
    A type of $phrase$ $structure$ $Grammar$ using $Combinatory$ $Logic$.
    \item[Head grammar]
    A subset of the $Linear$ $context-free$ $rewriting$ $system$ and a
    $phrase$ $structure$ $Grammar$.
    \end{description}

\paragraph{Type-2: Context-free}

    \begin{description}

    \item[Context-free]

    \item[Deterministic]

    \item[Visibly pushdown]

    \end{description}

\paragraph{Type-3: Regular}

    \begin{description}

    \item[Regular]

    \item[Star-free]

    \end{description}

\subsubsection{Analytic Grammars}

\subsubsection{Automata}

\subsection{Deductive Apparatus}

\subsubsection{Axiom Schemata}

\subsubsection{Inference Rules}

\section{Logic Systems}

A $Formal$ $Language$ in combination with a $Deductive$ $Apparatus$
gives rise to a $Logic$ $System$.

\subsection{Zeroth-order - Propositional}

\subsection{First-order - Predicate}

\subsection{Second-order - Plural}

\subsection{Higher-order}

\subsection{Algebraic Logic}


% --------------------------------------------------------------------

\part{Proof Theory}

% --------------------------------------------------------------------

\part{Model Theory}

% --------------------------------------------------------------------

\part{Formal Semantics}

% --------------------------------------------------------------------

\part{Set Theory}

% --------------------------------------------------------------------

\part{Graph Theory}

% --------------------------------------------------------------------

\part{Category Theory}

% --------------------------------------------------------------------

\end{document}

%%%%%%%%%%%%%%%%%%%%%%%%%%%%%%%%%%%%%%%%%%%%%%%%%%%%%%%%%%%%%%%%%%%%%%
