%%%%%%%%%%%%%%%%%%%%%%%%%%%%%%%%%%%%%%%%%%%%%%%%%%%%%%%%%%%%%%%%%%%%%%

\documentclass{article}

%%%%%%%%%%%%%%%%%%%%%%%%%%%%%%%%%%%%%%%%%%%%%%%%%%%%%%%%%%%%%%%%%%%%%%

\begin{document}

% --------------------------------------------------------------------

\title{Formal Logic}
\date{}
\maketitle

% --------------------------------------------------------------------

\tableofcontents

% --------------------------------------------------------------------

\part{Formal Language}

A $Formal$ $Language$, $L$, is a subset of a $Vocabulary$, $\Sigma^*$,
that is the set of all $Symbols$ over an $Alphabet$, $\Sigma$. The
structure of this $Syntax$ is outlined below.

\section{Syntax}

The $Syntax$ is that part of the $Language$ that refers only to the
literal strings of characters or symbols of the $Language$ with no
regard to their meaning or interpretation; only the fact that they can
be identified and differentiated from one-another is considered. The
entire contents of a $Language$ is determined by the set of all
terminal strings generated by a $Grammar$. This will be a subset of
the $Vocabulary$ of an $Alphabet$. The $Lexical$ $Elements$
making up this $Grammar$ are defined below.

    \subsection{Lexical Elements}

    \begin{description}

    \item[Glyph]

    a single character or letter

    \item[Alphabet ($\Sigma$)]

    set of all $Glyphs$ used in the $Syntax$ of the $Language$, not
    necessarily finite

    \item[Symbol]

    a word or finite string of $Glyphs$ over an $Alphabet$ (each
    $Alphabet$ has a $Symbol$ of length 0 called the $empty$ $symbol$
    denoted by $\varepsilon$

    \item[Vocabulary ($\Sigma^{*}$)]

    set of all $Symbols$ over an $Alphabet$

    \item[Production]

    a rule specifying a $Symbol$ substitution

    \end{description}

\subsection{Grammar}

The $Grammar$ of a $Language$ is a set of $Productions$ over the
$Symbols$ of a $Alphabet$.

\section{Logic (Semantics)}

% --------------------------------------------------------------------

\part{Proof Theory}

% --------------------------------------------------------------------

\part{Model Theory}

% --------------------------------------------------------------------

\part{Set Theory}

% --------------------------------------------------------------------

\part{Graph Theory}

% --------------------------------------------------------------------

\part{Category Theory}

% --------------------------------------------------------------------

\end{document}

%%%%%%%%%%%%%%%%%%%%%%%%%%%%%%%%%%%%%%%%%%%%%%%%%%%%%%%%%%%%%%%%%%%%%%
