%%%%%%%%%%%%%%%%%%%%%%%%%%%%%%%%%%%%%%%%%%%%%%%%%%%%%%%%%%%%%%%%%%%%%%

\documentclass{article}

%%%%%%%%%%%%%%%%%%%%%%%%%%%%%%%%%%%%%%%%%%%%%%%%%%%%%%%%%%%%%%%%%%%%%%

\begin{document}

\setcounter{secnumdepth}{5}
\setcounter{tocdepth}{5}

% --------------------------------------------------------------------

\title{Mathematical Logic}
\date{}
\maketitle

% --------------------------------------------------------------------

\tableofcontents

% --------------------------------------------------------------------

\part{Formal Systems}

\section{Formal Language Syntax}

A $Formal$ $Language$, $L$, is a possibly infinite subset of an
infinite $Vocabulary$, $\Sigma^*$, that is the set of all possible
finite $Symbols$ over an $Alphabet$ of $Glyphs$, $\Sigma$ (ie
$\Sigma^*$ is the Kleene star or Free monoid of $\Sigma$). The
$Syntactic$ structure of a $Language$ as defined by a $Grammar$ is
outlined below.

The $Syntax$ is that part of the $Language$ that refers only to the
literal strings of characters or $Symbols$ of the $Language$ with no
regard to their meaning or interpretation; only the condition that
they can be identified and differentiated from one-another is
required.

The entire content of a $Language$ is uniquely determined by the set
of all $Terminal$ strings generated by the rules of a $Grammar$. This
possibly infinite set will be a subset of the $Vocabulary$ over an
$Alphabet$.

\subsection{Lexical Elements}

The definition of a $Grammar$ and hence a $Language$ determined by it
will not involve any units smaller than the $Symbol$ and because of
the finite size of the $Alphabet$ available in most situations,
$Symbols$ are composed of 0 or more of the $Glyphs$ that uniquely
determine the $Alphabet$. $Alphabets$ of $Glyphs$ will be termed
'$Lexical$' to differentiate from $Syntactic$ $Elements$
($Vocabularies$ of $Symbols$).

    \begin{description}

    \item[Glyph] \hfill \\
    a single character or letter

    \item[Alphabet ($\Sigma$)] \hfill \\
    a possibly infinite set of all $Glyphs$ used to compose the
    $Symbols$ of a $Language$

    \end{description}

\subsection{Syntactic Elements}

Within a $Language$ defined by a $Grammar$ over a given $Vocabulary$,
the $Symbols$ will be divided into two disjoint subsets according to
whether they are $Terminal$ or $Non-terminal$ $Symbols$.

The definition of a $Non-terminal$ symbol is one for which a
$Production$ rule exists in the $Grammar$. Thus a $Grammar$ is
specified by a finite set of $Productions$, $P$, a finite set of
$Non-terminal$ $Symbols$, $N$, and a finite set of $Terminal$
$Symbols$, $T$.

    \begin{description}

    \item[Symbol] \hfill \\
    a word consisting of a finite string (ordered set) of $Glyphs$
    over an $Alphabet$

    \item[Vocabulary ($\Sigma^{*}$)] \hfill \\
    set of all $Symbols$ over an $Alphabet$

    \item[Expression] \hfill \\
    a finite ordered set of $Symbols$

    \item[Production] \hfill \\
    a rule specifying a $Non-terminal$ $Symbol$ substitution

    \item[Grammar] \hfill \\
    a finite set of $Productions$, $P$ over the $Symbols$ of a $Vocabulary$

    \end{description}

Two special $Symbols$ are recognized:

    \begin{description}

    \item[Empty Symbol ($\epsilon$)] \hfill \\
    the $Symbol$ of zero length and a $Terminal$ $Symbol$

    \item[Start Symbol ($S$)] \hfill \\
    a unique $Non-terminal$ $Symbol$

    \end{description}

A $Grammar$ $generates$ a $Language$ by the repeated application
of its $Production$ $Rules$ beginning with the $Start$ $Symbol$. A
sequence of rule applications is a $Derivation$.

The unrestricted form of a $Production$:
\[
    (N \cup T)^*N(N \cup T)^* \rightarrow (N \cup T)^*
\]
That is, a function from one $Expression$ to another, where the left
$Expression$ must contain at least one $Non-terminal$ $Symbol$.

Let:

\[
    \mathcal{A} = \{ Alphabets \},\: \mathcal{V} = \{ Vocabularies \}
\] \[
    \mathcal{G} = \{ Grammars \},\: \mathcal{L} = \{ Languages \}
\]

    \begin{description}

    \item Definition of the Kleene star or Free monoid over an
      $Alphabet$ where $++$ is the operation to $Concatenate$ two
      strings of $Glyphs$:
    \[
        \forall \: \Sigma \in \mathcal{A} \:
        \exists \: \Sigma^* \in \mathcal{V}
        : \Sigma^* = \bigcup_{i=0}^{|\Sigma|} \Sigma_i
        = (\Sigma,++)
    \]

    \item Definition of a $Language$ in terms of a $Vocabulary$:
    \[
        \forall \: L \in \mathcal{L} \:
        \exists \: \Sigma^* \in \mathcal{V}
        : L \subseteq \Sigma^*
    \]

    \item Existence of the $Empty$ $Symbol$, $\epsilon$:
    \[
        \forall \: \Sigma^* \in \mathcal{V} \:
        \exists ! \: \epsilon \in \Sigma^*
        : |\epsilon|=0
    \]

    \end{description}

\subsection{Formal Grammars}

\subsubsection{Chomsky Hierarchy}

Possible $Grammars$ are classified by how restrictive the $Production$
rules are. They may be organized into a hierarchy of sets under proper
inclusion, where $Type-0$ is an unrestricted grammar, covering all
possible formal grammars:

\[
    Type-0 \supset Type-1 \supset Type-2 \supset Type-3
\]

\paragraph{Type-0: Unrestricted}

    \subparagraph{Recursive}
    A completely unrestricted $Grammar$ is called $Recursively$ $enumerable$.

    \subparagraph{Decidable}

\paragraph{Type-1: Context-sensitive}

    \begin{description}

    \item[Context-sensitive]

    \item[Indexed]

    \item[Linear context-free]

    \item[Tree-adjoining]

    \end{description}

\paragraph{Type-2: Context-free}

    \begin{description}

    \item[Context-free]

    \item[Deterministic]

    \item[Visibly pushdown]

    \end{description}

\paragraph{Type-3: Regular}

    \begin{description}

    \item[Regular]

    \item[Star-free]

    \end{description}

\subsubsection{Analytic Grammars}

\subsubsection{Automata}

\subsection{Deductive Apparatus}

\subsubsection{Axiom Schemata}

\subsubsection{Inference Rules}

\section{Logic Systems}

A $Formal$ $Language$ in combination with a $Deductive$ $Apparatus$
gives rise to a $Logic$ $System$.

\subsection{Zeroth-order - Propositional}

\subsection{First-order - Predicate}

\subsection{Second-order - Plural}

\subsection{Higher-order}

\subsection{Algebraic Logic}


% --------------------------------------------------------------------

\part{Proof Theory}

% --------------------------------------------------------------------

\part{Model Theory}

% --------------------------------------------------------------------

\part{Formal Semantics}

% --------------------------------------------------------------------

\part{Set Theory}

% --------------------------------------------------------------------

\part{Graph Theory}

% --------------------------------------------------------------------

\part{Category Theory}

% --------------------------------------------------------------------

\end{document}

%%%%%%%%%%%%%%%%%%%%%%%%%%%%%%%%%%%%%%%%%%%%%%%%%%%%%%%%%%%%%%%%%%%%%%
