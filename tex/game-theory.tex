%%%%%%%%%%%%%%%%%%%%%%%%%%%%%%%%%%%%%%%%%%%%%%%%%%%%%%%%%%%%%%%%%%%%%%%%%%%%%%%%
%%%%%%%%%%%%%%%%%%%%%%%%%%%%%%%%%%%%%%%%%%%%%%%%%%%%%%%%%%%%%%%%%%%%%%%%%%%%%%%%
\part{Game Theory}\label{sec:game_theory}
%%%%%%%%%%%%%%%%%%%%%%%%%%%%%%%%%%%%%%%%%%%%%%%%%%%%%%%%%%%%%%%%%%%%%%%%%%%%%%%%
%%%%%%%%%%%%%%%%%%%%%%%%%%%%%%%%%%%%%%%%%%%%%%%%%%%%%%%%%%%%%%%%%%%%%%%%%%%%%%%%

(\url{https://www.youtube.com/watch?v=1swEiLHKxZA}) -- three Theories of Games:
\begin{itemize}
  \item Conway Games
  \item von Neumann/Nash Games
  \item Hyland/Abramsky Games -- Game Semantics (\S\ref{sec:game_semantics})
\end{itemize}
(TODO: xrefs)

2016 - Abramsky - \emph{Information, Processes and Games}

2019 - Shafer, Vovk
- \emph{Game-Theoretic Foundations for Probability and Finance}



% ==============================================================================
\section{Utility Function}\label{sec:utility_function}
% ==============================================================================

%FIXME: xref optimization objective function ???

\begin{itemize}
  \item Supermodular Function (\S\ref{sec:supermodular_set_function}) --
    Complementary Goods
\end{itemize}



% ==============================================================================
\section{Intelligent Agent}\label{sec:intelligent_agent}
% ==============================================================================

%FIXME: move section ???

% ------------------------------------------------------------------------------
\subsection{Percept}\label{sec:percept}
% ------------------------------------------------------------------------------

Input to an Intelligent Agent



% ------------------------------------------------------------------------------
\subsection{Action}\label{sec:agent_action}
% ------------------------------------------------------------------------------

Output of an Intelligent Agent



% ------------------------------------------------------------------------------
\subsection{Autonomous Agent}\label{sec:autonomous_agent}
% ------------------------------------------------------------------------------

\subsubsection{Agent-based Model}\label{sec:agent_based_model}



% ==============================================================================
\section{Game}\label{sec:game}
% ==============================================================================

elements of a \emph{Game}:
\begin{itemize}
  \item \emph{Players}
  \item \emph{Information} and \emph{Actions} available to each Player at each
    \emph{Decision Point}
  \item \emph{Payoffs} for each \emph{Outcome}
\end{itemize}
the Elements together with a \emph{Solution Concept}
(\S\ref{sec:solution_concept}) can be used to Deduce a Set of \emph{Equilibrium
  Strategies}

(TODO: clarify)

\fist Generative Adversarial Networks (GANs \S\ref{sec:gan}): a pair of Neural
  Networks that contest with each other in a Game



% ------------------------------------------------------------------------------
\subsection{Cooperative Game}\label{sec:cooperative_game}
% ------------------------------------------------------------------------------

Bondareva-Shapely Theorem



\subsubsection{Characteristic Function Form Game}
\label{sec:characteristic_function_form}

cf. Characteristic Function (Random Variables
\S\ref{sec:characteristic_function})



\subsubsection{Majority Game}\label{sec:majority_game}



% ------------------------------------------------------------------------------
\subsection{Non-cooperative Game}\label{sec:noncooperative_game}
% ------------------------------------------------------------------------------

\subsubsection{Extensive Form Game}\label{sec:extensive_form_game}

\subsubsection{Normal-form Game}\label{sec:normal_form_game}



% ------------------------------------------------------------------------------
\subsection{Succinct Game}\label{sec:succinct_game}
% ------------------------------------------------------------------------------

% ------------------------------------------------------------------------------
\subsection{Infinite Game}\label{sec:infinite_game}
% ------------------------------------------------------------------------------

% ------------------------------------------------------------------------------
\subsection{Subgame}\label{sec:subgame}
% ------------------------------------------------------------------------------

any ``part'' (Subset) of a Game that when seen in isolation constitutes a Game
in its own right

Subgame Perfect Equilibrium (\S\ref{sec:subgame_perfect})



% ==============================================================================
\section{Determinacy}\label{sec:determinacy}
% ==============================================================================

examines conditions under which one or other player has a winning Strategy



% ==============================================================================
\section{Decision Theory}\label{sec:decision_theory}
% ==============================================================================

(wiki):

concerned with choices of individual agents

\emph{Normative Decision Theory} -- decisions based on a given Set of uncertain
``beliefs'' and a Set of ``values''

\emph{Descriptive Decision Theory} -- analyzes decisions of existing agents

\emph{Ludic Fallacy} -- Decision Theory considers ``known unknowns'' but not
``unknown unknowns''

\emph{Stein's Paradox} -- when three or more Parameters are Estimated
simultaneously, there exist combined Estimators more accurate on average (having
lower Expected MSE) than any method handling the Parameters Separately

(Wasserman04, Ch.12)

choosing \emph{Statistical Estimators} (\S\ref{sec:estimation_theory})

Loss Functions (\S\ref{sec:objective_function}):
\begin{itemize}
  \item Squared Error
  \item Absolute Error
  \item $L_p$ Loss
  \item Zero-one Loss
  \item Kullback-Leibler Loss
\end{itemize}
(TODO: xrefs)

\fist Probability Theory (Part \ref{part:probability_theory}) -- (Savage54,
Jeffrey66) \emph{Expected Utility} representation of Subjective Probability:
``rational choice'' maximizes Expected Utility; these accounts presuppose a
connection between ``desire-like states'' and ``belief-like states'' rendered
explicit in the connections between Preferences and Probabilities
(\url{https://plato.stanford.edu/entries/probability-interpret/#SubPro})

\fist Uncertainty Analysis (\S\ref{sec:uncertainty_analysis})

\fist Predictive Inference (\S\ref{sec:predictive_inference})

(wiki) alternatives to the use of \emph{Probability} in Decision Theory:
\begin{itemize}
  \item Dempster-Shafer Theory (Evidence Theory \S\ref{sec:evidence_theory})
  \item Fuzzy Logic (\S\ref{sec:fuzzy_logic}), Possibility Theory
    (\S\ref{sec:possibility_theory})
  \item Info-gap Decision Theory (\S\ref{sec:info_gap})
  \item Quantum Theory (TODO: xref)
\end{itemize}
Probabilistic Decision Theory is Sensitive (\S\ref{sec:sensitivity_analysis}) to
\emph{Assumptions} about Probabilities of Events; Non-probabilistic Decision
Rules (\S\ref{sec:decision_rule}), such as Minimax (\S\ref{sec:minimax}), are
\emph{Robust} (\S\ref{sec:robust_statistic}) in that they don't make such
Assumptions (FIXME: clarify)



% ------------------------------------------------------------------------------
\subsection{Decision Rule}\label{sec:decision_rule}
% ------------------------------------------------------------------------------

Statistical Estimator (\S\ref{sec:estimator})

the possible values of a Decision Rule are called \emph{Actions}

uses an Optimality Criterion (\S\ref{sec:optimality_criterion}) on an Objective
Function to make a Choice

\begin{itemize}
  \item Bayes Estimator (\S\ref{sec:bayes_estimator})
\end{itemize}



\subsubsection{Admissable Decision Rule}\label{sec:admissable}

(wiki)

a Decision Rule for which there is no other Rule that is \emph{always}
``better''

cf. Pareto Optimality (\S\ref{sec:pareto_optimal})

\emph{Bayes Rules} (Estimators that minimize Bayes Risk) are Admissable -- every
unique Bayesian (\S\ref{sec:bayesian_inference}) ``Procedure'' (cf. Bayes
Estimator \S\ref{sec:bayes_estimator}) is Admissable and every Admissable
Statistical Procedure is either a Bayesian Procedure or a Limit of Bayesian
Procedures (Wald)

Posterior Mean is Admissable for any Strictly Positive Prior

if $\hat{\theta}$ has constant Risk (\S\ref{sec:risk}) and is Admissible, then
it is Minimax (\S\ref{sec:minimax});
Minimax Rules are ``close'' to Admissable: if $\hat{\theta}$ is Minimax then it
is not Strongly Inadmissable



\subsubsection{Minimax}\label{sec:minimax}

Non-probabilistic Decision Rule

Bayes Estimators (\S\ref{sec:bayes_estimator}) with constant Risk Function are
Minimax

for Parametric Models satisfying ``Regularity Conditions'', Mean Likelihood
Estimate (\S\ref{sec:mle}) is approximately Minimax

if $\hat{\theta}$ has constant Risk (\S\ref{sec:risk}) and is Admissible
(\S\ref{sec:admissable}), then it is Minimax;
Minimax Rules are ``close'' to Admissable: if $\hat{\theta}$ is Minimax then it
is not Strongly Inadmissable

Minimax Theorem

\fist Generative Adversarial Networks (GANs \S\ref{sec:gan})



% ------------------------------------------------------------------------------
\subsection{Risk}\label{sec:risk}
% ------------------------------------------------------------------------------

\emph{Average Loss} (\S\ref{sec:objective_function})

Expected Loss

cf. \emph{Uncertainty} (\S\ref{sec:uncertainty})

cf. \emph{Empirical Risk} (\S\ref{sec:empirical_risk}), \emph{Generalization
Error} (\emph{Out-of-sample Error} \S\ref{sec:generalization_error})

\[
  R(\theta,\hat{\theta}) = E_\theta\Big(L(\theta,\hat{\theta})\Big) =
    \int L(\theta,\hat{\theta}(x))f(x;\theta) dx
\]

when the Loss Function is Squared Error, Risk is MSE (\S\ref{sec:msd})

if $\hat{\theta}$ has constant Risk and is Admissible (\S\ref{sec:admissable}),
then it is Minimax (\S\ref{sec:minimax})

\emph{Maximum Risk} -- choosing Estimator $\hat{\theta}$ to minimize Maximum
Risk leads to Minimax Estimators

\emph{Bayes Risk} -- chooseing Estimator $\hat{\theta}$ to minimize Bayes Risk
leads to Bayes Estimators (\S\ref{sec:bayes_estimator})

the Expected Value of an ``undesirable'' Outcome (\emph{Absolute Risk}) or Loss
Function

\begin{itemize}
  \item Mean Integrated Squared Error (MISE \S\ref{sec:mise}) -- $L^2$ Risk
    Function \fist Density Estimation (\S\ref{sec:density_estimation}):
    $Risk = Bias^2 + Variance$
  \item ...
\end{itemize}



\subsubsection{Risk Function}\label{sec:risk_function}

\subsubsection{Risk Ratio}\label{sec:risk_ratio}

or \emph{Relative Risk}

cf. Odds Ratio (\S\ref{sec:odds_ratio})



\subsubsection{Risk Difference}\label{sec:risk_difference}

cf. Odds Ratio (\S\ref{sec:odds_ratio})



\subsubsection{Empirical Risk Minimization (ERM)}\label{sec:erm}

Probability Inequalities (Classification Error \S\ref{sec:classification_error})



% ------------------------------------------------------------------------------
\subsection{Regret}\label{sec:regret}
% ------------------------------------------------------------------------------

% ------------------------------------------------------------------------------
\subsection{Decision Tree}\label{sec:decision_tree}
% ------------------------------------------------------------------------------

``Hierarchical Axis-parallel Classifiers'' (\S\ref{sec:classification}) --
Partitioning of Covariate Space; paths from Root to Leaf represent
\emph{Classification Rules}

Impurity, Gini Index

highly Non-linear Classifier --
Bootstrap Aggregation (Bagging \S\ref{sec:bootstrap_aggregation}) to reduce
Variance



% ------------------------------------------------------------------------------
\subsection{Markov Decision Process}\label{sec:markov_decision}
% ------------------------------------------------------------------------------

Markov Process (\S\ref{sec:markov_process})



% ------------------------------------------------------------------------------
\subsection{Uncertainty Analysis}\label{sec:uncertainty_analysis}
% ------------------------------------------------------------------------------

\fist Robust Statistics (\S\ref{sec:robust_statistic})



\subsubsection{Uncertainty}\label{sec:uncertainty}

cf. Probability (\S\ref{sec:probability})

(\url{https://plato.stanford.edu/entries/logic-probability/}): Probabilistic
Semantics (\S\ref{sec:probabilistic_semantics}) for Logical Consequence Relation
yields \emph{Probability Preserving} (dually, \emph{Uncertainty Propagating})
Deductive Validity (\S\ref{sec:validity}), rather than Truth Preserving
(\S\ref{sec:truth_preservation})

\emph{Knightian Uncertainty}

cf. \emph{Risk} (\S\ref{sec:risk})

``the idea that all Uncertainty must be explainable in terms of Probability is a
wrong-way reduction'' --
(\url{https://confusopoly.com/2019/04/03/the-optimizers-curse-wrong-way-reductions/})



\paragraph{Propagation of Error}\label{sec:error_propagation}\hfill

or \emph{Propagation of Uncertainty}

\fist cf. Delta Method (\S\ref{sec:delta_method})



\subsubsection{Sensitivity Analysis}\label{sec:sensitivity_analysis}

or \emph{``What-if'' Analysis}

cf. Hypothetical Propositions (\S\ref{sec:proposition})

\fist Info-gap Decision Theory (\S\ref{sec:info_gap}): application of
Sensitivity Analysis of the Stability Radius (\S\ref{sec:stability_radius}) type
to Perturbations in the value of a given Estimate of a Parameter of interest
(FIXME: clarify)



\subsubsection{Evidince Theory}\label{sec:evidence_theory}

or \emph{Dempster-Shafer Theory (DST)} or \emph{Theory of Belief Functions}

Transferable Belief Model



\subsubsection{Possibility Theory}\label{sec:possibility_theory}

alternative to Probability Theory (Part \ref{part:probability_theory}) for
dealing with certain types of Uncertainty

Possibility Measure

Necessity Measure

cf. Alethic Logic (\S\ref{sec:alethic_logic})



% ------------------------------------------------------------------------------
\subsection{Info-gap Decision Theory}\label{sec:info_gap}
% ------------------------------------------------------------------------------

Non-probabilistic Decision Theory seeking to optimize \emph{Robustness}
(\S\ref{sec:robust_statistic}) to ``failure'' under severe Uncertainty
(\S\ref{sec:uncertainty_analysis})

application of Sensitivity Analysis (\S\ref{sec:sensitivity_analysis}) of the
Stability Radius (\S\ref{sec:stability_radius}) type to Perturbations in the
value of a given Estimate of a Parameter of interest (FIXME: clarify)



% ------------------------------------------------------------------------------
\subsection{Reinforcement Learning}\label{sec:reinforcement_learning}
% ------------------------------------------------------------------------------

learn to select an action to maximize ``payoff''

Markov Decision Process (\S\ref{sec:mdp})

Dynamic Programming (\S\ref{sec:dynamic_programming})

cf. other Learning Processes (ANNs \S\ref{sec:learning_process}):
\begin{itemize}
  \item Unsupervised Learning -- Dentity Estimation
    (\S\ref{sec:density_estimation}), Cluster Analysis
    (\S\ref{sec:cluster_analysis})
  \item Supervised Learning -- Classification (\S\ref{sec:classification}),
    Regression Analysis (\S\ref{sec:regression_analysis})
\end{itemize}

\emph{An Outsider's Tour of Reinforcement Learning} -
\url{http://www.argmin.net/2018/06/25/outsider-rl/}:

- \url{http://www.argmin.net/2018/01/29/taxonomy/} --
  Reinforcement Learning as a form of Predictive Analytics
  (\S\ref{sec:predictive_analytics})
- \url{http://www.argmin.net/2018/02/01/control-tour/} --
  Reinforcement Learning as Optimal Control (\S\ref{sec:optimal_control})



\subsubsection{Experience-Weighted Attraction (EWA)}\label{sec:ewa}

2013 - Galla, Farmer - \emph{Complex Dynamics in Learning Complicated Games}



% ------------------------------------------------------------------------------
\subsection{Markov Decision Process (MDP)}\label{sec:mdp}
% ------------------------------------------------------------------------------

Controlled (\S\ref{sec:control_theory}) Markov Model (\S\ref{sec:markov_model})



\subsubsection{Partially Observable Markov Decision Process (POMDP)}
\label{sec:pomdp}



% ==============================================================================
\section{Solution Concept}\label{sec:solution_concept}
% ==============================================================================

2009 - Pavlovic - \emph{A Semantical Approach to Equilibria and Rationality} --
``Game-theoretic Equilibria are mathematical expressions of Rationality''
%FIXME move ?



% ------------------------------------------------------------------------------
\subsection{Equilibrium Concept}\label{sec:equilibrium_concept}
% ------------------------------------------------------------------------------

\subsubsection{Pareto Optimal}\label{sec:pareto_optimal}

\fist cf. Admissable Decision Rules (\S\ref{sec:admissable})



\subsubsection{Correlated Equilibrium}\label{sec:correlated_equilibrium}

Correlated Equilibria

Aumann1987 - \emph{Correlated Equilibrium as an Expression of Bayesian
  Rationality} -- \emph{Correlated Equilibrium} ``does away with'' the
``dichotomy usually perceived'' between the \emph{Bayesian}
(\S\ref{sec:bayesian_inference}) and \emph{Game-theoretic} world-views



\paragraph{Nash Equilibrium}\label{sec:nash_equilibrium}\hfill

Nash Equilibria



\subparagraph{Subgame-perfect Equilibrium}\label{sec:subgame_perfect}\hfill

\emph{Subgame-perfect Nash Equilibrium}

Subgame (\S\ref{sec:subgame})

2020 - Hedges - \emph{Subgame perfection made difficult} -
\url{https://julesh.com/2020/05/26/subgame-perfection-made-difficult/}



\subparagraph{Evolutionarily Stable Strategy}
\label{sec:evolutionarily_stable}\hfill

ESS are stable states for a large class of ``\emph{Adaptive Dynamics}''
(\emph{Evolutionary Invasion Analysis} -- evolution of asexually reproducing
populations)

%FIXME: xref evolutionary game theory ???



% ==============================================================================
\section{Combinatorial Game Theory}\label{sec:combinatorial_game_theory}
% ==============================================================================

Combinatorial Game Theory - Conway  - \emph{On Numbers and Games}



% ------------------------------------------------------------------------------
\subsection{Poset Game}\label{sec:poset_game}
% ------------------------------------------------------------------------------

\subsubsection{Nim}\label{sec:nim}

\paragraph{Nimber}\label{sec:nimber}\hfill

$GF(2^n)$ (\S\ref{sec:finite_field})



% ==============================================================================
\section{Contract Theory}\label{sec:contract_theory}
% ==============================================================================

% ------------------------------------------------------------------------------
\subsection{Option Theory}\label{sec:option_theory}
% ------------------------------------------------------------------------------

\begin{quote}
  in a way Option Theory is ``Mathematicsl Contract Theory''
\end{quote}
(Taleb 2020)



% ==============================================================================
\section{Compositional Game Theory}\label{sec:compositional_game_theory}
% ==============================================================================

2015 - \emph{Coalgebraic Semantics of Reflexive Economics} -
\url{https://www.dagstuhl.de/no_cache/en/program/calendar/semhp/?semnr=15042} --
(FIXME)

2015 - Ghani, Hedges, Winschel, Zahn - \emph{Compositional Game Theory}

Emergent Behavior in Open Games --
2017 - \emph{On compositionality} -
\url{https://julesh.com/2017/04/22/on-compositionality/}

on the \emph{Non}-compositionality of Differential Equations:
\begin{quote}
  More specifically, I claim that compositionality is strictly necessary
  for working at scale. In a non-compositional setting, a technique for a
  solving a problem may be of no use whatsoever for solving the problem one
  order of magnitude larger. To demonstrate that this worst case scenario can
  actually happen, consider the theory of differential equations: a technique
  that is known to be effective for some class of equations will usually be of
  no use for equations removed from that class by even a small modification. In
  some sense, differential equations is the ultimate non-compositional theory.
\end{quote}

\url{https://julesh.com/2017/09/29/a-first-look-at-open-games/} \fist Open
Systems (\S\ref{sec:open_system})

\url{https://julesh.com/2017/11/09/compositional-game-theory-reading-list/}


\url{https://julesh.com/2018/01/16/towards-compositional-game-theory/}

\asterism

2020 - Hedges - \emph{Compositional Game Theory} -
\url{https://www.youtube.com/watch?v=5Qny8YmLUzk}

\emph{Graphical Language} -- two \emph{Composition Operators} on Games: ... TODO



% ------------------------------------------------------------------------------
\subsection{Event Structure}\label{sec:event_structure}
% ------------------------------------------------------------------------------

1989 - Winskel

2014 - Guiterrez, Woolridge - \emph{Equilibria of Concurrent Games on Event
  Structures}

``Event Structures form a canonical model of \emph{Concurrent Behavior} which
has a natural Game-theoretic Interpretation''



% ------------------------------------------------------------------------------
\subsection{Higher-order Sequential Game}\label{sec:higherorder_sequential_game}
% ------------------------------------------------------------------------------

2010 - Escard\'o, Oliva - \emph{Sequential Games and Optimal Strategies} -
\url{http://rspa.royalsocietypublishing.org/content/early/2010/11/26/rspa.2010.0471}

Simultaneous Games



% ------------------------------------------------------------------------------
\subsection{Open Game}\label{sec:open_game}
% ------------------------------------------------------------------------------

or ``\emph{Pregames}''

2014 - Hedges - \emph{String Diagrams for Game Theory: a (very) preliminary
  report} -- with Haskell code

2015 - Hedges - \emph{String Diagrams for Game Theory}

Open Games form a Symmetric Monoidal Category

Process Theory

Behavioral Control Theory (\S\ref{sec:behavioral_control})

Emergent Behavior in Open Games --
2017 - \emph{On compositionality} -
\url{https://julesh.com/2017/04/22/on-compositionality/}

\url{https://julesh.com/2018/04/02/the-pre-history-of-open-games/}

Event Structures (\S\ref{sec:event_structure})

2014 - Gutierrez, Wooldridge - \emph{Equilibria of Concurrent Games on Event
  Structures}



% ==============================================================================
\section{Mechanism Design}\label{sec:mechanism_design}
% ==============================================================================

or \emph{Reverse Game Theory}

Algorithmic Mechanism Design (\S\ref{sec:algorithmic_mechanism_design})



% ==============================================================================
\section{Scenario Analysis}\label{sec:scenario_analysis}
% ==============================================================================

based on \emph{hypothetical} Data

cf. Predictive Analytics (\S\ref{sec:predictive_analytics}): based on
\emph{historical} Data



% ==============================================================================
\section{Algorithmic Game Theory}\label{sec:algorithmic_game_theory}
% ==============================================================================

% ------------------------------------------------------------------------------
\subsection{Algorithmic Mechanism Design}
\label{sec:algorithmic_mechanism_design}
% ------------------------------------------------------------------------------



% ==============================================================================
\section{Evolutionary Game Theory}\label{sec:evolutionary_game_theory}
% ==============================================================================

cf. Genetic Algorithms (TODO: xref)



% ------------------------------------------------------------------------------
\subsection{Replicator Equation}\label{sec:replicator_equation}
% ------------------------------------------------------------------------------



% ==============================================================================
\section{Quantum Game Theory}\label{sec:quantum_game_theory}
% ==============================================================================

Non-local Games

2001 - Kobayashi, Masumoto - \emph{Quantum Multi-prover Interactive Proof
  Systems with Limited Prior Entanglement}

2010 - Cleve, Hoyer, Toner, Watrous -
\emph{Consequences and Limits of Nonlocal Strategies}

