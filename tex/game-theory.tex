%%%%%%%%%%%%%%%%%%%%%%%%%%%%%%%%%%%%%%%%%%%%%%%%%%%%%%%%%%%%%%%%%%%%%%%%%%%%%%%%
%%%%%%%%%%%%%%%%%%%%%%%%%%%%%%%%%%%%%%%%%%%%%%%%%%%%%%%%%%%%%%%%%%%%%%%%%%%%%%%%
\part{Game Theory}\label{sec:game_theory}
%%%%%%%%%%%%%%%%%%%%%%%%%%%%%%%%%%%%%%%%%%%%%%%%%%%%%%%%%%%%%%%%%%%%%%%%%%%%%%%%
%%%%%%%%%%%%%%%%%%%%%%%%%%%%%%%%%%%%%%%%%%%%%%%%%%%%%%%%%%%%%%%%%%%%%%%%%%%%%%%%

(\url{https://www.youtube.com/watch?v=1swEiLHKxZA}) -- three Theories of Games:
\begin{itemize}
  \item Conway Games
  \item von Neumann/Nash Games
  \item Hyland/Abramsky Games -- Game Semantics (\S\ref{sec:game_semantics})
\end{itemize}
(TODO: xrefs)



% ==============================================================================
\section{Utility Function}\label{sec:utility_function}
% ==============================================================================

\begin{itemize}
  \item Supermodular Function (\S\ref{sec:supermodular_set_function}) --
    Complementary Goods
\end{itemize}



% ==============================================================================
\section{Game}\label{sec:game}
% ==============================================================================

% ------------------------------------------------------------------------------
\subsection{Non-cooperative Game}\label{sec:noncooperative_game}
% ------------------------------------------------------------------------------

% ------------------------------------------------------------------------------
\subsection{Cooperative Game}\label{sec:cooperative_game}
% ------------------------------------------------------------------------------

Bondareva-Shapely Theorem



\subsubsection{Majority Game}\label{sec:majority_game}



% ------------------------------------------------------------------------------
\subsection{Infinite Game}\label{sec:infinite_game}
% ------------------------------------------------------------------------------



% ==============================================================================
\section{Determinacy}\label{sec:determinacy}
% ==============================================================================

% ==============================================================================
\section{Decision Theory}\label{sec:decision_theory}
% ==============================================================================

\fist Probability Theory (Part \ref{sec:probability_theory}) -- (Savage54,
Jeffrey66) \emph{Expected Utility} representation of Subjective Probability:
``rational choice'' maximizes Expected Utility; these accounts presuppose a
connection between ``desire-like states'' and ``belief-like states'' rendered
explicit in the connections between Preferences and Probabilities
(\url{https://plato.stanford.edu/entries/probability-interpret/#SubPro})

\fist Uncertainty Analysis (\S\ref{sec:uncertainty_analysis})

\fist Predictive Inference (\S\ref{sec:predictive_inference})

(wiki) alternatives to the use of \emph{Probability} in Decision Theory:
\begin{itemize}
  \item Dempster-Shafer Theory (Evidence Theory \S\ref{sec:evidence_theory})
  \item Fuzzy Logic (\S\ref{sec:fuzzy_logic}), Possibility Theory
    (\S\ref{sec:possibility_theory})
  \item Info-gap Decision Theory (\S\ref{sec:info_gap})
  \item Quantum Theory (TODO: xref)
\end{itemize}
Probabilistic Decision Theory is Sensitive (\S\ref{sec:sensitivity_analysis}) to
\emph{Assumptions} about Probabilities of Events; Non-probabilistic Decision
Rules (\S\ref{sec:decision_rule}), such as Minimax (\S\ref{sec:minimax}), are
\emph{Robust} (\S\ref{sec:robust_statistic}) in that they don't make such
Assumptions (FIXME: clarify)



% ------------------------------------------------------------------------------
\subsection{Decision Rule}\label{sec:decision_rule}
% ------------------------------------------------------------------------------

uses an Optimality Criterion (\S\ref{sec:optimality_criterion}) on an Objective
Function to make a Choice

\begin{itemize}
  \item Bayes Estimator (\S\ref{sec:bayes_estimator})
\end{itemize}



\subsubsection{Admissable Decision Rule}\label{sec:admissable_decision_rule}

(wiki)

a Decision Rule for which there is no other Rule that is \emph{always}
``better''

cf. Pareto Optimality (\S\ref{sec:pareto_optimal})

every unique Bayesian (\S\ref{sec:bayesian_inference}) ``Procedure'' (cf. Bayes
Estimator \S\ref{sec:bayes_estimator}) is Admissable and every Admissable
Statistical Procedure is either a Bayesian Procedure or a Limit of Bayesian
Procedures (Wald)



\subsubsection{Minimax}\label{sec:minimax}

Non-probabilistic Decision Rule

Minimax Theorem



% ------------------------------------------------------------------------------
\subsection{Info-gap Decision Theory}\label{sec:info_gap}
% ------------------------------------------------------------------------------

Non-probabilistic Decision Theory seeking to optimize \emph{Robustness}
(\S\ref{sec:robust_statistic}) to ``failure'' under severe Uncertainty
(\S\ref{sec:uncertainty_analysis})

application of Sensitivity Analysis (\S\ref{sec:sensitivity_analysis}) of the
Stability Radius (\S\ref{sec:stability_radius}) type to Perturbations in the
value of a given Estimate of a Parameter of interest (FIXME: clarify)



% ==============================================================================
\section{Solution Concept}\label{sec:solution_concept}
% ==============================================================================

2009 - Pavlovic - \emph{A Semantical Approach to Equilibria and Rationality} --
``Game-theoretic Equilibria are mathematical expressions of Rationality''
%FIXME move ?



% ------------------------------------------------------------------------------
\subsection{Equilibrium Concept}\label{sec:equilibrium_concept}
% ------------------------------------------------------------------------------

\subsubsection{Pareto Optimal}\label{sec:pareto_optimal}

\fist cf. Admissable Decision Rules (\S\ref{sec:admissable_decision})



\subsubsection{Correlated Equilibrium}\label{sec:correlated_equilibrium}

Correlated Equilibria

Aumann1987 - \emph{Correlated Equilibrium as an Expression of Bayesian
  Rationality} -- \emph{Correlated Equilibrium} ``does away with'' the
``dichotomy usually perceived'' between the \emph{Bayesian}
(\S\ref{sec:bayesian_inference}) and \emph{Game-theoretic} world-views



\paragraph{Nash Equilibrium}\label{sec:nash_equilibrium}\hfill

Nash Equilibria



\subparagraph{Subgame-perfect Equilibrium}\label{sec:subgame_perfect}\hfill

\emph{Subgame-perfect Nash Equilibrium}



\subparagraph{Evolutionarily Stable Strategy (ESS)}\label{sec:ess}\hfill

ESS are stable states for a large class of ``\emph{Adaptive Dynamics}''
(\emph{Evolutionary Invasion Analysis} -- evolution of asexually reproducing
populations)



% ==============================================================================
\section{Combinatorial Game Theory}\label{sec:combinatorial_game_theory}
% ==============================================================================

Combinatorial Game Theory - Conway  - \emph{On Numbers and Games}



% ==============================================================================
\section{Compositional Game Theory}\label{sec:compositional_game_theory}
% ==============================================================================

2015 - \emph{Coalgebraic Semantics of Reflexive Economics} -
\url{https://www.dagstuhl.de/no_cache/en/program/calendar/semhp/?semnr=15042} --
(FIXME)

2015 - Ghani, Hedges, Winschel, Zahn - \emph{Compositional Game Theory}

Emergent Behavior in Open Games --
2017 - \emph{On compositionality} -
\url{https://julesh.com/2017/04/22/on-compositionality/}

on the \emph{Non}-compositionality of Differential Equations:
\begin{quote}
  More specifically, I claim that compositionality is strictly necessary
  for working at scale. In a non-compositional setting, a technique for a
  solving a problem may be of no use whatsoever for solving the problem one
  order of magnitude larger. To demonstrate that this worst case scenario can
  actually happen, consider the theory of differential equations: a technique
  that is known to be effective for some class of equations will usually be of
  no use for equations removed from that class by even a small modification. In
  some sense, differential equations is the ultimate non-compositional theory.
\end{quote}

\url{https://julesh.com/2017/09/29/a-first-look-at-open-games/} \fist Open
Systems (\S\ref{sec:open_system})

\url{https://julesh.com/2017/11/09/compositional-game-theory-reading-list/}


\url{https://julesh.com/2018/01/16/towards-compositional-game-theory/}



% ------------------------------------------------------------------------------
\subsection{Event Structure}\label{sec:event_structure}
% ------------------------------------------------------------------------------

1989 - Winskel

2014 - Guiterrez, Woolridge - \emph{Equilibria of Concurrent Games on Event
  Structures}

``Event Structures form a canonical model of \emph{Concurrent Behavior} which
has a natural Game-theoretic Interpretation''



% ------------------------------------------------------------------------------
\subsection{Higher-order Sequential Game}\label{sec:higherorder_sequential_game}
% ------------------------------------------------------------------------------

2010 - Escard\'o, Oliva - \emph{Sequential Games and Optimal Strategies} -
\url{http://rspa.royalsocietypublishing.org/content/early/2010/11/26/rspa.2010.0471}

Simultaneous Games



% ------------------------------------------------------------------------------
\subsection{Open Game}\label{sec:open_game}
% ------------------------------------------------------------------------------

or ``\emph{Pregames}''

2014 - Hedges - \emph{String Diagrams for Game Theory: a (very) preliminary
  report} -- with Haskell code

2015 - Hedges - \emph{String Diagrams for Game Theory}

Open Games form a Symmetric Monoidal Category

Process Theory

Behavioral Control Theory (\S\ref{sec:behavioral_control_theory})

Emergent Behavior in Open Games --
2017 - \emph{On compositionality} -
\url{https://julesh.com/2017/04/22/on-compositionality/}

\url{https://julesh.com/2018/04/02/the-pre-history-of-open-games/}

Event Structures (\S\ref{sec:event_structure})

2014 - Gutierrez, Wooldridge - \emph{Equilibria of Concurrent Games on Event
  Structures}



% ==============================================================================
\section{Mechanism Design}\label{sec:mechanism_design}
% ==============================================================================

or \emph{Reverse Game Theory}
