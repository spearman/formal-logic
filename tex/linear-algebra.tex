%%%%%%%%%%%%%%%%%%%%%%%%%%%%%%%%%%%%%%%%%%%%%%%%%%%%%%%%%%%%%%%%%%%%%%
%%%%%%%%%%%%%%%%%%%%%%%%%%%%%%%%%%%%%%%%%%%%%%%%%%%%%%%%%%%%%%%%%%%%%%
\part{Linear Algebra}\label{sec:linear_algebra}
%%%%%%%%%%%%%%%%%%%%%%%%%%%%%%%%%%%%%%%%%%%%%%%%%%%%%%%%%%%%%%%%%%%%%%
%%%%%%%%%%%%%%%%%%%%%%%%%%%%%%%%%%%%%%%%%%%%%%%%%%%%%%%%%%%%%%%%%%%%%%

% ====================================================================
\section{Matrix Theory}\label{sec:matrix_theory}
% ====================================================================

% --------------------------------------------------------------------
\subsection{Matrix Multiplication}\label{sec:matrix_multiplication}
% --------------------------------------------------------------------

The Multiplication of an $m \times n$ Matrix $A$ with entries $a_{xy}$
on the left by an $n \times p$ Matrix $B$ with entries $b_{wv}$ on the
right results in an $m \times p$ Matrix $C$ with the entry $c_{ij}$
defined as:
\[
  \sum_{k=1}^n a_{ik} b_{kj}
\]
The columns of the resulting Matrix $C$ are the result of Multiplying
the corresponding Column of $B$ on the left by the Matrix $A$. And the
rows of the resulting Matrix $C$ are the result of Multiplying the
corresponding row of $A$ on the right by the Matrix $B$.



% --------------------------------------------------------------------
\subsection{Permutation Matrix}\label{sec:permutation_matrix}
% --------------------------------------------------------------------

A \emph{Permutation Matrix} is a Square Binary Matrix with exactly one
$1$ in each Row and each Column and $0$s elsewhere.



% --------------------------------------------------------------------
\subsection{Upper Triangular Matrix}\label{sec:upper_triangular}
% --------------------------------------------------------------------

% --------------------------------------------------------------------
\subsection{Determinant}\label{sec:determinant}
% --------------------------------------------------------------------

% --------------------------------------------------------------------
\subsection{Jacobian Matrix}\label{sec:jacobian_matrix}
% --------------------------------------------------------------------



% ====================================================================
\section{Vector Space}\label{sec:vector_space}
% ====================================================================

\emph{Span}

\emph{Finite Dimensional Vector Space} - has a Span

\emph{Infinite Dimensional Vector Space} - does not have a Span

\emph{Linear Independence}

\emph{Basis} (\S\ref{sec:basis}) - Spans and is Linearly Independent

All Bases of a Vector Space $\mathbf{V}$ have the same number of
Elements equal to the \emph{Dimension} of $\mathbf{V}$,
$dim(\mathbf{V})$. The Dimension of a Vector Space is uniquely defined
because for any Vector Space, a Basis exists, and all Bases of a
Vector space have equal Cardinality (\S\ref{sec:cardinality}).

For a Finite Dimensional Vector Space a Subset of a Span defines a
Basis, and a Linearly Independent Subset can be extended to form a
Basis.

The number of Elements in a Spanning Subset of $\mathbf{V}$ is greater
than or equal to the Dimension of $\mathbf{V}$.

The number of Elements in a Linearly Independent Subset of
$\mathbf{V}$ is less than or equal to the Dimension of $\mathbf{V}$.

A Basis defines an Isomorphism of Vector Spaces:
\[
    \mathbf{V} \xrightarrow{f} F^n
\]

\emph{Tensor Product}, \emph{Outer Product}



% --------------------------------------------------------------------
\subsection{Basis}\label{sec:basis}
% --------------------------------------------------------------------

% --------------------------------------------------------------------
\subsection{Inner Product}\label{sec:inner_product}
% --------------------------------------------------------------------

% --------------------------------------------------------------------
\subsection{Outer Product}\label{sec:outer_product}
% --------------------------------------------------------------------

Tensor Product (\S\ref{sec:tensor_product}) of two Vectors



% --------------------------------------------------------------------
\subsection{Norm}\label{sec:vector_space_norm}
% --------------------------------------------------------------------

\emph{Seminorm}

\emph{Quasinorm}



% ====================================================================
\section{Algebra}\label{sec:algebra}
% ====================================================================

\emph{Algebra over a Commutative Ring}

\emph{Algebra over a Field}

Coalgebra (\S\ref{sec:coalgebra})

$F$-algebra (\S\ref{sec:f_algebra})



% --------------------------------------------------------------------
\subsection{Unital Algebra}\label{sec:unital_algebra}
% --------------------------------------------------------------------

% --------------------------------------------------------------------
\subsection{Zero Algebra}\label{sec:zero_algebra}
% --------------------------------------------------------------------

% --------------------------------------------------------------------
\subsection{Associative Algebra}\label{sec:associative_algebra}
% --------------------------------------------------------------------

% --------------------------------------------------------------------
\subsection{Non-associative Algebra}
\label{sec:nonassociative_algebra}
% --------------------------------------------------------------------

% --------------------------------------------------------------------
\subsection{Coalgebra}\label{sec:coalgebra}
% --------------------------------------------------------------------



% ====================================================================
\section{Linear Map}\label{sec:linear_map}
% ====================================================================
