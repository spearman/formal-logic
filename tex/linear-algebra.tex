%%%%%%%%%%%%%%%%%%%%%%%%%%%%%%%%%%%%%%%%%%%%%%%%%%%%%%%%%%%%%%%%%%%%%%
%%%%%%%%%%%%%%%%%%%%%%%%%%%%%%%%%%%%%%%%%%%%%%%%%%%%%%%%%%%%%%%%%%%%%%
\part{Linear Algebra}\label{part:linear_algebra}
%%%%%%%%%%%%%%%%%%%%%%%%%%%%%%%%%%%%%%%%%%%%%%%%%%%%%%%%%%%%%%%%%%%%%%
%%%%%%%%%%%%%%%%%%%%%%%%%%%%%%%%%%%%%%%%%%%%%%%%%%%%%%%%%%%%%%%%%%%%%%

Set Theory (Part \ref{part:set_theory}) is Linear Algebra over the
``Field with one Element''



% ====================================================================
\section{Module}\label{sec:module}
% ====================================================================

A \emph{Module} is a Unital Ring (\S\ref{sec:unital_ring}), $R$,
together with an Abelian Group (\S\ref{sec:abelian_group}), $(M, +)$,
and an Operation called \emph{Scalar Multiplication} which is either:
\[ R \times M \rightarrow M \]
for a \emph{Left $R$-module $M$}, $_R M$, or:
\[ M \times R \rightarrow M \]
for a \emph{Right $R$-module $M$}, $M_R$.

The Scalar Multiplication Operator is required that for all $r,s \in
R$ and $x,y \in M$ in a Left $R$-module $M$:
\begin{enumerate}
    \item $r(x + y) = rx + ry$
    \item $(r + s)x = rx + sx$
    \item $(rs)x = r(sx)$
    \item $1_Rx = x$
\end{enumerate}
or in a Right $R$-module $M$:
\begin{enumerate}
    \item $(x + y)r = xr + yr$
    \item $x(r + s) = xr + xs$
    \item $x(rs) = (sx)r$
    \item $x 1_R = x$
\end{enumerate}
where $1_R$ is the Multiplicative Identity for $R$. If the Ring is not
required to be Unital, then item (4) above can be ommitted, but can be
explicitly required by stating that we are talking about a
\emph{Unital Left/Right $R$-module $M$}.

\emph{Bimodule}

If $R$ is Commutative, then Left $R$-modules are the same as Right
$R$-modules and simply called \emph{$R$-modules}.

A Module Homomorphism is a \emph{Linear Map} (\S\ref{sec:linear_map})

Vector Space (\S\ref{sec:vector_space})

Multilinear Algebra (\S\ref{sec:multilinear_algebra}):
\emph{Multilinear Map} (\S\ref{sec:multilinear_map})



% --------------------------------------------------------------------
\subsection{Linear Map}\label{sec:linear_map}
% --------------------------------------------------------------------

(or \emph{Linear Transformation}) is a Module Homomorphism
(\S\ref{sec:module}) $V \rightarrow W$ preserving Addition and Scalar
Multiplication.

Linear Maps are Morphisms in the Category of Modules over a given Ring
(\S\ref{sec:ring})

a Linear Map between Topological Vector Spaces
(\S\ref{sec:topological_vector}) can be Continuous
(\S\ref{sec:continuous}) and if the Domain and Codomain are the same
it is a Continuous Linear Operator (\S\ref{sec:continuous_linear})

(Fong16):

%FIXME: are linear relations different from linear maps in general?

\emph{Linear Relation} $L : U \rightsquigarrow V$ is a Subspace $L
\subseteq U \oplus V$

Linear Relations as Corelations in $\cat{Vect}$; Category of Vector
Spaces and Linear Relations $\cat{LinRel}$



\subsubsection{Linear Operator}\label{sec:linear_operator}

A \emph{Linear Operator} is an Endomorphic Linear Map, i.e. a Linear
Map from a Module to itself $V \rightarrow V$



\subsubsection{Continuous Linear Operator}\label{sec:continuous_linear}

A \emph{Continuous Linear Operator} is a Continuous
(\S\ref{sec:continuous_map}) Linear Operator between Topological
Vector Spaces (\S\ref{sec:topological_vector})

a C$^*$-algebra (\S\ref{sec:cstar_algebra}) is a Complex Algebra $A$
of Continuous Linear Operators on a Complex Hilbert Space
(\S\ref{sec:hilbert_space}) with the additional Properties that $A$ is
Topologically Closed in the Norm Topology of Operators and Closed
under the Operation of taking Adjoints of Operators


(wiki):

a Linear Operator on a Normed Vector Space
(\S\ref{sec:normed_vectorspace}) is Continuous if and only if it is a
Bounded Linear Operator (\S\ref{sec:bounded_linear_operator}), e.g.
when the Domain is Finite-dimensional: every Linear Operator on a
Finite-dimensional Space is Continuous



\subsubsection{Semilinear Map}\label{sec:semilinear_map}

(or \emph{Semilinear Transformation})

Sesquilinear Form (\S\ref{sec:sesquilinear_form})

generalizes the class of Antilinear Maps (\S\ref{sec:antilinear_map})



\paragraph{Antilinear Map}\label{sec:antilinear_map}\hfill

or \emph{Conjugate-linear Map}

is a Mapping $f : V \rightarrow W$ between Complex Vector Spaces $V$ and $W$
when:
\[
  f (ax+by) = \bar{a}f(x) + \bar{b}f(y)
\]
for all Complex Numbers $a,b \in \comps$ with Complex Conjugates $\bar{a},
\bar{b}$ and all $x,y \in V$

may be equivalently described in terms of the Linear Map $\bar{f} : V
\rightarrow \bar{W}$ from $V$ to the Complex Conjugate Vector Space $\bar{W}$

Quantum mechanics: Time Reversal

Spinor Calculus

\fist Antiunitary Operator (\S\ref{sec:antiunitary_operator})



\subsubsection{Linear Form}\label{sec:linear_form}

(or \emph{Linear Functional})

Distribution (\S\ref{sec:distribution})



\subsubsection{Short Linear Map}\label{sec:short_linear}

$\cat{Hilb}$



% --------------------------------------------------------------------
\subsection{Additive Map}\label{sec:additive_map}
% --------------------------------------------------------------------

$f(x + y) = f(x) + f(y)$

Special case of Subadditive Function (\S\ref{sec:subadditive_function})

Norm (\S\ref{sec:norm})



% --------------------------------------------------------------------
\subsection{Free Module}\label{sec:free_module}
% --------------------------------------------------------------------

A \emph{Free Module} is a Freely Generated (\S\ref{sec:free_object})
Module over a given Basis (\S\ref{sec:basis}).

Free Vector Space



\subsubsection{Group Ring}\label{sec:group_ring}



% --------------------------------------------------------------------
\subsection{Bimodule}\label{sec:bimodule}
% --------------------------------------------------------------------

Abelian Group (\S\ref{sec:abelian_group}) that is both a Left and a
Right Module such that the Left and Right Multiplications are
compatible.

Rings $R$, $S$, $R$-$S$-bimodule is an Abelian Group $M$ such that:

\begin{enumerate}
\item $M$ is a Left $R$-module and a Right $S$-module
\item $\forall r \in R, s \in S, m \in M, (rm)s = r(ms)$
\end{enumerate}

An $R$-$R$-bimodule is known as an $R$-bimodule

Generalization of Algebra Homomorphism
(\S\ref{sec:algebra_homomorphism})

Categorical generalization: Profunctor (\S\ref{sec:profunctor})



% ====================================================================
\section{Vector Space}\label{sec:vector_space}
% ====================================================================

\emph{Span}

\emph{Finite-dimensional Vector Space} -- has a Span; all
Finite-dimensional Vector Spaces are Nuclear Spaces
(\S\ref{sec:nuclear_space})

\emph{Infinite Dimensional Vector Space} -- does not have a Span

\emph{Linear Independence}

\emph{Basis} (\S\ref{sec:basis}) - Spans and is Linearly Independent

Field (\S\ref{sec:field})

every Vector Space is a \emph{Free Vector Space}: every Vector Space
looks like a Function $X \rightarrow R$ where $R$ is a Field and $X$
some Set of Basis Vectors, i.e. Vectors can be decomposed into a sum
of Scalars times the choice of Basis Vectors %FIXME

All Bases of a Vector Space $\mathbf{V}$ have the same number of
Elements equal to the \emph{Dimension} of $\mathbf{V}$,
$dim(\mathbf{V})$. The Dimension of a Vector Space is uniquely defined
because for any Vector Space, a Basis exists, and all Bases of a
Vector space have equal Cardinality (\S\ref{sec:cardinality}).

For a Finite Dimensional Vector Space a Subset of a Span defines a
Basis, and a Linearly Independent Subset can be extended to form a
Basis.

The number of Elements in a Spanning Subset of $\mathbf{V}$ is greater
than or equal to the Dimension of $\mathbf{V}$.

The number of Elements in a Linearly Independent Subset of
$\mathbf{V}$ is less than or equal to the Dimension of $\mathbf{V}$.

A Basis defines an Isomorphism of Vector Spaces:
\[
    \mathbf{V} \xrightarrow{f} F^n
\]

\emph{Tensor Product}, \emph{Outer Product} (\S\ref{sec:outer_product})

Scalar (0th-order Tensor \S\ref{sec:scalar}) is an Element of the
Field used to define a Vector Space

Vector (1st-order Tensor \S\ref{sec:vector}) is an Element of a Vector
Space



% --------------------------------------------------------------------
\subsection{Linear Combination}\label{sec:linear_combination}
% --------------------------------------------------------------------

% --------------------------------------------------------------------
\subsection{Basis}\label{sec:basis}
% --------------------------------------------------------------------

\subsubsection{Covariant Transformation}
\label{sec:covariant_transformation}

Coordinate-free (\S\ref{sec:coordinate_free})

Covariant Transformation Law

change of Basis

precise Transformation Law determines the Valence
(\S\ref{sec:valence}) of a Tensor (\S\ref{sec:linear_tensor})



% --------------------------------------------------------------------
\subsection{Linear Span}\label{sec:linear_span}
% --------------------------------------------------------------------

(or \emph{Linear Hull} or \emph{Span})

Intersection of all Subspaces containing Set of Vectors in a Vector
Space

Closed Linear Span (\S\ref{sec:closed_linear_span})



% --------------------------------------------------------------------
\subsection{Scalar Field}\label{sec:scalar_field}
% --------------------------------------------------------------------

Vector Field (\S\ref{sec:vector_field})

Tensor Field (\S\ref{sec:tensor_field})



\subsubsection{Scalar Function}\label{sec:scalar_function}



% --------------------------------------------------------------------
\subsection{Vector Field}\label{sec:vector_field}
% --------------------------------------------------------------------

assignment of a Vector to each Point in a (Subset of a) Space

Scalar Field (\S\ref{sec:scalar_field})

Tensor Field (\S\ref{sec:tensor_field})



% --------------------------------------------------------------------
\subsection{Normed Vector Space}\label{sec:normed_vectorspace}
% --------------------------------------------------------------------

Bounded Linear Operator (\S\ref{sec:bounded_linear_operator})

Functional Analysis (\S\ref{sec:functional_analysis})

Hilbert Space (\S\ref{sec:hilbert_space})



\subsubsection{Norm}\label{sec:norm}

\emph{Seminorm}

\emph{Quasinorm}

Subadditive Function (\S\ref{sec:subadditive_function})



\subsubsection{Triangle Inequality}\label{sec:triangle_inequality}

Real Line (\S\ref{sec:real_line}) with Absolute Value as Norm:\\
$|x + y| \leq |x| + |y|$



% --------------------------------------------------------------------
\subsection{Inner Product Space}\label{sec:innerproduct_space}
% --------------------------------------------------------------------

an \emph{Inner Product Space} is a Normed Vector Space with Norm
induced by a \emph{Inner Product}

Orthogonality %FIXME section ?

generalizes case of Euclidean Space (\S\ref{sec:euclidean_space}) to
Vector Spaces of any, possibly Infinite, Dimension

Functional Analysis (\S\ref{sec:functional_analysis})



\subsubsection{Inner Product}\label{sec:inner_product}

associates each pair of Vectors in a Vector Space with a Scalar known
as the \emph{Inner Product} of the Vectors

an Inner Product on a Real Vector Space is a Positive-definite
(\S\ref{sec:definite_quadratic}) Symmetric Bilinear Form
(\S\ref{sec:symmetric_bilinear}) and a Positive-definite Hermitian
Form (\S\ref{sec:hermitian_form})

Induces a Norm (\S\ref{sec:norm})



\subsubsection{Hilbert Space}\label{sec:hilbert_space}

Complete Metric Space (\S\ref{sec:complete_metric_space}) with an
Inner Product (\S\ref{sec:inner_product})

Functional Analysis (\S\ref{sec:functional_analysis}): can be defined
as a Banach Space (\S\ref{sec:banach_space})

Topological Vector Space (\S\ref{sec:topological_vector})

Hermitian Adjoint (\S\ref{sec:hermitian_adjoint})

$\cat{Hilb}$

$\cat{FdHilb}$



\paragraph{Hilbert Tensor Product}\label{sec:hilbert_tensor}\hfill

Topological Tensor Product (\S\ref{sec:topological_tensor})



\paragraph{Antiunitary Operator}\label{sec:antiunitary_operator}\hfill

a Bijective Antilinear Map (\S\ref{sec:antlinear_map}) $U$ between Complex
Hilbert Spaces $H_1$ and $H_2$:
\[
  U : H_1 \rightarrow H_2
\]
such that:
\[
  \langle{Ux,Uy}\rangle = \overline{\langle{x,y}\rangle}
\]

Quantum Theory: Time-reversal Symmetry (Wigner's Theorem, Ray Space) %FIXME



\subparagraph{Hilbert-Schmidt Mapping}\label{sec:hilbert_schmidt}\hfill

\emph{Weakly Hilbert-Schmidt Mapping}:
\[
  L : H_1 \otimes H_2 \rightarrow H
\]
is a Bilinear Map for which a $d$ exists such that
$\sum_{i,j=1}^\infty | \langle L(e_i,f_j), u \rangle |^2 \leq d^2
||u||^2$ for all $u \in K$ and one (implies all) Orthonormal Basis
$e_1, e_2, \ldots$ of $H_1$ and $f_1, f_2, \ldots$ of $H_2$

% FIXME

1970 - Blute, Panangaden:

the Ideal of Hilbert-Schmidt Maps contained in the Category of Hilbert
Spaces is an example of a \emph{Nuclear Ideal}: an Ideal contained in
an ambient Monoidal Dagger Category with the structure of a Compact
Closed Category except it is lacking Identities



\subsubsection{Unitary Space}\label{sec:unitary_space}

Inner Product Space over the Field of Complex Numbers



% --------------------------------------------------------------------
\subsection{Scalar Product Space}\label{sec:scalar_product_space}
% --------------------------------------------------------------------

% --------------------------------------------------------------------
\subsection{Hermitian Product Space}\label{sec:hermitian_product_space}
% --------------------------------------------------------------------

% --------------------------------------------------------------------
\subsection{Outer Product}\label{sec:outer_product}
% --------------------------------------------------------------------

Tensor Product (\S\ref{sec:tensor_product}) of two Vectors



% --------------------------------------------------------------------
\subsection{Real Vector Space}\label{sec:real_vector_space}
% --------------------------------------------------------------------

Vector Space over Scalar Field $F = \reals$

\fist \emph{Real Coordinate Space} (\S\ref{sec:real_coordinate_space})
$\reals^n$ -- prototypical Real Vector Space; Models Euclidean Space
(\S\ref{sec:euclidean_space}) with Cartesian Coordinates
(\S\ref{sec:cartesian_coordinates})



% --------------------------------------------------------------------
\subsection{Complex Vector Space}\label{sec:complex_vector_space}
% --------------------------------------------------------------------

Vector Space over Scalar Field $F = \comps$



% --------------------------------------------------------------------
\subsection{Symplectic Vector Space}\label{sec:symplectic_vectorspace}
% --------------------------------------------------------------------

Symplectic Geometry (\S\ref{sec:symplectic_geometry})

``Black-box Functor'': Category of Circuits $\rightarrow$ Category of
Symplectic Vector Spaces % FIXME Baez 15' Passive Linear Networks



% --------------------------------------------------------------------
\subsection{Finite-dimensional Vector Space}
\label{sec:finite_dimensional_vectorspace}
% --------------------------------------------------------------------

Finite-dimensional Vector spaces of Equal Dimension are Isomorphic
%FIXME



% --------------------------------------------------------------------
\subsection{Graded Vector Space}\label{sec:graded_vectorspace}
% --------------------------------------------------------------------

Graded Algebra (\S\ref{sec:graded_algebra})



% ====================================================================
\section{$R$-algebra}\label{sec:r_algebra}
% ====================================================================

Algebra over a Commutative Unital Ring

Module (\S\ref{sec:module}) with a \emph{Bilinear Product}
(\S\ref{sec:bilinear_product})



% --------------------------------------------------------------------
\subsection{Bilinear Product}\label{sec:bilinear_product}
% --------------------------------------------------------------------

% --------------------------------------------------------------------
\subsection{Algebra Homomorphism}\label{sec:algebra_homomorphism}
% --------------------------------------------------------------------

Generalized by Bimodules (\S\ref{sec:bimodule})



% --------------------------------------------------------------------
\subsection{Non-associative $R$-algebra}
\label{sec:nonassociative_r_algebra}
% --------------------------------------------------------------------

For Commutative Unital Ring $R$, $R$-module $V$ with Bilinear Product
(\S\ref{sec:bilinear_product}) $V \otimes V \rightarrow V$



% --------------------------------------------------------------------
\subsection{Associative $R$-algebra}\label{sec:associative_r_algebra}
% --------------------------------------------------------------------

\emph{Associative Unital $R$-algebra}, $R$-module $V$ with Bilinear
Product (\S\ref{sec:bilinear_product}) $p : V \otimes V \rightarrow V$
Linear Map $i : R \rightarrow V$ satisfying Associative and Unit Laws



% ====================================================================
\section{$K$-algebra}\label{sec:k_algebra}
% ====================================================================

\emph{Algebra over a Field}

Vector Space (\S\ref{sec:vector_space}) with a \emph{Bilinear Product}
(\S\ref{sec:bilinear_product})

Coalgebra (\S\ref{sec:coalgebra})

$F$-algebra (\S\ref{sec:f_algebra})

Every Finite Dimensional Associative Division Algebra over $\reals$ is
Isomorphic to either $\reals$, $\comps$, or $\quats$ (Theorem of
Frobenius) %FIXME



% --------------------------------------------------------------------
\subsection{Unital Algebra}\label{sec:unital_algebra}
% --------------------------------------------------------------------

% --------------------------------------------------------------------
\subsection{Zero Algebra}\label{sec:zero_algebra}
% --------------------------------------------------------------------

% --------------------------------------------------------------------
\subsection{Associative Algebra}\label{sec:associative_algebra}
% --------------------------------------------------------------------

Coalgebra (\S\ref{sec:coalgebra}): Dual to a Unital Associative
Algebra



\subsubsection{Frobenius Algebra}\label{sec:frobenius_algebra}

Representation Theory (\S\ref{sec:representation_theory})

Module Theory (\S\ref{sec:module})

Frobenius Algebras can be defined in any Monoidal Category (or
Polycategory \S\ref{sec:polycategory}) and are sometimes called
\emph{Frobenius Monoids} (\S\ref{sec:frobenius_monoid})



% --------------------------------------------------------------------
\subsection{Non-associative Algebra}
\label{sec:nonassociative_algebra}
% --------------------------------------------------------------------

% --------------------------------------------------------------------
\subsection{Coalgebra}\label{sec:coalgebra}
% --------------------------------------------------------------------

Dual to Unital Associative Algebra (\S\ref{sec:associative_algebra})

Cf. $F$-coalgebra (\S\ref{sec:f_coalgebra})



% --------------------------------------------------------------------
\subsection{Bialgebra}\label{sec:bialgebra}
% --------------------------------------------------------------------

Unital Associative Algebra and Coalgebra Vector Space over Field $K$



\subsubsection{Quasi-bialgebra}\label{sec:quasi_bialgebra}

\subsubsection{Hopf Algebra}\label{sec:hopf_algebra}

simultaneously a Unital Associative Algebra and a Counital
Coassociative Coalgebra with compatibility of these Structures making
it a Bialgebra, also equipped with an Anti-automorphism satisfying a
certain Property %FIXME what property?

Algebraic Topology



% ====================================================================
\section{Graded Algebra}\label{sec:graded_algebra}
% ====================================================================

Graded Vector Space (\S\ref{sec:graded_vectorspace})



% ====================================================================
\section{Exterior Algebra}\label{sec:exterior_algebra}
% ====================================================================

%FIXME multilinear algebra?

Grassmann Algebra (\S\ref{sec:grassmann_algebra})



% --------------------------------------------------------------------
\subsection{Bivector}\label{sec:bivector}
% --------------------------------------------------------------------

$2$-vector



% --------------------------------------------------------------------
\subsection{Exterior Product}\label{sec:exterior_product}
% --------------------------------------------------------------------

(\emph{Wedge Product})



% ====================================================================
\section{Linear Equation}\label{sec:linear_equation}
% ====================================================================

% ====================================================================
\section{Matrix Theory}\label{sec:matrix_theory}
% ====================================================================

a \emph{Matrix} represents a \emph{Linear Map}
(\S\ref{sec:linear_map}) between two Vector Spaces
(\S\ref{sec:vector_space})

\fist Matrix (Abstract Algebra \S\ref{sec:matrix})



% --------------------------------------------------------------------
\subsection{Matrix Multiplication}\label{sec:matrix_multiplication}
% --------------------------------------------------------------------

The Multiplication of an $m \times n$ Matrix $A$ with entries $a_{xy}$
on the left by an $n \times p$ Matrix $B$ with entries $b_{wv}$ on the
right results in an $m \times p$ Matrix $C$ with the entry $c_{ij}$
defined as:
\[
  \sum_{k=1}^n a_{ik} b_{kj}
\]
The columns of the resulting Matrix $C$ are the result of Multiplying
the corresponding Column of $B$ on the left by the Matrix $A$. And the
rows of the resulting Matrix $C$ are the result of Multiplying the
corresponding row of $A$ on the right by the Matrix $B$.



% --------------------------------------------------------------------
\subsection{Permutation Matrix}\label{sec:permutation_matrix}
% --------------------------------------------------------------------

A \emph{Permutation Matrix} is a Square Binary Matrix with exactly one
$1$ in each Row and each Column and $0$s elsewhere.



% --------------------------------------------------------------------
\subsection{Upper Triangular Matrix}\label{sec:upper_triangular}
% --------------------------------------------------------------------

% --------------------------------------------------------------------
\subsection{Determinant}\label{sec:determinant}
% --------------------------------------------------------------------

% --------------------------------------------------------------------
\subsection{Jacobian Matrix}\label{sec:jacobian_matrix}
% --------------------------------------------------------------------



% ====================================================================
\section{Differential Algebra}\label{sec:differential_algebra}
% ====================================================================

% --------------------------------------------------------------------
\subsection{Derivation}\label{sec:derivation}
% --------------------------------------------------------------------

% --------------------------------------------------------------------
\subsection{Differential Ring}\label{sec:differential_ring}
% --------------------------------------------------------------------

% --------------------------------------------------------------------
\subsection{Differential Field}\label{sec:differential_field}
% --------------------------------------------------------------------

% --------------------------------------------------------------------
\subsection{Differential Galois Theory}\label{sec:differential_galois}
% --------------------------------------------------------------------

Galois Theory (\S\ref{sec:galois_theory})

Extensions of Differential Fields

Galois Groups (\S\ref{sec:galois_group}) of Differential Equations



% ====================================================================
\section{Multilinear Algebra}\label{sec:multilinear_algebra}
% ====================================================================

% --------------------------------------------------------------------
\subsection{Dual Space}\label{sec:dual_space}
% --------------------------------------------------------------------

$V \cong V^*$

$V \cong V^**$ ``Naturally'' (\S\ref{sec:natural_transformation})

Contravariant Representable Functor
(\S\ref{sec:representable_functor}):
\[
  (-)^* = Vect(-,\mathbb{R}) :
    \mathbf{Vect}^op \rightarrow \mathbf{Vect}
\]

\[
  A^* = \pow(A) \cong \mathbf{Set}(A,2)
\]\cite{awodey06}



% --------------------------------------------------------------------
\subsection{Multilinear Map}\label{sec:multilinear_map}
% --------------------------------------------------------------------

\subsubsection{Bilinear Map}\label{sec:bilinear_map}

Module (\S\ref{sec:module}), Vector Space (\S\ref{sec:vector_space})

$R$-Algebra (\S\ref{sec:r_algebra})

$K$-Algebra (\S\ref{sec:k_algebra})

Bilinear Product (\S\ref{sec:bilinear_product})



% --------------------------------------------------------------------
\subsection{Multilinear Form}\label{sec:multilinear_form}
% --------------------------------------------------------------------

\subsubsection{Bilinear Form}\label{sec:bilinear_form}

Inner Product (\S\ref{sec:inner_product})



\paragraph{Symmetric Bilinear Form}\label{sec:symmetric_bilinear}
\hfill

Symmetric Bilinear Forms over a Vector Space correspond one-to-one
with Quadratic Forms (\S\ref{sec:quadratic_form}) over the Vector
Space

an Inner Product (\S\ref{sec:inner_product}) on a Real Vector Space is
a Positive-definite (\S\ref{sec:definite_quadratic}) Symmetric
Bilinear Form



\paragraph{Degenerate Bilinear Form}
\label{sec:degenerate_bilinear_form}\hfill

Pseudo-Riemannian Manifold (\S\ref{sec:pseudo_riemannian})



\subsubsection{Quadratic Form}\label{sec:quadratic_form}

Homogenous Polynomial (\S\ref{sec:homogenous_polynomial})

Quadratic Forms over a Vector Space correspond one-to-one with
Symmetric Bilinear Forms (\S\ref{sec:symmetric_bilinear}) over
the Vector Space



\paragraph{Definite Quadratic Form}\label{sec:definite_quadratic}\hfill

a Quadratic Form over some Real Vector Space that has the same Sign
for every Non-zero Vector

\emph{Positive Definite}

\emph{Negative Definite}

\emph{Semidefinite}

an Inner Product (\S\ref{sec:inner_product}) on a Real Vector Space is
a Positive-definite Symmetric Bilinear
(\S\ref{sec:symmetric_bilinear}) Form



\paragraph{Quadratic Space}\label{sec:quadratic_space}\hfill



\subsubsection{Sesquilinear Form}\label{sec:sesquilinear_form}

Semilinear Map (\S\ref{sec:semilinear_map})



\paragraph{Hermitian Form}\label{sec:hermitian_form}\hfill

an Inner Product (\S\ref{sec:inner_product}) is a Positive-definite
(\S\ref{sec:definite_quadratic}) Hermitian Form



% ====================================================================
\section{Tensor Calculus}\label{sec:tensor_calculus}
% ====================================================================

% --------------------------------------------------------------------
\subsection{Tensor}\label{sec:linear_tensor}
% --------------------------------------------------------------------

wikipedia: Geometric Objects (???) that describe Linear Relations
between Scalars, Vectors and other Tensors; Linear Relations such as
Dot Product, Cross Product, Linear Maps

%FIXME: merge with sec:tensor ???

\fist Tensor (Abstract Algebra \S\ref{sec:tensor})

\fist Metric Tensor (\S\ref{sec:metric_tensor})

must be independent of a particular choice of Coordinate System
(Coordinate-free \S\ref{sec:coordinate_free}): the particular Covariant
Transformation Law (\S\ref{sec:covariant_transformation}) determines
the \emph{Valence} (\S\ref{sec:valence}) of a Tensor

may be represented by:
\begin{itemize}
  \item Multidimensional Arrays (Basis Independence not apparent)
  \item Multilinear Maps (intrinsic Basis Independence)
  \item Elements of (Abstract) Tensor Products
\end{itemize}

\fist
\url{https://jeremykun.com/2014/01/17/how-to-conquer-tensorphobia/}
%FIXME cite

Tensors are Elements (Vectors) of a Vector Space given by
combining two ``smaller'' Vector Spaces via a Tensor Product

$v \otimes w \in V \otimes W$

Scalar Multiplication: $s(v \otimes w) = (sv \otimes w) = (v \otimes
sw)$; generalizing to  $n$-fold Tensor Products, Scalars can be moved
around all the coordinates freely

Addition Operation: $(v \otimes w) + (v' \otimes w) = (v + v') \otimes
w)$ \emph{or} $(v \otimes w) + (v \otimes w') = v \otimes (w + w')$,
otherwise $(x \otimes y) + (z \otimes w)$ is a ``new'' Tensor
(Vector); generalizing to $n$-fold Tensor Products, Addition can be
combined if all but one of the coordinates are the same in the
Addends.

Element of a Tensor Space $V_1 \otimes \cdots \otimes V_n$ as a Sum:
\[
  \Sigma_k a_{1,k} \otimes a_{2,k} \otimes cdots \otimes a_{n,k}
\]

A Rank $1$ or \emph{Pure Tensor} is one that can be expressed as a
One-term Sum, i.e. just $a_1 \otimes \cdots \otimes a_n$


\asterism


represented as an ``organized'' Multidimensional Array of Numerical
(?Scalar) Values

\emph{Order} (or \emph{Degree} or \emph{Rank}) of a Tensor $x$ is the
Dimensionality of the Array needed to represent it or equivalently the
minimum number of Terms to represent $x$ as a Sum of Pure Tensors (the
Zero Element is Rank $0$ by convention).

Rank $0$ Tensor -- Scalar (\S\ref{sec:scalar})

Rank $1$ Tensor (\emph{Pure Tensor}) -- Vector (\S\ref{sec:vector})

Rank $2$ Tensor -- Linear Map (\S\ref{sec:linear_map}) ? %FIXME correct?

Computing Tensor Rank is $NP$-hard when $k = \rats$ and $NP$-complete
when $k$ is a Finite Field %FIXME


\asterism


Multilinear Maps


\asterism


a Type (\S\ref{sec:valence}) $(n,m)$ Tensor $T$ is defined as an
Element of the Tensor Product (\S\ref{sec:tensor_product}) of Vector
Spaces $V$ and Dual Spaces $V^*$:
\[
  T \in V_1 \otimes \cdots \otimes V_n
    \otimes V^*_1 \otimes \cdots \otimes V^*_m
\]

the Tensor Product is Initial with respect to Multilinear Mappings
from the Direct Product

Tensors (Abstract Algebra \S\ref{sec:tensor})

Tensor Products (Category Theory \S\ref{sec:tensor_product}):
\begin{itemize}
  \item Modules
  \item Vector Spaces
  \item Graded Vector Spaces
  \item $R$-algebras
  \item Sheaves of Modules
  \item Quadratic Forms
  \item Multilinear Form
  \item Topological Vector Spaces
\end{itemize}

Monoidal Category (\S\ref{sec:monoidal_category}): general context for
Tensor Products



% --------------------------------------------------------------------
\subsection{Valence}\label{sec:valence}
% --------------------------------------------------------------------

precise Covariant Transformation Law
(\S\ref{sec:covariant_transformation}) determines the \emph{Valence}
(or \emph{Type}) of a Tensor

Tensor Type: Pair of Natural Numbers $(n,m)$ where $n$ is the number
of Contravariant Indicies and $m$ is the number of Covariant Indices

\emph{Total Order} is the sum of $n$ and $m$



% --------------------------------------------------------------------
\subsection{Module Tensor Product}\label{sec:module_tensor}
% --------------------------------------------------------------------

Balanced Product



% --------------------------------------------------------------------
\subsection{Tensor Field}\label{sec:tensor_field}
% --------------------------------------------------------------------

Tensor assigned to each Point in a Space

Scalar Field (\S\ref{sec:scalar_field})

Vector Field (\S\ref{sec:vector_field})



% --------------------------------------------------------------------
\subsection{Tensor Algebra}\label{sec:tensor_algebra}
% --------------------------------------------------------------------

of a Vector Space



% --------------------------------------------------------------------
\subsection{Eigenconfiguration}\label{sec:eigenconfiguration}
% --------------------------------------------------------------------

% --------------------------------------------------------------------
\subsection{Grassmann Algebra}\label{sec:grassmann_algebra}
% --------------------------------------------------------------------

Exterior Algebra (\S\ref{sec:exterior_algebra})

\subsubsection{$p$-vector}\label{sec:p_vector}

\subsubsection{Multivector}\label{sec:multivector}



% --------------------------------------------------------------------
\subsection{Ricci Calculus}\label{sec:ricci_calculus}
% --------------------------------------------------------------------

% --------------------------------------------------------------------
\subsection{Spinor}\label{sec:spinor}
% --------------------------------------------------------------------



% ====================================================================
\section{Numerical Linear Algebra}\label{sec:numerical_linear_algebra}
% ====================================================================

% --------------------------------------------------------------------
\subsection{Iterative Method}\label{sec:iterative_method}
% --------------------------------------------------------------------

\subsubsection{Stationary Iterative Method}
\label{sec:stationary_iterative}

\paragraph{Gauss-Seidel Method}\label{sec:gauss_seidel}\hfill

(or \emph{Liebmann Method} or \emph{Method of Successive
  Displacement})

Iterative Method for solving a Linear System of Equations
(\S\ref{sec:system_of_linear_equations})

Convergence only guaranteed if the Matrix is either Diagonally
Dominant (\S\ref{sec:diagonally_dominant}) or Symmetric
(\S\ref{sec:symmetric_matrix}) and Positive Definite
(\S\ref{sec:positive_definite})



\subparagraph{Projected Gauss-Seidel Method}\hfill
\label{sec:projected_gauss_seidel}

Gauss-Seidel applied to Linear Complementarity Problem (LCP
\S\ref{sec:linear_complementarity})



\subparagraph{Non-linear Gauss-Seidel Method}
\label{sec:nonlinear_gauss_seidel}



\paragraph{Jacobi Method}\label{sec:jacobi_method}\hfill

Diagonally Dominant (\S\ref{sec:diagonally_dominant}) System of Linear
Equations (\S\ref{sec:system_of_linear_equations})



\subparagraph{Projected Jacobi Method}
\label{sec:projected_jacobi_method}



\subsubsection{Krylov Subspace Method}\label{sec:krylov_subspace_method}



% ====================================================================
\section{Super Linear Algebra}\label{sec:super_linear_algebra}
% ====================================================================

% --------------------------------------------------------------------
\subsection{Superalgebra}\label{sec:superalgebra}
% --------------------------------------------------------------------

$Z_2$-graded Algebra



\subsubsection{Clifford Algebra}\label{sec:clifford_algebra}

\url{https://golem.ph.utexas.edu/category/2014/07/the_tenfold_way.html} -- kinds
of matter corresponding to Real Clifford Algebras:
\begin{itemize}
\item $Cl_0$ (equivalent to $\reals$)
\item $Cl_1$
\item $Cl_2$
\item $Cl_3$
\item $Cl_4$ (equivalent to $\quats$)
\item $Cl_5$
\item $Cl_6$
\item $Cl_7$
\end{itemize}
and Complex Clifford Algebras:
\begin{itemize}
\item $\comps{l}_0$ (equivalent to $\comps$)
\item $\comps{l}_1$ (Superalgebra created by adding an Odd Square Root of $-1$
  to the purely Even Algebra $\comps$)
\end{itemize}
