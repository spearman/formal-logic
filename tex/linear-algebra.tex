%%%%%%%%%%%%%%%%%%%%%%%%%%%%%%%%%%%%%%%%%%%%%%%%%%%%%%%%%%%%%%%%%%%%%%
%%%%%%%%%%%%%%%%%%%%%%%%%%%%%%%%%%%%%%%%%%%%%%%%%%%%%%%%%%%%%%%%%%%%%%
\part{Linear Algebra}\label{sec:linear_algebra}
%%%%%%%%%%%%%%%%%%%%%%%%%%%%%%%%%%%%%%%%%%%%%%%%%%%%%%%%%%%%%%%%%%%%%%
%%%%%%%%%%%%%%%%%%%%%%%%%%%%%%%%%%%%%%%%%%%%%%%%%%%%%%%%%%%%%%%%%%%%%%

% ====================================================================
\subsection{Module}\label{sec:module}
% ====================================================================

A \emph{Module} is a Unital Ring (\S\ref{sec:unital_ring}), $R$,
together with an Abelian Group (\S\ref{sec:abelian_group}), $(M, +)$,
and an Operation called \emph{Scalar Multiplication} which is either:
\[ R \times M \rightarrow M \]
for a \emph{Left $R$-module $M$}, $_R M$, or:
\[ M \times R \rightarrow M \]
for a \emph{Right $R$-module $M$}, $M_R$.

The Scalar Multiplication Operator is required that for all $r,s \in
R$ and $x,y \in M$ in a Left $R$-module $M$:
\begin{enumerate}
    \item $r(x + y) = rx + ry$
    \item $(r + s)x = rx + sx$
    \item $(rs)x = r(sx)$
    \item $1_Rx = x$
\end{enumerate}
or in a Right $R$-module $M$:
\begin{enumerate}
    \item $(x + y)r = xr + yr$
    \item $x(r + s) = xr + xs$
    \item $x(rs) = (sx)r$
    \item $x 1_R = x$
\end{enumerate}
where $1_R$ is the Multiplicative Identity for $R$. If the Ring is not
required to be Unital, then item (4) above can be ommitted, but can be
explicitly required by stating that we are talking about a
\emph{Unital Left/Right $R$-module $M$}.

\emph{Bimodule}

If $R$ is Commutative, then Left $R$-modules are the same as Right
$R$-modules and simply called \emph{$R$-modules}.

A Module Homomorphism is an \emph{Linear Map} (\S\ref{sec:linear_map})

\emph{Bilinear Map} (\S\ref{sec:bilinear_map})

Vector Space (\S\ref{sec:vector_space})

\emph{Multilinear Map}



% --------------------------------------------------------------------
\subsection{Linear Map}\label{sec:linear_map}
% --------------------------------------------------------------------

(or \emph{Linear Transformation}) is a Module (\S\ref{sec:module})
Homomorphism (\S\ref{sec:homomorphism}) $V \rightarrow W$ preserving
Addition and Scalar Multiplication.

Linear Maps are Morphisms in the Category of Modules over a given Ring
(\S\ref{sec:ring}).



\subsubsection{Linear Operator}\label{sec:linear_operator}

A \emph{Linear Operator} is an Endomorphic Linear Map, i.e. a Linear
Map from a Module to itself $V \rightarrow V$



\subsubsection{Bilinear Map}\label{sec:bilinear_map}

Module (\S\ref{sec:module}), Vector Space (\S\ref{sec:vector_space})

$R$-Algebra (\S\ref{sec:r_algebra})

$K$-Algebra (\S\ref{sec:k_algebra})

Bilinear Product (\S\ref{sec:bilinear_product})



\subsubsection{Bilinear Form}\label{sec:bilinear_form}

Inner Product (\S\ref{sec:inner_product})



% --------------------------------------------------------------------
\subsection{Module Tensor Product}\label{sec:module_tensor}
% --------------------------------------------------------------------

% --------------------------------------------------------------------
\subsection{Free Module}\label{sec:free_module}
% --------------------------------------------------------------------

A \emph{Free Module} is a Freely Generated (\S\ref{sec:free_object})
Module over a given Basis (\S\ref{sec:basis}).



\subsubsection{Group Ring}\label{sec:group_ring}



% --------------------------------------------------------------------
\subsection{Bimodule}\label{sec:bimodule}
% --------------------------------------------------------------------



% ====================================================================
\section{Vector Space}\label{sec:vector_space}
% ====================================================================

\emph{Span}

\emph{Finite Dimensional Vector Space} - has a Span

\emph{Infinite Dimensional Vector Space} - does not have a Span

\emph{Linear Independence}

\emph{Basis} (\S\ref{sec:basis}) - Spans and is Linearly Independent

Field (\S\ref{sec:field})

All Bases of a Vector Space $\mathbf{V}$ have the same number of
Elements equal to the \emph{Dimension} of $\mathbf{V}$,
$dim(\mathbf{V})$. The Dimension of a Vector Space is uniquely defined
because for any Vector Space, a Basis exists, and all Bases of a
Vector space have equal Cardinality (\S\ref{sec:cardinality}).

For a Finite Dimensional Vector Space a Subset of a Span defines a
Basis, and a Linearly Independent Subset can be extended to form a
Basis.

The number of Elements in a Spanning Subset of $\mathbf{V}$ is greater
than or equal to the Dimension of $\mathbf{V}$.

The number of Elements in a Linearly Independent Subset of
$\mathbf{V}$ is less than or equal to the Dimension of $\mathbf{V}$.

A Basis defines an Isomorphism of Vector Spaces:
\[
    \mathbf{V} \xrightarrow{f} F^n
\]

\emph{Tensor Product}, \emph{Outer Product} (\S\ref{sec:outer_product})



% --------------------------------------------------------------------
\subsection{Vector}\label{sec:vector}
% --------------------------------------------------------------------

Element of a Vector Space



% --------------------------------------------------------------------
\subsection{Linear Combination}\label{sec:linear_combination}
% --------------------------------------------------------------------

% --------------------------------------------------------------------
\subsection{Basis}\label{sec:basis}
% --------------------------------------------------------------------

% --------------------------------------------------------------------
\subsection{Norm}\label{sec:norm}
% --------------------------------------------------------------------

\emph{Seminorm}

\emph{Quasinorm}



% --------------------------------------------------------------------
\subsection{Dual Space}\label{sec:dual_space}
% --------------------------------------------------------------------

$V \cong V^*$

$V \cong V^**$ ``Naturally'' (\S\ref{sec:natural_transformation})

Contravariant Representable Functor
(\S\ref{sec:representable_functor}):
\[
  (-)^* = Vect(-,\mathbb{R}) :
    \mathbf{Vect}^op \rightarrow \mathbf{Vect}
\]

\[
  A^* = \mathcal{P}(A) \cong \mathbf{Set}(A,2)
\]\cite{awodey06}



% --------------------------------------------------------------------
\subsection{Inner Product Space}\label{sec:innerproduct_space}
% --------------------------------------------------------------------

Vector Space with an \emph{Inner Product}



\subsubsection{Inner Product}\label{sec:inner_product}

\subsubsection{Hilbert Space}\label{sec:hilbert_space}

Real or Complex Inner Product Space that is also a Complete Metric
Space (\S\ref{sec:complete_metric_space})

Functional Analysis (\S\ref{sec:functional_analysis}): can be defined
as a Banach Space (\S\ref{sec:banach_space})



% --------------------------------------------------------------------
\subsection{Outer Product}\label{sec:outer_product}
% --------------------------------------------------------------------

Tensor Product (\S\ref{sec:tensor_product}) of two Vectors



% --------------------------------------------------------------------
\subsection{Normed Vector Space}\label{sec:normed_vectorspace}
% --------------------------------------------------------------------

\subsubsection{Triangle Inequality}\label{sec:triangle_inequality}

Real Line (\S\ref{sec:real_line}) with Absolute Value as Norm:\\
$|x + y| \leq |x| + |y|$



% --------------------------------------------------------------------
\subsection{Symplectic Vector Space}\label{sec:symplectic_vectorspace}
% --------------------------------------------------------------------

Symplectic Geometry (\S\ref{sec:symplectic_geometry})

``Black-box Functor'': Category of Circuits $\rightarrow$ Category of
Symplectic Vector Spaces % FIXME Baez 15' Passive Linear Networks



% --------------------------------------------------------------------
\subsection{Real Coordinate Space}\label{sec:real_space}
% --------------------------------------------------------------------

Euclidean Space %FIXME



\subsubsection{Convex Set}\label{sec:convex_set}

\subsubsection{$n$-sphere}\label{sec:n_sphere}

$S^n = \{ x \in \reals^{n+1} : \|x\| = r \}$

$S^n = \reals \cup \{\infty\}$



\paragraph{Unit Circle}\label{sec:unit_circle}

\paragraph{Unit Sphere}\label{sec:unit_sphere}
\hfill \\

Unit Ball (\S\ref{sec:unit_ball})



% ====================================================================
\section{$R$-algebra}\label{sec:r_algebra}
% ====================================================================

Algebra over a Commutative Unital Ring

Module (\S\ref{sec:module}) with a \emph{Bilinear Product}
(\S\ref{sec:bilinear_product})



% --------------------------------------------------------------------
\subsection{Bilinear Product}\label{sec:bilinear_product}
% --------------------------------------------------------------------

% --------------------------------------------------------------------
\subsection{Non-associative $R$-algebra}
\label{sec:nonassociative_r_algebra}
% --------------------------------------------------------------------

For Commutative Unital Ring $R$, $R$-module $V$ with Bilinear Product
(\S\ref{sec:bilinear_product}) $V \otimes V \rightarrow V$



% --------------------------------------------------------------------
\subsection{Associative $R$-algebra}\label{sec:associative_r_algebra}
% --------------------------------------------------------------------

\emph{Associative Unital $R$-algebra}, $R$-module $V$ with Bilinear
Product (\S\ref{sec:bilinear_product}) $p : V \otimes V \rightarrow V$
Linear Map $i : R \rightarrow V$ satisfying Associative and Unit Laws



% ====================================================================
\section{$K$-algebra}\label{sec:k_algebra}
% ====================================================================

\emph{Algebra over a Field}

Vector Space (\S\ref{sec:vector_space}) with a \emph{Bilinear Product}
(\S\ref{sec:bilinear_product})

Coalgebra (\S\ref{sec:coalgebra})

$F$-algebra (\S\ref{sec:f_algebra})

Every Finite Dimensional Associative Division Algebra over $\reals$ is
Isomorphic to either $\reals$, $\comps$, or $\quats$ (Theorem of
Frobenius) %FIXME



% --------------------------------------------------------------------
\subsection{Unital Algebra}\label{sec:unital_algebra}
% --------------------------------------------------------------------

% --------------------------------------------------------------------
\subsection{Zero Algebra}\label{sec:zero_algebra}
% --------------------------------------------------------------------

% --------------------------------------------------------------------
\subsection{Associative Algebra}\label{sec:associative_algebra}
% --------------------------------------------------------------------

% --------------------------------------------------------------------
\subsection{Non-associative Algebra}
\label{sec:nonassociative_algebra}
% --------------------------------------------------------------------

% --------------------------------------------------------------------
\subsection{Coalgebra}\label{sec:coalgebra}
% --------------------------------------------------------------------

% --------------------------------------------------------------------
\subsection{Bialgebra}\label{sec:bialgebra}
% --------------------------------------------------------------------

Unital Associative Algebra and Coalgebra Vector Space over Field $K$



\subsubsection{Quasi-bialgebra}\label{sec:quasi_bialgebra}



% ====================================================================
\section{Linear Equation}\label{sec:linear_equation}
% ====================================================================

% ====================================================================
\section{Matrix Theory}\label{sec:matrix_theory}
% ====================================================================

% --------------------------------------------------------------------
\subsection{Matrix Multiplication}\label{sec:matrix_multiplication}
% --------------------------------------------------------------------

The Multiplication of an $m \times n$ Matrix $A$ with entries $a_{xy}$
on the left by an $n \times p$ Matrix $B$ with entries $b_{wv}$ on the
right results in an $m \times p$ Matrix $C$ with the entry $c_{ij}$
defined as:
\[
  \sum_{k=1}^n a_{ik} b_{kj}
\]
The columns of the resulting Matrix $C$ are the result of Multiplying
the corresponding Column of $B$ on the left by the Matrix $A$. And the
rows of the resulting Matrix $C$ are the result of Multiplying the
corresponding row of $A$ on the right by the Matrix $B$.



% --------------------------------------------------------------------
\subsection{Permutation Matrix}\label{sec:permutation_matrix}
% --------------------------------------------------------------------

A \emph{Permutation Matrix} is a Square Binary Matrix with exactly one
$1$ in each Row and each Column and $0$s elsewhere.



% --------------------------------------------------------------------
\subsection{Upper Triangular Matrix}\label{sec:upper_triangular}
% --------------------------------------------------------------------

% --------------------------------------------------------------------
\subsection{Determinant}\label{sec:determinant}
% --------------------------------------------------------------------

% --------------------------------------------------------------------
\subsection{Jacobian Matrix}\label{sec:jacobian_matrix}
% --------------------------------------------------------------------
