%%%%%%%%%%%%%%%%%%%%%%%%%%%%%%%%%%%%%%%%%%%%%%%%%%%%%%%%%%%%%%%%%%%%%%%%%%%%%%%%
%%%%%%%%%%%%%%%%%%%%%%%%%%%%%%%%%%%%%%%%%%%%%%%%%%%%%%%%%%%%%%%%%%%%%%%%%%%%%%%%
\part{Fractal Geometry}\label{part:fractal_geometry}
%%%%%%%%%%%%%%%%%%%%%%%%%%%%%%%%%%%%%%%%%%%%%%%%%%%%%%%%%%%%%%%%%%%%%%%%%%%%%%%%
%%%%%%%%%%%%%%%%%%%%%%%%%%%%%%%%%%%%%%%%%%%%%%%%%%%%%%%%%%%%%%%%%%%%%%%%%%%%%%%%

Iterated Functions (\S\ref{sec:iterated_function})

Scale Invariance (\S\ref{sec:scale_invariance})

Geometric Measure Theory (\S\ref{sec:gmt})

Chaos Theory (\S\ref{sec:chaos_theory})

cf. Parabolic Fractal Distribution (Probability Theory
\S\ref{sec:parabolic_fractal_distribution})

(Mandelbrot82) ``Geometric face'' of Harmonic Analysis
(\S\ref{sec:harmonic_analysis})

\asterism

1993 - Dube - \emph{Undecidable Problems in Fractal Geometry} \fist
Computability Theory (Part \ref{part:recursion_theory}: Recursion Theory)

for every Turing Machine (\S\ref{sec:turing_machine}), there exists a Fractal
Set (\S\ref{sec:fractal}) viewed in a certain sense as ``Geometrically
Encoding'' the Complement of the Language Accepted by the Machine

Computationally Universal, Fractal-based Geometric Model of Computation
(\S\ref{sec:computation_model})

Geometric display of Complex Systems (\S\ref{sec:complex_system}) as ``Fractal
images''; Fractal images as the Chaotic Set of a Dynamical System

Turing Machine as Dynamical System, Fractal image Encoding of its ``Chaotic
Set'' (Words not Accepted by the Turing Machine or may lead to Infinite
behavior, i.e. Non-halting)

\asterism

2016 - Laba, Falconer - \emph{Harmonic Analysis and Additive Combinatorics on
  Fractals}; Additive Combinatorics (\S\ref{sec:additive_combinatorics})



% ==============================================================================
\section{Iterated Function System (IFS)}\label{sec:ifs}
% ==============================================================================

an \emph{Iterated Function System} is a Finite Set of Contraction Mappings
(\S\ref{sec:contraction_map}) on a Complete Metric Space
(\S\ref{sec:complete_metric_space})

Iterated Function (\S\ref{sec:iterated_function})

inverse problem: Fractal Compression

1992 - Duvall, Husch - \emph{Attractors of Iterated Function Systems}

1999 - Tino, Dorffner - \emph{Recurrent Neural Networks with Iterated Function
  System Dynamics} -- RNNs (\S\ref{sec:rnn})

\fist Fractal Prediction Machine (Reservoir Computing
\S\ref{sec:fractal_prediction_machine})

\asterism

1993 - Dube - \emph{Undecidable Problems in Fractal Geometry} \fist
Computability Theory (Part \ref{part:recursion_theory}: Recursion Theory)

two Undecidable questions about Iterated Function Systems:
\begin{enumerate}
  \item test if the Attractor of a given IFS \emph{Intersects}
    (\S\ref{sec:intersection}) a given Line Segment
  \item test if a given IFS is \emph{Totally Disconnected}
    (\S\ref{sec:totally_disconnected})
\end{enumerate}

Proof by Reducing the Post Correspondence Problem
(\S\ref{sec:post_correspondence}) and Interpreting Strings as Numbers and
Concatenation as Compositions of Affine Transformations
(\S\ref{sec:affine_transformation})

IFS: a collection of $N$ Contractive Affine Transformations with Rational
Coefficients defines a unique Fractal Set which is the Fixed Point
(\emph{Attractor} \S\ref{sec:attractor_repeller}) of a mapping applying the $N$
Affine Transformations and then taking the Union; Approximation (Fixed-point
Iteration ???)

note: allowing for Transformations other than Affine Transformations allows IFSs
to generate Julia Sets (\S\ref{sec:julia_set})

constructing an $\omega$-language (\S\ref{sec:omega_language}) from the Set of
Contractive Affine Transformations $\Sigma$ of an IFS on a Metric Space $(X, d)$
yields a Surjective mapping $\phi : \Sigma^\omega \rightarrow X$, assigning an
``\emph{Address}'' $\sigma \in \Sigma^\omega$ to each Point in the Attractor,
$A \subseteq X$, of the IFS, $\forall x \in X$:
\[
  \phi(\sigma) =
    \lim_{i \rightarrow \infty} \sigma_1(\sigma_2(\ldots(\sigma_i(x))\ldots)) =
    a \in A
\]
that is, from the \emph{Contractivity} of the $\sigma_i$ Transformations, the
Limit $a$ is \emph{independent} of $x$

Determinstic Iterative Algorithm for describing $A$:

initialize a Set $S_1 = \{ B \}$ of Subsets of $X$ with $B \subseteq X$ and
Iteratively apply all $N$ Transformations to each Subset in $S_n$, taking the
Union of the results to be $S_{n+1}$

cf. Fixed-point Iteration \S\ref{sec:fixedpoint_iteration}; in this case the
output of each iteration is $N$ Subsets instead of a single Value

equivalently:
\[
  A = \bigcap_{n=1}^{\infty} S_n
\]

at the $n$-th step, $S_n$ will contain the result of all possible Sequences of
Transformations of length $n$ on the starting Set $B$, i.e.:
\[
  w_{i_1}(w_{i_2}(\ldots(w_{i_n}(B))\ldots))
\]
for all Sequencces $i_1, i_2, \ldots, i_n$

\emph{Geometric Model of Computation}

\textbf{Lemma 1} \emph{
 For an IFS $\{\reals^2; w_1, w_2, \ldots, w_N\}$ consisting of Affine
 Transformations with Rational Coefficients, and $L$ a Line Segment with
 Rational Endpoints, there is a (Semi)-procedure that Terminates if and only if
 the Attractor $A$ of the IFS does not Intersect $L$.
}

\textbf{Proof} Due to the Contractivity of the Transformations, if at any step
$n$, $L$ does not Intersect, then $L$ will not Intersect on step $n+1$, and so
the Attractor does not Intersect $L$. Conversely, if the Attractor does not
Intersect $L$, then at some step $n$ the intermediate Subsets do not Intersect.
If at any step of running the Deterministic Algorithm on the IFS, none of the
Subsets Intersect $L$, then the Semi-procedure Terminates.

since the Procedure Halts (Accepts) when an Input does not Intersect, the
Attractor represents a graphical ``encoding'' of the Strings \emph{not} in the
Language

cf. other graphical Models of Computation: any Turing Machine can be represented
as a Set of Wang Dominoes that Tile the Plane if and only if the Turing Machine
does not Halt, however in this case the Tiles are more like ``Symbols'' and does
not define in the limiting case a unqiue pattern encoding information about the
behavior of the Machine on all Inputs

one-to-one correspondence between Concatenation of Strings and application of
corresponding Affine Transformations (TODO)

\textbf{Thm. 3, Thm. 4} -- Implies it is Undecidable to test if the Intersection
of the Attractors of two IFSs is empty

an IFS is called \emph{Totally Disconnected} if and only if every Point in $A$
has a \emph{unique} Address, i.e.
$\forall i, j \in \{1, 2, \ldots, N\}, i \neq j, w_i(A) \cap w_j(A) = \varnothing$

note that being Totally Disconnected is a Property of the IFS, \emph{not} the
Attractor

if an IFS with two Affine Transformations is Totally Disconnected, then its
Attractor is also Totally Disconnected and is therefore a Cantor Set, and if the
IFS is \emph{not} Totally Disconnected, then its Attractor is a Connected Set

a Julie Aset is defined by an IFS with two Transformations that are not Affine
and is therefore either a Conator Set or Connected

an IFS with more than two Transformations might have an Attractor that is
neither a Cantor Set nor Connected

\textbf{Lemma 4} \emph{There exists a Semi-procedure that will Terminate if and
  only if a given IFS is Totally Disconnected}

\textbf{Proof} An IFS is Totally Disconnected if and only if during the
Deterministic Algorithm there is an $i$ such that all the Subsets obtained by
applying all possible Sequences of Transformations of length $i$ are mutually
Disjoint

\textbf{Thm. 5} \emph{It is Undecidable to test if a given IFS is Totally
  Disconnected}
-- i.e. there does not exist a Semi-procedure which Terminates if a given IFS is
\emph{not} Totally Disconnected



% ------------------------------------------------------------------------------
\subsection{Partitioned Iterated Function System (PIFS)}\label{sec:pifs}
% ------------------------------------------------------------------------------

% ------------------------------------------------------------------------------
\subsection{Recurrent Iterated Function System (RIFS)}\label{sec:rifs}
% ------------------------------------------------------------------------------

1989 - Barnsley, Elton, Hardin - \emph{Recurrent Iterated Function Systems}



% ------------------------------------------------------------------------------
\subsection{Mutually Recusrive Function System (MRFS)}\label{sec:mrfs}
% ------------------------------------------------------------------------------

equivalent to RIFS

Projective IFS

defines a ``richer class'' of Fractals and are useful in image compression; the
problem of testing Membership of a given Rational Point in the Attractor of a
given MRFS is Undecidable (Dube 1993)

cf. Mutually Recursive Type (\S\ref{sec:mutually_recursive})



% ==============================================================================
\section{Fractal}\label{sec:fractal}
% ==============================================================================

(Mandelbrot 1975): an object whose Hausdorff-Besicovitch Dimension
(\S\ref{sec:hausdorff_dimension}) is greater than its Topological Dimension
(Lebesgue Covering Dimension \S\ref{sec:lebesgue_dimension}); note this
requirement is not met by Fractal Space-filling Curves, e.g. the Hilbert Curve

Complex Dynamics:
\begin{itemize}
  \item Julia Sets (\S\ref{sec:julia_set})
  \item Mandelbrot Set (\S\ref{sec:mandelbrot_set})
\end{itemize}

\asterism

L-systems (\emph{Lindenmayer Systems} \S\ref{sec:l_system}) -- a kind of
Parallel Rewriting System

%FIXME xrefs:

in $\reals$:
\begin{itemize}
  \item Prouhet-Thue-Morse System
\end{itemize}

in $\reals^2$:
\begin{itemize}
  \item Space-filling Curves
  \item Median Space-filling Curves
  \item Tilings (Sphinx Tiling, Penrose Tiling)
\end{itemize}

\fist Strange Attractors (\S\ref{sec:strange_attractor}): a \emph{Fractal
  Attractor} is an Attractor whose ``local structure'' is Fractal

\fist Scaling Distributions (\S\ref{sec:scaling_distribution})

\asterism

(Dube 1993):
\begin{quote}
  for every Turing Machine (\S\ref{sec:turing_machine}), there exists a Fractal
  Set viewed in a certain sense as ``Geometrically Encoding'' the Complement of
  the Language Accepted by the Machine
\end{quote}



% ------------------------------------------------------------------------------
\subsection{Self-affine Fractal}\label{sec:selfaffine_fractal}
% ------------------------------------------------------------------------------

Mandelbrot97E

``Roughness'' (\S\ref{sec:rugosity}) in terms of Self-affinity
(\S\ref{sec:self_affinity}) is measured by Roughness Exponent (Hurst Exponent
\S\ref{sec:hurst_exponent}) and a Scale Factor (similar to Root Mean Square
\S\ref{sec:mean_square})

\begin{itemize}
  \item Devil Staircase (\S\ref{sec:devil_staircase})
  \item Fractional Brownian Motion (\S\ref{sec:fractional_brownian})
  \item ...
\end{itemize}



\subsubsection{Self-similar Fractal}\label{sec:selfsimilar_fractal}

Self-similarity (\S\ref{sec:self_similarity})

Mandelbrot97E: Fractal notion of Self-similarity (\S\ref{sec:self_similarity}):
Isotropic Reduction (Homothety \S\ref{sec:homothety}) followed by Rotation and
Translation

\emph{Self-similarity Dimension}:
\[
  D = \frac{-\ln N}{\ln r(N)}
\]
where $r(N)$ is the Similarity Ratio for each of the $N$ parts of the
Self-similarity Relation; for a Self-similar Measurable Set, the portion of the
Set that is Contained in a Sphere of Radius $R$ is of Measure proportional to
$R^D$

Fractal Dimension (\S\ref{sec:fractal_dimension}): Self-similar Fractals have a
\emph{unique} Fractal Dimension (Self-affine Fractals may have several depending
on which ``aspect'' is being considered)



\subsubsection{Devil Staircase}\label{sec:devil_staircase}

L\'evy Staircase Function (\S\ref{sec:levy_staircase})

\fist Multifractal Function (\S\ref{sec:multifractal_function}) -- lacks the
flat steps of a Devil Staircase

Compound Processes (\S\ref{sec:compound_process}) -- a Devil Staircase Function
used as a Subordinator (Non-negative Directing Function with Statistically
Independent Increments) is called a \emph{Fractal Time}
(\S\ref{sec:fractal_time})



% ------------------------------------------------------------------------------
\subsection{Fractal String}\label{sec:fractal_string}
% ------------------------------------------------------------------------------

\subsubsection{Cantor Set}\label{sec:cantor_set}

a (Locally) Compact Separated (Hausdorff) Space (\S\ref{sec:compact_separated})
and a Locale (\S\ref{sec:locale})

\emph{$p$-adic Numbers} (\S\ref{sec:padic_number})



\paragraph{Fat Cantor Set}\label{sec:fat_cantor_set}\hfill

Nowhere Dense, Positive Measure



% ------------------------------------------------------------------------------
\subsection{Fractal Time}\label{sec:fractal_time}
% ------------------------------------------------------------------------------

Berger, Mandelbrot 1963

(Mandelbrot97E) ``Unifractal'' Models of Price Variation: $L$-Stable Motion
(Infinite Variance, ``Tail-driven Variability'' \S\ref{sec:lsm}) + FBM
(Long-range Dependence \S\ref{sec:fbm})

Scaling/Dependence Exponent dependent on Time $t$

Compound Processes (\S\ref{sec:compound_process}) -- a Devil Staircase Function
(\S\ref{sec:devil_staircase}) used as a Subordinator (Non-negative Directing
Function with Statistically Independent Increments) is called a \emph{Fractal
  Time}; a Wiener Process (\S\ref{sec:wiener_process}) as a Directed Function of
a Fractal Time Directing Function yields an $L$-stable Motion (\S\ref{sec:lsm})
with $\alpha$ ``fed in'' by the L\'evy Staircase (\S\ref{sec:levy_staircase})

\fist Multifractal Time (\S\ref{sec:multifractal_time})



% ------------------------------------------------------------------------------
\subsection{Fractal Curve}\label{sec:fractal_curve}
% ------------------------------------------------------------------------------

Space-filling Curves (\S\ref{sec:space_filling_curve})

cf. Differentiable Curves (\S\ref{sec:differentiable_curve})



\subsubsection{Koch Snowflake}\label{sec:koch_snowflake}

Self-similar (\S\ref{sec:self_similarity})



% ==============================================================================
\section{Fractal Analysis}\label{sec:fractal_analysis}
% ==============================================================================

application to Multifractal Analysis (\S\ref{sec:multifractal_system})



% ------------------------------------------------------------------------------
\subsection{Fractal Dimension}\label{sec:fractal_dimension}
% ------------------------------------------------------------------------------

expresses differences in Non-topological aspects of \emph{Form}
(\S\ref{sec:form})

cf. Topological (Lebesgue Covering) Dimension
(\S\ref{sec:topological_dimension})

Hausdorff Dimension (\S\ref{sec:hausdorff_dimension}) -- measure of
``roughness''

\fist a \emph{Multifractal System} (\S\ref{sec:multifractal_system}) is a
``Fractal System'' (FIXME: clarify) for which the Fractal Dimension Exponent is
not enough to describe its Dynamics, instead being described by a (Continuous)
\emph{Singularity Spectrum} (\S\ref{sec:singularity_spectrum})

Mandelbrot97E

Fractal Co-dimension (\S\ref{sec:fractal_codimension}) --
Hurst Exponent (``Index of Long-range Dependence'' \S\ref{sec:hurst_exponent})
represents structure over asymptotically longer periods, Fractal Dimension
represents structure over asymptotically shorter periods

\emph{Self-similarity Dimension} (\S\ref{sec:self_similarity}):
\[
  D = \frac{-\ln N}{\ln r(N)}
\]
where $r(N)$ is the Similarity Ratio for each of the $N$ parts of the
Self-similarity Relation;
for a Self-similar Measurable Set, the portion of the Set that is Contained in a
Sphere of Radius $R$ is of Measure proportional to $R^D$

Self-similar Fractals (\S\ref{sec:selfsimilar_fractal}) have a \emph{unique}
Fractal Dimension; Self-affine Fractals may have several depending on which
``aspect'' is being considered, e.g. for FBM (\S\ref{sec:fractional_brownian})
$B_H$, the Box Dimension (\S\ref{sec:box_dimension}) of the Graph,
(\emph{Graph Dimension}) $D_G = 2 - H$, and the Box Dimension of the ``trail'',
(\emph{Trail Dimension}, i.e. the Dimension of the ``motion'' itself without the
Time axis) $D_T = \frac{1}{H}$; in the Multifractal (\S\ref{sec:multifractal})
case, $D_G$ and $D_T$ cease to be functionally related

in the Self-affine case, Fractal Dimension splits into ``Local'', as given
above, and ``Global'' forms



\subsubsection{Box Dimension}\label{sec:box_dimension}

\emph{Minkowski-Bouligand Dimension} or \emph{Box-counting Dimension}



\subsubsection{Hausdorff Dimension}\label{sec:hausdorff_dimension}

or \emph{Hausdorff-Besicovitch Dimension}

measure of ``\emph{rougness}'' or \emph{Fractal Dimension}

cf. Topological Dimension (\S\ref{sec:topological_dimension})

agrees with Inductive Dimension (\S\ref{sec:inductive_dimension}) of Topological
Space

Outer Measure (\S\ref{sec:outer_measure})

(Mandelbrot 1975): defines a ``Fractal'' as an object whose Hausdorff Dimension
is greater than its Topological Dimension (Lebesgue Covering Dimension
\S\ref{sec:lebesgue_dimension}); note this requirement is not met by Fractal
Space-filling Curves, e.g. the Hilbert Curve

cf. Box Dimension (\S\ref{sec:box_dimension})



% ------------------------------------------------------------------------------
\subsection{Fractal Co-dimension}\label{sec:fractal_codimension}
% ------------------------------------------------------------------------------

Hurst Exponent (``Index of Long-range Dependence'' or ``Roughness Exponent''
\S\ref{sec:hurst_exponent})

Fractal Dimension (\S\ref{sec:fractal_dimension}) -- Hurst Exponent represents
structure over asymptotically longer periods, Fractal Dimension represents
structure over asymptotically shorter periods

cf. Codimension (\S\ref{sec:codimension})



\subsubsection{Rugosity}\label{sec:rugosity}

%FIXME: move this section ???

Ratio of ``true'' Surface Area, $A_r$, to Projected Surface Area, $A_g$:
\[
  f_r = \frac{A_r}{A_g}
\]

Mandelbrot97E, Ch.6: ``Roughness'' in terms of Self-affinity
(\S\ref{sec:self_affinity}) is Measured by Roughness Exponent (Hurst Exponent
\S\ref{sec:hurst_exponent}) and a Scale Factor (similar to Root Mean Square
\S\ref{sec:mean_square})



\subsubsection{Lacunarity}\label{sec:lacunarity}

%FIXME: move this section ???

(wiki): ``gap'' or ``lake''; in addition to quantifying ``gaps'', also
quantifies features such as ``rotational invariance'' or more general
``heterogeneity''

application to Multifractal Analysis (\S\ref{sec:multifractal_system})



% ==============================================================================
\section{Multifractal System}\label{sec:multifractal_system}
% ==============================================================================

(wiki):

generalized framework of Fractional Brownian Motions
(\S\ref{sec:fractional_brownian})

generalization of a ``Fractal System'' in which the single Exponent of Fractal
Dimension (\S\ref{sec:fractal_dimension}) is not enough to describe its
``Dynamics'',  instead a (Continuous) \emph{Singularity Spectrum}
(\S\ref{sec:singularity_spectrum}) is needed (FIXME: clarify)

origin of ``Multifractality'' in Sequential (Time Series) data attributed to
``convergence effects'' related to the Central Limit Theorem
(\S\ref{sec:central_limit}) that have Foci of Convergence in the family of
Tweedie Exponential Dispersion Models (Probability Distributions
\S\ref{sec:tweedie_distribution})

Fractal Analysis (\S\ref{sec:fractal_analysis}), Lacunarity Analysis
(\S\ref{sec:lacunarity})

Generalized Index of Long-range Dependence (Generalized Hurst Exponent
\S\ref{sec:generalized_hurst}) $H(q)$ is a Non-linear Function of $q$, the Time
Series is a Multifractal System

Statistical Physics

\fist Markov Switching MultiFractal (Stochastic Volatility Model
\S\ref{sec:msmf})



% ------------------------------------------------------------------------------
\subsection{Multifractal Function}\label{sec:multifractal_function}
% ------------------------------------------------------------------------------

1972 - Mandelbrot - \emph{Possible Refinement of the Lognormal Hypothesis
  Concerning the Distribution of Energy Dissipation in Intermittent Turbulence}

(Mandelbrot97E)

Non-decreasing, Continuous, Non-differentiable

lacks the flat steps of a Devil Staircase (\S\ref{sec:devil_staircase})

cf. Fractional Brownian Motion (\S\ref{sec:fractional_brownian})

$M(t)$



\subsubsection{Multifractal Measure}\label{sec:multifractal_measure}

increments of a Multifractal Function

intermittent aspects of Turbulence, Variability in Variance of price increments



% ------------------------------------------------------------------------------
\subsection{Multifractal Time}\label{sec:multifractal_time}
% ------------------------------------------------------------------------------

(Mandelbrot97E) ``Unifractal'' Models of Price Variation: $L$-Stable Motion
(Infinite Variance, ``Tail-driven Variability'' \S\ref{sec:lsm}) + FBM
(Long-range Dependence \S\ref{sec:fbm})

Scaling/Dependence Exponent dependent on Time $t$

Fractal Time (\S\ref{sec:fractal_time}) -- cf. L\'evy Staircase Function
(\S\ref{sec:levy_staircase})

Process $Z(t)$ and Non-decreasing Function of Time $\theta(t)$ gives Compound
Process $\tilde{Z}(\theta(t))$ where $\tilde{Z}(\theta)$ and $\theta(t)$ are
taken to be Independent Scaling (\S\ref{sec:scaling_distribution})

``flicker noise'' (1/f, Pink Noise)

concentration of discontinuities; cf. ARCH (AutoRegressive Conditional
Heteroskedasticity \S\ref{sec:arch}) Models

models of price variation:

the Distribution $L(t, T) = \ln Z(t + T) - \ln Z(t)$ (cf. $L$-stable Motion
\S\ref{sec:lsm}) is observed to be increasingly ``sharp peaked'' and Long-tailed
as $T \to 0$

\emph{alternative explanation for drift}: Long-term Dependence and
``clustering'' of large changes of $L(t, T)$ means that changes of small $p$ are
unlikely to be observed in a finite sample as $T$ becomes large, reducing the
Histograms Tail

``Scale Factors'' $\sigma(q) = \Big(\expect(L^q(t, T))\Big)^{\frac{1}{q}}$

\emph{Uniscaling} (Fractal \S\ref{sec:scaling_distribution}) -- $q < \alpha$ are
Powers of $T$ with Exponent \emph{independent} of $q$

--FIXME: $\alpha$ ???

\emph{Multiscaling} (Multifractal \S\ref{sec:multiscaling}) -- the Exponents of
the Scale Factors \emph{depend} on $q$

Continuous Compounding (\S\ref{sec:compound_process})



% ------------------------------------------------------------------------------
\subsection{Singularity Spectrum}\label{sec:singularity_spectrum}
% ------------------------------------------------------------------------------

Function used to describe the Fractal Dimension (\S\ref{sec:fractal_dimension})
of a Subset of Points of a Function belonging to a ``group'' of Points that have
the same H\"older Exponent (TODO: xref)
