%%%%%%%%%%%%%%%%%%%%%%%%%%%%%%%%%%%%%%%%%%%%%%%%%%%%%%%%%%%%%%%%%%%%%%%%%%%%%%%%
%%%%%%%%%%%%%%%%%%%%%%%%%%%%%%%%%%%%%%%%%%%%%%%%%%%%%%%%%%%%%%%%%%%%%%%%%%%%%%%%
\part{Fractal Geometry}\label{sec:fractal_geometry}
%%%%%%%%%%%%%%%%%%%%%%%%%%%%%%%%%%%%%%%%%%%%%%%%%%%%%%%%%%%%%%%%%%%%%%%%%%%%%%%%
%%%%%%%%%%%%%%%%%%%%%%%%%%%%%%%%%%%%%%%%%%%%%%%%%%%%%%%%%%%%%%%%%%%%%%%%%%%%%%%%

Iterated Functions (\S\ref{sec:iterated_function})

Scale Invariance (\S\ref{sec:scale_invariance})

Chaos Theory (\S\ref{sec:chaos_theory})

cf. Parabolic Fractal Distribution (Probability Theory
\S\ref{sec:parabolic_fractal_distribution})

2016 - Laba, Falconer - \emph{Harmonic Analysis and Additive Combinatorics on
  Fractals}; Harmonic Analysis (\S\ref{sec:harmonic_analysis}), Additive
Combinatorics (\S\ref{sec:additive_combinatorics})



% ==============================================================================
\section{Geometric Measure Theory}\label{sec:geometric_measure_theory}
% ==============================================================================

\fist Measure Theory (Part \ref{part:measure_theory})

\fist Hausdorff Measure (\S\ref{sec:hausdorff_measure})



% ==============================================================================
\section{Iterated Function System (IFS)}\label{sec:ifs}
% ==============================================================================

an \emph{Iterated Function System} is a Finite Set of Contraction Mappings
(\S\ref{sec:contraction_map}) on a Complete Metric Space

Iterated Function (\S\ref{sec:iterated_function})

inverse problem: Fractal Compression

1992 - Duvall, Husch - \emph{Attractors of Iterated Function Systems}



% ------------------------------------------------------------------------------
\subsection{Partitioned Iterated Function System (PIFS)}\label{sec:pifs}
% ------------------------------------------------------------------------------



% ==============================================================================
\section{Fractal}\label{sec:fractal}
% ==============================================================================

(Mandelbrot 1975): an object whose Hausdorff-Besicovitch Dimension
(\S\ref{sec:hausdorff_dimension}) is greater than its Topological Dimension
(Lebesgue Covering Dimension \S\ref{sec:lebesgue_dimension}); note this
requirement is not met by Fractal Space-filling Curves, e.g. the Hilbert Curve

\asterism

L-systems (\emph{Lindenmayer Systems} \S\ref{sec:l_system}) -- a kind of
Parallel Rewriting System (\S\ref{sec:parallel_rewrite})

%FIXME xrefs:

in $\reals$:
\begin{itemize}
  \item Prouhet-Thue-Morse System
\end{itemize}

in $\reals^2$:
\begin{itemize}
  \item Space-filling Curves
  \item Median Space-filling Curves
  \item Tilings (Sphinx Tiling, Penrose Tiling)
\end{itemize}

\fist Strange Attractors (\S\ref{sec:strange_attractor}): a \emph{Fractal
  Attractor} is an Attractor whose ``local structure'' is Fractal



% ------------------------------------------------------------------------------
\subsection{Fractal String}\label{sec:fractal_string}
% ------------------------------------------------------------------------------

\subsubsection{Cantor Set}\label{sec:cantor_set}

a (Locally) Compact Hausdorff Space (\S\ref{sec:hausdorff_space}) and a Locale
(\S\ref{sec:locale})



% ==============================================================================
\section{Fractal Analysis}\label{sec:fractal_analysis}
% ==============================================================================

application to Multifractal Analysis (\S\ref{sec:multifractal_system})



% ------------------------------------------------------------------------------
\subsection{Fractal Dimension}\label{sec:fractal_dimension}
% ------------------------------------------------------------------------------

Hausdorff Dimension (\S\ref{sec:hausdorff_dimension})

\fist a \emph{Multifractal System} (\S\ref{sec:multifractal_system}) is a
``Fractal System'' (FIXME: clarify) for which the Fractal Dimension Exponent is
not enough to describe its Dynamics, instead being described by a (Continuous)
\emph{Singularity Spectrum} (\S\ref{sec:singularity_spectrum})



\subsubsection{Box Dimension}\label{sec:box_dimension}

\emph{Minkowski-Bouligand Dimension} or \emph{Box-counting Dimension}



% ------------------------------------------------------------------------------
\subsection{Lacunarity}\label{sec:lacunarity}
% ------------------------------------------------------------------------------

(wiki): ``gap'' or ``lake''; in addition to quantifying ``gaps'', also
quantifies features such as ``rotational invariance'' or more general
``heterogeneity''

application to Multifractal Analysis (\S\ref{sec:multifractal_system})



% ==============================================================================
\section{Multifractal System}\label{sec:multifractal_system}
% ==============================================================================

(wiki):

generalized framework of Fractional Brownian Motions
(\S\ref{sec:fractional_brownian})

generalization of a ``Fractal System'' in which the single Exponent of Fractal
Dimension (\S\ref{sec:fractal_dimension}) is not enough to describe its
``Dynamics'',  instead a (Continuous) \emph{Singularity Spectrum}
(\S\ref{sec:singularity_spectrum}) is needed (FIXME: clarify)

origin of ``Multifractality'' in Sequential (Time Series) data attributed to
``convergence effects'' related to the Central Limit Theorem
(\S\ref{sec:central_limit}) that have Foci of Convergence in the family of
Tweedie Exponential Dispersion Models (Probability Distributions
\S\ref{sec:tweedie_distribution})

Fractal Analysis (\S\ref{sec:fractal_analysis}), Lacunarity Analysis
(\S\ref{sec:lacunarity})

Statistical Physics



% ------------------------------------------------------------------------------
\subsection{Singularity Spectrum}\label{sec:singularity_spectrum}
% ------------------------------------------------------------------------------

Function used to describe the Fractal Dimension (\S\ref{sec:fractal_dimension})
of a Subset of Points of a Fnuction belonging to a ``group'' of Points that have
the same H\"older Exponent (TODO: xref)
