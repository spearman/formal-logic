%%%%%%%%%%%%%%%%%%%%%%%%%%%%%%%%%%%%%%%%%%%%%%%%%%%%%%%%%%%%%%%%%%%%%%
%%%%%%%%%%%%%%%%%%%%%%%%%%%%%%%%%%%%%%%%%%%%%%%%%%%%%%%%%%%%%%%%%%%%%%
\part{Automata Theory}\label{sec:automata_theory}
%%%%%%%%%%%%%%%%%%%%%%%%%%%%%%%%%%%%%%%%%%%%%%%%%%%%%%%%%%%%%%%%%%%%%%
%%%%%%%%%%%%%%%%%%%%%%%%%%%%%%%%%%%%%%%%%%%%%%%%%%%%%%%%%%%%%%%%%%%%%%

% ====================================================================
\section{State Transition Systems} \label{sec:state_transition_system}
% ====================================================================

A \emph{State Transition System} can have an infinite number of
\emph{States} and \emph{Transitions}, represented as the pair
\[
    (S,\rightarrow)
\]
where $S$ is a set of States and $\rightarrow \subseteq S \times S$.
This is identical to an \emph{un-indexed Abstract Rewriting
  System}(\S\ref{sec:abstract_rewrite}).

\emph{Finite Automata} may be seen as State Transition Systems with an
initial State and a number of final \emph{Accept} states indicating
\emph{Word} (Expression) membership for a Language.

\emph{Labeled State Transition Systems} have an additional set of
\emph{Labels}, $\Lambda$
\[(S,\Lambda,\rightarrow)\]
and $\rightarrow \subseteq S \times \Lambda \times S$.

\emph{Action Programming Languages} add a set of \emph{Fluents}, $F$, and
\emph{Values}, $V$, and a function mapping $F \times S$ to $V$.



% ====================================================================
\section{Semiautomata}
% ====================================================================

A State Transition System may be formulated as a \emph{Semiautomata}
\[
    (Q,\Sigma,T)
\]
where $\Sigma$ is a non-empty \emph{input Symbols}, $Q$ is the set of
States, and $T$ is a \emph{transition function} $T:Q \times \Sigma
\rightarrow Q$.

A Semiautomaton induces a Monoid called the \emph{input Monoid}:
\[
    M(Q,\Sigma,T) = \{T_w | w \in \Sigma^*\}
\]



% ====================================================================
\section{Automata} \label{subsec:automata}
% ====================================================================

An \emph{Automaton} reads input strings, \emph{Words} (Expressions),
and either accepts or rejects depending on whether a Word is a member
of the Language recognized by that Automaton. By convention the
Vocabulary of Expressions from Formal Languages will be re-cast as an
Alphabet of Words, $\Sigma$. The Set of all Words accepted by an
Automaton are those that are \emph{Recognized} by the Automaton.

Automata may be arranged in a hierarchy according to increasing power:
\[
    DFA = NFA \subset DPDA-I \subset NPDA-I \subset LBA \subset DPDA-II =
\]\[
    = NPDA-II = DTM = NTM = PTM = MDTM
\]
where
\begin{itemize}
\item DFA = Deterministic Finite Automata
\item NFA = Non-deterministic Finite Automata
\item DPDA = Deterministic Push Down Automata with 1
  or 2 push-down stores
\item NPDA = Non-deterministic Push Down Automata
  with 1 or 2 push-down stores
\item LBA = Linear Bounded Automata
\item DTM = Deterministic Turing Machine
\item NTM = Non-deterministic Turing Machine
\item PTM = Probabilistic Turing Machine
\item MDTM = Multidimensional Turing Machine
\end{itemize}

% --------------------------------------------------------------------
\subsection{Finite Automata}
% --------------------------------------------------------------------

\emph{Finite Automata} are \emph{Finite State Machines} and take a
finite input string of Symbols and either accepts or rejects the
input depending on the final State of the computation. Finite
Automata are able to recognize Regular Languages(\S\ref{subsec:regular_language}).

\subsubsection{Deterministic Finite Automata}\label{subsec:dfa}
\emph{Deterministic Finite Automata} have the restriction that an
input Symbol has a transition function to a single State.
Deterministic Finite Automata recognize Regular
Languages(\S\ref{subsec:regular_language}).

Representation of a Deterministic Finite Automaton as a 5-tuple:
\[
    (Q,\Sigma,\delta,q_0,F)
\]
where
\begin{itemize}
\item $Q$ is a finite set of States
\item $\Sigma$ is the Alphabet
\item $\delta$ is the transition function $\delta: Q \times
  \Sigma \rightarrow Q$
\item $q_0 \in Q$ is the initial State
\item $F \subseteq Q$ is the set of final Accept States.
\end{itemize}

Running for a given input $w = a_1,a_2, \cdots , a_n \in \Sigma^*$
produces a sequence of States $q_0,q_1,q_2,\cdots , q_n$ where $q_i
\in Q$ such that $q_i = \delta (q_{i-1},a_i)$ and $w$ is accepted if
$q_n \in F$.

A recursive definition using \emph{composition} of transition
functions
\[
    \widehat{\delta}(q,\varepsilon) = q
\]\[
    \widehat{\delta}(q,wa) = \delta_a(\widehat{\delta}(q,w))
\]
where $w \in \Sigma^*$, $a \in \Sigma$ and $q \in Q$. Repeated
application describes the \emph{Transition Monoid} or
\emph{Transformation Semigroup}.

A kind of Deterministic Finite Automata that recognizes Local
Languages(\S\ref{subsec:k_testable}) is called a \emph{Local Automaton}.

\subsubsection{Nondeterministic Finite Automata}\label{subsec:ndfa}
\emph{Nondeterministic Finite Automata} are Finite State Machines that
may transition from one State to a number of different states, given
as an element of the powerset of $Q$, $\mathcal{P}(Q)$.

Representation of a Nondeterministic Finite Automaton as a
5-tuple:
\[
    (Q,\Sigma,\Delta,q_0,F)
\]
where
\begin{itemize}
\item $Q$ is a finite set of States
\item $\Sigma$ is the Alphabet
\item $\Delta$ is a \emph{transition relation} $\Delta: Q \times
  \Sigma \rightarrow \mathcal{P}(Q)$
\item $q_0 \in Q$ is the initial State
\item $F \subseteq Q$ is the set of final Accept States.
\end{itemize}

A Word, $w=a_1,a_2,\cdots,a_n$, is accepted when there exists a
sequence of States, $r_0,r_1,\cdots,r_n$ such that
\begin{enumerate}
\item $r_0 = q_0$
\item $r_{i+1} \in \Delta(r_i, a_{i+1})$, for $i = 0, \cdots, n-1$
\item $r_n \in F$
\end{enumerate}

A DFA may be seen as a NFA which restricts transitions to allow only
one State, and can be constructed from a NFA with $n$ States using
\emph{powerset construction}, requiring up to $2^n$ States. Both types
recognize the same Regular Languages(\S\ref{subsec:regular_language}).

\paragraph{NFA-$\varepsilon$} is a NFA that allows transitions
without consuming input Symbols. A transition that changes state
without consuming input is an $\varepsilon$ $move$. Each State $q$
defines an $\varepsilon$-\emph{closure}, $E(q)$, which is the set of
States that are reachable by $\varepsilon$ moves.

The Languages recognized by NFA-$\varepsilon$ are the same as NFA/DFA.

% --------------------------------------------------------------------
\subsection{Pushdown Automata}\label{subsec:pushdown_automata}
% --------------------------------------------------------------------

\emph{Pushdown Automata} add to Finite Automata a \emph{Stack} as a
parameter for choice of States and can recognize Context-free
Languages(\S\ref{subsec:context_free_language}).

Adding a second Stack makes a Pushdown Automata equal in power to a
Turing Machine.

Unlike Finite Automata, Deterministic PDA are not equivalent to
Nondeterministic PDA. The general representation for a PDA is
\[
    M = (Q, \Sigma, \Gamma, q_0, Z_0, F, \delta)
\]
where
\begin{itemize}
\item $Q$ is a finite set of States
\item $\Sigma$ is a finite set of input Symbols
\item $\Gamma$ is a finite set of Stack Symbols
\item $q_0 \in Q$ is the initial State
\item $Z_0 \in \Gamma$ is the initial Stack Symbol
\item $F \subseteq Q$ is the set of final Accept States
\item $\delta$ is the transition function $\delta: (Q \times (\Sigma
  \cup \{\varepsilon\}) \times \Gamma) \rightarrow \mathcal{P}(Q \times
  \Gamma^*)$
\end{itemize}

An element $(p,a,Z,q,\alpha)\in\delta$, with $M$ in State $p \in Q$,
input $a \in \Sigma \cup \{\varepsilon\}$, and top stack Symbol $Z \in
\Gamma$ results in the following:
\begin{enumerate}
\item read $a$
\item change state to $q$
\item pop $Z$
\item push $\alpha \in \Gamma^*$
\end{enumerate}

\subsubsection{Deterministic Pushdown Automata}\label{subsec:deterministic_pda}
\emph{Deterministic Pushdown Automata} have the restriction of only
one derivation per accepted input Word. This allows recognition of a
subset of Context-free Languages termed
Deterministic(\S\ref{subsec:deterministic_cfg}). Such Languages can be
parsed in linear time and Parsers for such Languages can be generated
automatically(\S\ref{subsec:parser_generator}).

A Pushdown Automata is Deterministic iff both
\begin{enumerate}
\item $\forall q \in Q, a \in \Sigma \cup {\varepsilon}, x \in
  \Gamma \vdash |\delta(q,a,x)| \leq 1$
\item $\forall q \in Q, x \in \Gamma \vdash |\delta(q,\varepsilon,x)|
  \neq 0 \Rightarrow \forall a \in \Sigma \vdash |\delta(q,a,x)|=0$
\end{enumerate}

% --------------------------------------------------------------------
\subsection{Linear Bounded Automata} \label{subsec:linear_bounded_automata}
% --------------------------------------------------------------------

\emph{Linear Bounded Automata} are Turing Machines restricted to an
input of finite length and are acceptors for Context-sensitive
Languages(\S\ref{subsec:context_sensitive}) which require that
Production Rules do not increase the size of the Expression as a
result; therefore the size of the input is sufficient for calculation.

% --------------------------------------------------------------------
\subsection{Turing Machines}\label{subsec:turing_machine}
% --------------------------------------------------------------------

A Turing Machine operates on an infinite \emph{storage tape}, which
acts as the read input as well as write storage. Pushdown Automata
with 2 Stacks are equivalent to Turing Machines.

\subsubsection{Nondeterministic Turing Machines}
\emph{Nondeterministic Turing Machines} (\emph{NTM}s) can be defined
as
    \[
        M = (Q, \Sigma, q_0, \sqcup, A, \delta)
    \]
where
\begin{itemize}
\item $Q$ is a finite set of States
\item $\Sigma$ is the finite Alphabet
\item $q_0 \in Q$ is the initial State
\item $\sqcup \in \Sigma$ is the blank Symbol
\item $F \subseteq Q$ is the set of final Accept States
\item $\delta \subseteq (Q \setminus F \times \Sigma) \times (Q \times
  \Sigma \times \{L,R\})$ and $L$ and $R$ are left and right shift.
\end{itemize}

The operation of $M$ in State $q_i$ and current read input $a_j$ is a
transition function, $q_i a_j \rightarrow q_{i1} a_{j1} d_k$. Note
that for an NTM, $\delta$ is a relation and more than one function can
exist for each possible input/State combination. The result is to
write the new Symbol $a_{j1}$ in the current position and shift the
storage left or right as specified by $d_k$, afterwards assuming State
$q_{i1}$.

\subsubsection{Deterministic Turing Machines}
\emph{Deterministic Turing Machines} (\emph{DTM}s) have one possible
output transition per unique input/State combination, thus $\delta$ is
a \emph{partial function} rather than a \emph{relation}:
\[
    \delta : Q \setminus F \times \Sigma \rightarrow Q \times
    \Sigma \times {L,R}
\]
The computational power of DTMs and NTMs is equivalent (they can solve
the same problems) as NTMs include DTMs as a special case. An
equivalent accepting computation in a DTM is generally exponential to
the length of the shortest accepting computation of an NTM.

\subsubsection{Probabilistic Turing Machines}
A \emph{Probabilistic Turing Machine} adds to transitions a
probability distribution (or a tape with random Symbols). It is an
open question whether this is more powerful than a DTM
($\mathsf{BPP}=\mathsf{P}$ ?)  but it is useful in the definition of
\emph{interactive proof systems}. %FIXME ref

\subsubsection{Multidimensional Turing Machines}
\emph{Multidimensional Turing Machines} allow for tapes of varying
topologies. This requires additional shift directions (i.e. $\{L, R, U,
D\}$ for a 2-dimensional tape) but does not increase the computing
power; even an $\infty$-\emph{dimensional} Turing Machine can be
simulated by a DTM.
