%%%%%%%%%%%%%%%%%%%%%%%%%%%%%%%%%%%%%%%%%%%%%%%%%%%%%%%%%%%%%%%%%%%%%%
%%%%%%%%%%%%%%%%%%%%%%%%%%%%%%%%%%%%%%%%%%%%%%%%%%%%%%%%%%%%%%%%%%%%%%
\part{Graph Theory}
%%%%%%%%%%%%%%%%%%%%%%%%%%%%%%%%%%%%%%%%%%%%%%%%%%%%%%%%%%%%%%%%%%%%%%
%%%%%%%%%%%%%%%%%%%%%%%%%%%%%%%%%%%%%%%%%%%%%%%%%%%%%%%%%%%%%%%%%%%%%%

% ====================================================================
\section{Graphs} \label{sec:graphs}
% ====================================================================

A \emph{Graph}, $G$, is an Ordered Pair of Sets, $G = (V,E)$ where:
\begin{enumerate}
\item $V$ is the Set of \emph{Vertices} of generic Elements
\item $E$ is the Set of \emph{Edges} which are a ordered pairs of
  Vertices, that is $E \subseteq V \times V$
\end{enumerate}
The Vertices of an Edge are \emph{Endpoints} of that Edge, and the
Edge is \emph{Incident} to those Vertices. Two Vertices $A,B$ are
\emph{Adjacent} when
\[
    (A,B) \in E \vee (B,A) \in E
\]
A \emph{Self-loop} is an Edge with both Endpoints the same Vertex.

The \emph{Eccentricity}, $\epsilon$, of a Vertex, $v$, is the greatest
distance between $v$ and any other Vertex.

The \emph{Radius} of a Graph is the minimum Eccentricity of any
Vertex in the Graph.

The \emph{Diameter} of a Graph is the maximum Eccentricity of any
Vertex in the Graph.



% --------------------------------------------------------------------
\subsection{Multigraph} \label{subsec:multigraph}
% --------------------------------------------------------------------

A \emph{Multigraph} (or \emph{Pseudo-graph}) is a Graph where $E$ is a
Multiset where Edges with Multiplicity $>1$ are called
\emph{Parallel}.



% --------------------------------------------------------------------
\subsection{Simple Graph} \label{subsec:simple_graph}
% --------------------------------------------------------------------

A \emph{Simple Graph} has no Self-loops and no Parallel Edges.



% --------------------------------------------------------------------
\subsection{Undirected Graph} \label{subsec:undirected_graph}
% --------------------------------------------------------------------

In an \emph{Undirected Graph}, it is required that for $A,B \in V$:
\[
    (A,B) \in E \leftrightarrow (B,A) \in E
\]
This is the definition of a Symmetric Relation
(\S\ref{subsec:binary_relation}) on $V$.

\subsubsection{Tree}\label{subsec:graph_tree}



% --------------------------------------------------------------------
\subsection{Directed Graph} \label{subsec:directed_graph}
% --------------------------------------------------------------------

\emph{Reachability}

\emph{Directed Path}

\emph{Dominator}

\emph{Postdominator}

\emph{Immediate Dominator} or \emph{Idom}

A \emph{Dominator Tree} is a Tree where each Node's Children are those
Nodes it Immediately Dominates.

\emph{Topological Ordering} is possible if and only if the Graph is a
\emph{Directed Acyclic Graph} (\S\ref{subsec:dag}).

A \emph{Knot} in a Directed Graph is a collection of Vertices and
Edges where every Vertex has outgoing Edges and all outgoing Edges
terminate at other Vertices of the Knot.



% --------------------------------------------------------------------
\subsubsection{Directed Acyclic Graph} \label{subsec:dag}
% --------------------------------------------------------------------

A Partial Ordering, $\leq_P$, of a DAG, $G$, may be defined as the
Reachability Relation on $G$ by taking the objects as the Vertices
$u,v,... \in V$ and defining $(u,v) \in \leq_P$ if and only if there
is a Directed Path from $u$ to $v$.

A \emph{Topological Ordering} or \emph{Topological Sorting} is a Total
Ordering, $\leq_T$, of the Vertices such that $\forall (u,v) \in E
\implies (u,v) \in \leq_T$. A Topological Ordering is a Linear
Extension of the DAG's Partial Ordering. % FIXME ref linear extension



% --------------------------------------------------------------------
\subsection{Complete Graph} \label{subsec:complete_graph}
% --------------------------------------------------------------------

% --------------------------------------------------------------------
\subsection{Subgraph} \label{subsec:subgraphs}
% --------------------------------------------------------------------

% --------------------------------------------------------------------
\subsection{Finite Graph} \label{subsec:finite_graph}
% --------------------------------------------------------------------

% --------------------------------------------------------------------
\subsection{Dipole and Cycle Graphs} \label{subsec:dipole_cycle_graph}
% --------------------------------------------------------------------



% ====================================================================
\section{Trees} \label{sec:trees}
% ====================================================================

% --------------------------------------------------------------------
\subsection{Forest} \label{subsec:forest}
% --------------------------------------------------------------------
