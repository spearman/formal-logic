%%%%%%%%%%%%%%%%%%%%%%%%%%%%%%%%%%%%%%%%%%%%%%%%%%%%%%%%%%%%%%%%%%%%%%
%%%%%%%%%%%%%%%%%%%%%%%%%%%%%%%%%%%%%%%%%%%%%%%%%%%%%%%%%%%%%%%%%%%%%%
\part{Graph Theory}\label{part:graph_theory}
%%%%%%%%%%%%%%%%%%%%%%%%%%%%%%%%%%%%%%%%%%%%%%%%%%%%%%%%%%%%%%%%%%%%%%
%%%%%%%%%%%%%%%%%%%%%%%%%%%%%%%%%%%%%%%%%%%%%%%%%%%%%%%%%%%%%%%%%%%%%%

Programming Languages: \fist see DOT graph description language (used by
Graphviz graph visualization software)



% ====================================================================
\section{Graph}\label{sec:graph}
% ====================================================================

A \emph{Graph}, $G$, is an Ordered Pair of Sets, $G = (V,E)$ where:
\begin{enumerate}
  \item $V$ is the Set of \emph{Vertices} (\S\ref{sec:vertex}) of
    generic Elements
  \item $E$ is the Set of \emph{Edges} (\S\ref{sec:edge}) which are a
    Ordered Pairs of Vertices, that is $E \subseteq V^2$
\end{enumerate}
The Vertices of an Edge are \emph{Endpoints} of that Edge, and the
Edge is \emph{Incident} to those Vertices. Two Vertices $a,b$ are
\emph{Adjacent} when:
\[
    (a,b) \in E \vee (b,a) \in E
\]
A \emph{Self-loop} is an Edge with both Endpoints the same Vertex.

$\Delta_V$ -- the Diagonal Subset of $V$, i.e. Pairs $(x,x)$

$V^2/\Delta_V$ -- Complement of $\Delta_V$, i.e. Pairs
$(x,y)$ with $x \neq y$

$\angleover{V}{2}$ -- Quotient Set of $V^2$ where $(x,y)$ is
Identified with $(y,x)$, i.e. the Set of Unordered Pairs $\{x,y\}$ of
Vertices

${V}\choose{2}$ -- Quotient Set of $V^2/\Delta_V$ where $(x,y)$ is
Identified with $(y,x)$, i.e. the Set of Unordered Pairs where $x \neq
y$

The \emph{Eccentricity}, $\epsilon$, of a Vertex, $v$, is the greatest
distance between $v$ and any other Vertex.

The \emph{Radius} of a Graph is the minimum Eccentricity of any
Vertex in the Graph.

The \emph{Diameter} of a Graph is the maximum Eccentricity of any
Vertex in the Graph.

The \emph{Geometric Realization} of a Graph $X$ is a Topological Space
$|X|$ %FIXME

for any Graph $\Gamma$, the Configuration Space
(\S\ref{sec:configuration_space}) $Conf_n(\Gamma)$ is an Eilenberg-Maclane
Space (\S\ref{sec:eilenberg_maclane_space}) of Type $K(\pi,1)$ and Strong
Deformation Retracts into a Subspace of Dimension $b(\Gamma)$ where $b(\Gamma)$
is the number of Vertices of Degree at least 3, and $Uconf_n(\Gamma)$ and
$Conf_n(\Gamma)$ Deformation Retract to Non-positively Curved Cubical Compexes
of Dimension at most $min(n,b(\gamma))$



% --------------------------------------------------------------------
\subsection{Vertex}\label{sec:vertex}
% --------------------------------------------------------------------

(or \emph{Node})



% --------------------------------------------------------------------
\subsection{Edge}\label{sec:edge}
% --------------------------------------------------------------------

(or \emph{Line})

a Directed (Ordered) Edge in a Directed Graph is called an Arc
(\S\ref{sec:arc})



\subsubsection{Adjacency}\label{sec:adjacency}

$x \sim y$

Adjacency Relation

Directed Loop Graph (\S\ref{sec:directed_loop_graph}) determined
entirely by the Adjacency Relation



\paragraph{Adjacency Matrix}\label{sec:adjacency_matrix}\hfill

Spectral Graph Theory (\S\ref{sec:spectral_graph_theory}): two Graphs are
Isospectral (\S\ref{sec:isospectral}) if their Adjacency Matrices have the same
Spectrum (\S\ref{sec:matrix_spectrum}); the Families of Complete Graphs
(\S\ref{sec:complete_graph}) and Finite Star-like Trees
(\S\ref{sec:starlike_tree}) are completely determined by their Spectrum



\subsubsection{Loop}\label{sec:graph_loop}

Edge with same base and end Vertex

Loop Graph (\S\ref{sec:loop_graph})

Directed Loop (\S\ref{sec:directed_loop})

\fist See also \emph{Loop} (Topology \S\ref{sec:loop})



\subsubsection{Parallel Edge}\label{sec:parallel_edge}

(or \emph{Multi-edge})

Multigraph (\S\ref{sec:multigraph})



% --------------------------------------------------------------------
\subsection{Path}\label{sec:graph_path}
% --------------------------------------------------------------------

\emph{Directed Path} (\S\ref{sec:directed_path}) in a Directed Graph
(\S\ref{sec:directed_graph})

\fist See also \emph{Path} (Topology \S\ref{sec:topology})

a Graph is Well-founded if it has no Infinite Path

Two Vertices are \emph{Connected} (\S\ref{sec:connectivity}) if there
is a Path between them. A Graph is Connected if there is a Path
between all pairs of Vertices.



\subsubsection{Connectivity}\label{sec:connectivity}

A pair of Vertices is \emph{Connected} if there is a Path
(\S\ref{sec:graph_path}) between them. A Graph is Connected if there
is a Path between all pairs of Vertices (there are no Unreachable
Vertices \S\ref{sec:reachability}).

If the underlying Undirected Graph of a Directed Graph is Connected
then the Directed Graph is \emph{Weakly Connected}
(\S\ref{sec:weakly_connected})

Strongly Connected (\S\ref{sec:strongly_connected})



\subsubsection{Connected Component}\label{sec:connected_component}

Equivalence Classes given by the Reachability
(\S\ref{sec:reachability}) Relation

\emph{Strongly Connected Component}
(\S\ref{sec:strongly_connected_component}) in a Directed Graph
(\S\ref{sec:directed_graph})

\fist a \emph{Connected Component} (\S\ref{sec:connected_space}) in Topology is
a Maximal Connected Subset (Ordered by Inclusion)



\subsubsection{Reachability}\label{sec:reachability}

Reachability in an Undirected Graph is an Equivalence Relation:
Connected Components are the Equivalence Classes

if there are no Unreachable Vertices the Graph is Connected
(\S\ref{sec:connectivity})

In a Directed Graph (\S\ref{sec:directed_graph}) $G = (V,E)$, the
Reachability Relation is the Transitive Closure of $E$. If $G$ is
Acyclic (\S\ref{sec:dag}) then the Reachability Relation is a Partial
Order (\S\ref{sec:partial_order}), i.e. it is Anti-symmetric
(\S\ref{sec:antisymmetric_relation}).

Transitive Reduction (\S\ref{sec:transitive_reduction_graph}) of a
Directed Graph



\subsubsection{Cycle}\label{sec:cycle}

\fist Cycle Rank (Undirected Graphs \S\ref{sec:cycle_rank}): Minimum number of
Edges that must be removed from the Graph to break all its Cycles making it
into a Tree (\S\ref{sec:tree_graph})

Directed Cycle \S\ref{sec:directed_cycle}



\paragraph{Closed Walk}\label{sec:closed_walk}\hfill

\paragraph{Simple Cycle}\label{sec:simple_cycle}\hfill

\paragraph{Circuit}\label{sec:circuit}\hfill

\paragraph{Girth}\label{sec:girth}\hfill

\paragraph{Chordless Cycle}\label{sec:chordless_cycle}\hfill

\paragraph{Peripheral Cycle}\label{sec:peripheral_cycle}\hfill

\paragraph{Cycle Space}\label{sec:cycle_space}\hfill



% --------------------------------------------------------------------
\subsection{Subgraph}\label{sec:subgraph}
% --------------------------------------------------------------------

\subsubsection{Induced Subgraph}\label{sec:induced_subgraph}



% --------------------------------------------------------------------
\subsection{Finite Graph}\label{sec:finite_graph}
% --------------------------------------------------------------------

$V$ and $E$ are Finite Sets



% --------------------------------------------------------------------
\subsection{Well-founded Graph}\label{sec:wellfounded_graph}
% --------------------------------------------------------------------

no Infinite Paths (\S\ref{sec:graph_path})



% --------------------------------------------------------------------
\subsection{Simple Graph}\label{sec:simple_graph}
% --------------------------------------------------------------------

A \emph{Simple Graph} has no Self-loops and no Parallel Edges.

Injective Function $d : E \hookrightarrow {V \choose 2}$

Simple Directed Graph (\S\ref{sec:simple_directed})



\subsubsection{Cycle Graph}\label{sec:cycle_graph}

\emph{Cycle Graph} (or \emph{Circular Graph}) consists of a single
Cycle (\S\ref{sec:cycle})

$C_n$ -- Cycle Graph with $n$ Vertices

the Cycle Graph $C_n$ is Dual to the Dipole Graph
(\S\ref{sec:dipole_graph}) $D_n$



\subsubsection{Sparse Graph}\label{sec:sparse_graph}

\paragraph{Expander Graph}\label{sec:expander_graph}\hfill

%FIXME move section ?

Finite, Undirected Multigraph in which every Subset of the Vertices that is not
``too large'' has a ``large'' Boundary

Strong Connectivity Properties quantified using Vertex, Edge or Spectral
Expansion

Combinatorics (Part \ref{part:combinatorics})

P2P Overlay Topologies



\subsubsection{Dense Graph}\label{sec:dense_graph}

\paragraph{Graphon}\label{sec:graphon}

``Continuous Graphs''



% --------------------------------------------------------------------
\subsection{Loop Graph}\label{sec:loop_graph}
% --------------------------------------------------------------------

Self-loops (\S\ref{sec:graph_loop}) allowed, no Parallel Edges

Injective Function $d : E \hookrightarrow \angleover{V}{2}$

Undirected Loop Graph is given by a Symmetric Relation on $V$

Directed Loop Graph (\S\ref{sec:directed_loop_graph}) $d : E
\hookrightarrow V^2$ -- arbitrary Binary Relation on $V$



% --------------------------------------------------------------------
\subsection{Multigraph}\label{sec:multigraph}
% --------------------------------------------------------------------

A \emph{Multigraph} is a Graph where $E$ is a Multiset where Edges
with Multiplicity $>1$ are called \emph{Parallel}
(\S\ref{sec:parallel_edge}).

no Self-loops, Parallel Edges allowed

arbitrary Function $d : E \rightarrow {V \choose 2}$

Directed Multigraph (\S\ref{sec:directed_multigraph}) $d : E
\rightarrow V^2 / \Delta_V$



\subsubsection{Dipole Graph}\label{sec:dipole_graph}

two Vertices connected with any number of Parallel Edges

$n$ Edges

Order-$n$ Dipole Graph, $D_n$

$D_n$ is Dual to the Cycle Graph (\S\ref{sec:cycle_graph}) $C_n$



% --------------------------------------------------------------------
\subsection{Pseudograph}\label{sec:pseudograph}
% --------------------------------------------------------------------

Self-loops and Parallel Edges allowed

arbitrary Function $d : E \rightarrow \angleover{V}{2}$

Directed Pseudograph (Quiver \S\ref{sec:quiver}) $d : E \rightarrow
V^2$



% --------------------------------------------------------------------
\subsection{Multipartite Graph}\label{sec:multipartite_graph}
% --------------------------------------------------------------------

\subsubsection{Bigraph}\label{sec:bigraph}

or \emph{Bipartite Graph}



\paragraph{Complete Bigraph}\label{sec:complete_bigraph}\hfill

\subparagraph{Utility Graph}\label{sec:utility_graph}\hfill

Complete Bigraph $K_{3,3}$

Non-planar: the Planar Graphs (\S\ref{sec:planar_graph}) are exactly the Graphs
that do not contain the Utility Graph or the Complete Graph $K_5$



\subparagraph{Star}\label{sec:star}\hfill

Complete Bigraph $K_{1,k}$

is a Tree with one internal Node and $k$ Leaves



\subparagraph{Biclique}\label{sec:biclique}\hfill

Complete Bipartite Subgraph

cf. Clique (Complete Subgraph \S\ref{sec:clique})



% --------------------------------------------------------------------
\subsection{Weighted Graph}\label{sec:weighted_graph}
% --------------------------------------------------------------------

some algorithms: \url{https://blog.evjang.com/2018/08/dijkstras.html}



% ====================================================================
\section{Undirected Graph}\label{sec:undirected_graph}
% ====================================================================

In an \emph{Undirected Graph}, it is required that for $a,b \in V$:
\[
    (a,b) \in E \leftrightarrow (b,a) \in E
\]
This is the definition of a Symmetric Relation
(\S\ref{sec:symmetric_relation}) on $V$.



% --------------------------------------------------------------------
\subsection{Rank}\label{sec:rank}
% --------------------------------------------------------------------

\subsection{Rank Polynomial}\label{sec:rank_polynomial}



% --------------------------------------------------------------------
\subsection{Cycle Rank}\label{sec:cycle_rank}
% --------------------------------------------------------------------

Minimum number of Edges that must be removed from the Graph to break all its
Cycles (\S\ref{sec:cycle}) making it into a Tree (\S\ref{sec:tree_graph})

\fist cf. Feedback Arc Set (\S\ref{sec:feedback_arc_set})

\emph{Cyclomatic Complexity} -- software metric indicating the ``complexity''
of a program by measuring the number of linearly independent \emph{Paths} in
the Control-flow Graph (\S\ref{sec:control_flow})



% --------------------------------------------------------------------
\subsection{Complete Graph}\label{sec:complete_graph}
% --------------------------------------------------------------------

Simple (\S\ref{sec:simple_graph}) Undirected Graph where every Pair of
distinct Vertices is Connected by a unique Edge.

Complete Digraph (\S\ref{sec:complete_digraph})

Orientation (assigning an Order on each Edge) of a Complete Graph
gives a Tournament (\S\ref{sec:tournament})

every Complete Graph is its own Maximal Clique (\S\ref{sec:maximal_clique})

the Planar Graphs (\S\ref{sec:planar_graph}) are exactly the Graphs that do not
contain the Complete Graph $K_5$ or the Utility Graph
(\S\ref{sec:utility_graph}) $K_{3,3}$

the Family of Graphs of Complete Graphs are completely determined by their
Spectra (\S\ref{sec:graph_spectrum})



\subsubsection{Clique}\label{sec:clique}

a Subset of Vertices whose Induced Subgraph (\S\ref{sec:induced_subgraph}) is
Complete

cf. \emph{Biclique} (Complete Bipartite Subgraph \S\ref{sec:biclique})



\paragraph{Maximal Clique}\label{sec:maximal_clique}\hfill

a \emph{Maximal Clique} is a Clique that cannot be extended by including
another Vertex

every Complete Graph is its own Maximal Clique



% --------------------------------------------------------------------
\subsection{Tree Graph}\label{sec:tree_graph}
% --------------------------------------------------------------------

any two Vertices are connected by exactly one Path (\S\ref{sec:path})-- no
Cycles (\S\ref{sec:cycle})

Cycle Rank (\S\ref{sec:cycle_rank}): Minimum number of Edges that must be
removed from a Graph to break all its Cycles (\S\ref{sec:cycle}) making it into
a Tree

Rooted Tree (\S\ref{sec:rooted_tree})

$1$-dimensional CW-complex (\S\ref{sec:cw_complex})

the Orientation of an Undirected Tree is a \emph{Polytree}
(\S\ref{sec:polytree})

a Tree with one Internal Node and $k$ Leaves is a Complete Bigraph called a
\emph{Star} (\S\ref{sec:star})



\subsubsection{Forest}\label{sec:forest}

\subsubsection{Starlike Tree}\label{sec:starlike_tree}

the Family of Graphs of Finite Starlike Trees are completely determined by
their Spectra (\S\ref{sec:graph_spectrum})



\subsubsection{Spanning Tree}\label{sec:spanning_tree}

cf. Spanning Tree Protocol (STP)



% --------------------------------------------------------------------
\subsection{Circulant Graph}\label{sec:circulant_graph}
% --------------------------------------------------------------------

% --------------------------------------------------------------------
\subsection{Unit Distance Graph}\label{sec:unit_distance_graph}
% --------------------------------------------------------------------

a Matchstick Graph (\S\ref{sec:matchstick_graph}) is a Unit Distance Graph that
is also a Planar Graph (\S\ref{sec:planar_graph})



% --------------------------------------------------------------------
\subsection{Matchstick Graph}\label{sec:matchstick_graph}
% --------------------------------------------------------------------

simultaneously a Unit Distance Graph (\S\ref{sec:unit_distance_graph}) and a
Planar Graph (\S\ref{sec:planar_graph})



% ====================================================================
\section{Directed Graph}\label{sec:directed_graph}
% ====================================================================

(or \emph{Digraph})

$(V,A)$

Vertices $V$: Set of Node Elements

Arrows $A$: Set of Ordered Pairs of Vertices

Directed Graph
$d : E \hookrightarrow V^2 / \Delta_V$

Directed Loop Graph (\S\ref{sec:directed_loop}) $d : E \hookrightarrow
V^2$

Directed Multigraph (\S\ref{sec:directed_multigraph}) $d : E
\rightarrow V^2 / \Delta_V$

Directed Pseudograph (Quiver \S\ref{sec:quiver}) $d : E \rightarrow
V^2$



\emph{Dominator}

\emph{Postdominator}

\emph{Immediate Dominator} or \emph{Idom}

A \emph{Dominator Tree} is a Tree where each Node's Children are those
Nodes it Immediately Dominates.

\emph{Topological Ordering} is possible if and only if the Graph is a
\emph{Directed Acyclic Graph} (\S\ref{sec:dag}).

A \emph{Knot} in a Directed Graph is a collection of Vertices and
Edges where every Vertex has outgoing Edges and all outgoing Edges
terminate at other Vertices of the Knot.



% --------------------------------------------------------------------
\subsection{Arc}\label{sec:arc}
% --------------------------------------------------------------------

Ordered Edge (\S\ref{sec:edge})



\subsubsection{Directed Loop}\label{sec:directed_loop}

Directed Loop (\S\ref{sec:directed_loop})



% --------------------------------------------------------------------
\subsection{Directed Path}\label{sec:directed_path}
% --------------------------------------------------------------------

Path (\S\ref{sec:graph_path})



% --------------------------------------------------------------------
\subsection{Directed Cycle}\label{sec:directed_cycle}
% --------------------------------------------------------------------

Cycle \S\ref{sec:cycle}

Directed Acyclic Graph (\S\ref{sec:dag}) has no Directed Cycles

a Directed Cycle Graph is a Cayley Graph (\S\ref{sec:cayley_graph})
for a Cyclic Group (\S\ref{sec:cyclic_group})



% --------------------------------------------------------------------
\subsection{In/Outdegree}\label{sec:inoutdegree}
% --------------------------------------------------------------------

\emph{Source}: Indegree $0$

\emph{Sink}: Outdegree $0$

\emph{Internal}: neither a Source nor a Sink


Trees: Branching Factor (\S\ref{sec:branching_factor})



\subsubsection{Degree Sequence}\label{sec:degree_sequence}

Isomorphic Directed Graphs have the same Degree Sequence



% --------------------------------------------------------------------
\subsection{Weakly Connected}\label{sec:weakly_connected}
% --------------------------------------------------------------------

A Directed Graph is \emph{Weakly Connected} if its underlying
Undirected Graph is Connected (\S\ref{sec:connectivity})



% --------------------------------------------------------------------
\subsection{Strongly Connected}\label{sec:strongly_connected}
% --------------------------------------------------------------------

A pair of Vertices $u,v \in V$ are \emph{Strongly Connected} if there
is a Directed Path in each direction between them, i.e. $u$ is
Reachable (\S\ref{sec:reachability}) from $v$ and $v$ is Reachable
from $u$.

The Strongly-connected (Binary) Relation is an Equivalence Relation
and the Equivalence Classes of Vertices are \emph{Strongly Connected
  Component} (\S\ref{sec:strongly_connected_component})

A Directed Graph is Strongly Connected if all pairs of Vertices are
Strongly Connected.



\subsubsection{Strongly Connected Component}
\label{sec:strongly_connected_component}

A \emph{Strongly Connected Component} is an Equivalence Class of the
Strongly-connected Relation on the Vertices of a Directed Graph.
Strongly Connected Components are then Subgraphs in which every Vertex
is Reachable (\S\ref{sec:reachability}) from every other Vertex.

Connected Component (\S\ref{sec:connected_component})



% --------------------------------------------------------------------
\subsection{Complete Digraph}\label{sec:complete_digraph}
% --------------------------------------------------------------------

every pair of distinct Vertices is Connected by a pair of unique
Edges:
\[
  v,u \in V \Rightarrow \exists! (v,u), (u,v) \in E
\]



% --------------------------------------------------------------------
\subsection{Transitive Digraph}\label{sec:transitive_digraph}
% --------------------------------------------------------------------

(wolfram): an Unlabeled Transitive Digraph is called a \emph{Digraph Topology}
(\S\ref{sec:digraph_topology})



% --------------------------------------------------------------------
\subsection{Transitive Reduction}
\label{sec:transitive_reduction_graph}
% --------------------------------------------------------------------

Transitive Reduction (\S\ref{sec:transitive_reduction}) of the
Reachability Relation (\S\ref{sec:reachability})

not uniquely defined for Directed Graphs with Cycles

Transitive Reduction of a Directed Acyclic Graph (\S\ref{sec:dag}) $G
= (V,E)$ is unique and consists of the Edges of $G$ that from the only
Path between their Endpoints and so it is always a Subgraph of $G$.
This generates the Covering Relation (\S\ref{sec:covering_relation})
of the Partial Order defined by the Reachability Relation.



% --------------------------------------------------------------------
\subsection{Simple Directed Graph}\label{sec:simple_directed}
% --------------------------------------------------------------------

Simple Graph (\S\ref{sec:simple_graph})

Irreflexive Relation on $V$



% --------------------------------------------------------------------
\subsection{Directed Loop Graph}\label{sec:directed_loop_graph}
% --------------------------------------------------------------------

Loop Graph (\S\ref{sec:loop_graph})

arbitrary Binary Relation on $V$

completely determined by Adjacency Relation (\S\ref{sec:adjacency})



% --------------------------------------------------------------------
\subsection{Directed Multigraph}\label{sec:directed_multigraph}
% --------------------------------------------------------------------

Multigraph (\S\ref{sec:multigraph})

the Category of Directed Multigraphs is an example of an Adhesive Category
(\S\ref{sec:adhesive_category})



% --------------------------------------------------------------------
\subsection{Quiver}\label{sec:quiver}
% --------------------------------------------------------------------

Directed Multigraph

(FIXME: is this the same as a directed multigraph ???)

Directed Pseudograph (\S\ref{sec:pseudograph})

Paracategory (\S\ref{sec:paracategory})

nLab:

it is always possible to Interpret any kind of Graph as a Quiver



% --------------------------------------------------------------------
\subsection{Directed Acyclic Graph}\label{sec:dag}
% --------------------------------------------------------------------

no Directed Cycles (\S\ref{sec:directed_cycle})

A Partial Ordering, $\leq_P$, of a DAG, $G$, may be defined as the
\emph{Reachability Relation} on $G$ by taking the objects as the
Vertices $u,v,... \in V$ and defining $(u,v) \in \leq_P$ if and only
if there is a Directed Path from $u$ to $v$.

A \emph{Topological Ordering} or \emph{Topological Sorting} is a Total
Ordering, $\leq_T$, of the Vertices such that $\forall (u,v) \in E
\Rightarrow (u,v) \in \leq_T$. A Topological Ordering is a Linear
Extension of the DAG's Partial Ordering. % FIXME ref linear extension

Dependency Graph between Types in a Typing Context (\S\ref{sec:type_context})
are DAGs

\fist Boolean Networks (\S\ref{sec:boolean_network}) -- dependency structure



% --------------------------------------------------------------------
\subsection{Feedback Arc Set}\label{sec:feedback_arc_set}
% --------------------------------------------------------------------

\fist cf. Cycle Rank (Undirected Graphs \S\ref{sec:cycle_rank}): Minimum number
of Edges that must be removed from the Graph to break all its Cycles
(\S\ref{sec:cycle}) making it into a Tree (\S\ref{sec:tree_graph})




% --------------------------------------------------------------------
\subsection{Symmetric Digraph}\label{sec:symmetric_digraph}
% --------------------------------------------------------------------

Inverse for every Arc

a Simple Symmetric Digraph is equivalent to an Undirected Graph with
Edges replaced by pairs of Inverse Arcs



% --------------------------------------------------------------------
\subsection{Oriented Graph}\label{sec:oriented_graph}
% --------------------------------------------------------------------

\subsubsection{Strong Orientation}\label{sec:strong_orientation}

results in a Strongly Connected Graph (\S\ref{sec:strongly_connected})



\subsubsection{Acyclic Orientation}\label{sec:acyclic_orientation}

results in a Directed Acyclic Graph (\S\ref{sec:dag})



\subsubsection{Transitive Orientation}\label{sec:transitive_orientation}

resulting Directed Graph is its own Transitive Closure
(\S\ref{sec:transitive_closure})



\subsubsection{Eulerian Orientation}\label{sec:eulerian_orientation}

Orientation in which each Vertex has an equal In-degree and Out-degree



\subsubsection{Tournament}\label{sec:tournament}

Orientation (assignment of an Order to each Edge) of a Complete Graph
(\S\ref{sec:complete_graph})



\subsubsection{Polytree}\label{sec:polytree}

Orientation of an Undirected Tree (\S\ref{sec:tree_graph})



% --------------------------------------------------------------------
\subsection{Tolerance Relation}\label{sec:tolerance_relation}
% --------------------------------------------------------------------

%FIXME possibly move

\subsubsection{Dependency Relation}\label{sec:dependency_relation}

Finite Tolerance Relation

Independency Relation

Trace Monoid (\S\ref{sec:trace_monoid})



\subsubsection{Dependency Graph}\label{sec:dependency_graph}

The Monoid (\S\ref{sec:monoid}) of Dependency Graphs is Isomorphic to
Trace Monoids (\S\ref{sec:trace_monoid}), i.e. Free
Partially-commutative Monoids.



% --------------------------------------------------------------------
\subsection{De Bruijn Graph}\label{sec:debruijn_graph}
% --------------------------------------------------------------------

represents overlaps between Sequences of Symbols

graphical depiction of the construction of all possible Transitions
driven by Strings in the Free Group of a Semiautomaton
(\S\ref{sec:semiautomaton})

%FIXME clarify which free group above



% --------------------------------------------------------------------
\subsection{Discrete Laplace Operator}\label{sec:discrete_laplace}
% --------------------------------------------------------------------

Laplace Operator (\S\ref{sec:laplace_operator})

Dynamical Systems (\S\ref{sec:dynamical_system})

\emph{Boundary}: Subset of Vertices where the Function has specified
Values called the \emph{Boundary Condition}

\emph{Interior}: Non-boundary Vertices

$\gamma$: Function assigning a Value to each Edge such that for every
Vertex the Sum of its Outgoing Edges is $1$

\[
  \Delta(u)(v) = \sum_{w \in N(v)} \gamma(v,w)(u(v) - u(w)) = 0
\]
where $N(v)$ are the Neighbors of $v$


Discrete Mean Value Property:
\[
  u(v) = \sum_{w \in N(v)} \gamma (v,w)u(w)
\]

Any Function $\alpha$ on a Graph which satisfies the Mean Value
Property also satisfies the Discrete Laplacian



\subsubsection{Laplacian Matrix}\label{sec:laplacian_matrix}

Finite-dimensional Graph

\fist Spectral Graph Theory (\S\ref{sec:spectral_graph_theory})



% --------------------------------------------------------------------
\subsection{Flow Graph}\label{sec:flow_graph}
% --------------------------------------------------------------------

\fist Data Flow (\S\ref{sec:data_flow}), Control Flow
(\S\ref{sec:control_flow}), Signal FLow (\S\ref{sec:signal_flow})



\subsubsection{Directed Network}\label{sec:directed_network}

\fist Network (\S\ref{sec:network}): a Graph in which Nodes and/or Edges have
\emph{Attributes}



\subsubsection{Closed Flow Graph}\label{sec:closed_flowgraph}

\subsubsection{Open Flow Graph}\label{sec:open_flowgraph}

\paragraph{Linear Signal Flow Graph}\label{sec:linear_signal_flow}\hfill

Linear Time-invariant Systems

Coates Graph ??? %FIXME



\paragraph{State Transition SFG}\label{sec:state_transition_sfg}\hfill

(or \emph{State Diagram})



% ====================================================================
\section{Pointed Graph}\label{sec:pointed_graph}
% ====================================================================

A \emph{Pointed Graph} (or \emph{Rooted Graph}) is a Graph where one
Vertex distinguished as its \emph{Point}.

Undirected or Directed

\emph{Mostowski's Collapsing Lemma}: \cite{aczel88}

Every Well-founded Graph has a Unique Decoration %FIXME xref, move this?



% --------------------------------------------------------------------
\subsection{Accessible Pointed Graph}\label{sec:accessible_pointed}
% --------------------------------------------------------------------

An \emph{Accessible Pointed Graph} (or \emph{Flow Graph}) is a Pointed
Graph where all other Vertices are Reachable from the Root Vertex.

Anti-foundation Axioms (\S\ref{sec:anti_foundation})

\emph{Unfolding} an Accessible Pointed Graph to a Rooted Tree
(\S\ref{sec:rooted_tree})



% --------------------------------------------------------------------
\subsection{Rooted Tree}\label{sec:rooted_tree}
% --------------------------------------------------------------------

A \emph{Rooted Tree} is an Accessible Pointed Graph such that the Path
from the Point (now called the \emph{Root}) to another Vertex is
always Unique.

Accessible Pointed Graph may be \emph{Unfolded} to a Rooted Tree

Tree (\S\ref{sec:tree_graph})



\subsubsection{Branching Factor}\label{sec:branching_factor}

the number of Leaf Nodes for a Rooted Tree of Branching Factor $b$ and
Depth $d$ is $b^d$ %FIXME



% --------------------------------------------------------------------
\subsection{Multiply Rooted Graph}\label{sec:multiply_rooted}
% --------------------------------------------------------------------



% ====================================================================
\section{Cubic Graph}\label{sec:cubic_graph}
% ====================================================================

\subsubsection{Snark}\label{sec:snark}



% ====================================================================
\section{Symmetric Graph}\label{sec:symmetric_graph}
% ====================================================================

% ====================================================================
\section{Labelled Graph}\label{sec:labelled_graph}
% ====================================================================

\fist Global Graphs (Graphical Choreographies
\S\ref{sec:graphical_choreography})



% ====================================================================
\section{Hypergraph}\label{sec:hypergraph}
% ====================================================================

A \emph{Hypergraph} is a generalization of a Graph in which Edges can
join any number of Vertices.

\fist Hypergraph Category (\S\ref{sec:hypergraph_category}) --
compositional structure



% ====================================================================
\section{Graph Coloring}\label{sec:graph_coloring}
% ====================================================================

% --------------------------------------------------------------------
\subsubsection{Chromatic Number}\label{sec:chromatic_number}
% --------------------------------------------------------------------

%FIXME



% ====================================================================
\section{Geometric Graph Theory}\label{sec:geometric_graph_theory}
% ====================================================================

% --------------------------------------------------------------------
\subsection{Graph Drawing}\label{sec:graph_drawing}
% --------------------------------------------------------------------

\subsubsection{Network Diagram}\label{sec:network_diagram}

%FIXME equivalent with topological graph ?

Open Systems (\S\ref{sec:network_diagram})



% ====================================================================
\section{Topological Graph Theory}\label{sec:topological_graph_theory}
% ====================================================================

% --------------------------------------------------------------------
\subsection{Topological Graph}\label{sec:topological_graph}
% --------------------------------------------------------------------

%FIXME equivalent with network diagram ?



% --------------------------------------------------------------------
\subsection{Planar Graph}\label{sec:planar_graph}
% --------------------------------------------------------------------

can be drawn in ``the Plane'' without any crossing Edges (they meet only at
their Endpoints)

%FIXME: euclidean plane, cartesian plane ???

the Planar Graphs are exactly those Graphs that contain neither the Utility
Graph (\S\ref{sec:utility_graph}) $K_{3,3}$, nor the Complete Graph
(\S\ref{sec:complete_graph}) $K_5$

a Matchstick Graph (\S\ref{sec:matchstick_graph}) is a Planar Graph that
is also a Unit Distance Graph (\S\ref{sec:unit_distance_graph})



% --------------------------------------------------------------------
\subsection{Toroidal Graph}\label{sec:toroidal_graph}
% --------------------------------------------------------------------

Torus (\S\ref{sec:torus})

Chromatic Number (\S\ref{sec:chromatic_number}) at most $7$, e.g. the Complete
Graph $K_7$



% ====================================================================
\section{Spectral Graph Theory}\label{sec:spectral_graph_theory}
% ====================================================================

study of Properties of a Graph in relation to the Characteristic Polynomial
(\S\ref{sec:characteristic_polynomial}), Eigenvalues (\S\ref{sec:eigenvalue}),
and Eigenvectors (\S\ref{sec:eigenvector}) of \emph{Matrices} associated with
the Graph, such as the Adjacency Matrix (\S\ref{sec:adjacency_matrix}) or
Laplacian Matrix (\S\ref{sec:laplacian_matrix})



% --------------------------------------------------------------------
\subsection{Graph Spectrum}\label{sec:graph_spectrum}
% --------------------------------------------------------------------

the Spectrum (Set of Eigenvalues \S\ref{sec:matrix_spectrum}) of the Adjacency
Matrix (FIXME: other associated matrices ???)

the Graphs in the Family of Complete Graphs (\S\ref{sec:complete_graph}) and
the Family of Finite Starlike Trees (\S\ref{sec:starlike_tree}) are completely
determined by their Spectra



\subsubsection{Isospectral}\label{sec:isospectrum}

two Graphs with same Adjacency Matrix Spectra (\S\ref{sec:matrix_spectrum})



% ====================================================================
\section{Network Theory}\label{sec:network_theory}
% ====================================================================

Complex Systems (\S\ref{sec:complex_system})

2016 - Baez - \emph{Network Theory} (Articles) -
\url{http://math.ucr.edu/home/baez/networks/}

Petri Nets (\S\ref{sec:petri_net})



% --------------------------------------------------------------------
\subsection{Network}\label{sec:network}
% --------------------------------------------------------------------

A \emph{Network} is a Graph in which Nodes and/or Edges have \emph{Attributes}
(Appendix \S\ref{sec:attribute}) \fist cf. Attributes (Database Theory
\S\ref{sec:database_attribute}).

\fist Directed Network (\S\ref{sec:directed_network})

\asterism

2018 - Baez,Coya,Rebro - \emph{Props in Network Theory}

\url{https://golem.ph.utexas.edu/category/2018/04/props_in_network_theory.html}:

\emph{Props} (PROPs: PROduct and Permutations Categories
\S\ref{sec:prop_category}) -- Strict Symmetric Monoidal Categories with Natural
Numbers as Objects and Addition as the Tensor Product

each kind of Network corresponds to a Prop and each Network of
that kind is a \emph{Morphism} in the Prop

a \emph{Network} with $m$ \emph{Inputs} and $n$ \emph{Outputs} is a Morphism
from $m$ to $n$

\emph{Composition} of Networks (Morphisms) connects Networks together \emph{in
  series}

\emph{Tensoring} of Networks (Morphisms) puts Networks together \emph{in
  parallel}

Black-boxing Theorem -- \emph{Systems} (\S\ref{sec:systems_theory}) can be seen
as Morphisms in a Category where Composition uses the Outputs of one System as
the Inputs of another

Morphisms of Props clarify relation between \emph{Circuit Diagrams} and
\emph{Signal-flow Diagrams} (\S\ref{sec:signal_flow}) in Control Theory

cf. Bond Graphs (\S\ref{sec:bond_graph}) for Dynamical Systems

``Black-box Functor'': Category of Circuits $\rightarrow$ Category of
Symplectic Vector Spaces (Baez2015 - \emph{Passive Linear Networks})

in an Electrical Circuit, there is associated to each ``\emph{Wire}'' the
Attributes $\phi$ Potential and $I$ Current; Black-boxing a Circuit records
only the Relation imposed between Variables on its Input and Output Wires, that
is a Relation between Even-dimensional Vector Spaces (Pairs $(\phi,I)$); in
this case these Vector Spaces are \emph{Symplectic}
(\S\ref{sec:symplectic_vectorspace}), i.e. they are equipped with a
Nondegenerate Alternating (Skew-symmetric) Bilinear Form
called a \emph{Symplectic Form} (\S\ref{sec:symplectic_form})

the Black-box Functor respects the Symplectic Structure: it is a
\emph{'Lagrangian' Relation} (\S\ref{sec:lagrangian_system})

for any sort of System goverened by a Minimum Principle, Black-boxing should
give a Functor to some Category where the Morphisms are Lagrangian Relations

$\blacksquare : \cat{Circ}_k \rightarrow \cat{LagRel}_k$

Black-boxing Non-linear Circuits, i.e. including Voltage and Current Sources,
gives Lagrangian Affine Relations between Symplectic Vector Spaces



% --------------------------------------------------------------------
\subsection{Boolean Network}\label{sec:boolean_network}
% --------------------------------------------------------------------

%FIXME: move to dynamical systems ???

a particular kind of Sequential Dynamical System
(\S\ref{sec:sequential_dynamical_system})

2016 - Walker, Kim, Davies - \emph{The Information Architecture of the Cell}
