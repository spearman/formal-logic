%%%%%%%%%%%%%%%%%%%%%%%%%%%%%%%%%%%%%%%%%%%%%%%%%%%%%%%%%%%%%%%%%%%%%%
%%%%%%%%%%%%%%%%%%%%%%%%%%%%%%%%%%%%%%%%%%%%%%%%%%%%%%%%%%%%%%%%%%%%%%
\part{Graph Theory}\label{part:graph_theory}
%%%%%%%%%%%%%%%%%%%%%%%%%%%%%%%%%%%%%%%%%%%%%%%%%%%%%%%%%%%%%%%%%%%%%%
%%%%%%%%%%%%%%%%%%%%%%%%%%%%%%%%%%%%%%%%%%%%%%%%%%%%%%%%%%%%%%%%%%%%%%

% ====================================================================
\section{Graph}\label{sec:graph}
% ====================================================================

A \emph{Graph}, $G$, is an Ordered Pair of Sets, $G = (V,E)$ where:
\begin{enumerate}
  \item $V$ is the Set of \emph{Vertices} (\S\ref{sec:vertex}) of
    generic Elements
  \item $E$ is the Set of \emph{Edges} (\S\ref{sec:edge}) which are a
    Ordered Pairs of Vertices, that is $E \subseteq V^2$
\end{enumerate}
The Vertices of an Edge are \emph{Endpoints} of that Edge, and the
Edge is \emph{Incident} to those Vertices. Two Vertices $a,b$ are
\emph{Adjacent} when:
\[
    (a,b) \in E \vee (b,a) \in E
\]
A \emph{Self-loop} is an Edge with both Endpoints the same Vertex.

$\Delta_V$ -- the Diagonal Subset of $V$, i.e. Pairs $(x,x)$

$V^2/\Delta_V$ -- Complement of $\Delta_V$, i.e. Pairs
$(x,y)$ with $x \neq y$

$\angleover{V}{2}$ -- Quotient Set of $V^2$ where $(x,y)$ is
Identified with $(y,x)$, i.e. the Set of Unordered Pairs $\{x,y\}$ of
Vertices

${V}\choose{2}$ -- Quotient Set of $V^2/\Delta_V$ where $(x,y)$ is
Identified with $(y,x)$, i.e. the Set of Unordered Pairs where $x \neq
y$

The \emph{Eccentricity}, $\epsilon$, of a Vertex, $v$, is the greatest
distance between $v$ and any other Vertex.

The \emph{Radius} of a Graph is the minimum Eccentricity of any
Vertex in the Graph.

The \emph{Diameter} of a Graph is the maximum Eccentricity of any
Vertex in the Graph.



% --------------------------------------------------------------------
\subsection{Vertex}\label{sec:vertex}
% --------------------------------------------------------------------

(or \emph{Node})



% --------------------------------------------------------------------
\subsection{Edge}\label{sec:edge}
% --------------------------------------------------------------------

(or \emph{Line})

an Directed (Ordered) Edge in a Directed Graph is called an Arc
(\S\ref{sec:arc})



\subsubsection{Loop}\label{sec:graph_loop}

Edge with same base and end Vertex

Loop Graph (\S\ref{sec:loop_graph})

Directed Loop (\S\ref{sec:directed_loop})

\fist See also \emph{Loop} (Topology \S\ref{sec:loop})



\subsubsection{Adjacency}\label{sec:adjacency}

$x \sim y$

Adjacency Relation

Directed Loop Graph (\S\ref{sec:directed_loop_graph}) determined
entirely by the Adjacency Relation

Adjacency Matrix



% --------------------------------------------------------------------
\subsection{Path}\label{sec:graph_path}
% --------------------------------------------------------------------

\emph{Directed Path} (\S\ref{sec:directed_path}) in a Directed Graph
(\S\ref{sec:directed_graph})

\fist See also \emph{Path} (Topology \S\ref{sec:topology})



\subsubsection{Reachability}\label{sec:reachability}

Reachability in an Undirected Graph is an Equivalence Relation:
Connected Components are the Equivalence Classes

In a Directed Graph (\S\ref{sec:directed_graph}) $G = (V,E)$, the
Reachability Relation is the Transitive Closure of $E$. If $G$ is
Acyclic (\S\ref{sec:dag}) then the Reachability Relation is a Partial
Order (\S\ref{sec:partial_order}), i.e. it is Anti-symmetric
(\S\ref{sec:antisymmetric_relation}).

Transitive Reduction (\S\ref{sec:transitive_reduction_graph}) of a
Directed Graph



% --------------------------------------------------------------------
\subsection{Subgraph}\label{sec:subgraphs}
% --------------------------------------------------------------------

% --------------------------------------------------------------------
\subsection{Finite Graph}\label{sec:finite_graph}
% --------------------------------------------------------------------

$V$ and $E$ are Finite Sets



% --------------------------------------------------------------------
\subsection{Simple Graph}\label{sec:simple_graph}
% --------------------------------------------------------------------

A \emph{Simple Graph} has no Self-loops and no Parallel Edges.

Injective Function $d : E \hookrightarrow {V \choose 2}$

Simple Directed Graph (\S\ref{sec:simple_directed})



% --------------------------------------------------------------------
\subsection{Loop Graph}\label{sec:loop_graph}
% --------------------------------------------------------------------

Self-loops (\S\ref{sec:graph_loop}) allowed, no Parallel Edges

Injective Function $d : E \hookrightarrow \angleover{V}{2}$

Undirected Loop Graph is given by a Symmetric Relation on $V$

Directed Loop Graph (\S\ref{sec:directed_loop_graph}) $d : E
\hookrightarrow V^2$ -- arbitrary Binary Relation on $V$



% --------------------------------------------------------------------
\subsection{Multigraph}\label{sec:multigraph}
% --------------------------------------------------------------------

A \emph{Multigraph} is a Graph where $E$ is a Multiset where Edges
with Multiplicity $>1$ are called \emph{Parallel}.

no Self-loops, Parallel Edges allowed

arbitrary Function $d : E \rightarrow {V \choose 2}$

Directed Multigraph (\S\ref{sec:directed_multigraph}) $d : E
\rightarrow V^2 / \Delta_V$



% --------------------------------------------------------------------
\subsection{Pseudograph}\label{sec:pseudograph}
% --------------------------------------------------------------------

Self-loops and Parallel Edges allowed

arbitrary Function $d : E \rightarrow \angleover{V}{2}$

Directed Pseudograph (Quiver \S\ref{sec:quiver}) $d : E \rightarrow
V^2$



% --------------------------------------------------------------------
\subsection{Undirected Graph} \label{sec:undirected_graph}
% --------------------------------------------------------------------

In an \emph{Undirected Graph}, it is required that for $a,b \in V$:
\[
    (a,b) \in E \leftrightarrow (b,a) \in E
\]
This is the definition of a Symmetric Relation
(\S\ref{sec:symmetric_relation}) on $V$.



\subsubsection{Connected Component} \label{sec:connected_component}

Equivalence Classes given by the Reachability
(\S\ref{sec:reachability}) Relation

\emph{Strongly Connected Component} (\S\ref{sec:strongly_connected})
in a Directed Graph (\S\ref{sec:directed_graph})



\subsubsection{Complete Graph} \label{sec:complete_graph}

Simple (\S\ref{sec:simple_graph}) Undirected Graph where every Pair of
distinct Vertices is Connected by a unique Edge.

Complete Digraph (\S\ref{sec:complete_digraph})



\subsubsection{Tree Graph}\label{sec:tree_graph}

Pointed Tree (\S\ref{sec:pointed_tree})



\paragraph{Forest} \label{sec:forest}



\subsubsection{Circulant Graph} \label{sec:circulant_graph}



% --------------------------------------------------------------------
\subsection{Directed Graph} \label{sec:directed_graph}
% --------------------------------------------------------------------

(or \emph{Digraph})

$(V,A)$

Vertices $V$: Set of Node Elements

Arrows $A$: Set of Ordered Pairs of Vertices

Directed Graph
$d : E \hookrightarrow V^2 / \Delta_V$

Directed Loop Graph (\S\ref{sec:directed_loop}) $d : E \hookrightarrow
V^2$

Directed Multigraph (\S\ref{sec:directed_multigraph}) $d : E
\rightarrow V^2 / \Delta_V$

Directed Pseudograph (Quiver \S\ref{sec:quiver}) $d : E \rightarrow
V^2$



\emph{Dominator}

\emph{Postdominator}

\emph{Immediate Dominator} or \emph{Idom}

A \emph{Dominator Tree} is a Tree where each Node's Children are those
Nodes it Immediately Dominates.

\emph{Topological Ordering} is possible if and only if the Graph is a
\emph{Directed Acyclic Graph} (\S\ref{sec:dag}).

A \emph{Knot} in a Directed Graph is a collection of Vertices and
Edges where every Vertex has outgoing Edges and all outgoing Edges
terminate at other Vertices of the Knot.



\subsubsection{Arc}\label{sec:arc}

Ordered Edge (\S\ref{sec:edge})



\paragraph{Directed Loop}\label{sec:directed_loop}
\hfill \\

Directed Loop (\S\ref{sec:directed_loop})



\subsubsection{Directed Path}\label{sec:directed_path}

Path (\S\ref{sec:graph_path})



\subsubsection{Cycle}\label{sec:cycle}

\paragraph{Closed Walk}\label{sec:closed_walk}

\paragraph{Simple Cycle}\label{sec:simple_cycle}
\hfill \\

(or \emph{Directed Cycle})



\subsubsection{Strongly Connected}\label{sec:strongly_connected}

A pair of Vertices $u,v \in V$ are \emph{Strongly Connected} if there
is a Directed Path in each direction between them, i.e. $u$ is
Reachable (\S\ref{sec:reachability}) from $v$ and $v$ is Reachable
from $u$.

The Strongly-connected (Binary) Relation is an Equivalence Relation
and the Equivalence Classes of Vertices are \emph{Strongly Connected
  Component} (\S\ref{sec:strongly_connected_component})

A Directed Graph is Strongly Connected if all pairs of Vertices are
Strongly Connected.



\paragraph{Strongly Connected Component}
\label{sec:strongly_connected_component}
\hfill \\

A \emph{Strongly Connected Component} is an Equivalence Class of the
Strongly-connected Relation on the Vertices of a Directed Graph.
Strongly Connected Components are then Subgraphs in which every Vertex
is Reachable (\S\ref{sec:reachability}) from every other Vertex.

Connected Component (\S\ref{sec:connected_component})



\subsubsection{Complete Digraph}\label{sec:complete_digraph}

every pair of distinct Vertices is Connected by a pair of unique
Edges:
\[
  v,u \in V \Rightarrow \exists! (v,u), (u,v) \in E
\]



\subsubsection{Transitive Reduction}
\label{sec:transitive_reduction_graph}

Transitive Reduction (\S\ref{sec:transitive_reduction}) of the
Reachability Relation (\S\ref{sec:reachability})

not uniquely defined for Directed Graphs with Cycles

Transitive Reduction of a Directed Acyclic Graph (\S\ref{sec:dag}) $G
= (V,E)$ is unique and consists of the Edges of $G$ that from the only
Path between their Endpoints and so it is always a Subgraph of $G$.
This generates the Covering Relation (\S\ref{sec:covering_relation})
of the Partial Order defined by the Reachability Relation.



\subsubsection{Directed Acyclic Graph} \label{sec:dag}

A Partial Ordering, $\leq_P$, of a DAG, $G$, may be defined as the
\emph{Reachability Relation} on $G$ by taking the objects as the
Vertices $u,v,... \in V$ and defining $(u,v) \in \leq_P$ if and only
if there is a Directed Path from $u$ to $v$.

A \emph{Topological Ordering} or \emph{Topological Sorting} is a Total
Ordering, $\leq_T$, of the Vertices such that $\forall (u,v) \in E
\Rightarrow (u,v) \in \leq_T$. A Topological Ordering is a Linear
Extension of the DAG's Partial Ordering. % FIXME ref linear extension



\subsubsection{Tournament}\label{sec:tournament}

\subsubsection{Simple Directed Graph}\label{sec:simple_directed}

Simple Graph (\S\ref{sec:simple_graph})

Irreflexive Relation on $V$



\subsubsection{Directed Loop Graph}\label{sec:directed_loop_graph}

Loop Graph (\S\ref{sec:loop_graph})

arbitrary Binary Relation on $V$

completely determined by Adjacency Relation (\S\ref{sec:adjacency})



\subsubsection{Directed Multigraph}\label{sec:directed_multigraph}

Multigraph (\S\ref{sec:multigraph})



\subsubsection{Quiver}\label{sec:quiver}

Directed Pseudograph (\S\ref{sec:pseudograph})

Paracategory (\S\ref{sec:paracategory})

nCatLab:

it is always possible to Interpret any kind of Graph as a Quiver



% --------------------------------------------------------------------
\subsection{Rooted Graph} \label{sec:rooted_graph}
% --------------------------------------------------------------------

Undirected or Directed

\emph{Mostowski's Collapsing Lemma}: \cite{aczel88}

Every Well-founded Graph has a Unique Decoration %FIXME xref, move this?



\subsubsection{Pointed Graph} \label{sec:pointed_graph}

A \emph{Pointed Graph} (or \emph{Accessible Pointed Graph} or
\emph{Flow Graph}) is a Rooted Graph where all other Vertices are
Reachable from the Root Vertex.

Anti-foundation Axioms (\S\ref{sec:anti_foundation})



\subsubsection{Pointed Tree} \label{sec:pointed_tree}



% --------------------------------------------------------------------
\subsection{Tolerance Relation}\label{sec:tolerance_relation}
% --------------------------------------------------------------------

\subsubsection{Dependency Relation}\label{sec:dependency_relation}

Finite Tolerance Relation

Independency Relation

Trace Monoid (\S\ref{sec:trace_monoid})



\subsubsection{Dependency Graph}\label{sec:dependency_graph}

The Monoid (\S\ref{sec:monoid}) of Dependency Graphs is Isomorphic to
Trace Monoids (\S\ref{sec:trace_monoid}), i.e. Free
Partially-commutative Monoids.



% --------------------------------------------------------------------
\subsection{Dipole and Cycle Graphs} \label{sec:dipole_cycle_graph}
% --------------------------------------------------------------------

% --------------------------------------------------------------------
\subsection{Cubic Graph} \label{sec:cubic_graph}
% --------------------------------------------------------------------

\subsubsection{Snark} \label{sec:snark}



% --------------------------------------------------------------------
\subsection{Symmetric Graph} \label{sec:symmetric_graph}
% --------------------------------------------------------------------



% ====================================================================
\section{Hypergraph} \label{sec:hypergraph}
% ====================================================================
