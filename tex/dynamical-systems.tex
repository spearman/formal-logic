%%%%%%%%%%%%%%%%%%%%%%%%%%%%%%%%%%%%%%%%%%%%%%%%%%%%%%%%%%%%%%%%%%%%%%
%%%%%%%%%%%%%%%%%%%%%%%%%%%%%%%%%%%%%%%%%%%%%%%%%%%%%%%%%%%%%%%%%%%%%%
\part{Dynamical Systems}\label{sec:dynamical_systems}
%%%%%%%%%%%%%%%%%%%%%%%%%%%%%%%%%%%%%%%%%%%%%%%%%%%%%%%%%%%%%%%%%%%%%%
%%%%%%%%%%%%%%%%%%%%%%%%%%%%%%%%%%%%%%%%%%%%%%%%%%%%%%%%%%%%%%%%%%%%%%

A \emph{Dynamical System} is defined as a tuple $(T,M,\Phi)$ where $T$
is a Monoid (\S\ref{sec:monoid}), M is a Set and $\Phi$ is a Function
(\S\ref{sec:set_function}).



% ====================================================================
\section{Systems Theory}\label{sec:systems_theory}
% ====================================================================

%FIXME new tex file?

Operad Theory (\S\ref{sec:operad_theory}): Systems of Systems

Operad (\S\ref{sec:operad}): compositional style

Algebra (??? \S\ref{sec:universal_algebra}): System type



% --------------------------------------------------------------------
\subsection{State Variable}\label{sec:state_variable}
% --------------------------------------------------------------------

% --------------------------------------------------------------------
\subsection{Steady State}\label{sec:steady_state}
% --------------------------------------------------------------------

% --------------------------------------------------------------------
\subsection{Closed System}\label{sec:closed_system}
% --------------------------------------------------------------------

% --------------------------------------------------------------------
\subsection{Open System}\label{sec:open_system}
% --------------------------------------------------------------------

(2016 - Fong - The Algebra of Open and Interconnected Systems):

\fist Hypergraph (\S\ref{sec:hypergraph}), Hypergraph Category
(\S\ref{sec:hypergraph_category})

Interconnection (``Integration'') of Systems modelled by Cospans
(\S\ref{sec:cospan})

Principle of Compositionality

``Network-style Diagrammatic Languages''

Network Diagram (\S\ref{sec:network_diagram}), ``Network-style
Diagrammatic Languages'' -- Electrical Circuits

other examples: Signal Flow Graphs (\S\ref{sec:signal_flowgraph}),
Markov Processes (\S\ref{sec:markov_process}), Automata
(\S\ref{sec:automaton}), Petri Nets (\S\ref{sec:petri_net}), Chemical
Reaction Networks
%FIXME

\emph{Components} with multiple Input/Output Terminals (possibly
labelled with some Type) connected to form a larger \emph{Network}

Components form \emph{Hyperedges} between labelled Vertices

\begin{itemize}
  \item each Terminal of an Open System may make ``Measurements''
    appropriate to the ``Type'' of the Terminal
  \item given a collection of Terminals, the \emph{Universum} is the
    Set of all possible Measurement outcomes
  \item each Open System has a collection of Terminals (and a Universum)
  \item the Semantics of an Open System is the Subset of Measurement
    outcomes on the Terminals that are ``permitted'' by the System,
    known as the \emph{Behavior} of the System
\end{itemize}

%FIXME universum = phase space?

``Laws'' (e.g. Ohm's Law) are mechanisms for Partitioning Behaviors
into \emph{Possible} and \emph{Impossible} Behaviors

given a Universum $\class{U}$, a Behavior of a System is an Element
of the Power Set $\pow(\class{U})$ (representing all possible
Measurements of the System), and a Law is an Element of
$\pow(\pow(\class{U}))$ representing all possible Behaviors of a
\emph{Class} of Systems

\emph{Interconnection} of Terminals asserts the Identification of
Variables at the Identified Terminals

Algebra of Semantic Objects and Homomorphism from Syntax to Semantics
(Principle of Compositionality \S\ref{sec:compositionality})



% ====================================================================
\section{Phase Space}\label{sec:phase_space}
% ====================================================================

% --------------------------------------------------------------------
\subsection{Attractor \& Repeller}\label{sec:attractor_repeller}
% --------------------------------------------------------------------



% ====================================================================
\section{Reaction-diffusion System}\label{sec:reaction_diffusion}
% ====================================================================

% ====================================================================
\section{Ergodic Theory}\label{sec:ergodic_theory}
% ====================================================================

% --------------------------------------------------------------------
\subsection{Normal Number}\label{sec:normal_number}
% --------------------------------------------------------------------



% ====================================================================
\section{Hamiltonian System}\label{sec:hamiltonian_system}
% ====================================================================

% ====================================================================
\section{Chaos Theory}\label{sec:chaos_theory}
% ====================================================================
