%%%%%%%%%%%%%%%%%%%%%%%%%%%%%%%%%%%%%%%%%%%%%%%%%%%%%%%%%%%%%%%%%%%%%%
%%%%%%%%%%%%%%%%%%%%%%%%%%%%%%%%%%%%%%%%%%%%%%%%%%%%%%%%%%%%%%%%%%%%%%
\part{Analytic Geometry}\label{sec:analytic_geometry}
%%%%%%%%%%%%%%%%%%%%%%%%%%%%%%%%%%%%%%%%%%%%%%%%%%%%%%%%%%%%%%%%%%%%%%
%%%%%%%%%%%%%%%%%%%%%%%%%%%%%%%%%%%%%%%%%%%%%%%%%%%%%%%%%%%%%%%%%%%%%%

%FIXME: this section needs to be made consistent with respect to
% what is found under ``synthetic geometry''

of \emph{Coordinate Geometry} or \emph{Cartesian Geometry}

Catenary

Tractrix

Involute:Integral::Evolute:Derivative


\fist \emph{Geometry of Physics}:
\url{https://ncatlab.org/nlab/show/geometry+of+physics}



% ====================================================================
\section{Coordinate System}\label{sec:coordinate_system}
% ====================================================================

Coordinate Space (\S\ref{sec:coordinate_space})

Homogeneous Coordinates (Projective Coordinates
\S\ref{sec:homogeneous_coordinate})

\fist Canonical Coordinates (\S\ref{sec:canonical_coordinate}): Coordinates on
a Phase Space (\S\ref{sec:phase_space}) which can be used to desribe a Physical
System at any given Point in Time (Hamiltonian Mechanics
\S\ref{sec:hamiltonian_system}); can be generalized to definition of Coordinates
on the Phase Space as a Cotangent Bundle (\S\ref{sec:cotangent_bundle}) of a
Manifold


% --------------------------------------------------------------------
\subsection{Cartesian Coordinate}\label{sec:cartesian_coordinate}
% --------------------------------------------------------------------

a Euclidean Space (\S\ref{sec:euclidean_space}) with Cartesian Coordinates is
Modelled by a Real Coordinate Space (\S\ref{sec:real_coordinate_space})

cf. Homogeneous Coordinates (\S\ref{sec:homogeneous_coordinate}) in Projective
Geometry (\S\ref{sec:projective_geometry})



\subsubsection{Cartesian Space}\label{sec:cartesian_space}

(ncat):

A \emph{Cartesian Space} $\reals^n$ is a Finite Cartesian Product
(\S\ref{sec:cartesian_product}) of the Real Line $\reals$ with itself, where
$n$ is some Natural Number (possibly Zero).

The canonical Euclidean Metric (\S\ref{sec:euclidean_metric}) on $\reals^n$
gives an $n$-dimensional Euclidean Space (\S\ref{sec:euclidean_space}) and
induces the natural Euclidean Topology (\S\ref{sec:euclidean_topology}).

A canonical Smooth Structure (\S\ref{sec:smooth_structure}) on $\reals^n$ makes
it a Smooth Manifold (\S\ref{sec:smooth_manifold}). For all $n$, the Open
$n$-ball with standard Smooth Structure is Diffeomorphic to the Cartesian Space
$\reals^n$ with its standard Smooth Structure.

Thm. \emph{For $n \in \nats$ with $n \neq 4$, there is a Unique (up to
  Isomorphism) Smooth Structure on Cartesian Space $\reals^n$}.

Thm. \emph{On $\reals^4$ there exist Exotic Smooth Structures}.

A Cartesian Space is canonically a Vector Space (\S\ref{sec:vector_space}) over
the Field of Real Numbers.



% --------------------------------------------------------------------
\subsection{Polar Coordinate}\label{sec:polar_coordinate}
% --------------------------------------------------------------------

\fist Complex Numbers (\S\ref{sec:complex_number})



\subsubsection{Log-polar Coordinate}\label{sec:logpolar_coordinate}

\subsubsection{Cylindrical Coordinate}\label{sec:cylindrical_coordinate}



% --------------------------------------------------------------------
\subsection{Grassmann Cooridnate}\label{sec:grassmann_coordinate}
% --------------------------------------------------------------------

Pl\"ucker Embedding



\subsubsection{Pl\"ucker Cooridnate}\label{sec:plucker_coordinate}



% --------------------------------------------------------------------
\subsection{Coordinate-free}\label{sec:coordinate_free}
% --------------------------------------------------------------------

% --------------------------------------------------------------------
\subsection{Orthogonal Coordinate}\label{sec:orthogonal_coordinate}
% --------------------------------------------------------------------

% --------------------------------------------------------------------
\subsection{Curvilinear Coordinate}\label{sec:curvilinear_coordinate}
% --------------------------------------------------------------------

\subsubsection{Skew Coordinate}\label{sec:skew_coordinate}



% --------------------------------------------------------------------
\subsection{Local Coordinate}\label{sec:local_coordinate}
% --------------------------------------------------------------------

\fist Atlases (Manifolds \S\ref{sec:atlas})



% --------------------------------------------------------------------
\subsection{Reference Frame}\label{sec:reference_frame}
% --------------------------------------------------------------------

abstract Coordinate System and a Set of ``Physical Reference Points'' that
uniquely fix (``locate and orient'') the Coordinate System and standardizes
measurements



\subsubsection{Inertial Frame}\label{sec:inertial_frame}

Spherical Wave Transformation (\S\ref{sec:spherical_wave_transformation})
leaves the form of Spherical Waves (\S\ref{sec:spherical_wave}) Invariant in
all Inertial Frames



% ====================================================================
\section{Coordinate Space}\label{sec:coordinate_space}
% ====================================================================

a Space with a Coordinate System (\S\ref{sec:coordinate_system})



% --------------------------------------------------------------------
\subsection{Real Coordinate Space}\label{sec:real_coordinate_space}
% --------------------------------------------------------------------

\emph{Real Coordinate Space} or \emph{Real $n$-space}

$\reals^n$

Models Euclidean Space (\S\ref{sec:euclidean_space}) with Cartesian
Coordinates (\S\ref{sec:cartesian_coordinate})

$\reals^0$ -- Singleton Empty Column Vector-- Zero (Initial) Vector
Space

$\reals^1$ -- Real Line (\S\ref{sec:real_line})

$\reals^2$ -- Cartesian Plane (\S\ref{sec:cartesian_plane})

$\reals^3$



% --------------------------------------------------------------------
\subsection{Real Line}\label{sec:real_line}
% --------------------------------------------------------------------

$\reals^1$

``the'' \emph{Continuum}

\fist Real Analysis (\S\ref{sec:real_analysis})

\fist Numerical Analysis (\S\ref{sec:numerical_analysis}): approximation of the
Continuum

Mimesis -- quality of a Numerical Method which imitates some properties of the
``Continuum Problem''

FIXME: explain



\subsubsection{Extended Real Line}\label{sec:extended_real_line}

$\overline{\reals}$ or $[-\infty, +\infty]$ or $\reals \cup \{
-\infty, +\infty\}$

%FIXME move section?



\subsubsection{Localic Real Line}\label{sec:localic_real_line}

Locale (\S\ref{sec:locale}) of Real Numbers: the Locale of all Surjections from
the Discrete Space $\nats$ to the Real Line $\reals$



% --------------------------------------------------------------------
\subsection{Cartesian Plane}\label{sec:cartesian_plane}
% --------------------------------------------------------------------

$\reals^2$

Jordan Curve (Simple Closed Curve \S\ref{sec:simple_closed_curve}): a
Non-self-intersecting Continuous Loop in the Cartesian Plane



% ====================================================================
\section{Euclidean Space}\label{sec:euclidean_space}
% ====================================================================

\emph{Euclidean Space} $\mathbf{E}^n$ is the Metric Space with the Euclidean
Metric (\S\ref{sec:euclidean_metric}) on the Cartesian Space
(\S\ref{sec:cartesian_space}) $\reals^n$. This Metric Induces the standard
Euclidean Topology (\S\ref{sec:euclidean_topology}).

Euclidean Space is a Homogeneous Space (\S\ref{sec:homogeneous_space}) for its
Symmetry Groups (\S\ref{sec:symmetry_group})

Isomotries of Euclidean Space:
\begin{itemize}
  \item Point Reflection (Central Inversion) -- an Affine Transformation
  \item TODO
  ...
\end{itemize}

$E(n)$ -- Euclidean Group (\S\ref{sec:euclidean_group}): Isometry Group
(\S\ref{sec:isometry_group}) $ISO(n)$; makes Euclidean Geometry
(\S\ref{sec:euclidean_geometry}) a case of Klein Geometry
(\S\ref{sec:klein_geometry})

cf. Affine Spaces (\S\ref{sec:affine_space}): generalizes properties
of Euclidean Spaces

Reflection Groups (\S\ref{sec:reflection_group}) are Discrete Groups Generated
by a Set of Reflections (\S\ref{sec:reflection}) of a Finite-dimensional
Euclidean Space

The Coxeter Groups (\S\ref{sec:coxeter_group}) are precisely the Finite
Euclidean Reflection Groups

Crystallographic Groups (\S\ref{sec:crystallographic_group}) are Cocompact
(\S\ref{sec:cocompact_space}), Discrete Subgroups (\S\ref{sec:discrete_group})
of the Isometries (\S\ref{sec:isometry}) of some Euclidean Space



% --------------------------------------------------------------------
\subsection{Euclidean Metric}\label{sec:euclidean_metric}
% --------------------------------------------------------------------

or \emph{Euclidean Norm}

canonical Metric on $n$-dimensional Cartesian Space
(\S\ref{sec:cartesian_space}) $\reals^n$

$p$-norm

induces the natural Euclidean Topology



% --------------------------------------------------------------------
\subsection{Euclidean Topology}\label{sec:euclidean_topology}
% --------------------------------------------------------------------

or \emph{Standard Topology}

natural Topology Induced by the Euclidean Metric

a Set is Open if and only if it contains an Open Ball around each of its Points:
Open Balls form a Base of the Topology



% --------------------------------------------------------------------
\subsection{Euclidean Plane}\label{sec:euclidean_plane}
% --------------------------------------------------------------------

with Cartesian Coordinates Modelled by $\reals^2$ (Cartesian Plane
\S\ref{sec:cartesian_plane})

\fist Extended Euclidean Plane (Real Projective Plane
\S\ref{sec:real_projective_plane})

Frieze Groups (\S\ref{sec:frieze_group}) and Wallpaper Groups
(\S\ref{sec:wallpaper_group}) are Discrete Subgroups
(\S\ref{sec:discrete_group}) of the Isometry Group (\S\ref{sec:isometry_group})
of the Euclidean Plane



\subsubsection{Plane Curve}\label{sec:plane_curve}

Curvature (\S\ref{sec:curvature}) of a Plane Curve $\vec{r}(t) = [x(t),y(t)]$:
\[
  \kappa = \frac{x'(t)y''(t) - y'(t)x''(t)} {(x'(t)^2 + y'(t)^2)^{\frac{3}{2}}}
\]

\fist Simple Closed Curve (\S\ref{sec:simple_closed_curve})



\paragraph{Implicit Curve}\label{sec:implicit_curve}\hfill

a Plane Curve defined by an \emph{Implicit Equation}
(\S\ref{sec:implicit_equation}) relating Coordinate Variables $x$ and $y$:
\[
  F(x,y) = 0
\]
if $F(x,y)$ is a Polynomial in two Variables, then the corresponding Curve is an
\emph{Algebraic Curve} (\S\ref{sec:algebraic_curve})

\fist cf. Implicit Surface (\S\ref{sec:implicit_surface})



\paragraph{Smooth Plane Curve}\label{sec:smooth_plane_curve}\hfill

\paragraph{Algebraic Plane Curve}\label{sec:algebraic_plane_curve}\hfill



\subsubsection{Tessellation}\label{sec:tessellation}

\fist Translation (\S\ref{sec:translation}), Symmetry Group
(\S\ref{sec:symmetry_group})



\paragraph{Uniform Tiling}\label{sec:uniform_tiling}\hfill

Fundamental Domain (\S\ref{sec:fundamental_domain})



% --------------------------------------------------------------------
\subsection{3-space}\label{sec:3_space}
% --------------------------------------------------------------------

``Euclidean Space''

with Cartesian Coordinates Modelled by $\reals^3$

``Model'' Riemannian Manifold (\S\ref{sec:riemannian_manifold})

cf. Minkowski Space (\S\ref{sec:minkowski_space}) $\reals^{n-1,1}$
with the Flat Minkowski Metric (\S\ref{sec:minkowski_metric}) as the
``Model'' Lorentzian Manifold



\subsubsection{Implicit Surface}\label{sec:implicit_surface}

\fist cf. Implicit Curve (\S\ref{sec:implicit_curve})

$F(x,y,z) = 0$

\fist cf. Parametric Equation (\S\ref{sec:parametric_equation})

\fist Surface ($2$-manifold \S\ref{sec:surface})



\subsubsection{Parametric Curve}\label{sec:parametric_curve}

\fist Curve (Topology \S\ref{sec:curve}) -- a Topological Space Homeomorphic to
a Line (\S\ref{sec:algebraic_line})

a Parametric Function (\S\ref{sec:parametric_function}) on $\reals$:
\[
  \vec{r} : \reals \rightarrow \reals^3
\]

Line Integral (\S\ref{sec:line_integral})

Integral Curve (\S\ref{sec:integral_curve}), Isocline (\S\ref{sec:isocline})



\subsubsection{Parametric Surface}\label{sec:parametric_surface}

3-dimensional Parametric Function (\S\ref{sec:parametric_function}) on
$\reals^2$:
\[
  \vec{r} : \reals^2 \rightarrow \reals^3
\]

Surface Integral (\S\ref{sec:surface_integral})

Stokes' Theorem, Divergence Theorem (TODO)



\paragraph{Non-algebraic Surface}\label{sec:nonalgebraic_surface}\hfill

if $F(x,y,z)$ is Polynomial in $x$, $y$, $z$, then the surface is
Algebraic, otherwise it is Non-algebraic



\subsubsection{Minkowski Sum}\label{sec:minkowski_sum}

or \emph{Dilation}

\emph{Brunn-Minkowski Inequality} --
\url{https://golem.ph.utexas.edu/category/2017/01/the_brunnminkowski_inequality.html}



% --------------------------------------------------------------------
\subsection{Euclidean Vector}\label{sec:euclidean_vector}
% --------------------------------------------------------------------

a \emph{Vector} (\S\ref{sec:vector}) of a Euclidean Space, being a specific
kind of Vector Space (\S\ref{sec:vector_space}), is defined as an
``\emph{Arrow}'' in Euclidean Space defined by an \emph{Initial Point} and a
\emph{Terminal Point}

Positive-definite (\S\ref{sec:positive_definite}) Inner Product



\subsubsection{Bound Vector}\label{sec:bound_vector}

a \emph{Bound Vector} is a Euclidean Vector with \emph{fixed} Initial and
Terminal Points

generalized as \emph{Tangent Vectors} (\S\ref{sec:tangent_space}) in the
context of Tangent Spaces

if the Euclidean Space has an Origin then a Free Vector is equivalent to a
Bound Vector of the same Magnitude and Direction with Initial Point at the
Origin



\subsubsection{Free Vector}\label{sec:free_vector}

a \emph{Free Vector} is a Euclidean Vector without a fixed Initial Point, e.g.
a Vector only defined by \emph{Magnitude} and \emph{Direction}

if the Euclidean Space has an Origin then a Free Vector is equivalent to a
Bound Vector of the same Magnitude and Direction with Initial Point at the
Origin



% --------------------------------------------------------------------
\subsection{Pseudo-euclidean Space}\label{sec:pseudo_euclidean}
% --------------------------------------------------------------------

\subsubsection{Minkowski Space}\label{sec:minkowski_space}

$\reals^{n-1,1}$

with the Flat Minkowski Metric (\S\ref{sec:minkowski_metric}) as the
``Model'' Lorentzian Manifold (\S\ref{sec:lorentzian_manifold})

cf. Euclidean Space (\S\ref{sec:euclidean_space}) $\reals^n$ as the
``Model'' Riemannian Manifold (\S\ref{sec:riemannian_manifold})



\paragraph{Minkowski Metric}\label{sec:minkowski_metric}\hfill

Flat

Curved



\paragraph{de Sitter Space}\label{sec:desitter_space}\hfill

Minkowski Space analog of Sphere (\S\ref{sec:euclidean_sphere}) in ordinary
Euclidean Space



% ====================================================================
\section{$n$-sphere}\label{sec:n_sphere}
% ====================================================================

$S^n = \{ x \in \reals^{n+1} : \|x\| = r \}$

$S^n = \reals \cup \{\infty\}$

familiy of Manifolds

\emph{Hypersphere}

$S^0$ is Isomorphic to the $1$-dimensional Orthogonal Group $O(1)$, a
two-point Discrete Space (\S\ref{sec:discrete_space})



% --------------------------------------------------------------------
\subsection{Unit Circle}\label{sec:unit_circle}
% --------------------------------------------------------------------

$S^1$



% --------------------------------------------------------------------
\subsection{Unit Sphere}\label{sec:unit_sphere}
% --------------------------------------------------------------------

Unit Ball (\S\ref{sec:unit_ball})

$S^2$

Quaternion (\S\ref{sec:quaternion}) Solution for $\sqrt{-1}$



% --------------------------------------------------------------------
\subsection{Unit Glome}\label{sec:unit_glome}
% --------------------------------------------------------------------

$S^3$

Group of Unit Quaternions (\S\ref{sec:quaternion})



% ====================================================================
\section{$n$-torus}\label{sec:n_torus}
% ====================================================================

family of Manifolds



% ====================================================================
\section{Dimensional Analysis}\label{sec:dimensional_analysis}
% ====================================================================

% --------------------------------------------------------------------
\subsection{Subspace}\label{sec:subspace}
% --------------------------------------------------------------------

%FIXME move this section? vector spaces? topological spaces?

\fist Linear Subspace (\S\ref{sec:linear_subspace})

\fist Topological Subspace (\S\ref{sec:subspace_topology})



\subsubsection{Convex Subspace}\label{sec:convex_subspace}

\subsubsection{Hyperplane}\label{sec:hyperplane}

Subspace one Dimension less than Ambient Space

Linear, Affine, or Projective Space %FIXME xref

cf. Affine Manifold (\S\ref{sec:affine_manifold}), Affine Hyperplane
(\S\ref{sec:affine_hyperplane})

Left-regular Band Monoid (\S\ref{sec:graphic_monoid})

a \emph{Half-space} (\S\ref{sec:half_space}) is either of the two parts into
which a Hyperplane divides an Affine Space (\S\ref{sec:affine_space})

a \emph{Contact Structure} is a Smooth Field $\xi$ of Hyperplanes which is a
Subbundle of the Tangent Bundle of an Odd-dimensional Manifold: $\xi \subseteq
T X$

\fist Hypersurface (\S\ref{sec:hypersurface})

a Hyperplane is an Affine Set (\S\ref{sec:affine_set}) and a Convex Set
(\S\ref{sec:convex_set})



\paragraph{Supporting Hyperplane}\label{sec:supporting_hyperplane}\hfill



% --------------------------------------------------------------------
\subsection{Ambient Space}\label{sec:ambient_space}
% --------------------------------------------------------------------

%FIXME move this section?



% --------------------------------------------------------------------
\subsection{Nondimensionalization}\label{sec:nondimensionalization}
% --------------------------------------------------------------------

\subsubsection{Buckingham $\pi$ Theorem}\label{sec:buckingham_pi}



% ====================================================================
\section{Algebraic Curve}\label{sec:algebraic_curve}
% ====================================================================

An \emph{Algebraic Curve} is an Implicit Curve (\S\ref{sec:implicit_curve})
defined by an Equation $F(x,y) = 0$ such that $F(x,y)$ is a Polynomial
(\S\ref{sec:polynomial})



% --------------------------------------------------------------------
\subsection{Line}\label{sec:algebraic_line}
% --------------------------------------------------------------------

\fist cf. Primitive Notion of a \emph{Line} (\S\ref{sec:line}) in Synthetic
Geometry

\fist cf. Curve (Topology \S\ref{sec:curve}) -- a Topological Space Homeomorphic
to a Line

a ``straight curve''

$2$ Points determine a Line

\fist Projective Line (\S\ref{sec:projective_line}): Line extended by a Point at
Infinity; a One-dimensional Projective Space

cf. $5$ Points determine a Conic (\S\ref{sec:conic}), $9$ Points determine a
Cubic (\S\ref{sec:cubic_plane_curve})



% --------------------------------------------------------------------
\subsection{Conic}\label{sec:conic}
% --------------------------------------------------------------------

given any five Points in the Plane in General Linear Position
(\S\ref{sec:general_position})--i.e. no three are Collinear--there is a unique
non-degenerate Conic passing through them

Conics can be represented by Points in $5$-dimensional Projective Space
(\S\ref{sec:projective_space})

\fist Conic Section (\S\ref{sec:conic_section}), Quadrics (\S\ref{sec:quadric})



% --------------------------------------------------------------------
\subsection{Plane Algebraic Curve}\label{sec:plane_algebraic_curve}
% --------------------------------------------------------------------

\subsubsection{Elliptic Curve}\label{sec:elliptic_curve}

a Non-singular (no Cusps or Self-intersections) Plane Algebraic Curve defined
by an Equation of the form:
\[
  y^2 = x^3 + ax + b
\]

formally a Smooth, Projective, Algebraic Curve of Genus
(\S\ref{sec:scheme_genus}) $1$ with a given Rational Point
(\S\ref{sec:rational_point}) on it

an Elliptic Curve is an Abelian Variety (\S\ref{sec:abelian_variety}) with
Multiplication defined Algebraically with respect to which it is an Abelian
Group and the specified Point $O$ (taken to be the Curve's ``Point at
Infinity'' in the Projective Plane) serves as the Identity Element

Elliptic Curves are examples of Algebraic Groups (\S\ref{sec:algebraic_group})

\fist Algebraic Manifolds (\S\ref{sec:algebraic_manifold})

\fist Arithmetic Cryptography (\S\ref{sec:arithmetic_cryptography}) based on
the Discrete Logarithm Problem (\S\ref{sec:discrete_logarithm}) on Elliptic
Curves or more general Abelian Varieties (\S\ref{sec:abelian_variety}) over
Finite Fields (\S\ref{sec:finite_field})



\subsubsection{Cubic Plane Curve}\label{sec:cubic_plane_curve}

Cubic Curves form a Projective Space (\S\ref{sec:projective_space}) of
Dimension $9$ over any given Field $K$; for $9$ Points in General Position
(\S\ref{sec:general_position}) the Cubic Curve is unique and non-degenerate

cf. $2$ Points determine a Line (\S\ref{sec:algebraic_line}), $5$ Points
determine a Conic (\S\ref{sec:conic})



% --------------------------------------------------------------------
\subsection{Skew Curve}\label{sec:skew_curve}
% --------------------------------------------------------------------

a Space Curve which lies in \emph{no} Plane

\begin{itemize}
  \item Twisted Cubic (\S\ref{sec:twisted_cubic})
\end{itemize}



% --------------------------------------------------------------------
\subsection{Rational Curve}\label{sec:rational_curve}
% --------------------------------------------------------------------

\subsubsection{Twisted Cubic}\label{sec:twisted_cubic}

example of a Skew Curve(\S\ref{sec:skew_curve})



% --------------------------------------------------------------------
\subsection{Family of Curves}\label{sec:curve_family}
% --------------------------------------------------------------------

Set of Curves each of which is given by a Function or Parameterization in which
one or more Parameters is Variable

\begin{itemize}
  \item Solutions to Differential Equations (\S\ref{sec:differential_equation})
  \item non-degenerate Conic Sections (\S\ref{sec:conic_section})
\end{itemize}

\fist generalized as a Linear System of Divisors (Algebraic Geometry
\S\ref{sec:linear_system_of_divisors})



% ====================================================================
\section{Differential Geometry}\label{sec:differential_geometry}
% ====================================================================

Differentiable Manifolds (\S\ref{sec:differentiable_manifold})

\fist Differential Topology (\S\ref{sec:differential_topology})

\fist Synthetic Differential Geometry
(\S\ref{sec:synthetic_differential_geometry}): formalization of Differential
Geometry in the language of Topos Theory (\S\ref{sec:topos_theory})



% --------------------------------------------------------------------
\subsection{Tangency}\label{sec:tangency}
% --------------------------------------------------------------------

% --------------------------------------------------------------------
\subsection{Transversality}\label{sec:transversality}
% --------------------------------------------------------------------

formalizes idea of a ``Generic Intersection'' from Differential Topology
(\S\ref{sec:differential_topology})

cf. Transversal (Combinatorics \S\ref{sec:transversal})

General Position (\S\ref{sec:general_position})

defined by ``Linearizations'' of the Intersecting Spaces at the Point of
Intersection

two Submanifolds of a given Finite-dimensional Smooth Manifold are said to
Intersect \emph{Transversally} if at every Point of Intersection their separate
Tangent Spaces at the Point taken together gives the Tangent Space of the
Ambient Manifold at that Point

Non-intersecting (Sub-)manifolds that do not Intersect are ``Vacuously
Transverse''

if an Intersection is Transverse



% --------------------------------------------------------------------
\subsection{Geodesic}\label{sec:geodesic}
% --------------------------------------------------------------------

generalization of straight Lines to Curved Spaces



% --------------------------------------------------------------------
\subsection{Connection}\label{sec:connection}
% --------------------------------------------------------------------

precise notion of ``Transporting'' ``Data'' along a Curve or Family of Curves
in a \emph{Parallel} and consistent manner

allows comparison of ``Local Geometry'' at different Points

\emph{Local Theory}:
\begin{itemize}
  \item Parallel Transport
  \item Holonomy
\end{itemize}

\emph{Infinitesimal Theory}:
\begin{itemize}
  \item Differentiation of ``Geometric Data''; a Covariant Derivative
    (\S\ref{sec:covariant_derivative}) is a way of specifying a Derivative of a
    Vector Field along another Vector Field on a Manifold-- see Affine
    Connection (\S\ref{sec:affine_connection})
\end{itemize}



\subsubsection{Parallel Transport}\label{sec:parallel_transport}

(wiki):

Parallel Transport supplies:
\begin{itemize}
  \item a local realization of Connection
  \item a local realization of Curvature-- \emph{Holonomy}
    (\S\ref{sec:holonomy})
\end{itemize}

\fist cf. Tangent Spaces (\S\ref{sec:tangent_space})



\paragraph{Holonomy}\label{sec:holonomy}\hfill

The \emph{Holonomy} of a Connection on a Smooth Manifold
(\S\ref{sec:smooth_manifold}) measures the extent to which Parallel Transport
around a Closed Loop fails to preserve the ``Geometrical Data'' being
Transported.

\emph{Path-dependence}

\emph{Local realization} of Curvature

for Flat Connections (\S\ref{sec:flat_connection}), the associated Holonomy is
a type of Monodromy (\S\ref{sec:monodromy})

a \emph{Holonomic Constraint} (\S\ref{sec:holonomic_constraint}) is one where
there is \emph{no} Geometric Holonomy (Path-dependence)

cf. Non-integrability (Non-integrable Systems \S\ref{sec:nonintegrable_system})

\fist cf. Holonomic Functions (\S\ref{sec:holonomic_function}), Holonomic
Modules (\S\ref{sec:holonomic_module})

\url{https://physics.stackexchange.com/a/410034/36436}:

\emph{Holonomy Group} of a Connection is the Set of Transformations an
``object'' can ``experience'' when it is Parallel Transported in a Loop-- if
the associated Holonomy Groups of a Constraint are ``Non-trivial'', then the
Constraint cannot be Holonomic (\S\ref{sec:holonomic_constraint}) because the
Orientation of the object will depend on the Loop traversed, not just the
current State (FIXME: clarify)



\subsubsection{Affine Connection}\label{sec:affine_connection}

(wiki):

means of ``Transporting'' Vectors Tangent to a Manifold from one Point to
another along a Curve

typically given as a \emph{Covariant Derivative}
(\S\ref{sec:covariant_derivative})



\subsubsection{Flat Connection}\label{sec:flat_connection}

a Connection is Flat if its Curvature Form $\omega$ vanishes, or equivalently
if the Structure Group can be reduced to the same underlying Group but with the
Discrete Topology (FIXME: clarify, xref)

associated Holonomy (\S\ref{sec:holonomy}) is a type of Monodromy
(\S\ref{sec:monodromy})



% --------------------------------------------------------------------
\subsection{Geometric Flow}\label{sec:geometric_flow}
% --------------------------------------------------------------------

Gradient Flow (\S\ref{sec:gradient_flow}) associated to a Functional
(\S\ref{sec:functional}) on a Manifold



% --------------------------------------------------------------------
\subsection{Riemannian Geometry}\label{sec:riemannian_geometry}
% --------------------------------------------------------------------

Riemannian Manifold (\S\ref{sec:riemannian_manifold})

Pseudo-Riemannian Manifold (\S\ref{sec:pseudo_riemannian})

Symplectic Form (\S\ref{sec:symplectic_form}) in Symplectic Geometry
(\S\ref{sec:symplectic_geometry}) plays the role analagous to Metric Tensor
(\S\ref{sec:metric_tensor}) in Riemannian Geometry



\subsubsection{Curvature}\label{sec:reimannian_curvature}

%FIXME: xref curvature

\emph{Extrinsic Curvature} -- defined for objects \emph{embedded} in another
Space (usually Euclidean Space) in a way that relates to the Radius of
Curvature of Circles that \emph{touch} the Object

\emph{Intrinsic Curvature} -- defined in terms of the \emph{lengths} of Curves
within a Riemannian Manifold



\paragraph{Sectional Curvature}\label{sec:sectional_curvature}\hfill

%FIXME: xref curvature

Riemannian Manifolds (\S\ref{sec:riemannian_manifold}) with Constant Sectional
Curvature:
\begin{itemize}
  \item Euclidean Geometry (\S\ref{sec:euclidean_geometry}) -- Constant
    Vanishing Sectional Curvature
  \item Hyperbolic Geometry (\S\ref{sec:hyperbolic_geometry}) -- Constant
    Negative Sectional Curvature
  \item Elliptic Geometry (\S\ref{sec:elliptic_geometry}) -- Constant Positive
    Sectional Curvature
\end{itemize}



\subsubsection{Symmetric Space}\label{sec:symmetric_space}

a Pseudo-Riemannian Manifold (\S\ref{sec:pseudo_riemannian}) whose Group of
Symmetries (\S\ref{sec:symmetry_group}) contains an \emph{Inversion Symmetry}
about every Point



% --------------------------------------------------------------------
\subsection{Symplectic Geometry}\label{sec:symplectic_geometry}
% --------------------------------------------------------------------

Symplectic Manifold (\S\ref{sec:symplectic_manifold}) is a Smooth
Even-dimensional Manifold equipped with a Closed Non-degenerate $2$-form called
the \emph{Symplectic Form} (\S\ref{sec:symplectic_form})

the Symplectic Form in Symplectic Geometry plays the role analagous to Metric
Tensor (\S\ref{sec:metric_tensor}) in Riemannian Geometry
(\S\ref{sec:riemannian_geometry})

the Exterior Derivative (\S\ref{sec:exterior_derivative}) of a Tautological
$1$-form (\S\ref{sec:tautological_1form}) defined on the Cotangent Bundle
(\S\ref{sec:cotangent_bundle}) $T * Q$ of a Manifold $Q$ defines a Symplectic
Form giving $T * Q$ the Structure of a Symplectic Manifold

arises in formalisms of Classical Mechanics (e.g. Hamiltonian Systems
\S\ref{sec:hamiltonian_system}) considering either the Even-dimensional Phase
Space (\S\ref{sec:phase_space}) of a Mechanical System or the Odd-dimensional
Constant-energy Hypersurface

\fist cf. Contact Geometry (\S\ref{sec:contact_geometry}): Odd-dimensional
counterpart of Symplectic Geometry

\fist Holonomic Modules (\S\ref{sec:holonomic_module}): the Characteristic
Variety $Ch(M)$ (FIXME: xref) of any $D$-module $M$, when seen as a Subvariety
of the Cotangent Bundle $T^*X$ of $X$, is an \emph{Involutive Variety} (FIXME:
xref), i.e. the Module is Holonomic if and only if $Ch(M)$ is a Lagrangian
Submanifold (\S\ref{sec:lagrangian_submanifold})



% --------------------------------------------------------------------
\subsection{Contact Geometry}\label{sec:contact_geometry}
% --------------------------------------------------------------------

\fist cf. Symplectic Geometry (\S\ref{sec:symplectic_geometry}):
Even-dimensional counterpart of Contact Geometry

geometric structure on Smooth Manifolds given by a Hyperplane Distribution
(\S\ref{sec:tangent_bundle_distribution}) in the Tangent Bundle satisfying the
\emph{Complete Non-integrability} condition; may be equivalently given
(Locally) as the Kernel of a Differential $1$-form
(\S\ref{sec:differential_form}) and the Non-integrability condition trnslates
into a \emph{Maximal Non-degeneracy Condition}) on the Form

Hypersurface (\S\ref{sec:hypersurface}) a Hypersurface of ``Contact Type'';
Liouville Vector Field (Canonical Vector Field on a Tangent Bundle
\S\ref{sec:liouville_vector_field}); \fist Contact Geometry
(\S\ref{sec:contact_geometry})

\emph{Contact Structure} (\S\ref{sec:contact_structure})

\fist Contact Manifold (\S\ref{sec:contact_manifold}) --
\url{http://www.map.mpim-bonn.mpg.de/Contact_manifold}

Non-holonomic Constraints: Non-integrable



\subsubsection{Contact Structure}\label{sec:contact_structure}

$\xi \subset TM$

a \emph{Contact Structure} is a Smooth Field $\xi$ of Hyperplanes
(\S\ref{sec:hyperplane}) which is a Subbundle of the Tangent Bundle of an
Odd-dimensional Manifold: $\xi \subseteq T X$

(if $X$ has Dimension $2n+1$, $\xi$ has Dimension $2n$) %FIXME: correct?

$\xi$ is a Distribution (\S\ref{sec:tangent_bundle_distribution}) ??? FIXME

the Field of Hyperplanes is Maximally Non-integrable, i.e. the opposite of
being Tangent to a Foliation
%FIXME clarify

if $Y$ is a Submanifold of $X$, $Y \subset X$, and the Tangent Space of $Y$ at
every Point is inside the Distribution $\xi$, $T Y \subseteq \xi$, then the
Dimension of $Y \leq n$ (FIXME: 2n ???); if $dim(Y) = n$ then $Y$ is Legendrian
(FIXME: explain, xref)

(wiki): equivalently such a Distribution can be given Locally by the Kernel of
a Differential $1$-form and the Non-integrability Condition translates into a
Maximal Non-degeneracy Condition for the $1$-form

A \emph{Contact Manifold} (\S\ref{sec:contact_manifold}) is a Pair $(M,\xi)$ of
an Odd-dimensional Smooth Manifold $M$ with a Contact Structure $\xi$.

2002 - Etnyre - *Introductory Lectures on Contact Geometry*

a \emph{Plane Field} $\xi$ on $M$ is a Subbundle of the Tangent Bundle $TM$
such that $\xi_p = T_p M \cap \xi$ is a 2-dimensional Subspace of $T_pM$ for
each $p \in M$



% --------------------------------------------------------------------
\subsection{Lie Theory}\label{sec:lie_theory}
% --------------------------------------------------------------------

Lie Group (\S\ref{sec:lie_group})

Lie Algebra (\S\ref{sec:lie_algebra})

Lie Group-Lie Algebra Correspondence

uses Lie Groups used for analysing the Continuous Symmetries of
Differential Equations %FIXME

cf. Galois Theory (\S\ref{sec:galois_theory}) uses Permutation Groups
for analysing the Discrete Symmetries of Algebraic Equations %FIXME



\subsubsection{Lie Group}\label{sec:lie_group}\hfill

``Manifolds with Group Structure''

Continuous Transformation Group
(\S\ref{sec:continuous_transformation_group}) that is a Smooth
Differentiable Manifold (\S\ref{sec:differentiable_manifold})

Any Discrete Group (\S\ref{sec:discrete_group}) can be viewed as a
$0$-dimensional Lie Group %FIXME

a Euclidean Vector Space (\S\ref{sec:vector_space}) with the Group Operation of
Vector Addition is an example of a Non-compact (\S\ref{sec:compact_space}) Lie
Group

the Tangent Space (\S\ref{sec:tangent_space}) of a Lie Group can be naturally
given the structure of a Lie Algebra (\S\ref{sec:lie_algebra}) and can be used
to classify Compact Lie Groups

\begin{itemize}
  \item Indefinite Orthogonal Group (\S\ref{sec:indefinite_orthogonal_group})
\end{itemize}



\paragraph{Lie Subgroup}\label{sec:lie_subgroup}\hfill

naturally an Immersed Submanifold (\S\ref{sec:immersed_submanifold})



\paragraph{Dynkin Diagram}\label{sec:dynkin_diagram}\hfill

cf. Coxeter-Dynkin Diagrams (\S\ref{sec:coxeter_dynkin_diagram})



\paragraph{Special Unitary Group}\label{sec:special_unitary}\hfill

$\mathrm{SU}(n)$ -- the Lie Group of $n \times n$ Unitary Matrices
(\S\ref{sec:unitary_matrix}) with Determinant $1$

Unitary Group (\S\ref{sec:unitary_group})



\subsubsection{Lie Algebra}\label{sec:lie_algebra}

(or \emph{Infinitesimal Group})

Ininitesimal Transformations
(\S\ref{sec:infinitesimal_transformation})

Vector Space (\S\ref{sec:vector_space}) with a Non-associative
Multiplication called a \emph{Lie Bracket} $[x,y]$

when an Algebraic Product is defined on the Space, the Lie Bracket is
the Commutator $[x,y] = xy - yx$ %FIXME

the Tangent Space (\S\ref{sec:tangent_space}) of a Lie Group
(\S\ref{sec:lie_group}) can be naturally given the structure of a Lie Algebra
and can be used to classify Compact Lie Groups



\paragraph{Universal Enveloping Algebra}
\label{sec:universal_enveloping_algebra}\hfill

most general Unital Associative Algebra containing all Representations
(\S\ref{sec:representation_theory}) of a Lie Algebra
--FIXME: different from ``representation theory'' ???

\fist ``the'' Weyl Algebra (Symplectic Clifford Algebra
\S\ref{sec:weyl_algebra}) is a Quotient of the Universal Enveloping Algebra of
the Heisenberg Algebra (the Lie Algebra of the Heisenberg Group) by setting the
Central Element of the Heisenberg Algebra $([X,Y])$ equal to the Unit of the
Universal Enveloping Algebra ($1$)



\subparagraph{Algebra of Symbols}\label{sec:symbol_algebra}\hfill

$\star(\lie{g})$

Star Product $\star$

e.g. Moyal Product (Phase-space Star product of Phase Space Formulation of
Quantum Mechanics)



% ====================================================================
\section{Conformal Geometry}\label{sec:conformal_geometry}
% ====================================================================

Two dimensions: Geometry of Riemann Surfaces
(\S\ref{sec:riemann_surface})

\fist Conformal Manifold (\S\ref{sec:conformal_manifold})



% --------------------------------------------------------------------
\subsection{Euclidean Sphere}\label{sec:euclidean_sphere}
% --------------------------------------------------------------------

cf. \emph{de Sitter Space} (\S\ref{sec:desitter_space}): Minkowski Space analog
of Sphere in Euclidean Space



% --------------------------------------------------------------------
\subsection{Torus}\label{sec:torus}
% --------------------------------------------------------------------

$S^1 \times S^1$

Ring Torus

Horn Torus

Spindle Torus

\fist Toroidal Graph (\S\ref{sec:toroidal_graph})



% --------------------------------------------------------------------
\subsection{Stereographic Projection}\label{sec:stereographic_projection}
% --------------------------------------------------------------------

\begin{enumerate}
  \item Lines and Circles on the Plane map to Circles on the Sphere
  \item Being Conformal, projection preserves Angles
\end{enumerate}

Complex Plane (\S\ref{sec:complex_plane})



% --------------------------------------------------------------------
\subsection{Inversive Geometry}\label{sec:inversive_geometry}
% --------------------------------------------------------------------

% --------------------------------------------------------------------
\subsection{Conformal Symmetry}\label{sec:conformal_symmetry}
% --------------------------------------------------------------------

\emph{Conformal Invariance}



\subsubsection{Spherical Wave Transformation}
\label{sec:spherical_wave_transformation}

(wiki):

Bateman \& Cunningham -- Conformal Symmetry of Maxwell's Equations; generic
expression of Conformal Symmetry is called a \emph{Spherical Wave
  Transformation}

leaves the form of Spherical Waves (\S\ref{sec:spherical_wave}) Invariant in
all Inertial Frames (\S\ref{sec:inertial_frame})



\subsubsection{Scale Invariance}\label{sec:scale_invariance}

\emph{Dilation}: Scale Transformation (Homothety \S\ref{sec:homothety})

``Universality'', ``Self-similarity''

\fist Homogeneous Functions (Projective Geometry
\S\ref{sec:homogeneous_function})

\fist Fractal Geometry (Part \ref{sec:fractal_geometry})



\paragraph{Renormalization Group}\label{sec:renormalization_group}\hfill

Chaos Theory (\S\ref{sec:chaos_theory}), Statistical Mechanics



% ====================================================================
\section{Real Analytic Geometry}\label{sec:real_analytic_geometry}
% ====================================================================

% ====================================================================
\section{Non-archimedean Analytic Geometry}
\label{sec:nonarchimedean_analytic_geometry}
% ====================================================================

Analytic Geometry over Non-archimedean Fields (\S\ref{sec:nonarchimedean_field})

\fist Global Analytic Geometry (\S\ref{sec:global_analytic_geometry})
