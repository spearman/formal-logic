%%%%%%%%%%%%%%%%%%%%%%%%%%%%%%%%%%%%%%%%%%%%%%%%%%%%%%%%%%%%%%%%%%%%%%%%%%%%%%%%
%%%%%%%%%%%%%%%%%%%%%%%%%%%%%%%%%%%%%%%%%%%%%%%%%%%%%%%%%%%%%%%%%%%%%%%%%%%%%%%%
\part{Analytic Geometry}\label{part:analytic_geometry}
%%%%%%%%%%%%%%%%%%%%%%%%%%%%%%%%%%%%%%%%%%%%%%%%%%%%%%%%%%%%%%%%%%%%%%%%%%%%%%%%
%%%%%%%%%%%%%%%%%%%%%%%%%%%%%%%%%%%%%%%%%%%%%%%%%%%%%%%%%%%%%%%%%%%%%%%%%%%%%%%%

or \emph{Coordinate Geometry} or \emph{Cartesian Geometry}

\fist cf. Synthetic Geometry (Coordinate-free Geometry, Part
\ref{part:synthetic_geometry})

\fist Geometric Calculus (\S\ref{sec:geometric_calculus})

Catenary

Tractrix

Involute:Integral::Evolute:Derivative

\fist \emph{Geometry of Physics}:
\url{https://ncatlab.org/nlab/show/geometry+of+physics}

\asterism

Analytic Geometry and Algebraic Geometry (\S\ref{sec:algebraic_geometry}): TODO



% ==============================================================================
\section{Coordinate System}\label{sec:coordinate_system}
% ==============================================================================

Coordinate Space (\S\ref{sec:coordinate_space})

see also:
\begin{itemize}
  \item Canonical Coordinates (\S\ref{sec:canonical_coordinate}): Coordinates on
    a Phase Space (\S\ref{sec:phase_space}) which can be used to desribe a
    Physical System at any given Point in Time (Hamiltonian Mechanics
    \S\ref{sec:hamiltonian_system}); can be generalized to definition of
    Coordinates on the Phase Space as a Cotangent Bundle
    (\S\ref{sec:cotangent_bundle}) of a Manifold
  \item Generalized Coordinates (\S\ref{sec:generalized_coordinate}) in
    Lagrangian Mechanics (\S\ref{sec:lagrangian_system}) -- related to Canonical
    Coordinates of Hamiltonian Mechanics by the Hamilton-Jacobi Equations
    (\S\ref{sec:hamilton_jacobi})
\end{itemize}

\fist a \emph{Manifold} (Geometric Topology \S\ref{sec:manifold}) intuitively is
a Topological Space with Points labelled by Coordinates; see \emph{Coordinate
  Charts} (\emph{a.k.a.} \emph{Local Frames} \S\ref{sec:local-frame})



% ------------------------------------------------------------------------------
\subsection{Curvilinear Coordinates}\label{sec:curvilinear_coordinates}
% ------------------------------------------------------------------------------

\subsubsection{Orthogonal Coordinates}\label{sec:orthogonal_coordinates}

\emph{Orthonormal Frame}



\paragraph{Cartesian Coordinates}\label{sec:cartesian_coordinates}\hfill

Cartesian Space (\S\ref{sec:cartesian_space}) -- Finite Cartesian Product
(\S\ref{sec:cartesian_product}) $\reals^n$ of the Real Line
(\S\ref{sec:real_line}) $\reals$ with itself

a Euclidean Space (\S\ref{sec:euclidean_space}) with Cartesian Coordinates is
Modelled by a Real Coordinate Space (\S\ref{sec:real_coordinate_space})

cf. Homogeneous Coordinates (\S\ref{sec:homogeneous_coordinates}) in Projective
Geometry (\S\ref{sec:projective_geometry})

Cartesian Coordinates are Affine Coordinates (\S\ref{sec:affine_coordinates})
relative to an Orthonormal Frame

see Covariant (\S\ref{sec:vector_covariance}) and Contravariant
(\S\ref{sec:vector_contravariance}) Vector Components of Non-cartesian
Coordinates; these can be converted into one another using the Metric Tensor
(\S\ref{sec:metric_tensor})



\paragraph{Polar Coordinate}\label{sec:polar_coordinates}\hfill

\fist Complex Numbers (\S\ref{sec:complex_number})



\subparagraph{Log-polar Coordinate}\label{sec:logpolar_coordinates}\hfill



\paragraph{Spherical Coordinate}\label{sec:spherical_coordinates}\hfill

\paragraph{Cylindrical Coordinate}\label{sec:cylindrical_coordinates}\hfill



\subsubsection{Skew Coordinates}\label{sec:skew_coordinates}



% ------------------------------------------------------------------------------
\subsection{Homogeneous Coordinates}\label{sec:homogeneous_coordinates}
% ------------------------------------------------------------------------------

(or \emph{Projective Coordinates})

Projective Geometry (\S\ref{sec:projective_geometry}), Projective
Space (\S\ref{sec:projective_space}) -- Points (including Points at
Infinity) can be represented using Finite Coordinates; allows for
Affine Transformations (\S\ref{sec:affine_transformation})

\fist cf. Cartesian Coordinates (\S\ref{sec:cartesian_coordinates}) in
Euclidean Geometry (\S\ref{sec:euclidean_geometry})

\fist cf. Homogeneous Functions (\S\ref{sec:homogeneous_function})



\subsubsection{Barycentric Coordinates}\label{sec:barycentric_coordinates}

Barycentric Coordinates are \emph{not unique}

Affine Coordinates (\S\ref{sec:affine_coordinate}) \emph{are unique} --
$\sum \lambda_i = 1$



\paragraph{Barycentric Subdivision}\label{sec:barycentric_subdivision}\hfill

standard way of dividing Convex Polytopes into Simplices with the same Dimension

\url{http://www.ams.org/publicoutreach/feature-column/fc-2017-06} -- repeated
Barycentric Subdivision of a Triangle will eventually produce Triangles of every
shape



\subsubsection{Trilinear Cooridnates}\label{sec:trilinear_coordinates}

\subsubsection{Grassmann Cooridnates}\label{sec:grassmann_coordinates}

Pl\"ucker Embedding



\paragraph{Pl\"ucker Cooridnates}\label{sec:plucker_coordinates}\hfill



% ------------------------------------------------------------------------------
\subsection{Local Coordinate}\label{sec:local_coordinate}
% ------------------------------------------------------------------------------

``measurement indices'' into a Local Coordinate System or Local Coordinate Space
(FIXME: clarify)

\fist Atlases (Manifolds \S\ref{sec:atlas})



% ------------------------------------------------------------------------------
\subsection{Reference Frame}\label{sec:reference_frame}
% ------------------------------------------------------------------------------

abstract Coordinate System and a Set of ``Physical Reference Points'' that
uniquely fix (``locate and orient'') the Coordinate System and standardizes
measurements

cf. \emph{Local Frames} (\emph{Coordinate Charts} \S\ref{sec:coordinate_chart})



\subsubsection{Inertial Frame}\label{sec:inertial_frame}

Spherical Wave Transformation (\S\ref{sec:spherical_wave_transformation})
leaves the form of Spherical Waves (\S\ref{sec:spherical_wave}) Invariant in
all Inertial Frames



\subsubsection{Rapidity}\label{sec:rapidity}



% ==============================================================================
\section{Coordinate Space}\label{sec:coordinate_space}
% ==============================================================================

a Space with a Coordinate System (\S\ref{sec:coordinate_system})



% ------------------------------------------------------------------------------
\subsection{Cartesian Space}\label{sec:cartesian_space}
% ------------------------------------------------------------------------------

(ncat):

A \emph{Cartesian Space} $\reals^n$ is a Finite Cartesian Product
(\S\ref{sec:cartesian_product}) of the Real Line $\reals$ with itself, where
$n$ is some Natural Number (possibly Zero).

The canonical Euclidean Metric (\S\ref{sec:euclidean_metric}) on $\reals^n$
gives an $n$-dimensional Euclidean Space (\S\ref{sec:euclidean_space}) and
induces the natural Euclidean Topology (\S\ref{sec:euclidean_topology}).

A canonical Smooth Structure (\S\ref{sec:smooth_structure}) on $\reals^n$ makes
it a Smooth Manifold (\S\ref{sec:smooth_manifold}). For all $n$, the Open
$n$-ball with standard Smooth Structure is Diffeomorphic to the Cartesian Space
$\reals^n$ with its standard Smooth Structure.

Thm. \emph{For $n \in \nats$ with $n \neq 4$, there is a Unique (up to
  Isomorphism) Smooth Structure on Cartesian Space $\reals^n$}.

Thm. \emph{On $\reals^4$ there exist Exotic Smooth Structures}.

A Cartesian Space is canonically a Vector Space (\S\ref{sec:vector_space}) over
the Field of Real Numbers.



% ------------------------------------------------------------------------------
\subsection{Real Coordinate Space}\label{sec:real_coordinate_space}
% ------------------------------------------------------------------------------

\emph{Real Coordinate Space} or \emph{Real $n$-space}

$\reals^n$

Models Euclidean Space (\S\ref{sec:euclidean_space}) with Cartesian
Coordinates (\S\ref{sec:cartesian_coordinates})

$\reals^0$ -- Singleton Empty Column Vector-- Zero (Initial) Vector
Space

$\reals^1$ -- Real Line (\S\ref{sec:real_line})

$\reals^2$ -- Cartesian Plane (\S\ref{sec:cartesian_plane})

$\reals^3$



% ------------------------------------------------------------------------------
\subsection{Real Line}\label{sec:real_line}
% ------------------------------------------------------------------------------

$\reals^1$

a Cartesian Space (\S\ref{sec:cartesian_space}) is a Finite Cartesian Product
(\S\ref{sec:cartesian_product}) $\reals^n$ of the Real Line with itself

``the'' \emph{Continuum}

\fist Real Analysis (\S\ref{sec:real_analysis})

\fist Numerical Analysis (\S\ref{sec:numerical_analysis}): approximation of the
Continuum

Mimesis -- quality of a Numerical Method which imitates some properties of the
``Continuum Problem''

FIXME: explain

\emph{Cantor-Dedekind Axiom} -- thesis that the Real Numbers are
Order-isomorphic to the Linear Continuum (Order Theory
\S\ref{sec:linear_continuum}) in Geometry; a consequence is that the
Decidability of the Ordered Real Field can be seen as an Algorithm to solve any
problem in Euclidean Geometry

the Real Unit Interval $[0,1]$ is a One-dimensional Analytic Manifold and can be
interpreted as a generalization of the Boolean Domain $\{0, 1\}$ (Fuzzy Logic
\S\ref{sec:fuzzy_logic})

the Real Closed Interval (Unit 0-disc) $[-1,1]$ is a Semigroup
(\S\ref{sec:semigroup}) under Multiplication ($0$ does not have an Inverse)



\subsubsection{Extended Real Line}\label{sec:extended_real_line}

$\overline{\reals}$ or $[-\infty, +\infty]$ or $\reals \cup \{
-\infty, +\infty\}$

Extended Real Numbers (\S\ref{sec:extended_real})

%FIXME move section?



\subsubsection{Localic Real Line}\label{sec:localic_real_line}

Locale (\S\ref{sec:locale}) of Real Numbers: the Locale of all Surjections from
the Discrete Space $\nats$ to the Real Line $\reals$



% ------------------------------------------------------------------------------
\subsection{Cartesian Plane}\label{sec:cartesian_plane}
% ------------------------------------------------------------------------------

$\reals^2$

Jordan Curve (Simple Closed Curve \S\ref{sec:simple_closed_curve}): a
Non-self-intersecting Continuous Loop in the Cartesian Plane



\subsubsection{Planar Algebra}\label{sec:planar_algebra}

Kinematic Geometries (\S\ref{sec:kinematic_geometry})



% ------------------------------------------------------------------------------
\subsection{Subspace}\label{sec:subspace}
% ------------------------------------------------------------------------------

%FIXME move this section? vector spaces? topological spaces?

\fist Linear Subspace (\S\ref{sec:linear_subspace})

\fist Topological Subspace (\S\ref{sec:subspace_topology})



\subsubsection{Convex Subspace}\label{sec:convex_subspace}

\subsubsection{Hyperplane}\label{sec:hyperplane}

Subspace one Dimension less than Ambient Space

Linear, Affine, or Projective Space %FIXME xref

cf. Affine Manifold (\S\ref{sec:affine_manifold}), Affine Hyperplane
(\S\ref{sec:affine_hyperplane})

Left-regular Band Monoid (\S\ref{sec:graphic_monoid})

a \emph{Half-space} (\S\ref{sec:half_space}) is either of the two parts into
which a Hyperplane divides an Affine Space (\S\ref{sec:affine_space})

a \emph{Contact Structure} is a Smooth Field $\xi$ of Hyperplanes which is a
Subbundle of the Tangent Bundle of an Odd-dimensional Manifold: $\xi \subseteq
T X$

\fist Hypersurface (\S\ref{sec:hypersurface})

a Hyperplane is an Affine Set (\S\ref{sec:affine_set}) and a Convex Set
(\S\ref{sec:convex_set})



\paragraph{Supporting Hyperplane}\label{sec:supporting_hyperplane}\hfill



% ------------------------------------------------------------------------------
\subsection{Ambient Space}\label{sec:ambient_space}
% ------------------------------------------------------------------------------

%FIXME move this section?



% ==============================================================================
\section{Projective Space}\label{sec:projective_space}
% ==============================================================================

(wiki):
an \emph{$n$-dimensional Projective Space} is the Set of $1$-dimensional Linear
Subspaces (i.e. ``Vector Lines'' \S\ref{sec:linear_subspace}) in an
$n+1$-dimensional Vector Space $V$; this can be seen as the Quotient Set of $V
\ \{0\}$ by the Equivalence Relation of being on the same Vector Line, or as a
Unit Sphere with Antipodal Points identified

a $1$-dimensional Projective Space is the \emph{Projective Line}
(\S\ref{sec:projective_line}), and a $2$-dimensional Projective Space is the
\emph{Projective Plane} (\S\ref{sec:projective_plane})

Homogeneous Coordinates (\S\ref{sec:homogeneous_coordinates}) as a Model for
Projective Geometry (\S\ref{sec:projective_geometry})

a Projective Space is a Homogeneous Space (\S\ref{sec:homogeneous_space}) for
its Symmetry Groups (\S\ref{sec:symmetry_group}) %FIXME

\fist Projective Varieties (\S\ref{sec:projective_variety})

Projective Linear Group (\S\ref{sec:projective_linear_group}) $PGL(n,F)$ --
Quotient of $GL(n,F)$ by its Center, the Subgroup of Nonzero Scalar Matrices
$Z(n,F)$, which is the induced Action (\S\ref{sec:group_action}) on the
associated Projective Space

the Set of One-dimensional (Linear) Subspaces (\S\ref{sec:linear_subspace}) of a
Finite-dimensional Vector Space $V$ is the Projective Space of $V$

a Set of $n$ Vectors in an $n$-dimensional Vector Space are Linearly
Independent if and only if the Points they define in Projective Space of
Dimension $n-1$ are in General Linear Position (\S\ref{sec:general_position})

\begin{itemize}
  \item Conics (\S\ref{sec:conic}) can be represented by Points in
    $5$-dimensional Projective Space
  \item Cubic Curves (\S\ref{sec:cubic_plane_curve}) can be represented by
    Points in $9$-dimensional Projective Space
\end{itemize}

may be defined Axiomatically: TODO



% ------------------------------------------------------------------------------
\subsection{Projective Transformation}\label{sec:projective_transformation}
% ------------------------------------------------------------------------------

a \emph{Projective Transformation} (or \emph{Homography}) is an Isomorphism of
Projective Spaces induced by an Isomorphism of the Vector Spaces from which the
Projective Spaces derive

an example of a Collineation (\S\ref{sec:collineation}): Bijection mapping Lines
to Lines

\emph{Fundamental Theorem of Projective Geometry} (TODO)



% ------------------------------------------------------------------------------
\subsection{Real Projective Space}\label{sec:real_projective_space}
% ------------------------------------------------------------------------------

Compact (\S\ref{sec:compact_space}) Smooth Manifold
(\S\ref{sec:smooth_manifold})



\subsubsection{Real Projective Plane}\label{sec:real_projective_plane}

(or \emph{Extended Euclidean Plane})

can be thought of as the Two-sphere Quotiented out by the Antipodal Map (TODO:
explain)



% ------------------------------------------------------------------------------
\subsection{Complex Projective Space}\label{sec:complex_projective_space}
% ------------------------------------------------------------------------------

\begin{itemize}
  \item the Space of Pure States of a Two-level Quantum System (Qubit
    \S\ref{sec:qubit}) is the Complex Projective Line $\mathbb{CP}^1$ (i.e. the
    \emph{Bloch Sphere} \S\ref{sec:bloch_sphere}, a Unit $2$-sphere with
    Antipodal Points corresponding to a pair of mutually Orthogonal State
    Vectors)
  \item ...
\end{itemize}



% ------------------------------------------------------------------------------
\subsection{Grassmanian}\label{sec:grassmanian}
% ------------------------------------------------------------------------------

(wiki):

The \emph{Grassmanian} $Gr(k,V)$ of $k$-dimensional Subspaces of an
$n$-dimensional Vector Space $V$ is a Space which \emph{parameterizes} all
$k$-dimensional Linear Subspaces (\S\ref{sec:linear_subspace}) of $V$, e.g.
the Grassmanian $Gr(1,V)$ is the Space of Lines through the Origin in $V$ (and
equivalent to the Projective Space of Dimension $n-1$).

When $V$ is a Real or Complex Vector Space, the Grassmanians are Compact Smooth
Manifolds (\S\ref{sec:smooth_manifold}).



% ------------------------------------------------------------------------------
\subsection{Projective Hilbert Space}\label{sec:projective_hilbert_space}
% ------------------------------------------------------------------------------

Hilbert Space (\S\ref{sec:hilbert_space})

\fist Quantum Systems (\S\ref{sec:quantum_system}) -- a
Projective Hilbert Space $P(H)$ of a Complex Hilbert Space $H$ is the Set of
Equivalence Classes of Vectors $v \in H$ by the Relation $\sim$:
\[
  v = \lambda w \Rightarrow v \sim w
\]
for $v,w \neq 0$ and some Non-zero Complex Number $\lambda$, where the
Equivalence Classes are called \emph{(Projective) Rays} (\S\ref{sec:ray}); such
a Ray represents a \emph{Pure Quantum State} (\S\ref{sec:quantum_state}); if the
Hilbert Space is chosen as a Function Space, then the representatives are called
\emph{Wave Functions} (\S\ref{sec:wave_function})



% ==============================================================================
\section{Affine Space}\label{sec:affine_space}
% ==============================================================================

An \emph{Affine Space} generalizes properties of Euclidean Spaces
(\S\ref{sec:euclidean_space}) to be \emph{independent} of the concepts of
Distance (cf. Norm \S\ref{sec:norm}) and Measure of Angles (cf. Rotation
\S\ref{sec:rotation}), keeping only properties relating to
\emph{Parallelism}

\fist Affine Geometry (\S\ref{sec:affine_geometry})

(wiki):

an Affine Space is a Set $A$ of \emph{Points} and a Vector Space $\vec{A}$ of
\emph{(Free) Vectors} with a Transitive, Free Action
$A \times \vec{A} \rightarrow A$ of the Additive Group of $\vec{A}$ on the
Set $A$ with the Properties:
\begin{enumerate}
\item Right Identity
\item Associativity
\item $\forall a \in A$, the mapping $\vec{A} \rightarrow A; v \mapsto a + v$ is
  a Bijection
\item $\forall v \in \vec{A}$, the mapping $A \rightarrow A : a \mapsto a + v$
  is a Bijection
\item Subtraction: for every $a, b \in A$, there exists a unique $v \in \vec{A}$
  denoted $b - a$ such that $b = a + b$
\end{enumerate}
where (1.) and (2.) are defining Properties of a Right Group Action, (3.)
characterizes Free and Transitive Actions and (4.) follows from the first three,
and (5.) is an equivalent form of (3.)

an equivalent definition is that an Affine Space is a Principal Homogeneous
Space (\S\ref{sec:principal_homogeneous_space}) for its Symmetry Groups
(\S\ref{sec:symmetry_group}), i.e. for the Action of the Additive Group of a
Vector Space

any Vector Space (\S\ref{sec:vector_space}) may be considered an Affine Space
over itself

Manifolds (\S\ref{sec:manifold}) built by ``gluing together'' Charts
(\S\ref{sec:chart}, Open Subsets of Real Affine Spaces)

cf. Algebraic Varieties (\S\ref{sec:algebraic_variety}) built by ``gluing
together'' Affine Varities (\S\ref{sec:affine_variety})

\fist a \emph{Smooth Scheme} (\S\ref{sec:scheme}) over a Field is well
approximated by Affine Space near any Point (cf. Manifolds)

Affine Group (\S\ref{sec:affine_group}) $Aff(n,F)$ -- Extension of the General
Linear Group $GL(n,F)$ by Group of Translations in $F^n$ is the Group of all
Affine Transformations (\S\ref{sec:affine_transformation}) on the Affine Space
underlying the Vector Space $F^n$



% ------------------------------------------------------------------------------
\subsection{Affine Coordinates}\label{sec:affine_coordinates}
% ------------------------------------------------------------------------------

\begin{itemize}
  \item Barycentric Coordinates (\S\ref{sec:barycentric_coordinates})
  \item Cartesian Coordinates (\S\ref{sec:cartesian_coordinates}) are Affine
    Coordinates relative to an Orthonormal Frame
\end{itemize}



\subsubsection{Affine Frame}\label{sec:affine_frame}



% ------------------------------------------------------------------------------
\subsection{Affine Subspace}\label{sec:affine_subspace}
% ------------------------------------------------------------------------------

\subsubsection{Flat}\label{sec:flat}

a Subset of an $n$-dimensional Space that is Congruent



% ------------------------------------------------------------------------------
\subsection{Affine Combination}\label{sec:affine_combination}
% ------------------------------------------------------------------------------

a Linear Combination (\S\ref{sec:linear_combination}) is Affine when the sum of
the coefficients is equal to $1$

\fist Convex Combinations (Convex Geometry \S\ref{sec:convex_combination}) -- an
Affine Combination where the coefficients are non-negative



\subsubsection{Affine Independence}\label{sec:affine_independence}

Affine analog of Linear Independence (\S\ref{sec:linear_independence})



% ------------------------------------------------------------------------------
\subsection{Affine Basis}\label{sec:affine_basis}
% ------------------------------------------------------------------------------

a Set of Affinely Independent Points

$d+1$ Points in General Position (\S\ref{sec:general_position}) in Affine
$d$-space are an Affine Basis



\subsubsection{General Position}\label{sec:general_position}

or \emph{General Linear Position}

\fist cf. Generic Points (Algebraic Varieties \S\ref{sec:generic_point})

\fist cf. Differential Topoplogy (\S\ref{sec:differential_topology}): Tangency
(\S\ref{sec:tangency}), Transversality (\S\ref{sec:transversality})

(wiki):

notion of \emph{Genericity} for a Set of Points or other Geometric Objects

$d+1$ Points in General Position in Affine $d$-space are an Affine Basis

a Set of Points in a $d$-dimensional Affine Space (\S\ref{sec:affine_space}),
e.g. $d$-dimensional Euclidean Space, is in \emph{General (Linear) Position} if
no $k$ of them lie in a $(k-2)$-dimensional Flat (Affine Subspace
\S\ref{sec:flat}) for $k = 2,3,\ldots,d+1$; if the condition holds for some
$k_0$ then it must also hold for all $k$ with $2 \leq k \leq k_0$, so for a
Set containing at least $d+1$ Points in a $d$-dimensional Affine Space to be in
General Position, it is sufficient to know that no Hyperplane contains more
than $d$ Points-- i.e. the Points do not Satisfy any more Linear Relations than
they must (FIXME: clarify)

a Set of $n$ Vectors in an $n$-dimensional Vector Space are Linearly
Independent if and only if the Points they define in Projective Space
(\S\ref{sec:projective_space}) of Dimension $n-1$ are in General Linear
Position

given any five Points in the Plane in General Linear Position (i.e. no three
are Collinear), there is a unique non-degenerate Conic (\S\ref{sec:conic})
passing through them



\subsubsection{Special Position}\label{sec:special_position}



% ------------------------------------------------------------------------------
\subsection{Affine Transformation}\label{sec:affine_transformation}
% ------------------------------------------------------------------------------

(or \emph{Affine Map} or \emph{Affinity}) is a Function between Affine Spaces
which preserves:
\begin{itemize}
  \item Co-linearity between Points
  \item Parallelism between Lines
  \item Convexity of Sets
  \item Ratios of Lengths
  \item Barycenters of weighted collections of Points
\end{itemize}

preserves Ratios of Distances betwen Points lying on a Straight Line

does not necessarily preserve Angles between Lines or Distances between Points

for Affine Spaces $X$ and $Y$

Affine Functions are both Convex and Concave (\S\ref{sec:convex_function})

\fist Affine Group (\S\ref{sec:affine_group})

\fist All Linear Transformations (\S\ref{sec:linear_transformation}) are
Affine, but not every Affine Transformation is Linear; Affine Transformations
are not required to preserve the Zero Point in a Linear Space (FIXME: clarify)

Affine Transformations of Quaternions (\S\ref{sec:quaternion_function}) have
the form:
\[
  f(q) = aq + b, \;\;\; a,b,q \in \quats
\]

Affine Group (\S\ref{sec:affine_group}) $Aff(n,F)$ -- Extension of the General
Linear Group $GL(n,F)$ by Group of Translations in $F^n$ is the Group of all
Affine Transformations on the Affine Space underlying the Vector Space $F^n$

Affine Transformations:
\begin{itemize}
  \item Rigid Transformation (\S\ref{sec:rigid_transformation}) --
    Angle-preserving, Distance Preserving
    \begin{itemize}
      \item Translation (\S\ref{sec:translation})
      \item Rotation (\S\ref{sec:rotation})
      \item Reflection (\S\ref{sec:reflection})
    \end{itemize}
  \end{itemize}
  \item Homothety (Homogeneous Dilation or Central Similarity
    \S\ref{sec:homothety}) -- Preserves Lines and Parallelism
    \begin{itemize}
      \item Uniform Scaling (Linear Homothety \S\ref{sec:uniform_scaling})
    \end{itemize}
  \item Anisotropic Scaling (\S\ref{sec:anisotropic_scaling})
    \begin{itemize}
      \item Directional Scaling (\S\ref{sec:directional_scaling}) -- Stretching
    \end{itemize}
  \item Shear Transformation (Transvection \S\ref{sec:transvection}) --
    Measure-preserving, Co-linearity Preserving
\end{itemize}
and combinations of the above in any combination and sequence

the Invertible Affine Transformations form the Affine Group
(\S\ref{sec:affine_group})

cf. Affine Logic (\S\ref{sec:affine_logic})

\fist Iterated Function Systems (\S\ref{sec:ifs}) as Finite Sets of Contractive
Affine Transformations



\subsubsection{Shear Transformation}\label{sec:shear_transformation}

%FIXME: move this section ???

or \emph{Transvection}



\subsubsection{Self-affinity}\label{sec:self_affinity}

Symmetry under Affine Transformation

cf. Self-similarity (\S\ref{sec:self_similarity})



% ==============================================================================
\section{Euclidean Space}\label{sec:euclidean_space}
% ==============================================================================

\emph{Euclidean Space} $\xspace{E}^n$ is the Metric Space with the Euclidean
Metric (\S\ref{sec:euclidean_metric}) on the Cartesian Space
(\S\ref{sec:cartesian_space}) $\reals^n$ ($n$-dimensional Real Coordinate Space
\S\ref{sec:real_coordinate_space}). This Metric Induces the standard Euclidean
Topology (\S\ref{sec:euclidean_topology}).

Euclidean Geometry (\S\ref{sec:euclidean_geometry}) with a Coordinate System
%FIXME

Euclidean Space is a Homogeneous Space (\S\ref{sec:homogeneous_space}) for its
Symmetry Groups (\S\ref{sec:symmetry_group}) --FIXME

Isomotries of Euclidean Space:
\begin{itemize}
  \item Point Reflection (Central Inversion) -- an Affine Transformation
  \item TODO
  ...
\end{itemize}

$E(n)$ -- Euclidean Group (\S\ref{sec:euclidean_group}): Isometry Group
(\S\ref{sec:isometry_group}) $ISO(n)$; makes Euclidean Geometry
(\S\ref{sec:euclidean_geometry}) a case of Klein Geometry
(\S\ref{sec:klein_geometry})

cf. Affine Spaces (\S\ref{sec:affine_space}): generalizes properties
of Euclidean Spaces

Reflection Groups (\S\ref{sec:reflection_group}) are Discrete Groups Generated
by a Set of Reflections (\S\ref{sec:reflection}) of a Finite-dimensional
Euclidean Space

The Coxeter Groups (\S\ref{sec:coxeter_group}) are precisely the Finite
Euclidean Reflection Groups

Crystallographic Groups (\S\ref{sec:crystallographic_group}) are Cocompact
(\S\ref{sec:cocompact_space}), Discrete Subgroups (\S\ref{sec:discrete_group})
of the Isometries (\S\ref{sec:isometry}) of some Euclidean Space



% ------------------------------------------------------------------------------
\subsection{Euclidean Metric}\label{sec:euclidean_metric}
% ------------------------------------------------------------------------------

or \emph{Euclidean Norm}

canonical Metric on $n$-dimensional Cartesian Space
(\S\ref{sec:cartesian_space}) $\reals^n$

$p$-norm

induces the natural Euclidean Topology



% ------------------------------------------------------------------------------
\subsection{Euclidean Topology}\label{sec:euclidean_topology}
% ------------------------------------------------------------------------------

or \emph{Standard Topology}

natural Topology Induced by the Euclidean Metric

a Set is Open if and only if it contains an Open Ball around each of its Points:
Open Balls form a Base of the Topology; i.e. for any Metric Space, the Open
Balls form a Base for a Topology on that Space and the Euclidean Topology on
$\reals^n$ is the Topology ``generated'' by those Balls



% ------------------------------------------------------------------------------
\subsection{Rigid Transformation}\label{sec:rigid_transformation}
% ------------------------------------------------------------------------------

A \emph{Rigid Transformation} (or \emph{Euclidean Isometry}) is an Isometry
(Equi-affine Similarity \S\ref{sec:isometry}) associated with the Euclidean
Distance Metric of a Euclidean Space, i.e. a \emph{Distance-preserving
  Transformation}.

The Euclidean Group (\S\ref{sec:euclidean_group}) $\mathrm{ISO}(n)$ is the
Symmetry Group of $n$-dimensional Euclidean Space with Rigid Transformations as
its Elements.

a Rigid Transformation is a Similarity Transformation
(\S\ref{sec:similarity_transformation}) where the Scaling Factor $r = 1$

cf. Affine Transformations (\S\ref{sec:affine_transformation})

\fist a Rigid Motion (\S\ref{sec:rigid_motion}) is a Positive Affine
Transformation



\subsubsection{Translation}\label{sec:translation}

\fist Periodic Functions (\S\ref{sec:periodic_function}) are Functions with
Graphs that exhibit Translational Symmetry (\S\ref{sec:symmetry_group})

Group of Dilations (\S\ref{sec:dilation}) -- Translations
(\S\ref{sec:translation}) and Hometheties of an Affine Space (i.e. Affine
Transformations where the Image of every Line is Parallel to that Line)



\subsubsection{Rotation}\label{sec:rotation}

Discrete Rotational Symmetry (\S\ref{sec:discrete_symmetry})



\subsubsection{Reflection}\label{sec:reflection}

Discrete Symmetry (\S\ref{sec:discrete_symmetry})

Reflection Groups (\S\ref{sec:reflection_group}) are Discrete Groups Generated
by a Set of Reflections of a Finite-dimensional Euclidean Space
(\S\ref{sec:reflection})

The Coxeter Groups (\S\ref{sec:coxeter_group}) are precisely the Finite
Euclidean Reflection Groups

\begin{itemize}
  \item Elementary Reflector (Householder Transformation
    \S\ref{sec:elementary_reflector}) -- (Linear, Orthogonal) Reflection about
    a Plane containing the Origin
\end{itemize}



\subsubsection{Congruence}\label{sec:congruence}

(wiki):

two Sets of Points are called \emph{Congruent} if and only if one can be
Transformed into the other by an Isometry, i.e. a combination of Rigid Motions
(\S\ref{sec:rigid_transformation}), viz. Translation, Rotation, and Reflection

\fist a Similarity (\S\ref{sec:similarity_transformation}) between objects
additionally allows for Scaling (\S\ref{sec:scaling})



% ------------------------------------------------------------------------------
\subsection{Homothety}\label{sec:homothety}
% ------------------------------------------------------------------------------

\emph{Homothetic Transformation} or \emph{Homogeneous Dilation}

Affine Transformation (\S\ref{sec:affine_transformation}) determined by a
Homothety Center Point $S$ and a Nonzero Scalar $\lambda$ called the
\emph{Ratio} fixing $S$ and sending any Point $M$ to a Point $N$ such that
$\vec{SN}$ is on the same Line as $\vec{SM}$ but \emph{Scaled} by a factor of
$\lambda$:
\[
  M \mapsto S + \lambda\vec{SM}
\]

Group of Dilations (\S\ref{sec:dilation}) -- Translations
(\S\ref{sec:translation}) and Hometheties of an Affine Space (i.e. Affine
Transformations where the Image of every Line is Parallel to that Line)

\fist Projective Geometry (\S\ref{sec:projective_geometry}): a Homothety is a
Similarity Transformation (\S\ref{sec:similarity_transformation}) that leaves
the Line at Infinity Pointwise Invariant

\fist Scale Invariance (\S\ref{sec:scale_invariance})



% ------------------------------------------------------------------------------
\subsection{Scaling}\label{sec:scaling}
% ------------------------------------------------------------------------------

a \emph{Linear Homothety}

see also:
\begin{itemize}
  \item Scale Analysis (Approximation Theory \S\ref{sec:scale_analysis})
  \item Scale Invariance (\S\ref{sec:scale_invariance})
  \item Statistical Scale Analysis (\S\ref{sec:statistical_scale})
  \item Multidimensional Scaling (\S\ref{sec:multidimensional_scaling})
\end{itemize}



\subsubsection{Uniform Scaling}\label{sec:uniform_scaling}

special case of Homothety (\S\ref{sec:homothety})



\subsubsection{Anisotropic Scaling}\label{sec:anisotropic_scaling}

\paragraph{Directional Scaling}\label{sec:directional_scaling}\hfill

or \emph{Stretching}



% ------------------------------------------------------------------------------
\subsection{Similarity Transformation}\label{sec:similarity_transformation}
% ------------------------------------------------------------------------------

two geometric objects are \emph{Similar} if they have the same ``Shape'' under
Rigid Transformations (Euclidean Isometry \S\ref{sec:rigid_transformation}) and
Uniform Scaling (Linear Homothety \S\ref{sec:scaling})

in Euclidean Geometry, a Similarity Transformation (or \emph{Similitude}) is a
Bijection on the Space to itself that multiplies all Distances by the same
positive Real Number $r$:
\[
  d(f(x), f(y)) = r d(x, y)
\]
where $d$ is the Euclidean Distance; when $r = 1$ the Similarity is a
Rigid Transformation

in Projective Geometry (\S\ref{sec:projective_geometry}) a Similarity
Transformation fixes a given Elliptic Involution (FIXME: clarify)

in Projective Geometry a Homothety (\S\ref{sec:homothety}) is a Similarity
Transformation that leaves the Line at Infinity Pointwise Invariant

\fist cf. Similarity Measure (\S\ref{sec:similarity_measure})



\subsubsection{Self-similarity}\label{sec:self_similarity}

Symmetry under Similarity Transformation

cf. Self-affinity (\S\ref{sec:self_affinity})

\fist Self-similar (Stochastic) Process (\S\ref{sec:self_similar})



% ------------------------------------------------------------------------------
\subsection{Euclidean Plane}\label{sec:euclidean_plane}
% ------------------------------------------------------------------------------

with Cartesian Coordinates Modelled by $\reals^2$ (Cartesian Plane
\S\ref{sec:cartesian_plane})

\fist Extended Euclidean Plane (Real Projective Plane
\S\ref{sec:real_projective_plane})

Frieze Groups (\S\ref{sec:frieze_group}) and Wallpaper Groups
(\S\ref{sec:wallpaper_group}) are Discrete Subgroups
(\S\ref{sec:discrete_group}) of the Isometry Group (\S\ref{sec:isometry_group})
of the Euclidean Plane



\subsubsection{Plane Curve}\label{sec:plane_curve}

Curvature (\S\ref{sec:curvature}) of a Plane Curve $\vec{r}(t) = [x(t),y(t)]$:
\[
  \kappa = \frac{x'(t)y''(t) - y'(t)x''(t)} {(x'(t)^2 + y'(t)^2)^{\frac{3}{2}}}
\]

\fist Simple Closed Curve (\S\ref{sec:simple_closed_curve})



\paragraph{Inflection Point}\label{sec:inflection_point}\hfill

\paragraph{Implicit Curve}\label{sec:implicit_curve}\hfill

a Plane Curve defined by an \emph{Implicit Equation}
(\S\ref{sec:implicit_equation}) relating Coordinate Variables $x$ and $y$:
\[
  F(x,y) = 0
\]
if $F(x,y)$ is a Polynomial in two Variables, then the corresponding Curve is an
\emph{Algebraic Curve} (\S\ref{sec:algebraic_curve})

any Parameterized Curve (\S\ref{sec:parametric_curve}) can also be defined as an
Implicit Curve

\fist cf. Implicit Surface (\S\ref{sec:implicit_surface})



\paragraph{Smooth Plane Curve}\label{sec:smooth_plane_curve}\hfill

\paragraph{Algebraic Plane Curve}\label{sec:algebraic_plane_curve}\hfill

an Irreducible Plane Curve of Degree $d$ with $s$ Singular Points has Geometric
Genus (\S\ref{sec:geometric_genus}):
\[
  g = \frac{(d-1)(d-2)}{2} - s
\]



\paragraph{Parabola}\label{sec:parabola}\hfill

\paragraph{Cardioid}\label{sec:cardioid}\hfill

can be defined as the Inversion of a Parabola across the Unit Circle

\paragraph{Nephroid}\label{sec:nephroid}\hfill

\paragraph{Deltoid}\label{sec:deltoid}\hfill

or Tricuspoid



\paragraph{Sinusoid}\label{sec:sinusoid}\hfill

or \emph{Sine Wave}; the Graph of a Sine Function
(\S\ref{sec:trigonometric_function})

the Sine Wave is the \emph{only} Periodic Waveform (\S\ref{sec:waveform}) that
has the Property that it retains its ``wave shape'' when added to another Sine
Wave of the same Frequency and arbitrary Phase and Magnitude (FIXME: clarify)



\subsubsection{Tessellation}\label{sec:tessellation}

\fist Translation (\S\ref{sec:translation}), Symmetry Group
(\S\ref{sec:symmetry_group})



\paragraph{Uniform Tiling}\label{sec:uniform_tiling}\hfill

Fundamental Domain (\S\ref{sec:fundamental_domain})



% ------------------------------------------------------------------------------
\subsection{3-space}\label{sec:3_space}
% ------------------------------------------------------------------------------

``Euclidean Space''

with Cartesian Coordinates Modelled by $\reals^3$

``Model'' Riemannian Manifold (\S\ref{sec:riemannian_manifold})

cf. Minkowski Space (\S\ref{sec:minkowski_space}) $\reals^{n-1,1}$
with the Flat Minkowski Metric (\S\ref{sec:minkowski_metric}) as the
``Model'' Lorentzian Manifold



\subsubsection{Implicit Surface}\label{sec:implicit_surface}

\fist cf. Implicit Curve (\S\ref{sec:implicit_curve})

$F(x,y,z) = 0$

\fist cf. Parametric Surface (\S\ref{sec:parametric_surface})

\fist Surface ($2$-manifold \S\ref{sec:surface})



\paragraph{Non-algebraic Surface}\label{sec:nonalgebraic_surface}\hfill

if $F(x,y,z)$ is Polynomial in $x$, $y$, $z$, then the surface is
Algebraic, otherwise it is Non-algebraic



\subsubsection{Minkowski Sum}\label{sec:minkowski_sum}

or \emph{Dilation}

\emph{Brunn-Minkowski Inequality} --
\url{https://golem.ph.utexas.edu/category/2017/01/the_brunnminkowski_inequality.html}



% ------------------------------------------------------------------------------
\subsection{Euclidean Vector}\label{sec:euclidean_vector}
% ------------------------------------------------------------------------------

a \emph{Vector} (\S\ref{sec:vector}) of a Euclidean Space, being a specific
kind of Vector Space (\S\ref{sec:vector_space}), is defined as an
``\emph{Arrow}'' in Euclidean Space defined by an \emph{Initial Point} and a
\emph{Terminal Point}

Positive-definite (\S\ref{sec:positive_definite}) Inner Product

\fist Root Systems (\S\ref{sec:root_system})



\subsubsection{Bound Vector}\label{sec:bound_vector}

a \emph{Bound Vector} is a Euclidean Vector with \emph{fixed} Initial and
Terminal Points

generalized as \emph{Tangent Vectors} (\S\ref{sec:tangent_space}) in the
context of Tangent Spaces

if the Euclidean Space has an Origin then a Free Vector is equivalent to a
Bound Vector of the same Magnitude and Direction with Initial Point at the
Origin



\subsubsection{Free Vector}\label{sec:free_vector}

a \emph{Free Vector} is a Euclidean Vector without a fixed Initial Point, e.g.
a Vector only defined by \emph{Magnitude} and \emph{Direction}

if the Euclidean Space has an Origin then a Free Vector is equivalent to a
Bound Vector of the same Magnitude and Direction with Initial Point at the
Origin



% ------------------------------------------------------------------------------
\subsection{Pseudo-euclidean Space}\label{sec:pseudo_euclidean}
% ------------------------------------------------------------------------------

\subsubsection{Minkowski Space}\label{sec:minkowski_space}

$\reals^{n-1,1}$

with the Flat Minkowski Metric (\S\ref{sec:minkowski_metric}) as the
``Model'' Lorentzian Manifold (\S\ref{sec:lorentzian_manifold})

cf. Euclidean Space (\S\ref{sec:euclidean_space}) $\reals^n$ as the
``Model'' Riemannian Manifold (\S\ref{sec:riemannian_manifold})



\paragraph{Minkowski Metric}\label{sec:minkowski_metric}\hfill

Flat

Curved



\paragraph{de Sitter Space}\label{sec:desitter_space}\hfill

Minkowski Space analog of Sphere (\S\ref{sec:euclidean_sphere}) in ordinary
Euclidean Space



% ==============================================================================
\section{Hyperbolic Space}\label{sec:hyperbolic_space}
% ==============================================================================

Homogeneous Space (\S\ref{sec:homogeneous_space}) with Constant Negative
Sectional Curvature (\S\ref{sec:sectional_curvature})

Upper Half-space (\S\ref{sec:half_space})



% ==============================================================================
\section{Parametric Equation}\label{sec:parametric_equation}
% ==============================================================================

A \emph{Parametric Equation} expresses the Coordinates of the Points making up a
Geometric Object

Parametric Equations can describe Curves (\S\ref{sec:parametric_curve}),
Surfaces (\S\ref{sec:parametric_surface}), and also Manifolds
(\S\ref{sec:manifold}) and Algebraic Varieties (\S\ref{sec:algebraic_variety})
of higher Dimensions with the number of Parameters $m$ being equal to the
Dimension of the Manifold or Variety, and the number of Equations



\subsubsection{Parameterization}\label{sec:parameterization}

A \emph{Parameterization} (or \emph{Parametric Representation}) is Set of
Parametric Equations used to express the Coordinates of the Points that make up
a Geometric Object such as a Curve (\S\ref{sec:parametric_curve}) or Surface
(\S\ref{sec:parametric_surface}).

also the process of finding a Parametric Representation for a Manifold or
Variety from an Implicit Equation (\S\ref{sec:implicit_equation}) is called
\emph{Parameterization}; the inverse process is the \emph{Implicitization}
(\S\ref{sec:implicitization})

cf. a \emph{Set of Simultaneous Equations} (\emph{System of Equations}
\S\ref{sec:equation_system})

\url{https://math.stackexchange.com/questions/662009/what-is-the-difference-between-vector-valued-functions-and-parametric-equations}:

A Parametrization for a portion of a Submanifold $M$ in Euclidean Space is a
Map:
\[
  \varphi : U \subset R^m \rightarrow M \subset R^n
\]
with additional Properties:
\begin{itemize}
  \item $U$ is an Open Set (\S\ref{sec:open_set})
  \item $\varphi$ is a Homeomorphism (\S\ref{sec:homeomorphism}) onto its Image
  \item everywhere the Rank of $D\varphi = m$ (FIXME: clarify)
\end{itemize}

\fist cf. a \emph{Vector-valued Function} (\S\ref{sec:vector_function}) is a
Map:
\[
  f : U \subset R^m \rightarrow V \subset R^n
\]
i.e. a Parameterization is always in the form of a Vector Valued Function, but
conversely Vector-valued Functions are used to Parameterize Varieties
(\S\ref{sec:algebraic_variety}) %FIXME: clarify



\paragraph{Parametric Curve}\label{sec:parametric_curve}\hfill

\fist Curve (Topology \S\ref{sec:curve}) -- a Topological Space Homeomorphic to
a Line (\S\ref{sec:algebraic_line})

any Parameterized Curve can also be defined as an Implicit Curve
(\S\ref{sec:implicit_curve})

Line Integral (\S\ref{sec:line_integral})

Integral Curve (\S\ref{sec:integral_curve}), Isocline (\S\ref{sec:isocline})



\subparagraph{Unit Speed Curve}\label{sec:unit_speed_curve}\hfill

in 3 Dimensions, Satisfies the Differential Equation
(\S\ref{sec:differential_equation}):
\[
  x''(t) + y''(t) + z''(t) = 1
\]
where $x(t)$, $y(t)$, and $z(t)$ are Unknown Functions of the single
Independendent Variable $t$



\subparagraph{Space-filling Curve}\label{sec:space_filling_curve}\hfill



\paragraph{Parametric Surface}\label{sec:parametric_surface}\hfill

3-dimensional Parametric Function on
$\reals^2$:
\[
  \vec{r} : \reals^2 \rightarrow \reals^3
\]

Surface Integral (\S\ref{sec:surface_integral})

Stokes' Theorem, Divergence Theorem (TODO)



% ==============================================================================
\section{Curvature}\label{sec:curvature}
% ==============================================================================

Radius of Curvature $R$

\emph{Curvature} $\kappa = \frac{1}{R}$

for Unit Tangent Vector $\hat{T}$, Arclength $s$:
\[
  \kappa = \|\frac{d\hat{T}}{ds}\|
    = \|\frac{\frac{d\hat{T}}{dt}}{\frac{d\vec{r}}{dt}}\|
\]
where $\vec{r}(t)$ is some Parametric Function

for a Plane Curve (\S\ref{sec:plane_curve}) $\vec{r}(t) = [x(t),y(t)]$:
\[
  \kappa = \frac{x'(t)y''(t) - y'(t)x''(t)} {(x'(t)^2 + y'(t)^2)^{\frac{3}{2}}}
\]
where the Numerator is equal to $\vec{r}'(t) \times \vec{r}''(t)$ and:
\[
  \kappa = \frac{\vec{r}' \times \vec{r}''}{\|\vec{r}'\|^3}
\]

\fist Line (Geometry \S\ref{sec:line}) -- a Primitive Geometric Object of Zero
Curvature



% ==============================================================================
\section{$n$-sphere}\label{sec:n_sphere}
% ==============================================================================

$S^n = \{ x \in \reals^{n+1} : \|x\| = r \}$

$S^n = \reals \cup \{\infty\}$

Elliptic Manifold (\S\ref{sec:elliptic_manifold})

familiy of Manifolds

\emph{Hypersphere}

$S^0$ is Isomorphic to the $1$-dimensional Orthogonal Group $O(1)$, a
two-point Discrete Space (\S\ref{sec:discrete_space})

the Sum of the Volumes of Even-dimensional Unit $n$-spheres of Radius $r$
Converges to $e^\pi$:
\[
  \sum_{m=0}^\infty V_{2m}(S^{(2m-1)}) = e^\pi
\]

Unit Spheres in Dimension $\reals^n$: when $n = 1, 2, 4$ Spheres are
\emph{Groups} (\S\ref{sec:group})



the Real Closed Interval (Unit 0-disc) $[-1,1]$ is a Semigroup
(\S\ref{sec:semigroup}) under Multiplication ($0$ does not have an Inverse)



% ------------------------------------------------------------------------------
\subsection{Unit Circle}\label{sec:unit_circle}
% ------------------------------------------------------------------------------

$S^1$



% ------------------------------------------------------------------------------
\subsection{Unit Disk}\label{sec:unit_disk}
% ------------------------------------------------------------------------------

$D_1$



% ------------------------------------------------------------------------------
\subsection{Unit Sphere}\label{sec:unit_sphere}
% ------------------------------------------------------------------------------

Unit Ball (\S\ref{sec:unit_ball})

$S^2$

Quaternion (\S\ref{sec:quaternion}) Solution for $\sqrt{-1}$

\begin{itemize}
  \item Bloch Sphere (\S\ref{sec:bloch_sphere}) -- geometric representation of
    Pure State Space of a Two-level Quantum System (Qubit \S\ref{sec:qubit}):
    the Complex Projective Line (\S\ref{sec:complex_projective_space})
    $\mathbb{CP}^1$, a unit $2$-sphere with Antipodal Points corresponding to a
    point of mutually Orthogonal State Vectors and North and South Poles
    typically chosen to correspond to the standard Basis Vectors $|0\rangle$ and
    $|1\rangle$
\end{itemize}



% ------------------------------------------------------------------------------
\subsection{Unit Glome}\label{sec:unit_glome}
% ------------------------------------------------------------------------------

$S^3$

Diffeomorphic to the Special Unitary Group (\S\ref{sec:unitary_group}) $SU(2)$

Group of Unit Quaternions (\S\ref{sec:quaternion})

mapping from Unit $3$-sphere in the Two-dimensional State Space $\comps^2$
(FIXME: clarify) to the Bloch Sphere (\S\ref{sec:bloch_sphere}, i.e. $2$-sphere
\S\ref{sec:unit_sphere}) is the \emph{Hopf Fibration}
(\S\ref{sec:hopf_fibration})



% ==============================================================================
\section{Pseudosphere}\label{sec:pseudosphere}
% ==============================================================================

Hyperbolic Manifold (\S\ref{sec:hyperbolic_manifold})

Tractricoid



% ==============================================================================
\section{$n$-torus}\label{sec:n_torus}
% ==============================================================================

family of Manifolds



% ==============================================================================
\section{Differential Geometry}\label{sec:differential_geometry}
% ==============================================================================

Differentiable Manifolds (\S\ref{sec:differentiable_manifold})

\fist Differential Calculus (\S\ref{sec:differential_calculus})

\fist Differential Topology (\S\ref{sec:differential_topology})

\fist Differential Curve (\S\ref{sec:differential_curve})

\fist Differential Algebra (\S\ref{sec:differential_algebra})

\fist Geometric Calculus (Geometric Algebra \S\ref{sec:geometric_calculus})

\fist Synthetic Differential Geometry
(\S\ref{sec:synthetic_differential_geometry}) -- formalization of Differential
Geometry in the language of Topos Theory (\S\ref{sec:topos_theory});
Differentials (\S\ref{sec:differential}) in ``Smooth Models'' of Set Theory
(TODO: explain)



% ------------------------------------------------------------------------------
\subsection{Tangency}\label{sec:tangency}
% ------------------------------------------------------------------------------

% ------------------------------------------------------------------------------
\subsection{Transversality}\label{sec:transversality}
% ------------------------------------------------------------------------------

formalizes idea of a ``Generic Intersection'' from Differential Topology
(\S\ref{sec:differential_topology})

cf. Transversal (Combinatorics \S\ref{sec:transversal})

General Position (\S\ref{sec:general_position})

defined by ``Linearizations'' of the Intersecting Spaces at the Point of
Intersection

two Submanifolds of a given Finite-dimensional Smooth Manifold are said to
Intersect \emph{Transversally} if at every Point of Intersection their separate
Tangent Spaces at the Point taken together gives the Tangent Space of the
Ambient Manifold at that Point

Non-intersecting (Sub-)manifolds that do not Intersect are ``Vacuously
Transverse''

if an Intersection is Transverse



% ------------------------------------------------------------------------------
\subsection{Geodesic}\label{sec:geodesic}
% ------------------------------------------------------------------------------

generalization of straight Lines to Curved Spaces



% ------------------------------------------------------------------------------
\subsection{Differentiable Curve}\label{sec:differentiable_curve}
% ------------------------------------------------------------------------------

a $1$-manifold

\fist Curve (\S\ref{sec:curve}) -- a Topological Space Homeomorphic to a Line
(\S\ref{sec:algebraic_line})



\subsubsection{Regular Curve}\label{sec:regular_curve}

\subsubsection{Fundamental Theorem of Space Curves}
\label{sec:fundamental_curve_theorem}

``\emph{Every Regular Curve in Three-dimensional Space with Non-zero
Curvature is completely determined by a pair of Scalar Fields, Curvature
($\kappa$) and Torsion ($\tau$)}''



% ------------------------------------------------------------------------------
\subsection{Connection}\label{sec:connection}
% ------------------------------------------------------------------------------

precise notion of ``Transporting'' ``Data'' along a Curve or Family of Curves
in a \emph{Parallel} and consistent manner

allows comparison of ``Local Geometry'' at different Points

\emph{Local Theory}:
\begin{itemize}
  \item Parallel Transport
  \item Holonomy
\end{itemize}

\emph{Infinitesimal Theory}:
\begin{itemize}
  \item Differentiation of ``Geometric Data''; a Covariant Derivative
    (\S\ref{sec:covariant_derivative}) is a way of specifying a Derivative of a
    Vector Field along another Vector Field on a Manifold-- see Affine
    Connection (\S\ref{sec:affine_connection})
\end{itemize}



\subsubsection{Parallel Transport}\label{sec:parallel_transport}

(wiki):

Parallel Transport supplies:
\begin{itemize}
  \item a local realization of Connection
  \item a local realization of Curvature-- \emph{Holonomy}
    (\S\ref{sec:holonomy})
\end{itemize}

\fist cf. Tangent Spaces (\S\ref{sec:tangent_space})

a Simply-connected Riemannian Manifold is a \emph{Symmetric Space}
(\S\ref{sec:symmetric_space}) if and only if its \emph{Curvature Tensor}
(\S\ref{sec:riemannian_curvature}) is Invariant under Parallel Transport



\paragraph{Holonomy}\label{sec:holonomy}\hfill

The \emph{Holonomy} of a Connection on a Smooth Manifold
(\S\ref{sec:smooth_manifold}) measures the extent to which Parallel Transport
around a Closed Loop fails to preserve the ``Geometrical Data'' being
Transported.

\emph{Path-dependence}

\emph{Local realization} of Curvature

for Flat Connections (\S\ref{sec:flat_connection}), the associated Holonomy is
a type of Monodromy (\S\ref{sec:monodromy})

a \emph{Holonomic Constraint} (\S\ref{sec:holonomic_constraint}) is one where
there is \emph{no} Geometric Holonomy (Path-dependence)

cf. Non-integrability (Non-integrable Systems \S\ref{sec:nonintegrable_system})

\fist cf. Holonomic Functions (\S\ref{sec:holonomic_function}), Holonomic
Modules (\S\ref{sec:holonomic_module})

\url{https://physics.stackexchange.com/a/410034/36436}:

\emph{Holonomy Group} of a Connection is the Set of Transformations an
``object'' can ``experience'' when it is Parallel Transported in a Loop-- if
the associated Holonomy Groups of a Constraint are ``Non-trivial'', then the
Constraint cannot be Holonomic (\S\ref{sec:holonomic_constraint}) because the
Orientation of the object will depend on the Loop traversed, not just the
current State (FIXME: clarify)



\subsubsection{Affine Connection}\label{sec:affine_connection}

(wiki):

means of ``Transporting'' Vectors Tangent to a Manifold from one Point to
another along a Curve

\emph{Connects} nearby Tangent Spaces (\S\ref{sec:tangent_space}) permitting
Tangent Vector Fields (\S\ref{sec:vector_field}) to be Differentiated
(\S\ref{sec:derivative}) as if they were Functions on the Manifold with Values
in a fixed Vector Space

typically given as a \emph{Covariant Derivative}
(\S\ref{sec:covariant_derivative})



\subsubsection{Flat Connection}\label{sec:flat_connection}

a Connection is Flat if its Curvature Form $\omega$ vanishes, or equivalently
if the Structure Group can be reduced to the same underlying Group but with the
Discrete Topology (FIXME: clarify, xref)

associated Holonomy (\S\ref{sec:holonomy}) is a type of Monodromy
(\S\ref{sec:monodromy})



% ------------------------------------------------------------------------------
\subsection{Geometric Flow}\label{sec:geometric_flow}
% ------------------------------------------------------------------------------

Gradient Flow (\S\ref{sec:gradient_flow}) associated to a Functional
(\S\ref{sec:functional}) on a Manifold



% ------------------------------------------------------------------------------
\subsection{Klein Geometry}\label{sec:klein_geometry}
% ------------------------------------------------------------------------------

a \emph{Klein Geometry} is a pair $(G,H)$ of a Lie Group
(\S\ref{sec:lie_group}) $G$ and $H$ a Closed Lie Subgroup of $G$ such
that the (Left) Coset Space (\S\ref{sec:coset_space}) $G|H$ is
Connected (\S\ref{sec:connected_space})

$G|H$ is called the \emph{Space} of the Geometry and is a Smooth
Manifold (\S\ref{sec:smooth_manifold}) of Dimension:
\[
  dim(X) = dim(G) - dim(H)
\]

\emph{Erlangen Program} -- Projective Geometry as least restrictive
``unifying frame'' for other Geometries considered; from least to more
restrictive: Projective Geometry, Affine Geometry, Euclidean Geometry

examples: (TODO)

Projective Geometry (\S\ref{sec:projective_geometry})

Conformal Geometry (\S\ref{sec:conformal_geometry}) on a Sphere

Hyperbolic Geometry

Elliptic Geometry

Spherical Geometry

Affine Geometry

Euclidean Geometry



% ------------------------------------------------------------------------------
\subsection{Riemannian Geometry}\label{sec:riemannian_geometry}
% ------------------------------------------------------------------------------

Riemannian Manifold (\S\ref{sec:riemannian_manifold})

Pseudo-Riemannian Manifold (\S\ref{sec:pseudo_riemannian})

Symplectic Form (\S\ref{sec:symplectic_form}) in Symplectic Geometry
(\S\ref{sec:symplectic_geometry}) plays the role analagous to Metric Tensor
(\S\ref{sec:metric_tensor}) in Riemannian Geometry



\subsubsection{Curvature}\label{sec:reimannian_curvature}

%FIXME: xref curvature

\emph{Extrinsic Curvature} -- defined for objects \emph{embedded} in another
Space (usually Euclidean Space) in a way that relates to the Radius of
Curvature of Circles that \emph{touch} the Object

\emph{Intrinsic Curvature} -- defined in terms of the \emph{lengths} of Curves
within a Riemannian Manifold

Scalar Field (\S\ref{sec:scalar_field}) $\kappa$; Fundamental Theorem of Curves

a Simply-connected Riemannian Manifold is a Symmetric Space
(\S\ref{sec:symmetric_space}) if and only if its \emph{Curvature Tensor} is
Invariant under Parallel Transport



\paragraph{Sectional Curvature}\label{sec:sectional_curvature}\hfill

%FIXME: xref curvature

Riemannian Manifolds (\S\ref{sec:riemannian_manifold}) with Constant Sectional
Curvature are called \emph{Space Forms} (\S\ref{sec:space_form}):
\begin{itemize}
  \item $0$: Euclidean Space (\S\ref{sec:euclidean_space}) -- Constant
    Vanishing Sectional Curvature
  \item $-1$: Hyperbolic Space (\S\ref{sec:hyperbolic_space}) -- Constant
    Negative Sectional Curvature; Hyperbolic Manifold
    (\S\ref{sec:hyperbolic_manifold})
  \item $+1$: Elliptic Geometry (\S\ref{sec:elliptic_space}) -- Constant
    Positive Sectional Curvature; $n$-spheres (\S\ref{sec:n_sphere});
    (\S\ref{sec:elliptic_manifold})
\end{itemize}



\subsubsection{Symmetric Space}\label{sec:symmetric_space}

a Pseudo-Riemannian Manifold (\S\ref{sec:pseudo_riemannian}) whose Group of
Symmetries (\S\ref{sec:symmetry_group}) contains an \emph{Inversion Symmetry}
about every Point

a Simply-connected Riemannian Manifold is a Symmetric Space if and only if its
\emph{Curvature Tensor} (\S\ref{sec:riemannian_curvature}) is Invariant under
Parallel Transport

examples:
\begin{itemize}
  \item Euclidean Space
  \item Spheres
  \item Hyperbolic Spaces
  \item Projective Spaces
  \item Anti-de Sitter Space
  \item ...
\end{itemize}



% ------------------------------------------------------------------------------
\subsection{Symplectic Geometry}\label{sec:symplectic_geometry}
% ------------------------------------------------------------------------------

Symplectic Manifold (\S\ref{sec:symplectic_manifold}) is a Smooth
Even-dimensional Manifold equipped with a Closed Non-degenerate $2$-form called
the \emph{Symplectic Form} (\S\ref{sec:symplectic_form})

the Symplectic Form in Symplectic Geometry plays the role analagous to Metric
Tensor (\S\ref{sec:metric_tensor}) in Riemannian Geometry
(\S\ref{sec:riemannian_geometry})

the Exterior Derivative (\S\ref{sec:exterior_derivative}) of a Tautological
$1$-form (\S\ref{sec:tautological_1form}) defined on the Cotangent Bundle
(\S\ref{sec:cotangent_bundle}) $T * Q$ of a Manifold $Q$ defines a Symplectic
Form giving $T * Q$ the Structure of a Symplectic Manifold

arises in formalisms of Classical Mechanics (e.g. Hamiltonian Systems
\S\ref{sec:hamiltonian_system}) considering either the Even-dimensional Phase
Space (\S\ref{sec:phase_space}) of a Mechanical System or the Odd-dimensional
Constant-energy Hypersurface

(wiki): Symplectic Geometry \& Topology is at the ``boundary'' between the
fields of Topology and Geometry

\fist cf. Contact Geometry (\S\ref{sec:contact_geometry}): Odd-dimensional
counterpart of Symplectic Geometry

\fist Holonomic Modules (\S\ref{sec:holonomic_module}): the Characteristic
Variety $Ch(M)$ (FIXME: xref) of any $D$-module $M$, when seen as a Subvariety
of the Cotangent Bundle $T^*X$ of $X$, is an \emph{Involutive Variety} (FIXME:
xref), i.e. the Module is Holonomic if and only if $Ch(M)$ is a Lagrangian
Submanifold (\S\ref{sec:lagrangian_submanifold})

Darboux's Theorem (\S\ref{sec:darbouxs_theorem})

\emph{Geometric Quantization}

1999 - Echeverria-Enriquez, Munoz-Lecanda, Roman-Roy, Victoria-Monge -
\emph{Mathematical Foundations of Geometric Quantization}

\url{https://johncarlosbaez.wordpress.com/2018/12/01/geometric-quantization-part-1/}:
a Space of Quantum States can be seen as a Space of Classical States

Quantum States are not really ``Vectors'' in a Hilbert Space $H$-- from a
certain point of view they are really $1$-dimensional Subspaces of a Hilbert
space, i.e. the Set of Quantum States is the \emph{Projective} Space $P H$; when
such a Projective Space is Finite-dimensional, it is the simplest example of a
\emph{K\"ahler Manifold} equipped with a \emph{Holomorphic Hermitian Line
  Bundle} whose Curvature is the Imaginary Part of the K\"ahler Structure

Geometrically Quantizing $PH$ gets back the Hilbert Space $H$

taking a Quantum System with Hilbert Space $H$ can be thought of as a Classical
System whose \emph{Symplectic Manifold} of States is $P H$, which can be
Geometrically Quantized to get $H$ back

1997 - Astekar, Schilling - \emph{Geometrical Formulation of Quantum Mechanics}

$N$th Quantization -- \url{http://math.ucr.edu/home/baez/nth_quantization.html}

$0$th Quantization: apply $F : Set \rightarrow Hilb$ (takes any Set and returns
a Hilbert Space) to the Empty Set, giving the $0$-dimensional Hilbert Space
$\xspace{0}$ where the only Vector is the Zero Vector, and the corresponding
System has no States

$1$st Quantization: apply $K : Hilb \rightarrow Hilb$ (\emph{Fock Space} of $H$)
to $\xspace{0}$ gives $K(\xspace{0}) = \comps$, which is the Hilbert Space of a
System with no ``degrees of freedom'', i.e. only one State (modulo Phase)-- cf.
Photons; only Ket is $|\rangle$

$2$nd Quantization: apply $K$ again $K(K(\xspace{0})) = K(\comps)$ gives the
Hilbert Space of an arbitrary Finite collection of indistinguishable
``\emph{Quanta}''; Kets are $|0\rangle, |1\rangle, |2\rangle, |3\rangle,
\ldots$; this is the Hilbert Space of a \emph{Quantized Harmonic Oscillator}

$3$rd Quantization: $K(K(K(\xspace{0}))) = K(K(\comps))$ is the Hilbert Space of
the ``right-moving modes'' of a String (FIXME: clarify)

$4$th Quantization: Hilbert Space for a $2$-brane

\url{https://johncarlosbaez.wordpress.com/2018/12/26/geometric-quantization-part-2/}

\url{https://johncarlosbaez.wordpress.com/2018/12/27/geometric-quantization-part-3/}:

Quantization and Projectivization are Adjoint Functors



% ------------------------------------------------------------------------------
\subsection{Contact Geometry}\label{sec:contact_geometry}
% ------------------------------------------------------------------------------

\fist cf. Symplectic Geometry (\S\ref{sec:symplectic_geometry}):
Even-dimensional counterpart of Contact Geometry

geometric structure on Smooth Manifolds given by a Hyperplane Distribution
(\S\ref{sec:tangent_bundle_distribution}) in the Tangent Bundle satisfying the
\emph{Complete Non-integrability} condition; may be equivalently given
(Locally) as the Kernel of a Differential $1$-form
(\S\ref{sec:differential_form}) and the Non-integrability condition trnslates
into a \emph{Maximal Non-degeneracy Condition}) on the Form

Hypersurface (\S\ref{sec:hypersurface}) a Hypersurface of ``Contact Type'';
Liouville Vector Field (Canonical Vector Field on a Tangent Bundle
\S\ref{sec:liouville_vector_field}); \fist Contact Geometry
(\S\ref{sec:contact_geometry})

\emph{Contact Structure} (\S\ref{sec:contact_structure})

\fist Contact Manifold (\S\ref{sec:contact_manifold}) --
\url{http://www.map.mpim-bonn.mpg.de/Contact_manifold}

Non-holonomic Constraints: Non-integrable



\subsubsection{Contact Structure}\label{sec:contact_structure}

$\xi \subset TM$

a \emph{Contact Structure} is a Smooth Field $\xi$ of Hyperplanes
(\S\ref{sec:hyperplane}) which is a Subbundle of the Tangent Bundle of an
Odd-dimensional Manifold: $\xi \subseteq T X$

(if $X$ has Dimension $2n+1$, $\xi$ has Dimension $2n$) %FIXME: correct?

$\xi$ is a Distribution (\S\ref{sec:tangent_bundle_distribution}) ??? FIXME

the Field of Hyperplanes is Maximally Non-integrable, i.e. the opposite of
being Tangent to a Foliation
%FIXME clarify

if $Y$ is a Submanifold of $X$, $Y \subset X$, and the Tangent Space of $Y$ at
every Point is inside the Distribution $\xi$, $T Y \subseteq \xi$, then the
Dimension of $Y \leq n$ (FIXME: 2n ???); if $dim(Y) = n$ then $Y$ is Legendrian
(FIXME: explain, xref)

(wiki): equivalently such a Distribution can be given Locally by the Kernel of
a Differential $1$-form and the Non-integrability Condition translates into a
Maximal Non-degeneracy Condition for the $1$-form

A \emph{Contact Manifold} (\S\ref{sec:contact_manifold}) is a Pair $(M,\xi)$ of
an Odd-dimensional Smooth Manifold $M$ with a Contact Structure $\xi$.

2002 - Gadgil - \emph{Contact Structures on Elliptic 3-manifolds}
-- Elliptic Manifolds (\S\ref{sec:elliptic_manifold})

2002 - Etnyre - *Introductory Lectures on Contact Geometry*

a \emph{Plane Field} $\xi$ on $M$ is a Subbundle of the Tangent Bundle $TM$
such that $\xi_p = T_p M \cap \xi$ is a 2-dimensional Subspace of $T_pM$ for
each $p \in M$



% ------------------------------------------------------------------------------
\subsection{Lie Theory}\label{sec:lie_theory}
% ------------------------------------------------------------------------------

Lie Group (\S\ref{sec:lie_group})

Lie Algebra (\S\ref{sec:lie_algebra})

Lie Group-Lie Algebra Correspondence

uses Lie Groups used for analysing the Continuous Symmetries of
Differential Equations %FIXME

cf. Galois Theory (\S\ref{sec:galois_theory}) uses Permutation Groups
for analysing the Discrete Symmetries of Algebraic Equations %FIXME



\subsubsection{Root System}\label{sec:root_system}

for a Finite-dimensional Euclidean Vector Space (\S\ref{sec:euclidean_vector})
with standard Euclidean Inner Product $(\cdot,\cdot)$, a \emph{Root System},
$\Phi$, in $E$ is a Finite Set of Non-zero Vectors called \emph{Roots} that
satisfy:
\begin{enumerate}
  \item the Roots Span (\S\ref{sec:linear_span}) $E$
  \item the only Scalar Multiples of a Root $\alpha \in \Phi$ belonging to
    $\Phi$ are $\alpha$ itself and its inverse $-\alpha$
  \item for all Roots $\alpha \in \Phi$, $\Phi$ is closed under Reflection
    through the Hyperplane perpendicular to $\alpha$
\end{enumerate}

adding the \emph{Integrality Condition} defines a \emph{Crystallographic Root
  System}:
\begin{enumerate}
  \setcounter{enumi}{3}
  \item (\textbf{Integrality}) for Roots $\alpha, \beta \in \Phi$, the
    Projection of $\beta$ onto the Line through $\alpha$ is an Integer or
    Half-integer multiple of $\alpha$
\end{enumerate}

Semisimple Lie Algebras (\S\ref{sec:semisimple_lie})

Lie Groups (\S\ref{sec:lie_group})



\subsubsection{Lie Algebra}\label{sec:lie_algebra}

(or \emph{Infinitesimal Group})

Ininitesimal Transformations
(\S\ref{sec:infinitesimal_transformation})

Vector Space (\S\ref{sec:vector_space}) with a Non-associative
Multiplication called a \emph{Lie Bracket} $[x,y]$

when an Algebraic Product is defined on the Space, the Lie Bracket is
the Commutator $[x,y] = xy - yx$ %FIXME

the Tangent Space (\S\ref{sec:tangent_space}) of a Lie Group
(\S\ref{sec:lie_group}) can be naturally given the structure of a Lie Algebra
and can be used to classify Compact Lie Groups

(\url{https://twitter.com/sigfig/status/1111288427985752070}):

the Class of PDEs (\S\ref{sec:pde}) that admit Feedback Control Systems
(\S\ref{sec:control_system}) are the PDEs that have Finite-dimensional Lie
Algebras

a ``Controllability Manifold'' exists for any System with a Computable Lie
Bracket

Poisson Algebra (\S\ref{sec:poisson_algebra})



\paragraph{Semisimple Lie Algebra}\label{sec:semisimple_lie}\hfill

Root Systems (\S\ref{sec:root_system})



\paragraph{Universal Enveloping Algebra}
\label{sec:universal_enveloping_algebra}\hfill

most general Unital Associative Algebra containing all Representations
(\S\ref{sec:representation_theory}) of a Lie Algebra
--FIXME: different from ``representation theory'' ???

\fist ``the'' Weyl Algebra (Symplectic Clifford Algebra
\S\ref{sec:weyl_algebra}) is a Quotient of the Universal Enveloping Algebra of
the Heisenberg Algebra (the Lie Algebra of the Heisenberg Group) by setting the
Central Element of the Heisenberg Algebra $([X,Y])$ equal to the Unit of the
Universal Enveloping Algebra ($1$)



\subparagraph{Algebra of Symbols}\label{sec:symbol_algebra}\hfill

$\star(\lie{g})$

Star Product $\star$

e.g. Moyal Product (Phase-space Star product of Phase Space Formulation of
Quantum Mechanics)



\subsubsection{Lie Group}\label{sec:lie_group}

``Manifolds with Group Structure''

Continuous Transformation Group
(\S\ref{sec:continuous_transformation_group}) that is a Smooth
Differentiable Manifold (\S\ref{sec:differentiable_manifold})

Any Discrete Group (\S\ref{sec:discrete_group}) can be viewed as a
$0$-dimensional Lie Group %FIXME

a Euclidean Vector Space (\S\ref{sec:vector_space}) with the Group Operation of
Vector Addition is an example of a Non-compact (\S\ref{sec:compact_space}) Lie
Group

the Tangent Space (\S\ref{sec:tangent_space}) of a Lie Group can be naturally
given the structure of a Lie Algebra (\S\ref{sec:lie_algebra}) and can be used
to classify Compact Lie Groups

\begin{itemize}
  \item Indefinite Orthogonal Group (\S\ref{sec:indefinite_orthogonal_group})
\end{itemize}



\paragraph{Lie Subgroup}\label{sec:lie_subgroup}\hfill

naturally an Immersed Submanifold (\S\ref{sec:immersed_submanifold})



\paragraph{Dynkin Diagram}\label{sec:dynkin_diagram}\hfill

cf. Coxeter-Dynkin Diagrams (\S\ref{sec:coxeter_dynkin_diagram})



\paragraph{Special Unitary Group}\label{sec:special_unitary}\hfill

$\mathrm{SU}(n)$ -- the Lie Group of $n \times n$ Unitary Matrices
(\S\ref{sec:unitary_matrix}) with Determinant $1$

Unitary Group (\S\ref{sec:unitary_group})



\paragraph{Poincar\'e Group}\label{sec:poincare_group}\hfill

10-dimensional Non-commutative Lie Group

``Symmetry'' (invariance) of the speed of light under all frames of reference



% ==============================================================================
\section{Conformal Geometry}\label{sec:conformal_geometry}
% ==============================================================================

Two dimensions: Geometry of Riemann Surfaces
(\S\ref{sec:riemann_surface})

\fist Conformal Manifold (\S\ref{sec:conformal_manifold})



% ------------------------------------------------------------------------------
\subsection{Euclidean Sphere}\label{sec:euclidean_sphere}
% ------------------------------------------------------------------------------

cf. \emph{de Sitter Space} (\S\ref{sec:desitter_space}): Minkowski Space analog
of Sphere in Euclidean Space



% ------------------------------------------------------------------------------
\subsection{Torus}\label{sec:torus}
% ------------------------------------------------------------------------------

$S^1 \times S^1$

Ring Torus

Horn Torus

Spindle Torus

\fist Toroidal Graph (\S\ref{sec:toroidal_graph})



% ------------------------------------------------------------------------------
\subsection{Stereographic Projection}\label{sec:stereographic_projection}
% ------------------------------------------------------------------------------

\begin{enumerate}
  \item Lines and Circles on the Plane map to Circles on the Sphere
  \item Being Conformal, projection preserves Angles
\end{enumerate}

Complex Plane (\S\ref{sec:complex_plane})



% ------------------------------------------------------------------------------
\subsection{Conformal Symmetry}\label{sec:conformal_symmetry}
% ------------------------------------------------------------------------------

\emph{Conformal Invariance}



\subsubsection{Spherical Wave Transformation}
\label{sec:spherical_wave_transformation}

(wiki):

Bateman \& Cunningham -- Conformal Symmetry of Maxwell's Equations; generic
expression of Conformal Symmetry is called a \emph{Spherical Wave
  Transformation}

leaves the form of Spherical Waves (\S\ref{sec:spherical_wave}) Invariant in
all Inertial Frames (\S\ref{sec:inertial_frame})



\subsubsection{Scale Invariance}\label{sec:scale_invariance}

Scaling Transformation (\S\ref{sec:scaling}), Homothety (Homogeneous Dilation
\S\ref{sec:homothety})

``Universality'', ``Self-similarity'' (cf. Holography)

\fist Homogeneous Functions (Projective Geometry
\S\ref{sec:homogeneous_function})

\fist Standardized Moments (Probability Distributions
\S\ref{sec:standardized_moment})

Power Law (Scaling Law \S\ref{sec:power_law}) Distributions, Pareto Distribution
(\S\ref{sec:pareto_distribution})

Self-Organized Criticality (SOC \S\ref{sec:soc}): Property of Dynamical Systems
(\S\ref{sec:dynamical_system}) with a Critical Point (TODO) as an Attractor
(\S\ref{sec:attractor_repeller}); ``macroscopic behavior'' displays Spatial
and/or Temporal Scale-Invariance

\fist Fractal Geometry (Part \ref{sec:fractal_geometry})

cf. Scale Analysis (Approximation Theory \S\ref{sec:scale_analysis}),
Statistical Scale Analysis (\S\ref{sec:statistical_scale})



\paragraph{Renormalization Group}\label{sec:renormalization_group}\hfill

Chaos Theory (\S\ref{sec:chaos_theory}), Statistical Mechanics



% ==============================================================================
\section{Real Analytic Geometry}\label{sec:real_analytic_geometry}
% ==============================================================================

% ==============================================================================
\section{Non-archimedean Analytic Geometry}
\label{sec:nonarchimedean_analytic_geometry}
% ==============================================================================

Analytic Geometry over Non-archimedean Fields (\S\ref{sec:nonarchimedean_field})

\fist Global Analytic Geometry (\S\ref{sec:global_analytic_geometry})



% ==============================================================================
\section{Dimensional Analysis}\label{sec:dimensional_analysis}
% ==============================================================================

%FIXME

% ------------------------------------------------------------------------------
\subsection{Nondimensionalization}\label{sec:nondimensionalization}
% ------------------------------------------------------------------------------

\subsubsection{Buckingham $\pi$ Theorem}\label{sec:buckingham_pi}
