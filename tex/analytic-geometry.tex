%%%%%%%%%%%%%%%%%%%%%%%%%%%%%%%%%%%%%%%%%%%%%%%%%%%%%%%%%%%%%%%%%%%%%%
%%%%%%%%%%%%%%%%%%%%%%%%%%%%%%%%%%%%%%%%%%%%%%%%%%%%%%%%%%%%%%%%%%%%%%
\part{Analytic Geometry}\label{sec:analytic_geometry}
%%%%%%%%%%%%%%%%%%%%%%%%%%%%%%%%%%%%%%%%%%%%%%%%%%%%%%%%%%%%%%%%%%%%%%
%%%%%%%%%%%%%%%%%%%%%%%%%%%%%%%%%%%%%%%%%%%%%%%%%%%%%%%%%%%%%%%%%%%%%%

Catenary

Tractrix

Involute:Integral::Evolute:Derivative



% ====================================================================
\section{Coordinate System}\label{sec:coordinate_system}
% ====================================================================

Coordinate Space (\S\ref{sec:coordinate_space})



% --------------------------------------------------------------------
\subsection{Cartesian Coordinate}\label{sec:cartesian_coordinate}
% --------------------------------------------------------------------

a Euclidean Space (\S\ref{sec:euclidean_space}) with Cartesian
Coordinates is Modelled by a Real Coordinate Space
(\S\ref{sec:real_coordinate_space})

cf. Homogenous Coordinates (\S\ref{sec:homogenous_coordinate}) in
Projective Geometry (\S\ref{sec:projective_geometry})



% --------------------------------------------------------------------
\subsection{Polar Coordinate}\label{sec:polar_coordinate}
% --------------------------------------------------------------------

% --------------------------------------------------------------------
\subsection{Homogenous Coordinate}\label{sec:homogenous_coordinate}
% --------------------------------------------------------------------

Projective Geometry (\S\ref{sec:projective_geometry}), Projective
Space (\S\ref{sec:projective_space}) -- Points (including Points at
Infinity) can be represented using Finite Coordinates; allows for
Affine Transformations (\S\ref{sec:affine_transformation})

cf. Cartesian Coordinates (\S\ref{sec:cartesian_coordinate}) in
Euclidean Geometry (\S\ref{sec:euclidean_geometry})



% --------------------------------------------------------------------
\subsection{Grassmann Cooridnate}\label{sec:grassmann_coordinate}
% --------------------------------------------------------------------

Pl\"ucker Embedding



\subsubsection{Pl\"ucker Cooridnate}\label{sec:plucker_coordinate}



% --------------------------------------------------------------------
\subsection{Coordinate-free}\label{sec:coordinate_free}
% --------------------------------------------------------------------

% --------------------------------------------------------------------
\subsection{Orthogonal Coordinate}\label{sec:orthogonal_coordinate}
% --------------------------------------------------------------------

% --------------------------------------------------------------------
\subsection{Curvilinear Coordinate}\label{sec:curvilinear_coordinate}
% --------------------------------------------------------------------

\subsubsection{Skew Coordinate}\label{sec:skew_coordinate}



% ====================================================================
\section{Euclidean Space}\label{sec:euclidean_space}
% ====================================================================

$\mathbf{E}^n$

Euclidean Space is a Homogenous Space (\S\ref{sec:homogenous_space}) for its
Symmetry Groups (\S\ref{sec:symmetry_groups})

Isomotries of Euclidean Space:
\begin{itemize}
  \item Point Reflection (Central Inversion) -- an Affine Transformation
  \item TODO
  ...
\end{itemize}

$E(n)$ -- Euclidean Group (\S\ref{sec:euclidean_group}): Isometry Group
(\S\ref{sec:isometry_group}) $ISO(n)$; makes Euclidean Geometry
(\S\ref{sec:euclidean_geometry}) a case of Klein Geometry
(\S\ref{sec:klein_geometry})

cf. Affine Spaces (\S\ref{sec:affine_space}): generalizes properties
of Euclidean Spaces

Reflection Groups (\S\ref{sec:reflection_group}) are Discrete Groups Generated
by a Set of Reflections (\S\ref{sec:reflection}) of a Finite-dimensional
Euclidean Space

The Coxeter Groups (\S\ref{sec:coxeter_group}) are precisely the Finite
Euclidean Reflection Groups

Crystallographic Groups (\S\ref{sec:crystallographic_group}) are Cocompact
(\S\ref{sec:cocompact_space}), Discrete Subgroups (\S\ref{sec:discrete_group})
of the Isometries (\S\ref{sec:isometry}) of some Euclidean Space



% --------------------------------------------------------------------
\subsection{Euclidean Plane}\label{sec:euclidean_plane}
% --------------------------------------------------------------------

with Cartesian Coordinates Modelled by $\reals^2$ (Cartesian Plane
\S\ref{sec:cartesian_plane})

\fist Extended Euclidean Plane (Real Projective Plane
\S\ref{sec:real_projective_plane})

Frieze Groups (\S\ref{sec:frieze_group}) and Wallpaper Groups
(\S\ref{sec:wallpaper_group}) are Discrete Subgroups
(\S\ref{sec:discrete_group}) of the Isometry Group (\S\ref{sec:isometry_group})
of the Euclidean Plane



\subsubsection{Plane Curve}\label{sec:plane_curve}

\paragraph{Smooth Plane Curve}\label{sec:smooth_plane_curve}\hfill

\paragraph{Algebraic Plane Curve}\label{sec:algebraic_plane_curve}\hfill



% --------------------------------------------------------------------
\subsection{3-space}\label{sec:3_space}
% --------------------------------------------------------------------

``Euclidean Space''

with Cartesian Coordinates Modelled by $\reals^3$

``Model'' Riemannian Manifold (\S\ref{sec:riemannian_manifold})

cf. Minkowski Space (\S\ref{sec:minkowski_space}) $\reals^{n-1,1}$
with the Flat Minkowski Metric (\S\ref{sec:minkowski_metric}) as the
``Model'' Lorentzian Manifold



\subsubsection{Implicit Surface}\label{sec:implicit_surface}

$F(x,y,z) = 0$



\paragraph{Parametric Surface}\label{sec:parametric_surface}\hfill

3-dimensional Parametric Equation (\S\ref{sec:parametric_equation})



\paragraph{Non-algebraic Surface}\label{sec:nonalgebraic_surface}\hfill

if $F(x,y,z)$ is Polynomial in $x$, $y$, $z$, then the surface is
Algebraic, otherwise it is Non-algebraic



\subsubsection{Minkowski Sum}\label{sec:minkowski_sum}

or \emph{Dilation}

\emph{Brunn-Minkowski Inequality} --
\url{https://golem.ph.utexas.edu/category/2017/01/the_brunnminkowski_inequality.html}



% --------------------------------------------------------------------
\subsection{Pseudo-euclidean Space}\label{sec:pseudo_euclidean}
% --------------------------------------------------------------------

\subsubsection{Minkowski Space}\label{sec:minkowski_space}

$\reals^{n-1,1}$

with the Flat Minkowski Metric (\S\ref{sec:minkowski_metric}) as the
``Model'' Lorentzian Manifold (\S\ref{sec:lorentzian_manifold})

cf. Euclidean Space (\S\ref{sec:euclidean_space}) $\reals^n$ as the
``Model'' Riemannian Manifold (\S\ref{sec:riemannian_manifold})



\paragraph{Minkowski Metric}\label{sec:minkowski_metric}\hfill

Flat

Curved



% ====================================================================
\section{Coordinate Space}\label{sec:coordinate_space}
% ====================================================================

a Space with a Coordinate System (\S\ref{sec:coordinate_system})



% --------------------------------------------------------------------
\subsection{Real Coordinate Space}\label{sec:real_coordinate_space}
% --------------------------------------------------------------------

\emph{Real Coordinate Space} or \emph{Real $n$-space}

$\reals^n$

Models Euclidean Space (\S\ref{sec:euclidean_space}) with Cartesian
Coordinates (\S\ref{sec:cartesian_coordinates})

$\reals^0$ -- Singleton Empty Column Vector-- Zero (Initial) Vector
Space

$\reals^1$ -- Real Line (\S\ref{sec:real_line})

$\reals^2$ -- Cartesian Plane (\S\ref{sec:cartesian_plane})

$\reals^3$



% --------------------------------------------------------------------
\subsection{Convex Set}\label{sec:convex_set}
% --------------------------------------------------------------------

%FIXME

% --------------------------------------------------------------------
\subsection{Real Line}\label{sec:real_line}
% --------------------------------------------------------------------

$\reals^1$

\fist Real Analysis (\S\ref{sec:real_analysis})



\subsubsection{Extended Real Line}\label{sec:extended_real_line}

$\overline{\reals}$ or $[-\infty, +\infty]$ or $\reals \cup \{
-\infty, +\infty\}$

%FIXME move section?



% --------------------------------------------------------------------
\subsection{Cartesian Plane}\label{sec:cartesian_plane}
% --------------------------------------------------------------------

$\reals^2$



% ====================================================================
\section{$n$-sphere}\label{sec:n_sphere}
% ====================================================================

$S^n = \{ x \in \reals^{n+1} : \|x\| = r \}$

$S^n = \reals \cup \{\infty\}$

familiy of Manifolds



\subsection{Unit Circle}\label{sec:unit_circle}

\subsection{Unit Sphere}\label{sec:unit_sphere}

Unit Ball (\S\ref{sec:unit_ball})



% ====================================================================
\section{$n$-torus}\label{sec:n_torus}
% ====================================================================

family of Manifolds



% ====================================================================
\section{Projective Space}\label{sec:projective_space}
% ====================================================================

Homogenous Coordinates (\S\ref{sec:homogenous_coordinate})

Projective Space is a Homogenous Space (\S\ref{sec:homogenous_space}) for its
Symmetry Groups (\S\ref{sec:symmetry_groups})



% --------------------------------------------------------------------
\subsection{Projective Transformation}
\label{sec:projective_transformation}
% --------------------------------------------------------------------

a \emph{Projective Transformation} (or \emph{Homography}) is an
Isomorphism of Projective Spaces induced by an Isomorphism of the
Vector Spaces from which the Projective Spaces derive

Collineation



% --------------------------------------------------------------------
\subsection{Projective Plane}\label{sec:projective_plane}
% --------------------------------------------------------------------

includes Euclidean, Elliptic, and Hyperbolic Planes

Symmetry Group contains Symmetry Groups for Euclidean, Elliptic, and
Hyperbolic Planes



% --------------------------------------------------------------------
\subsection{Fano Plane}\label{sec:fano_plane}
% --------------------------------------------------------------------

% --------------------------------------------------------------------
\subsection{Grassmanian}\label{sec:grassmanian}
% --------------------------------------------------------------------

\subsubsection{Real Projective Space}\label{sec:real_projective_space}

Compact (\S\ref{sec:compact_space}) Smooth Manifold
(\S\ref{sec:smooth_manifold})



\paragraph{Real Projective Plane}\label{sec:real_projective_plane}\hfill

(or \emph{Extended Euclidean Plane})



\subsubsection{Complex Projective Space}
\label{sec:complex_projective_space}



% ====================================================================
\section{Affine Space}\label{sec:affine_space}
% ====================================================================

%FIXME move section?

generalizes properties of Euclidean Spaces (\S\ref{sec:euclidean_space})

Affine Space is a Homogenous Space (\S\ref{sec:homogenous_space}) for its
Symmetry Groups (\S\ref{sec:symmetry_groups})



% --------------------------------------------------------------------
\subsection{Affine Transformation}\label{sec:affine_transformation}
% --------------------------------------------------------------------

(or \emph{Affine Map} or \emph{Affinity}) is a Function between Affine
Spaces which preserves Points, Straight Lines, and Planes, and Sets of
Parallel Lines remain Parallel

preserves Ratios of Distances betwen Points lying on a Straight Line

does not necessarily preserve Angles between Lines or Distances
between Points

Affine Transformations:
\begin{itemize}
\item Translation
\item Scaling
\item Homothety (Homogenous Dilation or Central Similarity)
\item Similarity Transformation (???)
\item Reflection
\item Rotation
\item Shear Mapping (???)
\end{itemize}
and combinations of the above in any combination and sequence

cf. Affine Logic (\S\ref{sec:affine_logic})



% ====================================================================
\section{Dimensional Analysis}\label{sec:dimensional_analysis}
% ====================================================================

% --------------------------------------------------------------------
\subsection{Subspace}\label{sec:subspace}
% --------------------------------------------------------------------

%FIXME move this section? vector spaces? topological spaces?



\subsubsection{Convex Subspace}\label{sec:convex_subspace}

\subsubsection{Hyperplane}\label{sec:hyperplane}

Subspace one Dimension less than Ambient Space

Linear, Affine, or Projective Space %FIXME xref

Left-regular Band Monoid (\S\ref{sec:graphic_monoid})



% --------------------------------------------------------------------
\subsection{Ambient Space}\label{sec:ambient_space}
% --------------------------------------------------------------------

%FIXME move this section?



% ====================================================================
\section{Algebraic Curve}\label{sec:algebraic_curve}
% ====================================================================

% --------------------------------------------------------------------
\subsection{Conic}\label{sec:conic}
% --------------------------------------------------------------------



% ====================================================================
\section{Conformal Geometry}\label{sec:conformal_geometry}
% ====================================================================

Two dimensions: Geometry of Riemann Surfaces
(\S\ref{sec:riemann_surface})



% --------------------------------------------------------------------
\subsection{Euclidean Sphere}\label{sec:euclidean_sphere}
% --------------------------------------------------------------------

% --------------------------------------------------------------------
\subsection{Torus}\label{sec:torus}
% --------------------------------------------------------------------

$S^1 \times S^1$

Ring Torus

Horn Torus

Spindle Torus



% --------------------------------------------------------------------
\subsection{Stereographic Projection}\label{sec:stereographic_projection}
% --------------------------------------------------------------------

\begin{enumerate}
  \item Lines and Circles on the Plane map to Circles on the Sphere
  \item Being Conformal, projection preserves Angles
\end{enumerate}

Complex Plane (\S\ref{sec:complex_plane})



% --------------------------------------------------------------------
\subsection{Inversive Geometry}\label{sec:inversive_geometry}
% --------------------------------------------------------------------



% ====================================================================
\section{Real Analytic Geometry}\label{sec:real_analytic_geometry}
% ====================================================================
