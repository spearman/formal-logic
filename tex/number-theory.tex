%%%%%%%%%%%%%%%%%%%%%%%%%%%%%%%%%%%%%%%%%%%%%%%%%%%%%%%%%%%%%%%%%%%%%%
%%%%%%%%%%%%%%%%%%%%%%%%%%%%%%%%%%%%%%%%%%%%%%%%%%%%%%%%%%%%%%%%%%%%%%
\part{Number Theory}\label{sec:number_theory}
%%%%%%%%%%%%%%%%%%%%%%%%%%%%%%%%%%%%%%%%%%%%%%%%%%%%%%%%%%%%%%%%%%%%%%
%%%%%%%%%%%%%%%%%%%%%%%%%%%%%%%%%%%%%%%%%%%%%%%%%%%%%%%%%%%%%%%%%%%%%%

% ====================================================================
\section{Number}\label{sec:number}
% ====================================================================

% --------------------------------------------------------------------
\subsection{Natural Number}\label{sec:natural_number}
% --------------------------------------------------------------------

$\mathbb{N} = \{ 1,2,3,\ldots \}$

Well-ordering (\S\ref{sec:well_order})



\subsubsection{Principle of Mathematical Induction}
\label{sec:induction_principle}



% --------------------------------------------------------------------
\subsection{Integer}\label{sec:integer}
% --------------------------------------------------------------------

$\mathbb{Z} = \{ \ldots, -2, -1, 0, 1, 2, \ldots \}$



\subsubsection{Divisibility}\label{sec:divisibility}

$(a,b,q \in \mathbb{Z}), b|a \Leftrightarrow a = qb$

Properties:

\begin{itemize}
\item $\forall a \in \mathbb{Z}, 1|a$
\item $\forall a \in \mathbb{Z}, a|0$
\item Linear Combinations (\S\ref{sec:linear_combination}): $b|a_1
  \wedge b|a_2 \Rightarrow (\forall x,y \in \mathbb{Z})b|a_1 x + a_2
  y$
\end{itemize}



\paragraph{Greatest Common Divisor}\label{sec:gcd}
\hfill \\

Greatest Common Divisor (GCD)

$gcd(0,n) = n$

\emph{Linear Combination Theorem}: The GCD of two non-zero Integers
$a$ and $b$ is equal to the smallest possible Positive Linear
Combination (\S\ref{sec:linear_combination}) of $a$ and $b$.

Corollary 1: The Set of Linear Combinations of $a$ and $b$ is equal to
the Set of Multiples of $gcd(a,b)$.

Corollary 2: If $1$ is the smallest Positive Linear Combination of $a$
and $b$ then $gcd(a,b) = 1$ and $a$ and $b$ are Coprime
(\S\ref{sec:coprime}).

Corollary 3: If $a|c$ and $b|c$ and $gcd(a,b) = 1$, then $ab|c$.

Corollary 4: (Euclid's Lemma) If $a | bc$ and $gcd(a,b) = 1$ then
$a|c$.

Corollary 5: $gcd(\frac{a}{d}, \frac{b}{d}) = 1$



\subparagraph{Euclid's Algorithm}\label{sec:euclids_algorithm}



\subsubsection{Prime Number}\label{sec:prime_number}

$\mathbb{P}$

If $p \in \mathbb{P}$, then $gcd(p,a) = 1$ or $gcd(p,a) = p$ for any
$a \in \mathbb{Z}$.

If $p \nmid b$ and $p \in \mathbb{P}$ then $gcd (p,b) = 1$.

Corollary to Euclid's Lemma: If $p \in \mathbb{P}$ and $p|a_1 a_2
\ldots a_n$ then $p|a_i$ for some $1 \leq i \leq n$.



\paragraph{Coprime}\label{sec:coprime}
\hfill \\

Two Integers are \emph{Coprime} if $gcd (a,b) = 1$.



\paragraph{Prime Decomposition}\label{sec:prime_decomposition}
\hfill \\

Standard Prime Factorization: $n = p_1^{k_1} p_2^{k_2} \ldots
p_n^{k_n}$.

\emph{Fundamental Theorem of Arithmetic}: Any Integer $\geq 2$ can be
written as a Unique Product of Prime Numbers (Standard Prime
Factorization).

The Uniqueness of the result of the Fundamental Theorem does not hold
in Quotient Sets over the Integers.



\paragraph{Prime Power}\label{sec:prime_power}
\hfill \\

A \emph{Prime Power} is a Prime Number raised to the power of a
positive Integer.



\subsubsection{Diophantine Equation}\label{sec:diophantine_equation}

\paragraph{Linear Diophantine Equation}
\label{sec:linear_diophantine}
\hfill \\
Monomial (\S\ref{sec:monomial}) Equations with Integer Solutions

$c = ax + by$ has a solution in $\mathbb{Z}$ when $gcd(a,b)|c$. Given
$d=gcd(a,b)|c$ and a solution $(x_0, y_0)$, then all solutions are of
the form $(x_0 + \frac{b}{d}t, y_0 - \frac{a}{d}t)$ with $t \in
\mathbb{Z}$.



\paragraph{Exponential Diophantine Equation}
\label{sec:exponential_diophantine}
\hfill \\



% --------------------------------------------------------------------
\subsection{Rational Number}\label{sec:rational}
% --------------------------------------------------------------------

$\mathbb{Q}$

A \emph{Rational Number} $q \in \mathbb{Q}$ is a ratio of Integers
$m,n \in \mathbb{Z}$:
\[
  q = \frac{n}{m}
\]

\textbf{Lowest Form}: If $m$ and $n$ are not Coprime with $gcd (m,n) =
d$, an equivalent Rational Number can be written where the Numerator
and Denominator are Coprime by:
\[
  q = \frac{n}{m} = \frac{\frac{n}{d}}{\frac{m}{d}}
\]



% --------------------------------------------------------------------
\subsection{Real Number}\label{sec:real_number}
% --------------------------------------------------------------------

Three definitions:
\begin{enumerate}
  \item \emph{Axiomatically}: an Ordered Field
    (\S\ref{sec:ordered_field}) with the Least Upperbound Property
  \item \emph{Dedekind Cuts}: Partitions of the Rational Numbers into
    Upper and Lower Sets
  \item \emph{Cauchy Sequences}: a Sequence of Elements that given any
    ``small'' Positive Distance, all but a Finite number of Elements
    of the Sequence are less than that given Distance from eachother
\end{enumerate}



\subsubsection{Irrational Number}\label{sec:irrational}

$\mathbb{R}/\mathbb{Q}$

The Square Root of any Non-square Natural Number is Irrational.



\subsubsection{Algebraic Number}\label{sec:algebraic_number}

\subsubsection{Transcendental Number}\label{sec:transcendental}

\subsubsection{Computable Number}\label{sec:computable_real}

also \emph{Recursive Numbers} or \emph{Computable Reals}

\emph{Real Closed Field} (\S\ref{sec:closed_field})



% --------------------------------------------------------------------
\subsection{Complex Number}\label{sec:complex_number}
% --------------------------------------------------------------------

\subsubsection{Gaussian Integer}\label{sec:gaussian_integer}



% --------------------------------------------------------------------
\subsection{Cardinal Number}\label{sec:cardinal_number}
% --------------------------------------------------------------------

\subsubsection{Limit Cardinal}\label{sec:limit_cardinal}

\subsubsection{Inaccessible Cardinal}\label{sec:inaccessible_cardinal}

\subsubsection{Measurable Cardinal}\label{sec:measurable_cardinal}

\subsubsection{Regular Cardinal}\label{sec:regular_cardinal}

A \emph{Regular Cardinal} is a Cardinal Number that is equal to its
own Cofinality (\S\ref{sec:cofinality}).



% --------------------------------------------------------------------
\subsection{Ordinal Number}\label{sec:ordinal_number}
% --------------------------------------------------------------------

\subsubsection{Limit Ordinal}\label{sec:limit_ordinal}

\subsubsection{Regular Ordinal}\label{sec:regular_ordinal}

\subsubsection{Admissible Number}\label{sec:admissible_ordinal}

\subsubsection{Von Neumann Ordinal}\label{sec:vonneumann_ordinal}

\subsubsection{Proof-theoretic Ordinal}\label{sec:proof_ordinal}



% --------------------------------------------------------------------
\subsection{Transfinite Number}\label{sec:transfinite_number}
% --------------------------------------------------------------------

% --------------------------------------------------------------------
\subsection{Infinitesimal}\label{sec:infinitesimal}
% --------------------------------------------------------------------

% --------------------------------------------------------------------
\subsection{Hyperreal}\label{sec:hyperreal}
% --------------------------------------------------------------------

% --------------------------------------------------------------------
\subsection{Surreal Number}\label{sec:surreal_number}
% --------------------------------------------------------------------

% --------------------------------------------------------------------
\subsection{$p$-adic Number}\label{sec:padic_number}
% --------------------------------------------------------------------



% ====================================================================
\section{Arithmetic}\label{sec:arithmetic}
% ====================================================================

% --------------------------------------------------------------------
\subsection{Skolem Arithmetic}\label{sec:skolem_arithmetic}\cite{skolem23}
% --------------------------------------------------------------------

% --------------------------------------------------------------------
\subsection{First-order Arithmetic}\label{sec:firstorder_arithmetic}
% --------------------------------------------------------------------

First-order Logic (\S\ref{sec:firstorder_logic})



\subsubsection{Peano Arithmetic}\label{sec:peano_arithmetic}

\subsubsection{Presburger Arithmetic}\label{sec:presburger_arithmetic}

Decidable, weaker than Peano Arithmetic



% --------------------------------------------------------------------
\subsection{Second-order Arithmetic}\label{sec:second_order_arithmetic}
% --------------------------------------------------------------------

% --------------------------------------------------------------------
\subsection{Modular Arithmetic}\label{sec:modular_arithmetic}
% --------------------------------------------------------------------

\subsubsection{Modular Congruence}\label{sec:modular_congruence}

$a \equiv b (\mathrm{mod} n)$



% --------------------------------------------------------------------
\subsection{Heyting Arithmetic}\label{sec:heyting_arithmetic}
% --------------------------------------------------------------------

Peano Arithmetic with Intuitionistic Logic



% ====================================================================
\section{Analytic Number Theory}\label{sec:analytic_number_theory}
% ====================================================================

% --------------------------------------------------------------------
\subsection{Dirichlet L-Series}\label{sec:l_series}
% --------------------------------------------------------------------



% ====================================================================
\section{Algebraic Number Theory}\label{sec:algebraic_number_theory}
% ====================================================================

% ====================================================================
\section{Additive Number Theory}\label{sec:additive_number_theory}
% ====================================================================

% ====================================================================
\section{Computationsl Number Theory}\label{sec:computational_number_theory}
% ====================================================================

% --------------------------------------------------------------------
\subsection{Integer Factorization}\label{sec:integer_factorization}
% --------------------------------------------------------------------



% ====================================================================
\section{Probabilistic Number Theory}\label{sec:probabilistic_number_theory}
% ====================================================================

% ====================================================================
\section{Sieve Theory}\label{sec:sieve_theory}
% ====================================================================

% ====================================================================
\section{Finitism}\label{sec:finitism}
% ====================================================================

% --------------------------------------------------------------------
\subsection{Strict Finitism}\label{sec:strict_finitism}
% --------------------------------------------------------------------

% --------------------------------------------------------------------
\subsection{Ultrafinitism}\label{sec:ultrafinitism}
% --------------------------------------------------------------------
