%%%%%%%%%%%%%%%%%%%%%%%%%%%%%%%%%%%%%%%%%%%%%%%%%%%%%%%%%%%%%%%%%%%%%%
%%%%%%%%%%%%%%%%%%%%%%%%%%%%%%%%%%%%%%%%%%%%%%%%%%%%%%%%%%%%%%%%%%%%%%
\part{Number Theory}\label{sec:number_theory}
%%%%%%%%%%%%%%%%%%%%%%%%%%%%%%%%%%%%%%%%%%%%%%%%%%%%%%%%%%%%%%%%%%%%%%
%%%%%%%%%%%%%%%%%%%%%%%%%%%%%%%%%%%%%%%%%%%%%%%%%%%%%%%%%%%%%%%%%%%%%%

%FiXME

Continuous: Numerical Value

Discrete: Prime Factorization



% ====================================================================
\section{Number}\label{sec:number}
% ====================================================================

% --------------------------------------------------------------------
\subsection{Natural Number}\label{sec:natural_number}
% --------------------------------------------------------------------

$\mathbb{N} = \{ 1,2,3,\ldots \}$

Well-ordering Principle (\S\ref{sec:wellordering_principle})

Principle of Mathematical Induction \S\ref{sec:induction_principle})
(Minimalisation Axiom)



\subsubsection{Peano Axioms}\label{sec:peano_axioms}

(or \emph{Peano Postulates})



\subsubsection{Principle of Mathematical Induction}
\label{sec:induction_principle}

Mathematical Induction (\S\ref{sec:mathematical_induction}) on the
Natural Numbers



\subsubsection{Well-ordering Principle}\label{sec:wellordering_principle}

Well-ordering (\S\ref{sec:well_order})



\subsubsection{Pairing Function}\label{sec:pairing_function}

A \emph{Pairing Function} is a Primitive Recursive
(\S\ref{sec:primitive_recursive}) Bijection:
\[
  \pi : \nats \times \nats \rightarrow \nats
\]



\paragraph{Cantor Pairing Function}\label{sec:cantor_pairing}\hfill



\subsubsection{Fibonacci Sequence}\label{sec:fibonacci_sequence}

\subsubsection{Extended Natural Number}\label{sec:extended_natural}



% --------------------------------------------------------------------
\subsection{Integer}\label{sec:integer}
% --------------------------------------------------------------------

$\ints = \{ \ldots, -2, -1, 0, 1, 2, \ldots \}$

\fist Diophantine Equations (\S\ref{sec:diophantine_equation})

\fist a Free Abelian Group (\S\ref{sec:free_abelian_group}) is exactly a Free
Module over the Ring $\ints$ of Integers

\fist Integer Lattice (\S\ref{sec:integer_lattice})

the Principal Ideals (\S\ref{sec:principal_ideal}) of the Ring of Integers
correspond one-for-one with Non-negative Integers

the Prime Ideals (\S\ref{sec:prime_ideal}) of the Ring of Integers are the Sets
containing all the Multiples of a given Prime Number, together with the Zero
Ideal $\{0\}$

Ring (\S\ref{sec:ring}) of Integers

(from \#math IRC user unyu): by ``Initiality'' there is excactly one Ring
Homomorphism $f : \ints \rightarrow R$ for any Ring $R$



\subsubsection{Parity}\label{sec:parity}

Even, Odd

\fist Even Functions (\S\ref{sec:even_function}), Odd Functions
(\S\ref{sec:odd_function})



\subsubsection{Divisibility}\label{sec:divisibility}

$(a,b,q \in \mathbb{Z}), b|a \Leftrightarrow a = qb$

Properties:

\begin{itemize}
\item $\forall a \in \mathbb{Z}, 1|a$
\item $\forall a \in \mathbb{Z}, a|0$
\item Linear Combinations (\S\ref{sec:linear_combination}): $b|a_1
  \wedge b|a_2 \Rightarrow (\forall x,y \in \mathbb{Z})b|a_1 x + a_2
  y$
\end{itemize}



\paragraph{Greatest Common Divisor}\label{sec:gcd}\hfill

Greatest Common Divisor (GCD)

$gcd(0,n) = n$

\emph{Linear Combination Theorem}: The GCD of two non-zero Integers
$a$ and $b$ is equal to the smallest possible Positive Linear
Combination (\S\ref{sec:linear_combination}) of $a$ and $b$.

Corollary 1: The Set of Linear Combinations of $a$ and $b$ is equal to
the Set of Multiples of $gcd(a,b)$.

Corollary 2: If $1$ is the smallest Positive Linear Combination of $a$
and $b$ then $gcd(a,b) = 1$ and $a$ and $b$ are Coprime
(\S\ref{sec:coprime}).

Corollary 3: If $a|c$ and $b|c$ and $gcd(a,b) = 1$, then $ab|c$.

Corollary 4: (Euclid's Lemma) If $a | bc$ and $gcd(a,b) = 1$ then
$a|c$.

Corollary 5: $gcd(\frac{a}{d}, \frac{b}{d}) = 1$



\subparagraph{Euclid's Algorithm}\label{sec:euclids_algorithm}\hfill



\subsubsection{Prime Number}\label{sec:prime_number}

$\primes$

A Number that is \emph{not} Prime is called \emph{Composite}.

The Cardinality of $\primes$ is Countably Infinite.

The Series (\S\ref{sec:series}) $\sum_{p \in \primes} \frac{1}{p}$
Diverges The corresponding Series over Twin Primes
(\S\ref{sec:twin_prime}) Converges.

If $p \in \primes$, then $gcd(p,a) = 1$ or $gcd(p,a) = p$ for any
$a \in \mathbb{Z}$.

If $p \nmid b$ and $p \in \primes$ then $gcd (p,b) = 1$.

Corollary to Euclid's Lemma: If $p \in \primes$ and $p|a_1 a_2
\ldots a_n$ then $p|a_i$ for some $1 \leq i \leq n$.

Given any $l \in \mathbb{N}$, a Sequence of $l$ consecutive Composite
Numbers may be found.

\emph{Bertrand's Postulate} (\emph{Bertrand-Chebyshev Theorem}):
\[
  (\forall n \in \mathbb{Z}^{>3}, \exists p \in \primes) n < p < 2n
\]

Given an Arithmetic Sequence $S = \{ a + bx \;|\; x \in \mathbb{Z}^+\}$:
\begin{enumerate}
  \item If $gcd(a,b) > 1$ then there are no Primes in $S$.
  \item If $gcd(a,b) = 1$ then there are Infinitely many Primes in $S$
    (Dirichlet)
  \item For any $l \in \mathbb{N}$, there exists a consecutive
    Sequence of Primes of Length $l$ in $S$ (Green \& Tao)
\end{enumerate}

Sieve Theory (\S\ref{sec:sieve_theory})

For all $p \in \primes$, $gcd(p, (p-1)!) = 1$

If $n \notin \primes$ and $n > 4$ then $gcd(n,(n-1)!) = n$

\emph{Wilson's Theorem}:
\[
  (p - 1)! = -1 \mathrm{mod}\;p
\]
for $p \in \primes$.

$x^2 \equiv -1 \mathrm{mod}\;p$ if and only if $p \equiv 1 mod 4$

For Composite $k$, $2^k-1$ is Composite

For $k$ such that $k$ is not a Power of $2$, then $2^k+1$ is Composite



\paragraph{Coprime}\label{sec:coprime}\hfill

Two Integers are \emph{Coprime} if $gcd (a,b) = 1$.



\subparagraph{ABC Conjecture}\label{sec:abc_conjecture}\hfill

for every $\varepsilon > 0$, there exist only Finitely many triples $(a,b,c)$ of
Coprime Positive Integers with $a + b = c$ such that
$c > \mathrm{rad}(abc)^{1-\varepsilon}$



\paragraph{Prime Factor}\label{sec:prime_factor}\hfill

Prime Decomposition (\S\ref{sec:prime_decomposition})

Multiplicity



\paragraph{Semiprime}\label{sec:semiprime}\hfill

a Natural Number that is the Product of two (not necessarily distinct) Prime
Numbers

by the Fundamental Theorem of Arithmetic
(\S\ref{sec:fundamental_arithmetic_theorem}) the Product of two Primes is
always a Semi-prime

\fist Arithmetic Cryptography (\S\ref{sec:arithmetic_cryptography}): it is
difficult (computationally) to Factorize a Natural Number into a Product of two
large Prime Numbers



\paragraph{Prime Counting Function}\label{sec:prime_counting}\hfill

\paragraph{Prime Divisors Function}\label{sec:prime_divisors}\hfill

$\omega(n)$ Number of unique Prime Factors of $n$: Additive

$\Omega(n)$ Number of Prime Factors of $n$: Completely Additive

$p \in \primes$, $\omega(p^j) = 1$



\paragraph{Prime Power}\label{sec:prime_power}\hfill

A \emph{Prime Power} is a Prime Number raised to the power of a
positive Integer.

The number of Elements (Order) $q$ of a Finite Field (\S\ref{sec:finite_field})
is always a Prime Power $p^k$ for some Positive Integer $k$, and all Finite
Fields of a given Order are Isomorphic.



\paragraph{Fundamental Theorem of Arithmetic}
\label{sec:fundamental_arithmetic_theorem}\hfill

or \emph{Unique Prime-factorization Theorem}:

``\emph{Every Integer greater than $1$ is either a Prime Number or can be
  uniquely represented as the Product of Prime Numbers.}''

by this Theorem, the Product of any two Prime Numbers is Semi-prime
(\S\ref{sec:semiprime})



\paragraph{Twin Prime}\label{sec:twin_prime}\hfill

\paragraph{Fermat Prime}\label{sec:fermat_prime}\hfill

$2^{2^i} + 1$



\paragraph{Mersenne Prime}\label{sec:mersenne_prime}\hfill

\paragraph{Goldbach Conjecture}\label{sec:goldbach_conjecture}\hfill

Partial result:
\[
  (\forall n \in 2\mathbb{N}, \exists p,q \vee p,q,r \in \primes)
  n = p + q \vee n = p + q * r
\]



\paragraph{Prime Number Theorem}\label{sec:prime_number_theorem}\hfill



\subsubsection{Perfect Number}\label{sec:perfect_number}

$n$ is Perfect if the Sigma Function (\S\ref{sec:sigma_function})
$\sigma(n) = 2n$

Mersenne Prime (\S\ref{sec:mersenne_prime})



\subsubsection{Partition}\label{sec:integer_partition}

cf. Partition of a Set (\S\ref{sec:partition})



\subsubsection{Composite Number}\label{sec:composite_number}

\paragraph{Kn\"odel Number}\label{sec:knodel_number}\hfill

For Positive Integer $n$, a \emph{Kn\"odel Number} $m$ is a Composite
number such that every $i < m$ Coprime to $m$ satisfies $i^{m-n}
\equiv 1 \mod m$

Set of all Kn\"odel Numbers for $n$, denoted $K_n$

Every Composite Number is a Kn\"odel Number by setting $n$ to $m -
\varphi(m)$



\subparagraph{Carmichael Number}\label{sec:carmichael_number}\hfill

Composite Number $n$ satisfying $b^{n-1} \equiv 1 \mod n$ for all
Integers $1 < b < n$ Relatively Prime to $n$

Subset $K_1$ of Kn\"odel Numbers



\subsubsection{Integer Decomposition}\label{sec:integer_decomposition}

\paragraph{Prime Decomposition}\label{sec:prime_decomposition}\hfill

Standard Prime Factorization (\S\ref{sec:prime_factor}): $n =
p_1^{k_1} p_2^{k_2} \ldots p_n^{k_n}$.

\emph{Fundamental Theorem of Arithmetic}: Any Integer $\geq 2$ can be
written as a Unique Product of Prime Numbers (Standard Prime
Factorization).

The Uniqueness of the result of the Fundamental Theorem does not hold
in Quotient Sets over the Integers.



\subsubsection{Integer Coefficient Polynomial}
\label{sec:integer_coefficient}

$\mathbb{Z}[x]$

Polynomial (\S\ref{sec:polynomial})

For any $p(x) \in \mathbb{Z}[x]$ and $a \equiv b \mod n$, $p(a) \equiv
p(b) \mod n$.

If $p$ is Prime, then a Polynomial of Degree $n$ in
$\mathbb{Z}/p\mathbb{Z}[x]$ has at most $n$ Zeroes in
$\mathbb{Z}/p\mathbb{Z}$.

For Prime $p$, if $d|p-1$ then $x^d - 1$ has exactly $d$ solutions in
$\mathbb{Z}/p\mathbb{Z}$.



\subsubsection{Pythagorean Triple}\label{sec:pythagorean_triple}

\paragraph{Primitive Pythagorean Triple}\label{sec:primitive_pythagorean}\hfill

$(x,y,z)$ is a Primitive Pythagorean Triple if and only if $\exists
s,t \in \mathbb{N}$ such that:
\begin{enumerate}
  \item $s > t$
  \item $gcd (s,t) = 1$
  \item $s \not\equiv t \mod 2$
\end{enumerate}



\subsubsection{Factorial}\label{sec:factorial}

Stirling's Approximation



% --------------------------------------------------------------------
\subsection{Rational Number}\label{sec:rational}
% --------------------------------------------------------------------

$\mathbb{Q}$

A \emph{Rational Number} $q \in \mathbb{Q}$ is a ratio of Integers
$m,n \in \mathbb{Z}$:
\[
  q = \frac{n}{m}
\]

multiple Representations

\textbf{Lowest Form}: If $m$ and $n$ are not Coprime with $gcd (m,n) =
d$, an equivalent Rational Number can be written where the Numerator
and Denominator are Coprime by:
\[
  q = \frac{n}{m} = \frac{\frac{n}{d}}{\frac{m}{d}}
\]

Rational Numbers are Dense (\S\ref{sec:density}) in Real Numbers

Field of Rational Numbers \fist Arithmetic (Algebraic) Geometry
(\S\ref{sec:arithmetic_geometry}): study of Zero Sets (\S\ref{sec:zero_set}) of
Systems of Polynomial Equations (\S\ref{sec:polynomial_equation_system}) over
the Rationals

\emph{Artin-Schreier Theorem} (\S\ref{sec:artin_schreier}):
an Ordered Field (\S\ref{sec:ordered_field}) $F$ has an Algebraic Extension
(\S\ref{sec:algebraic_extension}) called the \emph{Real Closure} $K$ of $F$
such that $K$ is a Real Closed Field with Ordering as an Extension of the given
Ordering on $F$ and is Unique up to Unique Isomorphism of Fields identical to
$F$

the Real Closure of the Ordered Field of Rational Numbers is the Field
$\algs$ of Real Algebraic Numbers (\S\ref{sec:algebraic_number})

\fist Diophantine Geometry (\S\ref{sec:diophantine_geometry}): Rational Points
(\S\ref{sec:rational_point})



\subsubsection{Bernoulli Number}\label{sec:bernoulli_number}

Sequences of Rationals

%FIXME move this section?



% --------------------------------------------------------------------
\subsection{Real Number}\label{sec:real_number}
% --------------------------------------------------------------------

one of three possible Finite-dimensional Associative Division Algebras
(\S\ref{sec:associative_division_algebra}) over the Real Numbers \fist
Frobenius Theorem (\S\ref{sec:frobenius_theorem})

Convergent Infinite Series (\S\ref{sec:convergent_series}) $\sum_{i =
  1}^{\infty} \frac{d_i}{10^i}$

Three definitions:
\begin{enumerate}
  \item \emph{Axiomatically}: the Unique-up-to-Isomorphism Ordered Field
    (\S\ref{sec:ordered_field}) with the Least Upperbound Property
    (Dedekind-completeness \S\ref{sec:least_upperbound})
  \item \emph{Dedekind Cuts}: Partitions of the Rational Numbers into
    Upper and Lower Sets
  \item \emph{Cauchy Sequences} (Constructive): a Sequence of Elements that
    given any ``small'' Positive Distance, all but a Finite number of Elements
    of the Sequence are less than that given Distance from eachother
\end{enumerate}

Real Line (\S\ref{sec:real_line})

Completeness Axioms (Maximalisation Axioms)

Properties:

Triangle Inequality (\S\ref{sec:triangle_inequality}):\\
$|x + y| \leq |x| + |y|$

Reverse Triangle Inequality: $|x - y| \geq ||x| - |y||$

\emph{Archimedean Property} (\S\ref{sec:archimedean_property}): the Reals are
an Archimedean Field (\S\ref{sec:archimedean_field}), i.e. for any Real Number,
there is an Integer larger than it in Absolute Value

Density (\S\ref{sec:density}): Real Numbers with usual Topology have the
Rational Numbers (\S\ref{sec:rational}) as a Countable Dense Subset and the
Irrational Numbers (\S\ref{sec:irrational}) as an Uncountably Dense Subset

$\reals$ and $\comps$ are the only Connected Locally Compact
Topological Fields (\S\ref{sec:topological_field})

Tarski-Seidenberg Theorem (\S\ref{sec:tarski_seidenberg}): the First-order
Theory of the Real Field is Decidable; Euclidean Geometry without the ability
to measure Angles is also a Model of the Real Field Axioms and therefore also
Decidable

adding $\sin$ or Exponential Functions can change the Decidability of the Theory

\emph{Cantor-Dedekind Axiom} -- thesis that the Real Numbers
(\S\ref{sec:real_number}) are Order-isomorphic to the Linear Continuum (Order
Theory \S\ref{sec:linear_continuum}) in Geometry; a consequence is that the
Decidability of the Ordered Real Field can be seen as an Algorithm to solve any
problem in Euclidean Geometry (\S\ref{sec:euclidean_geometry})



\subsubsection{Fundamental Property of Reals}
\label{sec:fundamental_property}

For $A \neq \varnothing$ and $A \subset \mathbb{R}$:
\begin{itemize}
  \item if $A$ has an Upper Bound (\S\ref{sec:upper_bound}), then it
    has a Least Upper Bound (\S\ref{sec:least_upperbound}).
  \item if $A$ has a Lower Bound (\S\ref{sec:lower_bound}), then it
    has a Greatest Lower Bound (\S\ref{sec:greatest_lowerbound}).
\end{itemize}

Implies that there is exactly one Positive Solution for $\sqrt{x}, x
\in \mathbb{R}$



\subsubsection{Perfect Square}\label{sec:perfect_square}

Square of two Natural Numbers

All other Square Roots are Irrational (\S\ref{sec:irrational})



\subsubsection{Irrational Number}\label{sec:irrational}

$\mathbb{R}/\mathbb{Q}$

The Square Root of any Non-square Natural Number is Irrational.



\subsubsection{Computable Number}\label{sec:computable_real}

also \emph{Recursive Numbers} or \emph{Computable Reals}

\emph{Real Closed Field} (\S\ref{sec:closed_field})



\subsubsection{Dual Number}\label{sec:dual_number}

extension of Real Numbers adjoining new Element $\varepsilon$ called the
\emph{Dual Unit} with the Property:
\[
  \varepsilon^2 = 0
\]
i.e. $\varepsilon$ is \emph{Nilpotent} (\S\ref{sec:nilpotent})

(wiki):

every Dual Number has the form $z = a + b\varepsilon$ where $a$ and $b$ are
uniquely determined Real Numbers

the collection of Dual Numbers forms a Two-dimensional Commutative
(\S\ref{sec:commutative_algebra}) Unital Associative Algebra
(\S\ref{sec:associative_algebra}) over the Real Numbers

the Dual Numbers can also be thought of as the Exterior Algebra
(\S\ref{sec:exterior_algebra}) of a One-dimensional Vector Space; generalized to
$n$ Dimensions gives the Grassmann Numbers (\S\ref{sec:grassmann_number})

the Algebra of Dual Numbers is a Ring that is a Local Ring
(\S\ref{sec:local_ring}) since the Principal Ideal generated by $\varepsilon$ is
its only Maximal Ideal

Dual Numbers form the Coefficients of the Dual Quaternions
(\S\ref{sec:dual_quaternion})

using Matrices:
\[
  \varepsilon = \begin{bmatrix}
    0 & 1 \\
    0 & 0 \\
  \end{bmatrix}
  \;\;\;
  a + b\varepsilon = \begin{bmatrix}
    a & b \\
    0 & a \\
  \end{bmatrix}
\]

Forward-mode Automatic Differentiation is accomplished by augmenting the Algebra
of Real Numbers by adding a component to every number which will represent the
\emph{Derivative} (\S\ref{sec:derivative}) of a Function at the number, and
extending all Arithmetic Operators to take this into account, resuling in the
Algebra of Dual Numbers; generalized by theory of Operational Calculus
(\S\ref{sec:operational_calculus}) on (Differentiable) Programming Spaces
(\S\ref{sec:differentiable_programming}) through the Tensor Algebra of the Dual
(Vector) Space (\S\ref{sec:dual_space})



\paragraph{Grassmann Number}\label{sec:grassmann_number}\hfill

generalization to $n$ Dimensions of the Dual Numbers as the Exterior Algebra
(\S\ref{sec:exterior_algebra}) of a One-dimensional Vector Space

generated by \emph{Anti-commuting} Elements, e.g. Differential Forms
(\S\ref{sec:differential_form}) in Differential Geometry



\subsubsection{Extended Real Number}\label{sec:extended_real}

$\reals \cup \{-\infty, +\infty\}$

$\overline{\reals}$

Extended Real Line (\S\ref{sec:extended_real_line})

\begin{itemize}
  \item the Measure Function (\S\ref{sec:measure}) $\mu : \Sigma \rightarrow
    \overline{\reals}$ of a Measure Space
\end{itemize}



% --------------------------------------------------------------------
\subsection{Complex Number}\label{sec:complex_number}
% --------------------------------------------------------------------

\fist Complex Analysis (\S\ref{sec:complex_analysis})

one of three possible Finite-dimensional Associative Division Algebras
(\S\ref{sec:associative_division_algebra}) over the Real Numbers \fist
Frobenius Theorem (\S\ref{sec:frobenius_theorem})

$\comps$ can be viewed as an $\reals$-algebra (\S\ref{sec:k_algebra}, i.e. a
Vector Space that has a ``sensible way'' to define Vector Multiplication)

equivalent definitions: Algebraic Closure or Quadratic Extension of
the Real Numbers

Quaternions (\S\ref{sec:quaternion}) can be obtained by applying the
Cayley-Dickson Construction to Complex Numbers, i.e. the Direct Sum
(\S\ref{sec:direct_sum}) of the Complex Numbers with itself

$a + bi$

Complex Plane (\S\ref{sec:complex_plane})

\emph{Cartesian Form}: Ordered Pair $(a,b) \in \reals^2$

For $a \in \reals$, $(a,0)$ is Identified with the Real Number $a$

a Complex Number can be defined as a Polynomial (\S\ref{sec:polynomial}) in the
single Indeterminate with the Relation $i^2 + 1 = 0$ imposed, and Complex
Numbers can then be Added or Multiplied using Addition and Multiplication of
Polynomials

the Set of Complex Numbers is the Quotient Ring of the Polynomial Ring in the
Indeterminate $i$ by the Ideal generated by the Polynomial $i^2 + 1$

\emph{Polar Form}: Polar Coordinates (\S\ref{sec:polar_coordinates})

Argument (Angle) $\theta$

Modulus (Radius) $r$

$r(\cos\theta + i \sin\theta) = e^{i\theta}$

Algebraic Numbers (\S\ref{sec:algebraic_number})

Algebraically Closed Field (\S\ref{sec:algebraically_closed}) -- Fundamental
Theorem of Algebra (\S\ref{sec:fundamental_algebra_theorem}): extending the
Reals with the solution to the equation $i^2 + 1 = 0$ is Algebraically Closed

any Polynomial (\S\ref{sec:polynomial}) with Complex Coefficients has Complex
Roots (the Complex Numbers are Algebraically Closed)

Isomorphic to Algebraic Closure of $\rats_p$ (\S\ref{sec:padic_number})

$\reals$ and $\comps$ are the only Connected Locally Compact
Topological Fields (\S\ref{sec:topological_field})

Matrix Representation %FIXME

\url{https://johncarlosbaez.wordpress.com/2018/03/03/nonstandard-integers-as-complex-numbers/}:

any Field of Characteristic Zero (\S\ref{sec:ring_characteristic}) whose
Cardinality is less than that of the Continuum is \emph{Isomorphic} to some
Sub-field of the Complex Numbers

any Algebraically Closed Field with Cardinaliity equal to that of the Continuum
is Isomorphic to the Complex Numbers

if $D$ is Connected, the Meromorphic Functions
(\S\ref{sec:meromorphic_function}) on $D$ form a Field Extension of the Complex
Numbers



\subsubsection{Imaginary Number}\label{sec:imaginary_number}

An \emph{Imaginary Number} is a Complex Number that can be written as a Real
Number $b$ Multiplied by the Imaginary Unit $i$, defined by the Property:
\[
  i^2 = -1
\]
with:
\[
  bi = -b^2
\]



\subsubsection{Split-complex Number}\label{sec:split_complex}

or \emph{Hyperbolic number}



\subsubsection{Complex Arithmetic}\label{sec:complex_arithmetic}

Addition:

$(a,b) + (c,d) = (a + c, b + d)$

Multiplication:

$(a,b)*(c,d) = (ac - bd, ad + bc)$

$(0,1)*(0,1) = (-1, 0)$

in Polar Coordinates (\S\ref{sec:polar_coordinates}): Magnitude or
\emph{Modulus} of the Product is the Product of the two Absolute
Values (or Moduli) and the Angle or \emph{Argument} of the Product is
the Sum of the two Angles (or Arguments):
\[
  c_1 c_2 = r_1 r_2 e^{i\theta_1 + \theta_2}
\]

a Complex Number can be defined as a Polynomial (\S\ref{sec:polynomial}) in the
single Indeterminate with the Relation $i^2 + 1 = 0$ imposed, and Complex
Numbers can then be Added or Multiplied using Addition and Multiplication of
Polynomials



For $x,y,z \in \comps$:
\begin{itemize}
\item \emph{Commutativity}

$x + y = y + x$

$xy = yx$


\item \emph{Associativity}

$(x + y) + z = x + (y + z)$

$(xy)z = x(yz)$


\item \emph{Identities}

$x + 0 = x$

$x1 = x$


\item \emph{Additive Inverse}

$\forall x \in \comps \exists y \in \comps : x + y = 0$


\item \emph{Multiplicative Inverse}

$\forall x \neq 0 \in \comps \exists y \in \comps : xy = 1$


\item \emph{Distributive Property}

$z (x + y) = zx + zy$

\end{itemize}



\paragraph{Complex Conjugate}\label{sec:complex_conjugate}\hfill

The \emph{Complex Conjugate} of a Complex Number $a + bi$ is $a - bi$, that is
the same Real component and an Imaginary component of equal magnitude but
opposite sign

the Complex Conjugate of the Complex Product $x^*y$ is equal to the Complex
Product $y^*x$

the addition of Complex Conjugates $z + z^*$ results in a Real Number



\subsubsection{Extended Complex Number}\label{sec:extended_complex}

Complex Numbers plus a Value $\infty$

Modelled by Riemann Sphere (\S\ref{sec:riemann_sphere})



\subsubsection{Transcendental Number}\label{sec:transcendental}

Uncountable

Implies Irrational (\S\ref{sec:irrational})

$e$, $\pi$

Polynomial (\S\ref{sec:polynomial}) and Algebraic Expressions
(\S\ref{sec:algebraic_expression}) do not include Transcendental Numbers



\paragraph{Liouville Number}\label{sec:liouville_number}\hfill



% --------------------------------------------------------------------
\subsection{Bicomplex Number}\label{sec:bicomplex_number}
% --------------------------------------------------------------------

or \emph{Tessarines}

obtained by applying the Cayley-Dickson Construction to Complex Numbers, i.e.
the Direct Sum (\S\ref{sec:direct_sum}) of the Complex Numbers with itself



\subsubsection{Quaternion}\label{sec:quaternion}

$\quats$

one of three possible Finite-dimensional Associative Division Algebras
(\S\ref{sec:associative_division_algebra}) over the Real Numbers \fist
Frobenius Theorem (\S\ref{sec:frobenius_theorem})

\fist Dual Quaternions (\S\ref{sec:dual_quaternion})

a Pure Imaginary Quaternion is a Pseudovector (Axial Vector
\S\ref{sec:pseudovector}); Angular Velocity behaves like a Pseudovector

Conjugation is called the \emph{Spatial Inverse}

there are infinitely many Square Roots of $-1$: the Quaternion Solution for
$\sqrt{-1}$ is the Unit Sphere $S^2 \subset \reals^3$

only Negative Real Quaternions have Infinitely many Square Roots-- all others
have two (or one for the Zero Quaternion)



\paragraph{Versor}\label{sec:versor}

Unit Quaternion

Isomorphic with Special Unitary Group (\S\ref{sec:special_unitary}) $SU(2)$



\paragraph{Swing-twist Decomposition}\label{sec:swing_twist}

Decomposes Rotation, $q$, into two Rotations: the \emph{Twist}, $t$, around the
``Direction'' Vector and the \emph{Swing}, $s$, around the Axis that is
\emph{Perpendicular} to the ``Direction'' Vector:
\[
  q = s * t
\]

\url{https://stackoverflow.com/questions/3684269/component-of-a-quaternion-rotation-around-an-axis}

\url{http://www.euclideanspace.com/maths/geometry/rotations/for/decomposition/}



\subsubsection{Split-quaternion}\label{sec:split_quaternion}



% --------------------------------------------------------------------
\subsection{Octonian}\label{sec:octonian}
% --------------------------------------------------------------------

Normed Division Algebra (Composition Algebra \S\ref{sec:composition_algebra})

Hurwitz's Theorem (\S\ref{sec:hurwitzs_theorem})



% --------------------------------------------------------------------
\subsection{Cardinal Number}\label{sec:cardinal_number}
% --------------------------------------------------------------------

Generalization of Natural Numbers (\S\ref{sec:natural_number}) to possibly
Infinite magnitudes %FIXME wording from ncatlab

Cardinality (\S\ref{sec:cardinality})

nLab:

A Cardinal is a Set

Isomorphism Class of Sets

Set $A$, the Smallest possible Ordinal Rank (\S\ref{sec:ordinal_rank}) of any
Well-order on $A$. (Requires Axiom of Choice \S\ref{sec:choice_axiom} which
Implies the Well-ordering Theorem \S\ref{sec:wellorder_theorem}, i.e. every Set
is Well-orderable).

Well-ordered Cardinal

Well-founded Cardinal: Axiom of Choice or Axiom of Foundation
(\S\ref{sec:foundation_axiom})

$\mathbf{Card}$, $\cards$

$|\nats| = \aleph_0$

$|\pow\nats| = |\reals| = 2^{\aleph_0}$

\emph{Cardinal Exponentiation}

if $|X|$ is Infinite and $|2| \leq |A| \leq |X|$, then:
\[
  |A^X| = |X^X| = 2^|X| = |\pow(X)|
\]

the Set of all Continuous Functions from $\reals$ to $\reals$ has Cardinality
equal to $|\reals|$, while the Set of all Functions from $\reals$ to $\reals$
$\reals^\reals$ has strictly greater Cardinality

\url{https://johncarlosbaez.wordpress.com/2018/03/03/nonstandard-integers-as-complex-numbers/}:

any two Algebraically Closed Fields (\S\ref{sec:algebraically_closed}) of
Characteristic Zero (\S\ref{sec:ring_characteristic}) that have the same
Uncountable Cardinality must be Isomorphic

any Algebraically Closed Field with Cardinality equal to that of the Continuum
is Isomorphic to the Complex Numbers



\subsubsection{Finite Cardinal}\label{sec:finite_cardinal}

\subsubsection{Infinite Cardinal}\label{sec:infinite_cardinal}

\subsubsection{Transfinite Cardinal}\label{sec:transfinite_cardinal}

\paragraph{Large Cardinal}\label{sec:large_cardinal}\hfill

Large Cardinal Property: Property of Transfinite Cardinal Numbers

Large Cardinal Axiom: an Axiom stating there Exists a Cardinal (or
many Cardinals) with some Large Cardinal Property

Infinitary Combinatorics (\S\ref{sec:infinitary_combinatorics})

Propositions that Large Cardinals exist cannot be Proved in ZFC Set
Theory

``If you want more you have to assume more'' -- Dana Scott



\subparagraph{Rank-into-rank Cardinal}\label{sec:rank_into_rank}\hfill

I3 Rank Cardinal: Non-trivial Elementary Embedding
(\S\ref{sec:elementary_embedding}) of $V_\lambda$ into itself



\subparagraph{Shelf}\label{sec:shelf}

I3 Rank-into-rank Cardinal (\S\ref{sec:rank_into_rank})



\subsubsection{Successor Cardinal}\label{sec:successor_cardinal}



\subsubsection{Limit Cardinal}\label{sec:limit_cardinal}

Weak Limit Cardinal

Strong Limit Cardinal



\paragraph{Inaccessible Cardinal}\label{sec:inaccessible_cardinal}\hfill



\subsubsection{Cardinal Arithmetic}\label{sec:cardinal_arithmetic}

\subsubsection{Measurable Cardinal}\label{sec:measurable_cardinal}

\subsubsection{Regular Cardinal}\label{sec:regular_cardinal}

A \emph{Regular Cardinal} is a Cardinal Number that is equal to its
own Cofinality (\S\ref{sec:cofinality}).



\subsubsection{Continuum Hypothesis}\label{sec:continuum_hypothesis}

$2^{\aleph_0} = \aleph_1$

Generalized Continuum Hypothesis (\S\ref{sec:generalized_continuum})

cf. \emph{Cantor-Dedekind Axiom} -- thesis that the Real Numbers are
Order-isomorphic to the Linear Continuum (Order Theory
\S\ref{sec:linear_continuum}) in Geometry; a consequence is that the
Decidability of the Ordered Real Field can be seen as an Algorithm to solve any
problem in Euclidean Geometry (\S\ref{sec:euclidean_geometry})

Hyperreal Number (\S\ref{sec:hyperreal}) Field: $\reals^\nats /
\mathsf{M}$ where $\mathsf{M}$ is a Maximal Ideal
(\S\ref{sec:maximal_ideal}) \emph{not} resulting in a Field that is
Order-isomorphic (\S\ref{sec:order_isomorphism}) to $\reals$ -- the
uniqueness of this Field is equivalent to the Continuum Hypothesis



% --------------------------------------------------------------------
\subsection{Ordinal Number}\label{sec:ordinal_number}
% --------------------------------------------------------------------

Generalization of Natural Numbers (\S\ref{sec:natural_number}) to
possibly Infinite magnitudes %FIXME wording from ncatlab

$\mathbf{Ord}$, $\ords$, $\infty$

$\omega^{\omega^{\omega^{\cdots}}} = \epsilon_0$

Smallest solution to $x = \omega^x$

nLab:

An Ordinal is a Well-ordered Set

\emph{Ordinal Rank} (\S\ref{sec:ordinal_rank}): Isomorphism Class of
Well-ordered Sets (\S\ref{sec:well_order})

\begin{enumerate}
  \item Every Well-ordered Set has a Unique Ordinal Number for its
    Ordinal Rank
  \item Every Ordinal Number is the Ordinal Rank of some Well-ordered
    Set
  \item Two Well-ordered Sets have the same Ordinal Rank if they are
    Isomorphic as Well-ordered Sets
\end{enumerate}

Every Ordinal other than $0$ is either a Successor Ordinal
(\S\ref{sec:successor_ordinal}) or a Limit Ordinal
(\S\ref{sec:limit_ordinal})


\fist See also Ordinal Analysis (\S\ref{sec:ordinal_analysis})



\subsubsection{Ordinal Rank}\label{sec:ordinal_rank}

\emph{Order Rank} of a Set $S$ is its Isomorphism Class



\subsubsection{Finite Ordinal}\label{sec:finite_ordinal}

\subsubsection{Infinite Ordinal}\label{sec:infinite_ordinal}

\subsubsection{Transfinite Ordinal}\label{sec:transfinite_ordinal}

\subsubsection{Successor Ordinal}\label{sec:successor_ordinal}

Successor of an Ordinal Number $\alpha$ is the smallest Ordinal Number
greater than $\alpha$



\subsubsection{Limit Ordinal}\label{sec:limit_ordinal}

Ordinal Number that is neither Zero nor a Successor Ordinal



\paragraph{Inaccessible Ordinal}\label{sec:inaccessible_ordinal}\hfill



\subsubsection{Ordinal Arithmetic}\label{sec:ordinal_arithmetic}

$\alpha + 0 = \alpha$

$\alpha + S(\beta) = S(\alpha + \beta)$

for a Limit Ordinal $\lambda$:

$\alpha + \lambda = \bigcup_{\beta < \lambda} (\alpha + \beta)$

Ordinal Multiplication is not Commutative



\subsubsection{Von Neumann Ordinal}\label{sec:vonneumann_ordinal}

\subsubsection{Large Countable Ordinal}\label{sec:large_countable}

\paragraph{Ackermann Ordinal}\label{sec:ackermann_ordinal}\hfill



\subsubsection{Recursive Ordinal}\label{sec:recursive_ordinal}

\subsubsection{Regular Ordinal}\label{sec:regular_ordinal}

\subsubsection{Admissible Number}\label{sec:admissible_ordinal}

\subsubsection{Proof-theoretic Ordinal}\label{sec:proof_ordinal}

Arithmetical Transfinite Recursion $ATR_0$ -- Feferman-Sch\"utte
Ordinal (\S\ref{sec:feferman_schutte}) $\Gamma_0$

Ordinal Analysis (\S\ref{sec:ordinal_analysis})



\subsubsection{Generalized Continuum Hypothesis}
\label{sec:generalized_continuum}

if the General Continuum Hypothesis holds then all Real Closed Fields
(\S\ref{sec:real_closed}) with Cardinality of the Continuum and having the
$\eta_1$ Property (\S\ref{sec:eta_set}) are Order Isomorphic
(\S\ref{sec:order_isomorphism})

without the Continuum Hypothesis, if the Cardinality of the Continuum is
$\aleph_\beta$, then there is a Unique $\eta_\beta$ Field of Size $\eta_beta$

ZFC plus GCH or Axiom of Constructibility implies Axiom of Choice, making GCH a
strictly stronger claim than the Axiom of Choice



\subsubsection{Veblen Hierarchy}\label{sec:veblen_hierarchy}

Veblen Functions %FIXME

First Critical Epsilon Number $x = \epsilon_x$ %FIXME

$\phi_\gamma(\alpha)$



\paragraph{Feferman-Sch\"utte Ordinal}\label{sec:feferman_schutte}\hfill

$\Gamma_0$

smallest solution of $x = \phi_x(0)$



\paragraph{Small Veblen Ordinal}\label{sec:small_veblen}\hfill

\paragraph{Large Veblen Ordinal}\label{sec:large_veblen}\hfill



\subsubsection{Ordinal Collapsing Function}
\label{sec:ordinal_collapsing}

\paragraph{Bachmann-Howard Ordinal}\label{sec:bachmann_howard}



% --------------------------------------------------------------------
\subsection{Transfinite Number}\label{sec:transfinite_number}
% --------------------------------------------------------------------

Infinite but not Absolutely Infinite (\S\ref{sec:absolute_infinity})



% --------------------------------------------------------------------
\subsection{Absolute Infinity}\label{sec:absolute_infinity}
% --------------------------------------------------------------------

% --------------------------------------------------------------------
\subsection{Infinitesimal}\label{sec:infinitesimal}
% --------------------------------------------------------------------

Number Systems including Infinitesimals:
\begin{itemize}
  \item Hyperreal Numbers (\S\ref{sec:hyperreal_number})
  \item Surreal Numbers (\S\ref{sec:surreal_number})
  \item Superreal Numbers (TODO)
  \item Dual Numbers (\S\ref{sec:dual_number})
  \item Formal Power Series (\S\ref{sec:formal_power_series})
\end{itemize}

\fist Nonstandard Analysis (\S\ref{sec:nonstandard_analysis}) -- Differentials
(\S\ref{sec:differential}) as Infinitesimals in
Hyperreal Number Systems (\S\ref{sec:hyperreal})

\fist Smooth Infinitesimal Analysis (Synthetic Differential Geometry
\S\ref{sec:smooth_infinitesimal_analysis}) -- reformulation of Differential
Calculus (\S\ref{sec:differential_calculus}) in terms of Infinitesimals

\fist Infinitesimal Objects (\S\ref{sec:infinitesimal_object})

\fist Infinitesimal Interval (\S\ref{sec:infinitesimal_interval})



% --------------------------------------------------------------------
\subsection{Supernatural Number}\label{sec:supernatural_number}
% --------------------------------------------------------------------

% --------------------------------------------------------------------
\subsection{Hyperreal}\label{sec:hyperreal}
% --------------------------------------------------------------------

Real Closed Field (\S\ref{sec:real_closed})

\fist Nonstandard Analysis (\S\ref{sec:nonstandard_analysis}) -- Differentials
  (\S\ref{sec:differential}) as Infinitesimals (\S\ref{sec:infinitesimal}) in
  Hyperreal Number Systems

$\reals^\nats / \mathsf{M}$ where $\mathsf{M}$ is a Maximal Ideal
(\S\ref{sec:maximal_ideal}) \emph{not} resulting in a Field that is
Order-isomorphic (\S\ref{sec:order_isomorphism}) to $\reals$ --
uniqueness of this Field is equivalent to the Continuum Hypothesis
(\S\ref{sec:continuum_hypothesis})



% --------------------------------------------------------------------
\subsection{Surreal Number}\label{sec:surreal_number}
% --------------------------------------------------------------------

Properties of a Field (\S\ref{sec:field})



% --------------------------------------------------------------------
\subsection{$p$-adic Number}\label{sec:padic_number}
% --------------------------------------------------------------------

Algebraic Closure of $\rats_p$ is Isomorphic to the Complex Numbers
(\S\ref{sec:complex_number})

Non-archimedean Field (\S\ref{sec:nonarchimedean_field})



% ====================================================================
\section{Arithmetic}\label{sec:arithmetic}
% ====================================================================

Cardinal Arithmetic (\S\ref{sec:cardinal_arithmetic})

Ordinal Arithmetic (\S\ref{sec:ordinal_arithmetic})

Complex Arithmetic (\S\ref{sec:complex_arithmetic})

The wider class of Polynomial Expressions (\S\ref{sec:polynomial}) additionally
allows Positive Integer Exponents, and Algebraic Expressions
(\S\ref{sec:algebraic_expression}) allow for Negative and Rational ($n$-th
Root) Exponents, however they do not include Transcendental Numbers
(\S\ref{sec:transcendental}).

\fist cf. Arithmetic (Algebraic) Geometry
(\S\ref{sec:arithmetic_geometry})

as an Algebraic Structure (\S\ref{sec:algebraic_structure}), an Arithmetic
Structures is Structure over an Infinite Set $S$ with two Binary Operations of
Addition and Multiplication, which is also a Pointed Unary System with
Injective Successor as Unary Operation and Distinguished Element $0$

a Ring (\S\ref{sec:ring}) is a Set equipped with two Binary Operations that
generalize the Arithmetic Operations to ``non-numerical objects'' such as
Polynomials, Series, Matrices, and Functions



% --------------------------------------------------------------------
\subsection{Arithmetic Expression}\label{sec:arithmetic_expression}
% --------------------------------------------------------------------

an \emph{Arithmetic Expression} includes Constants, Variables, Elementary
Arithmetic Operations (Addition, Subtraction, Multiplication, and Division), and
Factorials



% --------------------------------------------------------------------
\subsection{Arithmetic Function}\label{sec:arithmetic_function}
% --------------------------------------------------------------------

(or \emph{Number-theoretic Function})



\subsubsection{Subadditive Function}\label{sec:subadditive_function}

Additive Map (\S\ref{sec:additive_map})

cf. Subadditive Set Function (\S\ref{sec:subadditive_set_function})

Subadditivity of Entropy (\S\ref{sec:entropy})



\subsubsection{Additive Function}\label{sec:additive_function}

an \emph{Additive Function} $f$ is one satisfying the Functional Equation
(\S\ref{sec:functional_equation}) $f(x+y) = f(x) + f(y)$



\subsubsection{Multiplicative Function}
\label{sec:multiplicative_function}

For Coprime $a,b \in \mathbb{Z}$:
\[
  f(ab) = f(a)f(b)
\]

Examples:
\begin{itemize}
  \item $f(n) = n^2$
  \item $\tau$-function (\S\ref{sec:tau_function})
  \item $\sigma$-function (\S\ref{sec:sigma_function})
  \item Euler's Totient Function (\S\ref{sec:eulers_totient})
\end{itemize}



\paragraph{M\"obius Function}\label{sec:mobius_function}\hfill

\[
  \mu(n) =
  \begin{cases}
  1     & \quad n = 1 \\
  0     & \quad \exists p \in \primes : p^2|n \\
  -1^m  & \quad n = \prod_{i = 1}^m q_i, q_i \in \primes \\
  \end{cases}
\]



\paragraph{Legendre Symbol}\label{sec:legendre_symbol}\hfill

Dirichlet Character (\S\ref{sec:dirichlet_character})



\subparagraph{Jacobi Symbol}\label{sec:jacobi_symbol}\hfill



\subsubsection{Divisor Function}\label{sec:divisor_function}

Number $d(n)$ of Divisors of $n$ -- Ramanujan15

$F(n) = \sum_{d|n}f(d)$

If $f$ is Multiplicative (\S\ref{sec:multiplicative_function}) then
$F$ is Multiplicative.



\paragraph{$\tau$-function}\label{sec:tau_function}\hfill

$\tau(n)$ Odd if and only if $n$ is Square



\paragraph{$\sigma$-function}\label{sec:sigma_function}\hfill

$\sigma_x(n) = \sum_{d | n} d^x$



\paragraph{Euler's Totient Function}\label{sec:eulers_totient}\hfill

\emph{Euler's Phi Function}

$\varphi(p^k) = p^k - p^{k-1}$

The Divisor Function (\S\ref{sec:divisor_function}) for $\varphi$
gives:
\[
  \sum_{d|n}\varphi(d) = n
\]

For Prime $p$ and $d|p-1$, there are exactly $\varphi(d)$ Elements $x
\in \mathbb{Z}/p$ with Order (\S\ref{sec:modular_order}) $O_p(x) = d$.

\subparagraph{Euler's Totient Theorem}\label{sec:totient_theorem}\hfill

If $gcd(a,n) = 1$ then $a^{\varphi(n)} \equiv 1 \mod n$



\subsubsection{Completely Additive Function}
\label{sec:completely_additive_function}

\subsubsection{Completely Multiplicative Function}
\label{sec:completely_multiplicative_function}

\subsubsection{M\"obius Inversion Formula}
\label{sec:mobius_inversion}

\subsubsection{Von Mangoldt Function}\label{sec:vonmangoldt_function}

$\Lambda(n)$

Neither Multiplicative nor Additive



\subsubsection{Dirichlet Convolution}\label{sec:dirichlet_convolution}



% --------------------------------------------------------------------
\subsection{Skolem Arithmetic}\label{sec:skolem_arithmetic}
\cite{skolem23}
% --------------------------------------------------------------------

% --------------------------------------------------------------------
\subsection{First-order Arithmetic}\label{sec:firstorder_arithmetic}
% --------------------------------------------------------------------

First-order Logic (\S\ref{sec:firstorder_logic})



\subsubsection{Presburger Arithmetic}\label{sec:presburger_arithmetic}

First-order Theory of the Natural Numbers

Decidable, weaker than Peano Arithmetic



\subsubsection{Robinson Arithmetic}\label{sec:robinson_arithmetic}

Peano Arithmetic without Induction



\subsubsection{Peano Arithmetic}\label{sec:peano_arithmetic}



% --------------------------------------------------------------------
\subsection{Second-order Arithmetic}\label{sec:second_order_arithmetic}
% --------------------------------------------------------------------

% --------------------------------------------------------------------
\subsection{Modular Arithmetic}\label{sec:modular_arithmetic}
% --------------------------------------------------------------------

Partition of $\mathbb{Z}$ into $\mathbb{Z}/n$ by $n \in \mathbb{N}$.

For Prime $p \in \primes$ (``Freshman's Dream''):
\[
  (x+y)^p \mod n \equiv x^p + y^p \mod n
\]



\subsubsection{Modular Congruence}\label{sec:modular_congruence}

$a \equiv b \mod n$ \emph{Congruent modulo $n$}

For two Integers $a,b \in \mathbb{Z}$, $a$ and $b$ are in the same
Quotient Set or \emph{Residue} (\S\ref{sec:residue}) if and only if
$(a - b)|n$, denoted $a \equiv b \mod n$.

\begin{enumerate}

  \item $(\forall a \in \mathbb{Z}) a \equiv a \mod n$

  \item $(\forall a,b \in \mathbb{Z}) a \equiv b \mod n
    \Leftrightarrow b \equiv a \mod n$

  \item $(\forall a,b,c \in \mathbb{Z}) a \equiv b \mod n
    \wedge b \equiv c \mod n \Rightarrow a \equiv c
    \mod n$

  \item
    If $a \equiv b \mod n$ and $c \equiv d (\mathrm{mod }
    n)$, then:
    \begin{itemize}
    \item $a + c \equiv b + d \mod n$
    \item $a - c \equiv b - d \mod n$
    \item $ac \equiv bd \mod n$
    \item $a^t \equiv b ^t \mod n$ for any $t \in
      \mathbb{N}$
    \end{itemize}

\end{enumerate}

Cancellation: If $ac \equiv bc \mod n$ then $a \equiv b
(\mathrm{mod } \frac{n}{d})$ where $d = gcd(n,c)$.



\paragraph{Chinese Remainder Theorem}\label{sec:chinese_remainder}\hfill

\paragraph{Simultaneous Congruence}\label{sec:simultaneous_congruence}\hfill



\subsubsection{Residue}\label{sec:residue}

\textbf{Complete Set of Residues}

Pigeon Hole Principle: A Set of Integers $\{a_1, a_2, \ldots, a_n\} =
T \subset \mathbb{Z}$ is a \emph{Complete Set of Residues} (CSR)
$\mod n$ if $|T| = n$ and no Elements of a particular Residue
$\mod n$ occur more than once.

For a CSR $\mod n$ $T = \{a_1, a_2, \ldots, a_n\}$ and an
Integer $c \in \mathbb{Z}$ Coprime with $n$ ($gcd(c,n) = 1$), for any
Integer $b \in \mathbb{Z}$, the Set $S = \{b + ca_1, b + ca_2, \ldots,
b + ca_n \}$ is also a CSR $\mod n$.



\textbf{Complete Set of Non-zero Residues}

$T = \{ a_1, a_2, \ldots, a_{n-1} \}$ Set of $n-1$ Integers, none are
$\equiv 0 \mod n$



\paragraph{Quadratic Residue}\label{sec:quadratic_residue}\hfill

\subparagraph{Quadratic Reciprocity}\label{sec:quadratic_reciprocity}\hfill



\paragraph{Least Residue}\label{sec:least_residue}\hfill

\paragraph{Reduced Residue}\label{sec:reduced_residue}\hfill



\subsubsection{Linear Congruence}\label{sec:linear_congruence}

$ax \equiv b \mod n$ has solutions if and only if $d = gcd(a,n) | b$.
If $d|b$ then there are exactly $d$ solutions in a given CSR $\mod n$.



\subsubsection{Fermat's Little Theorem}\label{sec:fermat_little}

$p \in \primes$, $a^p \equiv a \mod n$

If $p \in \primes$ and $a \in \mathbb{Z}$ and $p \nmid a$ then
$a^{p-1} \equiv 1 \mathbb{mod}\;p$.

For $p \in \primes$ and $a \in \mathbb{Z}$, $\{ a, 2a, \ldots, (p-1)a
\}$ is a Non-zero CSR.



\subsubsection{Modular Order}\label{sec:modular_order}

The \emph{Order} of $a \in \mathbb{Z}$ relative to $n \in \mathbb{Z}$,
denoted $O_n(a)$ is the smallest Exponent $j$ such that $a^j \equiv 1
\mod n$.

$a^e \equiv 1 \mod n \Leftrightarrow O_n(a) | e$

For $a, n$ with $gcd(a,n) = 1$, $O_n(a) | \varphi(n)$

$O_n(a^i) = \frac{O_n(a)}{gcd(i,O_n(a))}$



\subsubsection{Primitive Root}\label{sec:primitive_root}

A \emph{Primitive Root} is an $a \in \mathbb{Z}$ such that $gcd(a,n) =
1$ and the Sequence $a, a^2, a^3, \ldots$ gives a Complete Set of
Remainders $\mod n$

For $gcd(a,n) = 1$, $a$ is a Primitive Root $\mod n$ if and only if
$O_n(a) = \varphi(n)$ (the Order \S\ref{sec:modular_order} of $a$ is
equal to the number of Relatively Prime Integers $< n$).

If $n$ has at least one Primitive Root then it has
$\varphi(\varphi(n))$ Primitive Roots defined by the Set:
\[
  \{ a^i | 1 \leq i \leq \varphi(n), gcd(i,\varphi(n)) = 1 \}
\]

For Prime $p$, $p$ has $\varphi(p-1)$ Primitive Roots.

Index (Discrete Logarithm) %FIXME xref



% --------------------------------------------------------------------
\subsection{Heyting Arithmetic}\label{sec:heyting_arithmetic}
% --------------------------------------------------------------------

Peano Arithmetic with Intuitionistic Logic

G\"odel's \emph{Dialectica Interpretation}

Proof Interpretation of Heyting Arithmetic into a Finite-type
Extension of Primitive Recursive Arithmetic
(System T \S\ref{sec:system_t}))



% ====================================================================
\section{Exponentiation}\label{sec:exponentiation}
% ====================================================================

\fist Exponential (Category Theory \S\ref{sec:category_exponential})

Exponentiation is Right-adjoint to the Cartesian Product
(\S\ref{sec:cartesian_product})

a Polynomial Expression (\S\ref{sec:polynomial}) may contain
Positive Integer Exponents

an Algebraic Expression (\S\ref{sec:algebraic_expression}) may contain Negative
and Rational Exponents



% --------------------------------------------------------------------
\subsection{Tetration}\label{sec:tetration}
% --------------------------------------------------------------------

not an Elementary Function



% ====================================================================
\section{Logarithm}\label{sec:logarithm}
% ====================================================================

% --------------------------------------------------------------------
\subsection{Discrete Logarithm}\label{sec:discrete_logarithm}
% --------------------------------------------------------------------

for $a$ an Element of a Group $G$, an Integer $k$ that solves the Equation $b^k
= a$ is a \emph{Discrete Logarithm}:
\[
  k = \log_b a
\]

\emph{Discrete Logarithm Problem}

\fist Arithmetic Cryptography (\S\ref{sec:arithmetic_cryptography}) based on
the Discrete Logarithm Problem on Elliptic Curves (\S\ref{sec:elliptic_curve})
or more general Abelian Varieties (\S\ref{sec:abelian_variety}) over Finite
Fields (\S\ref{sec:finite_field}):

(wiki):

Public-key Cryptography based security on the assumption that the Discrete
Logarithm Problem over ``carefully chosen Groups'' has no efficient solution
(FIXME: clarify)


\emph{Index Calculus Algorithm}



% ====================================================================
\section{Analytic Number Theory}\label{sec:analytic_number_theory}
% ====================================================================

% --------------------------------------------------------------------
\subsection{Multiplicative Number Theory}
\label{sec:multiplicative_number_theory}
% --------------------------------------------------------------------

\subsubsection{Dirichlet Character}\label{sec:dirichlet_character}

\subsubsection{Dirichlet $L$-Series}\label{sec:l_series}

Meromorphic Functions (\S\ref{sec:meromorphic_function})



% --------------------------------------------------------------------
\subsection{Additive Number Theory}\label{sec:additive_number_theory}
% --------------------------------------------------------------------

% --------------------------------------------------------------------
\subsection{$p$-adic Analysis}\label{sec:padic_analysis}
% --------------------------------------------------------------------



% ====================================================================
\section{Algebraic Number Theory}\label{sec:algebraic_number_theory}
% ====================================================================

Arithmetic Topology (\S\ref{sec:arithmetic_topology})

three closely related fields:
\begin{itemize}
  \item Algebraic Number Theory
  \item Algebraic Geometry (Part \ref{part:algebraic_geometry})
  \item Commutative Algebra (\S\ref{sec:commutative_algebra})
\end{itemize}



% --------------------------------------------------------------------
\subsection{Algebraic Number}\label{sec:algebraic_number}
% --------------------------------------------------------------------

$\algs$

any Complex Number (\S\ref{sec:complex_number}) that is the Solution of $P$
Polynomial with Integer Coefficients (\S\ref{sec:integer_coefficient})

Countable

Algebraic Closure of (\S\ref{sec:algebraically_closed}) $\rats$

\emph{Artin-Schreier Theorem} (\S\ref{sec:artin_schreier}):
an Ordered Field (\S\ref{sec:ordered_field}) $F$ has an Algebraic Extension
(\S\ref{sec:algebraic_extension}) called the \emph{Real Closure} $K$ of $F$
such that $K$ is a Real Closed Field with Ordering as an Extension of the given
Ordering on $F$ and is Unique up to Unique Isomorphism of Fields identical to
$F$

the Real Closure of the Ordered Field of Rational Numbers is the Field
$\algs$ of Real Algebraic Numbers

Order-isomorphic (\S\ref{sec:order_isomorphism}) to the Rational
Numbers (\S\ref{sec:rational})



\subsubsection{Gaussian Integer}\label{sec:gaussian_integer}\hfill

\subsubsection{Constructible Number}\label{sec:constructible_number}\hfill



% --------------------------------------------------------------------
\subsection{Period Number}\label{sec:period_number}
% --------------------------------------------------------------------

Kontsevich-Zagier01

Countable Superset of Algebraic Numbers that have Finite descriptions
and includes many Transcendental Numbers (e.g. the Transcendental
Number $\pi$)



% ====================================================================
\section{Computationsl Number Theory}\label{sec:computational_number_theory}
% ====================================================================

% --------------------------------------------------------------------
\subsection{Integer Factorization}\label{sec:integer_factorization}
% --------------------------------------------------------------------



% ====================================================================
\section{Probabilistic Number Theory}\label{sec:probabilistic_number_theory}
% ====================================================================

% ====================================================================
\section{Sieve Theory}\label{sec:sieve_theory}
% ====================================================================

% --------------------------------------------------------------------
\subsection{Sieve of Eratosthenes}\label{sec:sieve}
% --------------------------------------------------------------------



% ====================================================================
\section{Finitism}\label{sec:finitism}
% ====================================================================

% --------------------------------------------------------------------
\subsection{Strict Finitism}\label{sec:strict_finitism}
% --------------------------------------------------------------------

% --------------------------------------------------------------------
\subsection{Ultrafinitism}\label{sec:ultrafinitism}
% --------------------------------------------------------------------
