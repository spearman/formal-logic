%%%%%%%%%%%%%%%%%%%%%%%%%%%%%%%%%%%%%%%%%%%%%%%%%%%%%%%%%%%%%%%%%%%%%%
%%%%%%%%%%%%%%%%%%%%%%%%%%%%%%%%%%%%%%%%%%%%%%%%%%%%%%%%%%%%%%%%%%%%%%
\part{Number Theory}\label{sec:number_theory}
%%%%%%%%%%%%%%%%%%%%%%%%%%%%%%%%%%%%%%%%%%%%%%%%%%%%%%%%%%%%%%%%%%%%%%
%%%%%%%%%%%%%%%%%%%%%%%%%%%%%%%%%%%%%%%%%%%%%%%%%%%%%%%%%%%%%%%%%%%%%%

% ====================================================================
\section{Number}\label{sec:number}
% ====================================================================

% --------------------------------------------------------------------
\subsection{Natural Number}\label{sec:natural_number}
% --------------------------------------------------------------------

$\mathbb{N} = \{ 1,2,3,\ldots \}$

Well-ordering (\S\ref{sec:well_order})



\subsubsection{Principle of Mathematical Induction}
\label{sec:induction_principle}



% --------------------------------------------------------------------
\subsection{Integer}\label{sec:integer}
% --------------------------------------------------------------------

$\mathbb{Z} = \{ \ldots, -2, -1, 0, 1, 2, \ldots \}$



\subsubsection{Divisibility}\label{sec:divisibility}

$(a,b,q \in \mathbb{Z}), b|a \Leftrightarrow a = qb$

Properties:

\begin{itemize}
\item $\forall a \in \mathbb{Z}, 1|a$
\item $\forall a \in \mathbb{Z}, a|0$
\item Linear Combinations (\S\ref{sec:linear_combination}): $b|a_1
  \wedge b|a_2 \Rightarrow (\forall x,y \in \mathbb{Z})b|a_1 x + a_2
  y$
\end{itemize}



\paragraph{Greatest Common Divisor}\label{sec:gcd}
\hfill \\

Greatest Common Divisor (GCD)

$gcd(0,n) = n$

\emph{Linear Combination Theorem}: The GCD of two non-zero Integers
$a$ and $b$ is equal to the smallest possible Positive Linear
Combination (\S\ref{sec:linear_combination}) of $a$ and $b$.

Corollary 1: The Set of Linear Combinations of $a$ and $b$ is equal to
the Set of Multiples of $gcd(a,b)$.

Corollary 2: If $1$ is the smallest Positive Linear Combination of $a$
and $b$ then $gcd(a,b) = 1$ and $a$ and $b$ are Coprime
(\S\ref{sec:coprime}).

Corollary 3: If $a|c$ and $b|c$ and $gcd(a,b) = 1$, then $ab|c$.

Corollary 4: (Euclid's Lemma) If $a | bc$ and $gcd(a,b) = 1$ then
$a|c$.

Corollary 5: $gcd(\frac{a}{d}, \frac{b}{d}) = 1$



\subparagraph{Euclid's Algorithm}\label{sec:euclids_algorithm}



\subsubsection{Prime Number}\label{sec:prime_number}

$\mathbb{P}$

A Number that is \emph{not} Prime is called \emph{Composite}.

The Cardinality of $\mathbb{P}$ is Countably Infinite.

The Series (\S\ref{sec:series}) $\sum_{p \in \mathbb{P}} \frac{1}{p}$
Diverges The corresponding Series over Twin Primes
(\S\ref{sec:twin_prime}) Converges.

If $p \in \mathbb{P}$, then $gcd(p,a) = 1$ or $gcd(p,a) = p$ for any
$a \in \mathbb{Z}$.

If $p \nmid b$ and $p \in \mathbb{P}$ then $gcd (p,b) = 1$.

Corollary to Euclid's Lemma: If $p \in \mathbb{P}$ and $p|a_1 a_2
\ldots a_n$ then $p|a_i$ for some $1 \leq i \leq n$.

Given any $l \in \mathbb{N}$, a Sequence of $l$ consecutive Composite
Numbers may be found.

\emph{Bertrand's Postulate} (\emph{Bertrand-Chebyshev Theorem}):
\[
  (\forall n \in \mathbb{Z}^{>3}, \exists p \in \mathbb{P}) n < p < 2n
\]

Given an Arithmetic Sequence $S = \{ a + bx \;|\; x \in \mathbb{Z}^+\}$:
\begin{enumerate}
  \item If $gcd(a,b) > 1$ then there are no Primes in $S$.
  \item If $gcd(a,b) = 1$ then there are Infinitely many Primes in $S$
    (Dirichlet)
  \item For any $l \in \mathbb{N}$, there exists a consecutive
    Sequence of Primes of Length $l$ in $S$ (Green \& Tao)
\end{enumerate}

Sieve Theory (\S\ref{sec:sieve_theory})

For all $p \in \mathbb{P}$, $gcd(p, (p-1)!) = 1$

If $n \notin \mathbb{P}$ and $n > 4$ then $gcd(n,(n-1)!) = n$

\emph{Wilson's Theorem}:
\[
  (p - 1)! = -1 \mathrm{mod}\;p
\]
for $p \in \mathbb{P}$.

$x^2 \equiv -1 \mathrm{mod}\;p$ if and only if $p \equiv 1 mod 4$



\paragraph{Coprime}\label{sec:coprime}
\hfill \\

Two Integers are \emph{Coprime} if $gcd (a,b) = 1$.



\paragraph{Prime Decomposition}\label{sec:prime_decomposition}
\hfill \\

Standard Prime Factorization: $n = p_1^{k_1} p_2^{k_2} \ldots
p_n^{k_n}$.

\emph{Fundamental Theorem of Arithmetic}: Any Integer $\geq 2$ can be
written as a Unique Product of Prime Numbers (Standard Prime
Factorization).

The Uniqueness of the result of the Fundamental Theorem does not hold
in Quotient Sets over the Integers.



\paragraph{Prime Counting Function}\label{sec:prime_counting}

\paragraph{Prime Power}\label{sec:prime_power}
\hfill \\

A \emph{Prime Power} is a Prime Number raised to the power of a
positive Integer.



\paragraph{Twin Prime}\label{sec:twin_prime}

\paragraph{Fermat Prime}\label{sec:fermat_prime}

\paragraph{Goldbach Conjecture}\label{sec:goldbach_conjecture}
\hfill \\

Partial result:
\[
  (\forall n \in 2\mathbb{N}, \exists p,q \vee p,q,r \in \mathbb{P})
  n = p + q \vee n = p + q * r
\]



\subsubsection{Diophantine Equation}\label{sec:diophantine_equation}

\paragraph{Linear Diophantine Equation}
\label{sec:linear_diophantine}
\hfill \\
Monomial (\S\ref{sec:monomial}) Equations with Integer Solutions

$c = ax + by$ has a solution in $\mathbb{Z}$ when $gcd(a,b)|c$. Given
$d=gcd(a,b)|c$ and a solution $(x_0, y_0)$, then all solutions are of
the form $(x_0 + \frac{b}{d}t, y_0 - \frac{a}{d}t)$ with $t \in
\mathbb{Z}$.



\paragraph{Exponential Diophantine Equation}
\label{sec:exponential_diophantine}
\hfill \\



\subsubsection{Integer Coefficient Polynomial}
\label{sec:integer_coefficient}

$\mathbb{Z}[x]$

Polynomial (\S\ref{sec:polynomial})

For any $p(x) \in \mathbb{Z}[x]$ and $a \equiv b \mod n$,
$p(a) \equiv p(b) \mathrm{mox }n$.



% --------------------------------------------------------------------
\subsection{Rational Number}\label{sec:rational}
% --------------------------------------------------------------------

$\mathbb{Q}$

A \emph{Rational Number} $q \in \mathbb{Q}$ is a ratio of Integers
$m,n \in \mathbb{Z}$:
\[
  q = \frac{n}{m}
\]

\textbf{Lowest Form}: If $m$ and $n$ are not Coprime with $gcd (m,n) =
d$, an equivalent Rational Number can be written where the Numerator
and Denominator are Coprime by:
\[
  q = \frac{n}{m} = \frac{\frac{n}{d}}{\frac{m}{d}}
\]



% --------------------------------------------------------------------
\subsection{Real Number}\label{sec:real_number}
% --------------------------------------------------------------------

Three definitions:
\begin{enumerate}
  \item \emph{Axiomatically}: an Ordered Field
    (\S\ref{sec:ordered_field}) with the Least Upperbound Property
  \item \emph{Dedekind Cuts}: Partitions of the Rational Numbers into
    Upper and Lower Sets
  \item \emph{Cauchy Sequences}: a Sequence of Elements that given any
    ``small'' Positive Distance, all but a Finite number of Elements
    of the Sequence are less than that given Distance from eachother
\end{enumerate}



\subsubsection{Irrational Number}\label{sec:irrational}

$\mathbb{R}/\mathbb{Q}$

The Square Root of any Non-square Natural Number is Irrational.



\subsubsection{Algebraic Number}\label{sec:algebraic_number}

\subsubsection{Transcendental Number}\label{sec:transcendental}

\subsubsection{Computable Number}\label{sec:computable_real}

also \emph{Recursive Numbers} or \emph{Computable Reals}

\emph{Real Closed Field} (\S\ref{sec:closed_field})



% --------------------------------------------------------------------
\subsection{Complex Number}\label{sec:complex_number}
% --------------------------------------------------------------------

\subsubsection{Gaussian Integer}\label{sec:gaussian_integer}



% --------------------------------------------------------------------
\subsection{Cardinal Number}\label{sec:cardinal_number}
% --------------------------------------------------------------------

\subsubsection{Limit Cardinal}\label{sec:limit_cardinal}

\subsubsection{Inaccessible Cardinal}\label{sec:inaccessible_cardinal}

\subsubsection{Measurable Cardinal}\label{sec:measurable_cardinal}

\subsubsection{Regular Cardinal}\label{sec:regular_cardinal}

A \emph{Regular Cardinal} is a Cardinal Number that is equal to its
own Cofinality (\S\ref{sec:cofinality}).



% --------------------------------------------------------------------
\subsection{Ordinal Number}\label{sec:ordinal_number}
% --------------------------------------------------------------------

\subsubsection{Limit Ordinal}\label{sec:limit_ordinal}

\subsubsection{Regular Ordinal}\label{sec:regular_ordinal}

\subsubsection{Admissible Number}\label{sec:admissible_ordinal}

\subsubsection{Von Neumann Ordinal}\label{sec:vonneumann_ordinal}

\subsubsection{Proof-theoretic Ordinal}\label{sec:proof_ordinal}



% --------------------------------------------------------------------
\subsection{Transfinite Number}\label{sec:transfinite_number}
% --------------------------------------------------------------------

% --------------------------------------------------------------------
\subsection{Infinitesimal}\label{sec:infinitesimal}
% --------------------------------------------------------------------

% --------------------------------------------------------------------
\subsection{Hyperreal}\label{sec:hyperreal}
% --------------------------------------------------------------------

Non-standard Analysis (\S\ref{sec:nonstandard_analysis})



% --------------------------------------------------------------------
\subsection{Surreal Number}\label{sec:surreal_number}
% --------------------------------------------------------------------

% --------------------------------------------------------------------
\subsection{$p$-adic Number}\label{sec:padic_number}
% --------------------------------------------------------------------



% ====================================================================
\section{Arithmetic}\label{sec:arithmetic}
% ====================================================================

% --------------------------------------------------------------------
\subsection{Arithmetic Function}\label{sec:arithmetic_function}
% --------------------------------------------------------------------

(or \emph{Number-theoretic Function})



\subsubsection{Multiplicative Function}
\label{sec:multiplicative_function}

For Coprime $a,b \in \mathbb{Z}$:
\[
  f(ab) = f(a)f(b)
\]

Examples:
\begin{itemize}
  \item $f(n) = n^2$
  \item $\tau$-function (\S\ref{sec:tau_function})
  \item $\sigma$-function (\S\ref{sec:sigma_function})
  \item Euler's Totient Function (\S\ref{sec:eulers_totient})
\end{itemize}



\subsubsection{Divisor Function}\label{sec:divisor_function}

$F(n) = \sum_{d|n}f(d)$

If $f$ is Multiplicative (\S\ref{sec:multiplicative_function}) then
$F$ is Multiplicative.



\paragraph{$\tau$-function}\label{sec:tau_function}

$\tau(n)$ Odd if and only if $n$ is Square



\paragraph{$\sigma$-function}\label{sec:sigma_function}

\paragraph{Euler's Totient Function}\label{sec:eulers_totient}
\hfill \\

\emph{Euler's Phi Function}

$\varphi(p^k) = p^k - p^{k-1}$

\subparagraph{Euler's Totient Theorem}\label{sec:totient_theorem}
\hfill \\

If $gcd(a,n) = 1$ then $a^{\varphi(n)} \equiv 1 \mod n$



% --------------------------------------------------------------------
\subsection{Skolem Arithmetic}\label{sec:skolem_arithmetic}
\cite{skolem23}
% --------------------------------------------------------------------

% --------------------------------------------------------------------
\subsection{First-order Arithmetic}\label{sec:firstorder_arithmetic}
% --------------------------------------------------------------------

First-order Logic (\S\ref{sec:firstorder_logic})



\subsubsection{Peano Arithmetic}\label{sec:peano_arithmetic}

\subsubsection{Presburger Arithmetic}\label{sec:presburger_arithmetic}

Decidable, weaker than Peano Arithmetic



% --------------------------------------------------------------------
\subsection{Second-order Arithmetic}\label{sec:second_order_arithmetic}
% --------------------------------------------------------------------

% --------------------------------------------------------------------
\subsection{Modular Arithmetic}\label{sec:modular_arithmetic}
% --------------------------------------------------------------------

Partition of $\mathbb{Z}$ into $\mathbb{Z}/n$ by $n \in \mathbb{N}$.

For Prime $p \in \mathbb{P}$ (``Freshman's Dream''):
\[
  (x+y)^p \mod n \equiv x^p + y^p \mod n
\]



\subsubsection{Modular Congruence}\label{sec:modular_congruence}

$a \equiv b \mod n$ \emph{Congruent modulo $n$}

For two Integers $a,b \in \mathbb{Z}$, $a$ and $b$ are in the same
Quotient Set or \emph{Residue} (\S\ref{sec:residue}) if and only if
$(a - b)|n$, denoted $a \equiv b \mod n$.

\begin{enumerate}

  \item $(\forall a \in \mathbb{Z}) a \equiv a \mod n$

  \item $(\forall a,b \in \mathbb{Z}) a \equiv b \mod n
    \Leftrightarrow b \equiv a \mod n$

  \item $(\forall a,b,c \in \mathbb{Z}) a \equiv b \mod n
    \wedge b \equiv c \mod n \Rightarrow a \equiv c
    \mod n$

  \item
    If $a \equiv b \mod n$ and $c \equiv d (\mathrm{mod }
    n)$, then:
    \begin{itemize}
    \item $a + c \equiv b + d \mod n$
    \item $a - c \equiv b - d \mod n$
    \item $ac \equiv bd \mod n$
    \item $a^t \equiv b ^t \mod n$ for any $t \in
      \mathbb{N}$
    \end{itemize}

\end{enumerate}

Cancellation: If $ac \equiv bc \mod n$ then $a \equiv b
(\mathrm{mod } \frac{n}{d})$ where $d = gcd(n,c)$.



\subsubsection{Residue}\label{sec:residue}

\textbf{Complete Set of Residues}

Pigeon Hole Principle: A Set of Integers $\{a_1, a_2, \ldots, a_n\} =
T \subset \mathbb{Z}$ is a \emph{Complete Set of Residues} (CSR)
$\mod n$ if $|T| = n$ and no Elements of a particular Residue
$\mod n$ occur more than once.

For a CSR $\mod n$ $T = \{a_1, a_2, \ldots, a_n\}$ and an
Integer $c \in \mathbb{Z}$ Coprime with $n$ ($gcd(c,n) = 1$), for any
Integer $b \in \mathbb{Z}$, the Set $S = \{b + ca_1, b + ca_2, \ldots,
b + ca_n \}$ is also a CSR $\mod n$.



\textbf{Complete Set of Non-zero Residues}

$T = \{ a_1, a_2, \ldots, a_{n-1} \}$ Set of $n-1$ Integers, none are
$\equiv 0 \mod n$



\subsubsection{Linear Congruence}\label{sec:linear_congruence}

$ax \equiv b \mod n$ has solutions if and only if $d =
gcd(a,n) | b$. If $d|b$ then there are exactly $d$ solutions in a
given CSR $\mod n$.



\subsubsection{Fermat's Little Theorem}\label{sec:fermat_little}

$p \in \mathbb{P}$, $a^p \equiv a \mod n$

If $p \in \mathbb{P}$ and $a \in \mathbb{Z}$ and $p \nmid a$ then $a^{p-1}
\equiv 1 \mathbb{mod}\;p$.

For $p \in \mathbb{P}$ and $a \in \mathbb{Z}$, $\{ a, 2a, \ldots,
(p-1)a \}$ is a Non-zero CSR.



\subsubsection{Modular Order}\label{sec:modular_order}

The \emph{Order} of $a \in \mathbb{Z}$ relative to $n \in \mathbb{Z}$,
denoted $O_n(a)$ is the smallest Exponent $j$ such that $a^j \equiv 1
\mod n$.

$a^e \equiv 1 \mod n \Leftrightarrow O_n(a) | e$

For $a, n$ with $gcd(a,n) = 1$, $O_n(a) | \varphi(n)$

$O_n(a^i) = \frac{O_n(a)}{gcd(i,O_n(a))}$



\subsubsection{Primitive Root}\label{sec:primitive_root}

A \emph{Primitive Root} is an $a \in \mathbb{Z}$ such that $gcd(a,n) =
1$ and the Sequence $a, a^2, a^3, \ldots$ gives a Complete Set of
Remainders $\mod n$

For $gcd(a,n) = 1$, $a$ is a Primitive Root $\mod n$ if and only if
$O_n(a) = \varphi(n)$ (the Order \S\ref{sec:modular_order} of $a$ is
equal to the number of Relatively Prime Integers $< n$).

If $n$ has at least one Primitive Root then it has
$\varphi(\varphi(n))$ Primitive Roots defined by the Set:
\[
  \{ a^i | 1 \leq i \leq \varphi(n), gcd(i,\varphi(n)) = 1 \}
\]



% --------------------------------------------------------------------
\subsection{Heyting Arithmetic}\label{sec:heyting_arithmetic}
% --------------------------------------------------------------------

Peano Arithmetic with Intuitionistic Logic



% ====================================================================
\section{Recurrence Relation}\label{sec:recurrence_relation}
% ====================================================================

Recursive Definition (\S\ref{sec:recursive_definition})



% ====================================================================
\section{Analytic Number Theory}\label{sec:analytic_number_theory}
% ====================================================================

% --------------------------------------------------------------------
\subsection{Dirichlet L-Series}\label{sec:l_series}
% --------------------------------------------------------------------



% ====================================================================
\section{Algebraic Number Theory}\label{sec:algebraic_number_theory}
% ====================================================================

% ====================================================================
\section{Additive Number Theory}\label{sec:additive_number_theory}
% ====================================================================

% ====================================================================
\section{Computationsl Number Theory}\label{sec:computational_number_theory}
% ====================================================================

% --------------------------------------------------------------------
\subsection{Integer Factorization}\label{sec:integer_factorization}
% --------------------------------------------------------------------



% ====================================================================
\section{Probabilistic Number Theory}\label{sec:probabilistic_number_theory}
% ====================================================================

% ====================================================================
\section{Sieve Theory}\label{sec:sieve_theory}
% ====================================================================

% --------------------------------------------------------------------
\subsection{Sieve of Eratosthenes}\label{sec:sieve}
% --------------------------------------------------------------------



% ====================================================================
\section{Finitism}\label{sec:finitism}
% ====================================================================

% --------------------------------------------------------------------
\subsection{Strict Finitism}\label{sec:strict_finitism}
% --------------------------------------------------------------------

% --------------------------------------------------------------------
\subsection{Ultrafinitism}\label{sec:ultrafinitism}
% --------------------------------------------------------------------
