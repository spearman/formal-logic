%%%%%%%%%%%%%%%%%%%%%%%%%%%%%%%%%%%%%%%%%%%%%%%%%%%%%%%%%%%%%%%%%%%%%%
%%%%%%%%%%%%%%%%%%%%%%%%%%%%%%%%%%%%%%%%%%%%%%%%%%%%%%%%%%%%%%%%%%%%%%
\part{Number Theory}\label{sec:number_theory}
%%%%%%%%%%%%%%%%%%%%%%%%%%%%%%%%%%%%%%%%%%%%%%%%%%%%%%%%%%%%%%%%%%%%%%
%%%%%%%%%%%%%%%%%%%%%%%%%%%%%%%%%%%%%%%%%%%%%%%%%%%%%%%%%%%%%%%%%%%%%%



% ====================================================================
\section{Arithmetic}\label{sec:arithmetic}
% ====================================================================

% --------------------------------------------------------------------
\subsection{First-order Arithmetic}\label{sec:firstorder_arithmetic}
% --------------------------------------------------------------------

\emph{First-order Logic} (\S\ref{sec:predicate_logic})

% --------------------------------------------------------------------
\subsection{Second-order Arithmetic}\label{sec:second_order_arithmetic}
% --------------------------------------------------------------------

% --------------------------------------------------------------------
\subsection{Modular Arithmetic}\label{sec:modular_arithmetic}
% --------------------------------------------------------------------

\subsubsection{Modular Congruence}\label{sec:modular_congruence}

$a \equiv b (\mathrm{mod} n)$



% --------------------------------------------------------------------
\subsection{Peano Arithmetic}\label{sec:peano_arithmetic}
% --------------------------------------------------------------------

% --------------------------------------------------------------------
\subsection{Heyting Arithmetic}\label{sec:heyting_arithmetic}
% --------------------------------------------------------------------



% ====================================================================
\section{Natural Numbers}\label{sec:natural_number}
% ====================================================================

% --------------------------------------------------------------------
\subsection{Principle of Mathematical Induction}
\label{sec:induction_principle}
% --------------------------------------------------------------------



% ====================================================================
\section{Integers}\label{sec:integer}
% ====================================================================

% --------------------------------------------------------------------
\subsection{Prime Number}\label{sec:prime_number}
% --------------------------------------------------------------------

A \emph{Prime Power} is a Prime Number raised to the power of a
positive Integer.



% ====================================================================
\section{Rationals}\label{sec:rational}
% ====================================================================



% ====================================================================
\section{Real Numbers}\label{sec:real_number}
% ====================================================================

% --------------------------------------------------------------------
\subsection{Computable Numbers}\label{sec:computable_real}
% --------------------------------------------------------------------

also \emph{Recursive Numbers} or \emph{Computable Reals}

\emph{Real Closed Field} (\S\ref{sec:closed_field})



% ====================================================================
\section{Ordinal Number}\label{sec:ordinal_number}
% ====================================================================

% --------------------------------------------------------------------
\subsection{Admissible Number}\label{sec:admissible_ordinal}
% --------------------------------------------------------------------

% --------------------------------------------------------------------
\subsection{Von Neumann Ordinal}\label{sec:vonneumann_ordinal}
% --------------------------------------------------------------------

% --------------------------------------------------------------------
\subsection{Proof-theoretic Ordinal}\label{sec:proof_ordinal}
% --------------------------------------------------------------------



% ====================================================================
\section{Cardinal Number}\label{sec:cardinal_number}
% ====================================================================



% ====================================================================
\section{Infinitesimal}\label{sec:infinitesimal}
% ====================================================================



% ====================================================================
\section{Complex Number}\label{sec:complex_number}
% ====================================================================

% --------------------------------------------------------------------
\subsection{Gaussian Integer}\label{sec:gaussian_integer}
% --------------------------------------------------------------------



% ====================================================================
\section{Hyperreal}\label{sec:hyperreal}
% ====================================================================



% ====================================================================
\section{Analytic Number Theory}\label{sec:analytic_number_theory}
% ====================================================================
% --------------------------------------------------------------------
\subsection{Dirichlet L-Series}\label{sec:l_series}
% --------------------------------------------------------------------
