%%%%%%%%%%%%%%%%%%%%%%%%%%%%%%%%%%%%%%%%%%%%%%%%%%%%%%%%%%%%%%%%%%%%%%
\part{Category Theory}\label{sec:category_theory}\cite{awodey06}\cite{maclane69}
%%%%%%%%%%%%%%%%%%%%%%%%%%%%%%%%%%%%%%%%%%%%%%%%%%%%%%%%%%%%%%%%%%%%%%

% --------------------------------------------------------------------
\section{Meta-categories}\label{sec:meta_category}
% --------------------------------------------------------------------

The Formal System of Category Theory consists of
\begin{itemize}
\item two Sorts (\S\ref{subsec:manysorted_logic}), \emph{Objects}
  $A,B,C,\ldots$, and \emph{Arrows} (corresponding to Morphisms)
  $f,g,h,\ldots$
\item four Operations: $dom(f)$, $cod(f)$, $1_A$, $\circ$
\end{itemize}
and seven Axioms:
\[
    dom(1_A) = A,\quad cod(1_A) = A
\]\[
    f \circ 1_{dom(f)} = f, \quad 1_{cod(f)} \circ f = f
\]\[
    dom(g\circ f) = dom(f), \quad cod(g \circ f) = cod(g)
\]\[
    h \circ (g \circ f) = (h \circ g) \circ f
\]
Because the Axioms are Self-dual, if a Sentence can be derived from
them, so can its Dual.

% --------------------------------------------------------------------
\section{Categories}\label{sec:categories}
% --------------------------------------------------------------------

A \emph{Category} is an \emph{Algebraic Structure}
(\S\ref{subsec:universal_algebra}) where Functions are Morphisms
(\S\ref{sec:category_morphisms}) between Objects and each Object has
its own \emph{Identity Morphism} and Morphisms are \emph{Associative}.

A Category $\mathbf{C}$ is defined as
\begin{itemize}
\item a Class $C_0$ of Objects
\item a Class $C_1$ of Morphisms (also called \emph{Arrows} or
  \emph{Maps}) with Identity Morphisms for each Object
\item a Binary Associative Composition Operation for every three
  Objects
\end{itemize}
$\mathbf{C}$ is considered \emph{Small} (as opposed to \emph{Large})
if both $C_0$ and $C_1$ can be represented as Sets.

\emph{Locally Small}

\emph{Concrete Category}

Example Categories:
\begin{itemize}
\item Sets and Functions
\item Finite Sets and Injective Functions
\item Posets and Monotonic Functions
\item Categories and Functors
\item Formulae and Deductions
\item Functors and Natural Transformations
\item Types and Computable Functions
\end{itemize}
The Category of all Sets and Set Mappings is denoted $\mathbf{Set}$
(and is \emph{Dual} to the Category of Complete, Atomic Boolean
Algebras (\S\ref{subsec:boolean_algebra}).

Preorders and Posets (\S\ref{sec:order_theory}) can represent
Categories by taking Elements as Objects and Morphisms as existing
between Pairs of Elements in the Ordering Relation. The Category of
Posets is denoted $\mathbf{Pos}$. A special case of a Poset Category
is a \emph{Discrete Category} which is a Category of Objects with only
Identity Morphisms.

A \emph{Monoid} (\S\ref{subsec:monoids}) is a Category with a single
Object and Morphisms for each Element in the Monoid such that
Composition of the Morphisms is the Binary Operation of the
Monoid. The Category $\mathbf{Mon}$ is the Category of all Monoids and
Functions that preserve the Monoid structure.

\emph{Simplicial}

% --------------------------------------------------------------------
\subsection{List of Categories}\label{subsec:categories_list}
% --------------------------------------------------------------------

\begin{description}
\item [0] no Objects or Morphisms
\item [1] one Object with Identity Morphism
\item [2] two Objects with Identity Morphisms and one Morphism between
  them
\item [3] three Objects with Identity Morphisms in a Commutative
  Triangle
\item [$\downarrow\downarrow$] two Objects, no Identity morphisms, and
  two parallel Morphisms between them
\item [Set] all Small Sets and Functions
\item [Set$_*$] all Pointed Sets and Base-point preserving Functions
\item [Rel] all Sets and Relations
\item [Pos] all Posets and Monotonic Functions
\item [Mag] all Magmas and Homomorphisms of Operations
\item [Med] all Medial Magmas and Homomorphisms of Operations
\item [Mon] all Monoids and Monoid Homomorphisms
\item [Grp] all Groups and Group Homomorphisms
\item [Ab] all Abelian Groups with Group Homomorphisms
\item [Rng] all Small Rings and Ring Morphisms
\item [Cat] all Small Categories and Functors
\item [Top] all Topological Spaces and Continuous Maps
\end{description}

% --------------------------------------------------------------------
\subsection{Finite Categories}\label{subsec:finite_categories}
% --------------------------------------------------------------------

A \emph{Finite Category} has a Finite number of Morphisms and
Objects. An Infinite number of Objects Implies and Infinite number of
Morphisms because each Object has its own Identity Morphism. A
Category such as
\[
    A
    \begin{matrix}
    \xrightarrow{\;\;f\;\;}\\
    \xleftarrow[\;\;g\;\;]{}
    \end{matrix}
    B
\]
Induces an Infinite Category because new Morphisms exist wherever the
Codomain of one Morphism is the Domain of another: $gf, gfgf, gfgfgf,
\ldots$ and $fg, fgfg, fgfgfg, \ldots$. See \emph{Finitely Presented
  Categories} \S\ref{subsec:finitely_presented}.

% --------------------------------------------------------------------
\subsection{Opposite Categories}\label{subsec:opposite_categories}
% --------------------------------------------------------------------

The \emph{Opposite Category} (or \emph{Dual}) of a Category
$\mathbf{C}$ is denoted $\mathbf{C^{op}}$ and has the same Objects as
$\mathbf{C}$ but the Domain and Codomain in each Morphism is
reversed. Objects and Morphisms of a Dual Category may be written with
over-lines to distinguish them from the original Category:
$\overline{f}: \overline{C} \rightarrow \overline{D}$. With this
notation the following Equalities may be expressed:
\[
    1_{\overline{C}} = \overline{1_C}
\]\[
    \overline{f} \circ \overline{g} = \overline{g \circ f}
\]
A Terminal Object in $\mathbf{C}$ is an Initial Object in
$\mathbf{C^{op}}$ and vice-versa.

% --------------------------------------------------------------------
\subsection{Comma Cateories}\label{subsec:comma_categories}
% --------------------------------------------------------------------

A \emph{Comma Category} is formed from a pair of Functors
(\S\ref{sec:category_functor}) that share a common Codomain. For three
Categories, $\mathbf{A}$, $\mathbf{B}$, and $\mathbf{C}$ and Functors
$S$ (\emph{Source}) and $T$ (\emph{Target}) in the following relation:
\[
    \mathbf{A} \xrightarrow{\;\; S\;\;} \mathbf{C} \xleftarrow{\;\;
      T\;\;} \mathbf{B}
\]
one can form a Comma Category $(S \downarrow T)$ with Objects as
Triples $(\alpha, \beta, f)$ where $\alpha$ is an Object in
$\mathbf{A}$, $\beta$ is an Object in $\mathbf{B}$, and $f : S(\alpha)
\rightarrow T(\beta)$ is a Morphism in $\mathbf{C}$ and with Morphisms
between Triples $(\alpha, \beta, f)$ to $(\alpha', \beta', f')$ as
pairs $(g,h)$ where $g : \alpha \rightarrow \alpha'$ is a Morphism in
$\mathbf{A}$ and $h : \beta \rightarrow \beta'$ is a Morphism in
$\mathbf{B}$.

% --------------------------------------------------------------------
\subsubsection{Arrow Categories}\label{subsec:arrow_categories}
% --------------------------------------------------------------------

An \emph{Arrow Category} of a Category $\mathbf{C}$, written
$\mathbf{C^{\rightarrow}}$, has for its Objects the Morphisms of
$\mathbf{C}$ and as Morphisms pairs of Objects such that their
underlying Morphisms in $\mathbf{C}$ are Composable.

% --------------------------------------------------------------------
\subsubsection{Slice Categories}\label{subsec:slice_categories}
% --------------------------------------------------------------------

A \emph{Slice Category} $\mathbf{C}/C$ of a Category $\mathbf{C}$ with
an Object $C$ has as Objects the Morphisms with Codomain $C$ and as
Morphisms those Morphisms in $\mathbf{C}$ between the Domains of the
underlying Morphisms of the Objects of $\mathbf{C}/C$. That is, for
Objects in the Slice Category corresonding to Morphisms $f$ and $f'$,
the Morphism in the Slice Category between the two is $g$ such that
\[
    f' \circ g = f
\]

\emph{Coslice}

% --------------------------------------------------------------------
\subsection{Congruence Categories}\label{subsec:category_congruence}
% --------------------------------------------------------------------

\emph{Congruence} on a Category is an Equivalence Relation on
Morphisms such that for two Morphisms $f,g \in \mathbf{C_1}$, $f \sim
g$ Implies:
\begin{itemize}
\item $dom(f) = dom(g)$
\item $cod(f) = cod(g)$
\item $\forall a,b \in \mathbf{C_1}, bfa \sim bga$
\end{itemize}
Such a Congruence defines a \emph{Congruence Category}
$\mathbf{C^{\sim}}$:
\[
    (\mathbf{C^{\sim}})_0 = \mathbf{C_0}
\]\[
    (\mathbf{C^{\sim}})_1 = \{\langle f,g \rangle | f \sim g\}
\]\[
    \tilde{1_\mathbf{C}} = \langle 1_\mathbf{C}, 1_\mathbf{C} \rangle
\]\[
    \langle f',g' \rangle \circ \langle f,g \rangle = \langle f'f,g'g \rangle
\]

% --------------------------------------------------------------------
\subsubsection{Quotient Categories}\label{subsec:category_quotient}
% --------------------------------------------------------------------

The \emph{Quotient Category} is defined for a Category $\mathbf{C}$
with Congruence Relation $\sim$ as $\mathbf{C}/\sim$:
\[
    (\mathbf{C}/\sim)_0 = \mathbf{C_0}
\]\[
    (\mathbf{C}/\sim)_1 = (\mathbf{C_1})/\sim
\]
where Morphisms are of the form $[f]$ where $f \in \mathbf{C_1}$.

% --------------------------------------------------------------------
\subsubsection{Kernel Category}\label{subsec:kernel_category}
% --------------------------------------------------------------------

Given a Functor $F : \mathbf{C} \rightarrow \mathbf{D}$, a Congruence
$\sim_F$ on $\mathbf{C}$ is defined as
\[
    f \sim_F g \leftrightarrow dom(f) = dom(g) \wedge cod(f) = cod(g)
    \wedge F(f) = F(g)
\]
The \emph{Kernel Category} of $F$ is then defined as the Congruence
Category of $\sim_F$
\[
    ker(F) = C^{\sim_F}
\]

% --------------------------------------------------------------------
\subsection{Free Categories}\label{subsec:free_categories}
% --------------------------------------------------------------------

A \emph{Free Category} $\mathbf{C}(G)$ is Generated by a Directed
Graph (\S\ref{subsec:directed_graph}) $G$ of Vertices and Edges under
Concatenation of Paths. For a Graph with a single Vertex, the Free
Category is the Free Monoid on the Edges of the Graph. A Graph with no
Edges generates a Discrete Category (\S\ref{sec:categories}). The Free
Category on a Graph with two Vertices and one Edge is the Finite
Category $\mathbf{2}$.

% --------------------------------------------------------------------
\subsubsection{Finitely Presented Category}\label{subsec:finitely_presented}
% --------------------------------------------------------------------

A \emph{Finitely Presented Category} is given by taking the Quotient
Category of a Free Category $\mathbf{C}(G)$ with the Congruence
$\sim_\Sigma$
\[
    \mathbf{C}(G) / \sim_{\Sigma} = \mathbf{C}(G,\Sigma)
\]
where $\Sigma$ is the finite Set of Relations
\[
    (g_1 \circ \ldots \circ g_n) = (g'_1 \circ \ldots \circ g'_m)
\]
for all $g_i \in G$ such that $dom(g_n) = dom(g'_m)$ and $cod(g_1) =
cod(g'_1)$.


% --------------------------------------------------------------------
\subsection{Category Product}\label{subsec:category_product}
% --------------------------------------------------------------------

A \emph{Product} of two Categories $\mathbf{C}$ and $\mathbf{D}$ is
denoted
\[
    \mathbf{C} \times \mathbf{D}
\]
has Objects of the form $(C,D)$ where $C \in \mathbf{C}$ and $D \in
\mathbf{D}$ and Morphisms $(f,g) : (C,D) \rightarrow (C',D')$ where $f
: C \rightarrow C' \in \mathbf{C}$ and $g : D \rightarrow D' \in
\mathbf{D}$. Composition and Identity are defined as:
\[
    (f',g') \circ (f,g) = (f' \circ f,g' \circ g)
\]\[
    1_{(C,D)} = (1_C, 1_D)
\]
$\mathbf{C} \times \mathbf{D}$ is a Product
(\S\ref{subsec:product_diagram}) in $\mathbf{Cat}$.

% --------------------------------------------------------------------
\section{Morphisms}\label{sec:category_morphisms}
% --------------------------------------------------------------------



% --------------------------------------------------------------------
\subsection{Constant Morphism}\label{subsec:constant_morphism}
% --------------------------------------------------------------------

A Morphism $f : X \rightarrow Y$ is a \emph{Constant Morphism} if for
any $g, h : W \rightarrow X$, $fg = fh$.

A Morphism $f : X \rightarrow Y$ is a \emph{Co-constant Morphism} if
for any $g, h : Y \rightarrow Z$, $gf = hf$.

\emph{Zero Morphism} is one that is both Constant and Co-constant.

% --------------------------------------------------------------------
\subsection{Kernel}\label{subsec:morphism_kernel}
% --------------------------------------------------------------------

The \emph{Kernel} of a Morphism $f : X \rightarrow Y$ is the most
general Morphism $k : K \rightarrow X$ such that $fk = 0_{KY}$ and for
any Morphism $k' : K' \rightarrow X$ such that $fk' = 0_{K'Y}$, there
exists a unique Morphism $u : K' \rightarrow K$ such that $ku = k'$.

The collection of all Morphisms between two Objects $X$ and $Y$ in a
Category $\mathbf{C}$ is called the \emph{Hom-set} and is denoted
$Hom(X,Y)$:
\[
    Hom(A,B) = \{f \in \mathbf{C} | f : A \rightarrow B\}
\]
Therefore
\[
    Hom : \mathbf{C^{op}} \times \mathbf{C} \rightarrow \mathbf{Set}
\]
Note that the Hom-set need not be a Set (it could be a Proper Class).

% --------------------------------------------------------------------
\subsection{Homomorphism}\label{subsec:homomorphism}
% --------------------------------------------------------------------

A \emph{Homomorphism} is a Structure-preserving Morphism between
Algebraic Structures (\S\ref{subsec:universal_algebra}). That is, for
an Homomorphism $f : A \rightarrow B$ where $A$ and $B$ have Operators
$*$ and $*'$ respectively, for any $a_i \in A$
\[
    f(a_1 * a_2) = f(a_1) *' f(a_2)
\]
Note that the Operators do not have to be the same.

\subsubsection{Monomorphism}

A \emph{Monomorphism} is a Homomorphism that is Injective, a
sufficient condition of which is that the Morphism has a Left Inverse
(\S\ref{subsec:inverse_functions}) or \emph{Retraction}. A
Monomorphism with a Left Inverse is called a \emph{Split
  Monomorphism}. A Monomorphism $f$ between Objects $A$ and $B$ is
denoted
\[
    f : A \rightarrowtail B
\]
and has the property for any two Morphisms $g, h : C \rightarrow A$,
$fg = fh$ Implies $g = h$.

\subsubsection{Epimorphism}

An \emph{Epimorphism} is a Surjective Homomorphism, a sufficient
condition of which is that the Morphism has a Right Inverse
(\S\ref{subsec:inverse_functions}) or \emph{Section}. An Epimorphism
with a Right Inverse is called a \emph{Split Epimorphism}. An
Epimorphism $f$ between Objects $A$ and $B$ is denoted
\[
    f : A \twoheadrightarrow B
\]
and has the property that for any two Morphisms $g, h : B \rightarrow
C$, $gf = hf$ Implies $g = h$.

An Object $P$ is \emph{Projective} if for any Epimorphism $e : E
\rightarrow X$ and Morphism $f : P \rightarrow X$, then $\exists
\overline{f} : P \rightarrow E$ such that $e \circ \overline{f} = f$.

\subsubsection{Isomorphism}\label{subsec:isomorphism}

A Morphism $f$ is an \emph{Isomorphism} if there exists another
Morphism $g$ such that $g = f^{-1}$. The Existance of an Isomorphism
between two Objects $A$ and $B$ is denoted $A \cong B$. An Isomorphism
is a Homomorphism that is both a Monomorphism and an Epimorphism. Note
that a Mono- and Epi-morphism need not be an Isomorphism.

\subsubsection{Endomorphism}

An \emph{Endomorphism} is a Homomorphism from an Object to itself.

\subsubsection{Automorphism}\label{subsec:automorphism}

An \emph{Automorphism} is both an Endomorphism and an Isomorphism,
that is, an Invertible Endomorphism.

% --------------------------------------------------------------------
\subsection{Factors}\label{subsec:morphism_factor}
% --------------------------------------------------------------------

% --------------------------------------------------------------------
\subsection{Subobjects}\label{subsec:category_subobjects}
% --------------------------------------------------------------------

A \emph{Subobject} of an Object $X$ in $\mathbf{C}$ is a Monomorphism
into $X$:
\[
    m : M \rightarrowtail X
\]
A Morphism between two Subobjects is a Morphism in the Quotient
Category $\mathbf{C}/X$ giving the Category of Subobjects of $X$ in
$\mathbf{C}$ as $Sub_{\mathbf{C}}(X)$. Because there is at most one
Morphism between Subobjects, $Sub_{\mathbf{C}}(X)$ is a Preorder
Category.

% --------------------------------------------------------------------
\section{Functor}\label{sec:category_functor}
% --------------------------------------------------------------------

A \emph{Functor} is a \emph{Homomorphism}
(\S\ref{subsec:homomorphism}) of Categories. A Functor $F$ between
Categories $\mathbf{C}$ and $\mathbf{D}$
\[
    F : \mathbf{C} \rightarrow \mathbf{D}
\]
is a pair of Maps for Objects and Morphisms of $\mathbf{C}$ to Objects
and Morphisms of $\mathbf{D}$ with the following Equivalences:
\begin{itemize}
\item $F(f : A \rightarrow B) = F(f) : F(A) \rightarrow F(B)$
\item $F(g \circ f) = F(g) \circ F(f)$
\item $F(1_A) = 1_{F(A)}$
\end{itemize}
Every Category has an Identity Functor $1_{\mathbf{C}} : \mathbf{C}
\rightarrow \mathbf{C}$ and the Category of all Small Categories and
Functors is denoted $\mathbf{Cat}$.

Functors between Poset Categories are Monotonic Functions
(\S\ref{subsec:monotonicity}).

A \emph{Forgetful Functor} is one that drops some Property of the
Input Category in the Output Category.

\emph{Faithful Functor}

\emph{Hom Functor}

\emph{Bifunctor} \emph{Multifunctor}

\emph{Product Functor}

% --------------------------------------------------------------------
\subsection{Contravariant Functor}\label{subsec:contravariant_functor}
% --------------------------------------------------------------------

A \emph{Contravariant Functor} is from a Dual Category to another
Category, eg $F : \mathbf{C^{op}} \rightarrow \mathbf{D}$ while the
Dual of a Contravariant Functor, a \emph{Covariant Functor}, is just
an ordinary Functor.

% --------------------------------------------------------------------
\subsection{Representable Functor}\label{subsec:representable_functor}
% --------------------------------------------------------------------

A \emph{Covariant Representable Functor} for an Object $A$ in a
Category $\mathbf{C}$ is defined as a Functor $h^A = Hom(A,-) :
\mathbf{C} \rightarrow \mathbf{Set}$
\[
    Hom(A,-) : Hom(A,X) \xrightarrow{f_*} Hom(A,Y)
\]
A \emph{Contravariant Representable Functor} for $A$ is a Functor $h_A
= Hom(-,A) : \mathbf{C^{op}} \rightarrow \mathbf{Set}$
\[
    Hom(-,A) : Hom(X,A) \xrightarrow{f^*} Hom(Y,A)
\]

% --------------------------------------------------------------------
\subsection{Natural Transformation}\label{subsec:natural_transformation}
% --------------------------------------------------------------------

% --------------------------------------------------------------------
\subsection{Adjunction}\label{subsec:adjunction}
% --------------------------------------------------------------------

An \emph{Adjoint Functor} is the Categorical analog of the Existential
Quantifier in Logic (\S\ref{subsec:firstorder_quantification}) and the
\emph{Image Operation} along a \emph{Continuous Function} in
\emph{Topology} (Part \ref{sec:topology}).

% --------------------------------------------------------------------
\section{Constructions}\label{sec:category_construction}
% --------------------------------------------------------------------

% --------------------------------------------------------------------
\subsection{Diagrams}\label{subsec:category_diagram}
% --------------------------------------------------------------------

A \emph{Diagram} consists of Objects and Morphisms that are Indexed by
a Category or equivalently a \emph{Functor} from an Index Category to
another Category. A Diagram is the Category Theory analogue of an
Indexed Family of Sets (\S\ref{subsec:index_set}).

A Diagram, $D$, with Index Category $\mathbf{J}$ in Category
$\mathbf{C}$:
\[
    D : \mathbf{J} \rightarrow \mathbf{C}
\]

% --------------------------------------------------------------------
\subsubsection{Commutative Diagram}\label{subsec:commutative_diagram}
% --------------------------------------------------------------------

% --------------------------------------------------------------------
\subsubsection{Cones}\label{subsec:category_cone}
% --------------------------------------------------------------------

\emph{Universal Cone}

% --------------------------------------------------------------------
\subsection{Universal Properties}\label{subsec:universal_property}
% --------------------------------------------------------------------

There are two \emph{Universal Properties} of Objects: \emph{Initial}
and \emph{Final}.

An Object $0$ in a Category $\mathbf{C}$ is Initial if for every other
Object $A$ in the Category there is a unique Morphism $0 \rightarrow
A$.

An Object $1$ in a Category $\mathbf{C}$ is Terminal if for every
other Object $A$ in the Category there is a unique Morphism $A
\rightarrow 1$.

An Object that is both an Initial and a Terminal Object is called a
\emph{Zero Object} (or \emph{Null Object}). A Category with a Zero
Object is called a \emph{Pointed Category}.

It follows for any pair of Objects that are both Terminal and/or both
Initial that an Isomorphism (\S\ref{subsec:isomorphism}) exists
between them; that is, Universal Objects are always unique up to
Isomorphism.

As an example, in $\mathbf{Set}$, all Singleton Sets are Terminal, and
as such they are all Isomorphic to one another. Also in $\mathbf{Set}$
the Empty Set is Initial as the only mapping from it to any other Set
is the Empty Function. In $\mathbf{Grp}$, a Trivial Group ${1}$ is a
Zero Object.

A \emph{Global Element} (also \emph{Point} or \emph{Constant}) of an
Object is a Morphism from a Terminal Object to that Object. A
\emph{General Element} (also \emph{Variable}) is a Morphism from an
arbitrary Domain Object $X$.

Given a Functor $U: \mathbf{D} \rightarrow \mathbf{C}$, an
\emph{Initial Morphism} is an Initial Object in the Comma Category $(X
\downarrow U)$ where $X \in \mathbf{C_0}$.

\emph{Terminal Morphism}

\emph{Initial Property} \emph{Terminal Property}

\emph{Universal Morphism}

\subsubsection{Universal Mapping Property}
\label{subsec:universal_mapping_property}

\emph{Existence}

\emph{Uniqueness}

% --------------------------------------------------------------------
\subsection{Limits \& Colimits}\label{subsec:category_limits}
% --------------------------------------------------------------------


% --------------------------------------------------------------------
\subsubsection{Products \& Coproducts}\label{subsec:product_diagram}
% --------------------------------------------------------------------

A \emph{Product} of two Objects $A \times B$:
\[
    A \xleftarrow{\;\;p_1\;\;} P \xrightarrow{\;\;p_2\;\;} B
\]
is a Product of $A$ and $B$ if and only if for any $A
\xleftarrow{\;\;z_1\;\;} Z \xrightarrow{\;\;z_2\;\;} B$
\[
    \exists!u : Z \rightarrow P
\]
with $p_i \circ u = z_i$. $u$ may also be written as $\langle z_1, z_2
\rangle$ as it is uniquely determined by $z_1$ and $z_2$.

Products are unique up to Isomorphism (\S\ref{subsec:isomorphism}).

The Diagram $A \xrightarrow{\;\;q_1\;\;} Q \xleftarrow{\;\;q_2\;\;} B$
is a \emph{Coproduct} $A + B$ if for any $A \xrightarrow{\;\;z_1\;\;} Z
\xleftarrow{\;\;z_2\;\;} B$
\[
    \exists!u : Q \rightarrow Z
\]
with $u \circ q_i = z_i$. $u$ may also be written as $[ z_1, z_2 ]$
and Coprojections $q_i$ may be called \emph{Injections} (although they
are not necessarily Injective Morphisms).

Disjoint Union %FIXME ref disjoint union

\emph{N-ary Products}

% --------------------------------------------------------------------
\paragraph{Exponentials}\label{subsec:category_exponential}
% --------------------------------------------------------------------

% --------------------------------------------------------------------
\subsubsection{Pushouts \& Pullbacks}\label{subsec:category_pullback}
% --------------------------------------------------------------------

% --------------------------------------------------------------------
\subsection{Cartesian Closed Categories}\label{subsec:cartesian_category}
% --------------------------------------------------------------------

% --------------------------------------------------------------------
\section{Group Theory}
% --------------------------------------------------------------------

A \emph{Group} is a Monoid with an Inverse for every Morphism (every
Morphism is an Isomorphism). A Group may be formed from a Set $X$ by
adding all Automorphisms (\S\ref{subsec:automorphism}) of $X$:
$Aut(X)$. \emph{Cayley's Theorem} states that every Group is
Isomorphic to a Group of Permutations (\S\ref{subsec:permutations}).

A Group $G$ within a Category $\mathbf{C}$ may be viewed as a Subset
of the Hom-set of an Object $X$:
\[
    G \subseteq Hom_{\mathbf{C}}(X,X)
\]

\emph{Trivial Group} ${1}$

% --------------------------------------------------------------------
\subsection{Group Homomorphism}\label{subsec:group_homomorphism}
% --------------------------------------------------------------------

A \emph{Group Homomorphism} preserves Group operations.

Taken as Monoidal Categories, two Groups $G, H$ may be related by a
Functor $f$ which is equivalent to a Group Homomorphism:
\[
    f : G \rightarrow H
\]

% --------------------------------------------------------------------
\subsection{Group Representations}\label{subsec:group_representation}
% --------------------------------------------------------------------

Given a Functor $R$ from a Group $G$ to a general Category
$\mathbf{C}$
\[
    R : G \rightarrow \mathbf{C}
\]
Such a Functor $R$ is termed a \emph{Representation} of $G$ in
$\mathbf{C}$.

% --------------------------------------------------------------------
\subsection{Coset}\label{subsec:group_coset}
% --------------------------------------------------------------------

The \emph{Coset} of a Subgroup (\S\ref{subsec:subgroup}) $H$ in a Group $G$
is defined as
\begin{description}
\item[Left Coset] $gH = {gh : h \in H}$
\item[Right Coset] $Hg = {hg : h \in H}$
\end{description}

% --------------------------------------------------------------------
\subsection{Subgroup}\label{subsec:subgroup}
% --------------------------------------------------------------------

A \emph{Subgroup} is a Subset of Group Elements that are still a Group
under the Group Operation.

Given a Group $G$ and Subgroup $H$, $H$ is a \emph{Normal Subgroup} if
and only if $\forall g \in G, gH = Hg$.

% --------------------------------------------------------------------
\subsection{Group Kernel}\label{subsec:group_kernel}
% --------------------------------------------------------------------

The \emph{Kernel} of a Group Homomorphism, $h : G \rightarrow H$,
denoted by $ker(h)$ is defined as:
\[
    ker(h) = {g \in G : f(g) = e_H}
\]
where $e_H$ is an Element of $H$, and so the Kernel is just the
Preimage of the Singleton Set $\{e_H\}$. Such a Kernel is a Normal
Subgroup.

% --------------------------------------------------------------------
\subsection{Monoids}\label{subsec:monoids}
% --------------------------------------------------------------------

The set of all Endomorphisms of an Object $X$ in a Category $C$ is
defined a Monoid and is denoted $End_C(X)$.

\emph{Free Monoid}

% --------------------------------------------------------------------
\section{Abstract Algebra}\label{sec:abstract_algebra}
% --------------------------------------------------------------------
\emph{Presentations}: Generators, Relations

\emph{Finitely Presented}
% --------------------------------------------------------------------
\subsection{Abstract Groups}\label{subsec:abstract_groups}
% --------------------------------------------------------------------
% --------------------------------------------------------------------
\subsection{Operad Theory}\label{subsec:operad_theory}
% --------------------------------------------------------------------
% --------------------------------------------------------------------
\subsection{Rings}\label{subsec:rings}
% --------------------------------------------------------------------
\emph{Ideals}

\emph{Principal Ideals}

\subsubsection{Fields}

\paragraph{Total Field}\label{subsec:total_field}

\paragraph{Closed Field}\label{subsec:closed_field}

% --------------------------------------------------------------------
\subsection{Initial Algebra}\label{subsec:initial_algebra}
% --------------------------------------------------------------------

% --------------------------------------------------------------------
\subsection{Algebraic Geometry}\label{subsec:algebraic_geometry}
% --------------------------------------------------------------------

\emph{Hilbert's Nullstellensatz}

\emph{Cauchy's Inequality}:
\[
    |\langle x,y \rangle|^2 \leq \langle x,x \rangle \cdot \langle
    y,y \rangle
\]

\subsubsection{Universal Algebraic Geometry}

% --------------------------------------------------------------------
\subsection{Heyting Algebra}\label{subsec:heyting_algebra}
% --------------------------------------------------------------------

\subsubsection{Boolean Algebra}\label{subsec:boolean_algebra}

% --------------------------------------------------------------------
\subsection{Relation Algebra}
% --------------------------------------------------------------------
% --------------------------------------------------------------------
\subsection{Relational Algebra}
% --------------------------------------------------------------------
\emph{Domain Relational Calculus}
% --------------------------------------------------------------------
\subsection{Quantale}
% --------------------------------------------------------------------

% --------------------------------------------------------------------
\section{Institution Theory}\label{subsec:institution_theory}
% --------------------------------------------------------------------
% --------------------------------------------------------------------
\section{Representation Theory}
% --------------------------------------------------------------------
