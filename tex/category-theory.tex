%%%%%%%%%%%%%%%%%%%%%%%%%%%%%%%%%%%%%%%%%%%%%%%%%%%%%%%%%%%%%%%%%%%%%%
\part{Category Theory}\label{sec:category_theory}\cite{awodey06}\cite{maclane69}
%%%%%%%%%%%%%%%%%%%%%%%%%%%%%%%%%%%%%%%%%%%%%%%%%%%%%%%%%%%%%%%%%%%%%%

% ====================================================================
\section{Metacategories}\label{sec:metacategory}
% ====================================================================

The Formal System of Category Theory consists of
\begin{itemize}
\item two Sorts (\S\ref{subsec:manysorted_logic}), \emph{Objects}
  $A,B,C,\ldots$, and \emph{Arrows} (corresponding to Morphisms)
  $f,g,h,\ldots$
\item four Operations: $dom(f)$, $cod(f)$, $1_A$, $\circ$
\end{itemize}
and seven Axioms:
\[
    dom(1_A) = A,\quad cod(1_A) = A
\]\[
    f \circ 1_{dom(f)} = f, \quad 1_{cod(f)} \circ f = f
\]\[
    dom(g\circ f) = dom(f), \quad cod(g \circ f) = cod(g)
\]\[
    h \circ (g \circ f) = (h \circ g) \circ f
\]



% --------------------------------------------------------------------
\subsection{Abstract Category}\label{subsec:abstract_category}
% --------------------------------------------------------------------

The \emph{Elementary Theory of Abstract Categories} refers to the
First-order Formulas (\S\ref{subsec:predicate_logic}) of the Language
of Metacategories.

A Sentence, $\Sigma$, in the Elementary Theory of Abstract Categories
can be made into a Dual Sentence, $\Sigma^*$, by replacing instances
of $cod$ by $dom$ and $dom$ by $cod$, and $h = g \circ f$ with $h = f
\circ g$. The Double-dual is Idempotent (\S\ref{subsec:idempotence}):
$\Sigma^** = \Sigma$.

Because the Axioms are Self-dual, if a Sentence can be derived from
them, so can its Dual by the \emph{Duality Principle}
(\S\ref{subsec:duality_principle}). \emph{Functors}
(\S\ref{sec:functor}) are Self-dual, and so they are not reversed as
are Morphisms in Categories.



% --------------------------------------------------------------------
\subsection{Precategory}\label{subsec:precategory}
% --------------------------------------------------------------------

A \emph{Precategory} (or \emph{Diagram Scheme}), $G$, is a Directed
Graph (\S\ref{subsec:directed_graph}) in the context of Category
Theory. A Precategory is used to generate a \emph{Free Category}
(\S\ref{subsec:free_category}). Here we will use the notation for
\emph{Categories} (\S\ref{sec:category}) to equate Vertices with
Objects $G_0$ and Edges with Morphisms $G_1$. A Precategory can then
be described as a pair of Functions, $cod$ and $dom$, from the Set of
Edges to the Set of Objects:
\[
    G = \{G_1 \rightrightarrows G_0\}
\]

A Morphism $D : G \rightarrow G'$ between Precategories is a pair of
Functions:
\[
    D_0 : G_0 \rightarrow G'_0, \;\;\; D_1 : G_1 \rightarrow G'_1
\]
such that $\forall f \in G_1$:
\[
    D_0(dom(f)) = dom(D_1(f))
\]\[
    D_0(cod(f)) = cod(D_1(f))
\]
Given two Graphs, $A$ and $B$, with the same Set of Objects, $O$, the
Product $\times_O$ over $O$ is defined as:
\[
    A \times_O B = \{ (g,f) | dom(g) = cod(f), g \in A, f \in B \}
\]
the set of Composable Morphisms.

A Category can be seen as a Graph, $G$, with two Morphisms:
\begin{enumerate}
    \item Mapping Composable Morphisms to their Composite:
        \[c : G \times_{G_0} G \rightarrow G\]
    \item Mapping Objects to their Identity Functions:
        \[i: G_0 \rightarrow G_1\]
\end{enumerate}
The \emph{Underlying Graph} of a Category, $\mathbf{C}$, is denoted
$U(\mathbf{C})$ where $U$ is the \emph{Underlying Graph Functor}
(\S\ref{sec:functor}).

For a Morphism of Graphs, $D : G \rightarrow U(\mathbf{B})$, there is
a corresponding Functor of Categories: $D' : \mathbf{C}_G \rightarrow
\mathbf{B}$.

The Category of all Small Directed Graphs and Graph Morphisms is
$\mathbf{Grph}$.

The Forgetful Functor $U : \mathbf{Cat} \rightarrow \mathbf{Grph}$
Maps a Categories to their underlying Graphs, effectively forgetting
which Morphisms are Identities and which are Composites.



% ====================================================================
\section{Category}\label{sec:category}
% ====================================================================

A \emph{Category} is an \emph{Algebraic Structure}
(\S\ref{subsec:universal_algebra}) where Functions are Morphisms
(\S\ref{sec:morphism}) between Objects and each Object has
its own \emph{Identity Morphism} and Morphisms are \emph{Associative}.

A Category $\mathbf{C}$ is defined as
\begin{itemize}
\item a Class $C_0$ of Objects
\item a Class $C_1$ of Morphisms (also called \emph{Arrows} or
  \emph{Maps}) with Identity Morphisms for each Object
\item a Binary Associative Composition Operation for every three
  Objects
\end{itemize}
$\mathbf{C}$ is considered \emph{Small} (as opposed to \emph{Large})
if both $C_0$ and $C_1$ can be represented as Sets. In the context of
Category Theory, a \emph{Small Set} (not to be confused with a Small
Category) is a member of a fixed \emph{Universe}
(\S\ref{sec:set_universe}) of Sets.

An equivalent definition of Categories can be given in terms of
\emph{Hom-sets} (\S\ref{subsec:hom_set}). A \emph{Locally Small
Category} is one for which all Hom-sets are themselves Sets and not
  Proper Classes.
% FIXME definition in terms of hom-sets

A \emph{Concrete Category} is pair, $(\mathbf{C},U)$, where
$\mathbf{C}$ is a Category and $U$ is a Faithful Functor $U :
\mathbf{C} \rightarrow \mathbf{Set}$.

Example Categories:
\begin{itemize}
\item Sets and Functions
\item Finite Sets and Injective Functions
\item Posets and Monotonic Functions
\item Categories and Functors
\item Formulae and Deductions
\item Functors and Natural Transformations
\item Types and Computable Functions
\end{itemize}
The Category of all Sets and Set Mappings is denoted $\mathbf{Set}$
(and is \emph{Dual} to the Category of Complete, Atomic Boolean
Algebras (\S\ref{subsec:boolean_algebra}).

Preorders and Posets (\S\ref{sec:order_theory}) can represent
Categories by taking Elements as Objects and Morphisms as existing
between Pairs of Elements in the Ordering Relation. The Category of
Posets is denoted $\mathbf{Pos}$. A special case of a Poset Category
is a \emph{Discrete Category} which is a Category of Objects with only
Identity Morphisms.

A \emph{Monoid} (\S\ref{subsec:monoid}) is a Category with a single
Object and Morphisms for each Element in the Monoid such that
Composition of the Morphisms is the Binary Operation of the
Monoid. The Category $\mathbf{Mon}$ is the Category of all Monoids and
Functions that preserve the Monoid structure.

\emph{Simplicial}



% --------------------------------------------------------------------
\subsection{List of Categories}\label{subsec:categories_list}
% --------------------------------------------------------------------

\begin{description}
\item [0] no Objects or Morphisms
\item [1] one Object with Identity Morphism
\item [2] two Objects with Identity Morphisms and one Morphism between
  them
\item [3] three Objects with Identity Morphisms in a Commutative
  Triangle
\item [$\downarrow\downarrow$] two Objects, no Identity morphisms, and
  two parallel Morphisms between them
\item [Set] all Small Sets and Functions
\item [Cls] all Classes and Functions between Classes
\item [Set$_*$] all Pointed Sets and Base-point preserving Functions
\item [Rel] all Sets and Relations
\item [Pos] all Posets and Monotonic Functions
\item [Finset] all Finite Sets and Functions (Full Subcategory of
  $\mathbf{Set}$)
\item [Finord] all Finite Ordinal Numbers and Functions between
  Ordinal Numbers (Skeleton of $\mathbf{FinSet}$)
\item [Mag] all Magmas and Homomorphisms of Operations
\item [Med] all Medial Magmas and Homomorphisms of Operations
\item [Mon] all Monoids and Monoid Homomorphisms
\item [Grp] all Groups and Group Homomorphisms
\item [Ab] all Abelian Groups with Group Homomorphisms
\item [Rng] all Small Rings and Ring Morphisms
\item [Grph] all Small Graphs and Graph Morphisms
\item [Cat] all Small Categories and Functors
\item [Cat'] all Large Categories and Functors
\item [Top] all Topological Spaces and Continuous Maps
\item [Top$_*$] all Pointed Topological Spaces and Continuous Maps
\item [Toph] all Topological Spaces and Homotopy Classes of Continuous
  Maps
\end{description}



% --------------------------------------------------------------------
\subsection{Category Equivalence}\label{subsec:category_equivalence}
% --------------------------------------------------------------------

\emph{Equivalence} between two Categories, $\mathbf{C}$ and
$\mathbf{D}$, is defined as a Pair of \emph{Functors}
(\S\ref{sec:functor}):
\[
    S : \mathbf{C} \rightarrow \mathbf{D}
\]\[
    T : \mathbf{D} \rightarrow \mathbf{C}
\]
and \emph{Natural Isomorphisms} (\S\ref{sec:natural_transformation}):
\[
    I_\mathbf{C} \cong T \circ S
\]\[
    I_\mathbf{D} \cong S \circ T
\]



% --------------------------------------------------------------------
\subsection{Finite Categories}\label{subsec:finite_categories}
% --------------------------------------------------------------------

A \emph{Finite Category} has a Finite number of Morphisms and
Objects. An Infinite number of Objects Implies and Infinite number of
Morphisms because each Object has its own Identity Morphism. A
Category such as
\[
    A
    \begin{matrix}
    \xrightarrow{\;\;f\;\;}\\
    \xleftarrow[\;\;g\;\;]{}
    \end{matrix}
    B
\]
Induces an Infinite Category because new Morphisms exist wherever the
Codomain of one Morphism is the Domain of another: $gf, gfgf, gfgfgf,
\ldots$ and $fg, fgfg, fgfgfg, \ldots$. See \emph{Finitely Presented
  Categories} \S\ref{subsec:finitely_presented}.



% --------------------------------------------------------------------
\subsection{Subcategory}\label{subsec:subcategory}
% --------------------------------------------------------------------

A \emph{Subcategory}, $\mathbf{S}$, of a Category, $\mathbf{C}$, is a
Category whose Objects and Morphisms are in $\mathbf{C}$. There is a
\emph{Faithful Functor} (\S\ref{subsec:faithful_functor}) called the
\emph{Inclusion Functor} which Maps Objects and Morphisms to
themselves:
\[
    I : \mathbf{S} \rightarrow \mathbf{C}
\]
If $I$ is a \emph{Full Functor}, then $S$ is a \emph{Full
  Subcategory}.



% --------------------------------------------------------------------
\subsection{Hom-set}\label{subsec:hom_set}
% --------------------------------------------------------------------

The collection of all Morphisms between two Objects $X$ and $Y$ in a
Category $\mathbf{C}$ is called the \emph{Hom-set} and is denoted
$Hom(X,Y)$:
\[
    Hom(X,Y) = \{f \in \mathbf{C} | f : X \rightarrow Y\}
\]
Therefore
\[
    Hom : \mathbf{C^{op}} \times \mathbf{C} \rightarrow \mathbf{Set}
\]
Note that the Hom-set need not be a Set (it could be a Proper Class).
A Category for which all Hom-sets are Sets is called a \emph{Locally
  Small Category}.



Some Properties of Hom-sets:
\begin{itemize}
\item $ (X,Y) \neq (X',Y') \rightarrow
  Hom(X,Y) \cap Hom(X',Y') = \varnothing$
\end{itemize}

A definition of Categories can be given in terms of Hom-sets. %FIXME



% ====================================================================
\section{Morphism}\label{sec:morphism}
% ====================================================================

% --------------------------------------------------------------------
\subsection{Homomorphism}\label{subsec:homomorphism}
% --------------------------------------------------------------------

A \emph{Homomorphism} is a Structure-preserving Morphism between
Algebraic Structures (\S\ref{subsec:universal_algebra}). That is, for
an Homomorphism $f : A \rightarrow B$ where $A$ and $B$ have Operators
$*$ and $*'$ respectively, for any $a_i \in A$
\[
    f(a_1 * a_2) = f(a_1) *' f(a_2)
\]
Note that the Operators do not have to be the same.



\subsubsection{Monomorphism}\label{subsec:monomorphism}

A \emph{Monomorphism} is a Homomorphism that is Injective, a
sufficient condition of which is that the Morphism has a Left Inverse
(\S\ref{subsec:inverse_functions}) or \emph{Retraction}. A
Monomorphism with a Left Inverse is called a \emph{Split Monomorphism}
and the Retraction a Split Epimorphism. A Monomorphism $f$ between
Objects $A$ and $B$ is denoted
\[
    f : A \rightarrowtail B
\]
and has the property for any two Morphisms $g, h : C \rightarrow A$,
$fg = fh$ Implies $g = h$. If $f$ is a Monomorphism then $gf$ and $hf$
are Idempotent (\S\ref{subsec:idempotence}).



\subsubsection{Epimorphism}\label{subsec:epimorphism}

A Homomorphism whose underlying function is Surjective is an
\emph{Epimorphism}, is a sufficient condition of which is that the
Morphism has a Right Inverse (\S\ref{subsec:inverse_functions}) or
\emph{Section}. An Epimorphism with a Right Inverse is called a
\emph{Split Epimorphism} and the Section a Split Monomorphism. An
Epimorphism $f$ between Objects $A$ and $B$ is denoted
\[
    f : A \twoheadrightarrow B
\]
and has the property that for any two Morphisms $g, h : B \rightarrow
C$, $gf = hf$ Implies $g = h$. Note that being an Epimorphism doesn't
necessitate the Morphism to be Surjective. If $f$ is an Epimorphism,
then $fg$ and $fh$ are Idempotent (\S\ref{subsec:idempotence}).

An Object $P$ is \emph{Projective} if for any Epimorphism $e : E
\rightarrow X$ and Morphism $f : P \rightarrow X$, then $\exists
\overline{f} : P \rightarrow E$ such that $e \circ \overline{f} = f$.



\subsubsection{Isomorphism}\label{subsec:isomorphism}

A Morphism $f$ is an \emph{Isomorphism} if there exists another
Morphism $g$ such that $g = f^{-1}$. The Existance of an Isomorphism
between two Objects $A$ and $B$ is denoted $A \cong B$. An Isomorphism
is a Homomorphism that is both a Monomorphism and an Epimorphism. Note
that a Mono- and Epi-morphism need not be an Isomorphism.

The Existence of an \emph{Invertible Morphism}
(\S\ref{subsec:bijective_function}) between Objects also Implies their
Isomorphism. A Category in which every Morphism is Invertible is a
\emph{Groupoid} (\S\ref{subsec:groupoid}).



\subsubsection{Endomorphism}

An \emph{Endomorphism} is a Homomorphism from an Object to itself.



\subsubsection{Automorphism}\label{subsec:automorphism}

An \emph{Automorphism} is both an Endomorphism and an Isomorphism,
that is, an Invertible Endomorphism.



% --------------------------------------------------------------------
\subsection{Constant Morphism}\label{subsec:constant_morphism}
% --------------------------------------------------------------------

A Morphism $f : X \rightarrow Y$ is a \emph{Constant Morphism} if for
any $g, h : W \rightarrow X$, $fg = fh$.

A Morphism $f : X \rightarrow Y$ is a \emph{Co-constant Morphism} if
for any $g, h : Y \rightarrow Z$, $gf = hf$.

\emph{Zero Morphism} is one that is both Constant and Co-constant.



% --------------------------------------------------------------------
\subsection{Kernel}\label{subsec:morphism_kernel}
% --------------------------------------------------------------------

The \emph{Kernel} of a Morphism $f : X \rightarrow Y$ is the most
general Morphism $k : K \rightarrow X$ such that $fk = 0_{KY}$ and for
any Morphism $k' : K' \rightarrow X$ such that $fk' = 0_{K'Y}$, there
exists a unique Morphism $u : K' \rightarrow K$ such that $ku = k'$.



% --------------------------------------------------------------------
\subsection{Factors}\label{subsec:morphism_factor}
% --------------------------------------------------------------------

% --------------------------------------------------------------------
\subsection{Subobjects}\label{subsec:category_subobjects}
% --------------------------------------------------------------------

A \emph{Subobject} of an Object $X$ in $\mathbf{C}$ is a Monomorphism
into $X$:
\[
    m : M \rightarrowtail X
\]
A Morphism between two Subobjects is a Morphism in the Quotient
Category $\mathbf{C}/X$ giving the Category of Subobjects of $X$ in
$\mathbf{C}$ as $Sub_{\mathbf{C}}(X)$. Because there is at most one
Morphism between Subobjects, $Sub_{\mathbf{C}}(X)$ is a Preorder
Category.



% ====================================================================
\section{Functor}\label{sec:functor}
% ====================================================================

A \emph{Functor} is a \emph{Homomorphism}
(\S\ref{subsec:homomorphism}) of Categories. A Functor $F$ between
Categories $\mathbf{C}$ and $\mathbf{D}$
\[
    F : \mathbf{C} \rightarrow \mathbf{D}
\]
is a pair of Maps for Objects and Morphisms of $\mathbf{C}$ to Objects
and Morphisms of $\mathbf{D}$ with the following Equivalences:
\begin{itemize}
\item $F(f : A \rightarrow B) = F(f) : F(A) \rightarrow F(B)$
\item $F(g \circ f) = F(g) \circ F(f)$
\item $F(1_A) = 1_{F(A)}$
\end{itemize}
Every Category has an Identity Functor $1_{\mathbf{C}} : \mathbf{C}
\rightarrow \mathbf{C}$ and the Category of all Small Categories and
Functors is denoted $\mathbf{Cat}$. Note that in constructions
(\S\ref{sec:category_construction}) where a Functor appears, if that
Functor is an Identity Functor, the Category instead may be
substituted.

Functors may also be defined in terms of the Object Mapping and Hom-sets:
\[
    F_{A,B} : Hom_{\mathbf{C}}(A,B) \rightarrow Hom_{\mathbf{D}}(F(A),F(B))
\]
where $F_{A,A}1_A = 1_{F(A)}$ and Composition is Commutative.

Functors between Poset Categories are Monotonic Functions
(\S\ref{subsec:monotonicity}).

A \emph{Forgetful Functor} is one that drops some Property of the
Input Category in the Output Category.

An \emph{Inclusion Functor} is a Functor from a Subcategory to its
containing Category (\S\ref{subsec:subcategory}).

A Functor on two Categories is a \emph{Bifunctor}
(\S\ref{subsec:bifunctor}) and generalised to more Categories is a
\emph{Multifunctor}.

\emph{Product Functor}



% --------------------------------------------------------------------
\subsection{Faithful \& Full Functor}\label{subsec:faithful_functor}
% --------------------------------------------------------------------

Given a Functor, $F : C \rightarrow D$, and the Induced Function
\[
    F_{X,Y} : \mathrm{Hom}_C(X,Y) \rightarrow \mathrm{Hom}_D(F(X),F(Y))
\]
then
\begin{itemize}
    \item $F$ is a \emph{Faithful Functor} if $F_{X,Y}$ is Injective
    \item $F$ is a \emph{Full Functor} if $F_{X,Y}$ is Surjective
    \item $F$ is a \emph{Fully Faithful Functor} if $F_{X,Y}$ is
      Bijective
\end{itemize}
A Faithful Functor may be considered an \emph{Embedding}, e.g. the
\emph{Inclusion Functor} for a Subcategory
(\S\ref{subsec:subcategory}).



% --------------------------------------------------------------------
\subsection{Contravariant \& Covariant Functor}
\label{subsec:contravariant_functor}
% --------------------------------------------------------------------

A \emph{Contravariant Functor} is from a Dual Category
(\S\ref{subsec:opposite_category}) to another Category, e.g. $F :
\mathbf{C^{op}} \rightarrow \mathbf{D}$, but expressed in terms of the
original Category:
\[
    \overline{F} : \mathbf{C} \rightarrow \mathbf{D}
\]

The Dual (\S\ref{subsec:abstract_category}) of a Contravariant
Functor, a \emph{Covariant Functor}.



\subsubsection{Presheaf}\label{subsec:presheaf}

A \emph{Presheaf} is a Contravariant Functor from an Opposite Category
to the Category $\mathbf{Set}$.



% --------------------------------------------------------------------
\subsection{Hom-functor}\label{subsec:hom_functor}
% --------------------------------------------------------------------

A \emph{Hom-functor} is a Functor from a Locally Small Category,
$\mathbf{C}$, to the Category $\mathbf{Set}$, and has a Covariant and
a Contravariant definition:

\begin{enumerate}
    \item \emph{Covariant Hom-functor}, for $A,f : X \rightarrow Y \in
      \mathbf{C}$:
\[
    h^A = Hom(A,-) : \mathbf{C} \rightarrow \mathbf{Set}
\]\[
    X \mapsto Hom(A,X)
\]\[
    f \mapsto Hom(A,f) : Hom(A,X) \rightarrow Hom(A,Y)
\]
    where $Hom(A,f)$ is defined for all $g \in Hom(A,X)$ as:
\[
    g \mapsto f \circ g
\]

    \item \emph{Contravariant Hom-functor} (also \emph{Functor of
      Points}, see \S\ref{subsec:general_element}), for $B,f : X
      \rightarrow Y \in \mathbf{C}$:
\[
    h_B = Hom(-,B) : \mathbf{C} \rightarrow \mathbf{Set}
\]\[
    X \mapsto Hom(X,B)
\]\[
    f \mapsto Hom(f,B) : Hom(Y,B) \rightarrow Hom(X,B)
\]
    where $Hom(f,B)$ is defined for all $g \in Hom(Y,B)$ as:
\[
    g \mapsto g \circ f
\]
\end{enumerate}

The Category of all Hom-functors and \emph{Natural Transformations}
(\S\ref{sec:natural_transformation}) between them, $\{ h^A | A \in
\mathbf{C} \}$, is a Subcategory of the Functor Category
(\S\ref{subsec:functor_category}) $\mathbf{Set^C}$, and is Isomorphic
to $\mathbf{C^{op}}$ (see \emph{Yoneda Embedding}
\S\ref{subsec:yoneda_embedding}).

Every Morphism $f : A' \rightarrow A$ determines a pair of Natural
Transformations:
\[
    Hom(f,-) : h^A \rightarrow h^{A'}
\]\[
    Hom(-,f) : h_{A'} \rightarrow h_A
\]

For any pair of Morphisms, $f : A' \rightarrow A$ and $g : B
\rightarrow B'$:
\[
    Hom(A',g) \circ Hom(f,B) = Hom(f,B') \circ Hom(A,g)
\]
is a path sending
\[
    h : A \rightarrow B
\]
to
\[
    g \circ h \circ f : A' \rightarrow B'
\]

The Covariant Bifunctor $Hom_{\mathbf{C}}(-,-) : \mathbf{C^{op}}
\times \mathbf{C} \rightarrow \mathbf{Set}$ is Contravariant in the
first argument and Covariant in the second.



\subsubsection{Internal Hom-functor}\label{subsec:internal_homfunctor}



\subsubsection{Representable Functor}\label{subsec:representable_functor}

%FIXME definition of 'representation'
%FIXME ref Naturally Isomorphic
A Functor $F : \mathbf{C} \rightarrow \mathbf{Set}$ is a
\emph{Representable Functor} if it is Naturally Isomorphic to the
Hom-functor for some Object $A \in \mathbf{C}$.

A \emph{Covariant Representable Functor} for an Object $A$ in a
Category $\mathbf{C}$ is defined as Naturally Isomorphic to the
Covariant Hom-functor $h^A = Hom(A,-) : \mathbf{C} \rightarrow
\mathbf{Set}$
\[
    Hom(A,-) : Hom(A,X) \xrightarrow{f_*} Hom(A,Y)
\]
A \emph{Contravariant Representable Functor} for $A$ is a Functor that
is Naturally Isomorphic to the Contravariant Hom-functor $h_A =
Hom(-,A) : \mathbf{C^{op}} \rightarrow \mathbf{Set}$
\[
    Hom(-,A) : Hom(X,A) \xrightarrow{f^*} Hom(Y,A)
\]

A \emph{Representation} of a Covariant Representable Functor, $F$, is
a pair $(A, \Phi)$ with Natural Isomorphism $\Phi : Hom(A,-)
\rightarrow F$.



\subsubsection{Yoneda Lemma}\label{subsec:yoneda_lemma}

For an arbitrary Covariant Functor $F : \mathbf{C} \rightarrow
\mathbf{Set}$:
\[
    \forall A \in \mathbf{C}, Nat(h^A,F) \cong F(A)
\]
If $F$ is a Covariant Hom-functor $h^B$, then:
\[
    \forall A \in \mathbf{C}, Nat(h^A,h^B) \cong Hom(B,A)
\]

For an arbitrary Contravariant Functor $G : \mathbf{C} \rightarrow
\mathbf{Set}$:
\[
    \forall A \in \mathbf{C}, Nat(h_B,G) \cong G(A)
\]
If $F$ is a Contravariant Hom-functor $h_B$, then:
\[
    \forall A \in \mathbf{C}, Nat(h_A,h_B) \cong Hom(A,B)
\]



\subsubsection{Yoneda Embedding}\label{subsec:yoneda_embedding}

The Fully Faithful Contravariant Functor $h^- : \mathbf{C} \rightarrow
\mathbf{Set^C}$ which maps each Object $A \in \mathbf{C}_0$ to the
Hom-functor $h^A$ and each $f \in \mathbf{C}_1$ to the Natural
Transformation $Hom(f,-)$ can also be interpreted as a Covariant
Functor $h^- : \mathbf{C^{op}} \rightarrow \mathbf{Set^C}$. Being a
Faithful Functor means $h^-$ gives an Embedding
(\S\ref{subsec:faithful_functor}) of $\mathbf{C^{op}}$ in
$\mathbf{Set^C}$.

By the Contravariant Yoneda's Lemma:
\[
    h_-: \mathbf{C} \rightarrow \mathbf{Set^{C^{op}}}
\]
called the \emph{Yoneda Embedding}.



% --------------------------------------------------------------------
\subsection{Adjunction}\label{subsec:adjunction}
% --------------------------------------------------------------------

An \emph{Adjoint Functor} is the Categorical analog of the Existential
Quantifier in Logic (\S\ref{subsec:firstorder_quantification}) and the
\emph{Image Operation} along a \emph{Continuous Function} in
\emph{Topology} (Part \ref{sec:topology}).



% ====================================================================
\section{Natural Transformation}\label{sec:natural_transformation}
% ====================================================================

A \emph{Natural Transformation} is a Morphism between Functors with
two Properties. Given a Natural Transformation, $\tau$, between two
Functors, $S$ and $T$, between two Categories, $\mathbf{C}$ and
$\mathbf{D}$:
\[
    \tau : S \rightarrow T
\]
\begin{enumerate}
    \item $\forall X \in \mathbf{C},
        \exists \tau_X : S(X) \rightarrow T(X) \in \mathbf{D}$
    \item $\forall f : X \rightarrow Y \in \mathbf{C},
        \tau_Y \circ S(f) = G(f) \circ \tau_X$
\end{enumerate}
where the Morphism $\tau_X$ is called the \emph{Component} of $\tau$
at $X$. When (2) holds, a Commutative Diagram is formed and the
Morphisms $\tau_X$ are said to be \emph{Natural} in $X$.

When every Component in $\tau$ is Invertible in $\mathbf{D}$, $\tau$
is a \emph{Natural Isomorphism} (or \emph{Natural Equivalence}), and:
\[
    \tau : S \cong T
\]



% --------------------------------------------------------------------
\subsection{Vertical Composition}\label{subsec:vertical_composition}
% --------------------------------------------------------------------

Given Natural Transformations $\eta : F \rightarrow G$ and $\epsilon :
G \rightarrow H$ between Functors $F,G,H : \mathbf{C} \rightarrow
\mathbf{D}$, the \emph{Vertical Composition} is given as:
\[
    \epsilon \eta : F \rightarrow H
\]
Vertical Composition may be explicitly denoted with the $\cdot$
Operator, $\epsilon \cdot \eta$. Vertical Composition is Associative
and has an Identity.



% --------------------------------------------------------------------
\subsection{Horizontal Composition}\label{subsec:horizontal_composition}
% --------------------------------------------------------------------

Given Functors $F,G : \mathbf{C} \rightarrow \mathbf{D}$ and $J,K :
\mathbf{D} \rightarrow \mathbf{E}$, and Natural Transformations $\eta
: F \rightarrow G$ and $\epsilon : J \rightarrow K$, the
\emph{Horizontal Composition} is given as:
\[
    \eta \epsilon : JF \rightarrow KG
\]
Horizontal Composition may be explicitly denoted with the $\circ$
Operator: $\eta \circ \epsilon$. Horizontal Composition is Associative
and has the same Identity as Vertical Composition.



% --------------------------------------------------------------------
\subsection{Interchange Law}\label{subsec:interchange_law}
% --------------------------------------------------------------------

Given three Categories, $\mathbf{B}$, $\mathbf{C}$, and $\mathbf{D}$,
and six Functors, $P,Q,R : \mathbf{B} \rightarrow \mathbf{C}$ and
$S,T,U : \mathbf{C} \rightarrow \mathbf{D}$, and four Natural
Transformations, $\sigma : P \rightarrow Q$, $\tau : Q \rightarrow R$,
$\sigma' : S \rightarrow T$, and $\tau' : T \rightarrow U$, the
following \emph{Interchange Law} applies:
\[
    (\tau' \cdot \sigma') \circ (\tau \cdot \sigma) =
    (\tau' \circ \tau) \cdot (\sigma' \circ \sigma)
\]



% ====================================================================
\section{Constructions}\label{sec:category_construction}
% ====================================================================

% --------------------------------------------------------------------
\subsection{Opposite Categories}\label{subsec:opposite_category}
% --------------------------------------------------------------------

The \emph{Opposite} or \emph{Dual} (\S\ref{subsec:abstract_category})
of a Category $\mathbf{C}$ is denoted $\mathbf{C^{op}}$ or
$\mathbf{C^*}$ and has the same Objects as $\mathbf{C}$ but the Domain
and Codomain in each Morphism is reversed. Objects and Morphisms of a
Dual Category may be written with over-lines to distinguish them from
the original Category: $\overline{f}: \overline{C} \rightarrow
\overline{D}$. With this notation the following Equalities may be
expressed:
\[
    1_{\overline{C}} = \overline{1_C}
\]\[
    \overline{f} \circ \overline{g} = \overline{g \circ f}
\]
A Terminal Object in $\mathbf{C}$ is an Initial Object in
$\mathbf{C^{op}}$ and vice-versa.

In a Dual Category the following are all Duals of eachother:
\begin{itemize}
    \item Monomorphisms and Epimorphisms (\S\ref{sec:morphism})
    \item Left and Right Inverses (\S\ref{sec:morphism})
    \item Initial and Terminal Objects (\S\ref{sec:universal_property})
\end{itemize}



% --------------------------------------------------------------------
\subsection{Category \& Functor Products}\label{subsec:category_product}
% --------------------------------------------------------------------

A \emph{Product}, $\times$, is a construction (or more specifically a
Functor) on Categories or Functors:
\[
    \times : \mathbf{Cat} \times \mathbf{Cat} \rightarrow \mathbf{Cat}
\]



\subsubsection{Product Category}\label{subsec:product_category}

A \emph{Product Category} can be constructed from two Categories,
$\mathbf{C}$ and $\mathbf{D}$, and is denoted:
\[
    \mathbf{C} \times \mathbf{D}
\]
has Objects of the form $(C,D)$ where $C \in \mathbf{C}$ and $D \in
\mathbf{D}$ and Morphisms $(f,g) : (C,D) \rightarrow (C',D')$ where $f
: C \rightarrow C' \in \mathbf{C}$ and $g : D \rightarrow D' \in
\mathbf{D}$. Composition and Identity are defined as:
\[
    (f',g') \circ (f,g) = (f' \circ f,g' \circ g)
\]\[
    1_{(C,D)} = (1_C, 1_D)
\]
$\mathbf{C} \times \mathbf{D}$ is a Product
(\S\ref{subsec:product_diagram}) in $\mathbf{Cat}$.



\paragraph{Projection}\label{subsec:projection_functor}
\hfill \\
A Product Category has a pair of \emph{Projections} which are Functors
from the Product Category to the original Categories:
\[
    \mathbf{C} \xleftarrow{\;\; P\;\;} \mathbf{C}\times\mathbf{D}
    \xrightarrow{\;\; Q\;\;} \mathbf{D}
\]
such that for $C,f \in \mathbf{C}, D,g \in \mathbf{D}$:
\[
    P(C,D) = C, \;\; P(f,g) = f
\]\[
    Q(C,D) = D, \;\; Q(f,g) = g
\]
Given any other Category, $\mathbf{B}$, there exists a unique Functor:
\[
    F : \mathbf{B} \rightarrow \mathbf{C} \times \mathbf{D}
\]
with:
\[
    PF = R : \mathbf{B} \rightarrow \mathbf{C}
\]\[
    QF = T : \mathbf{B} \rightarrow \mathbf{D}
\]
giving:
\[
    \forall h \in B, F(h) = (Rh,Th)
\]



\paragraph{Bifunctor}\label{subsec:bifunctor}
\hfill \\

A \emph{Bifunctor} is a Functor of two Variables from a Product
Category to an arbitrary Category:
\[
    S : \mathbf{C} \times \mathbf{D} \rightarrow \mathbf{A}
\]

A \emph{Multifunctor} is a generalized to $n$ or more Variables.



\subsubsection{Functor Product}\label{subsec:functor_product}

Give two Functors, $U : \mathbf{C} \rightarrow \mathbf{C'}$ and $V :
\mathbf{D} \rightarrow \mathbf{D'}$, a \emph{Functor Product} is
defined as:
\[
    U \times V : \mathbf{C} \times \mathbf{D}
    \rightarrow \mathbf{C'} \times \mathbf{D'}
\]
where:
\[
    (U \times V)(C,D) = (UC,VD), \;\; (U \times V)(f,g) = (Uf,Vg)
\]
and $(U \times V)$ is the unique Functor such that:
\[
    P'(U \times V) = UP, \;\; Q'(U \times V) = VQ
\]



% --------------------------------------------------------------------
\subsection{Functor Category}\label{subsec:functor_category}
% --------------------------------------------------------------------

Given two Categories, $\mathbf{C}$ and $\mathbf{D}$, a \emph{Functor
  Category} is a Category with Objects as Functors $T : \mathbf{C}
\rightarrow \mathbf{D}$ and Morphisms as Natural Transformations
between Functors:
\[
    \mathbf{D}^{\mathbf{C}} = Funct(\mathbf{C},\mathbf{D})
\]
The Hom-set in a Functor Category may be denoted:
\[
    Nat(S,T) = \mathbf{D}^{\mathbf{C}}(S,T) =
        \{ \tau | \tau : S \rightarrow T \}
\]



% --------------------------------------------------------------------
\subsection{Free Category}\label{subsec:free_category}
% --------------------------------------------------------------------

A \emph{Free Category}, $\mathbf{C}_G$, is Generated by a Directed
Graph or Precategory (\S\ref{subsec:precategory}), $G$, of Vertices
and Edges under Concatenation of Paths.

For a Graph with a single Vertex, the Free Category is the Free Monoid
on the Edges of the Graph. A Graph with no Edges generates a Discrete
Category (\S\ref{sec:category}). The Free Category on a Graph with
two Vertices and one Edge between them is the Finite Category
$\mathbf{2}$.



% --------------------------------------------------------------------
\subsection{Comma Category}\label{subsec:comma_category}
% --------------------------------------------------------------------

A \emph{Comma Category} is formed from a pair of Functors
(\S\ref{sec:functor}) that share a common Codomain. For three
Categories, $\mathbf{A}$, $\mathbf{B}$, and $\mathbf{C}$ and Functors
$S$ (\emph{Source}) and $T$ (\emph{Target}) in the following relation:
\[
    \mathbf{A} \xrightarrow{\;\; S\;\;} \mathbf{C} \xleftarrow{\;\;
      T\;\;} \mathbf{B}
\]
one can form a Comma Category $(S \downarrow T)$ with Objects as
Triples $(\alpha, \beta, f)$ where $\alpha$ is an Object in
$\mathbf{A}$, $\beta$ is an Object in $\mathbf{B}$, and $f : S(\alpha)
\rightarrow T(\beta)$ is a Morphism in $\mathbf{C}$ and with Morphisms
between Triples $(\alpha, \beta, f)$ to $(\alpha', \beta', f')$ as
pairs $(g,h)$ where $g : \alpha \rightarrow \alpha'$ is a Morphism in
$\mathbf{A}$ and $h : \beta \rightarrow \beta'$ is a Morphism in
$\mathbf{B}$.

When $S$ is a Functor, $\mathbf{1} \xrightarrow{\;\;S\;\;}
\mathbf{C}$, to a single Object $A \in \mathbf{C}$, the resulting
Comma Category may be denoted $(A \downarrow \mathbf{C})$ and is
called the Category of Objects under $A$. Here Objects are Morphisms
with Domain of $A$, and Morphisms are Commutative triangles with top
Vertex $A$.

The Category of Objects over $A$ is likewise $(\mathbf{C} \downarrow
A)$ and has as Objects Morphisms with Codomain $A$ and Morphisms are
Commutative triangles with a bottom Vertex $A$.

When both $S$ and $T$ are Functors from $\mathbf{1}$ to Objects $A$
and $B$ respectively, the result is a Discrete Category whose Objects
are $Hom(A,B)$.

The case where $S = T = 1_\mathbf{C}$, $(\mathbf{C} \downarrow
\mathbf{C})$, is the Category of all Morphisms of $\mathbf{C}$:
$\mathbf{C}^\mathbf{2}$.



\subsubsection{Arrow Categories}\label{subsec:arrow_categories}

An \emph{Arrow Category} of a Category $\mathbf{C}$, written
$\mathbf{C^{\rightarrow}}$, has for its Objects the Morphisms of
$\mathbf{C}$ and as Morphisms pairs of Objects such that their
underlying Morphisms in $\mathbf{C}$ are Composable.



\subsubsection{Slice Categories}\label{subsec:slice_categories}

A \emph{Slice Category} $\mathbf{C}/C$ of a Category $\mathbf{C}$ with
an Object $C$ has as Objects the Morphisms with Codomain $C$ and as
Morphisms those Morphisms in $\mathbf{C}$ between the Domains of the
underlying Morphisms of the Objects of $\mathbf{C}/C$. That is, for
Objects in the Slice Category corresonding to Morphisms $f$ and $f'$,
the Morphism in the Slice Category between the two is $g$ such that
\[
    f' \circ g = f
\]

\emph{Coslice}



% --------------------------------------------------------------------
\subsection{Quotient Category}\label{subsec:quotient_category}
% --------------------------------------------------------------------

%FIXME this probably needs a rewrite

The \emph{Quotient Category} is defined for a Category $\mathbf{C}$
with Congruence Relation $\sim$ as $\mathbf{C}/\sim$:
\[
    (\mathbf{C}/\sim)_0 = \mathbf{C_0}
\]\[
    (\mathbf{C}/\sim)_1 = (\mathbf{C_1})/\sim
\]
where Morphisms are of the form $[f]$ where $f \in \mathbf{C_1}$.

For a Category $\mathbf{C}$ with Graph $G$ and relations $R$,
$\mathbf{C}/R$ is called the Category with \emph{Generators} $G$ and
\emph{Relations} $R$.



\subsubsection{Congruence Category}\label{subsec:congruence_category}

\emph{Congruence} on a Category is an Equivalence Relation on
Morphisms such that for two Morphisms $f,g \in \mathbf{C_1}$, $f \sim
g$ Implies:
\begin{itemize}
\item $dom(f) = dom(g)$
\item $cod(f) = cod(g)$
\item $\forall a,b \in \mathbf{C_1}, bfa \sim bga$
\end{itemize}
Such a Congruence defines a \emph{Congruence Category}
$\mathbf{C^{\sim}}$:
\[
    (\mathbf{C^{\sim}})_0 = \mathbf{C}_0
\]\[
    (\mathbf{C^{\sim}})_1 = \{\langle f,g \rangle | f \sim g\}
\]\[
    \tilde{1_\mathbf{C}} = \langle 1_\mathbf{C}, 1_\mathbf{C} \rangle
\]\[
    \langle f',g' \rangle \circ \langle f,g \rangle = \langle f'f,g'g \rangle
\]



\subsubsection{Kernel Category}\label{subsec:kernel_category}

Given a Functor $F : \mathbf{C} \rightarrow \mathbf{D}$, a Congruence
$\sim_F$ on $\mathbf{C}$ is defined as
\[
    f \sim_F g \leftrightarrow dom(f) = dom(g) \wedge cod(f) = cod(g)
    \wedge F(f) = F(g)
\]
The \emph{Kernel Category} of $F$ is then defined as the Congruence
Category of $\sim_F$
\[
    ker(F) = C^{\sim_F}
\]



\subsubsection{Finitely Presented Category}\label{subsec:finitely_presented}

% FIXME free category notation?
A \emph{Finitely Presented Category} is given by taking the Quotient
Category of a Free Category $\mathbf{C}(G)$ with the Congruence
$\sim_\Sigma$
\[
    \mathbf{C}(G) / \sim_{\Sigma} = \mathbf{C}(G,\Sigma)
\]
where $\Sigma$ is the finite Set of Relations
\[
    (g_1 \circ \ldots \circ g_n) = (g'_1 \circ \ldots \circ g'_m)
\]
for all $g_i \in G$ such that $dom(g_n) = dom(g'_m)$ and $cod(g_1) =
cod(g'_1)$.



% --------------------------------------------------------------------
\subsection{Diagrams}\label{subsec:category_diagram}
% --------------------------------------------------------------------

A \emph{Diagram} consists of Objects and Morphisms that are Indexed by
a Category or equivalently a \emph{Functor} from an \emph{Index
  Category} to another Category. A Diagram is the Category Theory
analogue of an Indexed Family of Sets (\S\ref{subsec:index_set}).

A Diagram, $D$, with Index Category $\mathbf{J}$ in Category
$\mathbf{C}$:
\[
    D : \mathbf{J} \rightarrow \mathbf{C}
\]



\subsubsection{Commutative Diagram}\label{subsec:commutative_diagram}



\subsubsection{Cones}\label{subsec:category_cone}

\emph{Universal Cone}




% --------------------------------------------------------------------
\subsection{Monoidal Category}\label{subsec:monoidal_category}
% --------------------------------------------------------------------

\subsubsection{Closed Monoidal Category}\label{subsec:closed_monoidal}

\subsubsection{Cartesian Closed Categories}\label{subsec:cartesian_category}

\emph{Simply Typed $\lambda$-calculus} (\S\ref{subsec:simply_typed})



% ====================================================================
\section{Universal Properties}\label{sec:universal_property}
% ====================================================================

% --------------------------------------------------------------------
\subsection{Universal Mapping Property}
\label{subsec:universal_mapping_property}
% --------------------------------------------------------------------

\emph{Existence}

\emph{Uniqueness}



% --------------------------------------------------------------------
\subsection{Initial \& Terminal Objects}\label{subsec:initial_terminal}
% --------------------------------------------------------------------

There are two Universal Properties of Objects: \emph{Initial} and
\emph{Terminal}.

An Object $0$ in a Category $\mathbf{C}$ is Initial if for every other
Object $A$ in the Category there is a unique Morphism $0 \rightarrow
A$.

An Object $1$ in a Category $\mathbf{C}$ is Terminal if for every
other Object $A$ in the Category there is a unique Morphism $A
\rightarrow 1$.

An Object that is both an Initial and a Terminal Object is called a
\emph{Zero Object} (or \emph{Null Object}). A Category with a Zero
Object is called a \emph{Pointed Category}. For a Zero Object, $Z \in
\mathbf{C}$, there is a Unique Morphism between any two Objects, $A, B
\in \mathbf{C}$ called the \emph{Composite} in $Z$:
\[
    A \rightarrow Z \rightarrow B
\]

It follows for any pair of Objects that are both Terminal and/or both
Initial that an Isomorphism (\S\ref{subsec:isomorphism}) exists
between them; that is, Universal Objects are always unique up to
Isomorphism.

As an example, in $\mathbf{Set}$, all Singleton Sets are Terminal, and
as such they are all Isomorphic to one another. Given a Set $X$:
\[
    |X| = 1 \leftrightarrow \forall Y, |Hom_{\mathbf{Sets}}(Y,X)| = 1
\]
Also in $\mathbf{Set}$ the Empty Set is Initial as the only mapping
from it to any other Set is the Empty Function. In $\mathbf{Grp}$, a
Trivial Group ${1}$ is a Zero Object.



% --------------------------------------------------------------------
\subsection{Universal Morphism}\label{subsec:universal_morphism}
% --------------------------------------------------------------------

Given a Functor $S: \mathbf{D} \rightarrow \mathbf{C}$, an
\emph{Universal Morphism} to $S$ or \emph{Initial Morphism}, is an
Initial Object of the form $(X',u)$ in the Comma Category
(\S\ref{subsec:comma_category}) $(X \downarrow S)$ where $X \in
\mathbf{C_0}$, $u : X \rightarrow S(X')$ and $X' \in \mathbf{D}$.
%FIXME is X' initial and/or terminal in D?

$(X', u)$ satisfies the \emph{Initial Property}:
\[
    \forall Y' \in \mathbf{D}, \forall f : X \rightarrow S(Y') \in
    \mathbf{C}, \exists! g : X' \rightarrow Y' : S(g) \circ u = f
\]

The Dual concept of an Initial Morphism, an Universal Morphism from
$S$ or \emph{Terminal Morphism}, is a Terminal Object of the form
$(X',v)$ in the Comma Category $(S \downarrow X)$ where $v : S(X')
\rightarrow X \in \mathbf{C}$.

$(X',v)$ satisfies the \emph{Terminal Property}:
\[
    \forall Y' \in \mathbf{D}, \forall f : S(Y') \rightarrow X \in
    \mathbf{C}, \exists! g : Y' \rightarrow X' : v \circ S(g) = f
\]

%FIXME universality in terms of Hom sets



\subsubsection{Universal Element}\label{subsec:universal_element}

\emph{Representable Functor} (\S\ref{subsec:representable_functor})

For a Functor $H : \mathbf{D} \rightarrow \mathbf{Set}$, an
\emph{Universal Element} of $H$ is a pair of Objects $(A,X) \in
\mathbf{D}_0 \times \mathbf{Set}_0$ such that:
\[
    \forall (A',X') \in \mathbf{D}_0 \times \mathbf{Set}_0,
    \exists! f : A \rightarrow A' \in \mathbf{D} : H(f)(X) = X'
\]



% --------------------------------------------------------------------
\subsection{Global \& General Elements}\label{subsec:general_element}
% --------------------------------------------------------------------

A \emph{Global Element}, $a$, (also \emph{Point} or \emph{Constant})
of an Object, $A$, is a Morphism from a Terminal Object, $1$, to that
Object
\[
    a: 1 \rightarrow A
\]

A \emph{General Element}, $x$, (also \emph{Variable}) is a Morphism
from an arbitrary Domain Object, $X$
\[
    x: X \rightarrow A
\]



% --------------------------------------------------------------------
\subsection{Limits \& Colimits}\label{subsec:category_limits}
% --------------------------------------------------------------------

\emph{Limit} = \emph{Inverse Limit} = \emph{Projective Limit} =
\emph{Left Root}

\emph{Colimit} = \emph{Direct Limit} = \emph{Inductive Limit} =
\emph{Right Root}



\subsubsection{Products \& Coproducts}\label{subsec:product_diagram}

A \emph{Product} of two Objects $A \times B$:
\[
    A \xleftarrow{\;\;p_1\;\;} P \xrightarrow{\;\;p_2\;\;} B
\]
is a Product of $A$ and $B$ if and only if for any $A
\xleftarrow{\;\;z_1\;\;} Z \xrightarrow{\;\;z_2\;\;} B$
\[
    \exists!u : Z \rightarrow P
\]
with $p_i \circ u = z_i$. $u$ may also be written as $\langle z_1, z_2
\rangle$ as it is uniquely determined by $z_1$ and $z_2$.

Products are unique up to Isomorphism (\S\ref{subsec:isomorphism}).

The Diagram $A \xrightarrow{\;\;q_1\;\;} Q \xleftarrow{\;\;q_2\;\;} B$
is a \emph{Coproduct} $A + B$ if for any $A \xrightarrow{\;\;z_1\;\;} Z
\xleftarrow{\;\;z_2\;\;} B$
\[
    \exists!u : Q \rightarrow Z
\]
with $u \circ q_i = z_i$. $u$ may also be written as $[ z_1, z_2 ]$
and Coprojections $q_i$ may be called \emph{Injections} (although they
are not necessarily Injective Morphisms).

Disjoint Union %FIXME ref disjoint union

\emph{N-ary Products}

\paragraph{Exponentials}\label{subsec:category_exponential}



\subsubsection{Pushouts \& Pullbacks}\label{subsec:category_pullback}



\subsubsection{Equalizer}\label{subsec:limit_equalizer}



% ====================================================================
\section{Abstract Algebra}\label{sec:abstract_algebra}
% ====================================================================

\emph{Presentations}: Generators, Relations

\emph{Finitely Presented}



% --------------------------------------------------------------------
\subsection{Magma}\label{subsec:magma}
% --------------------------------------------------------------------

A \emph{Magma}, $M$, is an Algebraic Structure
(\S\ref{subsec:universal_algebra}) with a single Closed Binary
Operation, $M \times M \rightarrow M$.

% --------------------------------------------------------------------
\subsection{Semigroup}\label{subsec:semigroup}
% --------------------------------------------------------------------

A \emph{Semigroup} is Magma (\S\ref{subsec:magma}) with an Associative
Binary Operation. A Semigroup is differentiated from a Monoid
(\S\ref{subsec:monoid}) by not requiring an Identity Element, and from
a Group (\S\ref{subsec:group}) by not requiring Inverses.



% --------------------------------------------------------------------
\subsection{Monoid}\label{subsec:monoid}
% --------------------------------------------------------------------

A \emph{Monoid} is a Semigroup with an Identity Element. The set of
all Endomorphisms of an Object, $X$, in a Category, $C$,
\[
    Hom(X,X)
\]
defines a Monoid and is denoted $End_C(X)$.



\subsubsection{Free Monoid}\label{subsec:free_monoid}



% --------------------------------------------------------------------
\subsection{Abstract Groups}\label{subsec:abstract_groups}
% --------------------------------------------------------------------
% --------------------------------------------------------------------
\subsection{Operad Theory}\label{subsec:operad_theory}
% --------------------------------------------------------------------
% --------------------------------------------------------------------
\subsection{Ring}\label{subsec:ring}
% --------------------------------------------------------------------

\emph{Ideals}

\emph{Principal Ideals}



\subsubsection{Commutative Ring}

\emph{Determinant}



\subsubsection{Fields}

\paragraph{Total Field}\label{subsec:total_field}

\paragraph{Closed Field}\label{subsec:closed_field}



% --------------------------------------------------------------------
\subsection{Initial Algebra}\label{subsec:initial_algebra}
% --------------------------------------------------------------------

% --------------------------------------------------------------------
\subsection{Algebraic Geometry}\label{subsec:algebraic_geometry}
% --------------------------------------------------------------------

\emph{Hilbert's Nullstellensatz}

\emph{Cauchy's Inequality}:
\[
    |\langle x,y \rangle|^2 \leq \langle x,x \rangle \cdot \langle
    y,y \rangle
\]

\subsubsection{Universal Algebraic Geometry}

% --------------------------------------------------------------------
\subsection{Heyting Algebra}\label{subsec:heyting_algebra}
% --------------------------------------------------------------------

\subsubsection{Boolean Algebra}\label{subsec:boolean_algebra}

% --------------------------------------------------------------------
\subsection{Relation Algebra}
% --------------------------------------------------------------------
% --------------------------------------------------------------------
\subsection{Relational Algebra}
% --------------------------------------------------------------------
\emph{Domain Relational Calculus}
% --------------------------------------------------------------------
\subsection{Quantale}
% --------------------------------------------------------------------



% ====================================================================
\section{Group Theory}
% ====================================================================

% --------------------------------------------------------------------
\subsection{Group}\label{subsec:group}
% --------------------------------------------------------------------

A \emph{Group} is a Monoid (\S\ref{subsec:monoid}) with an Inverse for
every Morphism (every Morphism is an Isomorphism). A Group may be
formed from a Set $X$ by adding all Automorphisms
(\S\ref{subsec:automorphism}) of $X$: $Aut(X)$. \emph{Cayley's
  Theorem} states that every Group is Isomorphic to a Group of
Permutations (\S\ref{subsec:permutations}).

A Group, $G$, within a Category, $\mathbf{C}$, may be viewed as a
Subset of the Hom-set of an Object, $X$:
\[
    G \subseteq Hom_{\mathbf{C}}(X,X)
\]

\emph{Trivial Group} ${1}$



% --------------------------------------------------------------------
\subsection{Groupoid}\label{subsec:groupoid}
% --------------------------------------------------------------------

\emph{Fundamental Groupoid}



% --------------------------------------------------------------------
\subsection{Group Homomorphism}\label{subsec:group_homomorphism}
% --------------------------------------------------------------------

A \emph{Group Homomorphism} preserves Group operations.

Taken as Monoidal Categories, two Groups $G, H$ may be related by a
Functor $f$ which is equivalent to a Group Homomorphism:
\[
    f : G \rightarrow H
\]



% --------------------------------------------------------------------
\subsection{Group Representations}\label{subsec:group_representation}
% --------------------------------------------------------------------

Given a Functor $R$ from a Group $G$ to a general Category
$\mathbf{C}$
\[
    R : G \rightarrow \mathbf{C}
\]
Such a Functor $R$ is termed a \emph{Representation} of $G$ in
$\mathbf{C}$.



% --------------------------------------------------------------------
\subsection{Coset}\label{subsec:group_coset}
% --------------------------------------------------------------------

The \emph{Coset} of a Subgroup (\S\ref{subsec:subgroup}) $H$ in a Group $G$
is defined as
\begin{description}
\item[Left Coset] $gH = {gh : h \in H}$
\item[Right Coset] $Hg = {hg : h \in H}$
\end{description}



% --------------------------------------------------------------------
\subsection{Subgroup}\label{subsec:subgroup}
% --------------------------------------------------------------------

A \emph{Subgroup} is a Subset of Group Elements that are still a Group
under the Group Operation.

Given a Group $G$ and Subgroup $H$, $H$ is a \emph{Normal Subgroup} if
and only if $\forall g \in G, gH = Hg$.

\emph{Commutator Subgroup}



\subsubsection{Skeleton}



% --------------------------------------------------------------------
\subsection{Group Kernel}\label{subsec:group_kernel}
% --------------------------------------------------------------------

The \emph{Kernel} of a Group Homomorphism, $h : G \rightarrow H$,
denoted by $ker(h)$ is defined as:
\[
    ker(h) = {g \in G : f(g) = e_H}
\]
where $e_H$ is an Element of $H$, and so the Kernel is just the
Preimage of the Singleton Set $\{e_H\}$. Such a Kernel is a Normal
Subgroup.



% ====================================================================
\section{Institution Theory}\label{subsec:institution_theory}
% ====================================================================

% ====================================================================
\section{Representation Theory}
% ====================================================================
