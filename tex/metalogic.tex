%%%%%%%%%%%%%%%%%%%%%%%%%%%%%%%%%%%%%%%%%%%%%%%%%%%%%%%%%%%%%%%%%%%%%%
%%%%%%%%%%%%%%%%%%%%%%%%%%%%%%%%%%%%%%%%%%%%%%%%%%%%%%%%%%%%%%%%%%%%%%
\part{Metalogic}\label{sec:metalogic}
%%%%%%%%%%%%%%%%%%%%%%%%%%%%%%%%%%%%%%%%%%%%%%%%%%%%%%%%%%%%%%%%%%%%%%
%%%%%%%%%%%%%%%%%%%%%%%%%%%%%%%%%%%%%%%%%%%%%%%%%%%%%%%%%%%%%%%%%%%%%%

% ====================================================================
\section{Logical Reasoning}\label{sec:logical_reasoning}
% ====================================================================

\emph{Premise} (\emph{Precondition})

\emph{Material Conditional} (\emph{Implication Rule})

\emph{Logical Consequence} (\emph{Conclusion})

A Premise \emph{Logically Entails} (\S\ref{sec:logical_consequence}) a
Conclusion if and only if the negation of the Conclusion is
\emph{Logically Inconsistent} with the Premises.

Logical Entailment $\rightarrow$ Inductive Support

Monotonic:

    Deductive Reasoning

Non-monotonic:

    Inductive Reasoning

    Default Reasoning

    Abductive Reasoning



% --------------------------------------------------------------------
\subsection{Logical Constants}\label{sec:logical_constant}
% --------------------------------------------------------------------

\emph{Syncategorematic Symbol} (\emph{Term Logic} \S\ref{sec:term_logic})



% --------------------------------------------------------------------
\subsection{Logical Consequence}\label{sec:logical_consequence}
\cite{beall-restall05}
% --------------------------------------------------------------------

\emph{Logical Consequence} (\emph{Entailment}):
\[
    S \in \mathbf{L}, D \subset \mathbf{L}, S \leftrightarrow D
\]
\begin{itemize}
    \item 1. Logical Form (\S\ref{sec:normal_form})
    \item 2. A Priori
    \item 3. Modality (\S\ref{sec:modal_logic})
\end{itemize}



\subsubsection{Deductive Consequence}\label{sec:deductive_consequence}

\emph{Necessary}



\subsubsection{Inductive Consequence}\label{sec:inductive_consequence}

\emph{Sufficient}

\emph{Non-monotonic Logic} (\S\ref{sec:nonmonotonic_logic})



\subsubsection{Formal Consequence}\label{sec:formal_consequence}

\paragraph{Syntactic Consequence}\label{sec:syntactic_consequence}

\paragraph{Semantic Consequence}\label{sec:semantic_consequence}

\emph{Validity}



\subsubsection{Material Consequence}

\emph{Material Consequence} (\emph{Implication})



% --------------------------------------------------------------------
\subsection{Deductive Reasoning}\label{sec:deductive_reasoning}
% --------------------------------------------------------------------

\subsubsection{Deductive Inference}\label{sec:deductive_inference}

\paragraph{Mathematical Induction}\label{sec:mathematical_induction}

\emph{Mathematical Induction} as an Inference Rule is the
\emph{Implicative} process where a \emph{Base Case} is shown to extend
to the more general by means of Implication (the Inductive step). Note
that Mathematical Induction is not \emph{Inductive Reasoning} which is
an empirical or probabilistic Inference and not a form of Deduction.
The \emph{Principle of Mathematical Induction}
(\S\ref{sec:induction_principle}) is an application of Mathematical
Induction to the \emph{Natural Numbers} (\S\ref{sec:natural_number}).



% --------------------------------------------------------------------
\subsection{Inductive Reasoning}\label{sec:inductive_reasoning}
\cite{hawthorne08}
% --------------------------------------------------------------------

\emph{Inductive Support}

\subsubsection{Inductive Inference}\label{sec:inductive_inference}

\emph{Universal Inductive Inference}



% --------------------------------------------------------------------
\subsection{Abductive Reasoning}\label{sec:abductive_reasoning}
% --------------------------------------------------------------------

\subsubsection{Abductive Inference}\label{sec:abductive_inference}



% --------------------------------------------------------------------
\subsection{Default Reasoning}\label{sec:default_reasoning}
% --------------------------------------------------------------------
