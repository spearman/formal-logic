%%%%%%%%%%%%%%%%%%%%%%%%%%%%%%%%%%%%%%%%%%%%%%%%%%%%%%%%%%%%%%%%%%%%%%
%%%%%%%%%%%%%%%%%%%%%%%%%%%%%%%%%%%%%%%%%%%%%%%%%%%%%%%%%%%%%%%%%%%%%%
\part{Metalogic}\label{sec:metalogic}
%%%%%%%%%%%%%%%%%%%%%%%%%%%%%%%%%%%%%%%%%%%%%%%%%%%%%%%%%%%%%%%%%%%%%%
%%%%%%%%%%%%%%%%%%%%%%%%%%%%%%%%%%%%%%%%%%%%%%%%%%%%%%%%%%%%%%%%%%%%%%

% ====================================================================
\section{Logical Inference}\label{sec:logical_inference}
% ====================================================================

\emph{Logical Inference} (or \emph{Logical Reasoning})

\begin{enumerate}
\item \emph{Premise} (\emph{Precondition}, \emph{Antecedent})

\item \emph{Material Conditional} (\emph{Corresponding Conditional},
  \emph{Implication Rule}, \emph{Inference Rule} (?)
  \S\ref{sec:inference_rule})

\item \emph{Logical Consequence} (\emph{Conclusion},
  \emph{Consequent}, \emph{Idiomatic})
\end{enumerate}

The Premise and Conclusion are \emph{Truth-bearers}
(\S\ref{sec:truth_bearer}).

\begin{description}
\item [Coherent] \hfill\\
    \S\ref{sec:paraconsistent_inference}
    \emph{Paraconsistent Inference}

\item [Consistent] \hfill\\
    \S\ref{sec:deductive_inference} \emph{Deductive Inference}

\item [Non-monotonic] \hfill\\
    \S\ref{sec:deductive_inference} \emph{Inductive Inference}

    \S\ref{sec:deductive_inference} \emph{Abductive Inference}

\end{description}



% --------------------------------------------------------------------
\subsection{Logical Argument}\label{sec:logical_argument}
% --------------------------------------------------------------------

A \emph{Logical Argument} or \emph{Derivation} (see
\S\ref{sec:formal_proof} \emph{Formal Proof}) is a sequence of Logical
Inferences.



% --------------------------------------------------------------------
\subsection{Logical Form}\label{sec:logical_form}
% --------------------------------------------------------------------

\emph{Normal Form} (\S\ref{sec:normal_form})

\emph{Argument Form}



\subsubsection{Logical Constant}\label{sec:logical_constant}

\emph{Syncategorematic Symbol} (\emph{Term Logic} \S\ref{sec:term_logic})



% --------------------------------------------------------------------
\subsection{Logical Truth}\label{sec:logical_truth}
% --------------------------------------------------------------------

\emph{Interpretation} (\emph{Valuation}, \emph{Assignment})
\S\ref{sec:interpretation}

\emph{Truth Value} (\emph{Logical Value})

\emph{Valid}, \emph{Logically True}, \emph{Tautological}

\emph{Corresponding Conditional} is \emph{Logical Truth}

\emph{Satisfiable}



\subsubsection{Truth-bearer}\label{sec:truth_bearer}

\begin{itemize}
\item Proposition
\item Type-token
\item Judgement
\end{itemize}



\subsubsection{Analytic Truth}\label{sec:analytic_truth}



% --------------------------------------------------------------------
\subsection{Logical Consequence}\label{sec:logical_consequence}
\cite{beall-restall05}
% --------------------------------------------------------------------

\emph{Logical Consequence} (\emph{Entailment}):
\[
    S \in \mathbf{L}, D \subset \mathbf{L}, S \leftrightarrow D
\]
\begin{itemize}
    \item 1. Logical Form (\S\ref{sec:logical_form})
    \item 2. A Priori
    \item 3. Modality (\S\ref{sec:modal_logic})
\end{itemize}

A Premise \emph{Logically Entails} a Conclusion if and only if the
negation of the Conclusion is \emph{Logically Inconsistent} with the
Premises.



\subsubsection{Formal Consequence}\label{sec:formal_consequence}

\emph{Syntactic Consequence} (\S\ref{sec:syntactic_consequence})

\emph{Semantic Consequence} (\S\ref{sec:semantic_consequence})

\emph{Validity}



\subsubsection{Material Consequence}

\emph{Material Consequence} (\emph{Implication})



% --------------------------------------------------------------------
\subsection{Paraconsistent Inference}\label{sec:paraconsistent_inference}
\cite{priest-tanaka-weber13}
% --------------------------------------------------------------------

\subsubsection{Paraconsistent Consequence}\label{sec:paraconsistent_consequence}

A Logical Consequence Relation is a \emph{Paraconsistent Consequence
  Relation} if and only if it is not \emph{Explosive}. Such a Relation
is required to be \emph{Coherent} (\emph{Absolutely Consistent} or
\emph{Non-trivial}), meaning that no Paraconsistent Theory
(\S\ref{sec:formal_theory}) can include all Sentences.

A Consequence Relation $\vDash$ is \emph{Explosive} when:
\[
    \forall A, B. \{A, \neg A\} \vDash B
\]
which is the case with \emph{Consistent Consequence Relations}.




% --------------------------------------------------------------------
\subsection{Deductive Inference}\label{sec:deductive_inference}
% --------------------------------------------------------------------

\emph{Top-down Logic}

\subsubsection{Deductive Consequence}\label{sec:deductive_consequence}

A \emph{Deductive Consequence} is a Modally \emph{Necessary}
Conclusion drawn from given Premises and Inference Rules.

\emph{Consistency}

\emph{Principle of Explosion}

Deductive Consequence can be extended to provide \emph{Inductive
  Support} in \emph{Inductive Consequence}
(\S\ref{sec:inductive_consequence})



\subsubsection{Mathematical Induction}\label{sec:mathematical_induction}

\emph{Mathematical Induction} as an Inference Rule is the
\emph{Implicative} process where a \emph{Base Case} is shown to extend
to the more general by means of Implication (the Inductive step). Note
that Mathematical Induction is not \emph{Inductive Reasoning} which is
an empirical or probabilistic Inference and not a form of Deduction.
The \emph{Principle of Mathematical Induction}
(\S\ref{sec:induction_principle}) is an application of Mathematical
Induction to the \emph{Natural Numbers} (\S\ref{sec:natural_number}).



% --------------------------------------------------------------------
\subsection{Inductive Inference}\label{sec:inductive_inference}
\cite{hawthorne08}
% --------------------------------------------------------------------

\emph{Universal Inductive Inference}



\subsubsection{Inductive Consequence}\label{sec:inductive_consequence}

\emph{Inductive Consequence} (\emph{Inductive Support}) is an
extension of Deductive Consequence where the Modality of Necessity is
relaxed to that of \emph{Sufficiency}.



\subsubsection{Defeasible Inference}\label{sec:defeasible_inference}

\subsubsection{Probabilistic Inference}\label{sec:probabilistic_inference}

\subsubsection{Statistical Inference}\label{sec:statistical_inference}



% --------------------------------------------------------------------
\subsection{Abductive Inference}\label{sec:abductive_inference}
% --------------------------------------------------------------------



% --------------------------------------------------------------------
\subsection{Default Inference}\label{sec:default_inference}
% --------------------------------------------------------------------
