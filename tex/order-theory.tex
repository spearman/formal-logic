%%%%%%%%%%%%%%%%%%%%%%%%%%%%%%%%%%%%%%%%%%%%%%%%%%%%%%%%%%%%%%%%%%%%%%
%%%%%%%%%%%%%%%%%%%%%%%%%%%%%%%%%%%%%%%%%%%%%%%%%%%%%%%%%%%%%%%%%%%%%%
\part{Order Theory}\label{sec:order_theory}
%%%%%%%%%%%%%%%%%%%%%%%%%%%%%%%%%%%%%%%%%%%%%%%%%%%%%%%%%%%%%%%%%%%%%%
%%%%%%%%%%%%%%%%%%%%%%%%%%%%%%%%%%%%%%%%%%%%%%%%%%%%%%%%%%%%%%%%%%%%%%

\emph{Top Element}

\emph{Lexicographic Order}



% ====================================================================
\section{Ordering Relation}\label{sec:ordering_relation}
% ====================================================================

Order Theory (\S\ref{sec:order_theory})



% --------------------------------------------------------------------
\subsection{Weak Order}\label{sec:weak_order}
% --------------------------------------------------------------------

Total Preorder

Ordered Partition

Partition Refinement %FIXME

Ordered Bell Number



% --------------------------------------------------------------------
\subsection{Preorder}\label{sec:preorder}
% --------------------------------------------------------------------

Reflexive, Transitive


\subsubsection{Cartesian Closed Preorder}\label{sec:cartesian_preorder}

Heyting Algebra (\S\ref{sec:heyting_algebra})



% --------------------------------------------------------------------
\subsection{Partial Order}\label{sec:partial_order}
% --------------------------------------------------------------------

Reflexive, Antisymmetric, Transitive

The Signature (\S\ref{sec:signature}) of Partial Orderings is
$\{\prec\}$.



\subsubsection{Strict Order}\label{sec:strict_order}

A \emph{Strict Order} is an Irreflexive Partial Order (e.g. $<$).
There is a one-to-one correspondence between Strict and Non-strict
Partial Orders via the Irreflexive Kernel
(\S\ref{sec:reflexive_reduction}):
\[
  a \leq b \wedge a \neq b \Rightarrow a < b
\]
and conversely the Reflexive Closure (\S\ref{sec:reflexive_closure}):
\[
  a < b \vee a = b \Rightarrow a \leq b
\]



\subsubsection{Least Fixed Point}\label{sec:least_fixedpoint}

\subsubsection{Greatest Fixed Point}\label{sec:greatest_fixedpoint}

\subsubsection{Pre-fixpoint}\label{sec:prefixpoint}

\subsubsection{Post-fixpoint}\label{sec:postfixpoint}



% --------------------------------------------------------------------
\subsection{Total Order}\label{sec:total_order}
% --------------------------------------------------------------------

A \emph{Total Order} (or \emph{Linear Order}) adds the requirement of
Left-totality. A Totally Ordered Subset of some Partially Ordered Set
is called a \emph{Chain}.

Totally Ordered Set (\S\ref{sec:totally_ordered})



% --------------------------------------------------------------------
\subsection{Complete Partial Order}\label{sec:complete_partialorder}
% --------------------------------------------------------------------

% --------------------------------------------------------------------
\subsection{Well-founded Relation}\label{sec:well_founded}
% --------------------------------------------------------------------

\emph{Well-founded}

Every non-empty Subset has a Minimal Element



\subsubsection{Noetherian Relation}\label{sec:noetherian_relation}

\emph{Noetherian} (or \emph{Terminating})



% --------------------------------------------------------------------
\subsection{Cofinal Subset}\label{sec:cofinal_subset}
% --------------------------------------------------------------------

A Subset $B$ is \emph{Cofinal} in a Set $A$ with an Ordering Relation
$\leq$ if:
\[
  \forall a \in A, \exists b \in B : a \leq b
\]

Cofinality (Posets \S\ref{sec:cofinality})



% --------------------------------------------------------------------
\subsection{Coinitial}\label{sec:coinitial_subset}
% --------------------------------------------------------------------

A \emph{Coinitial Subset} is the Order Theoretic Dual of a Cofinal
Subset. A Subset $B$ is Coinitial in a Set $A$ with an Ordering
Relation $\leq$ if:
\[
  \forall a \in A, \exists b \in B : b \leq a
\]



% ====================================================================
\section{Setoid}\label{sec:setoid}
% ====================================================================

A \emph{Setoid} is a Set with an Equivalence Relation.



% ====================================================================
\section{Proset}\label{sec:proset}
% ====================================================================

A \emph{Proset} is a Set, $P$, with a Preorder
(\S\ref{sec:preorder}), $\lesssim$, defined on that Set.



% --------------------------------------------------------------------
\subsection{Directed Set}\label{sec:directed_set}
% --------------------------------------------------------------------

A \emph{Directed Set} (or \emph{Directed Preorder} or \emph{Filtered
  Set}) $D$ has a Preorder $\leq$ with the extra condition that any
two Elements of $D$ have an Upper Bound (\S\ref{sec:upper_bound}):
\[
  \forall x, y \in D, \exists z \in D : x \leq z \wedge y \leq z
\]



\subsubsection{Final Map}\label{sec:final_map}

\subsubsection{Cofinal Map}\label{sec:cofinal_map}



% ====================================================================
\section{Poset}\label{sec:poset}
% ====================================================================

$\prec$

A \emph{Poset} (or \emph{Partially Ordered Set}) is a Set, $P$, with a
Partial Order (\S\ref{sec:partial_order}), $\leq$, defined on that
Set (that is a Relation which is Reflexive, Antisymmetric, and
Transitive).

An Element $x \in P$ is a \emph{Greatest Element} when:
\[
  \forall y \in P, y \leq x
\]
and likewise the \emph{Least element} when:
\[
  \forall y \in P, x \leq y
\]

An Element $x \in P$ is a \emph{Maximal Element} when:
\[
  \forall y \in P, x \nleq y
\]
and a \emph{Minimal Element} when:
\[
  \forall y \in P, y \nleq x
\]

The definition of Posets gives rise to the concept of a Monotonic
Function (\S\ref{sec:monotonic_function}), that is a Function, $f$,
between Posets which is ordered in both the Domain and the Codomain:
\[
  x \leq y \Rightarrow f(x) \leq f(y)
\]

%FIXME

Meet and Join can be defined as Commutative, Associative, and
Idempotent Partial Binary Operations on pairs of Elements of $P$.

The Join/Meet of a Totally Ordered Set is its Maximal/Minimal Element.

The \emph{Nullary Meet} is the Top Element.



% --------------------------------------------------------------------
\subsection{Upper Bound}\label{sec:upper_bound}
% --------------------------------------------------------------------

\subsubsection{Least Upper Bound}\label{sec:least_upperbound}

The \emph{Supremum} (or \emph{Least Upper Bound} (LUB) or \emph{Join})
of a Subset of a Poset, $S \subseteq P$ is the (Unique) Least Element
of $P$ that is a Greatest Element of $S$, denoted $\vee S$ or $sup
(S)$.

The Join is an example of a Coproduct (\S\ref{sec:coproduct})



\paragraph{Least-upper-bound Property}\label{sec:leastupperbound_property}



% --------------------------------------------------------------------
\subsection{Lower Bound}\label{sec:lower_bound}
% --------------------------------------------------------------------

\subsubsection{Greatest Lower Bound}\label{sec:greatest_lowerbound}

The \emph{Infimum} (or \emph{Greatest Lower Bound} (GLB) or
\emph{Meet}) of a Subset of a Poset, $S \subseteq P$, is the (Unique)
Greatest Element of $P$ that is a Least Element of $S$, denoted
$\wedge S$ or $inf (S)$.



% --------------------------------------------------------------------
\subsection{Upper Set}\label{sec:upper_set}
% --------------------------------------------------------------------

An \emph{Upper Set} (or \emph{Upset} or \emph{Upward Closed Set}) is a
Subset, $U$, of a Poset, $P$, such that:
\[
  x \in U \wedge x \leq y \Rightarrow y \in U
\]

A Topos Sieve (\S\ref{sec:topos_sieve}) is a generalization of a Lower
Set.

The Upper Set generated from a Subset $Y \subseteq X$ is denoted by
$\uparrow Y$.



\subsubsection{Filter}\label{sec:filter}

A \emph{Filter}, $\mathcal{F}$, of a Set, $X$, is a Set of Subsets of
$X$:
\[
  \mathcal{F} \subseteq \pow(X)
\]
where:
\begin{itemize}
\item if $A \in \mathcal{F}$ and $B \supseteq A$, then $B \in
  \mathcal{F}$
\item if $A \in \mathcal{F}$ and $B \in \mathcal{F}$ then $A \cap B
  \in \mathcal{F}$
\item $X \in \mathcal{F}$ and $\emptyset \notin \mathcal{F}$
\end{itemize}

Dual is an \emph{Ideal} (\S\ref{sec:order_ideal})



\subsubsection{Ultrafilter}\label{sec:ultrafilter}

An \emph{Ultrafilter} $\mathcal{U}$ is a Filter such that for all $A
\subseteq X$, either $A \in \mathcal{U}$ or $X - A \in \mathcal{U}$.



\subsubsection{Linear Continuum}\label{sec:linear_continuum}

$\reals$



% --------------------------------------------------------------------
\subsection{Lower Set}\label{sec:lower_set}
% --------------------------------------------------------------------

A \emph{Lower Set} (or \emph{Downset} or \emph{Decreasing Set} or
\emph{Downward Closed Set}) $L$, is defined by:
\[
  y \in L \wedge x \leq y \Rightarrow x \in L
\]

The Lower Set generated from a Subset $Y \subseteq X$ is denoted by
$\downarrow Y$.



\subsubsection{Ideal}\label{sec:order_ideal}

Dual of a Filter (\S\ref{sec:filter})



\paragraph{Prime Ideal Theorem}\label{sec:prime_ideal}

Boolean Prime Ideal Theorem



% --------------------------------------------------------------------
\subsection{Poset Product}\label{sec:poset_product}
% --------------------------------------------------------------------

The Product of two Elements of a Poset is their Greatest Lower Bound.

The Coproduct of two Elements of a Poset is their Least Upper Bound.



% --------------------------------------------------------------------
\subsection{Monotonic Function}\label{sec:monotonic_function}
% --------------------------------------------------------------------

A \emph{Monotonic Function} is a Function between Posets
(\S\ref{sec:poset}) where the Ordering of Elements of the Domain
Implies the Ordering of Elements in the Image of those Elements under
the Function.



\subsubsection{Order Isomorphism}\label{sec:order_isomorphism}

Surjective Order Embedding



% --------------------------------------------------------------------
\subsection{Covering Relation}\label{sec:covering_relation}
% --------------------------------------------------------------------

\subsubsection{Hasse Diagram}\label{sec:hasse_diagram}



% --------------------------------------------------------------------
\subsection{Ascending Chain Condition}\label{sec:ascending_chain}
% --------------------------------------------------------------------

A Poset satisfies the \emph{Ascending Chain Condition} if every
strictly ascending Sequence of Elements eventually terminates. The
\emph{Descending Chain Condition} is defined likewise for descending
Sequences of Elements.



% --------------------------------------------------------------------
\subsection{Cofinality}\label{sec:cofinality}
% --------------------------------------------------------------------

The \emph{Cofinality} of a Poset $A$ is the least of the Cardinalities
of the Cofinal Subsets (\S\ref{sec:cofinal_subset}) of $A$.



% --------------------------------------------------------------------
\subsection{Galois Connection}\label{sec:galois_connection}
% --------------------------------------------------------------------

Adjoint Pairs (\S\ref{sec:adjoint}) in Poset Categories

Closure Operators are Monads (\S\ref{sec:monad})



% --------------------------------------------------------------------
\subsection{$\omega$-complete Partial Order}\label{sec:omega_cpo}
% --------------------------------------------------------------------

Poset in which every $\omega$-chain (\S\ref{sec:chain}) $x_1 \leq x_2
\leq \ldots$ has a Supremum.

A \emph{Strict} $\omega$-complete Partial Order has an Initial Object
$\bot$ and Continuous Maps between Strict $\omega$-complete Partial
Orders preserve $\bot$.

The Category of $\omega$-complete Partial Orders is Cartesian Closed
(\S\ref{sec:cartesian_closed}) but the Category of Strict
$\omega$-complete Partial Orders is not.

Denotational Semantics (\S\ref{sec:denotational_semantics})



\subsubsection{Directed Complete Partial Order}\label{sec:dcpo}



% --------------------------------------------------------------------
\subsection{Duality Principle}\label{sec:duality_principle}
% --------------------------------------------------------------------



% ====================================================================
\section{Totally Ordered Set}\label{sec:totally_ordered}
% ====================================================================

A \emph{Totally Ordered Set} is a Poset whose Ordering Relation is
Total (\S\ref{sec:total_order}).



% --------------------------------------------------------------------
\subsection{Interval}\label{sec:interval}
% --------------------------------------------------------------------

\emph{Open Interval} $(0,1)$: does not include Endpoints

\emph{Closed Interval} $[0,1]$: includes Endpoints

\emph{Degenerate Interval} $[1,1]$: a Set containing a single Element

Unit Interval (\S\ref{sec:unit_interval}): $[0,1]$



% --------------------------------------------------------------------
\subsection{Chain}\label{sec:chain}
% --------------------------------------------------------------------

A Totally Ordered Subset of some Poset is called a \emph{Chain}.

\paragraph{Complete Chain}\label{sec:complete_chain}



% --------------------------------------------------------------------
\subsection{Well-order}\label{sec:well_order}
% --------------------------------------------------------------------

A \emph{Well-order} is a Total Order on a Set $S$ such that every
non-empty Subset of $S$ has a Least Element.

\begin{itemize}
  \item $\mathbb{N}$
  \item $\mathbb{Z}^{\geq n}$
\end{itemize}

Well-ordering Axioms can be used to prove the Principle of
Mathematical Induction (\S\ref{sec:mathematical_induction}) as a
Theorem.


\subsubsection{Well-ordering Theorem}\label{sec:wellorder_theorem}

Axiom of Choice (\S\ref{sec:choice_axiom})



\subsubsection{Tree} \label{sec:tree}

A \emph{Tree} is a Poset $(T,<)$ such that $\forall t \in T$, the Set
$\{s \in T : s < t \}$ is Well-ordered (\S\ref{sec:well_order}) by
$<$.



\paragraph{Prefix Order}\label{sec:prefix_order}
\hfill \\

generalized Tree (\S\ref{sec:tree})



% ====================================================================
\section{Lattice Theory}\label{sec:lattice_theory}
% ====================================================================

% --------------------------------------------------------------------
\subsection{Lattice}\label{sec:lattice}
% --------------------------------------------------------------------

%FIXME xref join, meet
A \emph{Lattice} is a Poset for which every Element has both a Join
and a Meet. For a Lattice with an underlying Set $L$, the Algebraic
Structure (\S\ref{sec:universal_algebra}) $(L, \vee, \wedge)$ has
these Axiom Identities for all $a,b,c \in L$:
\begin{description}
\item[Commutative laws]
\[
    a \vee b = b \vee a
\] \[
    a \wedge b = b \wedge a
\]
\item[Associative laws]
\[
    a \vee (b \vee c) = (a \vee b) \vee c
\] \[
    a \wedge (b \wedge c) = (a \wedge b) \wedge c
\]
\item[Absorption laws]
\[
    a \vee (a \wedge b) = a
\] \[
    a \wedge (a \vee b) = a
\]
\end{description}
The Signature (\S\ref{sec:signature}) of Lattices is
$\{\wedge, \vee\}$.



% --------------------------------------------------------------------
\subsection{Semilattice}\label{sec:semilattice}
% --------------------------------------------------------------------

A \emph{Join-semilattice} (or \emph{Upper Semilattice}) is a Poset for
which all pairs have a Join (\S\ref{sec:least_upperbound}).

A \emph{Meet-semilattice} (or \emph{Lower Semilattice}) is a Poset for
which all pairs have a Meet.



% --------------------------------------------------------------------
\subsection{Complete Lattice}\label{sec:complete_lattice}
% --------------------------------------------------------------------

A \emph{Complete Lattice} is a Lattice for which every Subset has a
Meet and a Join.



\subsubsection{Knaster-Tarski Theorem}\label{sec:knaster_tarski}

\emph{Knaster-Tarski Theorem}



\subsubsection{Quantale}\label{sec:quantale}

\emph{Complete Residuated Semigroup}

Complete Lattice $Q$ with an Associative Binary Operation $* : Q
\times Q \rightarrow Q$ satisfying:
\[
  x*(\bigvee_{i \in I} y_i) = \bigvee_{i \in I}(x * y_i)
\]
and:
\[
  (\bigvee_{i \in I} y_i)*x = \bigvee_{i \in I}(y_i * x)
\]

Quantales give an Algebraic Semantics
(\S\ref{sec:algebraic_semantics}) for Linear Logic
(\S\ref{sec:linear_logic})



% --------------------------------------------------------------------
\subsection{Free Lattice}\label{sec:free_lattice}
% --------------------------------------------------------------------

% --------------------------------------------------------------------
\subsection{Bounded Lattice}\label{sec:bounded_lattice}
% --------------------------------------------------------------------

% --------------------------------------------------------------------
\subsection{Distributive Lattice}\label{sec:distributive_lattice}
% --------------------------------------------------------------------

Join and Meet Distribute over each other



\subsubsection{Infinite Distributive Law}
\label{sec:infinite_distributive}

Finite Meets Distribute over arbitrary Joins

Frames (\S\ref{sec:frame}), Locales (\S\ref{sec:locale}), Complete
Heyting Algebras (\S\ref{sec:complete_heyting})



% --------------------------------------------------------------------
\subsection{Completely Distributive Lattice}
\label{sec:completely_distributive}
% --------------------------------------------------------------------

arbitrary Joins Distribute over arbitrary Meets (Self-dual Property)

impossible without Axiom of Choice (\S\ref{sec:choice_axiom}) for a
Complete Lattice with more than one Element

\begin{itemize}
  \item any Complete Chain (\S\ref{sec:complete_chain}) e.g. Unit
    Interval $([0,1], \leq)$)
  \item Powerset Lattice $(\pow(X), \subseteq)$ for any Set
    $X$
  \item Free Completely Distributive Lattice
    (\S\ref{sec:free_completely_distributive_lattice}) over any Poset
    $C$
\end{itemize}



\subsection{Free Completely Distributive Lattice}
\label{sec:free_completely_distributive_lattice}

over a Poset $C$



% --------------------------------------------------------------------
\subsection{Residuated Lattice}\label{sec:residuated_lattice}
% --------------------------------------------------------------------

\subsubsection{Residuated Boolean Algebra}
\label{sec:residuated_boolean_algebra}



% --------------------------------------------------------------------
\subsection{Limit-preserving Function}\label{sec:limit_preserving}
% --------------------------------------------------------------------

\subsubsection{Scott-continuity}\label{sec:scott_continuity}

Function between Posets $f : P \rightarrow Q$ such that it Preserves
all Directed Suprema %FIXME



% ====================================================================
\section{Domain Theory}\label{sec:domain_theory}
% ====================================================================

% --------------------------------------------------------------------
\subsection{Scott Domain}\label{sec:scott_domain}
% --------------------------------------------------------------------

Denotational Semantics (\S\ref{sec:denotational_semantics})
