%%%%%%%%%%%%%%%%%%%%%%%%%%%%%%%%%%%%%%%%%%%%%%%%%%%%%%%%%%%%%%%%%%%%%%%%%%%%%%%%
%%%%%%%%%%%%%%%%%%%%%%%%%%%%%%%%%%%%%%%%%%%%%%%%%%%%%%%%%%%%%%%%%%%%%%%%%%%%%%%%
\part{Order Theory}\label{sec:order_theory}
%%%%%%%%%%%%%%%%%%%%%%%%%%%%%%%%%%%%%%%%%%%%%%%%%%%%%%%%%%%%%%%%%%%%%%%%%%%%%%%%
%%%%%%%%%%%%%%%%%%%%%%%%%%%%%%%%%%%%%%%%%%%%%%%%%%%%%%%%%%%%%%%%%%%%%%%%%%%%%%%%

%TODO: should this become a section of set theory?

\emph{Top Element}

\emph{Lexicographic Order}

\fist Category Theory as Proof-relevant Order Theory
(\url{https://ncatlab.org/nlab/show/proof+relevance})



% ==============================================================================
\section{Ordering Relation}\label{sec:ordering_relation}
% ==============================================================================

Order Theory (\S\ref{sec:order_theory})

$<$, $\leq$

$\ll$

$\subset$, $\subseteq$

$\sqsubset$, $\sqsubseteq$

$\prec$, $\preceq$



% ------------------------------------------------------------------------------
\subsection{Weak Order}\label{sec:weak_order}
% ------------------------------------------------------------------------------

Total Preorder

Ordered Partition

Partition Refinement %FIXME

Ordered Bell Number



% ------------------------------------------------------------------------------
\subsection{Preorder}\label{sec:preorder}
% ------------------------------------------------------------------------------

Reflexive, Transitive

Poset without Antisymmetry

Thin Category (\S\ref{sec:thin_category})

a Category Enriched (\S\ref{sec:enriched_category}) over Truth Values
is a Preorder

any Category with at most one Morphism between any two Objects is a Preorder

the number of possible Preorders on a Finite Set is equal to the number of
possible distinct Topologies (\S\ref{sec:finite_topological_space}) on the Set



\subsubsection{Cartesian Closed Preorder}\label{sec:cartesian_preorder}

Heyting Algebra (\S\ref{sec:heyting_algebra})



% ------------------------------------------------------------------------------
\subsection{Partial Order}\label{sec:partial_order}
% ------------------------------------------------------------------------------

Reflexive, Antisymmetric, Transitive

The Signature (\S\ref{sec:signature}) of Partial Orderings is
$\{\prec\}$.

The Reachability Relation (\S\ref{sec:reachability}) of an Acyclic
Directed Graph (\S\ref{sec:dag}) is a Partial Order.

2018 - Banerjea - \emph{Time is Partial} -
\url{http://shader-playground.timjones.io/} -- Distributed Consistency Models,
Weak Memory Models

the number of possible Partial Orders on a Finite Set is equal to the number of
possible Distinguishable ($T_0$) Topologies
(\S\ref{sec:distinguishable_space}) on the Set



\subsubsection{Strict Order}\label{sec:strict_order}

A \emph{Strict Order} is an Irreflexive Partial Order (e.g. $<$).
There is a one-to-one correspondence between Strict and Non-strict
Partial Orders via the Irreflexive Kernel
(\S\ref{sec:reflexive_reduction}):
\[
  a \leq b \wedge a \neq b \Rightarrow a < b
\]
and conversely the Reflexive Closure (\S\ref{sec:reflexive_closure}):
\[
  a < b \vee a = b \Rightarrow a \leq b
\]



\subsubsection{Pre-fixpoint}\label{sec:prefixpoint}

\subsubsection{Post-fixpoint}\label{sec:postfixpoint}



% ------------------------------------------------------------------------------
\subsection{Total Order}\label{sec:total_order}
% ------------------------------------------------------------------------------

A \emph{Total Order} (or \emph{Linear Order}) adds the requirement of
Left-totality (\S\ref{sec:left_total}).

Totally Ordered Set (\S\ref{sec:totally_ordered})

Ordering Principle (\S\ref{sec:ordering_principle})



\subsubsection{Well-order}\label{sec:well_order}

A \emph{Well-order} is a Total Order such that every non-empty Subset
of the Domain has a Least Element.

Well-ordered Set (\S\ref{sec:wellordered_set})

\begin{itemize}
  \item $\mathbb{N}$
  \item $\mathbb{Z}^{\geq n}$
\end{itemize}

Well-ordering Axioms can be used to prove the Principle of
Mathematical Induction (\S\ref{sec:mathematical_induction}) as a
Theorem.

Strict Well-order $\leftrightarrow$ Well-founded
(\S\ref{sec:well_founded}) Strict Total Order



% ------------------------------------------------------------------------------
\subsection{Complete Partial Order}\label{sec:complete_partialorder}
% ------------------------------------------------------------------------------

% ------------------------------------------------------------------------------
\subsection{Cyclic Order}\label{sec:cyclic_order}
% ------------------------------------------------------------------------------

unlike other Ordering Relations, a Cyclic Order is a \emph{Ternary Relation}
rather than a Binary Relation

Cycle (\S\ref{sec:cycle})

a Directed Cyclic Order (\S\ref{sec:directed_cycle}) is called a \emph{Necklace}



% ------------------------------------------------------------------------------
\subsection{Linear Extension}\label{sec:linear_extension}
% ------------------------------------------------------------------------------

\subsubsection{Order-extension Principle}
\label{sec:order_extension_principle}

Every Partial Order can be Extended to a Total Order



% ------------------------------------------------------------------------------
\subsection{Well-founded Relation}\label{sec:well_founded}
% ------------------------------------------------------------------------------

\emph{Well-founded}

Every non-empty Subset has a Minimal Element

Well-founded Set (\S\ref{sec:wellfounded_set})

Well-founded Graph (\S\ref{sec:wellfounded_graph})

Well-order (\S\ref{sec:well_order})



\subsubsection{Noetherian Relation}\label{sec:noetherian_relation}

\emph{Noetherian} (or \emph{Terminating})



% ------------------------------------------------------------------------------
\subsection{Cofinal Subset}\label{sec:cofinal_subset}
% ------------------------------------------------------------------------------

A Subset $B$ is \emph{Cofinal} in a Set $A$ with an Ordering Relation
$\leq$ if:
\[
  \forall a \in A, \exists b \in B : a \leq b
\]

Cofinality (Posets \S\ref{sec:cofinality})

cf. Archimedean Property (\S\ref{sec:archimedean_property})



% ------------------------------------------------------------------------------
\subsection{Coinitial}\label{sec:coinitial_subset}
% ------------------------------------------------------------------------------

A \emph{Coinitial Subset} is the Order Theoretic Dual of a Cofinal
Subset. A Subset $B$ is Coinitial in a Set $A$ with an Ordering
Relation $\leq$ if:
\[
  \forall a \in A, \exists b \in B : b \leq a
\]



% ==============================================================================
\section{Setoid}\label{sec:setoid}
% ==============================================================================

A \emph{Setoid} is a Set with an Equivalence Relation.



% ==============================================================================
\section{Proset}\label{sec:proset}
% ==============================================================================

A \emph{Proset} is a Set, $P$, with a Preorder
(\S\ref{sec:preorder}), $\lesssim$, defined on that Set.



% ------------------------------------------------------------------------------
\subsection{Directed Set}\label{sec:directed_set}
% ------------------------------------------------------------------------------

A \emph{Directed Set} (or \emph{Directed Preorder} or \emph{Filtered
  Set}) $D$ has a Preorder $\leq$ with the extra condition that any
two Elements of $D$ have an Upper Bound (\S\ref{sec:upper_bound}):
\[
  \forall x, y \in D, \exists z \in D : x \leq z \wedge y \leq z
\]

\fist A \emph{Net} (Topology \S\ref{sec:net}) is a Function from a Directed Set
into a Topological Space (\S\ref{sec:topological_space})

\fist generalized as Filtered Categories (\S\ref{sec:filtered_category})



\subsubsection{Final Map}\label{sec:final_map}

\subsubsection{Cofinal Map}\label{sec:cofinal_map}



% ==============================================================================
\section{Poset}\label{sec:poset}
% ==============================================================================

$\prec$

A \emph{Poset} (or \emph{Partially Ordered Set}) is a Set, $P$, with a
Partial Order (\S\ref{sec:partial_order}), $\leq$, defined on that
Set (that is a Relation which is Reflexive, Antisymmetric, and
Transitive).

a Preorder (\S\ref{sec:preorder}) with Antisymmetry

a Poset can be seen as a ``Decategorified Category'', i.e. a Category where any
two Objects have \emph{at most} one Morphism between them -
\url{http://comonad.com/reader/2018/computational-quadrinitarianism-curious-correspondences-go-cubical/}

(Corfield2018): Partially Ordered Sets can be considered as Categories
(\S\ref{sec:category}) ``enriched'' in Truth Values (\S\ref{sec:truth_value})
(FIXME: clarify)

a Skeletal (\S\ref{sec:skeletal_category}) Thin (\S\ref{sec:thin_category})
Category is a Poset

in a Category arising from a Poset $(P,\leq)$, \emph{Adjoint Pairs}
(\S\ref{sec:adjunction}) are \emph{Galois Connections}
(\S\ref{sec:galois_connection}) and \emph{Monads} (\S\ref{sec:monad}) are
\emph{Closure Operators} (TODO: xref)

An Element $x \in P$ is a \emph{Greatest Element} when:
\[
  \forall y \in P, y \leq x
\]
and likewise the \emph{Least element} when:
\[
  \forall y \in P, x \leq y
\]

An Element $x \in P$ is a \emph{Maximal Element} when:
\[
  \forall y \in P, x \nleq y
\]
and a \emph{Minimal Element} when:
\[
  \forall y \in P, y \nleq x
\]

The definition of Posets gives rise to the concept of a Monotonic
Function (\S\ref{sec:monotonic_function}), that is a Function, $f$,
between Posets which is ordered in both the Domain and the Codomain:
\[
  x \leq y \Rightarrow f(x) \leq f(y)
\]

Order Isomorphism (\S\ref{sec:order_isomorphism}): notion of
Isomorphism for Posets

%FIXME

Meet and Join can be defined as Commutative, Associative, and
Idempotent Partial Binary Operations on pairs of Elements of $P$.

The Join/Meet of a Totally Ordered Set is its Maximal/Minimal Element.

The \emph{Nullary Meet} is the Top Element.

\fist Algebraic Semantics (\S\ref{sec:algebraic_semantics}): the Logical
Connectives $\vee$ and $\wedge$ can be regarded as Axioms on a Poset with
$\wedge$ as Meet (GLB) and $\vee$ as Join (LUB), and the laws of Classical
Logic corresponding to requiring that such a Poset is a Boolean Algebra
(\S\ref{sec:boolean_algebra}); a Proof in Propositional Logic shows an Equation
that must hold in all Boolean Algebras

\fist \emph{Stone Duality} (\S\ref{sec:stone_duality}) -- Dualities between
Topological Spaces (\S\ref{sec:topological_space}) and Partially Ordered Sets



% ------------------------------------------------------------------------------
\subsection{Bounded Set}\label{sec:bounded_set}
% ------------------------------------------------------------------------------

\begin{itemize}
  \item Interval Arithmetic (\S\ref{sec:interval_arithmetic}) -- approach on
    putting Bounds on Rounding Errors and Measurement Errors in Numerical
    Analysis
\end{itemize}

see also:
\begin{itemize}
  \item Bounded Lattice (\S\ref{sec:bounded_lattice})
  \item Bounded Sequence (\S\ref{sec:bounded_sequence})
  \item Bounded Function (\S\ref{sec:bounded_function})
  \item Bounded Subset (Topology \S\ref{sec:bounded_subset})
\end{itemize}



\subsubsection{Upper Bound}\label{sec:upper_bound}



\paragraph{Least Upper Bound}\label{sec:least_upperbound}\hfill

The \emph{Supremum} (or \emph{Least Upper Bound} (LUB) or \emph{Join})
of a Subset of a Poset, $S \subseteq P$ is the (Unique) Least Element
of $P$ that is a Greatest Element of $S$, denoted $\vee S$ or $sup
(S)$.

The Join is an example of a Coproduct (\S\ref{sec:coproduct})

Axiomatic definition of the Real Numbers: the Real Numbers are the Unique up to
Isomorphism Dedekind-complete (Least-upper-bound Property) Ordered Field
(\S\ref{sec:ordered_field}) $(\reals, +, *, <)$



\subparagraph{Least-upper-bound Property}
\label{sec:leastupperbound_property}\hfill



\subsubsection{Lower Bound}\label{sec:lower_bound}

\paragraph{Greatest Lower Bound}\label{sec:greatest_lowerbound}\hfill

The \emph{Infimum} (or \emph{Greatest Lower Bound} (GLB) or
\emph{Meet}) of a Subset of a Poset, $S \subseteq P$, is the (Unique)
Greatest Element of $P$ that is a Least Element of $S$, denoted
$\wedge S$ or $inf (S)$.



% ------------------------------------------------------------------------------
\subsection{Upper Set}\label{sec:upper_set}
% ------------------------------------------------------------------------------

An \emph{Upper Set} (or \emph{Upset} or \emph{Upward Closed Set}) is a
Subset, $U$, of a Poset, $P$, such that:
\[
  x \in U \wedge x \leq y \Rightarrow y \in U
\]

A Topos Sieve (\S\ref{sec:topos_sieve}) is a generalization of a Lower
Set.

The Upper Set generated from a Subset $Y \subseteq X$ is denoted by
$\uparrow Y$.



\subsubsection{Filter}\label{sec:filter}

A \emph{Filter}, $\mathcal{F}$, of a Set, $X$, is a Set of Subsets of
$X$:
\[
  \mathcal{F} \subseteq \pow(X)
\]
where:
\begin{itemize}
\item if $A \in \mathcal{F}$ and $B \supseteq A$, then $B \in
  \mathcal{F}$
\item if $A \in \mathcal{F}$ and $B \in \mathcal{F}$ then $A \cap B
  \in \mathcal{F}$
\item $X \in \mathcal{F}$ and $\emptyset \notin \mathcal{F}$
\end{itemize}

Dual is an \emph{Ideal} (\S\ref{sec:order_ideal})

cf. Filtration (\S\ref{sec:filtration})



\subsubsection{Ultrafilter}\label{sec:ultrafilter}

An \emph{Ultrafilter} $\mathcal{U}$ is a Filter such that for all $A
\subseteq X$, either $A \in \mathcal{U}$ or $X - A \in \mathcal{U}$.

an Ultrafilter on a Set $X$ is like a Measure on $X$ where the Elements of the
Ultrafilter have Measure $1$ and every other Subset of $X$ has Measure $0$

Ultrafilters on a Set $X$ correspond one-to-one with the Finitely Additive
Probability Measures (\S\ref{sec:probability_measure}) on $X$
--\url{https://golem.ph.utexas.edu/category/2012/09/where_do_ultrafilters_come_fro.html}



\subsubsection{Linear Continuum}\label{sec:linear_continuum}

$\reals$

cf. \emph{Cantor-Dedekind Axiom} -- thesis that the Real Numbers
(\S\ref{sec:real_number}) are Order-isomorphic to the Linear Continuum (Order
Theory \S\ref{sec:linear_continuum}) in Geometry; a consequence is that the
Decidability of the Ordered Real Field can be seen as an Algorithm to solve any
problem in Euclidean Geometry (\S\ref{sec:euclidean_geometry})



% ------------------------------------------------------------------------------
\subsection{Lower Set}\label{sec:lower_set}
% ------------------------------------------------------------------------------

A \emph{Lower Set} (or \emph{Downset} or \emph{Decreasing Set} or
\emph{Downward Closed Set}) $L$, is defined by:
\[
  y \in L \wedge x \leq y \Rightarrow x \in L
\]

The Lower Set generated from a Subset $Y \subseteq X$ is denoted by
$\downarrow Y$.



\subsubsection{Ideal}\label{sec:order_ideal}

Dual of a Filter (\S\ref{sec:filter})



\paragraph{Maximal Ideal}\label{sec:maximal_ideal}\hfill

cf. Maximal Ring Ideal (\S\ref{sec:maximal_ring_ideal})



\paragraph{Prime Ideal}\label{sec:order_prime_ideal}\hfill

\fist using Isomorphism of the Categories of Boolean Algebras and Boolean Rings
the notion of Prime Ideal of a Poset coincides with the notion of Prime Ideal
for Rings (\S\ref{sec:prime_ideal})



\subparagraph{Prime Ideal Theorem}\label{sec:prime_ideal_theorem}\hfill

Boolean Prime Ideal Theorem



% ------------------------------------------------------------------------------
\subsection{Poset Product}\label{sec:poset_product}
% ------------------------------------------------------------------------------

The Product of two Elements of a Poset is their Greatest Lower Bound.

The Coproduct of two Elements of a Poset is their Least Upper Bound.



% ------------------------------------------------------------------------------
\subsection{Compact Element}\label{sec:compact_element}
% ------------------------------------------------------------------------------

% ------------------------------------------------------------------------------
\subsection{Monotonic Function}\label{sec:monotonic_function}
% ------------------------------------------------------------------------------

A \emph{Monotonic Function} is a Function between Posets (\S\ref{sec:poset})
where the Ordering of Elements of the Domain Implies the Ordering of Elements in
the Image of those Elements under the Function.



\subsubsection{Order Isomorphism}\label{sec:order_isomorphism}

Isomorphism for Posets

Surjective Order Embedding

if the General Continuum Hypothesis (\S\ref{sec:generalized_continuum}) holds
then all Real Closed Fields (\S\ref{sec:real_closed}) with Cardinality of the
Continuum and having the $\eta_1$ Property (\S\ref{sec:eta_set}) are Order
Isomorphic



% ------------------------------------------------------------------------------
\subsection{Covering Relation}\label{sec:covering_relation}
% ------------------------------------------------------------------------------

Transitive Reduction (\S\ref{sec:transitive_reduction_graph}) of a
Directed Acyclic Graph (\S\ref{sec:dag})



\subsubsection{Hasse Diagram}\label{sec:hasse_diagram}



% ------------------------------------------------------------------------------
\subsection{Ascending Chain Condition}\label{sec:ascending_chain}
% ------------------------------------------------------------------------------

A Poset satisfies the \emph{Ascending Chain Condition} if every
strictly ascending Sequence of Elements eventually terminates. The
\emph{Descending Chain Condition} is defined likewise for descending
Sequences of Elements.



% ------------------------------------------------------------------------------
\subsection{Cofinality}\label{sec:cofinality}
% ------------------------------------------------------------------------------

The \emph{Cofinality} of a Poset $A$ is the least of the Cardinalities
of the Cofinal Subsets (\S\ref{sec:cofinal_subset}) of $A$.



% ------------------------------------------------------------------------------
\subsection{Galois Connection}\label{sec:galois_connection}
% ------------------------------------------------------------------------------

Closure Operators as examples of \emph{Monads} (\S\ref{sec:monad}):
\emph{Adjoint Pairs} (\S\ref{sec:adjunction}) are Galois Connections and Monads
are Closure Operators (TODO: xref)



% ------------------------------------------------------------------------------
\subsection{Chain-complete Partial Order}\label{sec:cpo}
% ------------------------------------------------------------------------------

Every Chain (\S\ref{sec:chain}) has a Supremum

Admissible Relation: Preserves Chain-completeness



% ------------------------------------------------------------------------------
\subsection{$\omega$-complete Partial Order}\label{sec:omega_cpo}
% ------------------------------------------------------------------------------

Poset in which every $\omega$-chain (Countable Chain
\S\ref{sec:chain}) $x_1 \leq x_2 \leq \ldots$ has a Supremum.

A \emph{Strict} $\omega$-complete Partial Order has an Initial Object
$\bot$ and Continuous Maps between Strict $\omega$-complete Partial
Orders preserve $\bot$.

The Category of $\omega$-complete Partial Orders is Cartesian Closed
(\S\ref{sec:cartesian_closed}) but the Category of Strict
$\omega$-complete Partial Orders is not.

Denotational Semantics (\S\ref{sec:denotational_semantics})

Least Element: State of no Information, Output of a Computation that
does not Return any Result

A $\omega$-cpo is \emph{Flat} if no two distinct Elements are
Comparable



\subsubsection{Directed Complete Partial Order}\label{sec:dcpo}

Directed Set (\S\ref{sec:directed_set})

Domain Theory (\S\ref{sec:domain_theory})

Category of DCPOs is Cartesian Closed



% ------------------------------------------------------------------------------
\subsection{Duality Principle}\label{sec:duality_principle}
% ------------------------------------------------------------------------------



% ==============================================================================
\section{Totally Ordered Set}\label{sec:totally_ordered}
% ==============================================================================

A \emph{Totally Ordered Set} is a Poset whose Ordering Relation is
Total (\S\ref{sec:total_order}).

\fist Ordinary Generating Functions (OGFs \S\ref{sec:generating_function})
describe Structures on Totally Ordered Sets; cf. Exponential Generating
Functions (EGFs) describe structures on unordered Sets



% ------------------------------------------------------------------------------
\subsection{Interval}\label{sec:interval}
% ------------------------------------------------------------------------------

\emph{Open Interval} $(0,1)$: does not include Endpoints

$(-\infty, \infty)$, $(-\infty,a)$, $(a,\infty)$, $(a,b)$

\emph{Closed Interval} $[0,1]$: includes Endpoints

\emph{Degenerate Interval} $[1,1]$: a Set containing a single Element

Unit Interval (\S\ref{sec:unit_interval}): $[0,1]$

Connected Set (\S\ref{sec:connected_set})



% ------------------------------------------------------------------------------
\subsection{Chain}\label{sec:chain}
% ------------------------------------------------------------------------------

A Totally Ordered Subset of some Poset is called a \emph{Chain}.



\subsubsection{Complete Chain}\label{sec:complete_chain}



% ------------------------------------------------------------------------------
\subsection{Ordering Principle}\label{sec:ordering_principle}
% ------------------------------------------------------------------------------

Every Set can be Totally Ordered

Well-ordering Theorem (\S\ref{sec:wellorder_theorem})



% ------------------------------------------------------------------------------
\subsection{Well-ordered Set}\label{sec:wellordered_set}
% ------------------------------------------------------------------------------

A \emph{Well-ordered Set} is a Totally Ordered Set $S$ such that every
Non-empty (possibly Infinite) Subset of $S$ has a Least Element.

Well-order (\S\ref{sec:well_order})

The Ordinal Numbers (\S\ref{sec:ordinal_number}) are defined as
Isomorphism Classes of Well-ordered Sets.

The Well-ordering Theorem (\S\ref{sec:wellorder_theorem}) states that
every Set can be Well-ordered (as a consequence of the Axiom of
Choice).

\begin{itemize}
  \item $\mathbb{N}$
  \item $\mathbb{Z}^{\geq n}$
\end{itemize}



\subsubsection{Well-ordering Theorem}\label{sec:wellorder_theorem}

Axiom of Choice (\S\ref{sec:choice_axiom}): every Set can be
Well-ordered

Ordering Principle (\S\ref{sec:ordering_principle})



\subsubsection{Tree} \label{sec:tree}

A \emph{Tree} is a Poset $(T,<)$ such that $\forall t \in T$, the Set
$\{s \in T : s < t \}$ is Well-ordered (\S\ref{sec:well_order}) by
$<$.



\paragraph{Prefix Order}\label{sec:prefix_order}\hfill

generalized Tree



% ------------------------------------------------------------------------------
\subsection{$\eta$ Set}\label{sec:eta_set}
% ------------------------------------------------------------------------------

$\eta_1$ -- between any two Real Numbers there exists another (FIXME: correct
???)

if the General Continuum Hypothesis (\S\ref{sec:generalized_continuum}) holds
then all Real Closed Fields (\S\ref{sec:real_closed}) with Cardinality of the
Continuum and having the $\eta_1$ Property are Order Isomorphic
(\S\ref{sec:order_isomorphism})

without the Continuum Hypothesis, if the Cardinality of the Continuum is
$\aleph_\beta$, then there is a Unique $\eta_\beta$ Field of Size $\eta_beta$



% ==============================================================================
\section{Lattice Theory}\label{sec:lattice_theory}
% ==============================================================================

% ------------------------------------------------------------------------------
\subsection{Lattice}\label{sec:lattice}
% ------------------------------------------------------------------------------

%FIXME xref join, meet
A \emph{Lattice} is a Poset for which every Element has both a Join
and a Meet. For a Lattice with an underlying Set $L$, the Algebraic
Structure (\S\ref{sec:universal_algebra}) $(L, \vee, \wedge)$ has
these Axiom Identities for all $a,b,c \in L$:
\begin{description}
\item[Commutative laws]
\[
    a \vee b = b \vee a
\] \[
    a \wedge b = b \wedge a
\]
\item[Associative laws]
\[
    a \vee (b \vee c) = (a \vee b) \vee c
\] \[
    a \wedge (b \wedge c) = (a \wedge b) \wedge c
\]
\item[Absorption laws]
\[
    a \vee (a \wedge b) = a
\] \[
    a \wedge (a \vee b) = a
\]
\end{description}
The Signature (\S\ref{sec:signature}) of Lattices is
$\{\wedge, \vee\}$.

cf. Lattice Group (\S\ref{sec:lattice_group}), Lattice Graph
(\S\ref{sec:lattice_graph})



% ------------------------------------------------------------------------------
\subsection{Semilattice}\label{sec:semilattice}
% ------------------------------------------------------------------------------

A \emph{Join-semilattice} (or \emph{Upper Semilattice}) is a Poset for
which all pairs have a Join (\S\ref{sec:least_upperbound}).

A \emph{Meet-semilattice} (or \emph{Lower Semilattice}) is a Poset for
which all pairs have a Meet.

Finite-powerset Construction as the Free Semilattice
(\S\ref{sec:free_semilattice})



% ------------------------------------------------------------------------------
\subsection{Complete Lattice}\label{sec:complete_lattice}
% ------------------------------------------------------------------------------

A \emph{Complete Lattice} is a Lattice for which every Subset has a
Meet and a Join.



\subsubsection{Knaster-Tarski Theorem}\label{sec:knaster_tarski}

\emph{Knaster-Tarski Theorem}



\subsubsection{Quantale}\label{sec:quantale}

\emph{Complete Residuated Semigroup}

Complete Lattice $Q$ with an Associative Binary Operation $* : Q
\times Q \rightarrow Q$ satisfying:
\[
  x*(\bigvee_{i \in I} y_i) = \bigvee_{i \in I}(x * y_i)
\]
and:
\[
  (\bigvee_{i \in I} y_i)*x = \bigvee_{i \in I}(y_i * x)
\]

Quantales give an Algebraic Semantics
(\S\ref{sec:algebraic_semantics}) for Linear Logic
(\S\ref{sec:linear_logic})

Continuity Space (\S\ref{sec:continuity_space})



\subsubsection{Morphology}\label{sec:morphology}



% ------------------------------------------------------------------------------
\subsection{Free Lattice}\label{sec:free_lattice}
% ------------------------------------------------------------------------------

% ------------------------------------------------------------------------------
\subsection{Free Semilattice}\label{sec:free_semilattice}
% ------------------------------------------------------------------------------

Finite-powerset Construction as the Free Semilattice %FIXME

cf. Powerdomain Construction (\S\ref{sec:power_domain}) as the Free
Models of Theories of Non-determinism



% ------------------------------------------------------------------------------
\subsection{Bounded Lattice}\label{sec:bounded_lattice}
% ------------------------------------------------------------------------------

\subsubsection{Complemented Lattice}\label{sec:complemented_lattice}

Complement

Orthocomplementation

every Complemented Distributive Lattice (\S\ref{sec:distributive_lattice}) has a
unique Orthocomplementation and is a Boolean Algebra
(\S\ref{sec:boolean_algebra})

(Birkhoff35): any Finite-dimensional Complemented Modular Lattice
(\S\ref{sec:modular_lattice}) is the Direct Product of a Finite number of
Abstract Projective Geometries (\S\ref{sec:projective_geometry}) and a Finite
Boolean Algebra, and such a
Lattice is a single Projective Geometry if and only if it is Irreducible
(FIXME: correct ?; see Birkhoff, Von Neumann36 \S 12)



% ------------------------------------------------------------------------------
\subsection{Modular Lattice}\label{sec:modular_lattice}
% ------------------------------------------------------------------------------

\emph{Modular Law}:
\[
  a \leq b \Rightarrow a \vee (x \wedge b) = (a \vee x) \wedge b
\]

Quantum Logic (\S\ref{sec:quantum_logic}) -- the Physical Qualities (Equivalence
Classes of Experimental Properties) attributable to any Quantum-mechanical
System (\S\ref{sec:quantum_system}) form a Modular Lattice

(Birkhoff35): any Finite-dimensional Complemented
(\S\ref{sec:complemented_lattice}) Modular Lattice is the Direct Product of a
Finite number of Abstract Projective Geometries
(\S\ref{sec:projective_geometry}) and a Finite Boolean Algebra
(\S\ref{sec:boolean_algebra}), and such a
Lattice is a single Projective Geometry if and only if it is Irreducible
(FIXME: correct ?; see Birkhoff, Von Neumann36 \S 12)



\subsubsection{Distributive Lattice}\label{sec:distributive_lattice}

Join and Meet Distribute over each other

every Complemented (\S\ref{sec:complemented_lattice}) Distributive Lattice has a
unique Orthocomplementation and is a Boolean Algebra
(\S\ref{sec:boolean_algebra})



\paragraph{Infinite Distributive Law}\label{sec:infinite_distributive}\hfill

Finite Meets Distribute over arbitrary Joins

Frames (\S\ref{sec:frame}), Locales (\S\ref{sec:locale}), Complete
Heyting Algebras (\S\ref{sec:complete_heyting_algebra})



\paragraph{Completely Distributive Lattice}
\label{sec:completely_distributive}\hfill

arbitrary Joins Distribute over arbitrary Meets (Self-dual Property)

impossible without Axiom of Choice (\S\ref{sec:choice_axiom}) for a
Complete Lattice with more than one Element

\begin{itemize}
  \item any Complete Chain (\S\ref{sec:complete_chain}) e.g. Unit
    Interval $([0,1], \leq)$)
  \item Powerset Lattice $(\pow(X), \subseteq)$ for any Set
    $X$
  \item Free Completely Distributive Lattice
    (\S\ref{sec:free_completely_distributive_lattice}) over any Poset
    $C$
\end{itemize}



\subparagraph{Free Completely Distributive Lattice}
\label{sec:free_completely_distributive_lattice}\hfill

over a Poset $C$



% ------------------------------------------------------------------------------
\subsection{Residuated Lattice}\label{sec:residuated_lattice}
% ------------------------------------------------------------------------------

\subsubsection{Residuated Boolean Algebra}
\label{sec:residuated_boolean_algebra}



% ------------------------------------------------------------------------------
\subsection{Limit-preserving Function}\label{sec:limit_preserving}
% ------------------------------------------------------------------------------

\subsubsection{Scott-continuity}\label{sec:scott_continuity}

Function between Posets $f : P \rightarrow Q$ such that it Preserves
all Directed Suprema %FIXME



% ==============================================================================
\section{Domain Theory}\label{sec:domain_theory}
% ==============================================================================

Abramsky-Jung94 -- \emph{Domain Theory}

Denotational Semantics (\S\ref{sec:denotational_semantics})

Metric Spaces (\S\ref{sec:metric_space})

Directed (Sub-)Sets (\S\ref{sec:directed_set}) in Domain Theory,
Sequences in Metric Spaces

Least Upper-bound (\S\ref{sec:least_upperbound}), Limit
(\S\ref{sec:sequence_limit})

Directed Complete Partial Orders (\S\ref{sec:dcpo})

Denotational Semantics of the Actor Model (\S\ref{sec:actor_model})



% ------------------------------------------------------------------------------
\subsection{Domain}\label{sec:order_domain}
% ------------------------------------------------------------------------------

Special kind of Poset (\S\ref{sec:poset})

Information Ordering



% ------------------------------------------------------------------------------
\subsection{Scott Domain}\label{sec:scott_domain}
% ------------------------------------------------------------------------------

Denotational Semantics (\S\ref{sec:denotational_semantics}) of
$\lambda$-calculus (\S\ref{sec:untyped_lambda})

Scott-continuous Function (\S\ref{sec:scott_continuity})



% ------------------------------------------------------------------------------
\subsection{Power Domain}\label{sec:power_domain}
% ------------------------------------------------------------------------------

Denotational Semantics of:
\begin{itemize}
  \item Non-deterministic Computation
    (\S\ref{sec:nondeterministic_computation})
  \item Concurrent Computation (\S\ref{sec:concurrent_computation})
\end{itemize}

a Non-deterministic Function may be described as a Deterministic
Set-valud Function where each Set contains all the Values a the
Non-deterministic Function can yield for a given Argument

in a Concurrent Computation the Power Domain expresses the Set of all
possible Computations

\fist Vietoris Topology (\S\ref{sec:vietoris_topology})

Powerdomain Constructions understood abstractly as Free-models
(\S\ref{sec:free_model}) of Theories of Non-determinism; cf. the
Finite-powerset Construction as the Free Semilattice
(\S\ref{sec:free_semilattice})

a different Theory of Non-determinism gives rise to a different
Powerdomain Construction


\textbf{Power Domains as Free Models of Theories of Non-determinism}

variations of the Theory of Semilattices (\S\ref{sec:semilattice})

\fist note: not Algebraic Theories because some involve the Order of
the underlying Domain %FIXME

``Domain'' -- some Ordered structure

``Continuous Function'' -- some kind of Limit-preserving Function

all Theories have a single Sort $X$ and one Binary Operation:
\[
  \cup : X \times X \rightarrow X
\]
takes a pair and returns the Non-deterministic Choice of one of them

Plotkin Powertheory Axioms:
\begin{enumerate}
  \item \emph{Idempotency}: $x \cup x = x$
  \item \emph{Commutativity}: $x \cup y = y \cup x$
  \item \emph{Associativity}: $(x \cup y) \cup z = x \cup (y \cup z)$
\end{enumerate}
a Model of the Plotkin Powertheory is a Continuous Semilattice with a
Carrier Domain for which the Operation is Continuous (not necessarily
a Meet or Join Operation for the Order of the Domain); a Homomorphism
of Continuous Semilattices is a Continuous Function between their
Carriers that respects the Lattice Operator

a \emph{Plotkin Powerdomain} is a Model (when it exists) $P(D)$ on a
Domain $D$ is the Free Model of the Plotkin Powertheory over $D$,
equipped with a Continuous Function $X \rightarrow P(D)$ such that for
any other Continuous Semilattice $L$ over $D$, there is a unique
Continuous Semilattice Homomorphism $P(D) \rightarrow L$ making the
Diagram Commute:
\[
  X \rightarrow P(D) \rightarrow L \equiv X \rightarrow L
\]
%FIXME: the wikipedia article:
% https://en.wikipedia.org/wiki/Power_domains#Power_domains_as_free_models
% states that the diagram is ``evident'', although it doesn't
% explicitly mention X -> L which appears to be the other ``apparent''
% edge to the diagram

Lower (Hoare) Powertheory adds to the Plotkin Powertheory the
Inequality:
\begin{itemize}
  \item $x \leq x \cup y$
\end{itemize}
a Model of the Lower Powertheory is called an \emph{Inflationary
  Semilattice} with the additional requirement that the Operator
behaves ``somewhat like'' a Join for the Order %FIXME

Upper (Smyth) Powertheory adds to the Plotkin Powertheory the
Inequality:
\begin{itemize}
  \item $x \cup y \leq x$
\end{itemize}
a Model of the Upper Powertheory is called an \emph{Deflationary
  Semilattice} with the additional requirement that the Operator
behaves ``somewhat like'' a Meet for the Order %FIXME


\textbf{Power Domains for Concurrency}

\emph{Clinger Powerdomain} (\S\ref{sec:clinger_powerdomain}) -- Actor
Model

\emph{Timed-diagrams Powerdomain}
(\S\ref{sec:timed_diagrams_powerdomain}) -- Actor Model



\subsubsection{Clinger Powerdomain}\label{sec:clinger_powerdomain}

Clinger81 \emph{Foundations of Actor Semantics}

\fist Actor Model (\S\ref{sec:actor_model})



\subsubsection{Timed Diagrams Powerdomain}
\label{sec:timed_diagrams_powerdomain}

Hewitt06 \emph{What is Commitment? Physical, Organizational, and
  Social}

\fist Timed Diagrams Model (\S\ref{sec:timed_diagrams})



% ------------------------------------------------------------------------------
\subsection{Way-below Relation}\label{sec:waybelow_relation}
% ------------------------------------------------------------------------------

% ------------------------------------------------------------------------------
\subsection{Valuation}\label{sec:domain_valuation}
% ------------------------------------------------------------------------------

a map from the Class of Open Sets of a Topological Space to the Set of Positive
Reals including Infinity

\fist cf. \emph{Measure} (\S\ref{sec:measure}) -- a Borel Measure
(\S\ref{sec:borel_measure}) always restricts to a Valuation



% ------------------------------------------------------------------------------
\subsection{Continuity Space}\label{sec:continuity_space}
% ------------------------------------------------------------------------------



% ==============================================================================
\section{Formal Concept Analysis (FCA)}\label{sec:fca}
% ==============================================================================

%FIXME: move this section to information theory ???

\url{https://johncarlosbaez.wordpress.com/2020/05/07/formal-concepts/}

\url{http://www.upriss.org.uk/fca/fca.html}

\emph{Concept Hierarchy} (or \emph{Formal Ontology})

\url{https://golem.ph.utexas.edu/category/2014/02/galois_correspondences_and_enr.html}
(4th post in a series)



% ------------------------------------------------------------------------------
\subsection{Concept Lattice}\label{sec:concept_lattice}
% ------------------------------------------------------------------------------

% -----------------------------------------------------------------------------
\subsection{Formal Context}\label{sec:formal_context}
% -----------------------------------------------------------------------------

\emph{Formal Context}:
\[
  K = (G,M,I)
\]

$G$ -- \emph{Objects}

$M$ -- \emph{Attributes}

$I \subseteq G \times M$ -- \emph{Incidence}



% -----------------------------------------------------------------------------
\subsection{Formal Concept}\label{sec:formal_concept}
% -----------------------------------------------------------------------------

a \emph{Formal Concept} is a pair $(A,B)$ where:
\begin{itemize}
  \item $A \subseteq G$ -- a Set of Objects
  \item $B \subseteq M$ -- a Set of Attributes
  \item $A' = B$ -- where $A' = \{ m \in M \ |\ \forall g \in A, gIm \}$
  \item $B' = A$ -- where $B' = \{ g \in G \ |\ \forall m \in B, gIm \}$
\end{itemize}
that is:
\begin{itemize}
  \item every Object in $A$ has every Attribute in $B$
  \item for every Object in $G$ that is \emph{not} in $A$, there is
    \emph{some} Attribute in $B$ that the Object \emph{does not} have
  \item for every Attribute in $M$ that is \emph{not} in $B$, there is
    \emph{some} Object in $A$ that \emph{does not} have that Attribute
\end{itemize}



% ------------------------------------------------------------------------------
\subsection{Temporal Concept Analysis (TCA)}\label{sec:tca}
% ------------------------------------------------------------------------------
