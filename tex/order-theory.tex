%%%%%%%%%%%%%%%%%%%%%%%%%%%%%%%%%%%%%%%%%%%%%%%%%%%%%%%%%%%%%%%%%%%%%%
%%%%%%%%%%%%%%%%%%%%%%%%%%%%%%%%%%%%%%%%%%%%%%%%%%%%%%%%%%%%%%%%%%%%%%
\part{Order Theory}\label{sec:order_theory}
%%%%%%%%%%%%%%%%%%%%%%%%%%%%%%%%%%%%%%%%%%%%%%%%%%%%%%%%%%%%%%%%%%%%%%
%%%%%%%%%%%%%%%%%%%%%%%%%%%%%%%%%%%%%%%%%%%%%%%%%%%%%%%%%%%%%%%%%%%%%%

Ordering Relation (\S\ref{sec:ordering_relation})

\emph{Top Element}

\emph{Lexicographic Order}



% ====================================================================
\section{Setoid}\label{sec:setoid}
% ====================================================================

A \emph{Setoid} is a Set with an Equivalence Relation.



% ====================================================================
\section{Proset}\label{sec:proset}
% ====================================================================

A \emph{Proset} is a Set, $P$, with a Preorder
(\S\ref{sec:preorder}), $\lesssim$, defined on that Set.



% ====================================================================
\section{Poset}\label{sec:poset}
% ====================================================================

$\prec$

A \emph{Poset} (or \emph{Partially Ordered Set}) is a Set, $P$, with a
Partial Order (\S\ref{sec:partial_order}), $\leq$, defined on that
Set (that is a Relation which is Reflexive, Antisymmetric, and
Transitive).

An Element $x \in P$ is a \emph{Greatest Element} when:
\[
  \forall y \in P, y \leq x
\]
and likewise the \emph{Least element} when:
\[
  \forall y \in P, x \leq y
\]

An Element $x \in P$ is a \emph{Maximal Element} when:
\[
  \forall y \in P, x \nleq y
\]
and a \emph{Minimal Element} when:
\[
  \forall y \in P, y \nleq x
\]

The definition of Posets gives rise to the concept of a Monotonic
Function (\S\ref{sec:monotonic}), that is a Function, $f$, between
Posets which is ordered in both the Domain and the Codomain:
\[
  x \leq y \Rightarrow f(x) \leq f(y)
\]



% --------------------------------------------------------------------
\subsection{Hasse Diagram}\label{sec:hasse_diagram}
% --------------------------------------------------------------------

% --------------------------------------------------------------------
\subsection{Ascending Chain Condition}\label{sec:ascending_chain}
% --------------------------------------------------------------------

A Poset satisfies the \emph{Ascending Chain Condition} if every
strictly ascending Sequence of Elements eventually terminates. The
\emph{Descending Chain Condition} is defined likewise for descending
Sequences of Elements.



% --------------------------------------------------------------------
\subsection{Cofinality}\label{sec:cofinality}
% --------------------------------------------------------------------

The \emph{Cofinality} of a Poset $A$ is the least of the Cardinalities
of the Cofinal Subsets (\S\ref{sec:cofinal}) of $A$.



% --------------------------------------------------------------------
\subsection{Galois Connection}\label{sec:galois_connection}
% --------------------------------------------------------------------

Adjoint Pairs (\S\ref{sec:adjoint}) in Poset Categories

Closure Operators are Monads (\S\ref{sec:monad})



% --------------------------------------------------------------------
\subsection{Totally Ordered Set}\label{sec:totally_ordered}
% --------------------------------------------------------------------

A \emph{Totally Ordered Set} is a Poset whose Ordering Relation is
Total (\S\ref{sec:total_order}).



\subsubsection{Chain}\label{sec:chain}

A Totally Ordered Subset of some Poset is called a \emph{Chain}.



\subsubsection{Well-order}\label{sec:well_order}

A \emph{Well-order} is a Total Order on a Set $S$ such that every
non-empty Subset of $S$ has a Least Element.

\begin{itemize}
  \item $\mathbb{N}$
  \item $\mathbb{Z}^{\geq n}$
\end{itemize}

Well-ordering Axioms can be used to prove the Principle of
Mathematical Induction (\S\ref{sec:mathematical_induction}) as a
Theorem.


\paragraph{Well-ordering Theorem}\label{sec:wellorder_theorem}
\hfill \\

Axiom of Choice (\S\ref{sec:choice_axiom})



\subsubsection{Upper \& Lower Sets}\label{sec:upper_lower}

An \emph{Upper Set} (or \emph{Upset}) is a Subset, $U$, of a Poset,
$P$, such that:
\[
  x \in U \wedge x \leq y \Rightarrow y \in U
\]
The Dual notion of a \emph{Lower Set} (or \emph{Downset} or
\emph{Decreasing Set}), $L$, is defined likewise:
\[
  y \in L \wedge x \leq y \Rightarrow x \in L
\]
The Upper or Lower Set generated from a Subset $Y \subseteq X$ is
denoted by $\uparrow Y$ or $\downarrow Y$, respectively.

An Upper Set has the Property of being \emph{Upward Closed} and a
Lower Set the Property of being \emph{Downward Closed}.

A Topos Sieve (\S\ref{sec:topos_sieve}) is a generalization of a Lower
Set.



% --------------------------------------------------------------------
\subsection{Infimum \& Supremum}\label{sec:glb_lub}
% --------------------------------------------------------------------

The \emph{Infimum} (or \emph{Greatest Lower Bound} (GLB) or
\emph{Meet}) of a Subset of a Poset, $S \subseteq P$, is the Greatest
Element of $P$ that is a Least Element of $S$, denoted $\wedge S$.

The \emph{Supremum} (or \emph{Least Upper Bound} (LUB) or \emph{Join})
of a Subset of a Poset, $S \subseteq P$ is the Least Element of $P$
that is a Greatest Element of $S$, denoted $\vee S$.

Such Elements are Unique but not necessarily existing.

Meet and Join can be defined as Commutative, Associative, and
Idempotent Partial Binary Operations on pairs of Elements of $P$.

The Join/Meet of a Totally Ordered Set is its Maximal/Minimal Element.

The \emph{Nullary Meet} is the Top Element.



\subsubsection{Least-upper-bound Property}\label{sec:leastupperbound_property}



% --------------------------------------------------------------------
\subsection{Poset Product}\label{sec:poset_product}
% --------------------------------------------------------------------

The Product of two Elements of a Poset is their Greatest Lower Bound.

The Coproduct of two Elements of a Poset is their Least Upper Bound.



% --------------------------------------------------------------------
\subsection{Tree}\label{sec:tree}
% --------------------------------------------------------------------

A \emph{Tree} is a Poset $(T,<)$ such that $\forall t \in T$, the Set
$\{s \in T : s < t \}$ is Well-ordered (\S\ref{sec:well_order}) by
$<$.



% --------------------------------------------------------------------
\subsection{$\omega$-complete Partial Order}\label{sec:omega_cpo}
% --------------------------------------------------------------------

Poset in which every $\omega$-chain (\S\ref{sec:chain}) $x_1 \leq x_2
\leq \ldots$ has a Supremum.

A \emph{Strict} $\omega$-complete Partial Order has an Initial Object
$\bot$ and Continuous Maps between Strict $\omega$-complete Partial
Orders preserve $\bot$.

The Category of $\omega$-complete Partial Orders is Cartesian Closed
(\S\ref{sec:cartesian_closed}) but the Category of Strict
$\omega$-complete Partial Orders is not.



\subsubsection{Directed Complete Partial Order}\label{sec:dcpo}



% --------------------------------------------------------------------
\subsection{Duality Principle}\label{sec:duality_principle}
% --------------------------------------------------------------------



% ====================================================================
\section{Prefix Order}\label{sec:prefix_order}
% ====================================================================

% ====================================================================
\section{Order Isomorphism}\label{sec:order_isomorphism}
% ====================================================================

% ====================================================================
\section{Filter}\label{sec:filter}
% ====================================================================

Poset (\S\ref{sec:poset})

A \emph{Filter}, $\mathcal{F}$, of a Set, $X$, is a Set of Subsets of
$X$:
\[
  \mathcal{F} \subseteq \mathcal{P}(X)
\]
where:
\begin{itemize}
\item if $A \in \mathcal{F}$ and $B \supseteq A$, then $B \in
  \mathcal{F}$
\item if $A \in \mathcal{F}$ and $B \in \mathcal{F}$ then $A \cap B
  \in \mathcal{F}$
\item $X \in \mathcal{F}$ and $\emptyset \notin \mathcal{F}$
\end{itemize}

Dual is an \emph{Ideal} (\S\ref{sec:order_ideal})



% --------------------------------------------------------------------
\subsection{Ultrafilter}\label{sec:ultrafilter}
% --------------------------------------------------------------------

An \emph{Ultrafilter} $\mathcal{U}$ is a Filter such that for all $A
\subseteq X$, either $A \in \mathcal{U}$ or $X - A \in \mathcal{U}$.



% ====================================================================
\section{Ideal}\label{sec:order_ideal}
% ====================================================================

Dual of a Filter (\S\ref{sec:filter})



% --------------------------------------------------------------------
\subsection{Prime Ideal Theorem}\label{sec:prime_ideal}
% --------------------------------------------------------------------

Boolean Prime Ideal Theorem



% ====================================================================
\section{Law of Trichotomy}\label{sec:trichotomy_law}
% ====================================================================



% ====================================================================
\section{Domain Theory}\label{sec:domain_theory}
% ====================================================================

% --------------------------------------------------------------------
\subsection{Scott Domain}\label{sec:scott_domain}
% --------------------------------------------------------------------

Denotational Semantics (\S\ref{sec:denotational_semantics})
