%%%%%%%%%%%%%%%%%%%%%%%%%%%%%%%%%%%%%%%%%%%%%%%%%%%%%%%%%%%%%%%%%%%%%%
%%%%%%%%%%%%%%%%%%%%%%%%%%%%%%%%%%%%%%%%%%%%%%%%%%%%%%%%%%%%%%%%%%%%%%
\part{Mathematical Analysis}\label{part:mathematical_analysis}
%%%%%%%%%%%%%%%%%%%%%%%%%%%%%%%%%%%%%%%%%%%%%%%%%%%%%%%%%%%%%%%%%%%%%%
%%%%%%%%%%%%%%%%%%%%%%%%%%%%%%%%%%%%%%%%%%%%%%%%%%%%%%%%%%%%%%%%%%%%%%

% ====================================================================
\section{Asymptotic Analysis}\label{sec:asymptotic_analysis}
% ====================================================================

%FIXME move sequence, limit, etc. here ?



% ====================================================================
\section{Sequence}\label{sec:sequence}
% ====================================================================

A \emph{Sequence} can be defined as a Countable Totally Ordered
Multiset (\S\ref{sec:multiset}), that is, a collection of Elements (or
\emph{Terms}) in a given order where duplicate Elements are allowed.
The number of Elements in a Sequence is referred to as its
\emph{Length}.

A \emph{Finite Sequence} is called a \emph{Tuple} (\S\ref{sec:tuple}).

String (\S\ref{sec:string})

Series (\S\ref{sec:series})

Sequence (Topology) (\S\ref{sec:sequence_topology})

$a_n : \nats \rightarrow \reals$



% --------------------------------------------------------------------
\subsection{Subsequence}\label{sec:subsequence}
% --------------------------------------------------------------------

A Sequence $a_n$ Converges (\S\ref{sec:convergent_sequence}) to $l$
if and only if all Subsequences of $a_n$ Converge to $l$.



% --------------------------------------------------------------------
\subsection{Tuple}\label{sec:tuple}
% --------------------------------------------------------------------

% --------------------------------------------------------------------
\subsection{Limit}\label{sec:sequence_limit}
% --------------------------------------------------------------------

Can be defined in any Metric (\S\ref{sec:metric}) or Topological Space
(\S\ref{sec:topological_space}), generalized to a Topological Net
(\S\ref{sec:net}); see also Limits (\S\ref{sec:limit}) and Colimits
(\S\ref{sec:colimit}) in Category Theory.

The Limit of a Sequence is Unique

Bounded Sequence (\S\ref{sec:bounded_sequence})

$\lim a_n = 0 \Leftrightarrow \lim |a_n| = 0$

$\lim (a_n \pm b_n) = \lim a_n \pm \lim b_n$

$\lim (a_n b_n) = (\lim a_n) (\lim b_n)$

$\lim (\frac{a_n}{b_n}) = \frac{\lim a_n}{\lim b_n}$ for $\forall
n \in \nats, b_n \neq 0$ and $\lim b_n \neq 0$

$a_n \leq b_n$ for all $n$ Implies $\lim a_n \leq \lim b_n$, but it is
not the case that $a_n < b_n$ for all $n$ Implies $\lim a_n < \lim
b_n$ since $a_n$ and $b_n$ may still be equal in the Limit.

For an Open Interval $(x,y)$, $\lim a_n \in (x,y)$ Implies $a_n \in
(x,y)$ for all $n$, but for not for a Closed Interval.



\subsubsection{Limit Inferior}\label{sec:liminf}

$\liminf$

$\underline{\lim}$



\subsubsection{Limit Superior}\label{sec:limsup}

$\limsup$

$\overline{\lim}$



\subsubsection{Convergent Sequence}\label{sec:convergent_sequence}

$\forall \varepsilon > 0, \exists N : n \geq N \Rightarrow |(a_n - l)| <
\varepsilon$

A Sequence $a_n$ Converges to $l$ if and only if all Subsequences
(\S\ref{sec:subsequence}) of $a_n$ Converge to $l$.

If $\lim a_n = l$, then for $k \in \ints$, $\lim (a_{n+k}) = l$

If $a_n$ is Convergent then it is Bounded
(\S\ref{sec:bounded_sequence}).

If $F$ is a Closed Set (\S\ref{sec:closed_set}), then for any Sequence
$x_n$ in $F$ can Converge to $x$ if and only if $x$ is in $F$.



\paragraph{Squeeze Theorem}\label{sec:squeeze_theorem}\hfill

For Sequences $a_n \leq b_n \leq c_n$ and $\lim a_n = l$ and $\lim c_n
= l$, then $\lim b_n = l$



\subsubsection{Divergent Sequence}\label{sec:divergent_sequence}



% --------------------------------------------------------------------
\subsection{Bounded Sequence}\label{sec:bounded_sequence}
% --------------------------------------------------------------------

If $a_n$ is Convergent (\S\ref{sec:convergent_sequence}) then it is
Bounded.



\subsubsection{Bolzano-Weierstrass Theorem}
\label{sec:bolzano_weierstrass}

\emph{Sequential Compactness Theorem}

Lemma: Every Sequence $a_n$ of Real Numbers has a Monotone
Subsequence

The \emph{Bolzano-Weierstrass Theorem} states that a Bounded Sequence
$a_n$ has at least one Subsequence that Converges.



% --------------------------------------------------------------------
\subsection{Arithmetic Sequence}\label{sec:arithmetic_sequence}
% --------------------------------------------------------------------

(or \emph{Arithmetic Progression})

Arithmetic Series (\S\ref{sec:arithmetic_series})



% --------------------------------------------------------------------
\subsection{Geometric Sequence}\label{sec:geometric_sequence}
% --------------------------------------------------------------------

(or \emph{Geometric Progression})

Sum of Terms of a Geometric Sequence form a Geometric Series
(\S\ref{sec:geometric_series})



% --------------------------------------------------------------------
\subsection{Infinite Sequence}\label{sec:infinite_sequence}
% --------------------------------------------------------------------

\emph{Infinite Sequences} may be \emph{Singly Infinite}
(\S\ref{sec:singly_infinite}), having an initial Element but no final
Element, or \emph{Doubly Infinite} (\S\ref{sec:doubly_infinite})
having neither a first nor a last Element.



\subsubsection{Singly Infinite Sequence}\label{sec:singly_infinite}

A \emph{Singly Infinite Sequence} (or \emph{One-sided Infinite
  Sequence}) can be defined as a Function, $s$, with a Countably
Infinite Totally Ordered Set of Indices, $X$, for its Domain and a Set
of Elements, $Y$, for the Codomain:

  $s : X \rightarrow Y$ \\
where:

  $X = \{1,2,\ldots,n\}$

  $Y = \{a_1, a_2,\ldots,a_n\}$

  $s = \{(1,a_1), (2,a_2),\ldots, (n,a_n)\}$ \\
for some Countable $n \geq 0$.

Singly Infinite Sequences may be interpreted as Elements of the
Semigroup Ring of the Natural Numbers, $R[\mathbb{N}]$
(\S\ref{sec:group_ring}).



\subsubsection{Doubly Infinite Sequence}\label{sec:doubly_infinite}

A \emph{Doubly Infinite Sequence} (also \emph{Two-way Infinite} or
\emph{Bi-infinite Sequence}) may be defined as a Function from the Set
of all Integers $\mathbb{Z}$ into a Set, denoted
$(2n)^{\infty}_{n=-\infty}$.

Doubly Infinite Sequences may be interpreted as Elements of the Group
Ring of the Integers, $R[\mathbb{Z}]$ (\S\ref{sec:group_ring}.



% --------------------------------------------------------------------
\subsection{Monotone Sequence}\label{sec:monotone_sequence}
% --------------------------------------------------------------------

Monotone Function (\S\ref{sec:monotonic_function})

Increasing Sequence: $\forall n \in \nats, a_n \leq a_{n+1}$

An Increasing Sequence is Bounded above if and only if it is
Convergent.

Decreasing Sequence: $\forall n \in \nats, a_n \geq a_{n+1}$

A Decreasing Sequence is Bounded below if and only if it is
Convergent.



% --------------------------------------------------------------------
\subsection{Cauchy Sequence}\label{sec:cauchy_sequence}
% --------------------------------------------------------------------

Complete Metric Space (\S\ref{sec:complete_metric_space})



% --------------------------------------------------------------------
\subsection{Oscillation}\label{sec:oscillation}
% --------------------------------------------------------------------

% --------------------------------------------------------------------
\subsection{Sequence Space}\label{sec:sequence_space}
% --------------------------------------------------------------------

%FIXME xref function space, move to functional analysis?



% ====================================================================
\section{Series}\label{sec:series}
% ====================================================================

Sum of Terms of a Sequence (\S\ref{sec:sequence}) $a_n : \nats
\rightarrow \reals$



% --------------------------------------------------------------------
\subsection{Infinite Series}\label{sec:infinite_series}
% --------------------------------------------------------------------

Transcendental Numbers (\S\ref{sec:transcendental})



\subsubsection{Partial Sum}\label{sec:partial_sum}

Sequence $\{ a_1, a_2, a_3, \ldots \}$

$S_n = \sum_{k=1}^n a_k$



\subsubsection{Convergent Series}\label{sec:convergent_series}

Limit (\S\ref{sec:sequence_limit}) of Partial Sums $\{ S_1, S_2, S_3,
\ldots \}$ Converges (\S\ref{sec:convergent_sequence})



\subsubsection{Divergent Series}\label{sec:divergent_series}

\paragraph{Abelian Mean}\label{sec:abelian_mean}\hfill

\paragraph{Abel Summation}\label{sec:abel_summation}\hfill

Analytic Number Theory (\S\ref{sec:analytic_number_theory})

$a_n$ Sequence of Complex Numbers

$f(t)$ Differentiable Function (\S\ref{sec:differentiable_function})

$A(x) = \sum_{n \leq x} a_n$

$\sum_{n \leq x} a_n f(n) = A(x)f(x) - \int_1^x A(t)f'(t) dt$



% --------------------------------------------------------------------
\subsection{Arithmetic Series}\label{sec:arithmetic_series}
% --------------------------------------------------------------------

Arithmetic Progression (Arithmetic Sequence
\S\ref{sec:arithmetic_sequence})



% --------------------------------------------------------------------
\subsection{Geometric Series}\label{sec:geometric_series}
% --------------------------------------------------------------------

Constant Ratio between successive Terms

Terms form a Geometric Progression (Geometric Sequence
\S\ref{sec:geometric_sequence})

Sum Converges as long as Absolute Value of the Ratio of Terms is less
than $1$



\subsubsection{Hypergeometric Series}\label{sec:hypergeometric_series}

Hypergeometric Function (\S\ref{sec:hypergeometric_function})



% --------------------------------------------------------------------
\subsection{Harmonic Series}\label{sec:harmonic_series}
% --------------------------------------------------------------------

% --------------------------------------------------------------------
\subsection{Alternating Series}\label{sec:alternating_series}
% --------------------------------------------------------------------

\[
  \sum_{n=0}^\infty (-1)^n a_n
\]
or:
\[
  \sum_{n=0}^\infty (-1)^{n-1} a_n
\]



% --------------------------------------------------------------------
\subsection{Power Series}\label{sec:power_series}
% --------------------------------------------------------------------

Infinite Series of the form:
\[
  f(x) = \sum_{n=0}^\infty a_n (x - c)^n
\]



\subsubsection{Taylor Series}\label{sec:taylor_series}

Centered at $0$: \emph{Maclaurin Series}

Polynomial formed by some initial Terms of a Taylor Series: Taylor
Polynomial (\S\ref{sec:taylor_polynomial}); Taylor Series is the Limit
(\S\ref{sec:sequence_limit}) of the Taylor Polynomials with increasing
Degree.

Note that a Function may not be equal to its Taylor Series even if its
Taylor Series Converges at every Point.

A Function that is equal to its Taylor Series in an Open Interval
(\S\ref{sec:interval}, or Disc \S\ref{sec:disc}) is an Analytic
Function (\S\ref{sec:analytic_function}) in that Interval.



\paragraph{Binomial Series}\label{sec:binomial_series}\hfill



% --------------------------------------------------------------------
\subsection{General Dirichlet Series}\label{sec:general_dirichlet}
% --------------------------------------------------------------------

%FIXME: move to subsection?



\subsubsection{Dirichlet Series}\label{sec:dirichlet_series}



% --------------------------------------------------------------------
\subsection{Formal Power Series}\label{sec:formal_power_series}
% --------------------------------------------------------------------

% --------------------------------------------------------------------
\subsection{Telescoping Series}\label{sec:telescoping_series}
% --------------------------------------------------------------------



% ====================================================================
\section{Infinite Product}\label{sec:infinite_product}
% ====================================================================

Converges when the Sequence converges to $1$

$\prod_{n=1}^\infty a_n$ Converges if and only if $\sum_{n=1}^\infty
ln(a_n)$ Converges

$q_n = (1 + u_1)(1 + u_2)\cdots(1 + u_n)$ Converges if and only if
$\sum u_n$ Converges.



% ====================================================================
\section{Real Analysis}\label{sec:real_analysis}
% ====================================================================

$R^1$ -- Real Line (\S\ref{sec:real_line}): 1-dimensional Real
Coordinate Space



% --------------------------------------------------------------------
\subsection{Real Interval}\label{sec:real_interval}
% --------------------------------------------------------------------

Interval (\S\ref{sec:interval})

Interval Arithmetic (\S\ref{sec:interval_arithmetic})

$[a,b] = \bigcap_n (a - \frac{1}{n}, b + \frac{1}{n})$



\subsubsection{Interval Partition}\label{sec:interval_partition}

Closed Interval $[a,b]$

Finite Sequence (\S\ref{sec:sequence}) $(x_i) = \{ x_0, x_1, \ldots,
x_n \}$

$a = x_0 < x_1 < x_2 < \ldots < x_n = b$



% --------------------------------------------------------------------
\subsection{Maximum}\label{sec:maximum}
% --------------------------------------------------------------------

Upper Bound (\S\ref{sec:upper_bound})

Least Upper Bound (\S\ref{sec:least_upperbound})



% --------------------------------------------------------------------
\subsection{Minimum}\label{sec:minimum}
% --------------------------------------------------------------------

Lower Bound (\S\ref{sec:lower_bound})

Greatest Lower Bound (\S\ref{sec:greatest_lowerbound})



% --------------------------------------------------------------------
\subsection{Newton's Expansion}\label{sec:newtons_expansion}
% --------------------------------------------------------------------

% FIXME

$(1 + a)^n$ % ???



% --------------------------------------------------------------------
\subsection{Bernoulli's Inequality}\label{sec:bernoullis_inequality}
% --------------------------------------------------------------------

$n \in \nats$, $a \in \reals^+$, then:
\[
  (1 + a)^n \geq 1 + n a
\]


% --------------------------------------------------------------------
\subsection{Real-valued Function}\label{sec:real_function}
% --------------------------------------------------------------------

\subsubsection{Bounded Function}\label{sec:bounded_function}

\subsubsection{Function Limit}\label{sec:function_limit}

Limit Point (\S\ref{sec:limit_point}) of $D \in \reals$ is $a \in D$
such that:
\[
  \exists a_n \in D : a_n \neq a \wedge \lim a_n = a
\]

Function $f$ has a \emph{Limit} $l$ at $a$ if for all Sequences $a_n
\in D$ with $\lim a_n = a$ and $a_n \neq a$ for all $n$, $\lim f(a_n)
= l$



\subsubsection{Root}\label{sec:root}

Any Polynomial (\S\ref{sec:polynomial}) with Odd Degree has at least
one Real Root



\subsubsection{Newton's Method}\label{sec:newtons_method}



% --------------------------------------------------------------------
\subsection{Real-valued Continuous Function}\label{sec:real_continuous}
% --------------------------------------------------------------------

$f : D \subseteq \reals \rightarrow \reals$

Continuous at $l \in D$ if for every Sequence $a_n \in D$ such that
$\lim a_n = l$, then $\lim f(a_n) = f(l)$

$\lim f (a_n) = f (\lim a_n)$

Equivalently: Continuous at $l$ if and only if:
\[
  \forall \varepsilon > 0, \exists \delta :
  |x - l| < \delta \Rightarrow |f(x) - f(l)| < \varepsilon
\]

Equivalently: Continuous at $a$ if and only if $f$ has Limit
(\S\ref{sec:function_limit}) $f(a)$ at $a$: $\lim_{x \rightarrow
  a}f(x) = f(a)$

Continuous on an Interval if and only if the Range on that Interval is
also a single Interval

For $f,g$ Continuous on $D$ at $a \in D$, then $(f + g)$, $f g$,
$\frac{f}{g}$ (when $g(a) \neq 0$) are Continuous at $a$.

For $g$ Defined on the Range of $f$, $\{ f(x); x \in D\}$, if $f$ is
Continuous at $a \in D$ and $g$ Continuous at $f(a)$, then $g \circ f$
is Continuous at $a$: $\lim g(f(a_n)) = g(f(a))$

Differentiable (\S\ref{sec:differentiable_function}) at $a$ Implies
Continuous at $a$



\subsubsection{Local Extrema}\label{sec:local_extrema}

Local Maximum

Local Minimum

for Differentiable (\S\ref{sec:differentiable_function}) $f$ on an
Open Interval: $f'(a) = 0$ at Local Extrema



\paragraph{Second Derivative Test}\label{sec:second_derivative_test}\hfill



\subsubsection{Intermediate Value Theorem}
\label{sec:intermediate_value}

For Function $f(x)$ Continuous on Closed Interval $[a,b]$, for any
$f(a) < c < f(b)$, there is a $d \in (a,b)$ such that $f(d) = c$.



\subsubsection{Extreme Value Theorem}\label{sec:extreme_value}

\subsubsection{Modulus of Continuity}\label{sec:continuity_modulus}



% --------------------------------------------------------------------
\subsection{Differentiable Function}\label{sec:differentiable_function}
% --------------------------------------------------------------------

Differential Equation (\S\ref{sec:differential_equation})

For $f$ defined on Open Interval $(a,b) \subset \reals$, $f$ is
Differentiable at $x \in (a,b)$ if Limit $f'(x) = \lim_{h \rightarrow
  0} \frac{f (x+h) - f(x)}{h}$ exists.

Differentiable (\S\ref{sec:differentiable_function}) at $a$ Implies
Continuous at $a$

A Polynomial (\S\ref{sec:polynomial}), being the Sum of Differentiable
Functions, is Differentiable everywhere

For $f$ Differentiable on an Interval $I$, if $\forall x \in I, f'(x)
> 0$, $f$ is Strictly Increasing and if $\forall x \in I, f'(x) < 0$,
$f$ is Strictly Decreasing (see Monotonic Functions
\S\ref{sec:monotonic_function}).



\subsubsection{Derivative}\label{sec:derivative}

$(\frac{g}{f})' = \frac{f g' - f' g}{f^2}$



\subsubsection{Antiderivative}\label{sec:antiderivative}

\emph{Primitive Integral}

If $f$ is Continuous on $[a,b]$ then:
\[
  F(x) = \int_a^x f
\]
for all $x \in [a,b]$ and $F$ is Differentiable on $(a,b)$ and $F' =
f$.



\subsubsection{Differentiability Class}
\label{sec:differentiability_class}

$\mathcal{C}^1$ Continuous First Derivative

$\mathcal{C}^2$ Continuous Second Derivative



\subsubsection{Chain Rule}\label{sec:chain_rule}

\subsubsection{Cauchy Mean Value Theorem}
\label{sec:cauchy_mean_value}

For Functions $f$, $g$ Continuous on $[a,b]$, Differentiable on
$(a,b)$ then $\exists c \in (a,b)$ such that:
\[
  f'(c) (g(b) - g(a)) = g'(c) (f(b) - f(a))
\]



\subsubsection{Lagrange Mean Value Theorem}
\label{sec:lagrange_mean_value}

$f$ Continuous on $[a,b]$ and Differentiable on $(a,b)$ then $\exists
c \in (a,b)$ such that $f(b) - f(a) = f'(c)(b-a)$ or:
\[
  f'(c) = \frac{f(b) - f(a)}{b - a}
\]



\subsubsection{Rolle's Theorem}\label{sec:rolles_theorem}

for $f$ Continuous on $[a,b]$ and Differentiable on $(a,b)$ with $f(a)
= f(b)$, then $\exists c \in (a,b)$ such that $f'(c) = 0$



% --------------------------------------------------------------------
\subsection{Integrable Function}\label{sec:integrable_function}
% --------------------------------------------------------------------

$f$ Bounded on Closed Bounded $[a,b]$, $\forall \varepsilon >0$, there
exists a Partition (\S\ref{sec:interval_partition}) $P$ of $[a,b]$
such that $0 \leq U(f,P) - L(f,P) < \varepsilon$ %FIXME xref upper lower

Monotone Functions (\S\ref{sec:monotonic_function}) and Continuous
Functions (\S\ref{sec:continuous_function}) are always (Riemann)
Integrable

$L(f,P) \leq \int_a^b f \leq U(f,P)$

For $c$ a Constant, if $f$ is Integrable then $cf$ is Integrable: $c
\int_a^b f = \int_a^b c f$

For $f$, $g$ Integrable on $[a,b]$, then $f + g$ is Integrable:
$\int_a^b (f+g) = \int_a^b f + \int_a^b g$

For $f$ Integrable on $[a,b]$ and $a < c < b$ then $f$ is Integrable
on $[a,c]$ and $[c,b$ and $\int_a^b f = \int_a^c f + \int_c^b f$

For $f$, $g$ Integrable on $[a,b]$ and $f \leq g$ on $[a,b]$, then
$\int_a^b f \leq \int_a^b g$

For Continuous $g$ and Integrable $f$, then $g \circ f$ is Integrable

For Integrable $f$, $|f|$ is Integrable and $|\int_a^b f| \leq
\int_a^b |f|$



\subsubsection{Integral}\label{sec:integral}

Numerical Integration (\S\ref{sec:numerical_integration})

Integrable Function (\S\ref{sec:integrable_function})



\paragraph{Definite Integral}\label{sec:definite_integral}\hfill

\paragraph{Indefinite Integral}\label{sec:indefinite_integral}\hfill

\paragraph{Darboux Integral}\label{sec:darboux_integral}\hfill

Upper Darboux Sum

Lower Darboux Sum



\paragraph{Riemann Integral}\label{sec:riemann_integral}\hfill

$\int_a^b f(x) dx$

Interval Partition (\S\ref{sec:interval_partition})

Monotone Bounded Functions are Riemann Integrable

Continuous Functions are Reiemann Integrable

Riemann Integrable on $[a,b]$, Least Upper Bound of Lower Darboux Sums
is equal to the Greatest Lower Bound of the Upper Darboux Sums, value
denoted by $\int_a^b f$



\paragraph{Lebesgue Integral}\label{sec:lebesgue_integral}\hfill

\paragraph{Wallis Integral}\label{sec:wallis_integral}\hfill



\subsubsection{Integral Mean Value Theorem}
\label{sec:integral_mean_value}

For $f$ Continuous on $[a,b]$, there is a $c \in [a,b]$ such that
$\int_a^b f = f(c)(b - a)$



% --------------------------------------------------------------------
\subsection{Fundamental Theorem of Calculus}
\label{sec:fundamental_theorem}
% --------------------------------------------------------------------

$F' = f$

$\int^b_a f = F(b) - F(a)$



% ====================================================================
\section{Complex Analysis}\label{sec:complex_analysis}
% ====================================================================

Complex Numbers (\S\ref{sec:complex_number})

Complex Differentiable Functions are automatically Analytic %FIXME



% --------------------------------------------------------------------
\subsection{Euler's Formula}\label{sec:eulers_formula}
% --------------------------------------------------------------------

% --------------------------------------------------------------------
\subsection{Complex Surface}\label{sec:complex_surface}
% --------------------------------------------------------------------

\subsubsection{Enriques-Kodaira Classification}
\label{sec:enriques_kodaira}

\subsubsection{Complex Plane}\label{sec:complex_plane}

or \emph{Argand Plane}



\subsubsection{Disc}\label{sec:disc}\hfill

\subsubsection{Extended Complex Plane}\label{sec:extended_complex_plane}

Complex Plane with Point at Infinity

Stereographic Projection (\S\ref{sec:stereographic_projection})

Riemann Sphere (\S\ref{sec:riemann_sphere})



% --------------------------------------------------------------------
\subsection{Holomorphic Function}\label{sec:holomorphic_function}
% --------------------------------------------------------------------

Complex-valued Function of one or more Complex Variables that is
Complex-differentiable in a Neighborhood (\S\ref{sec:neighborhood}) of
every Point in its Domain. Implies that any Holomorphic Function is
Infinitely Differentiable (\S\ref{sec:smooth_function}) and equal to
its own Taylor Series (\S\ref{sec:taylor_series}).

Real or Imaginary part of any Homomorphic Function is a Harmonic
Function (\S\ref{sec:harmonic_function})



% --------------------------------------------------------------------
\subsection{Meromorphic Function}\label{sec:meromorphic_function}
% --------------------------------------------------------------------

\subsubsection{$L$-function}\label{sec:l_function}

\subsection{Gamma Function}\label{sec:gamma_function}

$\Gamma(\alpha) = \int_0^{\infty} x^{\alpha -1} e^{-x} dx$

for $\alpha > 0$

\begin{enumerate}
\item $\Gamma(\alpha) = (\alpha - 1) \Gamma(\alpha -1)$
\item $\Gamma(n) = (n-1)!$
\item $\Gamma(1) = 1$
\item $\Gamma(\sfrac{1}{2}) = \sqrt{pi}$
\end{enumerate}

Gamma Distribution (\S\ref{sec:gamma_distribution})



% --------------------------------------------------------------------
\subsection{Univalent Function}\label{sec:univalent_function}
% --------------------------------------------------------------------

Injective Holomorphic Function on an Open Subset of the Complex Plane



% --------------------------------------------------------------------
\subsection{Modular Form}\label{sec:modular_form}
% --------------------------------------------------------------------

Automorphic Form (\S\ref{sec:automorphic_form})



% --------------------------------------------------------------------
\subsection{$n$-th Root}\label{sec:nth_root}
% --------------------------------------------------------------------

% --------------------------------------------------------------------
\subsection{Riemann Surface}\label{sec:riemann_surface}
% --------------------------------------------------------------------

One-dimensional Complex Manifold (\S\ref{sec:complex_manifold})

the Geometry of Riemann Surfaces is given by Two-dimensional Conformal
Geometry (\S\ref{sec:conformal_geometry})



\subsubsection{Riemann Mapping Theorem}
\label{sec:riemann_mapping_theorem}

\subsubsection{Riemann Sphere}\label{sec:riemann_sphere}

``simplest'' Riemann Surface

Model of the Extended Complex Plane
(\S\ref{sec:extended_complex_plane})

can be thought of as the \emph{Complex Projective Line}
$\mathbb{CP}^1$: the Projective Space (\S\ref{sec:projective_space} of
all Complex Lines in $\comps^2$



\subsubsection{Bolza Surface}\label{sec:bolza_surface}



% --------------------------------------------------------------------
\subsection{Countour Integration}\label{sec:contour_integration}
% --------------------------------------------------------------------

Residue Calculus (???) %FIXME



% ====================================================================
\section{Differential Equation}\label{sec:differential_equation}
% ====================================================================

% --------------------------------------------------------------------
\subsection{Ordinary Differential Equation}
\label{sec:ode}
% --------------------------------------------------------------------

\subsubsection{Continuously Differentiable}
\label{sec:continuously_differentiable}

$C^1$



\subsubsection{Hypergeometric Function}
\label{sec:hypergeometric_function}

Hypergeometric Series (\S\ref{sec:hypergeometric_series})



% --------------------------------------------------------------------
\subsection{Smooth Function}\label{sec:smooth_function}
% --------------------------------------------------------------------

$C^{\infty}$



\subsubsection{Analytic Function}\label{sec:analytic_function}

\paragraph{Analytic Continuation}\label{sec:analytic_continuation}



% --------------------------------------------------------------------
\subsection{Partial Differential Equation}
\label{sec:partial_differential}
% --------------------------------------------------------------------

Finite Element Method (\S\ref{sec:finite_element_method}): solving
Boundary Value Problems (\S\ref{sec:boundary_value_problem}) for
Partial Differential Equations



\subsubsection{Elliptic Partial Differential Equation}
\label{sec:elliptic_partial_differential}

\paragraph{Poisson Equation}\label{sec:poisson_equation}\hfill

$\nabla^2 u = u_{xx} + u{yy} = f(x,y)$



\paragraph{Laplace's Equation}\label{sec:laplaces_equation}\hfill

Second-order Elliptic Partial Differential Equation

written:

\[ \nabla^2 \varphi = 0 \]

or:

\[ \Delta \varphi = 0 \]

where $\Delta = \nabla^2$ is the Laplace Operator
(\S\ref{sec:laplace_operator}) and $\varphi$ is a Scalar Function
(\S\ref{sec:scalar_function})

Harmonic Functions (\S\ref{sec:harmonic_function})

$\nabla^2 u = u_{xx} + u_{yy} = 0$ %FIXME



\subsubsection{Dirichlet Problem}\label{sec:dirichlet_problem}



% --------------------------------------------------------------------
\subsection{Boundary Value Problem}\label{sec:boundary_value_problem}
% --------------------------------------------------------------------

%FIXME harmonic analysis? dirichlet problem?

\subsubsection{Finite Element Method}\label{sec:finite_element_method}

\emph{Finite Element Method} or \emph{FEM} is a technique for solving
Boundary Value Problems for Partial Differential Equations
(\S\ref{sec:partial_differential})



% ====================================================================
\section{Harmonic Analysis}\label{sec:harmonic_analysis}
% ====================================================================

% --------------------------------------------------------------------
\subsection{Harmonic Function}\label{sec:harmonic_function}
% --------------------------------------------------------------------

Twice-continuously Differentiable Function
(\S\ref{sec:continuously_differentiable}) $f : U \rightarrow Reals$,
where $U$ is an Open Subset of $\reals^n$, satisfying \emph{Laplace's
  Equation} (\S\ref{sec:laplaces_equation})

2016 - Samantha Davies - \emph{Voltage, Temperature, and Harmonic
  Functions} -
\url{https://jeremykun.com/2016/09/26/voltage-temperature-and-harmonic-functions/}

Physics: Simple Harmonic Motion is a type of ``Oscillation'' where the
Force that restores an object to its Equilibrium is directly
proportional to the Displacement

Stochastic Processes (\S\ref{sec:stochastic_process})

Potential Theory (Mathematical Physics) %FIXME create section?

the Real or Imaginary part of any Homomorphic Function
(\S\ref{sec:holomorphic_function}) is a Harmonic Function

Harmonic Functions on $\reals$ are exactly the Linear Functions
(\S\ref{sec:polynomial_function})


\textbf{Mean Value Property}: the Value of a Harmonic Function at an
Interior Point is the Average of the Function's Values around the
Point

If $u$ is a Continuous Function satisfying the Mean Value Property on
a Region $\Omega$ then $u$ is Harmonic in $\Omega$

Any Function $\alpha$ on a Graph which satisfies the Mean Value
Property also satisfies the Discrete Laplacian
(\S\ref{sec:discrete_laplace})


\textbf{Maximum Principle}: a Non-constant Harmonic Function on a
Closed Bounded Region must attain Maximum and Minimum on its Boundary

Convex Optimization (\S\ref{sec:convex_optimization})


\textbf{Uniqueness}: if a Harmonic Function is Continuous on the
Boundary of a Closed Bounded Set and Harmonic in the Interior then the
Interior Values are Uniquely Determined by the Values of the Boundary


\textbf{Solution to a Dirichlet Problem}
(\S\ref{sec:dirichlet_problem})



% --------------------------------------------------------------------
\subsection{Automorphic Form}\label{sec:automorphic_form}
% --------------------------------------------------------------------

% --------------------------------------------------------------------
\subsection{Fourier Analysis}\label{sec:fourier_analysis}
% --------------------------------------------------------------------

\subsubsection{Fourier Series}\label{sec:fourier_series}



% --------------------------------------------------------------------
\subsection{Spherical Harmonics}\label{sec:spherical_harmonics}
% --------------------------------------------------------------------



% ====================================================================
\section{Functional Analysis}\label{sec:functional_analysis}
% ====================================================================

Inner Product Spaces (Normed Vector Spaces
\S\ref{sec:innerproduct_space})

Hilbert Space (\S\ref{sec:hilbert_space})

Topological Vector Space (\S\ref{sec:topological_vector})



% --------------------------------------------------------------------
\subsection{Differential Operator}\label{sec:differential_operator}
% --------------------------------------------------------------------

%FIXME move?

$\nabla$



\subsubsection{Theta Operator}\label{sec:theta_operator}

\subsubsection{Elliptic Operator}\label{sec:elliptic_operator}

\paragraph{Laplace Operator}\label{sec:laplace_operator}\hfill

$\Delta = \nabla^2$

Discrete Laplace Operator (Graph Theory \S\ref{sec:discrete_laplace})



\subsubsection{Schwarzian Derivative}\label{sec:schwarzian_derivative}

Non-linear Differential Operator



% --------------------------------------------------------------------
\subsection{Bounded Linear Operator}\label{sec:bounded_linear_operator}
% --------------------------------------------------------------------

an Operator between two Normed Vector Spaces
(\S\ref{sec:normed_vectorspace}) is a Bounded Linear Operator if and
only if it is a Continuous Linear Operator
(\S\ref{sec:continuous_linear})



\subsubsection{Hermitian Adjoint}\label{sec:hermitian_adjoint}

Complex Hilbert Space (\S\ref{sec:hilbert_space})



% --------------------------------------------------------------------
\subsection{Closed Linear Span}\label{sec:closed_linear_span}
% --------------------------------------------------------------------

Linear Span (\S\ref{sec:linear_span})



% --------------------------------------------------------------------
\subsection{Banach Space}\label{sec:banach_space}
% --------------------------------------------------------------------

Topological Vector Space (\S\ref{sec:topological_vector})



\subsubsection{$L^p$ Space}\label{sec:lp_space}



% --------------------------------------------------------------------
\subsection{Banach Algebra}\label{sec:banach_algebra}
% --------------------------------------------------------------------

\subsubsection{C$^*$-algebra}\label{sec:cstar_algebra}

A \emph{C$^*$-algebra} a Complex Algebra $A$ of Continuous Linear
Operators (\S\ref{sec:continuous_linear}) on a Complex Hilbert Space
(\S\ref{sec:hilbert_space}) with the additional Properties that $A$ is
Topologically Closed in the Norm Topology of Operators and Closed
under the Operation of taking Adjoints of Operators

Involutive Algebra (\S\ref{sec:involutive_algebra})

Involution Semigroup (\S\ref{sec:involution_semigroup})

(wiki):

Linear Logic (\S\ref{sec:linear_logic}) can be seen as the refining
the Interpretation of Classical Logic by replacing Boolean Algebras
(\S\ref{sec:boolean_algebra}) by C$^*$-algebras



\paragraph{von Neumann Algebra}\label{sec:vonneumann_algebra}\hfill



% --------------------------------------------------------------------
\subsection{Coherence Space}\label{sec:coherence_space}
% --------------------------------------------------------------------

note: not the same as Coherent Space (Spectral Space)

cf. Linear Logic (\S\ref{sec:linear_logic}) Semantics



% --------------------------------------------------------------------
\subsection{Spectral Theory}\label{sec:spectral_theory}
% --------------------------------------------------------------------

Hilbert Space (\S\ref{sec:hilbert_space})



\subsubsection{Spectrum}\label{sec:spectrum}



% ====================================================================
\section{Convex Analysis}\label{sec:convex_analysis}
% ====================================================================

Convex Optimization (\S\ref{sec:convex_optimization})



% ====================================================================
\section{Algebraic Analysis}\label{sec:algebraic_analysis}
% ====================================================================

% --------------------------------------------------------------------
\subsection{Generalized Function}\label{sec:generalized_function}
% --------------------------------------------------------------------

\subsubsection{Distribution}\label{sec:distribution}

Continuous Linear Functional (\S\ref{sec:linear_form})



% ====================================================================
\section{Numerical Analysis}\label{sec:numerical_analysis}
% ====================================================================

% --------------------------------------------------------------------
\subsection{Interval Arithmetic}\label{sec:interval_arithmetic}
% --------------------------------------------------------------------

\subsubsection{Unit Interval}\label{sec:unit_interval}

$[0,1]$, sometimes $I$

a Complete Metric Space (\S\ref{sec:complete_metric_space}),
Homeomorphic (\S\ref{sec:homeomorphism}) to the Extended Real Number
Line (\S\ref{sec:extended_real_line})



% --------------------------------------------------------------------
\subsection{Numerical Integration}\label{sec:numerical_integration}
% --------------------------------------------------------------------

% --------------------------------------------------------------------
\subsection{Spline Function}\label{sec:spline}
% --------------------------------------------------------------------

\subsubsection{Spline Interpolation}\label{sec:spline_interpolation}

cf. Polynomial Interpolation (???) %FIXME



\subsubsection{Cubic Spline}\label{sec:cubic_spline}

\paragraph{B-spline}\label{sec:b_spline}\hfill



% ====================================================================
\section{Ordinal Analysis}\label{sec:ordinal_analysis}
% ====================================================================

Ordinal Numbers (\S\ref{sec:ordinal_number})

Proof-theoretic Ordinal (\S\ref{sec:proof_ordinal})



% ====================================================================
\section{Non-standard Analysis}\label{sec:nonstandard_analysis}
% ====================================================================

Hyperreals (\S\ref{sec:hyperreal})

Real Closed Field (\S\ref{sec:real_closed})

$\reals^\nats / \mathsf{M}$ where $\mathsf{M}$ is a Maximal Ideal
(\S\ref{sec:maximal_ideal}) not leading to a Field that is
Order-isomorphic (\S\ref{sec:order_isomorphism}) to $\reals$-- the
uniqueness of this Field is equivalent to the Continuum Hypothesis
(\S\ref{sec:continuum_hypothesis})



% ====================================================================
\section{Calculus of Variations}\label{sec:calculus_of_variations}
% ====================================================================

% --------------------------------------------------------------------
\subsection{Functional}\label{sec:functional}
% --------------------------------------------------------------------



% ====================================================================
\section{Constructive Analysis}\label{sec:constructive_analysis}
% ====================================================================

\emph{Choice Sequence}

% ====================================================================
\section{Computable Analysis}\label{sec:computable_analysis}
% ====================================================================

% ====================================================================
\section{Algorithm Analysis}\label{sec:algorithm_analysis}
% ====================================================================

% --------------------------------------------------------------------
\subsection{Linear Dominance}\label{sec:linear_dominance}
% --------------------------------------------------------------------

\[
    f \lesssim g \Leftrightarrow
    \exists x_0 \exists c : \forall x > x_0, |f(x)| \leq c |g(x)|
\]

Pointwise Domination implies Linear Dominance:
\[
    f \leq g \Rightarrow f \lesssim g
\]

\[
    f \lesssim g \wedge g \lesssim f \Leftrightarrow f \sim g
\]



\subsubsection{Big-O Notation}\label{sec:bigo_notation}

\[
    O(f) = \{ g : g \lesssim f \}
\]

\[
    \Omega(f) = \{ g : g \gtrsim f \}
\]

\[
    \Theta(f) = \{ g : g \sim f \}
\]

\[
    O(1) \subset O(x) \subset O(x^2) \subset O(x^2) \ldots
\]



% ====================================================================
\section{Spatial Analysis}\label{sec:spatial_analysis}
% ====================================================================

% FIXME

% --------------------------------------------------------------------
\subsection{Boundary Problem}\label{sec:boundary_problem}
% --------------------------------------------------------------------

% FIXME
