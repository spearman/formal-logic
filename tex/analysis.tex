%%%%%%%%%%%%%%%%%%%%%%%%%%%%%%%%%%%%%%%%%%%%%%%%%%%%%%%%%%%%%%%%%%%%%%
%%%%%%%%%%%%%%%%%%%%%%%%%%%%%%%%%%%%%%%%%%%%%%%%%%%%%%%%%%%%%%%%%%%%%%
\part{Mathematical Analysis}\label{part:mathematical_analysis}
%%%%%%%%%%%%%%%%%%%%%%%%%%%%%%%%%%%%%%%%%%%%%%%%%%%%%%%%%%%%%%%%%%%%%%
%%%%%%%%%%%%%%%%%%%%%%%%%%%%%%%%%%%%%%%%%%%%%%%%%%%%%%%%%%%%%%%%%%%%%%

% ====================================================================
\section{Series}\label{sec:series}
% ====================================================================

Sum of Terms of a Sequence (\S\ref{sec:sequence}) $a_n : \nats
\rightarrow \reals$



% --------------------------------------------------------------------
\subsection{Infinite Series}\label{sec:infinite_series}
% --------------------------------------------------------------------

\subsubsection{Partial Sum}\label{sec:partial_sum}

Sequence $\{ a_1, a_2, a_3, \ldots \}$

$S_n = \sum_{k=1}^n a_k$



\subsubsection{Convergent Series}\label{sec:convergent_series}

Limit (\S\ref{sec:sequence_limit}) of Partial Sums $\{ S_1, S_2, S_3,
\ldots \}$ Converges (\S\ref{sec:convergent_sequence})



\subsubsection{Divergent Series}\label{sec:divergent_series}



% --------------------------------------------------------------------
\subsection{Arithmetic Series}\label{sec:arithmetic_series}
% --------------------------------------------------------------------

Arithmetic Progression (Arithmetic Sequence
\S\ref{sec:arithmetic_sequence})



% --------------------------------------------------------------------
\subsection{Geometric Series}\label{sec:geometric_series}
% --------------------------------------------------------------------

Constant Ratio between successive Terms

Terms form a Geometric Progression (Geometric Sequence
\S\ref{sec:geometric_sequence})

Sum Converges as long as Absolute Value of the Ratio of Terms is less
than $1$



\subsubsection{Hypergeometric Series}\label{sec:hypergeometric_series}

Hypergeometric Function (\S\ref{sec:hypergeometric_function})



% --------------------------------------------------------------------
\subsection{Harmonic Series}\label{sec:harmonic_series}
% --------------------------------------------------------------------

% --------------------------------------------------------------------
\subsection{Alternating Series}\label{sec:alternating_series}
% --------------------------------------------------------------------

\[
  \sum_{n=0}^\infty (-1)^n a_n
\]
or:
\[
  \sum_{n=0}^\infty (-1)^{n-1} a_n
\]



% --------------------------------------------------------------------
\subsection{Power Series}\label{sec:power_series}
% --------------------------------------------------------------------

Infinite Series of the form:
\[
  f(x) = \sum_{n=0}^\infty a_n (x - c)^n
\]



\subsubsection{Taylor Series}\label{sec:taylor_series}

Centered at $0$: \emph{Maclaurin Series}

Polynomial formed by some initial Terms of a Taylor Series: Taylor
Polynomial (\S\ref{sec:taylor_polynomial}); Taylor Series is the Limit
(\S\ref{sec:sequence_limit}) of the Taylor Polynomials with increasing
Degree.

Note that a Function may not be equal to its Taylor Series even if its
Taylor Series Converges at every Point.

A Function that is equal to its Taylor Series in an Open Interval
(\S\ref{sec:interval}, or Disc \S\ref{sec:disc}) is an Analytic
Function (\S\ref{sec:analytic_function}) in that Interval.



\paragraph{Binomial Series}\label{sec:binomial_series}



% --------------------------------------------------------------------
\subsection{Telescoping Series}\label{sec:telescoping_series}
% --------------------------------------------------------------------



% ====================================================================
\section{Real Analysis}\label{sec:real_analysis}
% ====================================================================

% --------------------------------------------------------------------
\subsection{Maximum}\label{sec:maximum}
% --------------------------------------------------------------------

Upper Bound (\S\ref{sec:upper_bound})

Least Upper Bound (\S\ref{sec:least_upperbound})



% --------------------------------------------------------------------
\subsection{Minimum}\label{sec:minimum}
% --------------------------------------------------------------------

Lower Bound (\S\ref{sec:lower_bound})

Greatest Lower Bound (\S\ref{sec:greatest_lowerbound})



% --------------------------------------------------------------------
\subsection{Newton's Expansion}\label{sec:newtons_expansion}
% --------------------------------------------------------------------

% FIXME

$(1 + a)^n$ % ???



% --------------------------------------------------------------------
\subsection{Bernoulli's Inequality}\label{sec:bernoullis_inequality}
% --------------------------------------------------------------------

$n \in \nats$, $a \in \reals^+$, then:
\[
  (1 + a)^n \geq 1 + n a
\]


% --------------------------------------------------------------------
\subsection{Modulus of Continuity}\label{sec:continuity_modulus}
% --------------------------------------------------------------------



% ====================================================================
\section{Complex Analysis}\label{sec:complex_analysis}
% ====================================================================

% --------------------------------------------------------------------
\subsection{Complex Plane}\label{sec:complex_plane}
% --------------------------------------------------------------------

\subsubsection{Disc}\label{sec:disc}



% --------------------------------------------------------------------
\subsection{Holomorphic Function}\label{sec:holomorphic_function}
% --------------------------------------------------------------------

Complex-valued Function of one or more Complex Variables that is
Complex-differentiable in a Neighborhood (\S\ref{sec:neighborhood}) of
every Point in its Domain. Implies that any Holomorphic Function is
Infinitely Differentiable (\S\ref{sec:smooth_function}) and
equal to its own Taylor Series (\S\ref{sec:taylor_series}).



% --------------------------------------------------------------------
\subsection{Meromorphic Function}\label{sec:meromorphic_function}
% --------------------------------------------------------------------

\subsubsection{$L$-function}\label{sec:l_function}



% --------------------------------------------------------------------
\subsection{Univalent Function}\label{sec:univalent_function}
% --------------------------------------------------------------------

Injective Holomorphic Function on an Open Subset of the Complex Plane



% --------------------------------------------------------------------
\subsection{Modular Form}\label{sec:modular_form}
% --------------------------------------------------------------------

Automorphic Form (\S\ref{sec:automorphic_form})



% --------------------------------------------------------------------
\subsection{Riemann Surface}\label{sec:riemann_surface}
% --------------------------------------------------------------------

\subsubsection{Bolza Surface}\label{sec:bolza_surface}



% ====================================================================
\section{Functional Analysis}\label{sec:functional_analysis}
% ====================================================================

% --------------------------------------------------------------------
\subsection{Banach Space}\label{sec:banach_space}
% --------------------------------------------------------------------

% --------------------------------------------------------------------
\subsection{Banach Algebra}\label{sec:banach_algebra}
% --------------------------------------------------------------------

\subsubsection{C$^*$-algebra}\label{sec:cstar_algebra}

Involutive Algebra (\S\ref{sec:involutive_algebra})

Involution Semigroup (\S\ref{sec:involution_semigroup})



% ====================================================================
\section{Differential Equations}\label{sec:differential_equations}
% ====================================================================

% --------------------------------------------------------------------
\subsection{Ordinary Differential Equation}
\label{sec:ode}
% --------------------------------------------------------------------

\subsubsection{Continuously Differentiable}
\label{sec:continuously_differentiable}

$C^1$



\subsubsection{Hypergeometric Function}
\label{sec:hypergeometric_function}

Hypergeometric Series (\S\ref{sec:hypergeometric_series})



% --------------------------------------------------------------------
\subsection{Smooth Function}\label{sec:smooth_function}
% --------------------------------------------------------------------

$C^{\infty}$



\subsubsection{Analytic Function}\label{sec:analytic_function}



% ====================================================================
\section{Harmonic Analysis}\label{sec:harmonic_analysis}
% ====================================================================

% --------------------------------------------------------------------
\subsection{Automorphic Form}\label{sec:automorphic_form}
% --------------------------------------------------------------------



% ====================================================================
\section{Algebraic Analysis}\label{sec:algebraic_analysis}
% ====================================================================

% --------------------------------------------------------------------
\subsection{Generalized Function}\label{sec:generalized_function}
% --------------------------------------------------------------------

\subsubsection{Distribution}\label{sec:distribution}



% ====================================================================
\section{Non-standard Analysis}\label{sec:nonstandard_analysis}
% ====================================================================

Hyperreals (\S\ref{sec:hyperreal})



% ====================================================================
\section{Numerical Analysis}\label{sec:numerical_analysis}
% ====================================================================

% --------------------------------------------------------------------
\subsection{Interval Arithmetic}\label{sec:interval_arithmetic}
% --------------------------------------------------------------------



% ====================================================================
\section{Constructive Analysis}\label{sec:constructive_analysis}
% ====================================================================

\emph{Choice Sequence}

% ====================================================================
\section{Computable Analysis}\label{sec:computable_analysis}
% ====================================================================

% ====================================================================
\section{Algorithm Analysis}\label{sec:algorithm_analysis}
% ====================================================================

% --------------------------------------------------------------------
\subsection{Linear Dominance}\label{sec:linear_dominance}
% --------------------------------------------------------------------

\[
    f \lesssim g \Leftrightarrow
    \exists x_0 \exists c : \forall x > x_0, |f(x)| \leq c |g(x)|
\]

Pointwise Domination implies Linear Dominance:
\[
    f \leq g \Rightarrow f \lesssim g
\]

\[
    f \lesssim g \wedge g \lesssim f \Leftrightarrow f \sim g
\]



\subsubsection{Big-O Notation}\label{sec:bigo_notation}

\[
    O(f) = \{ g : g \lesssim f \}
\]

\[
    \Omega(f) = \{ g : g \gtrsim f \}
\]

\[
    \Theta(f) = \{ g : g \sim f \}
\]

\[
    O(1) \subset O(x) \subset O(x^2) \subset O(x^2) \ldots
\]
