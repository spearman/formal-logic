%%%%%%%%%%%%%%%%%%%%%%%%%%%%%%%%%%%%%%%%%%%%%%%%%%%%%%%%%%%%%%%%%%%%%%
%%%%%%%%%%%%%%%%%%%%%%%%%%%%%%%%%%%%%%%%%%%%%%%%%%%%%%%%%%%%%%%%%%%%%%
\part{Mathematical Analysis}\label{part:mathematical_analysis}
%%%%%%%%%%%%%%%%%%%%%%%%%%%%%%%%%%%%%%%%%%%%%%%%%%%%%%%%%%%%%%%%%%%%%%
%%%%%%%%%%%%%%%%%%%%%%%%%%%%%%%%%%%%%%%%%%%%%%%%%%%%%%%%%%%%%%%%%%%%%%

\fist Multivariate Analysis (Statistics
\S\ref{sec:multivariate_analysis})



% ====================================================================
\section{Analytic Expression}\label{sec:analytic_expression}
% ====================================================================

Infinite Series (\S\ref{sec:infinite_series})

Continued Fraction

Gamma Function (\S\ref{sec:gamma_function}): extension of the Factorial
Function to Real and Complex Numbers

Bessel Function (\S\ref{sec:bessel_function})

Analytic Expressions exclude Differentials (\S\ref{sec:differential}),
Integrals (\S\ref{sec:integral}), Limits (\S\ref{sec:limits}), and Formal Power
Series (\S\ref{sec:formal_power_series})

Closed-form Expressions (TODO: xref) are a more restricted class of Mathematical
Expressions that can be Evaluated in a Finite number of Operations



% ====================================================================
\section{Asymptotic Analysis}\label{sec:asymptotic_analysis}
% ====================================================================

%FIXME move sequence, limit, etc. here ?



% ====================================================================
\section{Recurrence Relation}\label{sec:recurrence_relation}
% ====================================================================

Recursive Definition (\S\ref{sec:recursive_definition})



% --------------------------------------------------------------------
\subsection{Difference Equation}\label{sec:difference_equation}
% --------------------------------------------------------------------

\fist Differential Equation (\S\ref{sec:differential_equation})



\subsubsection{Matrix Difference Equation}\label{sec:matrix_difference_equation}

Difference Equation in which the Value of a Vector or Matrix of Variables at
one Point in Time is related to its own Value at one or more previous Points in
Time, using Matrices



% --------------------------------------------------------------------
\subsection{Logistic Map}\label{sec:logistic_map}
% --------------------------------------------------------------------



% ====================================================================
\section{Sequence}\label{sec:sequence}
% ====================================================================

A \emph{Sequence} can be defined as a Countable Totally Ordered
Multiset (\S\ref{sec:multiset}), that is, a collection of Elements (or
\emph{Terms}) in a given order where duplicate Elements are allowed.
The number of Elements in a Sequence is referred to as its
\emph{Length}.

A \emph{Finite Sequence} is called a \emph{Tuple} (\S\ref{sec:tuple}).

String (\S\ref{sec:string})

Series (\S\ref{sec:series})

Sequence Space (\S\ref{sec:sequence_space})

Sequence (Topology) (\S\ref{sec:sequence_topology})

$a_n : \nats \rightarrow \reals$



% --------------------------------------------------------------------
\subsection{Subsequence}\label{sec:subsequence}
% --------------------------------------------------------------------

A Sequence $a_n$ Converges (\S\ref{sec:convergent_sequence}) to $l$
if and only if all Subsequences of $a_n$ Converge to $l$.



% --------------------------------------------------------------------
\subsection{Tuple}\label{sec:tuple}
% --------------------------------------------------------------------

% --------------------------------------------------------------------
\subsection{Limit}\label{sec:sequence_limit}
% --------------------------------------------------------------------

Can be defined in any Metric (\S\ref{sec:metric}) or Topological Space
(\S\ref{sec:topological_space}), generalized to a Topological Net
(\S\ref{sec:net}); see also Limits (\S\ref{sec:limit}) and Colimits
(\S\ref{sec:colimit}) in Category Theory.

The Limit of a Sequence is Unique

Bounded Sequence (\S\ref{sec:bounded_sequence})

$\lim a_n = 0 \Leftrightarrow \lim |a_n| = 0$

$\lim (a_n \pm b_n) = \lim a_n \pm \lim b_n$

$\lim (a_n b_n) = (\lim a_n) (\lim b_n)$

$\lim (\frac{a_n}{b_n}) = \frac{\lim a_n}{\lim b_n}$ for $\forall
n \in \nats, b_n \neq 0$ and $\lim b_n \neq 0$

$a_n \leq b_n$ for all $n$ Implies $\lim a_n \leq \lim b_n$, but it is
not the case that $a_n < b_n$ for all $n$ Implies $\lim a_n < \lim
b_n$ since $a_n$ and $b_n$ may still be equal in the Limit.

For an Open Interval $(x,y)$, $\lim a_n \in (x,y)$ Implies $a_n \in
(x,y)$ for all $n$, but for not for a Closed Interval.



\subsubsection{Limit Inferior}\label{sec:liminf}

$\liminf$

$\underline{\lim}$



\subsubsection{Limit Superior}\label{sec:limsup}

$\limsup$

$\overline{\lim}$



\subsubsection{Convergent Sequence}\label{sec:convergent_sequence}

$\forall \varepsilon > 0, \exists N : n \geq N \Rightarrow |(a_n - l)| <
\varepsilon$

A Sequence $a_n$ Converges to $l$ if and only if all Subsequences
(\S\ref{sec:subsequence}) of $a_n$ Converge to $l$.

If $\lim a_n = l$, then for $k \in \ints$, $\lim (a_{n+k}) = l$

If $a_n$ is Convergent then it is Bounded
(\S\ref{sec:bounded_sequence}).

If $F$ is a Closed Set (\S\ref{sec:closed_set}), then for any Sequence
$x_n$ in $F$ can Converge to $x$ if and only if $x$ is in $F$.



\paragraph{Squeeze Theorem}\label{sec:squeeze_theorem}\hfill

For Sequences $a_n \leq b_n \leq c_n$ and $\lim a_n = l$ and $\lim c_n
= l$, then $\lim b_n = l$



\subsubsection{Divergent Sequence}\label{sec:divergent_sequence}



% --------------------------------------------------------------------
\subsection{Bounded Sequence}\label{sec:bounded_sequence}
% --------------------------------------------------------------------

If $a_n$ is Convergent (\S\ref{sec:convergent_sequence}) then it is
Bounded.



\subsubsection{Bolzano-Weierstrass Theorem}
\label{sec:bolzano_weierstrass}

\emph{Sequential Compactness Theorem}

Lemma: Every Sequence $a_n$ of Real Numbers has a Monotone
Subsequence

The \emph{Bolzano-Weierstrass Theorem} states that a Bounded Sequence
$a_n$ has at least one Subsequence that Converges.



% --------------------------------------------------------------------
\subsection{Arithmetic Sequence}\label{sec:arithmetic_sequence}
% --------------------------------------------------------------------

(or \emph{Arithmetic Progression})

Arithmetic Series (\S\ref{sec:arithmetic_series})



% --------------------------------------------------------------------
\subsection{Geometric Sequence}\label{sec:geometric_sequence}
% --------------------------------------------------------------------

(or \emph{Geometric Progression})

Sum of Terms of a Geometric Sequence form a Geometric Series
(\S\ref{sec:geometric_series})



% --------------------------------------------------------------------
\subsection{Infinite Sequence}\label{sec:infinite_sequence}
% --------------------------------------------------------------------

\emph{Infinite Sequences} may be \emph{Singly Infinite}
(\S\ref{sec:singly_infinite}), having an initial Element but no final
Element, or \emph{Doubly Infinite} (\S\ref{sec:doubly_infinite})
having neither a first nor a last Element.



\subsubsection{Singly Infinite Sequence}\label{sec:singly_infinite}

A \emph{Singly Infinite Sequence} (or \emph{One-sided Infinite
  Sequence}) can be defined as a Function, $s$, with a Countably
Infinite Totally Ordered Set of Indices, $X$, for its Domain and a Set
of Elements, $Y$, for the Codomain:

  $s : X \rightarrow Y$ \\
where:

  $X = \{1,2,\ldots,n\}$

  $Y = \{a_1, a_2,\ldots,a_n\}$

  $s = \{(1,a_1), (2,a_2),\ldots, (n,a_n)\}$ \\
for some Countable $n \geq 0$.

Singly Infinite Sequences may be interpreted as Elements of the
Semigroup Ring of the Natural Numbers, $R[\mathbb{N}]$
(\S\ref{sec:group_ring}).



\subsubsection{Doubly Infinite Sequence}\label{sec:doubly_infinite}

A \emph{Doubly Infinite Sequence} (also \emph{Two-way Infinite} or
\emph{Bi-infinite Sequence}) may be defined as a Function from the Set
of all Integers $\mathbb{Z}$ into a Set, denoted
$(2n)^{\infty}_{n=-\infty}$.

Doubly Infinite Sequences may be interpreted as Elements of the Group
Ring of the Integers, $R[\mathbb{Z}]$ (\S\ref{sec:group_ring}.



% --------------------------------------------------------------------
\subsection{Monotone Sequence}\label{sec:monotone_sequence}
% --------------------------------------------------------------------

Monotone Function (\S\ref{sec:monotonic_function})

Increasing Sequence: $\forall n \in \nats, a_n \leq a_{n+1}$

An Increasing Sequence is Bounded above if and only if it is
Convergent.

Decreasing Sequence: $\forall n \in \nats, a_n \geq a_{n+1}$

A Decreasing Sequence is Bounded below if and only if it is
Convergent.



% --------------------------------------------------------------------
\subsection{Cauchy Sequence}\label{sec:cauchy_sequence}
% --------------------------------------------------------------------

Complete Metric Space (\S\ref{sec:complete_metric_space})



% --------------------------------------------------------------------
\subsection{Oscillation}\label{sec:oscillation}
% --------------------------------------------------------------------



% ====================================================================
\section{Series}\label{sec:series}
% ====================================================================

Sum of Terms of a Sequence (\S\ref{sec:sequence}) $a_n : \nats
\rightarrow \reals$



% --------------------------------------------------------------------
\subsection{Infinite Series}\label{sec:infinite_series}
% --------------------------------------------------------------------

Transcendental Numbers (\S\ref{sec:transcendental})

some examples:
\begin{itemize}
  \item $\frac{1}{4} - \frac{1}{16} + \frac{1}{64} - \frac{1}{256} \cdots
    = \frac{1}{4}$
  \item $\frac{1}{2} - \frac{1}{4} + \frac{1}{8} - \frac{1}{16} \cdots
    = \frac{1}{3}$
  \item $1 - \frac{1}{2} + \frac{1}{3} - \frac{1}{4} \cdots  = \ln{2}$
  \item $1 - \frac{1}{3} + \frac{1}{5} - \frac{1}{7} \cdots  = \frac{\pi}{4}$
  \item $1 + \frac{1}{4} + \frac{1}{9} + \frac{1}{16} \cdots = \frac{\pi^2}{6}$
\end{itemize}



\subsubsection{Partial Sum}\label{sec:partial_sum}

Sequence $\{ a_1, a_2, a_3, \ldots \}$

$S_n = \sum_{k=1}^n a_k$



\subsubsection{Convergent Series}\label{sec:convergent_series}

Limit (\S\ref{sec:sequence_limit}) of Partial Sums $\{ S_1, S_2, S_3,
\ldots \}$ Converges (\S\ref{sec:convergent_sequence})



\subsubsection{Divergent Series}\label{sec:divergent_series}

\paragraph{Abelian Mean}\label{sec:abelian_mean}\hfill

\paragraph{Abel Summation}\label{sec:abel_summation}\hfill

Analytic Number Theory (\S\ref{sec:analytic_number_theory})

$a_n$ Sequence of Complex Numbers

$f(t)$ Differentiable Function (\S\ref{sec:differentiable_function})

$A(x) = \sum_{n \leq x} a_n$

$\sum_{n \leq x} a_n f(n) = A(x)f(x) - \int_1^x A(t)f'(t) dt$



% --------------------------------------------------------------------
\subsection{Arithmetic Series}\label{sec:arithmetic_series}
% --------------------------------------------------------------------

Arithmetic Progression (Arithmetic Sequence
\S\ref{sec:arithmetic_sequence})



% --------------------------------------------------------------------
\subsection{Geometric Series}\label{sec:geometric_series}
% --------------------------------------------------------------------

Constant Ratio between successive Terms

Terms form a Geometric Progression (Geometric Sequence
\S\ref{sec:geometric_sequence})

Sum Converges as long as Absolute Value of the Ratio of Terms is less
than $1$



\subsubsection{Hypergeometric Series}\label{sec:hypergeometric_series}

\paragraph{Hypergeometric Function}\label{sec:hypergeometric_function}\hfill

is a Solution of a Second-order Linear ODE (\S\ref{sec:linear_ode})

Hypergeometric Series (\S\ref{sec:hypergeometric_series})



% --------------------------------------------------------------------
\subsection{Harmonic Series}\label{sec:harmonic_series}
% --------------------------------------------------------------------

% --------------------------------------------------------------------
\subsection{Alternating Series}\label{sec:alternating_series}
% --------------------------------------------------------------------

\[
  \sum_{n=0}^\infty (-1)^n a_n
\]
or:
\[
  \sum_{n=0}^\infty (-1)^{n-1} a_n
\]



% --------------------------------------------------------------------
\subsection{Power Series}\label{sec:power_series}
% --------------------------------------------------------------------

Infinite Series of the form:
\[
  f(x) = \sum_{n=0}^\infty a_n (x - c)^n
\]

\fist Formal Power Series (\S\ref{sec:formal_power_series})



\subsubsection{Taylor Series}\label{sec:taylor_series}

Centered at $0$: \emph{Maclaurin Series}

Polynomial formed by some initial Terms of a Taylor Series: Taylor
Polynomial (\S\ref{sec:taylor_polynomial}); Taylor Series is the Limit
(\S\ref{sec:sequence_limit}) of the Taylor Polynomials with increasing
Degree.

Note that a Function may not be equal to its Taylor Series even if its
Taylor Series Converges at every Point.

A Function is \emph{Analytic} (\S\ref{sec:analytic_function}) if and only if
its Taylor Series about $x_0$ Converges to the Function in some Neighborhood
for every $x_0$ in its Domain.

A Function that is equal to its Taylor Series in an Open Interval
(\S\ref{sec:interval}, or Disc \S\ref{sec:disc}) is an Analytic Function in
that Interval.

Maclaurin Series:
\[
  \sum_{n=0}^\infty f^{(n)}(0) \frac{x^n}{n!}
\]



\paragraph{Binomial Series}\label{sec:binomial_series}\hfill



% --------------------------------------------------------------------
\subsection{General Dirichlet Series}\label{sec:general_dirichlet}
% --------------------------------------------------------------------

%FIXME: move to subsection?



\subsubsection{Dirichlet Series}\label{sec:dirichlet_series}



% --------------------------------------------------------------------
\subsection{Telescoping Series}\label{sec:telescoping_series}
% --------------------------------------------------------------------



% ====================================================================
\section{Infinite Product}\label{sec:infinite_product}
% ====================================================================

Converges when the Sequence converges to $1$

$\prod_{n=1}^\infty a_n$ Converges if and only if $\sum_{n=1}^\infty
ln(a_n)$ Converges

$q_n = (1 + u_1)(1 + u_2)\cdots(1 + u_n)$ Converges if and only if
$\sum u_n$ Converges.



% ====================================================================
\section{Real Analysis}\label{sec:real_analysis}
% ====================================================================

$R^1$ -- Real Line (\S\ref{sec:real_line}): 1-dimensional Real
Coordinate Space

(Infinitesimal) Calculus:
\begin{itemize}
\item Differential Calculus: Differentiable Functions
  (\S\ref{sec:differentiable_function})
\item Integral Calculus: Integrable Functions (\S\ref{sec:integrable_function})
\end{itemize}
uses notions of Convergent Infinite Sequences (\S\ref{sec:convergent_sequence})
and Convergent Infinite Series (\S\ref{sec:convergent_series}); related by
Fundamental Theorem of Calculus (\S\ref{sec:fundamental_calculus_theorem})

\fist cf. Calculus of Finite Differences
(\S\ref{sec:finite_differences_calculus})

\fist extension of Calculus in one Variable to Functions of several Variables:
Multivariable Calculus (\S\ref{sec:multivariable_calculus})



% --------------------------------------------------------------------
\subsection{Real Interval}\label{sec:real_interval}
% --------------------------------------------------------------------

Interval (\S\ref{sec:interval})

Interval Arithmetic (\S\ref{sec:interval_arithmetic})

$[a,b] = \bigcap_n (a - \frac{1}{n}, b + \frac{1}{n})$



\subsubsection{Interval Partition}\label{sec:interval_partition}

Closed Interval $[a,b]$

Finite Sequence (\S\ref{sec:sequence}) $(x_i) = \{ x_0, x_1, \ldots,
x_n \}$

$a = x_0 < x_1 < x_2 < \ldots < x_n = b$



% --------------------------------------------------------------------
\subsection{Maximum}\label{sec:maximum}
% --------------------------------------------------------------------

Upper Bound (\S\ref{sec:upper_bound})

Least Upper Bound (\S\ref{sec:least_upperbound})



% --------------------------------------------------------------------
\subsection{Minimum}\label{sec:minimum}
% --------------------------------------------------------------------

Lower Bound (\S\ref{sec:lower_bound})

Greatest Lower Bound (\S\ref{sec:greatest_lowerbound})



% --------------------------------------------------------------------
\subsection{Newton's Expansion}\label{sec:newtons_expansion}
% --------------------------------------------------------------------

% FIXME

$(1 + a)^n$ % ???



% --------------------------------------------------------------------
\subsection{Bernoulli's Inequality}\label{sec:bernoullis_inequality}
% --------------------------------------------------------------------

$n \in \nats$, $a \in \reals^+$, then:
\[
  (1 + a)^n \geq 1 + n a
\]


% --------------------------------------------------------------------
\subsection{Real Function}\label{sec:real_function}
% --------------------------------------------------------------------

or \emph{Real-valued Function}

\fist Vector-valued Functions (\S\ref{sec:vector_function})



\subsubsection{Bounded Function}\label{sec:bounded_function}

a Bounded Function on a Compact Interval $[a,b]$ is Riemann Integrable
(\S\ref{sec:integrable_function}) if and only if it is Continuous
(\S\ref{sec:continuous_function}) ``Almost Everywhere'', i.e. Set of Points of
Discontinuity has Lebesgue Measure Zero (\S\ref{sec:lebesgue_measure})



\subsubsection{Function Limit}\label{sec:function_limit}

Limit Point (\S\ref{sec:limit_point}) of $D \in \reals$ is $a \in D$
such that:
\[
  \exists a_n \in D : a_n \neq a \wedge \lim a_n = a
\]

Function $f$ has a \emph{Limit} $l$ at $a$ if for all Sequences $a_n
\in D$ with $\lim a_n = a$ and $a_n \neq a$ for all $n$, $\lim f(a_n)
= l$



\subsubsection{Level Set}\label{sec:level_set}

a special case of a Fiber (\S\ref{sec:fiber})



\subsubsection{Newton's Method}\label{sec:newtons_method}



% --------------------------------------------------------------------
\subsection{Real-valued Continuous Function}\label{sec:real_continuous}
% --------------------------------------------------------------------

$f : D \subseteq \reals \rightarrow \reals$

Continuous at $l \in D$ if for every Sequence $a_n \in D$ such that
$\lim a_n = l$, then $\lim f(a_n) = f(l)$

$\lim f (a_n) = f (\lim a_n)$

Equivalently: Continuous at $l$ if and only if:
\[
  \forall \varepsilon > 0, \exists \delta :
  |x - l| < \delta \Rightarrow |f(x) - f(l)| < \varepsilon
\]

Equivalently: Continuous at $a$ if and only if $f$ has Limit
(\S\ref{sec:function_limit}) $f(a)$ at $a$: $\lim_{x \rightarrow
  a}f(x) = f(a)$

Continuous on an Interval if and only if the Range on that Interval is
also a single Interval

For $f,g$ Continuous on $D$ at $a \in D$, then $(f + g)$, $f g$,
$\frac{f}{g}$ (when $g(a) \neq 0$) are Continuous at $a$.

For $g$ Defined on the Range of $f$, $\{ f(x); x \in D\}$, if $f$ is
Continuous at $a \in D$ and $g$ Continuous at $f(a)$, then $g \circ f$
is Continuous at $a$: $\lim g(f(a_n)) = g(f(a))$

Differentiable (\S\ref{sec:differentiable_function}) at $a$ Implies
Continuous at $a$



\subsubsection{Local Extrema}\label{sec:local_extrema}

Local Maximum

Local Minimum

for Differentiable (\S\ref{sec:differentiable_function}) $f$ on an
Open Interval: $f'(a) = 0$ at Local Extrema



\paragraph{Second Derivative Test}\label{sec:second_derivative_test}\hfill



\subsubsection{Intermediate Value Theorem}
\label{sec:intermediate_value}

For Function $f(x)$ Continuous on Closed Interval $[a,b]$, for any
$f(a) < c < f(b)$, there is a $d \in (a,b)$ such that $f(d) = c$.



\subsubsection{Extreme Value Theorem}\label{sec:extreme_value}

\subsubsection{Modulus of Continuity}\label{sec:continuity_modulus}



% --------------------------------------------------------------------
\subsection{Differentiable Function}\label{sec:differentiable_function}
% --------------------------------------------------------------------

For $f$ defined on Open Interval $(a,b) \subset \reals$, $f$ is
Differentiable at $x \in (a,b)$ if Limit $f'(x) = \lim_{h \rightarrow
  0} \frac{f (x+h) - f(x)}{h}$ exists.

Differentiable at $a$ Implies Continuous (\S\ref{sec:continuous_function}) at
$a$

\fist Differential Equation (\S\ref{sec:differential_equation})

A Polynomial (\S\ref{sec:polynomial}), being the Sum of Differentiable
Functions, is Differentiable everywhere

For $f$ Differentiable on an Interval $I$, if $\forall x \in I, f'(x)
> 0$, $f$ is Strictly Increasing and if $\forall x \in I, f'(x) < 0$,
$f$ is Strictly Decreasing (see Monotonic Functions
\S\ref{sec:monotonic_function}).



\subsubsection{Derivative}\label{sec:derivative}

$(\frac{g}{f})' = \frac{f g' - f' g}{f^2}$

often viewed as a Quotient of \emph{Differentials} (\S\ref{sec:differential}):
\[
  \frac{dy}{dx}
\]

\fist Vector Derivative (\S\ref{sec:vector_derivative})

\emph{Differentiation Operator}

a Differential Operator (\S\ref{sec:differential_operator}) is an Operator
defined as a Function of the Differentiation Operator



\subsubsection{Differential Operator}\label{sec:differential_operator}

%FIXME move?

$\nabla$

Operator defined as a Function of the Differentiation (Derivative
\S\ref{sec:derivative}) Operator



\paragraph{Theta Operator}\label{sec:theta_operator}\hfill

\paragraph{Elliptic Operator}\label{sec:elliptic_operator}\hfill

\subparagraph{Laplace Operator}\label{sec:laplace_operator}\hfill

$\Delta = \nabla^2$

Discrete Laplace Operator (Graph Theory \S\ref{sec:discrete_laplace})

the Set of Harmonic Functions (\S\ref{sec:harmonic_function}) on a given Open
Set $U$ can be seen as the Kernel of the Laplace Operator

\fist cf. Laplace Transform (\S\ref{sec:laplace_transform}) %FIXME same concept?



\paragraph{Schwarzian Derivative}\label{sec:schwarzian_derivative}\hfill

Non-linear Differential Operator



\subsubsection{Antiderivative}\label{sec:antiderivative}

\emph{Indefinite Integral} or \emph{Primitive Integral}

If $f$ is Continuous on $[a,b]$ then:
\[
  F(x) = \int_a^x f
\]
for all $x \in [a,b]$ and $F$ is Differentiable on $(a,b)$ and $F' = f$.

\fist related to Definite Integrals (\S\ref{sec:definite_integral}) through the
Fundamental Theorem of Calculus (\S\ref{sec:fundamental_calculus_theorem})



\subsubsection{Differentiability Class}\label{sec:differentiability_class}

$\mathcal{C}^k$

$\mathcal{C}^1$ -- Continuous First Derivative: Continuously Differentiable
Function (\S\ref{sec:continuously_differentiable})

$\mathcal{C}^2$ -- Continuous Second Derivative

$\mathcal{C}^\infty$ -- Continuous for all Derivatives: Smooth Function
(\S\ref{sec:smooth_function})



\subsubsection{Continuously Differentiable}
\label{sec:continuously_differentiable}

A Function $f$ is \emph{Continuously Differentiable} if the Derivative $f'(x)$
exists and is a Continuous Function.

Differentiability Class (\S\ref{sec:differentiability_class}) $C^1$



\subsubsection{Smooth Function}\label{sec:smooth_function}

Differentiability Class (\S\ref{sec:differentiability_class}) $C^{\infty}$



\paragraph{Analytic Function}\label{sec:analytic_function}\hfill

A Function is \emph{Analytic} if and only if its Taylor Series
(\S\ref{sec:analytic_function}) about $x_0$ Converges to the Function in some
Neighborhood for every $x_0$ in its Domain.

Complex Dynamics (\S\ref{sec:complex_dynamics})

Trigonometric Functions (\S\ref{sec:trigonometric_function})



\subparagraph{Analytic Continuation}\label{sec:analytic_continuation}



\subsubsection{Chain Rule}\label{sec:chain_rule}

(wiki):

Formula for computing the Derivative of the \emph{Composition} of two or more
Functions:
\[
  (f \circ g)' = (f' \circ g) g'
\]
or equivalently when $F(x) = f(g(x))$:
\[
  F'(x) = f'(g(x))g'(x)
\]

Leibniz Notation for Variable $z$ Dependent on Variable $y$ Dependent on
Independent Variable $x$:
\[
  \frac{dz}{dx} = \frac{dz}{dy} \cdot \frac{dy}{dx}
\]
and if $z = f(y)$ and $y = g(x)$ then:
\[
  \frac{dz}{dx} = \frac{dz}{dy} \cdot \frac{dy}{dx}
    = f'(y)g'(x) = f'(g(x))g'(x)
\]

\fist cf. \emph{Substitution Rule} (\S\ref{sec:substitution_rule}) for
Integration

Multivariable Chain Rule



\subsubsection{Cauchy Mean Value Theorem}
\label{sec:cauchy_mean_value}

For Functions $f$, $g$ Continuous on $[a,b]$, Differentiable on
$(a,b)$ then $\exists c \in (a,b)$ such that:
\[
  f'(c) (g(b) - g(a)) = g'(c) (f(b) - f(a))
\]



\subsubsection{Lagrange Mean Value Theorem}
\label{sec:lagrange_mean_value}

$f$ Continuous on $[a,b]$ and Differentiable on $(a,b)$ then $\exists
c \in (a,b)$ such that $f(b) - f(a) = f'(c)(b-a)$ or:
\[
  f'(c) = \frac{f(b) - f(a)}{b - a}
\]



\subsubsection{Rolle's Theorem}\label{sec:rolles_theorem}

for $f$ Continuous on $[a,b]$ and Differentiable on $(a,b)$ with $f(a)
= f(b)$, then $\exists c \in (a,b)$ such that $f'(c) = 0$



% --------------------------------------------------------------------
\subsection{Integrable Function}\label{sec:integrable_function}
% --------------------------------------------------------------------

%FIXME: is this the same as Riemann-integrable ???

a Bounded Function (\S\ref{sec:bounded_function}) on a Compact Interval $[a,b]$
is Riemann Integrable if and only if it is Continuous
(\S\ref{sec:continuous_function}) ``Almost Everywhere'', i.e. Set of Points of
Discontinuity has Lebesgue Measure Zero (\S\ref{sec:lebesgue_measure})

cf. Integral Equation (\S\ref{sec:integral_equation})

\fist Integrable System (\S\ref{sec:integrable_system})

$f$ Bounded on Closed Bounded $[a,b]$, $\forall \varepsilon >0$, there
exists a Partition (\S\ref{sec:interval_partition}) $P$ of $[a,b]$
such that $0 \leq U(f,P) - L(f,P) < \varepsilon$ %FIXME xref upper lower

Monotone Functions (\S\ref{sec:monotonic_function}) and Continuous
Functions (\S\ref{sec:continuous_function}) are always (Riemann)
Integrable

$L(f,P) \leq \int_a^b f \leq U(f,P)$

For $c$ a Constant, if $f$ is Integrable then $cf$ is Integrable: $c
\int_a^b f = \int_a^b c f$

For $f$, $g$ Integrable on $[a,b]$, then $f + g$ is Integrable:
$\int_a^b (f+g) = \int_a^b f + \int_a^b g$

For $f$ Integrable on $[a,b]$ and $a < c < b$ then $f$ is Integrable
on $[a,c]$ and $[c,b$ and $\int_a^b f = \int_a^c f + \int_c^b f$

For $f$, $g$ Integrable on $[a,b]$ and $f \leq g$ on $[a,b]$, then
$\int_a^b f \leq \int_a^b g$

For Continuous $g$ and Integrable $f$, then $g \circ f$ is Integrable

For Integrable $f$, $|f|$ is Integrable and $|\int_a^b f| \leq
\int_a^b |f|$



\subsubsection{Integral}\label{sec:integral}

\fist Multiple Integral (\S\ref{sec:multiple_integral}) in Multivariable
Calculus

cf. Numerical Integration (\S\ref{sec:numerical_integration})



\paragraph{Definite Integral}\label{sec:definite_integral}\hfill

\fist related to Antiderivative (\S\ref{sec:antiderivative}) through the
Fundamental Theorem of Calculus (\S\ref{sec:fundamental_calculus_theorem})



\paragraph{Indefinite Integral}\label{sec:indefinite_integral}\hfill

\fist Antiderivative (\S\ref{sec:antiderivative})



\paragraph{Darboux Integral}\label{sec:darboux_integral}\hfill

Upper Darboux Sum

Lower Darboux Sum



\paragraph{Riemann Integral}\label{sec:riemann_integral}\hfill

$\int_a^b f(x) dx$

Interval Partition (\S\ref{sec:interval_partition})

Monotone Bounded Functions are Riemann Integrable

Continuous Functions are Reiemann Integrable

Riemann Integrable on $[a,b]$, Least Upper Bound of Lower Darboux Sums
is equal to the Greatest Lower Bound of the Upper Darboux Sums, value
denoted by $\int_a^b f$



\paragraph{Lebesgue Integral}\label{sec:lebesgue_integral}\hfill

\paragraph{Wallis Integral}\label{sec:wallis_integral}\hfill



\subsubsection{Substitution Rule}\label{sec:substitution_rule}

Fundamental Theorem of Calculus (\S\ref{sec:fundamental_calculus_theorem})

\fist cf. \emph{Chain Rule} (\S\ref{sec:chain_rule}) for Differentiation



\subsubsection{Integral Mean Value Theorem}
\label{sec:integral_mean_value}

For $f$ Continuous on $[a,b]$, there is a $c \in [a,b]$ such that
$\int_a^b f = f(c)(b - a)$



\subsubsection{Integral Transform}\label{sec:integral_transform}

Transformation of an Equation from original Domain into another Domain where it
is solved more easily and then mapped back into the original domain using the
Inverse of the Integral Transform



\paragraph{Laplace Transform}\label{sec:laplace_transform}\hfill

takes a Function of a Real Variable $t$ (usually Time) to a Function of a
Complex Variable $s$ (usually Frequency + Phase)

Frequency-domain approach for Continuous Time Signals where the System may be
Stable \emph{or} Unstable (TODO: clarify)

\fist cf. Laplace Operator (\S\ref{sec:laplace_operator}) %FIXME same concept?

in Control Theory (\S\ref{sec:control_theory}), Systems are often Transformed
from the Time Domain to the Frequency Domain using Laplace Transform and the
Transformed System's Zeros (\S\ref{sec:complex_zero}) and Poles
(\S\ref{sec:complex_pole}) are Analyzed in the Complex Plane



% --------------------------------------------------------------------
\subsection{Fundamental Theorem of Calculus}
\label{sec:fundamental_calculus_theorem}
% --------------------------------------------------------------------

for Function $f$ Continuous (\S\ref{sec:continuous_function}) on the Interval
$[a,b]$, and $F$ with Derivative $f$:
\[
  F' = f
\]
on the Interval $(a,b)$, then:
\[
  \int_a^b f(x) dx = F(b) - F(a)
\]

Additionally, for every $x$ in the Interval $(a,b)$:
\[
  \frac{d}{dx}\int^x_a f(t) dt = f(x)
\]

\fist Subsitution Rule (\S\ref{sec:substitution_rule})

\fist Integral Theorems of Vector Calculus
(\S\ref{sec:integral_theorems}) in Multivariable Calculus



% --------------------------------------------------------------------
\subsection{Logistic Function}\label{sec:logistic_function}
% --------------------------------------------------------------------

%FIXME: move section ?

\subsubsection{Sigmoid Function}\label{sec:sigmoid_function}



% ====================================================================
\section{Complex Analysis}\label{sec:complex_analysis}
% ====================================================================

Complex Numbers (\S\ref{sec:complex_number})

Complex Differentiable Functions are automatically Analytic %FIXME

a Laplace Transform (\S\ref{sec:laplace_transform}) transforms a Function from
the Time Domain to the Complex Frequency Domain so it can be analyzed in the
Complex Plane



% --------------------------------------------------------------------
\subsection{Zero}\label{sec:complex_zero}
% --------------------------------------------------------------------

% --------------------------------------------------------------------
\subsection{Pole}\label{sec:complex_pole}
% --------------------------------------------------------------------

% --------------------------------------------------------------------
\subsection{Euler's Formula}\label{sec:eulers_formula}
% --------------------------------------------------------------------

% --------------------------------------------------------------------
\subsection{Complex Surface}\label{sec:complex_surface}
% --------------------------------------------------------------------

\subsubsection{Enriques-Kodaira Classification}
\label{sec:enriques_kodaira}

\subsubsection{Complex Plane}\label{sec:complex_plane}

or \emph{Argand Plane}



\subsubsection{Disc}\label{sec:disc}\hfill

\subsubsection{Extended Complex Plane}\label{sec:extended_complex_plane}

Complex Plane with Point at Infinity

Stereographic Projection (\S\ref{sec:stereographic_projection})

Riemann Sphere (\S\ref{sec:riemann_sphere})



% --------------------------------------------------------------------
\subsection{Holomorphic Function}\label{sec:holomorphic_function}
% --------------------------------------------------------------------

Complex-valued Function of one or more Complex Variables that is
Complex-differentiable in a Neighborhood (\S\ref{sec:neighborhood}) of
every Point in its Domain. Implies that any Holomorphic Function is
Infinitely Differentiable (\S\ref{sec:smooth_function}) and equal to
its own Taylor Series (\S\ref{sec:taylor_series}).

Real or Imaginary part of any Homomorphic Function is a Harmonic
Function (\S\ref{sec:harmonic_function})



% --------------------------------------------------------------------
\subsection{Meromorphic Function}\label{sec:meromorphic_function}
% --------------------------------------------------------------------

\subsubsection{Elliptic Function}\label{sec:elliptic_function}

Double-periodic Functions (\S\ref{sec:double_periodic})

Pendulum (Non-linear Dynamical System \S\ref{sec:nonlinear_dynamical_system})



\subsubsection{$L$-function}\label{sec:l_function}

\subsubsection{Gamma Function}\label{sec:gamma_function}

$\Gamma(\alpha) = \int_0^{\infty} x^{\alpha -1} e^{-x} dx$

for $\alpha > 0$

\begin{enumerate}
\item $\Gamma(\alpha) = (\alpha - 1) \Gamma(\alpha -1)$
\item $\Gamma(n) = (n-1)!$
\item $\Gamma(1) = 1$
\item $\Gamma(\sfrac{1}{2}) = \sqrt{pi}$
\end{enumerate}

Gamma Distribution (\S\ref{sec:gamma_distribution})



% --------------------------------------------------------------------
\subsection{Univalent Function}\label{sec:univalent_function}
% --------------------------------------------------------------------

Injective Holomorphic Function on an Open Subset of the Complex Plane



% --------------------------------------------------------------------
\subsection{Modular Form}\label{sec:modular_form}
% --------------------------------------------------------------------

Automorphic Form (\S\ref{sec:automorphic_form})



% --------------------------------------------------------------------
\subsection{$n$-th Root}\label{sec:nth_root}
% --------------------------------------------------------------------

\emph{Radical}



% --------------------------------------------------------------------
\subsection{Riemann Surface}\label{sec:riemann_surface}
% --------------------------------------------------------------------

One-dimensional Complex Manifold (\S\ref{sec:complex_manifold})

the Geometry of Riemann Surfaces is given by Two-dimensional Conformal
Geometry (\S\ref{sec:conformal_geometry})



\subsubsection{Riemann Mapping Theorem}
\label{sec:riemann_mapping_theorem}

\subsubsection{Riemann Sphere}\label{sec:riemann_sphere}

``simplest'' Riemann Surface

Model of the Extended Complex Plane
(\S\ref{sec:extended_complex_plane})

can be thought of as the \emph{Complex Projective Line}
$\mathbb{CP}^1$: the Projective Space (\S\ref{sec:projective_space} of
all Complex Lines in $\comps^2$



\subsubsection{Bolza Surface}\label{sec:bolza_surface}



% --------------------------------------------------------------------
\subsection{Countour Integration}\label{sec:contour_integration}
% --------------------------------------------------------------------

Residue Calculus (???) %FIXME



% ====================================================================
\section{Calculus of Finite Differences}\label{sec:finite_differences_calculus}
% ====================================================================

%FIXME

% --------------------------------------------------------------------
\subsection{Finite Difference}\label{sec:finite_difference}
% --------------------------------------------------------------------

Discrete equivalent of Differentiation (\S\ref{sec:differential})



% ====================================================================
\section{Functional Equation}\label{sec:functional_equation}
% ====================================================================

%FIXME: move this elsewhere ?

Functionals (\S\ref{sec:functional})

any Equation in which the Unknown (Variable \S\ref{sec:variable}) is a Function

e.g. an Additive Function (\S\ref{sec:additive_function}) $f$ is one satisfying
the Functional Equation $f(x + y) = f(x) + f(y)$


examples:
\begin{itemize}
  \item Schr\"oder's Equation (\S\ref{sec:schroeders_equation}): $\Psi(h(x)) =
    s\Psi(x)$ where $h(x)$ is a given Function and $\Psi(x)$ is Unknown;
    Solutions specify \emph{Flow} (\S\ref{sec:flow}), a generalization of
    Function Iteration Count (\S\ref{sec:iterated_function}) to a Continuous
    Parameter
\end{itemize}



% ====================================================================
\section{Differential Equation}\label{sec:differential_equation}
% ====================================================================

The \emph{Solution} to a Differential Equation $F(x,y,y',\ldots,y^{(n)}) = 0$
of Order $n$ is a \emph{Function} or \emph{Class of Functions} of the form $u :
I \subset \reals \rightarrow \reals$ where $u$ is an $n$-times Differentiable
Function (\S\ref{sec:differentiable_function}) on $I$ and $\forall x \in I$:
\[
  F(x,u,u',\ldots,u^{(n)}) = 0
\]
Such a $u$ defines an \emph{Integral Curve} (\S\ref{sec:integral_curve}) for
$F$.

\fist cf. the Solution of a Polynomial Equation
(\S\ref{sec:polynomial_equation}) is a Value or Set of Values

In a Dynamical System (\S\ref{sec:dynamical_system}), the \emph{Evolution
  Function} $\Phi^t$ is often the Solution of a Differential Equation of Motion
$\dot{x} = v(x)$ giving the Time Derivative of a Trajectory (Integral Curve)
$x(t)$ on the Phase Space starting at some Point $x_0$.

Extension, Maximal Solution, Global Solution, General Solution

\fist Partial Differential Equation (\S\ref{sec:pde}) in Multivariable Calculus

\fist Numerical Integration (\S\ref{sec:numerical_integration})

\fist Difference Equation (\S\ref{sec:difference_equation})

\fist Differential System (\S\ref{sec:differential_system})




% --------------------------------------------------------------------
\subsection{Differential}\label{sec:differential}
% --------------------------------------------------------------------

an Infinitesimally small quantity $dx$

cf. Derivative (\S\ref{sec:derivative}); often viewed as a \emph{Quotient} of
Differentials: $\frac{dy}{dx}$

\fist Dicrete equivalent: Finite Differences (\S\ref{sec:finite_difference})



% --------------------------------------------------------------------
\subsection{Ordinary Differential Equation}\label{sec:ode}
% --------------------------------------------------------------------

An \emph{Ordinary Differential Equation (ODE)} is an Equation containing a
Function of one Independent Variable and its Derivatives.

\fist A \emph{Partial} Differential Equation (PDE
\S\ref{sec:pde}) is a Differential Equation
containing Unknown Multivariable Functions and their Partial Derivatives.

\fist System of ODEs (\S\ref{sec:system_of_odes})

\fist Flow (\S\ref{sec:flow})

For $F$ a Function of $x$, $y$, and Derivatives (\S\ref{sec:derivative}) of
$y$, an Equation of the form:
\[
  F(x,y,y',\ldots,y^{(n-1)}) = y^{(n)}
\]
is a \emph{Explicit Ordinary Differential Equation} of \emph{Order $n$}, and an
\emph{Implicit Ordinary Differential Equation} of Order $n$ has the form:
\[
  F(x,y,y',\ldots,y^{(n)}) = 0
\]



\subsubsection{Slope Field}\label{sec:slope_field}

When the Differential Equation is represented as a Vector Field or Slope Field
(\S\ref{sec:slope_field}), the corresponding Integral Curves
(\S\ref{sec:integral_curve}) are Tangent to the Field at each point.



\subsubsection{Linear Ordinary Differential Equation}\label{sec:linear_ode}

An Ordinary Differential Equation is \emph{Linear} if $F$ can be written as a
Linear Combination (\S\ref{sec:linear_combination}) of the Derivatives of $y$:
\[
  y^{(n)} = \sum_{i=0}^{n-1} a_i(x)y^{(i)} + r(x)
\]
where $a_i(x)$ and $r(x)$ are Continuous Functions in $x$ and $r(x)$ is called
the \emph{Source Term}.

Homogenous

Non-homogenous



% --------------------------------------------------------------------
\subsection{Stochastic Differential Equation}\label{sec:sde}
% --------------------------------------------------------------------

\fist Stochastic Partial Differential Equation (\S\ref{sec:spde})



% --------------------------------------------------------------------
\subsection{Functional Differential Equation}\label{sec:fde}
% --------------------------------------------------------------------

a Differential Equation with \emph{Deviating Argument}

TODO



\subsubsection{Differential Difference Equation}\label{sec:dde}



% --------------------------------------------------------------------
\subsection{Singular Solution}\label{sec:singular_solution}
% --------------------------------------------------------------------

% --------------------------------------------------------------------
\subsection{Boundary Value Problem}\label{sec:boundary_value_problem}
% --------------------------------------------------------------------

%FIXME harmonic analysis? dirichlet problem?

\subsubsection{Finite Element Method}\label{sec:finite_element_method}

\emph{Finite Element Method} or \emph{FEM} is a technique for solving
Boundary Value Problems for Partial Differential Equations
(\S\ref{sec:partial_differential})

GetFEM++ -- LGPL C++ library



% --------------------------------------------------------------------
\subsection{Bessel function}\label{sec:bessel_function}
% --------------------------------------------------------------------

cf. Reverse Bessel Polynomials (\S\ref{sec:reverse_bessel_polynomial})



% ====================================================================
\section{Integral Equation}\label{sec:integral_equation}
% ====================================================================

cf. Integrable Function (\S\ref{sec:integrable_function})



% ====================================================================
\section{Multivariable Calculus}\label{sec:multivariable_calculus}
% ====================================================================

extension of (Infinitesimal) Calculus in one Variable to Functions with several
Variables

Calculus in one Variable:
\begin{itemize}
\item Differential Calculus: Differentiable Functions
  (\S\ref{sec:differentiable_function}), Differential Equations
  (\S\ref{sec:differential_equation})
\item Integral Calculus: Integrable Functions
  (\S\ref{sec:integrable_function}), Integral Equation
  (\S\ref{sec:integral_equation})
\end{itemize}

\fist Vector Fields (\S\ref{sec:vector_field})

\fist Differential Forms (\S\ref{sec:differential_form}): approach to
Multivariable Calculus that is \emph{independent} of Coordinates

\emph{Three-dimensional Graph}: Two-dimensional Input with One-dimensional
Output; Contour Plots (\S\ref{sec:contour})

Multivariable Chain Rule

2 Variables:
\[
  \frac{d}{dt} f(x(t),y(t)) =
    \frac{\partial{f}}{\partial{x}} \cdot \frac{dx}{dt}
      + \frac{\partial{f}}{\partial{y}} \cdot \frac{dy}{dt}
\]
Vector form where $\vec{v}(t) = [x(t),y(t)]$:
\[
  \frac{d}{dt}f(\vec{v}(t)) = \nabla{f(\vec{v}(t)}\bullet{\vec{v}'(t)}
\]
where $\nabla{f}$ is the Gradient (\S\ref{sec:gradient}) of $f$; this the same
as taking the Directional Derivative (\S\ref{sec:directional_derivative}):
\[
  \nabla_{\vec{v}'(t)}f(\vec{v}(t))
\]



% --------------------------------------------------------------------
\subsection{Contour}\label{sec:contour}
% --------------------------------------------------------------------

Contour Map



% --------------------------------------------------------------------
\subsection{Parametric Function}\label{sec:parametric_function}
% --------------------------------------------------------------------

\fist \emph{Vector Function} (\S\ref{sec:vector_function})



\subsubsection{Curvature}\label{sec:curvature}

Radius of Curvature $R$

\emph{Curvature} $\kappa = \frac{1}{R}$

for Unit Tangent Vector $\hat{T}$, Arclength $s$:
\[
  \kappa = \|\frac{d\hat{T}}{ds}\|
    = \|\frac{\frac{d\hat{T}}{dt}}{\frac{d\vec{r}}{dt}}\|
\]
where $\vec{r}(t)$ is some Parametric Function

for a Plane Curve (\S\ref{sec:plane_curve}) $\vec{r}(t) = [x(t),y(t)]$:
\[
  \kappa = \frac{x'(t)y''(t) - y'(t)x''(t)} {(x'(t)^2 + y'(t)^2)^{\frac{3}{2}}}
\]
where the Numerator is equal to $\vec{r}'(t) \times \vec{r}''(t)$ and:
\[
  \kappa = \frac{\vec{r}' \times \vec{r}''}{\|\vec{r}'\|^3}
\]



\subsubsection{Parametric Equation}\label{sec:parametric_equation}

\fist Parametric Surface (\S\ref{sec:parametric_surface})



% --------------------------------------------------------------------
\subsection{Multivariate Continuity}\label{sec:multivariate_continuity}
% --------------------------------------------------------------------

% --------------------------------------------------------------------
\subsection{Partial Derivative}\label{sec:partial_derivative}
% --------------------------------------------------------------------

the Matrix of all First-order Partial Derivatives of a Vector Function
(\S\ref{sec:vector_function}) is the \emph{Jacobian Matrix}
(\S\ref{sec:jacobian_matrix})

(\emph{Schwarz's Theorem}) if the Second Partial Derivative is Continuous at the
given Point, then the Partial Differentiations of the Function are Commutative
at that Point:
\[
  \frac{\partial^2f}{\partial{x}\partial{y}}
    = \frac{\partial^2f}{\partial{y}\partial{x}}
\]

Gradient (\S\ref{sec:gradient}): $\nabla f$

Partial Derivative of a Multivariable Vector-valued Function (TODO); Parametric
Surface (\S\ref{sec:parametric_surface})

Partial Derivative of Vector Fields (\S\ref{sec:vector_field}) ... TODO

the \emph{Hessian Matrix} (\S\ref{sec:hessian_matrix}) is the Square Matrix of
Second-order Partial Derivatives of a Scalar-valued Function



\subsubsection{Partial Differential}\label{sec:partial_differential}

\subsubsection{Partial Differential Equation}\label{sec:pde}

A \emph{Partial Differential Equation (PDE)} is a Differential Equation
containing Unknown Multivariable Functions and their Partial Derivatives.

\fist An \emph{Ordinary} Differential Equation (ODE \S\ref{sec:ode}) is a
Differential Equation with a single Independent Variable.

Finite Element Method (\S\ref{sec:finite_element_method}): solving
Boundary Value Problems (\S\ref{sec:boundary_value_problem}) for
Partial Differential Equations

\fist Numerical Integration (\S\ref{sec:numerical_integration})



\paragraph{Wave Equation}\label{sec:dirichlet_problem}\hfill

\subparagraph{Spherical Wave}\label{sec:spherical_wave}\hfill

Separation of Variables (Fourier Method)

Spherical Wave Transformation (\S\ref{sec:spherical_wave_transformation})
leaves the form of Spherical Waves Invariant in all Inertial Frames
(\S\ref{sec:inertial_frame})



\paragraph{Elliptic Partial Differential Equation}
\label{sec:elliptic_partial_differential}\hfill

\subparagraph{Poisson Equation}\label{sec:poisson_equation}\hfill

$\nabla^2 u = u_{xx} + u{yy} = f(x,y)$



\subparagraph{Laplace's Equation}\label{sec:laplaces_equation}\hfill

Second-order Elliptic Partial Differential Equation

written:
\[ \nabla^2 \varphi = 0 \]
or:
\[ \Delta \varphi = 0 \]

where $\Delta = \nabla^2$ is the Laplace Operator
(\S\ref{sec:laplace_operator}) and $\varphi$ is a Scalar Function
(\S\ref{sec:scalar_function})

a Harmonic Function (\S\ref{sec:harmonic_function}) is a Twice-continuously
Differentiable Function (\S\ref{sec:continuously_differentiable}) $f : U
\rightarrow Reals$, where $U$ is an Open Subset of $\reals^n$, satisfying
Laplace's Equation

$\nabla^2 u = u_{xx} + u_{yy} = 0$ %FIXME



\paragraph{Dirichlet Problem}\label{sec:dirichlet_problem}\hfill

\paragraph{Stochastic Partial Differential Equation}\label{sec:spde}\hfill

\fist Stochastic Differential Equation (\S\ref{sec:sde})



\subsubsection{Directional Derivative}\label{sec:directional_derivative}

$\frac{\partial{f}}{\partial{\vec{v}}}$

$\nabla_{\vec{v}} f = \vec{v}\bullet \nabla{f}$

Gradient (\S\ref{sec:gradient})



% --------------------------------------------------------------------
\subsection{Multiple Integral}\label{sec:multiple_integral}
% --------------------------------------------------------------------

Integral (\S\ref{sec:integral})

Multiple Integration



\subsubsection{Double Integral}\label{sec:double_integral}

\subsubsection{Triple Integral}\label{sec:triple_integral}

\subsubsection{Line Integral}\label{sec:line_integral}

\subsubsection{Surface Integral}\label{sec:surface_integral}

Double Integral analog of the Line Integral



\subsubsection{Volume Integral}\label{sec:volume_integral}

used to calculate ``\emph{Flux Densities}'' %FIXME



% --------------------------------------------------------------------
\subsection{Exterior Derivative}\label{sec:exterior_derivative}
% --------------------------------------------------------------------



\subsubsection{Differential Form}\label{sec:differential_form}

approach to Multivariable Calculus (\S\ref{sec:multivariable_calculus})
that is \emph{independent} of Coordinates



% --------------------------------------------------------------------
\subsection{Vector Calculus}\label{sec:vector_calclulus}
% --------------------------------------------------------------------

or \emph{Vector Analysis}

Gradient, Divergence, Curl

$\nabla$

FIXME:

Directional Derivative

Laplacian

Tensor Derivative



\subsubsection{Vector Function}\label{sec:vector_function}

or \emph{Vector-valued Function}

\fist Parametric Function (\S\ref{sec:parametric_function})

the Matrix of all First-order Partial Derivatives
(\S\ref{sec:partial_derivative}) of a Vector Function is the \emph{Jacobian
  Matrix} (\S\ref{sec:jacobian_matrix})

Partial Derivative of a Multivariable Vector-valued Function (TODO)



\paragraph{Root}\label{sec:function_root}\hfill

A \emph{Root} (or \emph{Zero}) of a Vector-valued Function (including Real- or
Complex-valued Functions) $f$ is an Element $x$ of the Domain of $f$ for which
$f(x)$ is \emph{Zero}.

\fist Root (Equations \S\ref{sec:equation_root})

Any Polynomial (\S\ref{sec:polynomial}) with Odd Degree has at least
one Real Root.

The Fundamental Theorem of Algebra (\S\ref{sec:fundamental_algebra_theorem})
states that every Polynomial of Degree $n$ has $n$ Complex Roots, counted with
their Multiplicities (??? FIXME).



\paragraph{Vector Derivative}\label{sec:vector_derivative}\hfill

$\frac{d\vec{r}(t)}{dt}$



\subsubsection{Gradient}\label{sec:gradient}

Partial Derivative (\S\ref{sec:partial_derivative})

Directional Derivative (\S\ref{sec:directional_derivative})

(wiki):

the \emph{Gradient} of a Scalar Field (\S\ref{sec:scalar_field}) $f$ is the
Vector Derivative (\S\ref{sec:vector_derivative}):
\[
  \nabla f =
    \frac{\partial f}{\partial x}\vec{e}_x +
    \frac{\partial f}{\partial y}\vec{e}_y +
    \frac{\partial f}{\partial z}\vec{e}_z
\]

Gradient of a Vector Field (FIXME)

Divergence (\S\ref{sec:divergence}) can be expressed as:
\[
  \nabla \cdot \vec{v} =
    \frac{\partial v_x}{\partial x} +
    \frac{\partial v_y}{\partial y} +
    \frac{\partial v_z}{\partial z}
\]

the Laplacian (\S\ref{sec:laplacian}) is the Divergence of the Gradient Field
of $f$: $\nabla \bullet \nabla f$

Curl (\S\ref{sec:curl}) can be expressed as:
\[
  \nabla\times\vec{v} =
    (\frac{\partial v_z}{\partial y}-\frac{\partial v_y}{\partial z})\vec{e}_x +
    (\frac{\partial v_x}{\partial z}-\frac{\partial v_z}{\partial x})\vec{e}_y +
    (\frac{\partial v_y}{\partial x}-\frac{\partial v_x}{\partial y})\vec{e}_z
\]



\paragraph{Gradient Flow}\label{sec:gradient_flow}\hfill

Flow (\S\ref{sec:flow})

Geometric Flow (\S\ref{sec:geometric_flow})



\subsubsection{Divergence}\label{sec:divergence}

wiki:

the \emph{Divergence} of a Vector Field (\S\ref{sec:vector_field})
$\vec{v}(x,y,z) = v_x\vec{e}_x + v_y\vec{e}_y + v_z\vec{e}_z$ is a
Scalar-valued Function (\S\ref{sec:scalar_function}):
\[
  \nabla \cdot \vec{v} =
    \frac{\partial v_x}{\partial x} +
    \frac{\partial v_y}{\partial y} +
    \frac{\partial v_z}{\partial z}
\]
where $\nabla$ is the Gradient Operator (\S\ref{sec:gradient})

\emph{Convergence}

...

\[
  \vec{v}(x,y,z) = \begin{bmatrix}
    p(x,y,z) \\
    q(x,y,z) \\
    r(x,y,z)
  \end{bmatrix}
\]

\[
  div\vec{v}(x,y,z)
    = \frac{\partial{p}}{\partial{x}}
    + \frac{\partial{q}}{\partial{y}}
    + \frac{\partial{r}}{\partial{z}}
\]

the Laplacian (\S\ref{sec:laplacian}) is the Divergence of the Gradient Field
of $f$: $\nabla \bullet \nabla f$


\subsubsection{Curl}\label{sec:curl}

wiki:

the \emph{Curl} of a Vector Field (\S\ref{sec:vector_field}) $\vec{v}(x,y,z) =
v_x\vec{e}_x + v_y\vec{e}_y + v_z\vec{e}_z$ is a Vector-valued Function
(\S\ref{sec:vector_function}):
\[
  \nabla\times\vec{v} =
    (\frac{\partial v_z}{\partial y}-\frac{\partial v_y}{\partial z})\vec{e}_x -
    (\frac{\partial v_x}{\partial z}-\frac{\partial v_z}{\partial x})\vec{e}_y +
    (\frac{\partial v_y}{\partial x}-\frac{\partial v_x}{\partial y})\vec{e}_z
\]
where $\nabla$ is the Gradient Operator (\S\ref{sec:gradient}) associating each
Point in the Vector Field with the proportional ``on-axis'' Torque to which a
``tiny pinwheel'' would be subjected if it were centered at the Point and can
be realized as a Pseudo-determinant (\S\ref{sec:pseudo_determinant}):

TODO

Right-hand Rule

Counter-clockwise: Positive

Clockwise: Negative



\subsubsection{Laplacian}\label{sec:laplacian}

the \emph{Laplacian}, $\Delta$, is the Divergence (\S\ref{sec:divergence}) of
the Gradient Field of $f$:
\[
  \Delta f = \nabla \bullet \nabla f
\]

analog of Second Derivative for Multivariable Scalar-valued Functions



\subsubsection{Integral Theorems of Vector Calculus}
\label{sec:integral_theorems}

cf. Fundamental Theorem of Calculus (\S\ref{sec:fundamental_calculus_theorem})

\paragraph{Gradient Theorem}\label{sec:gradient_theorem}\hfill

\paragraph{Stoke's Theorem}\label{sec:stokes_theorem}\hfill

\paragraph{Divergence Theorem}\label{sec:divergence_theorem}\hfill

\paragraph{Green's Theorem}\label{sec:greens_theorem}\hfill



% --------------------------------------------------------------------
\subsection{Matrix Calculus}\label{sec:matrix_calculus}
% --------------------------------------------------------------------

% --------------------------------------------------------------------
\subsection{Tensor Calculus}\label{sec:tensor_calculus}
% --------------------------------------------------------------------

or \emph{Tensor Analysis}



\subsubsection{Tensor}\label{sec:linear_tensor}

wikipedia: Geometric Objects (???) that describe Linear Relations
between Scalars, Vectors and other Tensors; Linear Relations such as
Dot Product, Cross Product, Linear Maps

%FIXME: merge with sec:tensor ???

\fist Tensor (Abstract Algebra \S\ref{sec:tensor})

\fist Metric Tensor (\S\ref{sec:metric_tensor})

must be independent of a particular choice of Coordinate System
(Coordinate-free \S\ref{sec:coordinate_free}): the particular Covariant
Transformation Law (\S\ref{sec:covariant_transformation}) determines
the \emph{Valence} (\S\ref{sec:valence}) of a Tensor

may be represented by:
\begin{itemize}
  \item Multidimensional Arrays (Basis Independence not apparent)
  \item Multilinear Maps (intrinsic Basis Independence)
  \item Elements of (Abstract) Tensor Products
\end{itemize}

\fist
\url{https://jeremykun.com/2014/01/17/how-to-conquer-tensorphobia/}
%FIXME cite

Tensors are Elements (Vectors) of a Vector Space given by
combining two ``smaller'' Vector Spaces via a Tensor Product

$v \otimes w \in V \otimes W$

Scalar Multiplication: $s(v \otimes w) = (sv \otimes w) = (v \otimes
sw)$; generalizing to  $n$-fold Tensor Products, Scalars can be moved
around all the coordinates freely

Addition Operation: $(v \otimes w) + (v' \otimes w) = (v + v') \otimes
w)$ \emph{or} $(v \otimes w) + (v \otimes w') = v \otimes (w + w')$,
otherwise $(x \otimes y) + (z \otimes w)$ is a ``new'' Tensor
(Vector); generalizing to $n$-fold Tensor Products, Addition can be
combined if all but one of the coordinates are the same in the
Addends.

Element of a Tensor Space $V_1 \otimes \cdots \otimes V_n$ as a Sum:
\[
  \Sigma_k a_{1,k} \otimes a_{2,k} \otimes cdots \otimes a_{n,k}
\]

A Rank $1$ or \emph{Pure Tensor} is one that can be expressed as a
One-term Sum, i.e. just $a_1 \otimes \cdots \otimes a_n$


\asterism


represented as an ``organized'' Multidimensional Array of Numerical
(?Scalar) Values

\emph{Order} (or \emph{Degree}) of a Tensor $x$ is the
Dimensionality of the Array needed to represent it or equivalently the
minimum number of Terms to represent $x$ as a Sum of Pure Tensors (the
Zero Element is Order $0$ by convention).

Order $0$ Tensor -- Scalar (\S\ref{sec:scalar})

Order $1$ Tensor -- Vector (\S\ref{sec:vector})

Order $2$ Tensor -- Linear Map (\S\ref{sec:linear_map}) ? %FIXME correct?

Computing Tensor Order is $NP$-hard when $k = \rats$ and $NP$-complete
when $k$ is a Finite Field %FIXME


\asterism


Multilinear Maps


\asterism


a Type (\S\ref{sec:valence}) $(n,m)$ Tensor $T$ is defined as an
Element of the Tensor Product (\S\ref{sec:tensor_product}) of Vector
Spaces $V$ and Dual Spaces $V^*$:
\[
  T \in V_1 \otimes \cdots \otimes V_n
    \otimes V^*_1 \otimes \cdots \otimes V^*_m
\]

the Tensor Product is Initial with respect to Multilinear Mappings
from the Direct Product

Tensors (Abstract Algebra \S\ref{sec:tensor})

Tensor Products (Category Theory \S\ref{sec:tensor_product}):
\begin{itemize}
  \item Modules
  \item Vector Spaces
  \item Graded Vector Spaces
  \item $R$-algebras
  \item Sheaves of Modules
  \item Quadratic Forms
  \item Multilinear Form
  \item Topological Vector Spaces
\end{itemize}

Monoidal Category (\S\ref{sec:monoidal_category}): general context for
Tensor Products



\subsubsection{Valence}\label{sec:valence}

precise Covariant Transformation Law
(\S\ref{sec:covariant_transformation}) determines the \emph{Valence}
(or \emph{Type}) of a Tensor

Tensor Type: Pair of Natural Numbers $(n,m)$ where $n$ is the number
of Contravariant Indicies and $m$ is the number of Covariant Indices

\emph{Total Order} is the sum of $n$ and $m$



\subsubsection{Module Tensor Product}\label{sec:module_tensor}

Balanced Product



\subsubsection{Tensor Field}\label{sec:tensor_field}

Tensor assigned to each Point in a Space

cf. Scalar Field (\S\ref{sec:scalar_field})

cf. Vector Field (\S\ref{sec:vector_field})



\subsubsection{Tensor Algebra}\label{sec:tensor_algebra}

of a Vector Space



\subsubsection{Eigenconfiguration}\label{sec:eigenconfiguration}

\subsubsection{Grassmann Algebra}\label{sec:grassmann_algebra}

Exterior Algebra (\S\ref{sec:exterior_algebra})

\paragraph{$p$-vector}\label{sec:p_vector}\hfill

\paragraph{Multivector}\label{sec:multivector}\hfill


\subsubsection{Ricci Calculus}\label{sec:ricci_calculus}

\subsubsection{Spinor}\label{sec:spinor}



% ====================================================================
\section{Harmonic Analysis}\label{sec:harmonic_analysis}
% ====================================================================

% --------------------------------------------------------------------
\subsection{Harmonic Function}\label{sec:harmonic_function}
% --------------------------------------------------------------------

Twice-continuously Differentiable Function
(\S\ref{sec:continuously_differentiable}) $f : U \rightarrow Reals$,
where $U$ is an Open Subset of $\reals^n$, satisfying \emph{Laplace's
  Equation} (\S\ref{sec:laplaces_equation})

the Set of Harmonic Functions on a given Open Set $U$ can be seen as the Kernel
of the Laplace Operator (\S\ref{sec:laplace_operator})

2016 - Samantha Davies - \emph{Voltage, Temperature, and Harmonic
  Functions} -
\url{https://jeremykun.com/2016/09/26/voltage-temperature-and-harmonic-functions/}

Physics: Simple Harmonic Motion is a type of ``Oscillation'' where the
Force that restores an object to its Equilibrium is directly
proportional to the Displacement

Stochastic Processes (\S\ref{sec:stochastic_process})

Potential Theory (Mathematical Physics) %FIXME create section?

the Real or Imaginary part of any Homomorphic Function
(\S\ref{sec:holomorphic_function}) is a Harmonic Function

Harmonic Functions on $\reals$ are exactly the Linear Functions
(\S\ref{sec:polynomial_function})


\textbf{Mean Value Property}: the Value of a Harmonic Function at an
Interior Point is the Average of the Function's Values around the
Point

If $u$ is a Continuous Function satisfying the Mean Value Property on
a Region $\Omega$ then $u$ is Harmonic in $\Omega$

Any Function $\alpha$ on a Graph which satisfies the Mean Value
Property also satisfies the Discrete Laplacian
(\S\ref{sec:discrete_laplace})


\textbf{Maximum Principle}: a Non-constant Harmonic Function on a
Closed Bounded Region must attain Maximum and Minimum on its Boundary

Convex Optimization (\S\ref{sec:convex_optimization})


\textbf{Uniqueness}: if a Harmonic Function is Continuous on the
Boundary of a Closed Bounded Set and Harmonic in the Interior then the
Interior Values are Uniquely Determined by the Values of the Boundary


\textbf{Solution to a Dirichlet Problem}
(\S\ref{sec:dirichlet_problem})



% --------------------------------------------------------------------
\subsection{Automorphic Form}\label{sec:automorphic_form}
% --------------------------------------------------------------------

% --------------------------------------------------------------------
\subsection{Fourier Analysis}\label{sec:fourier_analysis}
% --------------------------------------------------------------------

\fist Convolution (\S\ref{sec:convolution}), Cross-correlation
(\S\ref{sec:cross_correlation})



\subsubsection{Periodic Function}\label{sec:periodic_function}

a Function $f$ is \emph{Periodic} with Nonzero \emph{Period} $P$ if:
\[
  f(x+P) = f(x)
\]
for all $x$

the least positive $P$ is called the \emph{Fundamental Period}

Geometrically, a Periodic Function is a Function with a Graph exhibiting
\emph{Translational Symmetry} (\S\ref{sec:translation});
can be extended to Tessellations of the Plane (\S\ref{sec:tessellation})



\paragraph{Trigonometric Function}\label{sec:trigonometric_function}\hfill

Analytic Functions (\S\ref{sec:analytic_function})

the Sine Wave is the only Periodic Waveform that has the Property that it
retains its ``wave shape'' when added to another Sine Wave of the same
Frequency and arbitrary Phase and Magnitude (FIXME: clarify)

$\sin x = \sum_{n=0}^\infty \frac{(-1)^n x^{2n+1}}{(2n + 1)!}$

$\cos\theta= \sin(\frac{\pi}{2} - \theta)$

$\tan\theta = \frac{\sin\theta}{\cos\theta}$

$\sec\theta = \frac{1}{\cos\theta}$

$\csc\theta = \frac{1}{\sin\theta}$

$\cot\theta = \frac{\cos\theta}{\sin\theta}$

Integrals:

$\int_0^{2\pi} \sin(mt)dt = 0$ for any Integer $m$

$\int_0^{2\pi} \cos(mt)dt = 0$ for any Non-zero Integer $m$

$\int_0^{2\pi} \sin(mt)\cos(nt)dt = 0$ for any Integers $m,n$

$\int_0^{2\pi} \sin^2(mt)dt = \pi$ for any Non-zero Integer $m$

$\int_0^{2\pi} \sin(mt)\sin(nt)dt = 0$ for Integers $m,n$ such that $m\neq{n}$
or $m\neq{-n}$

$\int_0^{2\pi} \cos(mt)\cos(nt)dt = 0$ for Integers $m,n$ such that $m\neq{n}$
or $m\neq{-n}$

$\int_0^{2\pi} \cos(mt)dt = \pi$ for any Non-zero Integer $m$

\fist Fourier Series (\S\ref{sec:fourier_series})



\subsubsection{Double-periodic Function}\label{sec:double_periodic}

e.g. Elliptic Functions (\S\ref{sec:elliptic_function})



\subsubsection{Fourier Series}\label{sec:fourier_series}

representation of an arbitrary Periodic Function by a Series of weighted $cos$
and $sin$ (\S\ref{sec:trigonometric_function}) Terms

(wiki):

for a Function $s(x)$ Integrable on an Interval $[x_0,x_0 + P]$ that is
Periodic outside the Interval with Period $P$ (Frequency $1/P$), then it can be
approximated on the entire Real Line as a Series of ``Harmonically related''
Sinusoidal Functions (FIXME: clarify)

the $N$th Partial Sum:
\[
  s_N(x) = \frac{a_0}{2} + \sum_{n=1}^N \left(
    a_n\cos(\frac{2\pi{nx}}{P}) + b_n\sin(\frac{2\pi{nx}}{P})
  \right)
\]
where $a_0,\ldots,a_N$ and $b_1,\ldots,b_N$ are \emph{Fourier Coefficients}
computed by:
\begin{align*}
  a_n & = \frac{2}{P}\int_{x_0}^{x_0+P}s(x) \cdot \cos(\frac{2\pi{nx}}{P}) dx \\
  b_n & = \frac{2}{P}\int_{x_0}^{x_0+P}s(x) \cdot \sin(\frac{2\pi{nx}}{P}) dx
\end{align*}

the Coefficient $\frac{a_0}{2} = \frac{1}{P}\int_{x_0}^{x_0+P}s(x)dt$
represents the \emph{Average} of $s(x)$ over the Interval $[x_0,P]$

equivalently the $N$th Partial Sum can be defined as:
\[
  s_N(x) = \sum_{n=-N}^N c_n \cdot e^i\frac{2\pi{nx}}{P}
\]
where:
\[
  c_n \defeq \begin{cases}
    \frac{1}{2}(a_n - ib_n) &\ \text{for n > 0} \\
    \frac{1}{2}a_0          &\ \text{for n = 0} \\
    c_{|n|}^*               &\ \text{for n < 0}
  \end{cases}
\]
(FIXME: $^*$ is the complex conjugate ???)

the Infinite Sum $s_\infty(x)$ is called the \emph{Fourier Series}
representation of $s(x)$


\emph{Convolution Theorems}

Convolution (\S\ref{sec:convolution}), Cross-correlation
(\S\ref{sec:cross_correlation})



\subsubsection{Continuous Fourier Trasform}
\label{sec:continuous_fourier_transform}

\subsubsection{Discrete-Time Fourier Trasform (DTFT)}\label{sec:dtft}



% --------------------------------------------------------------------
\subsection{Spherical Harmonics}\label{sec:spherical_harmonics}
% --------------------------------------------------------------------



% ====================================================================
\section{Functional Analysis}\label{sec:functional_analysis}
% ====================================================================

Inner Product Spaces (Normed Vector Spaces
\S\ref{sec:innerproduct_space})

Hilbert Space (\S\ref{sec:hilbert_space})

Topological Vector Space (\S\ref{sec:topological_vector})



% --------------------------------------------------------------------
\subsection{Bounded Linear Operator}\label{sec:bounded_linear_operator}
% --------------------------------------------------------------------

a Linear Transformation between two Normed Vector Spaces
(\S\ref{sec:normed_vectorspace}) is a Bounded Linear Operator if and
only if it is a Continuous Linear Operator
(\S\ref{sec:continuous_linear})

an Orthogonal Projection (\S\ref{sec:orthogonal_projection}) is a Bounded
Linear Operator



\subsubsection{Hermitian Adjoint}\label{sec:hermitian_adjoint}

Adjoint Operator (\S\ref{sec:adjoint_operator})

Complex Hilbert Space (\S\ref{sec:hilbert_space})



% --------------------------------------------------------------------
\subsection{Closed Linear Span}\label{sec:closed_linear_span}
% --------------------------------------------------------------------

Linear Span (\S\ref{sec:linear_span})



% --------------------------------------------------------------------
\subsection{Sequence Space}\label{sec:sequence_space}
% --------------------------------------------------------------------

Vector Space (\S\ref{sec:vector_space}) whose Elements are Infinite Sequences
(\S\ref{sec:sequence}) of Real or Complex Numbers, or equivalently a Function
Space whose Elements are Functions from $\nats$ to the Field $K$ of Real or
Complex Numbers



% --------------------------------------------------------------------
\subsection{Banach Space}\label{sec:banach_space}
% --------------------------------------------------------------------

A \emph{Banach Space} is a Vector Space (\S\ref{sec:vector_space}) $X$ over the
Field of Real or Complex Numbers

a Complete (\S\ref{sec:complete_metric_space}) Normed Vector Space
(\S\ref{sec:normed_vectorspace})

Topological Vector Space (\S\ref{sec:topological_vector})



\subsubsection{$L^p$ Space}\label{sec:lp_space}



% --------------------------------------------------------------------
\subsection{Banach Algebra}\label{sec:banach_algebra}
% --------------------------------------------------------------------

\subsubsection{C$^*$-algebra}\label{sec:cstar_algebra}

A \emph{C$^*$-algebra} a Complex Algebra $A$ of Continuous Linear
Operators (\S\ref{sec:continuous_linear}) on a Complex Hilbert Space
(\S\ref{sec:hilbert_space}) with the additional Properties that $A$ is
Topologically Closed in the Norm Topology of Operators and Closed
under the Operation of taking Adjoints of Operators

Involutive Algebra (\S\ref{sec:involutive_algebra})

Involution Semigroup (\S\ref{sec:involution_semigroup})

(wiki):

Linear Logic (\S\ref{sec:linear_logic}) can be seen as the refining
the Interpretation of Classical Logic by replacing Boolean Algebras
(\S\ref{sec:boolean_algebra}) by C$^*$-algebras



\paragraph{von Neumann Algebra}\label{sec:vonneumann_algebra}\hfill



% --------------------------------------------------------------------
\subsection{Cross-correlation}\label{sec:cross_correlation}
% --------------------------------------------------------------------

or \emph{Sliding Dot Product}

Similarity Measure (\S\ref{sec:similarity_measure}) of two Series as a Function
of the Displacement of one relative to the other

for Continuous Signals the Cross-correlation Operator is the Adjoint Operator
of the Convolution (\S\ref{sec:convolution}) Operator

\fist Fourier Analysis (\S\ref{sec:fourier_analysis})



\subsubsection{Autocorrelation}\label{sec:autocorrelation}



% --------------------------------------------------------------------
\subsection{Convolution}\label{sec:convolution}
% --------------------------------------------------------------------

Operation on two Functions giving a modified Function giving the Integral of
the Pointwise Multiplication of the two Functions as a Function of the amount
that one of the original Functions is Translated (\S\ref{sec:translation})

for Continuous Signals the Cross-correlation (\S\ref{sec:cross_correlation})
Operator is the Adjoint Operator of the Convolution Operator

Image Processing

\fist Fourier Analysis (\S\ref{sec:fourier_analysis})



% --------------------------------------------------------------------
\subsection{Coherence Space}\label{sec:coherence_space}
% --------------------------------------------------------------------

note: not the same as Coherent Space (Spectral Space)

cf. Linear Logic (\S\ref{sec:linear_logic}) Semantics



% --------------------------------------------------------------------
\subsection{Spectral Theory}\label{sec:spectral_theory}
% --------------------------------------------------------------------

Hilbert Space (\S\ref{sec:hilbert_space})



\subsubsection{Spectrum}\label{sec:spectrum}



% ====================================================================
\section{Convex Analysis}\label{sec:convex_analysis}
% ====================================================================

Convex Optimization (\S\ref{sec:convex_optimization})



% ====================================================================
\section{Algebraic Analysis}\label{sec:algebraic_analysis}
% ====================================================================

% --------------------------------------------------------------------
\subsection{Generalized Function}\label{sec:generalized_function}
% --------------------------------------------------------------------

\subsubsection{Distribution}\label{sec:distribution}

Continuous Linear Functional (\S\ref{sec:linear_form})



% ====================================================================
\section{Numerical Analysis}\label{sec:numerical_analysis}
% ====================================================================

% --------------------------------------------------------------------
\subsection{Interval Arithmetic}\label{sec:interval_arithmetic}
% --------------------------------------------------------------------

\subsubsection{Unit Interval}\label{sec:unit_interval}

$[0,1]$, sometimes $I$

a Complete Metric Space (\S\ref{sec:complete_metric_space}),
Homeomorphic (\S\ref{sec:homeomorphism}) to the Extended Real Number
Line (\S\ref{sec:extended_real_line})



% --------------------------------------------------------------------
\subsection{Numerical Integration}\label{sec:numerical_integration}
% --------------------------------------------------------------------

methods for obtaining Numerical approximations to the solutions of
Time-dependent Ordinary (\S\ref{sec:differential_equation}) and Partial
(\S\ref{sec:pde}) Differential Equations

\emph{Explicit Methods} -- calculates the state of a system at a later time
from the state of the system at the current time:
\[
  Y(t+\Delta{t}) = F(Y(t))
\]
where $Y(t)$ is the current system state and $Y(t + \Delta{t})$ is the system
state after a (small) time step $\Delta{t}$

\emph{Implicit Methods} -- finds a solution by solving an equation involving
the current system state and a later system state:
\[
  G(Y(t), Y(t + \Delta{t})) = 0
\]

First-order Methods

Second-order Methods

Higher-order Methods



\subsubsection{Geometric Integrator}\label{sec:geometric_integrator}

a Numerical Integration Method that preserves Geometric properties of the exact
\emph{Flow} (\S\ref{sec:integral_curve}) of a Differential Equation



\paragraph{Symplectic Integrator}\label{sec:symplectic_integrator}\hfill

a Numerical Integration scheme for Hamiltonian Systems
(\S\ref{sec:hamiltonian_system})

forms the Subclass of Geometric Integrators that are by definition
\emph{Canonical Transformations}: a change of Canonical Coordinates that
preserves the form of Hamilton's Equations (\S\ref{sec:hamiltonian_system})
%FIXME xref



% --------------------------------------------------------------------
\subsection{Spline Function}\label{sec:spline}
% --------------------------------------------------------------------

\subsubsection{Spline Interpolation}\label{sec:spline_interpolation}

cf. Polynomial Interpolation (???) %FIXME



\subsubsection{Cubic Spline}\label{sec:cubic_spline}

\paragraph{B-spline}\label{sec:b_spline}\hfill



% ====================================================================
\section{Ordinal Analysis}\label{sec:ordinal_analysis}
% ====================================================================

Ordinal Numbers (\S\ref{sec:ordinal_number})

Proof-theoretic Ordinal (\S\ref{sec:proof_ordinal})



% ====================================================================
\section{Non-standard Analysis}\label{sec:nonstandard_analysis}
% ====================================================================

Hyperreals (\S\ref{sec:hyperreal})

Real Closed Field (\S\ref{sec:real_closed})

$\reals^\nats / \mathsf{M}$ where $\mathsf{M}$ is a Maximal Ideal
(\S\ref{sec:maximal_ideal}) not leading to a Field that is
Order-isomorphic (\S\ref{sec:order_isomorphism}) to $\reals$-- the
uniqueness of this Field is equivalent to the Continuum Hypothesis
(\S\ref{sec:continuum_hypothesis})



% ====================================================================
\section{Calculus of Variations}\label{sec:calculus_of_variations}
% ====================================================================

\fist Optimal Control Theory (\S\ref{sec:optimal_control})



% --------------------------------------------------------------------
\subsection{Functional}\label{sec:functional}
% --------------------------------------------------------------------

a mapping from a Space $X$ into the Real Numbers or sometimes into the Complex
Numbers for the purpose of establishing a Calculus-like structure on $X$

\fist cf. Linear Functional (Linear Form \S\ref{sec:linear_form})

\fist Functional Equations (\S\ref{sec:functional_equation})



% ====================================================================
\section{Constructive Analysis}\label{sec:constructive_analysis}
% ====================================================================

\emph{Choice Sequence}

% ====================================================================
\section{Computable Analysis}\label{sec:computable_analysis}
% ====================================================================

% ====================================================================
\section{Algorithm Analysis}\label{sec:algorithm_analysis}
% ====================================================================

% --------------------------------------------------------------------
\subsection{Linear Dominance}\label{sec:linear_dominance}
% --------------------------------------------------------------------

\[
    f \lesssim g \Leftrightarrow
    \exists x_0 \exists c : \forall x > x_0, |f(x)| \leq c |g(x)|
\]

Pointwise Domination implies Linear Dominance:
\[
    f \leq g \Rightarrow f \lesssim g
\]

\[
    f \lesssim g \wedge g \lesssim f \Leftrightarrow f \sim g
\]



\subsubsection{Big-O Notation}\label{sec:bigo_notation}

\[
    O(f) = \{ g : g \lesssim f \}
\]

\[
    \Omega(f) = \{ g : g \gtrsim f \}
\]

\[
    \Theta(f) = \{ g : g \sim f \}
\]

\[
    O(1) \subset O(x) \subset O(x^2) \subset O(x^2) \ldots
\]



% ====================================================================
\section{Spatial Analysis}\label{sec:spatial_analysis}
% ====================================================================

% FIXME

% --------------------------------------------------------------------
\subsection{Boundary Problem}\label{sec:boundary_problem}
% --------------------------------------------------------------------

% FIXME
