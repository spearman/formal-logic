%%%%%%%%%%%%%%%%%%%%%%%%%%%%%%%%%%%%%%%%%%%%%%%%%%%%%%%%%%%%%%%%%%%%%%
%%%%%%%%%%%%%%%%%%%%%%%%%%%%%%%%%%%%%%%%%%%%%%%%%%%%%%%%%%%%%%%%%%%%%%
\part{Computation Theory}\cite{czoo14}\label{sec:recursion_theory}
%%%%%%%%%%%%%%%%%%%%%%%%%%%%%%%%%%%%%%%%%%%%%%%%%%%%%%%%%%%%%%%%%%%%%%
%%%%%%%%%%%%%%%%%%%%%%%%%%%%%%%%%%%%%%%%%%%%%%%%%%%%%%%%%%%%%%%%%%%%%%

also \emph{Recursion Theory}

\emph{Church-Turing}



% ====================================================================
\section{Decision Problem}\label{sec:decision_problem}
% ====================================================================



% ====================================================================
\section{Effective Calculability}\label{sec:effective_method}
% ====================================================================

A \emph{Function} (\S\ref{sec:set_function}) or \emph{Decision
Problem} (\S\ref{sec:decision_problem}) is \emph{Effectively
Calculable} if there exists an \emph{Effective Method} for solving
that problem.

An \emph{Effective Method} is a mechanical procedure for solving a
Function or Decision Problem of a specific \emph{Complexity Class}
(\S\ref{sec:complexity_theory}).

See \emph{Computable Function} (\S\ref{sec:computable_function}),
\emph{Computation Models} (\S\ref{sec:computation_model}) and
\emph{Algorithms} (\S\ref{subsec:algorithm}).



% ====================================================================
\section{Noncomputable Function}\label{sec:noncomputable_function}
% ====================================================================

also \emph{Undecidable Function}

\emph{Algorithmically Unsolvable}

\emph{Turing Degree}



% ====================================================================
\section{Computable Function}\label{sec:computable_function}
% ====================================================================

\emph{Computable Number} (\S\ref{subsec:computable_real})

\emph{Computable Functions} are the formalized analogues of
\emph{Algorithms} (\S\ref{subsec:algorithm}); that is a Function that
is \emph{Effectively Calculable} by an Effective Method
(\S\ref{sec:effective_method}).

\emph{Computable Functions} may be broadly divided into two Classes by
whether they are \emph{Partial} or \emph{Total}. A Total Computable
Function is sometimes called a (Total) \emph{Recursive} or
\emph{Decidable Function}.

There are several equivalent definitions of the class of Computable
Functions, which have reference to different \emph{Models of
  Computation} (\S\ref{sec:computation_model}):
\begin{enumerate}
\item Turing-computable Functions
\item $\mu$-recursive Functions
\item $\lambda$-recursive Functions
\end{enumerate}

% --------------------------------------------------------------------
\subsection{Partial Computable Function}\label{subsec:partial_recursive}
% --------------------------------------------------------------------

A \emph{Partial Computable Function} (\emph{Partial Recursive
  Function}) is also called \emph{Semidecidable} or \emph{Recursively
  Enumerable} and are the Class of Functions capable of generating
\emph{Recursively Enumerable Sets}
(\S\ref{subsec:recursively_enumerable}). A Set, $T$, is
\emph{Co-recursively Enumerable} if the Compliment $\mathbb{N} \ T$ is
Recursively Enumerable.

\emph{Recursively Enumerable} $\mathsf{coRE}$

The Lattice (\S\ref{subsec:lattice}) of $\mathsf{RE}$ Sets under
Inclusion is denoted $\mathcal{E}$.

% --------------------------------------------------------------------
\subsection{Total Computable Function}\label{subsec:recursive_function}
% --------------------------------------------------------------------

\emph{Total Recursive Function}

% --------------------------------------------------------------------
\subsubsection{Primitive Recursion}\label{subsec:primitive_recursion}
% --------------------------------------------------------------------

The Class of \emph{Primitive Recursive Functions}, $\mathsf{PR}$, are
those given by three Axioms:
\begin{enumerate}
    \item The $0$-ary \emph{Constant Function}, $0$, is Primitive
      Recursive
    \item The $1$-ary \emph{Successor Function}, $S(k) = k + 1$, is
      Primitive Recursive
    \item The $n$-ary \emph{Projection Function}, $P_i^n = k_i$,
      (returning the $i$th argument) is Primitive Recursive
\end{enumerate}
and the Operations:
\begin{enumerate}
    \item \emph{Composition}: Given a $k$-ary function, $f$, and $k$
      $m$-ary Primitive Recursive Functions, $g_1, \ldots, g_k$,
    \item \emph{Primitive Recursion}:
\end{enumerate}



% ====================================================================
\section{Models of Computation}\label{sec:computation_model}
% ====================================================================

% --------------------------------------------------------------------
\subsection{$\mu$-recursive Function}\label{subsec:mu_recursive}
% --------------------------------------------------------------------

also \emph{Partially Recursive Functions}

% --------------------------------------------------------------------
\subsection{Turing Machine}
% --------------------------------------------------------------------

See Part \ref{sec:automata_theory} Automata Theory
\S\ref{subsec:turing_machine} Turing Machines

\subsubsection{Post-Turing Machine}\label{subsec:post_turing}



% --------------------------------------------------------------------
\subsection{$\lambda$-Calculus}\label{subsec:lambda_calculus}
% --------------------------------------------------------------------

\emph{$\lambda$-Calculus} uses three constructions:

\begin{enumerate}
    \item \emph{Variable}: $x$
    \item \emph{Abstraction}: $(\lambda x.N)$
    \item \emph{Application}: $(L M)$
\end{enumerate}



% --------------------------------------------------------------------
\subsection{Tag System}\label{subsec:tag_system}
% --------------------------------------------------------------------

also \emph{Post Tag Machine}

% --------------------------------------------------------------------
\subsection{Register Machine}\label{subsec:register_machine}
% --------------------------------------------------------------------

\subsubsection{Counter Machine}

\subsubsection{Pointer Machine}

\subsubsection{Random Access Machine}

\subsubsection{Random Access Stored Program Machine}



% ====================================================================
\section{Complexity Theory}\label{sec:complexity_theory}
% ====================================================================

% --------------------------------------------------------------------
\subsection{Algorithm}\label{subsec:algorithm}
% --------------------------------------------------------------------

An \emph{Algorithm} may be formalized as a sequence of operations that
can be simulated by a Turing-complete system and any function that is
computable by Algorithm is a Computable Function
(\S\ref{sec:computable_function}).

\subsubsection{Algorithmic Randomness}\label{subsec:algorithmic_randomness}

% --------------------------------------------------------------------
\subsection{Function Problem}\label{subsec:function_problem}
% --------------------------------------------------------------------

% --------------------------------------------------------------------
\subsection{Communication Complexity}\label{subsec:communication_complexity}
% --------------------------------------------------------------------

% --------------------------------------------------------------------
\subsection{Complexity Class}\label{subsec:complexity_class}
% --------------------------------------------------------------------

\[
    \mathsf{PR} \subset \mathsf{R} = \mathsf{RE} \cap \mathsf{coRE}
\]



% --------------------------------------------------------------------
\subsection{Hierarchies}\label{subsec:complexity_hierarchy}
% --------------------------------------------------------------------

\[
    \mathsf{AC^0} \subseteq \mathsf{L} \subseteq \mathsf{NC} \subseteq
    \mathsf{P} \subseteq \mathsf{PSPACE} \subseteq \mathsf{EXP}
    \subseteq \mathsf{ALL}
\]

\[
    \Delta_0 = \Sigma_0 = \Pi_0 = \mathsf{R}
\]\[
    \Sigma_1^0 \leftrightarrow \mathsf{RE}
\]\[
    \Pi_1^0 \leftrightarrow \mathsf{coRE}
\]

\emph{$\mathsf{NC}$} is the Class of problems decidable in
Polyalgorithmic Time on a Parallel Computer with a Polynomial number
of Processors.

\emph{Lightface} (no \emph{Set Parameters}) %FIXME explain ``set parameters''
\[
    \Sigma, \Pi, \Delta
\]

\emph{Boldface} (standard hierarchy of Borel Sets)
\[
    \mathbf{\Sigma}, \mathbf{\Pi}, \mathbf{\Delta}
\]

Formulas are considered in \emph{Prenex Normal Form}
(\S\ref{subsec:prenex_normal}).
\[
    \Sigma^k_n \cap \Pi^k_n = \Delta^k_n
\]
\begin{itemize}
    \item $k$ is the type of objects being Quantified over with type
      $0 = \mathbb{N}$, and type $i + 1$ are Functions from type $i
      \rightarrow \mathbb{N}$.
    \item $n$ is the number of alternating blocks of Existential and
      Universal Quantifiers
\end{itemize}
The outermost Quantifier block for $\Sigma$ Classes is Existential,
and for $\Pi$ Classes is Universal.



\subsubsection{Arithmetic Hierarchy}\label{subsec:arithmetic_hierarchy}

The \emph{Arithmetic Hierarchy} Classifies the Complexity of Formulas
in \emph{First-order Arithmetic}
(\S\ref{subsec:firstorder_arithmetic}). The Arithmetic Hierarchy is
extended by the \emph{Hyperarithmetic} and \emph{Analytic
  Hierarchies}.

Every Formula in the Language of First-order Arithmetic is in
$\Sigma^0_n$ or $\Pi^0_n$ for $n \in \mathbb{N}$. Formulas with only
\emph{Bounded Quantifiers} (\S\ref{subsec:firstorder_quantification})
are in $\Sigma^0_0$ or $\Pi^0_0$. $\Sigma^0_n$ and $\Pi^0_n$ are
defined Inductively for $n \in \mathbb{N}$:

\begin{itemize}
    \item $\phi = \exists x_1 \exists x_2 \cdots \exists x_k\psi, \psi
      \in \Pi^0_n \rightarrow \phi \in \Sigma^0_{n+1}$
    \item $\phi = \forall x_1 \forall x_2 \cdots \forall x_k\psi, \psi
      \in \Sigma^0_n \rightarrow \phi \in \Pi^0_{n+1}$
\end{itemize}

Thus a $\Sigma^0_n$ Formula begins with Existential Quantifiers and
alternates $n-1$ times between series of Existential and Universal
Quantifiers. The dual is true for $\Pi^0_n$.

By the addition of redundant Quantifiers, a Formula of $\Sigma^0_n$ or
$\Pi^0_n$ is also $\Sigma^0_m$ and $\Pi^0_m$ for $m > n$.

Sets of Natural Numbers that are definable in First-order Arithmetic
are assigned to Classes $\Sigma^0_n$ or $\Pi^0_n$, and to $\Delta^0_n$
if they are both $\Sigma^0_n$ and $\Pi^0_n$.

Turing Computable Sets are at level $\Delta^0_1$.

The \emph{Hyperarithmetic Sets} (\S\ref{subsec:hyperarithmetic_set})
are defined as $\Delta^1_1$.

\paragraph{Relativized Arithmetical Hierarchies}



\subsubsection{Analytic Hierarchy}\label{subsec:analytic_hierarchy}

The \emph{Analytic Hierarchy} extends the Arithmetic Hierarchy to
Second-order Formulas. The Classification is of \emph{Analytical Sets}
(\S\ref{subsec:analytical_set} of Natural Numbers by the difficulty of
specifying in Second-order Arithmetic (\S\ref{subsec:second_order}),
that is Second-order Formulas which include Quantifiers over
$\mathbb{N}$ and Functions from $\mathbb{N}$ to $\mathbb{N}$.

Every Formula in the Language of Second-order Arithmetic is in
$\Sigma^1_n$ or $\Pi^1_n$ for $n \in \mathbb{N}$.

Second-order Arithmetic with no Function Quantifiers:
\[
    \Delta^1_0 = \Sigma^1_0 = \Pi^1_0
\]

Second-order Arithmetic with Existential Function Quantifiers only:
\[
    \Sigma^1_1
\]

Second-order Arithmetic with Universal Function Quantifiers only:
\[
    \Pi^1_1
\]
Both $\Sigma^1_1$ and $\Pi^1_1$ are strictly larger than $\Delta^1_0$.

Second-order Arithmetic with both Existential and Universal Function
Quantifiers-- the class of \emph{Hyperarithmetic Sets}:
\[
    \Delta^1_1 = \Sigma^1_1 \cap \Pi^1_1
\]
is larger than $\Delta^1_0$.

$\Pi^1_1$ Formulas quantified over by Existential Functional
Quantifiers results in the Class $\Sigma^1_2$.

$\Sigma^1_1$ Formulas quantified over by Universal Functional
Quantifiers results in the Class $\Pi^1_2$.

Again, $\Delta^1_2$ is the Intersection of $\Sigma^1_2$ and $\Pi^1_2$
and Inductively this procedure extends to $\Sigma^1_n$, $\Pi^1_n$, and
$\Delta^1_n$.



\subsubsection{Projective Hierarchy}\label{subsec:projective_hierarchy}

\emph{Pointclass} (\S\ref{subsec:pointclass})

Pointclass of all \emph{Analytic Sets} (\S\ref{subsec:analytic_set})
\[
    \mathbf{\Sigma^1_1}
\]

\paragraph{Borel Hierarchy}



\subsubsection{Counting Hierarchy}

\subsubsection{Exponential Hierarchy}

\subsubsection{Fourier Hierarchy}

\subsubsection{Polynomial Hierarchy}

\subsubsection{Polynomial Communication Hierarchy}

\subsubsection{Boolean Hierarchy}




\subsubsection{Blum Complexity Measure}

\emph{Blum Axioms}



% ====================================================================
\section{Hypercomputation}\label{sec:hypercomputation}
% ====================================================================

% --------------------------------------------------------------------
\subsection{Oracle}\label{sec:oracle}
% --------------------------------------------------------------------

% --------------------------------------------------------------------
\subsection{$\alpha$-recursion Theory}\label{subsec:alpha_recursion}
% --------------------------------------------------------------------

% --------------------------------------------------------------------
\subsection{Hyperarithmetical Theory}\label{subsec:hyperarithmetical_theory}
% --------------------------------------------------------------------


