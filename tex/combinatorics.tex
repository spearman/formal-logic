%%%%%%%%%%%%%%%%%%%%%%%%%%%%%%%%%%%%%%%%%%%%%%%%%%%%%%%%%%%%%%%%%%%%%%
%%%%%%%%%%%%%%%%%%%%%%%%%%%%%%%%%%%%%%%%%%%%%%%%%%%%%%%%%%%%%%%%%%%%%%
\part{Combinatorics}\label{sec:combinatorics}
%%%%%%%%%%%%%%%%%%%%%%%%%%%%%%%%%%%%%%%%%%%%%%%%%%%%%%%%%%%%%%%%%%%%%%
%%%%%%%%%%%%%%%%%%%%%%%%%%%%%%%%%%%%%%%%%%%%%%%%%%%%%%%%%%%%%%%%%%%%%%

% ====================================================================
\section{Permutation}\label{sec:permutation}
% ====================================================================

% --------------------------------------------------------------------
\subsection{Cyclic Permutation}\label{sec:cyclic_permutation}
% --------------------------------------------------------------------

A Permutation is \emph{Cyclic} if and only if it consists of a single
Non-trivial Cycle (Cycle Length > 1).



\subsubsection{Transposition}\label{sec:transposition}

A \emph{Transposition} is a Cyclic Permutation of length 2 which
exchanges two Elements and leaves all other Elements fixed.



% ====================================================================
\section{Matroid}\label{sec:matroid}
% ====================================================================

% --------------------------------------------------------------------
\subsection{Antimatroid}\label{sec:antimatroid}
% --------------------------------------------------------------------

% --------------------------------------------------------------------
\subsection{Greedoid}\label{sec:greedoid}
% --------------------------------------------------------------------



% ====================================================================
\section{Ramsey's Theorem}\label{sec:ramseys_theorem}
% ====================================================================

% ====================================================================
\section{Combinatory Logic}\label{sec:combinatory_logic}
% ====================================================================

cf. Predicate Functor Logic (\S\ref{sec:pfl})

Set Theory (Part \S\ref{sec:set_theory})

Combinatory Logic may be given as a variant of $\lambda$-calculus
(\S\ref{sec:untyped_lambda}) where Lambda Expressions are replaced by
\emph{Combinators} (\S\ref{sec:combinator}) which are Primitive
Functions witout Bound Variables ($\lambda$-calculus without
Abstraction).

SKI Combinator Calculus (\S\ref{sec:ski_calculus})

$\lambda$-terms can be transformed into Combinatory Logic by
Abstraction Elimination (\S\ref{sec:abstraction_elimination})

It is Undecidable whether a given Combinatory Term has a Normal Form
(or whether two Combinatory Terms are Equivalent).

By the Curry-Howard Correspondence (\S\ref{sec:curry_howard}) a Typed
Combinatory Logic corresponds to a Hilbert System
(\S\ref{sec:hilbert_system}) in Proof Theory.



% --------------------------------------------------------------------
\subsection{Combinator}\label{sec:combinator}
% --------------------------------------------------------------------

\emph{Combinator}

Closed Primitive Function

Identity:
\[
  I x = x
\]
Constant:
\[
  K x y = x
\]
Application (Sequence):
\[
  S x y z = x z (y z)
\]

Identity may be defined in Terms of $S$ and $K$:
\[
  (S K K) x = I x
\]



\subsubsection{Fixed-point Combinator}\label{sec:fixedpoint_combinator}

\paragraph{$Y$-combinator}\label{sec:y_combinator}

In Simply-typed $\lambda$-calculus, the $Y$-combinator cannot be
assigned a Type.

Allows for \emph{Anonymous Recursion} in Programming Languages.



\subsubsection{Non-standard Fixed-point Combinator}
\label{sec:nonstandard_combinator}

A \emph{Non-standard Fixed-point Combinator} is a Combinator that has
the same Infinite Extension (B\"ohm Tree \S\ref{sec:bohm_tree}) as a
Fixed-point Combinator. Every Fixed-point Combinator is a Non-standard
Fixed-point Combinator, but not every Non-standard Fixed-point
Combinator is a Fixed-point Combinator. A Non-standard Fixed-point
Combinator that is not a Fixed-point Combinator is called a
\emph{Strictly Non-standard Fixed-point Combinator}.

The Set of Non-standard Fixed-point Combinators is not Recursively
Enumerable. \cite{goldberg05}



% --------------------------------------------------------------------
\subsection{Combinatory Term}\label{sec:combinatory_term}
% --------------------------------------------------------------------

\begin{itemize}
  \item a Variable $x$
  \item a Combinator $P$
  \item Application of Combinatory Terms $E_1 E_2$
\end{itemize}



\subsubsection{Abstraction Elimination}
\label{sec:abstraction_elimination}

\subsubsection{Bracket Abstraction}\label{sec:bracket_abstraction}
