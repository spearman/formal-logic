%%%%%%%%%%%%%%%%%%%%%%%%%%%%%%%%%%%%%%%%%%%%%%%%%%%%%%%%%%%%%%%%%%%%%%
%%%%%%%%%%%%%%%%%%%%%%%%%%%%%%%%%%%%%%%%%%%%%%%%%%%%%%%%%%%%%%%%%%%%%%
\part{Combinatorics}\label{part:combinatorics}
%%%%%%%%%%%%%%%%%%%%%%%%%%%%%%%%%%%%%%%%%%%%%%%%%%%%%%%%%%%%%%%%%%%%%%
%%%%%%%%%%%%%%%%%%%%%%%%%%%%%%%%%%%%%%%%%%%%%%%%%%%%%%%%%%%%%%%%%%%%%%

Discrete, Countable (\S\ref{sec:countably_infinite})

Abstract Structure (\S\ref{sec:abstract_structure}): Recognizable
Structure vs. Combinatoric description (requiring special arguments)

Formal Language Theory (Part \ref{part:formal_language}):
Combinatorics on Words (\S\ref{sec:combinatorics_on_words})

Set Theory (Part \ref{part:set_theory}): Infinitary Combinitorics
(\S\ref{sec:infinitary_combinatorics})



% ====================================================================
\section{Combination}\label{sec:combination}
% ====================================================================

Unordered choice of Elements

$k$-combination of $n$ Elements is given by the Binomial Coefficient
(\S\ref{sec:binomial_coefficient}) $\binom{n}{k}$, $n$ ``choose'' $k$,
equal to:
\[
  \frac{n!}{k!(n - k)!}
\]



% --------------------------------------------------------------------
\subsection{Binomial Coefficient}\label{sec:binomial_coefficient}
% --------------------------------------------------------------------

Binomial (\S\ref{sec:binomial})

$\binom{n}{k} = \frac{n!}{k!(n - k)!}$



% ====================================================================
\section{Permutation}\label{sec:permutation}
% ====================================================================

Ordered choice of Elements

The number of Permutations of $n$ Elements is $n!$

The number of Cyclic Permutations (\S\ref{sec:cyclic_permutation}) of
$n$ Elements is $(n-1)!$

The number of $k$ choices out of $n$ Elements when order is
significant is $\frac{n!}{(n-1)!}$

Permutation Group (\S\ref{sec:permutation_group})



% --------------------------------------------------------------------
\subsection{Cyclic Permutation}\label{sec:cyclic_permutation}
% --------------------------------------------------------------------

A Permutation is \emph{Cyclic} if and only if it consists of a single
Non-trivial Cycle (Cycle Length > 1).



\subsubsection{Transposition}\label{sec:transposition}

A \emph{Transposition} is a Cyclic Permutation of length 2 which
exchanges two Elements and leaves all other Elements fixed.



% ====================================================================
\section{Partition}\label{sec:partition}
% ====================================================================

Set Partitions (\S\ref{sec:set_partition})

$\binom{n}{n_1 n_2 \ldots n_k}$



% --------------------------------------------------------------------
\subsection{Bell Number}\label{sec:bell_number}
% --------------------------------------------------------------------

Bell Number

$B_{n+1} = \sum_{k=0}^n \binom{n}{k} B_k$



% ====================================================================
\section{Enumerative Combinatorics}
\label{sec:enumerative_combinatorics}
% ====================================================================

Counting Structures



% ====================================================================
\section{Analytic Combinatorics}\label{sec:analytic_combinatorics}
\cite{flajolet-sedgewick09}
% ====================================================================

% --------------------------------------------------------------------
\subsection{Combinatorial Species}\label{sec:combinatorial_species}
% --------------------------------------------------------------------



% ====================================================================
\section{Design Theory}\label{sec:combinatorial_design}
% ====================================================================

% ====================================================================
\section{Algebraic Combinatorics}\label{sec:algebraic_combinatorics}
% ====================================================================

% --------------------------------------------------------------------
\subsection{Matroid Theory}\label{sec:matroid_theory}
% --------------------------------------------------------------------

\subsubsection{Matroid}\label{sec:matroid}

\subsubsection{Antimatroid}\label{sec:antimatroid}

\subsubsection{Greedoid}\label{sec:greedoid}



% ====================================================================
\section{Extremal Combinatorics}\label{sec:extremal_combinatorics}
% ====================================================================

Combinatorial Optimization (\S\ref{sec:combinatorial_optimization})



% --------------------------------------------------------------------
\subsection{Ramsey Theory}\label{sec:ramsey_theory}
% --------------------------------------------------------------------

\subsubsection{Ramsey's Theorem}\label{sec:ramseys_theorem}



% ====================================================================
\section{Probabilistic Combinatorics}\label{sec:probabilistic_combinatorics}
% ====================================================================

Probability Theory (Part \S\ref{part:probability_theory})



% --------------------------------------------------------------------
\subsection{Probabilistic Method}\label{sec:probabilistic_method}
% --------------------------------------------------------------------



% ====================================================================
\section{Infinitary Combinatorics}\label{sec:infinitary_combinatorics}
% ====================================================================

Set Theory (Part \ref{part:set_theory})

\emph{Combinatorial Set Theory}



% ====================================================================
\section{Combinatorics on Words}\label{sec:combinatorics_on_words}
% ====================================================================

Formal Language Theory (Part \ref{part:formal_language})

Chomasky Hierarchy (\S\ref{sec:chomsky_hierarchy})



% --------------------------------------------------------------------
\subsection{Word Type}\label{sec:word_type}
% --------------------------------------------------------------------

\subsubsection{Sturmian Word}\label{sec:sturmian_word}

\subsubsection{Lyndon Word}\label{sec:lyndon_word}



% ====================================================================
\section{Combinatory Logic}\label{sec:combinatory_logic}
% ====================================================================

Higher-order Functions (\S\ref{sec:higherorder_function})

\fist cf. Predicate Functor Logic (\S\ref{sec:pfl})

Set Theory (Part \S\ref{part:set_theory})

Combinatory Logic may be given as a variant of $\lambda$-calculus
(\S\ref{sec:untyped_lambda}) where Lambda Expressions are replaced by
\emph{Combinators} (\S\ref{sec:combinator}) which are Primitive
Functions witout Bound Variables ($\lambda$-calculus without
Abstraction).

SKI Combinator Calculus (\S\ref{sec:ski_calculus})

$\lambda$-terms can be transformed into Combinatory Logic by
Abstraction Elimination (\S\ref{sec:abstraction_elimination})

It is Undecidable whether a given Combinatory Term has a Normal Form
(or whether two Combinatory Terms are Equivalent).

By the Curry-Howard Correspondence (\S\ref{sec:curry_howard}) a Typed
Combinatory Logic corresponds to a Hilbert System
(\S\ref{sec:hilbert_system}) in Proof Theory.



% --------------------------------------------------------------------
\subsection{Combinator}\label{sec:combinator}
% --------------------------------------------------------------------

\emph{Combinator} (or \emph{Constant})

Closed Primitive Function

Identity:
\[
  I x = x
\]
Constant:
\[
  K x y = x
\]
Application (Sequence):
\[
  S x y z = x z (y z)
\]

Identity may be defined in Terms of $S$ and $K$:
\[
  (S K K) x = I x
\]

$\Omega$-combinator: $(\lambda x. x x)$



\subsubsection{Fixed-point Combinator}\label{sec:fixedpoint_combinator}

\[
  y f = f (y f)
\]
represents a Solution to the Fixed Point Equation
(\S\ref{sec:fixed_point}):
\[
  x = f x
\]



\paragraph{$Y$-combinator}\label{sec:y_combinator}\hfill

In Simply-typed $\lambda$-calculus, the $Y$-combinator cannot be
assigned a Type.

\[
  Y = \lambda f.(\lambda x.f (x x)) (\lambda x.f (x x))
\]

Allows for \emph{Anonymous Recursion} in Programming Languages.
%FIXME xref?



\subsubsection{Non-standard Fixed-point Combinator}
\label{sec:nonstandard_combinator}

A \emph{Non-standard Fixed-point Combinator} is a Combinator that has
the same Infinite Extension (B\"ohm Tree \S\ref{sec:bohm_tree}) as a
Fixed-point Combinator. Every Fixed-point Combinator is a Non-standard
Fixed-point Combinator, but not every Non-standard Fixed-point
Combinator is a Fixed-point Combinator. A Non-standard Fixed-point
Combinator that is not a Fixed-point Combinator is called a
\emph{Strictly Non-standard Fixed-point Combinator}.

The Set of Non-standard Fixed-point Combinators is not Recursively
Enumerable. \cite{goldberg05}



% --------------------------------------------------------------------
\subsection{Combinatory Term}\label{sec:combinatory_term}
% --------------------------------------------------------------------

Curry \emph{Theory of Functionality}

\begin{itemize}
  \item a Variable $x$
  \item a Combinator (Constant) $P$
  \item Application of Combinatory Terms $E_1 E_2$
\end{itemize}
Variables and Constants are \emph{Atoms}.



\subsubsection{Contraction}\label{sec:combinatory_contraction}
\cite{seldin03}

Basic Combinators:
\begin{enumerate}
  \item $\mathsf{I} X = X$
  \item $\mathsf{K} X Y = X$
  \item $\mathsf{S} X Y Z = (X Z) (Y Z)$
\end{enumerate}

Weak Reduction, a Sequence of Zero or more Contractions: $M \rhd N$

Weak Conversion, a Sequence of Zero or more Contractions and reverse
Contractions: $M =_* N$



\subsubsection{Bracket Abstraction}\label{sec:bracket_abstraction}

\emph{Abstraction} of Term $M$ with respect to Variable $x$ denoted
$[x]M$, defined Inductively:\cite{seldin03}
\[
  [x]x \equiv \mathsf{I}
\]\[
  [x]c \equiv \mathsf{K}c
\]\[
  [x](M N) \equiv \mathsf{S}([x]M)([x]N)
\]
where $c$ is an Atom other than $x$ and $\equiv$ is an Equivalence of
Terms.

Theorem:
\[
  ([x]M)N \rhd [N/x]M
\]



\subsubsection{Abstraction Elimination}
\label{sec:abstraction_elimination}



\subsubsection{Type Assignment}\label{sec:combinatory_type}
\cite{seldin03}

\begin{itemize}
  \item ($\rightarrow \mathsf{I}$) $\mathsf{I}: A \rightarrow A$
  \item ($\rightarrow \mathsf{K}$)
    $\mathsf{K}: A \rightarrow (B \rightarrow A)$
  \item ($\rightarrow \mathsf{S}$)
    $\mathsf{S}: (A \rightarrow (B \rightarrow C))
      \rightarrow (A \rightarrow B) \rightarrow A \rightarrow C$
\end{itemize}

Derived Rule (cf. Modus Ponens):
\[
  \frac{
    [x:A] \quad m:B
  }{
    [x]m:A \rightarrow B
  }
\]
