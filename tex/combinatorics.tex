%%%%%%%%%%%%%%%%%%%%%%%%%%%%%%%%%%%%%%%%%%%%%%%%%%%%%%%%%%%%%%%%%%%%%%%%%%%%%%%%
%%%%%%%%%%%%%%%%%%%%%%%%%%%%%%%%%%%%%%%%%%%%%%%%%%%%%%%%%%%%%%%%%%%%%%%%%%%%%%%%
\part{Combinatorics}\label{part:combinatorics}
%%%%%%%%%%%%%%%%%%%%%%%%%%%%%%%%%%%%%%%%%%%%%%%%%%%%%%%%%%%%%%%%%%%%%%%%%%%%%%%%
%%%%%%%%%%%%%%%%%%%%%%%%%%%%%%%%%%%%%%%%%%%%%%%%%%%%%%%%%%%%%%%%%%%%%%%%%%%%%%%%

Taking \emph{Combination} (\S\ref{sec:combination}) as a Primitive Notion, one
may divide Mathematics into two branches:
\begin{itemize}
  \item \emph{Finitary Combinatorics} (\emph{Combinatorics on Words}
    \S\ref{sec:combinatorics_on_words}) -- Formal Language Theory (Part
    \ref{part:formal_language})
  \item \emph{Infinitary Combinatorics} (\S\ref{sec:infinitary_combinatorics}):
    Set Theory (Part \ref{part:set_theory})
\end{itemize}

\emph{Combinatorics as ``De-structured'' Mathematics} -- ``recognizable''
\emph{Structure}'' vs. ``\emph{Combinatoric} description'' (requiring ``special
arguments''):
\begin{itemize}
  \item Structure (Model Theory \S\ref{sec:abstract_structure})
  \item Algebraic Structure (Abstract Algebra \S\ref{sec:abstract_structure})
  \item Data Structure (Database Theory \S\ref{sec:data_structure})
  \item Abstract Structure (Appendix \S\ref{sec:abstract_structure})
\end{itemize}

central subfield of \emph{Enumerative Combinatorics}
(\S\ref{sec:enumerative_combinatorics}): computing the Cardinality
(\S\ref{sec:cardinality}) of Elements of \emph{Finite Sets}

subfield of \emph{Discrete Mathematics}, Countable
(\S\ref{sec:countably_infinite})

see also:
\begin{itemize}
  \item Discrete Geometry (Combinatorial Geometry \S\ref{sec:discrete_geometry})
  \item Combinatorial Group Theory (\S\ref{sec:combinatorial_group_theory})
  \item Combinatorial Optimization (\S\ref{sec:combinatorial_optimization})
  \item Combinatorial Game Theory (\S\ref{sec:combinatorial_game_theory})
  \item ... MORE?
\end{itemize}

\fist Discrete Uniform Law (Probability Theory \S\ref{sec:discrete_uniform_law})
%FIXME: clarify



% ==============================================================================
\section{Combination}\label{sec:combination}
% ==============================================================================

Unordered choice of Elements

$k$-combination of $n$ Elements is given by the Binomial Coefficient
(\S\ref{sec:binomial_coefficient}) $\binom{n}{k}$, $n$ ``choose'' $k$, equal to:
\[
  \frac{n!}{k!(n - k)!}
\]



% ------------------------------------------------------------------------------
\subsection{Binomial Coefficient}\label{sec:binomial_coefficient}
% ------------------------------------------------------------------------------

Binomial (\S\ref{sec:binomial})

$\binom{n}{k} = \frac{n!}{k!(n - k)!}$

number of ways to Partition a Set of Cardinality $n$ into two Subsets of
Cardinality $k$ and $n-k$

counting Partitions (\S\ref{sec:partition}) of prescribed Cardinalities:

$\binom{n}{n_1 n_2 \cdots n_k}$



\subsubsection{Catalan Number}\label{sec:catalan_number}



% ==============================================================================
\section{Permutation}\label{sec:permutation}
% ==============================================================================

Ordered choice of Elements

The number of Permutations of $n$ Elements is $n!$

The number of Cyclic Permutations (\S\ref{sec:cyclic_permutation}) of $n$
Elements is $(n-1)!$

The number of $k$ choices out of $n$ Elements when order is significant is
$\frac{n!}{(n-1)!}$

Permutation Group (\S\ref{sec:permutation_group})



% ------------------------------------------------------------------------------
\subsection{Cyclic Permutation}\label{sec:cyclic_permutation}
% ------------------------------------------------------------------------------

A Permutation is \emph{Cyclic} if and only if it consists of a single
Non-trivial Cycle (Cycle Length > 1).



\subsubsection{Transposition}\label{sec:transposition}

A \emph{Transposition} is a Cyclic Permutation of length 2 which exchanges two
Elements and leaves all other Elements fixed.



% ------------------------------------------------------------------------------
\subsection{Derangement}\label{sec:derangement}
% ------------------------------------------------------------------------------

a Permutation such that no Element appears in its original position, i.e. a
Permutation with no \emph{Fixed Points}



% ------------------------------------------------------------------------------
\subsection{Superpermutation}\label{sec:superpermutation}
% ------------------------------------------------------------------------------

% ------------------------------------------------------------------------------
\subsection{Permutation Pattern}\label{sec:permutation_pattern}
% ------------------------------------------------------------------------------

% ------------------------------------------------------------------------------
\subsection{Permutation Class}\label{sec:permutation_class}
% ------------------------------------------------------------------------------

\subsubsection{Wilf Class}\label{sec:wilf_class}

a Combinatorial Class (\S\ref{sec:combinatorial_class}) of Permutation Classes
enumerated by Length



% ==============================================================================
\section{Partition}\label{sec:partition}
% ==============================================================================

Set Partitions (\S\ref{sec:set_partition})

Binomial Coefficient (\S\ref{sec:binomial_coefficient})

$\binom{n}{n_1 n_2 \cdots n_k}$



% ------------------------------------------------------------------------------
\subsection{Bell Number}\label{sec:bell_number}
% ------------------------------------------------------------------------------

Bell Number

$B_{n+1} = \sum_{k=0}^n \binom{n}{k} B_k$



% ==============================================================================
\section{Transversal}\label{sec:transversal}
% ==============================================================================

given a Family of Sets (Collection), $C$, a \emph{Transversal} (or
\emph{Cross-section}) is a Set containing exactly one Element from each Member
of the Collection

cf. Transversality (\S\ref{sec:transversality})



% ==============================================================================
\section{Enumerative Combinatorics}\label{sec:enumerative_combinatorics}
% ==============================================================================

Counting Structures



% ------------------------------------------------------------------------------
\subsection{Classification Problem}\label{sec:classification_problem}
% ------------------------------------------------------------------------------

(wiki):

\emph{Classification Problem}: what are the ``objects'' of a given ``type'' up
to some Equivalence?



\subsubsection{Classification Theorem}\label{sec:classification_theorem}

(wiki):

a \emph{Classification Theorm} answers a Classification Problem for ``objects''
of a given ``type'', giving a non-redundant Enumeration of Equivalence Classes

\begin{itemize}
  \item \emph{Equivalence Problem} (\S\ref{sec:equivalence_problem}): given two
    objects, determine if they are Equivalent
  \item \emph{Complete Set of Invariants} (\S\ref{sec:invariant}): a Set of
    Invariants such that Equivalence of the Invariants in the Set is a
    \emph{sufficient condition} for Equivalence of objects; together with
    ``which Invariants are realizable'' solves the Classification problem, and a
    \emph{Computable Complete Set of Invariants} together with ``which
    Invariants are realizable'' solves the Classification Problem and the
    Equivalence Problem (FIXME: clarify)
  \item \emph{Canonical Form} (\S\ref{sec:canonical_form}): solves the
    Classification Problem and provides a distinguished (``Canonical'') Element
    of each Class
\end{itemize}

cf. Exceptional Objects (\S\ref{sec:exceptional}) -- Classifications often
result in a number of \emph{(Infinite) Series of Objects}, and a \emph{Finite
  Number} of \emph{Exceptions} that do not fit into any Series, and such
\emph{Exceptional Objects} are often related to Exceptional Objects in other
branches of Mathematics

\begin{itemize}
  \item Classification of Euclidean Plane Isometries
  \item Classification of Two-dimensional Closed Manifolds
  \item Thurston's Geometrization Conjecture
  \item Classification of Finite Simple Groups
  \item ... MORE
\end{itemize}

TODO: xrefs



% ------------------------------------------------------------------------------
\subsection{The Twelvefold Way}\label{sec:twelvefold_way}
% ------------------------------------------------------------------------------

(wiki):

Systematic classification of 12 related Enumerative problems concerning two
Finite Sets, including classical problems of counting:
\begin{itemize}
  \item Permutations (\S\ref{sec:permutation})
  \item Combinations (\S\ref{sec:combination})
  \item Multisets (\S\ref{sec:multiset})
  \item Partitions of a Set (\S\ref{sec:partition})
  \item Partitions of an Integer (\S\ref{sec:integer_partition})
\end{itemize}

Cases:
\begin{enumerate}
  \item Functions from $N$ to $X$
  \item Functions from $N$ to $X$, up to a Permutation of $N$
  \item Functions from $N$ to $X$, up to a Permutation of $X$
  \item Functions from $N$ to $X$, up to Permutations of $N$ and $X$
  \item Injective Functions from $N$ to $X$
  \item Injective Functions from $N$ to $X$, up to a Permutation of $N$
  \item Injective Functions from $N$ to $X$, up to a Permutation of $X$
  \item Injective Functions from $N$ to $X$, up to Permutations of $N$ and $X$
  \item Surjective Functions from $N$ to $X$
  \item Surjective Functions from $N$ to $X$, up to a Permutation of $N$
  \item Surjective Functions from $N$ to $X$, up to a Permutation of $X$
  \item Surjective Functions from $N$ to $X$, up to Permutations of $N$ and $X$
\end{enumerate}



% ------------------------------------------------------------------------------
\subsection{Lattice Path}\label{sec:lattice_path}
% ------------------------------------------------------------------------------

\subsubsection{Dyck Path}\label{sec:dyck_path}

\subsubsection{Schr\"oder Path}\label{sec:schroder_path}

\emph{Schr\"oder Paths and Reverse Bessel Polynomials} --
\url{https://golem.ph.utexas.edu/category/2017/08/schrder_paths_and_reverse_bessel_polynomials.html}

Reverse Bessel Polynomial (\S\ref{sec:reverse_bessel_polynomial})



% ==============================================================================
\section{Analytic Combinatorics}\label{sec:analytic_combinatorics}
% ==============================================================================

\cite{flajolet-sedgewick09}

Asymptotic Analysis (\S\ref{sec:asymptotic_analysis})



% ------------------------------------------------------------------------------
\subsection{Combinatorial Species}\label{sec:combinatorial_species}
% ------------------------------------------------------------------------------

Presheaf (\S\ref{sec:presheaf})

Categorification of Product Rule (\S\ref{sec:product_rule}) %TODO



% ------------------------------------------------------------------------------
\subsection{Combinatorial Class}\label{sec:combinatorial_class}
% ------------------------------------------------------------------------------

a Wilf Class (\S\ref{sec:wilf_class}) of Permutation Classes
(\S\ref{sec:permutation_class}) enumerated by Length is a Combinatorial Class



\subsubsection{Counting Sequence}\label{sec:counting_sequence}



% ==============================================================================
\section{Design Theory}\label{sec:combinatorial_design}
% ==============================================================================

% ==============================================================================
\section{Algebraic Combinatorics}\label{sec:algebraic_combinatorics}
% ==============================================================================

% ------------------------------------------------------------------------------
\subsection{Polynomial Sequence}\label{sec:polynomial_sequence}
% ------------------------------------------------------------------------------

Polynomial (\S\ref{sec:polynomial}) Sequence (\S\ref{sec:sequence})

\fist Formal Power Series (\S\ref{sec:formal_power_series})



\subsubsection{Orthogonal Polynomial Sequence}
\label{sec:orthogonal_polynomial_sequence}

\paragraph{Bessel Polynomial}\label{sec:bessel_polynomial}\hfill

\subparagraph{Reverse Bessel Polynomial}\label{sec:reverse_bessel_polynomial}
\hfill

cf. Bessel Functions (\S\ref{sec:bessel_function})

\emph{Schr\"oder Paths and Reverse Bessel Polynomials} --
\url{https://golem.ph.utexas.edu/category/2017/08/schrder_paths_and_reverse_bessel_polynomials.html}

Schr\"oder Path (\S\ref{sec:schroder_path})



% ------------------------------------------------------------------------------
\subsection{Matroid Theory}\label{sec:matroid_theory}
% ------------------------------------------------------------------------------

\subsubsection{Independence System}\label{sec:independence_system}

an Abstract Simplicial Complex (\S\ref{sec:abstract_complex}) in the context of
Matroid Theory



\subsubsection{Matroid}\label{sec:matroid}

(wiki): generalizes the notion of \emph{Linear Independence} in Vector Spaces
(\S\ref{sec:vector_space})



\subsubsection{Antimatroid}\label{sec:antimatroid}

\subsubsection{Greedoid}\label{sec:greedoid}



% ==============================================================================
\section{Extremal Combinatorics}\label{sec:extremal_combinatorics}
% ==============================================================================

Combinatorial Optimization (\S\ref{sec:combinatorial_optimization})



% ------------------------------------------------------------------------------
\subsection{Ramsey Theory}\label{sec:ramsey_theory}
% ------------------------------------------------------------------------------

\subsubsection{Ramsey's Theorem}\label{sec:ramseys_theorem}



% ==============================================================================
\section{Probabilistic Combinatorics}\label{sec:probabilistic_combinatorics}
% ==============================================================================

Probability Theory (Part \S\ref{part:probability_theory})



% ------------------------------------------------------------------------------
\subsection{Probabilistic Method}\label{sec:probabilistic_method}
% ------------------------------------------------------------------------------



% ==============================================================================
\section{Combinatorics on Words}\label{sec:combinatorics_on_words}
% ==============================================================================

or \emph{Finitary Combinatorics}

Formal Language Theory (Part \ref{part:formal_language})

Chomasky Hierarchy (\S\ref{sec:chomsky_hierarchy})



% ------------------------------------------------------------------------------
\subsection{Word Type}\label{sec:word_type}
% ------------------------------------------------------------------------------

\subsubsection{Sturmian Word}\label{sec:sturmian_word}

\subsubsection{Lyndon Word}\label{sec:lyndon_word}



% ==============================================================================
\section{Infinitary Combinatorics}\label{sec:infinitary_combinatorics}
% ==============================================================================

Set Theory (Part \ref{part:set_theory})

\emph{Combinatorial Set Theory}

Graphons (``Continuous Graphs'' \S\ref{sec:graphon})



% ==============================================================================
\section{Arithmetic Combinatorics}\label{sec:arithmetic_combinatorics}
% ==============================================================================

Ergodic Ramsey Theory



% ------------------------------------------------------------------------------
\subsection{Additive Combinatorics}\label{sec:additive_combinatorics}
% ------------------------------------------------------------------------------

only Addition and Subtraction

2016 - Laba, Falconer - \emph{Harmonic Analysis and Additive Combinatorics on
  Fractals}
