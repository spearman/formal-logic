%%%%%%%%%%%%%%%%%%%%%%%%%%%%%%%%%%%%%%%%%%%%%%%%%%%%%%%%%%%%%%%%%%%%%%%%%%%%%%%%
%%%%%%%%%%%%%%%%%%%%%%%%%%%%%%%%%%%%%%%%%%%%%%%%%%%%%%%%%%%%%%%%%%%%%%%%%%%%%%%%
\part{Combinatorics}\label{part:combinatorics}
%%%%%%%%%%%%%%%%%%%%%%%%%%%%%%%%%%%%%%%%%%%%%%%%%%%%%%%%%%%%%%%%%%%%%%%%%%%%%%%%
%%%%%%%%%%%%%%%%%%%%%%%%%%%%%%%%%%%%%%%%%%%%%%%%%%%%%%%%%%%%%%%%%%%%%%%%%%%%%%%%

Taking \emph{Combination} (\S\ref{sec:combination}) as a Primitive Notion, one
may divide Mathematics into two branches:
\begin{itemize}
  \item \emph{Finitary Combinatorics} (\emph{Combinatorics on Words}
    \S\ref{sec:combinatorics_on_words}) -- Formal Language Theory (Part
    \ref{part:formal_language})
  \item \emph{Infinitary Combinatorics} (\S\ref{sec:infinitary_combinatorics}):
    Set Theory (Part \ref{part:set_theory})
\end{itemize}

(wiki: \url{https://en.wikipedia.org/wiki/Structure_(category_theory)})

\emph{Combinatorics as ``De-structured'' Mathematics} -- ``recognizable''
\emph{Structure}'' vs. ``\emph{Combinatoric} description'' (requiring ``special
arguments''):
\begin{itemize}
  \item Structure (Model Theory \S\ref{sec:abstract_structure})
  \item Algebraic Structure (Abstract Algebra \S\ref{sec:abstract_structure})
  \item Data Structure (Database Theory \S\ref{sec:data_structure})
  \item Abstract Structure (Appendix \S\ref{sec:abstract_structure})
\end{itemize}

subfields:
\begin{itemize}
\item \emph{Enumerative Combinatorics} (\S\ref{sec:enumerative_combinatorics}):
  finding \emph{Counting Functions} (\S\ref{sec:counting_function}) to Compute
  (\S\ref{sec:computability}) the Cardinality (\S\ref{sec:cardinality}) of
  Elements of Finite Sets \emph{without} ``actually'' Counting them
\item \emph{Discrete Mathematics}: Countably Infinite
  (\S\ref{sec:countably_infinite}) Combinatorics
\end{itemize}

see also:
\begin{itemize}
  \item Discrete Geometry (Combinatorial Geometry \S\ref{sec:discrete_geometry})
  \item Geometric Probability (Continuous Combinatorics
    \S\ref{sec:geometric_probability}): analogies between \emph{Counting} and
    \emph{Measure} (\S\ref{sec:measure}) \fist Measure Theory (Part
    \ref{part:measure_theory})
  \item Combinatorial Group Theory (\S\ref{sec:combinatorial_group_theory})
  \item Combinatorial Optimization (\S\ref{sec:combinatorial_optimization})
  \item Combinatorial Game Theory (\S\ref{sec:combinatorial_game_theory})
  \item ... MORE?
\end{itemize}

\fist Discrete Uniform Law (Probability Theory \S\ref{sec:discrete_uniform_law})
%FIXME: clarify



% ==============================================================================
\section{Counting}\label{sec:counting}
% ==============================================================================

(wiki):

a means of establishing a Bijection (One-to-one correspondence
\S\ref{sec:bijective_function}) between the Subset of Natural Numbers
(\S\ref{sec:nats}) $\{1, 2, \ldots, n\}$ and a Set of Cardinality
(\S\ref{sec:cardinality}) $n$

\textbf{Thm.} no Bijection can exist between Sets of different Cardinality, and
two Bijections can be Composed to give another Bijection; therefore Counting the
same Set in different ``ways'' can never result in different Cardinality

Sets for which no Bijection exists for any Finite $n$ are called \emph{Infinite
  Sets}, otherwise they are Finite Sets

cf. Countable (\S\ref{sec:countably_infinite}), Uncountable
(\S\ref{sec:uncountably_infinite})

adding Elements to an Infinite Set does not necessarily increase its size
(Cardinality)



% ------------------------------------------------------------------------------
\subsection{Counting Principle}\label{sec:counting_principle}
% ------------------------------------------------------------------------------

a.k.a. \emph{Combinatorial Principles} or \emph{Combinatorial Rules}



\subsubsection{Multiplication Principle}\label{sec:multiplication_principle}

if there are $a$ possible outcomes for an event (``ways of doing something'',
e.g. choosing a Member of a Set) and $b$ possible outcomes for another event,
then there are $a \cdot b$ possible outcomes for the joint occurrence of both
events (``ways of doing both actions'')



\subsubsection{Addition Principle}\label{sec:addition_principle}

\subsubsection{Inclusion-exclusion Principle}\label{sec:inclusion_exclusion}

\subsubsection{Pigeonhole Principle}\label{sec:pigeonhole_principle}



% ==============================================================================
\section{Combinatorial Proof}\label{sec:combinatorial_proof}
% ==============================================================================

% ------------------------------------------------------------------------------
\subsection{Bijective Proof}\label{sec:bijective_proof}
% ------------------------------------------------------------------------------

% ------------------------------------------------------------------------------
\subsection{Double Counting}\label{sec:double_counting}
% ------------------------------------------------------------------------------



% ==============================================================================
\section{Enumerative Combinatorics}\label{sec:enumerative_combinatorics}
% ==============================================================================

(wiki): finding \emph{Counting Functions} (\S\ref{sec:counting_function}) to
Compute (\S\ref{sec:computability}) the Cardinality (\S\ref{sec:cardinality}) of
Elements of Finite Sets \emph{without} ``actually'' Counting
(\S\ref{sec:counting}) them

Counting Structures

main subject of study are \emph{Counting Sequences}
(\S\ref{sec:counting_sequence}) of Combinatorial Classes
(\S\ref{sec:combinatorial_class})

2013 SFSU - Ardila - \emph{Enumerative Combinatorics} -
\url{https://www.youtube.com/watch?v=pCJNjW8kMIg&list=PL-XzhVrXIVeSi7xym1XAfFIxOAaHVhjtP}



% ------------------------------------------------------------------------------
\subsection{Counting Function}\label{sec:counting_function}
% ------------------------------------------------------------------------------

$f : \nats \biject X$



% ------------------------------------------------------------------------------
\subsection{Combination}\label{sec:combination}
% ------------------------------------------------------------------------------

Unordered choice of Elements

$k$-combination of $n$ Elements is given by the Binomial Coefficient
(\S\ref{sec:binomial_coefficient}) $\binom{n}{k}$, $n$ ``choose'' $k$, equal to:
\[
  \frac{n!}{k!(n - k)!}
\]



\subsubsection{Binomial Coefficient}\label{sec:binomial_coefficient}

Binomial (\S\ref{sec:binomial})

$\binom{n}{k} = \frac{n!}{k!(n - k)!}$

number of ways to Partition a Set of Cardinality $n$ into two Subsets of
Cardinality $k$ and $n-k$

counting Partitions (\S\ref{sec:partition}) of prescribed Cardinalities:

$\binom{n}{n_1 n_2 \cdots n_k}$



\paragraph{Catalan Number}\label{sec:catalan_number}\hfill



% ------------------------------------------------------------------------------
\subsection{Permutation}\label{sec:permutation}
% ------------------------------------------------------------------------------

Ordered choice of Elements

The number of Permutations of $n$ Elements is $n!$

The number of Cyclic Permutations (\S\ref{sec:cyclic_permutation}) of $n$
Elements is $(n-1)!$

The number of $k$ choices out of $n$ Elements when order is significant is
$\frac{n!}{(n-1)!}$

Permutation Group (\S\ref{sec:permutation_group})



\subsubsection{Cyclic Permutation}\label{sec:cyclic_permutation}

A Permutation is \emph{Cyclic} if and only if it consists of a single
Non-trivial Cycle (Cycle Length > 1).



\subsubsection{Transposition}\label{sec:transposition}

A \emph{Transposition} is a Cyclic Permutation of length 2 which exchanges two
Elements and leaves all other Elements fixed.



\subsubsection{Derangement}\label{sec:derangement}

a Permutation such that no Element appears in its original position, i.e. a
Permutation with no \emph{Fixed Points}



\subsubsection{Superpermutation}\label{sec:superpermutation}

\subsubsection{Permutation Pattern}\label{sec:permutation_pattern}

\subsubsection{Permutation Class}\label{sec:permutation_class}

\paragraph{Wilf Class}\label{sec:wilf_class}\hfill

a Combinatorial Class (\S\ref{sec:combinatorial_class}) of Permutation Classes
enumerated by Length



% ------------------------------------------------------------------------------
\subsection{Partition}\label{sec:partition}
% ------------------------------------------------------------------------------

Set Partitions (\S\ref{sec:set_partition})

Binomial Coefficient (\S\ref{sec:binomial_coefficient})

$\binom{n}{n_1 n_2 \cdots n_k}$



\subsubsection{Bell Number}\label{sec:bell_number}

Bell Number

$B_{n+1} = \sum_{k=0}^n \binom{n}{k} B_k$



% ------------------------------------------------------------------------------
\subsection{Classification Problem}\label{sec:classification_problem}
% ------------------------------------------------------------------------------

(wiki):

\emph{Classification Problem}: what are the ``objects'' of a given ``type'' up
to some Equivalence?



\subsubsection{Classification Theorem}\label{sec:classification_theorem}

(wiki):

a \emph{Classification Theorm} answers a Classification Problem for ``objects''
of a given ``type'', giving a non-redundant Enumeration of Equivalence Classes

\begin{itemize}
  \item \emph{Equivalence Problem} (\S\ref{sec:equivalence_problem}): given two
    objects, determine if they are Equivalent
  \item \emph{Complete Set of Invariants} (\S\ref{sec:invariant}): a Set of
    Invariants such that Equivalence of the Invariants in the Set is a
    \emph{sufficient condition} for Equivalence of objects; together with
    ``which Invariants are realizable'' solves the Classification problem, and a
    \emph{Computable Complete Set of Invariants} together with ``which
    Invariants are realizable'' solves the Classification Problem and the
    Equivalence Problem (FIXME: clarify)
  \item \emph{Canonical Form} (\S\ref{sec:canonical_form}): solves the
    Classification Problem and provides a distinguished (``Canonical'') Element
    of each Class
\end{itemize}

cf. Exceptional Objects (\S\ref{sec:exceptional}) -- Classifications often
result in a number of \emph{(Infinite) Series of Objects}, and a \emph{Finite
  Number} of \emph{Exceptions} that do not fit into any Series, and such
\emph{Exceptional Objects} are often related to Exceptional Objects in other
branches of Mathematics

\begin{itemize}
  \item Classification of Euclidean Plane Isometries
  \item Classification of Two-dimensional Closed Manifolds
  \item Thurston's Geometrization Conjecture
  \item Classification of Finite Simple Groups
  \item ... MORE
\end{itemize}

TODO: xrefs



% ------------------------------------------------------------------------------
\subsection{The Twelvefold Way}\label{sec:twelvefold_way}
% ------------------------------------------------------------------------------

(wiki):

Systematic classification of 12 related Enumerative problems concerning two
Finite Sets, including classical problems of counting:
\begin{itemize}
  \item Permutations (\S\ref{sec:permutation})
  \item Combinations (\S\ref{sec:combination})
  \item Multisets (\S\ref{sec:multiset})
  \item Partitions of a Set (\S\ref{sec:partition})
  \item Partitions of an Integer (\S\ref{sec:integer_partition})
\end{itemize}

Cases:
\begin{enumerate}
  \item Functions from $N$ to $X$
  \item Functions from $N$ to $X$, up to a Permutation of $N$
  \item Functions from $N$ to $X$, up to a Permutation of $X$
  \item Functions from $N$ to $X$, up to Permutations of $N$ and $X$
  \item Injective Functions from $N$ to $X$
  \item Injective Functions from $N$ to $X$, up to a Permutation of $N$
  \item Injective Functions from $N$ to $X$, up to a Permutation of $X$
  \item Injective Functions from $N$ to $X$, up to Permutations of $N$ and $X$
  \item Surjective Functions from $N$ to $X$
  \item Surjective Functions from $N$ to $X$, up to a Permutation of $N$
  \item Surjective Functions from $N$ to $X$, up to a Permutation of $X$
  \item Surjective Functions from $N$ to $X$, up to Permutations of $N$ and $X$
\end{enumerate}



% ------------------------------------------------------------------------------
\subsection{Lattice Path}\label{sec:lattice_path}
% ------------------------------------------------------------------------------

\subsubsection{Dyck Path}\label{sec:dyck_path}

\subsubsection{Schr\"oder Path}\label{sec:schroder_path}

\emph{Schr\"oder Paths and Reverse Bessel Polynomials} --
\url{https://golem.ph.utexas.edu/category/2017/08/schrder_paths_and_reverse_bessel_polynomials.html}

Reverse Bessel Polynomial (\S\ref{sec:reverse_bessel_polynomial})



% ==============================================================================
\section{Analytic Combinatorics}\label{sec:analytic_combinatorics}
% ==============================================================================

\cite{flajolet-sedgewick09}

Asymptotic Analysis (\S\ref{sec:asymptotic_analysis})

1964 - Ulam - \emph{Combinatorial Analysis in Infinite Sets and Some Physical
  Theories}

2004 - Flajolet, Sedgewick - \emph{Analytic Combinatorics}



% ------------------------------------------------------------------------------
\subsection{Combinatorial Class}\label{sec:combinatorial_class}
% ------------------------------------------------------------------------------

a Countable Set of Mathematical Objects (\S\ref{sec:mathematical_object})
together with a \emph{Size Function} mapping each Object to a Non-negative
Integer such that there are \emph{Finitely many} Objects of each Size

two Combinatorial Classes are Isomorphic if they have the same \emph{Counting
  Sequence} (\S\ref{sec:counting_sequence}), i.e. the same number of Objects of
each Size

\begin{itemize}
  \item a Wilf Class (\S\ref{sec:wilf_class}) is a Combinatorial Class of
    Permutation Classes (\S\ref{sec:permutation_class}) enumerated by Length is
    a Combinatorial Class
  \item ...
\end{itemize}



\subsubsection{Counting Sequence}\label{sec:counting_sequence}

the Sequence (\S\ref{sec:sequence}) of numbers of Elements of Size $i = 0, 1, 2,
\ldots$; may be described as a \emph{Generating Function}
(\S\ref{sec:generating_function}) with those numbers as Coefficients

main subject of study of \emph{Enumerative Combinatorics}
(\S\ref{sec:numerative_combinatorics})



% ------------------------------------------------------------------------------
\subsection{Combinatorial Species}\label{sec:combinatorial_species}
% ------------------------------------------------------------------------------

1981 - Joyal - \emph{A Combinatorial Theory of Formal Series}

abstract method for analyzing Discrete Structures in terms of \emph{Generating
  Functions} (\S\ref{sec:generating_function})

Ordinary Generating Functions (OGFs) describe Structures on Totally Ordered Sets
(\S\ref{sec:totally_ordered})

Exponential Generating Functions (EGFs) describe structures on unordered Sets

Presheaf (\S\ref{sec:presheaf})

Categorification of Product Rule (\S\ref{sec:product_rule}) %TODO

\url{https://topologicalmusings.wordpress.com/2009/03/28/number-of-idempotent-endofunctions/},
\url{https://harrisonbrown.wordpress.com/2009/06/30/category-theory-and-combinatorics/}

1981 - Joyal - \emph{A Combinatorial Theory of Formal Series}, transl. Yorgey
2014: \url{https://github.com/byorgey/series-formelles}

a \emph{Combinatorial Species} is an Endofunctor $F : \cat{FinSet}_{\biject}
\rightarrow \cat{FinSet}_{\biject}$ called the \emph{Species of $F$-structures}
on the Groupoid $\cat{FinSet}_{\biject}$ of Finite Sets with Bijections as
Morphisms, where the Finite Set $F A$ is called the \emph{Set of $F$-structures
  on Finite Set $A$} or the \emph{Set of Structures of Species $F$ on Finite Set
  $A$}

note the Functor may also be defined as $F : \cat{FinSet}_\biject \rightarrow
\cat{FinSet}$ into the Category of Finite Sets and Total Functions (Joyal81
\S2.2)

for Finite Set $E \in \cat{FinSet}_{\biject}_0$ and Species $M$ and $M E$ is the
Set of all $M$-structures on $E$, $E$ is said to be the Underlying Set of a
particular $s \in M E$ or that it is \emph{Supported} by $E$; one can think of
$M$ as sending Finite Sets of \emph{Labels} to Finite Sets of \emph{Labelled
  Structures}

for Morphism $u : E \rightarrow F \in \cat{FinSet}_{\biject}_0$ is a Bijection
between Label Sets $E$ and $F$, and can be thought of as a \emph{Relabelling};
$M$ sends $u$ to a Bijection $M u$ between Sets of Labelled Structures where
each Structure corresponds to its Relabelled version; for example
$(M u)(s) = t \in M F$ for $s \in M E$ is a Relabelling of the $M$-structure
$s$ (via ``\emph{Transport}''), and $s$ and $t$ are Isomorphic; this Functorial
notion of Transport allows an \emph{Extensional} rather than Intensional
definition of ``Species'', i.e. Species are defined by their ``Behavior'' (what
can be ``done'' with them), not by their Properties, i.e. Species are things
which \emph{generate} a Set of Structures from a Set of Labels and which can be
Functorially Relabelled

the \emph{Type} $|s| \in \pi_0(M)$ of an $M$-structure $s \in M$ is the
Equivalence Class of Isomorphic Structures (i.e. under Relabelling) in $el(M)$,
the Groupoid of $M$-structures and Ismorphisms of $M$-structures (i.e. a
Connected Component of $el(M)$), where $\pi_0(M)$ is the Set of Types
(Equivalence Classes) of $M$-structures; since there are an Infinite number of
Finite Sets of Labels, each Type contains an Infinite Number of Isomorphic
Structures with the same ``shape''

one can define Species in HoTT (\S\ref{sec:hott}) such that Transport of Species
is given by Transport along Paths (Yorgey14)

2010 - Yorgey - \emph{Species and Functors and Types, Oh My!}

\url{http://hackage.haskell.org/package/species} -- Haskell DSL package for
describing and computing with Combinatorial Species

2014 - Yorgey - \emph{Combinatorial Species and Labelled Structures}

\url{https://homotopytypetheory.org/2016/07/20/combinatorial-species-and-finite-sets-in-hott/}:
Combinatorial Species can be seen as a Categorification of Generating Functions,
which, over a Semiring (\S\ref{sec:semiring}) $R$, are represented by Functions
$\nats \rightarrow R$ mapping Natural Numbers to Coefficients in $R$, and each
Natural Number corresponds to an Equivalence Class of \emph{Finite} Sets



% ------------------------------------------------------------------------------
\subsection{Symbolic Method}\label{sec:symbolic_method}
% ------------------------------------------------------------------------------

Asymptotic Combinatorics



% ==============================================================================
\section{Combinatorics on Words}\label{sec:combinatorics_on_words}
% ==============================================================================

or \emph{Finitary Combinatorics}

Formal Language Theory (Part \ref{part:formal_language})

Chomasky Hierarchy (\S\ref{sec:chomsky_hierarchy})



% ------------------------------------------------------------------------------
\subsection{Word Type}\label{sec:word_type}
% ------------------------------------------------------------------------------

\subsubsection{Sturmian Word}\label{sec:sturmian_word}

\subsubsection{Lyndon Word}\label{sec:lyndon_word}



% ==============================================================================
\section{Infinitary Combinatorics}\label{sec:infinitary_combinatorics}
% ==============================================================================

Set Theory (Part \ref{part:set_theory})

\emph{Combinatorial Set Theory}

Graphons (``Continuous Graphs'' \S\ref{sec:graphon})

\emph{Discrete Mathematics}: Countably Infinite (\S\ref{sec:countably_infinite})
Combinatorics



% ==============================================================================
\section{Algebraic Combinatorics}\label{sec:algebraic_combinatorics}
% ==============================================================================

% ------------------------------------------------------------------------------
\subsection{Polynomial Sequence}\label{sec:polynomial_sequence}
% ------------------------------------------------------------------------------

Polynomial (\S\ref{sec:polynomial}) Sequence (\S\ref{sec:sequence})

\fist Formal Power Series (\S\ref{sec:formal_power_series})



\subsubsection{Orthogonal Polynomial Sequence}
\label{sec:orthogonal_polynomial_sequence}

\paragraph{Bessel Polynomial}\label{sec:bessel_polynomial}\hfill

\subparagraph{Reverse Bessel Polynomial}\label{sec:reverse_bessel_polynomial}
\hfill

cf. Bessel Functions (\S\ref{sec:bessel_function})

\emph{Schr\"oder Paths and Reverse Bessel Polynomials} --
\url{https://golem.ph.utexas.edu/category/2017/08/schrder_paths_and_reverse_bessel_polynomials.html}

Schr\"oder Path (\S\ref{sec:schroder_path})



% ------------------------------------------------------------------------------
\subsection{Matroid Theory}\label{sec:matroid_theory}
% ------------------------------------------------------------------------------

\subsubsection{Independence System}\label{sec:independence_system}

an Abstract Simplicial Complex (\S\ref{sec:abstract_complex}) in the context of
Matroid Theory



\subsubsection{Matroid}\label{sec:matroid}

(wiki): generalizes the notion of \emph{Linear Independence} in Vector Spaces
(\S\ref{sec:vector_space})



\subsubsection{Antimatroid}\label{sec:antimatroid}

\subsubsection{Greedoid}\label{sec:greedoid}



% ==============================================================================
\section{Arithmetic Combinatorics}\label{sec:arithmetic_combinatorics}
% ==============================================================================

Ergodic Ramsey Theory



% ------------------------------------------------------------------------------
\subsection{Additive Combinatorics}\label{sec:additive_combinatorics}
% ------------------------------------------------------------------------------

only Addition and Subtraction

2016 - Laba, Falconer - \emph{Harmonic Analysis and Additive Combinatorics on
  Fractals}



% ==============================================================================
\section{Probabilistic Combinatorics}\label{sec:probabilistic_combinatorics}
% ==============================================================================

Probability Theory (Part \S\ref{part:probability_theory})



% ------------------------------------------------------------------------------
\subsection{Probabilistic Method}\label{sec:probabilistic_method}
% ------------------------------------------------------------------------------



% ==============================================================================
\section{Extremal Combinatorics}\label{sec:extremal_combinatorics}
% ==============================================================================

Combinatorial Optimization (\S\ref{sec:combinatorial_optimization})



% ------------------------------------------------------------------------------
\subsection{Ramsey Theory}\label{sec:ramsey_theory}
% ------------------------------------------------------------------------------

\subsubsection{Ramsey's Theorem}\label{sec:ramseys_theorem}



% ==============================================================================
\section{Combinatorial Design}\label{sec:combinatorial_design}
% ==============================================================================

\emph{Design Theory}
