%%%%%%%%%%%%%%%%%%%%%%%%%%%%%%%%%%%%%%%%%%%%%%%%%%%%%%%%%%%%%%%%%%%%%%
%%%%%%%%%%%%%%%%%%%%%%%%%%%%%%%%%%%%%%%%%%%%%%%%%%%%%%%%%%%%%%%%%%%%%%
\part{Type Theory}\label{sec:type_theory}
%%%%%%%%%%%%%%%%%%%%%%%%%%%%%%%%%%%%%%%%%%%%%%%%%%%%%%%%%%%%%%%%%%%%%%
%%%%%%%%%%%%%%%%%%%%%%%%%%%%%%%%%%%%%%%%%%%%%%%%%%%%%%%%%%%%%%%%%%%%%%

\emph{Type Theory} is the study of classes of Formal Systems
(\S\ref{sec:formal_system}) where each Term
(\S\ref{sec:term}) has a \emph{Type} and Operations
are restricted to Terms of specific Types.

As a Formal Theory (Part \ref{sec:proof_theory}), Judgements in Type
Theory are of three kinds\cite{hott13}:
\begin{enumerate}

\item \emph{Well-formed Context} (\S\ref{sec:type_context})
\[
    (\Gamma) ctx
\]
``$\Gamma$ is a Well-formed Context''

\item \emph{Propositional Equality}
\[
    \Gamma \vdash a : A
\]
``Given Contex $\Gamma$, $a$ is a Term of Type $A$''

\item \emph{Definitional (Judgemental) Equality}
\[
    \Gamma \vdash a \equiv b : A
\]
``Given Context $\Gamma$, $a$ and $b$ are definitionally equal Terms
of Type $A$''

\end{enumerate}
with a Deductive Apparatus (\S\ref{sec:deductive_apparatus})
consisting of Inference Rules only and no Axioms. Judgemental Equality
is an Equivalence Relation respected by Typing.

Inference Rules have the form:
\[
    \frac{J_1 \quad \cdots \quad J_k} {J} Name
\]
where $J_i$ are provided as derived hypothetical (metatheoretical)
Judgements and $J$ is the conclusion.

A tree constructed from Inference Rules forms a \emph{Derivation} of a
Judgement.



% ====================================================================
\section{Expression}\label{sec:type_expression}
% ====================================================================

An \emph{Expression} is an Equivalence Class of Syntactic forms which
differ in the names of \emph{Bound Variables}. That is, changing the
name of a Bound Variable everywhere within an Expression
(\emph{$\alpha$-conversion}) does not change the Expression.

A Variable is Bound in an Expression by an \emph{Abstraction}
expressing that the Variable is \emph{local} to the Expression:
\[
    \lambda x.B
\]
or
\[
    x.B
\]

\emph{Substitution}:
\[
    B[a/x]
\]
Substitute Term $a$ for Free occurrences of Variable $x$ in the Term
$B$. Generalized:
\[
    B[a_1,\ldots,a_n / x_1,\ldots,x_n]
\]



% ====================================================================
\section{Context}\label{sec:type_context}
% ====================================================================

A \emph{Context}, $\Gamma$, is a list of Assumptions of the form:
\[
    \Gamma = x_1 : A_1, x_2 : A_2, \ldots, x_n : A_n
\]
where each Element $x_i : A_i$ is an Assumption that the distinct
Variable $x_i$ has type $A_i$.

Judgements are formulated under the Assumptions of a particular
Context, $\Gamma$:
\[
    \Gamma \vdash a : A
\]
For an empty Context:
\[
    \vdash a : A
\]
or
\[
    . \vdash a : A
\]



% ====================================================================
\section{Type Universe}\label{sec:type_universe}
% ====================================================================

A \emph{Type Universe} is a Type whose Elements are Types. A hierarchy:
\[
    \mathcal{U}_0, \mathcal{U}_1, \mathcal{U}_2, \ldots
\]
is such that any Type in $\mathcal{U}_i$ is also in
$\mathcal{U}_{i+1}$. See $\mathcal{U}-INTRO$ and $\mathcal{U}-CUMUL$
\S\ref{sec:universe_rules}.



% ====================================================================
\section{Inference Rules}\label{sec:type_inference}
% ====================================================================

% --------------------------------------------------------------------
\subsection{Context Rules}
% --------------------------------------------------------------------

The following Rules of Inference allow for the determination of a
Well-formed Context:
\begin{enumerate}
\item
\[
    {
        \frac{}{(.)ctx}
    } ctx-EMP
\]
\item
\[
    {
        \frac
        {x_1:A_1, \ldots, x_{n-1}:A_{n-1} \vdash A_n : \mathcal{U}_i}
        {(x_1:A_1,\ldots,x_n:A_n) ctx}
    } ctx-EXT
\]
\end{enumerate}



% --------------------------------------------------------------------
\subsection{Structural Rules}
% --------------------------------------------------------------------

Given a Context, derive Typing Judgements listed in the Context:
\[
    {
        \frac
        {(x_1:A_1, \ldots, x_n:A_n)ctx}
        {x_1:A_1, \ldots, x_n:A_n \vdash x_i:A_i}
    } Vble
\]

Substitution for Typing Judgements:
\[
    {
        \frac
        {\Gamma \vdash a : A \;\;\;\;\;\;\;
        \Gamma,x:A,\Delta \vdash b : B}
        {\Gamma,\Delta[a/x] \vdash b[a/x] : B[a/x]}
    } Subst_1
\]

Weakening for Typing Judgements:
\[
    {
        \frac
        {\Gamma \vdash A : \mathcal{U}_i \;\;\;\;\;\;\;
        \Gamma,\Delta \vdash b : B}
        {\Gamma,x:A,\Delta \vdash b:B}
    } Wkg_1
\]

Substitution for Judgemental Equality:
\[
    {
        \frac
        {\Gamma \vdash a : A \;\;\;\;\;\;\;
        \Gamma,x:A,\Delta \vdash b \equiv c : B}
        {\Gamma,\Delta[a/x] \vdash b[a/x] \equiv c[a/x] : B[a/x]}
    } Subst_2
\]

Weakening for Judgemental Equality:
\[
    {
        \frac
        {\Gamma \vdash A : \mathcal{U}_i \;\;\;\;\;\;\;
        \Gamma,\Delta \vdash b \equiv c : B}
        {\Gamma, x:A, \Delta \vdash b \equiv c : B}
    } Wkg_2
\]



% --------------------------------------------------------------------
\subsection{Universe Rules}\label{sec:universe_rules}
% --------------------------------------------------------------------

\[
    {
        \frac
        {(\Gamma) ctx}
        {\Gamma \vdash \mathcal{U}_i : \mathcal{U}_{i+1}}
    } \mathcal{U}-INTRO
\]

\[
    {
        \frac
        {\Gamma \vdash A : \mathcal{U}_i}
        {\Gamma \vdash A : \mathcal{U}_{i+1}}
    } \mathcal{U}-CUMUL
\]



% --------------------------------------------------------------------
\subsection{Dependent Function Type Rules}\label{sec:dependent_rules}
% --------------------------------------------------------------------

\[
    {
        \frac
        {\Gamma \vdash A : \mathcal{U}_i \;\;\;\;\;\;\;
        \Gamma,x:A \vdash B : \mathcal{U}_i}
        {\Gamma \vdash \prod_{(x:A)} B : \mathcal{U}_i}
    } \Pi-FORM
\]\[
    {
        \frac
        {}
        {}
    } \Pi-INTRO
\]\[
    {
        \frac
        {}
        {}
    } \Pi-ELIM
\]\[
    {
        \frac
        {}
        {}
    } \Pi-COMP
\]\[
    {
        \frac
        {}
        {}
    } \Pi-UNIQ
\]



% ====================================================================
\section{Typed $\lambda$-Calculus}\label{sec:typed_lambda}
% ====================================================================

% --------------------------------------------------------------------
\subsection{Simply Typed $\lambda$-Calculus}\label{sec:simply_typed}
% --------------------------------------------------------------------

Isomorphic by Curry-Howard (\S\ref{sec:curry_howard}) to
Propositional Logic (\S\ref{sec:propositional_logic})

Propositional Logic $\leftrightarrow$ Simply Typed $\lambda$-Calculus:
\[
    \supset \leftrightarrow \rightarrow
\] \[
    \wedge \leftrightarrow \times
\] \[
    \vee \leftrightarrow +
\] \[
    False \leftrightarrow \bot
\]



% --------------------------------------------------------------------
\subsection{Second-order $\lambda$-Calculus}
\label{sec:secondorder_lambda_calculus}
% --------------------------------------------------------------------



% --------------------------------------------------------------------
\subsection{Calculus of Constructions}\label{sec:coq}
% --------------------------------------------------------------------



% ====================================================================
\section{Intuitionistic Type Theory}
% ====================================================================

\emph{Intuitionistic Type Theory} (also \emph{Constructive Type
  Theory} or \emph{Martin-L\"of Type Theory})

Inductive Types



% --------------------------------------------------------------------
\subsection{Types}
% --------------------------------------------------------------------

\emph{Function Type}

\emph{Higher-order Function}

\emph{Product Type}

\subsubsection{Empty Type $\mathbf{0}$}

\subsubsection{Unit Type $\mathbf{1}$}

\subsubsection{Natural Number Type}

\subsubsection{Identity Type}

\subsubsection{Dependent Function Type ($\Pi$-type)}

\[
    A \rightarrow B :\equiv \prod_{x:A} B
\]

\subsubsection{Dependent Pair Type ($\Sigma$-type)}

In Set Theory: Indexed Sum over a given Type.

\subsubsection{Coproduct Type}



% --------------------------------------------------------------------
\subsection{Intensional \& Extensional Type Theory}
\label{sec:intension_extension}
% --------------------------------------------------------------------

\emph{Canonical Form}

\emph{Normal Form}



% ====================================================================
\section{Pure Type System}\label{sec:pure_type_sytem}
% ====================================================================

Arbitrary number of Sorts and Dependencies

Generalization of the \emph{Lambda Cube} to more sorts than Terms and
Types

Not necessarily Strongly Normalizing (\S\ref{sec:normal_form})



% --------------------------------------------------------------------
\subsection{Lambda Cube}\label{sec:lambda_cube}
% --------------------------------------------------------------------

Simply-typed $\lambda$-calculus allows only Terms to depend on Types.



% --------------------------------------------------------------------
\subsection{Subtype}\label{sec:subtype}
% --------------------------------------------------------------------



% ====================================================================
\section{Homotopy Type Theory}\label{sec:homotopy_type}
% ====================================================================

- Function Extensionality
- Univalence Axiom
- Higher Inductive Types



% ====================================================================
\section{Curry-Howard Correspondence}\label{sec:curry_howard}
% ====================================================================

\begin{tabular}{| l | l |}
\hline
\textbf{Type Theory} & \textbf{Logic} \\ \hline \hline
$A$ (Type) & Proposition \\ \hline
$a : A$ (Term) & Proof \\
\hline
\end{tabular}

Natural Deduction - Typed Lambda Calculus

Propositional Logic - Simple Types

Predicate Logic - Dependent Types

Modal Logic - Monads

Classical-Intuitionistic Embedding - Continuation Passing Style
