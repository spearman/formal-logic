%%%%%%%%%%%%%%%%%%%%%%%%%%%%%%%%%%%%%%%%%%%%%%%%%%%%%%%%%%%%%%%%%%%%%%
%%%%%%%%%%%%%%%%%%%%%%%%%%%%%%%%%%%%%%%%%%%%%%%%%%%%%%%%%%%%%%%%%%%%%%
\part{Type Theory}\label{sec:type_theory}
%%%%%%%%%%%%%%%%%%%%%%%%%%%%%%%%%%%%%%%%%%%%%%%%%%%%%%%%%%%%%%%%%%%%%%
%%%%%%%%%%%%%%%%%%%%%%%%%%%%%%%%%%%%%%%%%%%%%%%%%%%%%%%%%%%%%%%%%%%%%%

\emph{Type Theory} is the study of Classes of Formal Systems
(\S\ref{sec:formal_system}) where each Term (\S\ref{sec:term}) has a
\emph{Type} (\S\ref{sec:type}) and Operations are restricted to Terms
of specific Types.

As a Formal Theory (\S\ref{sec:formal_theory}), Judgements
(\S\ref{sec:judgement}) in Type Theory are of three
kinds\cite{hott13}:
\begin{enumerate}

\item \emph{Well-formed Context} (\S\ref{sec:type_context}):
  \[
    (\Gamma) ctx
  \]
  ``$\Gamma$ is a Well-formed Context''

\item \emph{Propositional Equality} (Typing Judgement
    \S\ref{sec:typing_judgement}):
  \[
    \Gamma \vdash a : A
  \]
  ``Given Contex $\Gamma$, $a$ is a Term of Type $A$''

\item \emph{Definitional (Judgemental) Equality}:
  \[
    \Gamma \vdash a \equiv b : A
  \]
  ``Given Context $\Gamma$, $a$ and $b$ are Definitionally Equal Terms
  of Type $A$''

\end{enumerate}
with a Deductive Apparatus (\S\ref{sec:deductive_apparatus})
consisting of Inference Rules (\S\ref{sec:type_inference}) only and no
Axioms. Judgemental Equality is an Equivalence Relation respected by
Typing.



\textbf{Intensional \& Extensional Type Theory}

\begin{itemize}
\item \emph{Extensional Type Theory}: Definitional (Computational)
  Equality is not distinguished from Propositional (Proof) Equality
  but Type Checking (\S\ref{sec:type_checking}) is Undecidable
\item \emph{Intensional Type Theory}: Type Checking is Decidable but
  Extensional Reasoning must be carried out using Setoids
  (\S\ref{sec:setoid})
\item \emph{Homotopy Type Theory} (\S\ref{sec:homotopy_type}): Higher
  Inductive Types (\S\ref{sec:higher_inductive_type}) allow definition
  of Higher-order Constructors
\end{itemize}

Term Rewrite System (\S\ref{sec:term_rewriting})

\emph{Conversion Rules}

\emph{Canonical Form}

\emph{Normal Form}



% ====================================================================
\section{Expression}\label{sec:type_expression}
% ====================================================================

An \emph{Expression} is an Equivalence Class of Syntactic forms which
differ in the names of \emph{Bound Variables}. That is, changing the
name of a Bound Variable everywhere within an Expression
(\emph{$\alpha$-conversion}) does not change the Expression.

A Variable is Bound in an Expression by an \emph{Abstraction}
expressing that the Variable is \emph{local} to the Expression:
\[
  \lambda x.B
\]
or
\[
  x.B
\]

\emph{Substitution}:
\[
  B[a/x]
\]
Substitute Term $a$ for Free occurrences of Variable $x$ in the Term
$B$. Generalized:
\[
  B[a_1,\ldots,a_n / x_1,\ldots,x_n]
\]



% --------------------------------------------------------------------
\subsection{Term Constant}\label{sec:term_constant}
% --------------------------------------------------------------------

A \emph{Term Constant} is a Term that is a Base Type
(\S\ref{sec:type_constant})
% FIXME ???



% --------------------------------------------------------------------
\subsection{Well-typed Term}\label{sec:well_typed}
% --------------------------------------------------------------------

A \emph{Well-typed Term} (or \emph{Typable Term}) is a Term for which
a Typing Derivation (\S\ref{sec:typing_derivation}) exists.



% ====================================================================
\section{Context}\label{sec:type_context}
% ====================================================================

A \emph{Context}, $\Gamma$, is a list of Assumptions of the form:
\[
  \Gamma = x_1 : A_1, x_2 : A_2, \ldots, x_n : A_n
\]
where each Element $x_i : A_i$ is an Assumption that the distinct
Variable $x_i$ has type $A_i$.

Judgements are formulated under the Assumptions of a particular
Context, $\Gamma$:
\[
  \Gamma \vdash a : A
\]
For an empty Context:
\[
  \vdash a : A
\]
or
\[
  . \vdash a : A
\]



% ====================================================================
\section{Type}\label{sec:type}
% ====================================================================

% --------------------------------------------------------------------
\subsection{Type Constant}\label{sec:type_constant}
% --------------------------------------------------------------------

\emph{Type Constant} (or \emph{Base Type} or \emph{Atomic Type})



% --------------------------------------------------------------------
\subsection{Type Order}\label{sec:type_order}
% --------------------------------------------------------------------

\emph{Order}

In Simply-typed $\lambda$-calculus ($\lambda^\rightarrow$), the Order
of a Type $\tau$, denoted $o(\tau)$, is defined Inductively as:
\begin{itemize}
\item $o(T) = 0$ if $T$ is a Base Type
\item $o(\sigma \rightarrow \tau) = \text{max}(o(\sigma) + 1,
  o(\tau))$
\end{itemize}



% --------------------------------------------------------------------
\subsection{Inhabited Type}\label{sec:inhabited_type}
% --------------------------------------------------------------------

\emph{Inhabited}

In Simply-typed $\lambda$-calculus (\S\ref{sec:simply_typed}), a Type
is Inhabited if and only if its corresponding Proposition is a
Tautology of Minimal Implicative Logic (\S\ref{sec:minimal_logic}).

In Second-order $\lambda$-calculus (\S\ref{sec:system_f}), a
Type is Inhabited if and only if its corresponding Proposition is a
Tautology of Second-order Logic (\S\ref{sec:secondorder_logic}).



% --------------------------------------------------------------------
\subsection{Equality Type}\label{sec:equality_type}
% --------------------------------------------------------------------

The \emph{Equality Type} (or \emph{Identity Type}), denoted
\emph{Propositional Equality}, represents Equality of Types and Terms.



% --------------------------------------------------------------------
\subsection{Recursive Type}\label{sec:recursive_type}
% --------------------------------------------------------------------

% --------------------------------------------------------------------
\subsection{Universe Type}\label{sec:universe_type}
% --------------------------------------------------------------------

A \emph{Universe Type} contains all other Types, see Type Universe
(\S\ref{sec:type_universe}).



% --------------------------------------------------------------------
\subsection{Inductive Type}\label{sec:inductive_type}
% --------------------------------------------------------------------

Type Constructor (\S\ref{sec:type_operator})

Structural Recursion (\S\ref{sec:structural_recursion})

\emph{Coinductive Type}, Coinduction (\S\ref{sec:coinduction})

\emph{Induction Induction}

\emph{Induction Recursion}



% --------------------------------------------------------------------
\subsection{Polymorphic Type}\label{sec:polymorphic_type}
% --------------------------------------------------------------------

Second-order $\lambda$-calculus (\S\ref{sec:system_f})

Subtyping (\S\ref{sec:subtype})

Function Overloading

Parametric Polymorphism

Variance

Bounded Quantification

Parametricity



\subsubsection{Subtype}\label{sec:subtype}

System $F_<$ (System $F$ with Subtyping)



% --------------------------------------------------------------------
\subsection{Type Operator}\label{sec:type_operator}
% --------------------------------------------------------------------

Adding additional Type Operators to Simply-typed Lambda Calculus
results in \emph{Simply-typed Lambda Calculus with Type Operators}
``$\lambda \underline{\omega}$''

System F$_{\omega}$



\subsubsection{Higher-order Type Operator}
\label{sec:higherorder_typeoperator}



\subsubsection{Type Constructor}\label{sec:type_constructor}

\emph{Type Constructor}



\paragraph{Kind}\label{sec:kind}
\hfill \\

A \emph{Kind} is the Type of a Type Constructor, or the Type of a
Higher-order Type Operator \S\ref{sec:higherorder_typeoperator}).



\paragraph{Function Type}\label{sec:function_type}
\hfill \\

A \emph{Function Type} (or \emph{Arrow Type}) is a Higher-kinded



% --------------------------------------------------------------------
\subsection{Dependent Type}\label{sec:dependent_type}
% --------------------------------------------------------------------

A \emph{Dependent Type} depends on a Term or another Type.

Logical Quantifiers (\S\ref{sec:quantifier})

Intuitionistic Type Theory (\S\ref{sec:intuitionistic_type})

A \emph{Dependent Type} is the Type of a Dependent Function
(\S\ref{sec:dependent_function}).

Dependent Pairs



\subsubsection{Dependent Function}\label{sec:dependent_function}

For a Type $A : \mathcal{U}$ in Type Universe $\mathcal{U}$, a Family
of Types $B : A \rightarrow \mathcal{U}$ can be defined by a
\emph{Dependent Function} which assigns a Type $B(a) : \mathcal{U}$ to
Each Term $a : A$.



\subsubsection{Dependent Sum Type}\label{sec:dependent_sum}

\emph{$\Sigma$-type}



% ====================================================================
\section{Type Universe}\label{sec:type_universe}
% ====================================================================

A \emph{Type Universe} is a Type whose Elements are Types. A
hierarchy:
\[
  \mathcal{U}_0, \mathcal{U}_1, \mathcal{U}_2, \ldots
\]
is such that any Type in $\mathcal{U}_i$ is also in
$\mathcal{U}_{i+1}$.

See $\mathcal{U}-INTRO$ and $\mathcal{U}-CUMUL$
\S\ref{sec:homotopy_rules}.



% --------------------------------------------------------------------
\subsection{Girard's Paradox}\label{sec:girards_paradox}
% --------------------------------------------------------------------



% ====================================================================
\section{Typing Environment}\label{sec:typing_environment}
% ====================================================================

A \emph{Typing Environment} (or \emph{Context} or \emph{Variable
  Assignment}) $\Gamma$ is a Set of Typing Assumptions.

Closed Terms are those Terms that are Typable
(\S\ref{sec:typing_derivation}) in an Empty Context, e.g. the I, S,
and K Combinators (\S\ref{sec:combinator}).



% --------------------------------------------------------------------
\subsection{Typing Assumption}\label{sec:typing_assumption}
% --------------------------------------------------------------------

\emph{Typing Assumption})

\[
  e : \tau
\]
``Term $e$ has Type $\tau$''



% --------------------------------------------------------------------
\subsection{Type Annotation}\label{sec:type_annotation}
% --------------------------------------------------------------------

% --------------------------------------------------------------------
\subsection{Type Erasure}\label{sec:type_erasure}
% --------------------------------------------------------------------

% --------------------------------------------------------------------
\subsection{Reification}\label{sec:reification}
% --------------------------------------------------------------------

* opposite of Type Erasure
* see Type Inference (\S\ref{sec:type_inference})



% ====================================================================
\section{Typing Judgement}\label{sec:typing_judgement}
% ====================================================================

A \emph{Typing Judgement} is an instance of a \emph{Typing Relation}
(\S\ref{sec:typing_relation}).

A Typing Judgement expresses Propositional Equality. % FIXME

The Validity of a Typing Judgement is given by a Typing Derivation
(\S\ref{sec:typing_derivation}).



% --------------------------------------------------------------------
\subsection{Typing Relation}\label{sec:typing_relation}
% --------------------------------------------------------------------

\emph{Typing Relation} between Terms and Types

\[
  \Gamma \vdash e : \tau
\]
``$e$ is a (Well-typed \S\ref{sec:well_typed}) Term of Type $\tau$ in
Context $\Gamma$'',

Terms that are Well-typed in the Empty Context are Closed Terms.



% --------------------------------------------------------------------
\subsection{Typing Rule}\label{sec:typing_rule}
% --------------------------------------------------------------------

A \emph{Typing Rule} (or \emph{Type Rule}) describes how a Type is
assigned to a Syntactic Construction.

The notation for Type Rules is that of Sequent Notation
(\S\ref{sec:sequent}):
\[
  {
    \frac
    { \Gamma_1 \vdash e_1:\tau_1 \quad \cdots
      \quad \Gamma_n \vdash e_n:\tau_n }
    { \Gamma \vdash e:\tau }
  }
\]
where $\Gamma$ are Typing Environments and $e:\tau$ are Typing
Judgements.



% --------------------------------------------------------------------
\subsection{Typing Derivation}\label{sec:typing_derivation}
% --------------------------------------------------------------------

A \emph{Typing Derivation} shows the Validity of a Typing Judgement
and is constructed from Typing Rules (\S\ref{sec:typing_rule}).

A Term for which a Typing Derivation exists is called a
\emph{Well-typed Term} (\S\ref{sec:well_typed}).



% ====================================================================
\section{Type Inference}\label{sec:type_inference}
% ====================================================================



% ====================================================================
\section{Typed $\lambda$-Calculus}\label{sec:typed_lambda}
% ====================================================================

\emph{Typed $\lambda$-Calculus} allows for Lambda Terms to be assigned
Types (\S\ref{sec:type}).

Simply-typed $\lambda$-calculus (\S\ref{sec:simply_typed}) has only
one Type Constructor (\S\ref{sec:type_constructor}), $\rightarrow$
(see Function Type \S\ref{sec:function_type}).

Untyped $\lambda$-calculus (\S\ref{sec:untyped_lambda}) may be
considered a Typed $\lambda$-calculus with only one Type.



% --------------------------------------------------------------------
\subsection{Simply-typed $\lambda$-calculus}\label{sec:simply_typed}
% --------------------------------------------------------------------

\emph{Simply-typed $\lambda$-calculus} ($\lambda^\rightarrow$) is a
Typed Interpretation of $\lambda$-calculus with one Type Constructor
for Function Types: $\rightarrow$ (\S\ref{sec:function_type}).

Strongly Normalizing, not Turing Complete: addition of $fix_\alpha$
Operator or Recursive Types (\S\ref{sec:recursive_type}) gives Turing
Completeness

``Simply-typed'' is to distinguish from extensions such as:
\begin{itemize}
  \item System F (\S\ref{sec:system_f}): Polymorphic Types
  \item Logical Framework (\S\ref{sec:logical_framework}): Dependent
    Types
\end{itemize}
because Polymorphism and Dependency cannot be encoding using only
$\rightarrow$ and Type Variables.

The following extensions are still considered ``Simply-typed''
systems:
\begin{itemize}
  \item System T (\S\ref{sec:system_t}): Products, Coproducts, Natural
    Numbers
  \item PCF (\S\ref{sec:pcf}): Full Recursion
\end{itemize}

Simply-typed $\lambda$-calculus has the same Equational Theory of
$\beta\eta$-equivalence (\S\ref{sec:beta_reduction},
\S\ref{sec:eta_conversion}) as Untyped $\lambda$-calculus, subject to
Type restrictions.

$\beta$-reduction:
\[
  (\lambda x:s.t)u =_\beta t[x := u]
\]
holds in Context $\Gamma$ when $\Gamma$, $x:\sigma \rightarrow t:\tau$
and $\Gamma \vdash u:\sigma$.

$\eta$-reduction:
\[
  \lambda x:\sigma .t x =_\eta t
\]
holds when $\Gamma \vdash t : \sigma \rightarrow \tau$ and $x$ is not
Free in $t$.

With the Function Type Constructor $\rightarrow$ and the Base Types
$B$, the Types of a Simply-typed $\lambda$-calculus are defined (with
BNF Notation):
\[
  \tau ::= \tau \rightarrow \tau \;|\; T \in B
\]

\emph{Term Constant}

That is, Expressions of Simply-typed $\lambda$-calculus are defined
as:
\[
  e ::= x \;|\; \lambda x:t.e \;|\; e e \;|\; c
\]
where $x$ is a Variable and $\tau$ is a Type, and $c$ is a Term
Constant. These Expressions are, in order:
\begin{itemize}
  \item Variable Reference
  \item Abstraction
  \item Application
  \item Constant
\end{itemize}



\textbf{Intrinsic Interpretation}: \emph{Church-style}

* Only Well-typed Terms have Meaning (Meaning assigned to Typing
Derivations), therefore equivalent Terms having different Annotations
may have different Meanings, cf. Reification
(\S\ref{sec:reification}):



\textbf{Extrinsic Interpretation}: \emph{Curry-style}

* Terms Interpreted as in an Untyped Language, cf. Type Erasure
(\S\ref{sec:type_erasure})



\subsubsection{Simply-typed $\lambda$-calculus Typing Rules}
\label{sec:simplytyped_rules}

Typing Rules (\S\ref{sec:typing_rule}):
\begin{enumerate}
\item
  \[
    {
      \frac
      {x : \tau \in \Gamma}
      {\Gamma \vdash x : \tau}
    }
  \]
\item
  \[
    {
      \frac
      {c \;\text{is a constant of Base Type}\; T}
      {\Gamma \vdash c:T}
    }
  \]
\item
  \[
    {
      \frac
      {\Gamma, x:\sigma \vdash e:\tau}
      {\Gamma \vdash (\lambda x:\sigma.e):(\sigma \rightarrow \tau)}
    }
  \]
\item
  \[
    {
      \frac
      {\Gamma \vdash e_1:\sigma \rightarrow \tau \quad
        \Gamma \vdash e_2:\sigma}
      {\Gamma \vdash e_1 e_2 : \tau}
    }
  \]
\end{enumerate}



\textbf{Operational Semantics}

\textbf{Categorical Semantics}

Simply-typed $\lambda$-calculus is the Internal Language
(\S\ref{sec:internal_language}) of the Cartesian Closed Categories
(\S\ref{sec:cartesian_closed}).



\textbf{Proof-theoretic Semantics}

Simply-typed $\lambda$-calculus is Isomorphic by Curry-Howard
(\S\ref{sec:curry_howard}) to Minimal Logic
(\S\ref{sec:minimal_logic}):
\begin{itemize}
  \item Terms correspond to Proofs in Natural Deduction
    (\S\ref{sec:natural_deduction})
  \item (Inhabited) Types correspond to Tautologies in Minimal Logic
\end{itemize}

Minimal Logic $\leftrightarrow$ Simply Typed $\lambda$-Calculus:
\[
  \supset \leftrightarrow \rightarrow
\] \[
  \wedge \leftrightarrow \times
\] \[
  \vee \leftrightarrow +
\] \[
  False \leftrightarrow \bot
\]



\subsubsection{Simply-typed $\lambda$-calculus with Conjunctive Types}
\label{sec:simplytyped_conjunctive}

Subtyping (\S\ref{sec:subtype})

System $F_<$



\subsubsection{System T}\label{sec:system_t}

G\"odel

Proof Interpretation of Heyting Arithmetic into a Finite-type
Extension of Primitive Recursive Arithmetic
(\S\ref{sec:primitive_recursion})

All Recursive Functions in Peano Arithmetic are definable



\subsubsection{PCF}\label{sec:pcf}



% --------------------------------------------------------------------
\subsection{System F}\label{sec:system_f}
% --------------------------------------------------------------------

\emph{System F} (or \emph{Second-order $\lambda$-calculus})

Universal Quantification over all Types

Can describe all Functions that are Provably Total in Second-order
Logic (\S\ref{sec:secondorder_logic})



% --------------------------------------------------------------------
\subsection{Dependently-typed $\lambda$-Calculus}
\label{sec:dependently_typed}
% --------------------------------------------------------------------

Dependent Types (\S\ref{sec:dependent_type})

Dependently-typed $\lambda$-calculus is base of Intuitionistic Type
Theory (\S\ref{sec:intuitionistic_type}), Calculus of Constructions
(\S\ref{sec:coq}), and Logical Framework
(\S\ref{sec:logical_framework}).

Dependently-typed $\lambda$-calculus with a Type of all Types (the
simplest Pure Type System \S\ref{sec:pure_type_system}) is not
Strongly Normalizing due to Girard's Paradox
(\S\ref{sec:girards_paradox})



\subsubsection{Logical Framework}\label{sec:logical_framework}



% --------------------------------------------------------------------
\subsection{$\kappa$-calculus}\label{sec:kappa_calculus}
% --------------------------------------------------------------------

First-order fragment of Typed $\lambda$-calculus



% ====================================================================
\section{Intuitionistic Type Theory}\label{sec:intuitionistic_type}
% ====================================================================

\emph{Intuitionistic Type Theory} (also \emph{Constructive Type
  Theory} or \emph{Martin-L\"of Type Theory})

Inductive Types



% --------------------------------------------------------------------
\subsection{Constructive Type}\label{sec:constructive_type}
% --------------------------------------------------------------------

\emph{Function Type}

\emph{Higher-order Function}

\emph{Product Type}

\subsubsection{Empty Type $\mathbf{0}$}

\subsubsection{Unit Type $\mathbf{1}$}

\subsubsection{Natural Number Type}

\subsubsection{Identity Type}

\subsubsection{Dependent Function Type ($\Pi$-type)}

\[
  A \rightarrow B :\equiv \prod_{x:A} B
\]

\subsubsection{Dependent Pair Type ($\Sigma$-type)}

In Set Theory: Indexed Sum over a given Type.

\subsubsection{Coproduct Type}



% ====================================================================
\section{Calculus of Constructions}\label{sec:coq}
% ====================================================================



% ====================================================================
\section{Homotopy Type Theory}\label{sec:homotopy_type}
% ====================================================================

Homotopy Thoery (\S\ref{sec:homotopy_theory})

- Function Extensionality
- Univalence Axiom

- Values (Points), Paths, Homotopies


% --------------------------------------------------------------------
\subsection{Higher Inductive Type}\label{sec:higher_inductive_type}
% --------------------------------------------------------------------



% --------------------------------------------------------------------
\subsection{Inference Rules}\label{sec:homotopy_rules}
% --------------------------------------------------------------------

Inference Rules (\S\ref{sec:type_inference}) have the form:
\[
  \frac{J_1 \quad \cdots \quad J_k} {J} Name
\]
where $J_i$ are provided as derived hypothetical (metatheoretical)
Judgements and $J$ is the conclusion.

A Tree constructed from Inference Rules forms a Derivation
(\S\ref{sec:typing_derivation}) of a Judgement.



\textbf{Context Rules}

The following Rules of Inference allow for the determination of a
Well-formed Context:
\begin{enumerate}
\item
\[
  {
    \frac{}{(.)ctx}
  } ctx-EMP
\]
\item
\[
  {
    \frac
    {x_1:A_1, \ldots, x_{n-1}:A_{n-1} \vdash A_n : \mathcal{U}_i}
    {(x_1:A_1,\ldots,x_n:A_n) ctx}
  } ctx-EXT
\]
\end{enumerate}



\textbf{Structural Rules}

Given a Context, derive Typing Judgements
(\S\ref{sec:typing_judgement}) listed in the Context:
\[
  {
    \frac
    {(x_1:A_1, \ldots, x_n:A_n)ctx}
    {x_1:A_1, \ldots, x_n:A_n \vdash x_i:A_i}
  } Vble
\]

Substitution for Typing Judgements:
\[
  {
    \frac
    {\Gamma \vdash a : A \;\;\;\;\;\;\;
    \Gamma,x:A,\Delta \vdash b : B}
    {\Gamma,\Delta[a/x] \vdash b[a/x] : B[a/x]}
  } Subst_1
\]

Weakening for Typing Judgements:
\[
  {
    \frac
    {\Gamma \vdash A : \mathcal{U}_i \;\;\;\;\;\;\;
    \Gamma,\Delta \vdash b : B}
    {\Gamma,x:A,\Delta \vdash b:B}
  } Wkg_1
\]

Substitution for Judgemental Equality:
\[
  {
    \frac
    {\Gamma \vdash a : A \;\;\;\;\;\;\;
    \Gamma,x:A,\Delta \vdash b \equiv c : B}
    {\Gamma,\Delta[a/x] \vdash b[a/x] \equiv c[a/x] : B[a/x]}
  } Subst_2
\]

Weakening for Judgemental Equality:
\[
  {
    \frac
    {\Gamma \vdash A : \mathcal{U}_i \;\;\;\;\;\;\;
    \Gamma,\Delta \vdash b \equiv c : B}
    {\Gamma, x:A, \Delta \vdash b \equiv c : B}
  } Wkg_2
\]



\textbf{Universe Rules}

\[
  {
    \frac
    {(\Gamma) ctx}
    {\Gamma \vdash \mathcal{U}_i : \mathcal{U}_{i+1}}
  } \mathcal{U}-INTRO
\]

\[
  {
    \frac
    {\Gamma \vdash A : \mathcal{U}_i}
    {\Gamma \vdash A : \mathcal{U}_{i+1}}
  } \mathcal{U}-CUMUL
\]



\textbf{Dependent Function Type Rules}

\[
  {
    \frac
    {\Gamma \vdash A : \mathcal{U}_i \;\;\;\;\;\;\;
    \Gamma,x:A \vdash B : \mathcal{U}_i}
    {\Gamma \vdash \prod_{(x:A)} B : \mathcal{U}_i}
  } \Pi-FORM
\]\[
  {
    \frac
    {}
    {}
  } \Pi-INTRO
\]\[
  {
    \frac
    {}
    {}
  } \Pi-ELIM
\]\[
  {
    \frac
    {}
    {}
  } \Pi-COMP
\]\[
  {
    \frac
    {}
    {}
  } \Pi-UNIQ
\]



% ====================================================================
\section{Pure Type System}\label{sec:pure_type_system}
% ====================================================================

Arbitrary number of Sorts and Dependencies

Generalization of the \emph{Lambda Cube} to more sorts than Terms and
Types

Not necessarily Strongly Normalizing (\S\ref{sec:normalization})

Barendregt-Geuvers-Klop Conjecture



% --------------------------------------------------------------------
\subsection{Lambda Cube}\label{sec:lambda_cube}
% --------------------------------------------------------------------

Simply-typed $\lambda$-calculus allows only Terms to depend on Types.

All Strongly Normalizing



% ====================================================================
\section{Curry-Howard Correspondence}\label{sec:curry_howard}
% ====================================================================

\begin{tabular}{| l | l |}
\hline
\textbf{Type Theory} & \textbf{Logic} \\ \hline \hline
$A$ (Type) & Proposition \\ \hline
$a : A$ (Term) & Proof \\
\hline
\end{tabular}

Natural Deduction - Typed Lambda Calculus

Minimal Logic - Simple Types (Simply-typed $\lambda$-calculus
\S\ref{sec:simply_typed})

Predicate Logic - Dependent Types (Intuitionistic Type Theory
\S\ref{sec:intuitionistic_type})

Modal Logic - Monads

Classical-Intuitionistic Embedding - Continuation Passing Style



% ====================================================================
\section{Type System}\label{sec:type_system}
% ====================================================================

% --------------------------------------------------------------------
\subsection{Hindley-Milner Type System}\label{sec:hindley_milner}
% --------------------------------------------------------------------

% --------------------------------------------------------------------
\subsection{Type Checking}\label{sec:type_checking}
% --------------------------------------------------------------------

\subsubsection{Bi-directional Type Checking}
\label{sec:bidirectional_checking}

\emph{Checking}

\emph{Synthesis}
