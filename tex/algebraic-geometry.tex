%%%%%%%%%%%%%%%%%%%%%%%%%%%%%%%%%%%%%%%%%%%%%%%%%%%%%%%%%%%%%%%%%%%%%%
%%%%%%%%%%%%%%%%%%%%%%%%%%%%%%%%%%%%%%%%%%%%%%%%%%%%%%%%%%%%%%%%%%%%%%
\part{Algebraic Geometry}\label{part:algebraic_geometry}
%%%%%%%%%%%%%%%%%%%%%%%%%%%%%%%%%%%%%%%%%%%%%%%%%%%%%%%%%%%%%%%%%%%%%%
%%%%%%%%%%%%%%%%%%%%%%%%%%%%%%%%%%%%%%%%%%%%%%%%%%%%%%%%%%%%%%%%%%%%%%

\emph{Valuation}

\emph{Hilbert's Nullstellensatz}

\emph{Cauchy's Inequality}:
\[
    |\langle x,y \rangle|^2 \leq \langle x,x \rangle \cdot \langle
    y,y \rangle
\]

Study of Geometry using Commutative Rings
(\S\ref{sec:commutative_ring})

One-dimensional Spaces:

\begin{itemize}
\item a Field $k$ is a One-dimensional Vector Space over itself
\item the Projective Line over a Field $k$
\item the Complex Projective Line (Riemann Sphere) over the Field of Complex
  Numbers
\end{itemize}



% ====================================================================
\section{Polynomial}\label{sec:polynomial}
% ====================================================================

A \emph{Polynomial Expression} is an Expression containing Variables (or
\emph{Indeterminates}) and Coefficients with only Arithmetic Operations
(\S\ref{sec:arithmetic}) and Positive Integer Exponents
(\S\ref{sec:exponentiation}) where the \emph{Degree} (\S\ref{sec:degree}) of
the Polynomial is the largest such Exponent of any one Term with Nonzero
Coefficient.

\fist Algebraic Expressions (\S\ref{sec:algebraic_expression}) are a wider
class of Expressions that allow for Negative and Rational Exponents ($n$-th
Roots).

\fist Generalized to Formal Power Series (\S\ref{sec:formal_power_series}).

\fist Homogeneous Polynomial (Projective Geometry
\S\ref{sec:homogeneous_polynomial})

The Funadmental Theorem of Algebra (\S\ref{sec:fundamental_algebra_theorem})
states that every Polynomial of Degree $n$ has $n$ Complex Roots
(\S\ref{sec:function_root}), counted with their Multiplicities (??? FIXME).

a Complex Number (\S\ref{sec:complex_number}) can be defined as a Polynomial in
the single Indeterminate with the Relation $i^2 + 1 = 0$ imposed, and Complex
Numbers can then be Added or Multiplied using Addition and Multiplication of
Polynomials

an Algebraically Closed Field (\S\ref{sec:algebraically_closed}) $F$ contains a
Root (\S\ref{sec:root}) for every Non-constant Polynomial in the Ring of
Polynomials (\S\ref{sec:polynomial_ring}) $F[x]$ in the Variable $x$ with
Coefficients in $F$

the Set of Complex Numbers is the Quotient Ring of the Polynomial Ring in the
Indeterminate $i$ by the Ideal generated by the Polynomial $i^2 + 1$

any Polynomial with Complex Coefficients (\S\ref{sec:complex_number}) has
Complex Roots (i.e. the Complex Numbers are \emph{Algebraically Closed})

Univariate Polynomial (\S\ref{sec:univariate_polynomial}) of Degree
$n$ with Single Indeterminate $x$ and Coefficients $a_0, \ldots, a_n$
from a Ring:
\[
  a_n x^n + a_{n-1} x^{n-1} + \ldots + a_2 x^2 + a_1 x + a_0
\]
or:
\[
  \sum_{i=0}^n a_i x^i
\]

Continuous (\S\ref{sec:continuous_function})

Differentiable (\S\ref{sec:differentiable_function}) everywhere

Integer Coefficient Polynomial (\S\ref{sec:integer_coefficient})

Categorification: Polynomial Functor (\S\ref{sec:polynomial_functor})

testing Polynomial Equality:
\url{https://jeremykun.com/2017/04/24/testing-polynomial-equality/}

\fist Polynomial Sequences (Combinatorics \S\ref{sec:polynomial_sequence})

$R[x]$ -- Ring of Polynomials (\S\ref{sec:polynomial_ring}) in the
Indeterminate $x$ over the Ring $R$

a Linear Differential Equation (\S\ref{sec:linear_differential_equation}) is an
Ordinary Differential Equation that is defined by a Linear (i.e. Degree $1$)
Polynomial of the Unknown Function and its Derivatives



% --------------------------------------------------------------------
\subsection{Polynomial Expression}\label{sec:polynomial_expression}
% --------------------------------------------------------------------

\begin{itemize}
  \item number of Terms -- Monomial (\S\ref{sec:monomial}), Binomial
    (\S\ref{sec:binomial}), etc.
  \item Degree (\S\ref{sec:degree}) -- highest Degree of its individual Terms
    (Monomials)
  \item number of Indeterminates (\S\ref{sec:indeterminate}) -- Univariate
    (\S\ref{sec:univariate_polynomial}), Multivariate
    (\S\ref{sec:multivariate_polynomial})
\end{itemize}



\subsubsection{Indeterminate}\label{sec:indeterminate}

when a Polynomial Expression is considered as a Polynomial Function
(\S\ref{sec:polynomial_function}), the Indeterminates become ``Variables''



\subsubsection{Degree}\label{sec:degree}

highest Degree of its Monomials

\begin{itemize}
  \item Degree 1 -- Linear
  \item Degree 2 -- Quadratic
  \item ...
\end{itemize}



\subsubsection{Discriminant}\label{sec:discriminant}

\subsubsection{Monomial}\label{sec:monomial}

or \emph{Power Product}

a Polyonomial of a single Term

a Product of Powers of Variables with Non-negative Integer Exponents

$x^a y^b z^c \cdots$

the \emph{Degree} of a Monomial is the Sum of all the Exponents of the
Variables



\subsubsection{Binomial}\label{sec:binomial}

A \emph{Binomial} is a Polynomial which is the Sum of two Monomials
(\S\ref{sec:monomial})

Binomial Coefficient (\S\ref{sec:binomial_coefficient})



\paragraph{Univariate Binomial}\label{sec:univariate_binomial}\hfill

\[
  a x^n - b x^m
\]
% FIXME


\paragraph{Binomial Theorem}\label{sec:binomial_theorem}\hfill



\subsubsection{Univariate Polynomial}\label{sec:univariate_polynomial}

Polynomial with a single Indeterminate



\subsubsection{Multivariate Polynomial}\label{sec:multivariate_polynomial}

Polynomial with more than one Indeterminate



\paragraph{Bivariate Polynomial}\label{sec:bivariate_polynomial}\hfill

Polynomial with two Indeterminates



% --------------------------------------------------------------------
\subsection{Polynomial Arithmetic}\label{sec:polynomial_arithmetic}
% --------------------------------------------------------------------

Sum $P + Q$

Product $P Q$

Composition

Derivative

Antiderivative



\subsubsection{Polynomial Division}\label{sec:polynomial_division}

Euclidean Division

Polynomial Long Division



% --------------------------------------------------------------------
\subsection{Polynomial Function}\label{sec:polynomial_function}
% --------------------------------------------------------------------

Function defined by Evaluating a Polynomial Expression
(\S\ref{sec:polynomial_expression})

the Variables of a Polynomial Function are the Indeterminates
(\S\ref{sec:indeterminate}) of the defining Polynomial Expression

Degree $0$ or $1$ is a \emph{Linear Function} (\S\ref{sec:linear_function})

the Harmonic Functions (\S\ref{sec:harmonic_function}) on $\reals$ are
exactly the Linear Functions



\subsubsection{Linear Function}\label{sec:linear_function}

Polynomial of Degree $0$ or $1$



\subsubsection{Power Function}\label{sec:power_function}

%FIXME: move this section ?

\paragraph{Even Function}\label{sec:even_function}\hfill

Even Parity (\S\ref{sec:parity}) of the Exponent



\paragraph{Odd Function}\label{sec:odd_function}\hfill

Odd Parity (\S\ref{sec:parity}) of the Exponent



\subsubsection{Weierstrass Approximation Theorem}
\label{sec:weierstrass_approximation}

any given Continuous Complex-Valued Function defined on a Closed Interval
$[a,b]$ can be Uniformly Approximated as closely as desired by a Polynomial
Function



% --------------------------------------------------------------------
\subsection{Symmetric Polynomial}\label{sec:symmetric_polynomial}
% --------------------------------------------------------------------

% --------------------------------------------------------------------
\subsection{Homogeneous Polynomial}\label{sec:homogeneous_polynomial}
% --------------------------------------------------------------------

Quadratic Form (\S\ref{sec:quadratic_form})

the Space of Symmetric Tensors (\S\ref{sec:symmetric_tensor}) of Order $r$ on a
Finite-dimensional Vector Space is naturally Isomorphic to the Dual of the
Space of Homogeneous Polynomials of Degree $r$ on $V$

a Homogeneous Linear Differential Equation
(\S\ref{sec:homogeneous_linear_differential}) is a Linear Differential
Equation with a $0$ Constant Term, i.e. it is a Homogeneous Polynomial in the
Unknown Function and its Derivatives

cf. Homogeneous Difference Equation
(\S\ref{sec:homogeneous_differential_equation}),
Homogeneous System of Linear Equations (\S\ref{sec:homogeneous_system})



% --------------------------------------------------------------------
\subsection{Irreducible Polynomial}\label{sec:irreducible_polynomial}
% --------------------------------------------------------------------

% --------------------------------------------------------------------
\subsection{Taylor Polynomial}\label{sec:taylor_polynomial}
% --------------------------------------------------------------------

approximating a Smooth Function by a Polynomial with Terms from the
Taylor Series (\S\ref{sec:taylor_series}) of that Function

(wiki:)

``Linear Taylor Polynomials'' in the definition of Cotangent Spaces
(\S\ref{sec:cotangent_space}) -- two Smooth Functions are considered
``Equivalent'' at a Point $x$ if they have the same ``\emph{First-order
  Behavior}'' near $x$, analogous to their Linear Taylor Polynomials,
where First-order Behavior is defined as Equivalent if and only if the
Derivative of the Function $f-g$ \emph{vanishes} at $x$ --the Cotangent Space
then consists of all the possible ``\emph{First-order Behaviors}'' of a
Function near $x$



% --------------------------------------------------------------------
\subsection{Formal Power Series}\label{sec:formal_power_series}
% --------------------------------------------------------------------

\fist Power Series (\S\ref{sec:power_series})

\fist \emph{Perturbation Series} (Perturbation Methods
\S\ref{sec:perturbation_method}): Formal Power Series in some ``small''
Parameter quantifying the deviation from the exactly Solvable problem



% --------------------------------------------------------------------
\subsection{Polynomial Equation}\label{sec:polynomial_equation}
% --------------------------------------------------------------------

A \emph{Polynomial Equation} is an Equation of the form:
\[
  P = Q
\]
where $P$ and $Q$ are \emph{Polynomials} (\S\ref{sec:polynomial}) with
Coefficients in some Field (\S\ref{sec:field}).

\fist Polynomial Equations are sometimes called \emph{Algebraic Equations} when
they involve a Single Variable, but should not be confused with Algebraic
Expressions (\S\ref{sec:algebraic_expression}).

\fist System of Differential-Algebraic Equations (\S\ref{sec:system_of_daes})

The \emph{Solution} to a Polynomial Equation is a Value or Set of Values; cf.
the Solution to a Differential Equation (\S\ref{sec:differential_equation}) is
a Function or Class of Functions.

Polynomial Equations by Degree:
\begin{itemize}
  \item Degree 1 -- Linear Equation (\S\ref{sec:linear_equation})
  \item Degree 2 -- Quadratic Equation (\S\ref{sec:quadratic_equation})
  \item ...
\end{itemize}

Linear Equation (\S\ref{sec:linear_equation})

cf. Functional Equation --
\url{https://golem.ph.utexas.edu/category/2017/04/functional_equations_entropy_a.html},
Implicit Function %FIXME


\emph{Tarski-Seidenberg Theorem}\S\ref{sec:tarski_seidenberg}

the First-order Theory of the Real Closed Field (\S\ref{sec:real_closed}) is
Decidable: every Formula constructed from Polynomial Equations and Inequalities
by the Logical Connectives $\vee$, $\wedge$, $\neg$, and Quantifiers $\forall$,
$\exists$ is equivalent to a Formula without Quantifiers



\subsubsection{Root}\label{sec:equation_root}

\fist Root (Functions \S\ref{sec:function_root})

An Algebraic Function (\S\ref{sec:algebraic_function}) is a Function that can
be defined as the Root of a Polynomial Equation.

the Golden Ration $\varphi$ is the Root of the Polynomial $x^2 - x - 1$



\subsubsection{Diophantine Equation}\label{sec:diophantine_equation}

a Polynomial Equation of two or more Unknowns for which only the Integer
Solutions (where all Unknowns take Integer Values) are sought

\fist Number Theory

\emph{Hilbert's Tenth Problem}: there is no general Algorithm for Deciding
whether for any given Diophantine Equation there exists a Solution with all
Unknowns taking Integer values



\paragraph{Linear Diophantine Equation}
\label{sec:linear_diophantine}\hfill

Monomial (\S\ref{sec:monomial}) Equations with Integer Solutions

$c = ax + by$ has a solution in $\mathbb{Z}$ when $gcd(a,b)|c$. Given
$d=gcd(a,b)|c$ and a solution $(x_0, y_0)$, then all solutions are of
the form $(x_0 + \frac{b}{d}t, y_0 - \frac{a}{d}t)$ with $t \in
\mathbb{Z}$.



\paragraph{Exponential Diophantine Equation}\hfill
\label{sec:exponential_diophantine}



\subsubsection{Linear Equation}\label{sec:linear_equation}

System of Linear Equations (\S\ref{sec:system_of_linear_equations})



\subsubsection{Quadratic Equation}\label{sec:quadratic_equation}



% --------------------------------------------------------------------
\subsection{Fundamental Theorem of Algebra}
\label{sec:fundamental_algebra_theorem}
% --------------------------------------------------------------------

states that the Field of Complex Numbers is Algebraically Closed
(\S\ref{sec:algebraically_closed})



% --------------------------------------------------------------------
\subsection{Tarski-Seidenberg Theorem}\label{sec:tarski_seidenberg}
% --------------------------------------------------------------------

the First-order Theory of the Real Closed Field (\S\ref{sec:real_closed}) is
Decidable: every Formula constructed from Polynomial Equations and Inequalities
by the Logical Connectives $\vee$, $\wedge$, $\neg$, and Quantifiers $\forall$,
$\exists$ is equivalent to a Formula without Quantifiers

Euclidean Geometry (\S\ref{sec:euclidean_geometry}) without the ability to
measure Angles is also a Model of the Real Field Axioms and therefore also
Decidable

Quantifier Elimination over the Reals is only possible in Double Exponential
Time (Collins Cylindrical Algebraic Decomposition
\S\ref{sec:cylindrical_algebraic_decomposition})

the Semialgebraic Sets in $\reals^n$ form an $o$-minimal Structure (TODO)



% ====================================================================
\section{Posynomial}\label{sec:posynomial}
% ====================================================================

Geometric Programming (\S\ref{sec:geometric_programming})



% ====================================================================
\section{Algebraic Expression}\label{sec:algebraic_expression}
% ====================================================================

An \emph{Algebraic Expression} is an Expression including Arithmetic
(\S\ref{sec:arithmetic}) and Algebraic (Negative and Rational Exponent
\S\ref{sec:exponential}) Operations.

Algebraic Expressions do not include Transcendental Numbers
(\S\ref{sec:transcendental_number}), Trigonometric, Logarithmic, or Hyperbolic
Functions, nor Irrational Exponents; these are allowed by the wider class of
Closed-form, Analytic (\S\ref{sec:analytic_expression}), and unrestricted
Mathematical Expressions (\S\ref{sec:mathematical_expression}).

\fist Polynomial Expressions (\S\ref{sec:polynomial_expression}) are a narrower
class of Expressions that do not allow Rational Exponents ($n$-th Roots) or
Negative Integer Exponents.



% --------------------------------------------------------------------
\subsection{Algebraic Fraction}\label{sec:algebraic_fraction}
% --------------------------------------------------------------------

% --------------------------------------------------------------------
\subsection{Algebraic Function}\label{sec:algebraic_function}
% --------------------------------------------------------------------

An \emph{Algebraic Function} is a Function that can be defined as the Root
(\S\ref{sec:equation_root}) of a Polynomial Equation
(\S\ref{sec:polynomial_equation}).

all Algebraic Functions are Holonomic Functions
(\S\ref{sec:holonomic_function})



% ====================================================================
\section{Closed-form Expression}\label{sec:closed_form_expression}
% ====================================================================

(Laczkovich03) refinement of Richardson's Theorem
(\S\ref{sec:richardsons_theorem}); the use of $\pi$ can be removed and the use
of Composition reduced:

given an Expression $A(x)$ in the Ring generated by the Integers, $x$, $\sin
x^n$, and $\sin(x \sin x^n)$, the question whether $A(x) > 0$ for some $x$ and
whether $A(x) = 0$ for some $x$ are \emph{Unsolvable}



% --------------------------------------------------------------------
\subsection{Elementary Function}\label{sec:elementary_function}
% --------------------------------------------------------------------

the Set of Elementary Functions is Closed under Arithmetic Operations and
Composition



% ====================================================================
\section{System of Equations}\label{sec:system_of_equations}
% ====================================================================

a Set of \emph{Simultaneous Equations}

a Finite Set of Equations (\S\ref{sec:equation}) for which a ``common
solution'' is desired (FIXME: clarify)

\begin{itemize}
  \item System of Polynomial Equations
    (\S\ref{sec:system_of_polynomial_equations})
  \begin{itemize}
    \item System of Linear Equations (\S\ref{sec:system_of_linear_equations})
  \end{itemize}
  \item System of Nonlinear Equations
    (\S\ref{sec:system_of_nonlinear_equations}) -- either Polynomials of Degree
    $>1$ or Unknowns appear as Variables of a general Function which is not a
    Polynomial of Degree $1$
\end{itemize}

%FIXME: move the following ?

the System of Equations:
\begin{align*}
  u & = u(x,y) \\
  v & = v(x,y)
\end{align*}
is \emph{Invertible} if $f = u + iv$ is Analytic and $f'(z) \neq 0$ \fist
Conformal Mapping (\S\ref{sec:conformal_map})



% --------------------------------------------------------------------
\subsection{Unknown}\label{sec:unknown}
% --------------------------------------------------------------------

Variables of the Equations



\subsubsection{Underdetermined System}\label{sec:underdetermined_system}

fewer Equations than Unknowns



\subsubsection{Overdetermined System}\label{sec:overdetermined_system}

more Equations than Unknowns

Ordinary Least Squares (OLS \S\ref{sec:ordinary_least_squares})



% --------------------------------------------------------------------
\subsection{System of Polynomial Equations}
\label{sec:system_of_polynomials}
% --------------------------------------------------------------------

Polynomial Equation (\S\ref{sec:polynomial_equation})

Algebraically Closed (\S\ref{sec:algebraically_closed}) Field
Extension (\S\ref{sec:field_extension})



\subsubsection{System of Linear Equations}
\label{sec:system_of_linear_equations}

\emph{Linear System (LS)}

Linear Equation (\S\ref{sec:linear_equation})

\fist Linear Transformations (\S\ref{sec:linear_transformation})

a System of Linear Equations of the form:
\begin{align*}
  a_{11}x_1 + \cdots + a_{1n}x_n & = b_1 \\
  a_{21}x_1 + \cdots + a_{2n}x_n & = b_2 \\
                                 & \vdots \\
  a_{m1}x_1 + \cdots + a_{mn}x_n & = b_m
\end{align*}
is equivalent to the Matrix Equation:
\[
  \mathbf{A}\vec{x} = \vec{b}
\]

Algorithms:
\begin{itemize}
  \item Gaussian Elimination (Row Reduction \S\ref{sec:gaussian_elimination})
\end{itemize}

Numerical Methods:
\begin{itemize}
  \item Gauss-Seidel Method (\S\ref{sec:gauss_seidel})
\end{itemize}



\paragraph{Coefficient Matrix}\label{sec:coefficient_matrix}\hfill

Augmented Matrix

Reduced Row Echelon Form (\S\ref{sec:reduced_row_echelon})



\paragraph{Homogeneous System of Linear Equations}\hfill
\label{sec:homogeneous_system}

cf. Homogeneous Polynomial (\S\ref{sec:homogeneous_polynomial}), Homogeneous
Differential Equation (\S\ref{sec:homogeneous_differential_equation}),
Homogeneous Difference Equation (\S\ref{sec:homogeneous_differential_equation})



\subsubsection{Cylindrical Algebraic Decomposition}
\label{sec:cylindrical_algebraic_decomposition}

Complexity: Double Exponential (\S\ref{sec:double_exponential})

Quantifier Elimination over Reals \fist Tarski-Seidenberg Theorem
(\S\ref{sec:tarski_seidenberg})



% --------------------------------------------------------------------
\subsection{System of Nonlinear Equations}
\label{sec:system_of_nonlinear_equations}
% --------------------------------------------------------------------

Non-Linear System (NLS)

a Set of Simultaneous Equations in which the Unknowns or Unknown Functions
appear as Variables of a Polynomial of Degree higher than $1$, \emph{or} in the
Argument of a Function which is not a Polynomial of Degree $1$; i.e. a
Nonlinear System of Equations to be solved cannot be written as a Linear
Combination of Unknown Variables or Functions that appear in them

\fist Nonlinear Programming (\S\ref{sec:nonlinear_programming})

\fist Iterative Methods (\S\ref{sec:iterative_method})



% --------------------------------------------------------------------
\subsection{System of Ordinary Differential Equations}
\label{sec:system_of_odes}
% --------------------------------------------------------------------

a number of ``coupled'' Differential Equations (\S\ref{sec:ode})

\fist Dynamical Systems (\S\ref{sec:dynamical_system})

\fist the Jacobian Matrix (\S\ref{sec:jacobian_matrix}) for a System of DAEs
(\S\ref{sec:system_of_daes}) is a Singular (Non-invertible) Matrix
(\S\ref{sec:singular_matrix})



\subsubsection{System of Linear Differential Equations}
\label{sec:system_of_linear_odes}

a Holonomic Function (\S\ref{sec:holonomic_function}) is a Multivariable Smooth
Function that is a solution to a (System of) Homogeneous Linear Differential
Equation(s) with Polynomial Coefficients



% --------------------------------------------------------------------
\subsection{System of Differential-Algebraic Equations}
\label{sec:system_of_daes}
% --------------------------------------------------------------------

(wiki):

a System of Equations that either contains Differential Equations
(\S\ref{sec:differential_equation}) and Algebraic (Polynomial) Equations
(\S\ref{sec:polynomial_equation}), or is equivalent to such a System

\fist Constraints of Multibody Systems (\S\ref{sec:multibody_system})

general form of System of Differential equations for Vector-valued Functions
$x$ in one Independent Variable $t$:
\[
  \vec{f}(\vec{x}'(t), \vec{x}(t), t) = 0
\]
where $\vec{x} : [a,b] \rightarrow \reals^n$ is a Vector of $n$ Dependent
Variables:
\[
  \vec{x}(t) = (x_1(t), \ldots, x_n(t))
\]
and the System has $n$ Equations:
\[
  \vec{f} = (\vec{f}_1, \ldots, \vec{f}_n) : \reals^{2n+1} \rightarrow \reals^n
\]

Solution consists of a search for Consistent Initial Values and a computed
Trajectory (\S\ref{sec:trajectory})

a DAE is not completely Solvable for the Derivatives of \emph{all} Components
of the Function $\vec{x}$ because they may not \emph{appear}, i.e. \emph{some}
Equations are \emph{Algebraic} (Polynomial)

compared to general Systems of ODEs (\S\ref{sec:system_of_odes}), the Jacobian
Matrix (\S\ref{sec:jacobian_matrix}) for a System of DAEs is a Singular
(Non-invertible) Matrix (\S\ref{sec:singular_matrix})

in practical terms, the Solution to a DAE System depends on the Derivatives of
the \emph{Input Signal} (\S\ref{sec:signal_flow}), and not just the ``Signal
itself'' as in the case of ODEs
(FIXME: clarify); e.g. see Systems with Hysteresis such as the Schmitt Trigger



% ====================================================================
\section{Algebraic Variety}\label{sec:algebraic_variety}
% ====================================================================

Dimension 2: Algebraic Surface (\S\ref{sec:algebraic_surface})

Scheme Theory (\S\ref{sec:scheme_theory}): Zariski Topology
(\S\ref{sec:zariski_topology}) allows Algebraic Varieties to be built by
``gluing together'' Affine Varieties (\S\ref{sec:affine_variety})

cf. Manifolds (\S\ref{sec:manifold}) built by ``gluing together'' Charts
(\S\ref{sec:chart}, Open Subsets of Real Affine Spaces
\S\ref{sec:affine_space}); Submanifolds (\S\ref{sec:submanifold})

\fist Projective Variety (Homogeneous Polynomials \S\ref{sec:projective_variety})



% --------------------------------------------------------------------
\subsection{Regular Map}\label{sec:regular_map}
% --------------------------------------------------------------------

a Morphism from an Algebraic Variety to the Affine Line
(\S\ref{sec:affine_line})

a Regular Map between Complex Algebraic Varieties is a Holomorphic Map
(\S\ref{sec:holomorphic_function})



% --------------------------------------------------------------------
\subsection{Affine Variety}\label{sec:affine_variety}
% --------------------------------------------------------------------

% --------------------------------------------------------------------
\subsection{Solution Set}\label{sec:solution_set}
% --------------------------------------------------------------------

% --------------------------------------------------------------------
\subsection{Characteristic Set}\label{sec:characteristic_set}
% --------------------------------------------------------------------

\subsubsection{Regular Chain}\label{sec:regular_chain}



% --------------------------------------------------------------------
\subsection{Algebraic Surface}\label{sec:algebraic_surface}
% --------------------------------------------------------------------

Algebraic Variety of Dimension 2



% --------------------------------------------------------------------
\subsection{Zariski Tangent Space}\label{sec:zariski_space}
% --------------------------------------------------------------------

Tangent Space (Topology \S\ref{sec:tangent_space})

Tangent Space at a Point of an Algebraic Variety $V$ giving a Vector Space
(\S\ref{sec:vector_space}) $V$ with Dimension at least that of $V$

Points are called \emph{Singular Points} when the Dimension of the Tangent
Space is higher than $V$ at that Point, are called \emph{Non-singular Points}
when the Dimension of the Tangent Space is equal to that of $V$ at that Point



% --------------------------------------------------------------------
\subsection{Projective Algebraic Variety}
\label{sec:projective_algebraic_variety}
% --------------------------------------------------------------------

\subsubsection{Abelian Variety}\label{sec:abelian_variety}

Projective Algebraic Variety that is also a Group

as a Topological Group (\S\ref{sec:topological_group}), an Abelian Variety is a
Torus %FIXME



% --------------------------------------------------------------------
\subsection{Motive}\label{sec:motive}
% --------------------------------------------------------------------



% ====================================================================
\section{Scheme Theory}\label{sec:scheme_theory}
% ====================================================================

% --------------------------------------------------------------------
\subsection{Zariski Topology}\label{sec:zariski_topology}
% --------------------------------------------------------------------

allows Algebraic Varieties to be built by ``gluing together'' Affine Varieties
 (\S\ref{sec:affine_variety})

cf. Manifolds (\S\ref{sec:manifold}) built by ``gluing together'' Charts
(\S\ref{sec:chart}, Open Subsets of Real Affine Spaces
\S\ref{sec:affine_space})



% --------------------------------------------------------------------
\subsection{Scheme}\label{sec:scheme}
% --------------------------------------------------------------------

Etale Topos (\S\ref{sec:etale_topos})

Locale (\S\ref{sec:locale})



\subsubsection{Affine Scheme}\label{sec:affine_scheme}



% --------------------------------------------------------------------
\subsection{Algebraic Fundamental Group}\label{sec:algebraic_fundamental_group}
% --------------------------------------------------------------------

or \emph{\'Etale Fundamental Group}

cf. Etale Topos, Etale Topology

cf. Fundamental Group (Algebraic Topology \S\ref{sec:fundamental_group}) of a
Topological Space (\S\ref{sec:topological_space})



% ====================================================================
\subsection{Moduli Space}\label{sec:moduli_space}
% ====================================================================

Fine Moduli Space

Coarse Moduli Space

a Configuration Space (\S\ref{sec:configuration_space}) is a type of (Fine)
Moduli Space



% ====================================================================
\section{Real Algebraic Geometry}\label{sec:real_algebraic_geometry}
% ====================================================================

Real Analytic Geometry (\S\ref{sec:real_analytic_geometry})



% --------------------------------------------------------------------
\subsection{Semialgebraic Function}\label{sec:semialgebraic_function}
% --------------------------------------------------------------------

a Real Closed Ring (\S\ref{sec:real_closed_ring}) is a Commutative Ring $A$
that is a Subring of the Product of Real Closed Fields
(\S\ref{sec:real_closed}) which is Closed under Continuous Semi-algebraic
Functions defined over the Integers

the Real Closure of the Polynomial Ring (\S\ref{sec:polynomial_ring})
$\reals[T_1,\ldots,T_n]$ is the Ring of Continuous Semi-algebraic Functions
$\reals^n \rightarrow \reals$



% --------------------------------------------------------------------
\subsection{Semialgebraic Set}\label{sec:semialgebraic_set}
% --------------------------------------------------------------------

Decision Problem for the Existential Theory of the Reals (TODO) is equivalent
to the problem of testing whether a given Semialgebraic Set is Non-empty and is
$NP-hard$ lyingin $PSPACE$

Geometric Graph Theory



% --------------------------------------------------------------------
\subsection{Semialgebraic Geometry}\label{sec:semialgebraic_geometry}
% --------------------------------------------------------------------

\subsubsection{Semialgebraic Set}\label{sec:semialgebraic_set}



% ====================================================================
\section{Diophantine Geometry}\label{sec:diophantine_geometry}
% ====================================================================

% ====================================================================
\section{Arithmetic Geometry}\label{sec:arithmetic_geometry}
% ====================================================================

% ====================================================================
\section{Complex Algebraic Geometry}
\label{sec:complex_algebraic_geometry}
% ====================================================================

% ====================================================================
\section{Universal Algebraic Geometry}\label{sec:universal_geometry}
% ====================================================================

% ====================================================================
\section{Computational Algebraic Geometry}
\label{sec:computational_algebraic_geometry}
% ====================================================================

% ====================================================================
\section{Derived Algebraic Geometry}
\label{sec:derived_algebraic_geometry}
% ====================================================================

% ====================================================================
\section{Anabelian Geometry}\label{sec:anabelian_geometry}
% ====================================================================

% ====================================================================
\section{Intersection Theory}\label{sec:intersection_theory}
% ====================================================================

in Algebraic Geometry: Subvarieties are \emph{Intersected} on an
Algebraic Variety

in Algebraic Topology: Intersections within ``the'' Cohomology Ring
%FIXME
