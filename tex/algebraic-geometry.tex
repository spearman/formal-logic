%%%%%%%%%%%%%%%%%%%%%%%%%%%%%%%%%%%%%%%%%%%%%%%%%%%%%%%%%%%%%%%%%%%%%%
%%%%%%%%%%%%%%%%%%%%%%%%%%%%%%%%%%%%%%%%%%%%%%%%%%%%%%%%%%%%%%%%%%%%%%
\part{Algebraic Geometry}\label{part:algebraic_geometry}
%%%%%%%%%%%%%%%%%%%%%%%%%%%%%%%%%%%%%%%%%%%%%%%%%%%%%%%%%%%%%%%%%%%%%%
%%%%%%%%%%%%%%%%%%%%%%%%%%%%%%%%%%%%%%%%%%%%%%%%%%%%%%%%%%%%%%%%%%%%%%

\emph{Valuation}

\emph{Hilbert's Nullstellensatz}

\emph{Cauchy's Inequality}:
\[
    |\langle x,y \rangle|^2 \leq \langle x,x \rangle \cdot \langle
    y,y \rangle
\]

Study of Geometry using Commutative Rings
(\S\ref{sec:commutative_ring})

One-dimensional Spaces:

\begin{itemize}
\item a Field $k$ is a One-dimensional Vector Space over itself
\item the Projective Line over a Field $k$
\item the Complex Projective Line (Riemann Sphere) over the Field of Complex
  Numbers
\end{itemize}



% ====================================================================
\section{Polynomial}\label{sec:polynomial}
% ====================================================================

\emph{Polynomial}

Generalized to Formal Power Series (\S\ref{sec:formal_power_series})

Univariate Polynomial (\S\ref{sec:univariate_polynomial}) of Degree
$n$ with Single Indeterminate $x$ and Coefficients $a_0, \ldots, a_n$
from a Ring:
\[
  a_n x^n + a_{n-1} x^{n-1} + \ldots + a_2 x^2 + a_1 x + a_0
\]
or:
\[
  \sum_{i=0}^n a_i x^i
\]

Continuous (\S\ref{sec:continuous_function})

Differentiable (\S\ref{sec:differentiable_function}) everywhere

Integer Coefficient Polynomial (\S\ref{sec:integer_coefficient})

Categorification: Polynomial Functor (\S\ref{sec:polynomial_functor})

testing Polynomial Equality:
\url{https://jeremykun.com/2017/04/24/testing-polynomial-equality/}

\fist Polynomial Sequences (Combinatorics \S\ref{sec:polynomial_sequence})

\fist Polynomial Ring (\S\ref{sec:polynomial_ring})



% --------------------------------------------------------------------
\subsection{Indeterminate}\label{sec:indeterminate}
% --------------------------------------------------------------------

% --------------------------------------------------------------------
\subsection{Discriminant}\label{sec:discriminant}
% --------------------------------------------------------------------

% --------------------------------------------------------------------
\subsection{Polynomial Arithmetic}\label{sec:polynomial_arithmetic}
% --------------------------------------------------------------------

Sum $P + Q$

Product $P Q$

Composition

Derivative

Antiderivative



\subsubsection{Polynomial Division}\label{sec:polynomial_division}

Euclidean Division

Polynomial Long Division



% --------------------------------------------------------------------
\subsection{Monomial}\label{sec:monomial}
% --------------------------------------------------------------------

% --------------------------------------------------------------------
\subsection{Binomial}\label{sec:binomial}
% --------------------------------------------------------------------

A \emph{Binomial} is a Polynomial which is the Sum of two Monomials
(\S\ref{sec:monomial})

Binomial Coefficient (\S\ref{sec:binomial_coefficient})



\subsubsection{Univariate Binomial}\label{sec:univariate_binomial}

\[
  a x^n - b x^m
\]
% FIXME


\subsubsection{Binomial Theorem}\label{sec:binomial_theorem}



% --------------------------------------------------------------------
\subsection{Univariate Polynomial}\label{sec:univariate_polynomial}
% --------------------------------------------------------------------

% --------------------------------------------------------------------
\subsection{Bivariate Polynomial}\label{sec:bivariate_polynomial}
% --------------------------------------------------------------------

% --------------------------------------------------------------------
\subsection{Multivariate Polynomial}\label{sec:multivariate_polynomial}
% --------------------------------------------------------------------

% --------------------------------------------------------------------
\subsection{Symmetric Polynomial}\label{sec:symmetric_polynomial}
% --------------------------------------------------------------------

% --------------------------------------------------------------------
\subsection{Homogenous Polynomial}\label{sec:homogenous_polynomial}
% --------------------------------------------------------------------

Quadratic Form (\S\ref{sec:quadratic_form})



% --------------------------------------------------------------------
\subsection{Irreducible Polynomial}\label{sec:irreducible_polynomial}
% --------------------------------------------------------------------

% --------------------------------------------------------------------
\subsection{Taylor Polynomial}\label{sec:taylor_polynomial}
% --------------------------------------------------------------------

(wiki:)

``Linear Taylor Polynomials'' in the definition of Cotangent Spaces
(\S\ref{sec:cotangent_space}) -- two Smooth Functions are considered
``Equivalent'' at a Point $x$ if they have the same ``\emph{First-order
  Behavior}'' near $x$, analogous to their Linear Taylor Polynomials,
where First-order Behavior is defined as Equivalent if and only if the
Derivative of the Function $f-g$ \emph{vanishes} at $x$ --the Cotangent Space
then consists of all the possible ``\emph{First-order Behaviors}'' of a
Function near $x$



% --------------------------------------------------------------------
\subsection{Polynomial Function}\label{sec:polynomial_function}
% --------------------------------------------------------------------

Degree $0$ or $1$ is a \emph{Linear Function}

the Harmonic Functions (\S\ref{sec:harmonic_function}) on $\reals$ are
exactly the Linear Functions



% ====================================================================
\section{Algebraic Expression}\label{sec:algebraic_expression}
% ====================================================================

% --------------------------------------------------------------------
\subsection{Algebraic Fraction}\label{sec:algebraic_fraction}
% --------------------------------------------------------------------

% --------------------------------------------------------------------
\subsection{Algebraic Equation}\label{sec:algebraic_equation}
% --------------------------------------------------------------------

cf. Functional Equation --
\url{https://golem.ph.utexas.edu/category/2017/04/functional_equations_entropy_a.html},
Implicit Function %FIXME



\subsubsection{Polynomial Equation}\label{sec:polynomial_equation}

\emph{Polynomial Equation}

Linear Equation (\S\ref{sec:linear_equation})



\paragraph{Quadratic Equation}\label{sec:quadratic_equation}\hfill



\subsubsection{Parametric Equation}\label{sec:parametric_equation}

%FIXME move ?
%FIXME differential, integral calculus ?
%FIXME analysis ?

Parametric Surface (\S\ref{sec:parametric_surface})



% ====================================================================
\section{System of Equations}\label{sec:system_of_equations}
% ====================================================================

% --------------------------------------------------------------------
\subsection{System of Polynomial Equations}
\label{sec:system_of_polynomials}
% --------------------------------------------------------------------

Polynomial Equation (\S\ref{sec:polynomial_equation})

Algebraically Closed (\S\ref{sec:algebraically_closed}) Field
Extension (\S\ref{sec:field_extension})



\subsubsection{Cylindrical Algebraic Decomposition}
\label{sec:algebraic_decomposition}

Complexity: Double Exponential (\S\ref{sec:double_exponential})

Quantification over Reals



% --------------------------------------------------------------------
\subsection{System of Linear Equations}
\label{sec:system_of_linear_equations}
% --------------------------------------------------------------------

Linear Equation (\S\ref{sec:linear_equation})

Numerical Methods:
\begin{itemize}
  \item Gauss-Seidel Method (\S\ref{sec:gauss_seidel})
\end{itemize}



% ====================================================================
\section{Algebraic Variety}\label{sec:algebraic_variety}
% ====================================================================

Dimension 2: Algebraic Surface (\S\ref{sec:algebraic_surface})

Scheme Theory (\S\ref{sec:scheme_theory}): Zariski Topology
(\S\ref{sec:zariski_topology}) allows Algebraic Varieties to be built by
``gluing together'' Affine Varieties (\S\ref{sec:affine_variety})

cf. Manifolds (\S\ref{sec:manifold}) built by ``gluing together'' Charts
(\S\ref{sec:chart}, Open Subsets of Real Affine Spaces
\S\ref{sec:affine_space})




% --------------------------------------------------------------------
\subsection{Affine Variety}\label{sec:affine_variety}
% --------------------------------------------------------------------

% --------------------------------------------------------------------
\subsection{Solution Set}\label{sec:solution_set}
% --------------------------------------------------------------------

% --------------------------------------------------------------------
\subsection{Characteristic Set}\label{sec:characteristic_set}
% --------------------------------------------------------------------

\subsubsection{Regular Chain}\label{sec:regular_chain}



% --------------------------------------------------------------------
\subsection{Algebraic Surface}\label{sec:algebraic_surface}
% --------------------------------------------------------------------

Algebraic Variety of Dimension 2



% --------------------------------------------------------------------
\subsection{Projective Algebraic Variety}
\label{sec:projective_algebraic_variety}
% --------------------------------------------------------------------

\subsubsection{Abelian Variety}\label{sec:abelian_variety}

Projective Algebraic Variety that is also a Group

as a Topological Group (\S\ref{sec:topological_group}), an Abelian Variety is a
Torus %FIXME



% --------------------------------------------------------------------
\subsection{Motive}\label{sec:motive}
% --------------------------------------------------------------------



% ====================================================================
\section{Scheme Theory}\label{sec:scheme_theory}
% ====================================================================

% --------------------------------------------------------------------
\subsection{Zariski Topology}\label{sec:zariski_topology}
% --------------------------------------------------------------------

allows Algebraic Varieties to be built by ``gluing together'' Affine Varieties
 (\S\ref{sec:affine_variety})

cf. Manifolds (\S\ref{sec:manifold}) built by ``gluing together'' Charts
(\S\ref{sec:chart}, Open Subsets of Real Affine Spaces
\S\ref{sec:affine_space})



% --------------------------------------------------------------------
\subsection{Scheme}\label{sec:scheme}
% --------------------------------------------------------------------

Etale Topos (\S\ref{sec:etale_topos})

Locale (\S\ref{sec:locale})



\subsubsection{Affine Scheme}\label{sec:affine_scheme}



% --------------------------------------------------------------------
\subsection{Algebraic Fundamental Group}
\label{sec:algebraic_fundamental_group}
% --------------------------------------------------------------------

or \emph{\'Etale Fundamental Group}

cf. Etale Topos, Etale Topology

cf. Fundamental Group (Algebraic Topology \S\ref{sec:fundamental_group}) of a
Topological Space (\S\ref{sec:topological_space})



% ====================================================================
\subsection{Moduli Space}\label{sec:moduli_space}
% ====================================================================

Fine Moduli Space

Coarse Moduli Space

a Configuration Space (\S\ref{sec:configuration_space}) is a type of (Fine)
Moduli Space



% ====================================================================
\section{Real Algebraic Geometry}\label{sec:real_algebraic_geometry}
% ====================================================================

Real Analytic Geometry (\S\ref{sec:real_analytic_geometry})



% --------------------------------------------------------------------
\subsection{Semialgebraic Geometry}\label{sec:semialgebraic_geometry}
% --------------------------------------------------------------------

\subsubsection{Semialgebraic Set}\label{sec:semialgebraic_set}



% ====================================================================
\section{Diophantine Geometry}\label{sec:diophantine_geometry}
% ====================================================================

% ====================================================================
\section{Arithmetic Geometry}\label{sec:arithmetic_geometry}
% ====================================================================

% ====================================================================
\section{Complex Algebraic Geometry}
\label{sec:complex_algebraic_geometry}
% ====================================================================

% ====================================================================
\section{Universal Algebraic Geometry}\label{sec:universal_geometry}
% ====================================================================

% ====================================================================
\section{Derived Algebraic Geometry}
\label{sec:derived_algebraic_geometry}
% ====================================================================

% ====================================================================
\section{Anabelian Geometry}\label{sec:anabelian_geometry}
% ====================================================================

% ====================================================================
\section{Intersection Theory}\label{sec:intersection_theory}
% ====================================================================

in Algebraic Geometry: Subvarieties are \emph{Intersected} on an
Algebraic Variety

in Algebraic Topology: Intersections within ``the'' Cohomology Ring
%FIXME
