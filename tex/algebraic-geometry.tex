%%%%%%%%%%%%%%%%%%%%%%%%%%%%%%%%%%%%%%%%%%%%%%%%%%%%%%%%%%%%%%%%%%%%%%
%%%%%%%%%%%%%%%%%%%%%%%%%%%%%%%%%%%%%%%%%%%%%%%%%%%%%%%%%%%%%%%%%%%%%%
\part{Algebraic Geometry}\label{part:algebraic_geometry}
%%%%%%%%%%%%%%%%%%%%%%%%%%%%%%%%%%%%%%%%%%%%%%%%%%%%%%%%%%%%%%%%%%%%%%
%%%%%%%%%%%%%%%%%%%%%%%%%%%%%%%%%%%%%%%%%%%%%%%%%%%%%%%%%%%%%%%%%%%%%%

Study of Zero Sets (\S\ref{sec:zero_set}) of Systems of Polynomial Equations
(\S\ref{sec:polynomial_equation_system}) \fist Algebraic Varieties
(\S\ref{sec:algebraic_variety})

such Zero Sets may be considered over different Fields (\S\ref{sec:field}):
\begin{itemize}
  \item $\rats$ (Field of Rational Numbers \S\ref{sec:rational}) --
    Arithmetic (Algebraic) Geometry (\S\ref{sec:arithmetic_geometry})
  \item $\reals$ -- Real Algebraic Geometry
    (\S\ref{sec:real_algebraic_geometry})
  \item $\comps$ -- Algebraically Closed (\S\ref{sec:algebraically_closed})--
    Solutions always exist; any Complex Manifold (\S\ref{sec:complex_manifold})
    is an Analytic Variety (\S\ref{sec:analytic_variety})
\end{itemize}

Study of \emph{Geometry} using Commutative Rings (\S\ref{sec:commutative_ring})

\begin{itemize}
  \item Algebraic Number Theory (\S\ref{sec:algebraic_number_theory})
  \item Algebraic Geometry
  \item Commutative Algebra (\S\ref{sec:commutative_algebra})
\end{itemize}

cf. Geometric Algebra (Algebra over a Field \S\ref{sec:geometric_algebra})

2012 - Dolgachev - \emph{Classical Algebraic Geometry: a modern view}

\emph{Valuation}

\emph{Hilbert's Nullstellensatz}: fundamental correspondence between Ideals
(\S\ref{sec:ring_ideal}) of Polynomial Rings (\S\ref{sec:polynomial_ring}) and
Algebraic Sets (\S\ref{sec:algebraic_set})

\emph{Cauchy's Inequality}:
\[
    |\langle x,y \rangle|^2 \leq \langle x,x \rangle \cdot \langle
    y,y \rangle
\]

One-dimensional Spaces:
\begin{itemize}
\item a Field $k$ is a One-dimensional Vector Space over itself
\item the Projective Line over a Field $k$
\item the Complex Projective Line (Riemann Sphere) over the Field of Complex
  Numbers
\end{itemize}

\emph{Scheme Theory} (\S\ref{sec:scheme_theory})

\url{https://ncatlab.org/nlab/show/Global+analytic+geometry}

\emph{Global Analytic Geometry} (\S\ref{sec:global_analytic_geometry}) combines
Non-archimedean (\S\ref{sec:nonarchimedean_analytic_geometry}) and Archimedean
Analytic Geometry (Part \ref{part:analytic_geometry}) and contains Algebraic
Geometry as a \emph{Sub-theory}; treats all Places (\S\ref{sec:place}) ``on
equal footing'' in contrast to Scheme Theory



% ====================================================================
\section{Polynomial}\label{sec:polynomial}
% ====================================================================

A \emph{Polynomial Expression} is an Expression containing Variables (or
\emph{Indeterminates}) and Coefficients with only Arithmetic Operations
(\S\ref{sec:arithmetic}) and Positive Integer Exponents
(\S\ref{sec:exponentiation}) where the \emph{Degree} (\S\ref{sec:degree}) of
the Polynomial is the largest such Exponent of any one Term with Nonzero
Coefficient.

\fist Signomials (\S\ref{sec:signomial}) -- allows arbitrary Real Exponents but
requires that Independent Real Variables are Strictly Positive

\fist Algebraic Expressions (\S\ref{sec:algebraic_expression}) are a wider
class of Expressions that allow for Negative and Rational Exponents ($n$-th
Roots).

\fist Polynomial Interpolation (\S\ref{sec:polynomial_interpolation})

\fist Homogeneous Polynomial (Projective Geometry
\S\ref{sec:homogeneous_polynomial})

\fist generalized to Formal Power Series (\S\ref{sec:formal_power_series})

\fist ``Polynomial Mapping'' (Recurrence Relation
\S\ref{sec:recurrence_relation}), e.g. Logistic Map (\S\ref{sec:logistic_map})

\fist an \emph{Algebraic Curve} (\S\ref{sec:algebraic_curve}) is an Implicit
Curve defined by an Equation $F(x,y) = 0$ such that $F(x,y)$ is a Polynomial

The Funadmental Theorem of Algebra (\S\ref{sec:fundamental_algebra_theorem})
states that every Polynomial of Degree $n$ has $n$ Complex Roots
(\S\ref{sec:function_root}), counted with their Multiplicities (??? FIXME).

a Complex Number (\S\ref{sec:complex_number}) can be defined as a Polynomial in
the single Indeterminate with the Relation $i^2 + 1 = 0$ imposed, and Complex
Numbers can then be Added or Multiplied using Addition and Multiplication of
Polynomials

an Algebraically Closed Field (\S\ref{sec:algebraically_closed}) $F$ contains a
Root (\S\ref{sec:function_root}) for every Non-constant Polynomial in the Ring
of Polynomials (\S\ref{sec:polynomial_ring}) $F[x]$ in the Variable $x$ with
Coefficients in $F$

the Set of Complex Numbers is the Quotient Ring of the Polynomial Ring in the
Indeterminate $i$ by the Ideal generated by the Polynomial $i^2 + 1$

any Polynomial with Complex Coefficients (\S\ref{sec:complex_number}) has
Complex Roots (i.e. the Complex Numbers are \emph{Algebraically Closed})

Univariate Polynomial (\S\ref{sec:univariate_polynomial}) of Degree
$n$ with Single Indeterminate $x$ and Coefficients $a_0, \ldots, a_n$
from a Ring:
\[
  a_n x^n + a_{n-1} x^{n-1} + \ldots + a_2 x^2 + a_1 x + a_0
\]
or:
\[
  \sum_{i=0}^n a_i x^i
\]

Continuous (\S\ref{sec:continuous_function})

Differentiable (\S\ref{sec:differentiable_function}) everywhere: the Power Rule
combined with the Linearity Rules for Differentiation allows computing the
Derivative (\S\ref{sec:derivative}) of any Polynomial

Integer Coefficient Polynomial (\S\ref{sec:integer_coefficient})

Categorification: Polynomial Functor (\S\ref{sec:polynomial_functor})

testing Polynomial Equality:
\url{https://jeremykun.com/2017/04/24/testing-polynomial-equality/}

\fist Polynomial Sequences (Combinatorics \S\ref{sec:polynomial_sequence})

\fist Rank Polynomial (Graph Theory \S\ref{sec:rank_polynomial})

$R[x]$ -- Ring of Polynomials (Free Commutative Algebra
\S\ref{sec:polynomial_ring}) in the Indeterminate $x$ over the Ring $R$

a Linear Differential Equation (\S\ref{sec:linear_differential_equation}) is an
Ordinary Differential Equation that is defined by a Linear (i.e. Degree $1$)
Polynomial of the Unknown Function and its Derivatives



% --------------------------------------------------------------------
\subsection{Polynomial Expression}\label{sec:polynomial_expression}
% --------------------------------------------------------------------

\begin{itemize}
  \item number of \emph{Terms} (with non-zero Coefficients) -- Monomial
    (\S\ref{sec:monomial}), Binomial (\S\ref{sec:binomial}), etc.
  \item \emph{Degree} (\S\ref{sec:degree}) -- highest Degree of its individual
    Terms (Monomials)
  \item number of \emph{Indeterminates} (\S\ref{sec:indeterminate}) --
    Univariate (\S\ref{sec:univariate_polynomial}), Multivariate
    (\S\ref{sec:multivariate_polynomial})
\end{itemize}



\subsubsection{Coefficient}\label{sec:coefficient}

\emph{Leading Coefficient}



\subsubsection{Indeterminate}\label{sec:indeterminate}

when a Polynomial Expression is considered as a Polynomial Function
(\S\ref{sec:polynomial_function}), the Indeterminates become ``Variables''



\subsubsection{Degree}\label{sec:degree}

note sometimes the term ``Order'' is used to refer to the Degree of a
Polynomial

highest Degree of its Monomials

\begin{itemize}
  \item Degree 1 -- Linear
  \item Degree 2 -- Quadratic
  \item ...
\end{itemize}



\subsubsection{Resultant}\label{sec:resultant}

$\mathrm{Res}$



\subsubsection{Discriminant}\label{sec:discriminant}

The \emph{Discriminant} of a Polynomial $A(x)$ and is defined as a Quotient of
the Resultant of $A$ and its $A$'s Derivative $A'$ by the Leading Coefficient
$a_n$:
\[
  \mathrm{Disc}_x(A) = \frac{-\frac{n(n-1)}{2}}{a_n} \mathrm{Res}_x(A, A')
\]



\subsubsection{Monomial}\label{sec:monomial}

or \emph{Power Product}

a Polyonomial of a single Term

a Product of Powers of Variables with Non-negative Integer Exponents

$x^a y^b z^c \cdots$

the \emph{Degree} of a Monomial is the Sum of all the Exponents of the
Variables



\subsubsection{Binomial}\label{sec:binomial}

A \emph{Binomial} is a Polynomial which is the Sum of two Monomials
(\S\ref{sec:monomial})

Binomial Coefficient (\S\ref{sec:binomial_coefficient})



\paragraph{Univariate Binomial}\label{sec:univariate_binomial}\hfill

\[
  a x^n - b x^m
\]
% FIXME


\paragraph{Binomial Theorem}\label{sec:binomial_theorem}\hfill



\subsubsection{Univariate Polynomial}\label{sec:univariate_polynomial}

Polynomial with a single Indeterminate



\paragraph{Monic Polynomial}\label{sec:monic_polynomial}\hfill

\emph{Fundamental Theorem of Algebra link between ``Algebra'' and
  ``Geometry''}: a Monic Polynomial, an Algebraic Object, is determined by the
Set of its Roots (\S\ref{sec:function_root}), a Geometric Object, in the
Complex Plane (\S\ref{sec:complex_plane}).



\subsubsection{Multivariate Polynomial}\label{sec:multivariate_polynomial}

Polynomial with more than one Indeterminate



\paragraph{Bivariate Polynomial}\label{sec:bivariate_polynomial}\hfill

Polynomial with two Indeterminates



% --------------------------------------------------------------------
\subsection{Polynomial Arithmetic}\label{sec:polynomial_arithmetic}
% --------------------------------------------------------------------

Sum $P + Q$

Product $P Q$

Composition

Derivative

Antiderivative



\subsubsection{Polynomial Division}\label{sec:polynomial_division}

Euclidean Division

Polynomial Long Division



% --------------------------------------------------------------------
\subsection{Zero Polynomial}\label{sec:zero_polynomial}
% --------------------------------------------------------------------

$0$ (no Terms), Degree is either undefined or defined as \emph{Negative}: $-1$
or $-\infty$



% --------------------------------------------------------------------
\subsection{Polynomial Function}\label{sec:polynomial_function}
% --------------------------------------------------------------------

Function defined by Evaluating a Polynomial Expression
(\S\ref{sec:polynomial_expression})

the Variables of a Polynomial Function are the Indeterminates
(\S\ref{sec:indeterminate}) of the defining Polynomial Expression

Degree $0$ or $1$ is a \emph{Linear Function} (\S\ref{sec:linear_function})

the Harmonic Functions (\S\ref{sec:harmonic_function}) on $\reals$ are
exactly the Linear Functions

\emph{Weierstrass Approximation Theorem}
(\S\ref{sec:weierstrass_approximation}): every \emph{Continuous Function}
(\S\ref{sec:continuous_function}) defined on a Closed Interval can be Uniformly
Approximated (\S\ref{sec:uniform_convergence}) to arbitrary closeness by a
Polynomial Function

\emph{Stone-Weierstrass Theorem}: the Set of all Continuous Functions on a
Closed Interval is the Uniform Closure (cf. Uniform Norm \S\ref{sec:p_norm}) of
the Set of Polynomials on the Interval



\subsubsection{Linear Function}\label{sec:linear_function}

Polynomial of Degree $0$ or $1$

Second and Higher Derivatives are all Zero

\begin{itemize}
  \item \emph{Linear Predictor Function} (Linear Classifiers
    \S\ref{sec:linear_classifier})
  \item ...
\end{itemize}



\subsubsection{Quadratic Function}\label{sec:quadratic_function}

\fist Quadratic Equation (\S\ref{sec:quadratic_equation})

\fist Quadratic Form (\S\ref{sec:quadratic_form})

\begin{align*}
  f(\vec{x})        & = \frac{1}{2}\vec{x}^T P \vec{x} + \vec{q}^T\vec{x} \\
  \nabla f(\vec{x}) & = P\vec{x} + \vec{q} \\
  H(f(\vec{x}))     & = P \\
\end{align*}
where $H(f(x))$ is the Hessian Matrix (\S\ref{sec:hessian_matrix}) of
Second-order Partial Derivatives of $f$



\subsubsection{Power Function}\label{sec:power_function}

%FIXME: move this section ?

\paragraph{Even Function}\label{sec:even_function}\hfill

Even Parity (\S\ref{sec:parity}) of the Exponent



\paragraph{Odd Function}\label{sec:odd_function}\hfill

Odd Parity (\S\ref{sec:parity}) of the Exponent



\subsubsection{Weierstrass Approximation Theorem}
\label{sec:weierstrass_approximation}

any given Continuous (\S\ref{sec:continuous_function}) Complex-Valued Function
defined on a Closed Interval $[a,b]$ can be Uniformly Approximated
(\S\ref{sec:uniform_convergence}) as closely as desired by a Polynomial Function

\emph{Stone-Weierstrass Theorem}: the Set of all Continuous Functions on a
Closed Interval is the Uniform Closure (cf. Uniform Norm \S\ref{sec:p_norm}) of
the Set of Polynomials on the Interval



% --------------------------------------------------------------------
\subsection{Symmetric Polynomial}\label{sec:symmetric_polynomial}
% --------------------------------------------------------------------

\fist Commutative Algebra (\S\ref{sec:commutative_algebra})



\subsubsection{Elementary Symmetric Polynomial}\label{sec:elementary_symmetric}

any Symmetric Polynomial can be expressed as a Polynomial in Elementary
Symmetric Polynomials (i.e. as an expression involving only Addition and
Multiplications of Constants and Elementary Symmetric Polynomials)

there is \emph{one} Elementary Symmetric Polynomial of Degree $d$ in $n$
Variables for each Non-negative Integer $d \leq n$, and it is formed by adding
together all distinct Products of $d$ distinct Variables



% --------------------------------------------------------------------
\subsection{Homogeneous Polynomial}\label{sec:homogeneous_polynomial}
% --------------------------------------------------------------------

a \emph{Homogeneous Polynomial} is such that the Nonzero Terms all have the same
Degree

Homogeneous Function (\S\ref{sec:homogeneous_function})

the Space of Symmetric Tensors (\S\ref{sec:symmetric_tensor}) of Order $r$ on a
Finite-dimensional Vector Space is naturally Isomorphic to the Dual of the
Space of Homogeneous Polynomials of Degree $r$ on $V$

a Homogeneous Linear Differential Equation
(\S\ref{sec:homogeneous_linear_differential}) is a Linear Differential
Equation with a $0$ Constant Term, i.e. it is a Homogeneous Polynomial in the
Unknown Function and its Derivatives

cf. Homogeneous Difference Equation
(\S\ref{sec:homogeneous_differential_equation}),
Homogeneous System of Linear Equations
(\S\ref{sec:homogeneous_linear_equation_system})

Projective Varieties (\S\ref{sec:projective_variety})



\subsubsection{Quadratic Form}\label{sec:quadratic_form}

a \emph{Quadratic Form} is a Homogeneous Polynomial of Degree Two in a number of
Variables

note that Quadratic Functions (\S\ref{sec:quadratic_function}) are not
equivalent to Quadratic Forms since the former may not be Homogeneous

a Quadratic Form $f(\vec{x})$ on $n$ Real Variables $x_1,\ldots,x_n$ can always
be written as $\vec{x}^T M \vec{x}$ for $M$ a Symmetric Real Matrix
(\S\ref{sec:symmetric_matrix})

Matrix form $\vec{x}^T M \vec{x}$:
\[
  [x y]
  \begin{bmatrix}
    a & b \\
    c & d
  \end{bmatrix}
  \begin{bmatrix}
    x \\
    y
  \end{bmatrix}
  = ax^2 + 2bxy + cy^2
\]

Associated Symmetric Bilinear Form (\S\ref{sec:symmetric_bilinear}): Symmetric
Bilinear Forms over a Vector Space correspond one-to-one with Quadratic Forms
over the Vector Space

\emph{Diagonal Form}, \emph{Signature}

(\emph{Jacobi's Theorem}) any Symmetric Matrix can be transformed into a
Diagonal Matrix $B$ by an Orthogonal Matrix $S$:
\[
  B = S^T A S
\]
and the Diagonals of $B$ are uniquely determined

(\emph{Sylvester's Law of Inertia}) if $S$ is allowed to be any Invertible
Matrix then the Diagonals of $B$ may be made to be $0$, $1$, and $-1$, the
number of each type giving the Signature of the associated Quadratic Form

when the Diagonals are all $1$, the Quadratic Form is Positive Definite, and
when all are $-1$ it is Negative Definite
the Orthogonal Group (\S\ref{sec:orthogonal_group}) $O(V)$ -- preserves a
Non-degenerate Quadratic Form on $V$

Jacobi's Theorem and Sylvester's Law shows that any Positive Definite Quadratic
Form in $n$ Variables can be brought to the Sum of $n$ Squares by a suitable
Invertible Linear Transformation: Geometrically \emph{there is only one
  Positive Definite Real Quadratic Form of every Dimension}, with its Isometry
Group a Compact Orthogonal Group (\S\ref{sec:orthogonal_group}) $O(n)$

for Indefinite Forms, the corresponding Group is the Indefinite Orthogonal
Group $O(p,q)$ which is Non-compact


UC Math 352 Lec. 17 - \url{https://www.youtube.com/watch?v=5tY3G1895_c}:

Positive Definite Matrices (\S\ref{sec:positive_definite}) can be used to
Minimize a Complex Quadratic Form



\paragraph{Definite Quadratic Form}\label{sec:definite_quadratic}\hfill

a Quadratic Form over some Real Vector Space that has the same Sign
for every Non-zero Vector

\emph{Positive Definite}

\emph{Negative Definite}

\emph{Semidefinite}

\fist Positive-definite Matrices (\S\ref{sec:positive_definite})

\fist Sylvester's Law of Inertia (TODO: xref)

Generally a Real Function $f$ on $n$ Real Variables has local Minimum at
$\vec{x}_0$ if its Gradient (\S\ref{sec:gradient}) $\nabla f$ is Zero and its
Hessian Matrix (\S\ref{sec:hessian_matrix}) $H(f)$ is Positive Semi-definite at
that Point

\fist Quadratic Approximation (\S\ref{sec:quadratic_approximation})

\fist Multiple Regression (\S\ref{sec:multiple_regression})

\fist Quadratic Programming (\S\ref{sec:quadratic_programming})

an Inner Product (\S\ref{sec:inner_product}) on a Real Vector Space is
a Positive-definite Symmetric Bilinear
(\S\ref{sec:symmetric_bilinear}) Form



\paragraph{Quadratic Space}\label{sec:quadratic_space}\hfill

\paragraph{Genus}\label{sec:quadratic_genus}\hfill

\fist cf. Topological Genus (\S\ref{sec:genus}), Scheme Genus
(\S\ref{sec:scheme_genus})



\subparagraph{Spinor Genus}\label{sec:spinor_genus}\hfill



% --------------------------------------------------------------------
\subsection{Irreducible Polynomial}\label{sec:irreducible_polynomial}
% --------------------------------------------------------------------

\fist a Quadric (\S\ref{sec:quadric}) is a Hypersurface of Dimension $d$ in a
$d+1$-dimensional Space defined as the Zero Set (Function Root
\S\ref{sec:function_root}) of an Irreducible Polynomial of Degree $2$ in $d+1$
Variables



% --------------------------------------------------------------------
\subsection{Taylor Polynomial}\label{sec:taylor_polynomial}
% --------------------------------------------------------------------

approximating a Smooth Function by a Polynomial with Terms from the
Taylor Series (\S\ref{sec:taylor_series}) of that Function

(wiki:)

``Linear Taylor Polynomials'' in the definition of Cotangent Spaces
(\S\ref{sec:cotangent_space}) -- two Smooth Functions are considered
``Equivalent'' at a Point $x$ if they have the same ``\emph{First-order
  Behavior}'' near $x$, analogous to their Linear Taylor Polynomials,
where First-order Behavior is defined as Equivalent if and only if the
Derivative of the Function $f-g$ \emph{vanishes} at $x$ --the Cotangent Space
then consists of all the possible ``\emph{First-order Behaviors}'' of a
Function near $x$



% --------------------------------------------------------------------
\subsection{Formal Derivative}\label{sec:formal_derivative}
% --------------------------------------------------------------------

%FIXME: move section ???

cf. Formal Scheme (\S\ref{sec:formal_scheme})



% --------------------------------------------------------------------
\subsection{Formal Power Series}\label{sec:formal_power_series}
% --------------------------------------------------------------------

generalization of a Polynomial (\S\ref{sec:polynomial}) where the number of
Terms is allowed to be \emph{Infinite}

\fist Power Series (\S\ref{sec:power_series})

\fist \emph{Perturbation Series} (Perturbation Methods
\S\ref{sec:perturbation_method}): Formal Power Series in some ``small''
Parameter quantifying the deviation from the exactly Solvable problem



\subsubsection{Asymptotic Expansion}\label{sec:asymptotic_expansion}

or \emph{Asymptotic Series}



\subsubsection{Formal Dirichlet Series}\label{sec:formal_dirichlet_series}

Dirichlet Series (\S\ref{sec:dirichlet_series})



\subsubsection{Generating Function}\label{sec:generating_function}

or \emph{Generating Series}

1990 - Wilf - \emph{Generatingfunctionology}

(wiki):

an encoding of an Infinite Sequence (\S\ref{sec:infinite_sequence}) $a_n$ of
Numbers by treating them as Coefficients of a (Formal) Power Series (the
\emph{Generating Function})

Generating Functions are \emph{not} Functions in the sense of a mapping from
Domain to Codomain

\fist Combinatorial Species (\S\ref{sec:combinatorial_species}) -- abstract
method for analyzing Discrete Structures in terms of Generating Functions,
defined as an Endofunctor $F : \cat{FinSet}_{\biject} \rightarrow
\cat{FinSet}_{\biject}$ called the \emph{Species of $F$-structures} on the
Category $\cat{FinSet}_{\biject}$ of Finite Sets with Bijections as Morphisms,
where the Finite Set $F A$ is called the \emph{Set of $F$-structures on Finite
  Set $A$} or the \emph{Set of Structures of Species $F$ on Finite Set $A$}

\fist cf. Riemann Zeta Function (\S\ref{sec:riemann_zeta})

\fist Weil Conjectures (\S\ref{sec:weil_conjectures})

\begin{itemize}
  \item $G(a_n; x) = \Sum_{n=0}^\infty a_n x^n$ -- \emph{Ordinary Generating
    Functions (OGFs)}: describe Structures on Totally Ordered Sets
    (\S\ref{sec:totally_ordered}); when the Sequence is the Probability Mass
    Function (\S\ref{sec:probability_mass_function}) of a Discrete Random
    Variable (\S\ref{sec:discrete_random_variable}), it is a
    \emph{Probability-generating Function}
    (\S\ref{sec:probability_generating_function})

    can be generalized to Multi-dimensional Arrays instead of Sequences
  \item $EG(a_n; x) = \Sum_{n=0}^\infty a_n \frac{x^n}{n!}$ -- \emph{Exponential
    Generating Functions}: describe Structures on unordered (labelled) Sets
  \item $PG(a_n; x) = \Sum_{n=0}^\infty a_n e^{-x}\frac{x^n}{n!} = e^{-x}EG(a_n; x)$
    -- \emph{Poisson Generating Functions}
  \item $LG(a_n; x) = \Sum_{n=1}^\infty a_n \frac{x^n}{1 - x^n}$ --
    \emph{Lambert Series}
  \item $DG(a_n; s) = \Sum_{n=0}^\infty a_{p^n} x^n$ -- \emph{Bell Series}
  \item $DG(a_n; s) = \Sum_{n=1}^\infty \frac{a_n}{n^s}$ --
    \emph{Dirichlet Series}; has a Euler Product (\S\ref{sec:euler_product})
    expression in terms of the Function's Bell Series
\end{itemize}

every Sequence in principle has a Generating Function of each type, except that
Lambert and Dirichlet Series require Indices to start at $1$ rather than $0$



% --------------------------------------------------------------------
\subsection{Polynomial Equation}\label{sec:polynomial_equation}
% --------------------------------------------------------------------

A \emph{Polynomial Equation} is an Equation of the form:
\[
  P = Q
\]
where $P$ and $Q$ are \emph{Polynomials} (\S\ref{sec:polynomial}) with
Coefficients in some Field (\S\ref{sec:field}).

\fist Polynomial Equations are sometimes called \emph{Algebraic Equations} when
they involve a Single Variable, but should not be confused with Algebraic
Expressions (\S\ref{sec:algebraic_expression}).

\fist System of Differential-Algebraic Equations (\S\ref{sec:dae_system})

The \emph{Solution} to a Polynomial Equation is a Value or Set of Values; cf.
the Solution to a Differential Equation (\S\ref{sec:differential_equation}) is
a Function or Class of Functions.

Polynomial Equations by Degree:
\begin{itemize}
  \item Degree 1 -- Linear Equation (\S\ref{sec:linear_equation})
  \item Degree 2 -- Quadratic Equation (\S\ref{sec:quadratic_equation})
  \item ...
\end{itemize}

Linear Equation (\S\ref{sec:linear_equation})

cf. Functional Equation --
\url{https://golem.ph.utexas.edu/category/2017/04/functional_equations_entropy_a.html},
Implicit Function %FIXME


\emph{Tarski-Seidenberg Theorem}\S\ref{sec:tarski_seidenberg}

the First-order Theory of the Real Closed Field (\S\ref{sec:real_closed}) is
Decidable: every Formula constructed from Polynomial Equations and Inequalities
by the Logical Connectives $\vee$, $\wedge$, $\neg$, and Quantifiers $\forall$,
$\exists$ is equivalent to a Formula without Quantifiers



\subsubsection{Root}\label{sec:equation_root}

\fist Root (Functions \S\ref{sec:function_root})

An Algebraic Function (\S\ref{sec:algebraic_function}) is a Function that can
be defined as the Root of a Polynomial Equation.

the Golden Ration $\varphi$ is the Root of the Polynomial $x^2 - x - 1$

for Polynomials of Degree $1$ through $4$, there are Closed-form solutions
which produce all Roots

$5$th-degree and higher-degree Polynomials do not have a general Algebraic
Solution (\S\ref{sec:algebraic_expression}); \emph{Abel-Ruffini Theorem}



\subsubsection{Diophantine Equation}\label{sec:diophantine_equation}

a Polynomial Equation of two or more Unknowns for which only the Integer
Solutions (where all Unknowns take Integer Values) are sought

\fist Number Theory

\emph{Hilbert's Tenth Problem}: there is no general Algorithm for Deciding
whether for any given Diophantine Equation there exists a Solution with all
Unknowns taking Integer values



\paragraph{Linear Diophantine Equation}
\label{sec:linear_diophantine}\hfill

Monomial (\S\ref{sec:monomial}) Equations with Integer Solutions

$c = ax + by$ has a solution in $\mathbb{Z}$ when $gcd(a,b)|c$. Given
$d=gcd(a,b)|c$ and a solution $(x_0, y_0)$, then all solutions are of
the form $(x_0 + \frac{b}{d}t, y_0 - \frac{a}{d}t)$ with $t \in
\mathbb{Z}$.



\paragraph{Exponential Diophantine Equation}\hfill
\label{sec:exponential_diophantine}



\subsubsection{Linear Equation}\label{sec:linear_equation}

System of Linear Equations (\S\ref{sec:linear_equation_system})



\subsubsection{Quadratic Equation}\label{sec:quadratic_equation}

Quadratic Function (\S\ref{sec:quadratic_function})



% --------------------------------------------------------------------
\subsection{Fundamental Theorem of Algebra}
\label{sec:fundamental_algebra_theorem}
% --------------------------------------------------------------------

states that the Field of Complex Numbers is Algebraically Closed
(\S\ref{sec:algebraically_closed})

(wiki):

\emph{Link between ``Algebra'' and ``Geometry''}: a Monic Polynomial
(Univariate with Complex Coefficients \S\ref{sec:monic_polynomial}), an
Algebraic Object, is determined by the Set of its Roots
(\S\ref{sec:function_root}), a Geometric Object, in the Complex Plane
(\S\ref{sec:complex_plane}).

\emph{Hilbert's Nullstellensatz}: fundamental correspondence between Ideals
(\S\ref{sec:ring_ideal}) of Polynomial Rings (\S\ref{sec:polynomial_ring}) and
Algebraic Sets (\S\ref{sec:algebraic_set})



% --------------------------------------------------------------------
\subsection{Tarski-Seidenberg Theorem}\label{sec:tarski_seidenberg}
% --------------------------------------------------------------------

the First-order Theory of the Real Closed Field (\S\ref{sec:real_closed}) is
Decidable: every Formula constructed from Polynomial Equations and Inequalities
by the Logical Connectives $\vee$, $\wedge$, $\neg$, and Quantifiers $\forall$,
$\exists$ is equivalent to a Formula without Quantifiers

Euclidean Geometry (\S\ref{sec:euclidean_geometry}) without the ability to
measure Angles is also a Model of the Real Field Axioms and therefore also
Decidable

Quantifier Elimination over the Reals is only possible in Double Exponential
Time (Collins Cylindrical Algebraic Decomposition
\S\ref{sec:cylindrical_algebraic_decomposition})

the Semialgebraic Sets in $\reals^n$ form an $o$-minimal Structure (TODO)



% ====================================================================
\section{Signomial}\label{sec:signomial}
% ====================================================================

Algebraic Extension (\S\ref{sec:algebraic_extension}) of Polyonomials
(\S\ref{sec:polynomial}) permitting exponents to be arbitrary Real Numbers but
requiring Independent Variables to be Strictly Positive



% --------------------------------------------------------------------
\subsection{Posynomial}\label{sec:posynomial}
% --------------------------------------------------------------------

(or \emph{Posinomial})

compared to Polynomials (\S\ref{sec:polynomial}), Posynomials must have Positive
Real Independent Variables and Coefficients with arbitrary Real Exponents, while
Polynomials must have Non-negative Integer Exponents but Independent Variables
and Coefficients can be arbitrary Real Numbers

Inequality Constriants in Geometric Programming
(\S\ref{sec:geometric_programming})

Globally Convex



% ====================================================================
\section{Algebraic Expression}\label{sec:algebraic_expression}
% ====================================================================

An \emph{Algebraic Expression} is an Expression including Arithmetic
(\S\ref{sec:arithmetic}) and Algebraic (Negative and Rational Exponent
\S\ref{sec:exponentiation}) Operations.

Radicals ($n$-th Roots \S\ref{sec:radical})

Algebraic Expressions do not include Transcendental Numbers
(\S\ref{sec:transcendental}), Trigonometric, Logarithmic, or Hyperbolic
Functions, nor Irrational Exponents; these are allowed by the wider class of
Closed-form, Analytic (\S\ref{sec:analytic_expression}), and unrestricted
Mathematical Expressions (\S\ref{sec:mathematical_expression}).

\fist Polynomial Expressions (\S\ref{sec:polynomial_expression}) are a narrower
class of Expressions that do not allow Rational Exponents ($n$-th Roots) or
Negative Integer Exponents.

finding Roots (\S\ref{sec:equation_root}) of $5$th-degree and higher-degree
Polynomials do not have a general Algebraic Solution



% --------------------------------------------------------------------
\subsection{Algebraic Fraction}\label{sec:algebraic_fraction}
% --------------------------------------------------------------------

\subsubsection{Rational Fraction}\label{sec:rational_fraction}

Algebraic Fraction such that both numerator and denominator are Polynomials



% --------------------------------------------------------------------
\subsection{Algebraic Function}\label{sec:algebraic_function}
% --------------------------------------------------------------------

An \emph{Algebraic Function} is a Function that can be defined as the Root
(\S\ref{sec:equation_root}) of a Polynomial Equation
(\S\ref{sec:polynomial_equation}).

all Algebraic Functions are Holonomic Functions
(\S\ref{sec:holonomic_function})



\subsubsection{Rational Function}\label{sec:rational_function}

Rational Fraction (\S\ref{sec:rational_fraction})

the Function Field (\S\ref{sec:function_field}) of an Algebraic Variety $V$
consists of objects interpreted as Rational Functions on $V$

Partial Fraction Decomposition



% ====================================================================
\section{Closed-form Expression}\label{sec:closed_form_expression}
% ====================================================================

(Laczkovich03) refinement of Richardson's Theorem
(\S\ref{sec:richardsons_theorem}); the use of $\pi$ can be removed and the use
of Composition reduced:

given an Expression $A(x)$ in the Ring generated by the Integers, $x$, $\sin
x^n$, and $\sin(x \sin x^n)$, the question whether $A(x) > 0$ for some $x$ and
whether $A(x) = 0$ for some $x$ are \emph{Unsolvable}



% --------------------------------------------------------------------
\subsection{Elementary Function}\label{sec:elementary_function}
% --------------------------------------------------------------------

the Set of Elementary Functions is Closed under Arithmetic Operations and
Composition



% ====================================================================
\section{Linear System of Divisors}\label{sec:linear_system_of_divisors}
% ====================================================================

generalization of a Family of Curves (\S\ref{sec:curve_family})

Function Field (\S\ref{sec:function_field}) of an Algebraic Variety $V$

\begin{itemize}
  \item Linear System of Conic Sections (\S\ref{sec:conic_section}) --
    Dimension $5$
  \item ...
\end{itemize}



% --------------------------------------------------------------------
\subsection{Pencil}\label{sec:pencil}
% --------------------------------------------------------------------

dimension $1$

\fist a \emph{Projective Range} (\S\ref{sec:projective_range}) in Projective
Geometry is Dual of a Pencil of Lines on a given Point, e.g. a Correlation
(\S\ref{sec:correlation}) exchanges the Points of a Projective Range with the
Lines of a Pencil



% --------------------------------------------------------------------
\subsection{Net}\label{sec:linear_divisor_net}
% --------------------------------------------------------------------

dimension $2$




% --------------------------------------------------------------------
\subsection{Web}\label{sec:web}
% --------------------------------------------------------------------

dimension $3$



% ====================================================================
\section{System of Equations}\label{sec:equation_system}
% ====================================================================

a Set of \emph{Simultaneous Equations}

a Finite Set of Equations (\S\ref{sec:equation}) for which a ``common
solution'' is desired (FIXME: clarify)

\begin{itemize}
  \item System of Polynomial Equations
    (\S\ref{sec:polynomial_equation_system})
  \begin{itemize}
    \item System of Linear Equations (\S\ref{sec:linear_equation_system})
  \end{itemize}
  \item System of Nonlinear Equations
    (\S\ref{sec:nonlinear_equation_system}) -- either Polynomials of Degree
    $>1$ or Unknowns appear as Variables of a general Function which is not a
    Polynomial of Degree $1$
\end{itemize}

%FIXME: move the following ?

the System of Equations:
\begin{align*}
  u & = u(x,y) \\
  v & = v(x,y)
\end{align*}
is \emph{Invertible} if $f = u + iv$ is Analytic and $f'(z) \neq 0$ \fist
Conformal Mapping (\S\ref{sec:conformal_map})

Numerical Analysis (\S\ref{sec:numerical_analysis}) applied to Solving Systems
of Equations:
\begin{itemize}
  \item \emph{Direct Methods} (\S\ref{sec:direct_method}) for Solving Systems of
    Linear Equations using Matrix Decomposition
    (\S\ref{sec:matrix_decomposition}):
    \begin{itemize}
      \item Gaussian Elimination (\S\ref{sec:gaussian_elimination})
      \item LU Decomposition (\S\ref{sec:lu_decomposition})
      \item Cholesky Decomposition (\S\ref{sec:cholesky_decomposition})
      \item QR Decomposition (\S\ref{sec:qr_decomposition})
    \end{itemize}
  \item \emph{Iterative Methods} (\S\ref{sec:iterative_method}), preferred for
    larger Systems:
    \begin{itemize}
      \item Jacobi Method (\S\ref{sec:jacobi_method})
      \item Gauss-Seidel Method (\S\ref{sec:gauss_seidel})
      \item Successive Over-relaxation (\S\ref{sec:successive_over_relaxation})
      \item Conjugate Gradient Method (\S\ref{sec:conjugate_gradient_method})
    \end{itemize}
  \item \emph{Root-finding Algorithms} (\S\ref{sec:root_finding}), used to
    Solve Non-linear Equations:
    \begin{itemize}
      \item Newton's Method (\S\ref{sec:newtons_method})
      \item Linearization (\S\ref{sec:linearization})
    \end{itemize}
\end{itemize}



% --------------------------------------------------------------------
\subsection{Unknown}\label{sec:unknown}
% --------------------------------------------------------------------

Variables of the Equations



\subsubsection{Underdetermined System}\label{sec:underdetermined_system}

a System that has been \emph{Underconstrained}

fewer Equations than Unknowns; usually Infinite number of Solutions

Linear Systems (\S\ref{sec:linear_equation_system}): System Matrix $A$ is
Rectangular $m \times n$ with $m < n$ (of Full Rank), Unknown Vector $\vec{x}$
is an $n \times 1$ Column Vector, and RHS $\vec{b}$ is an $m \times 1$ Column
Vector

UC Math 352 Lec. 8 - \url{https://www.youtube.com/watch?v=56vogBfKJu0}

find the Solution that is the ``smallest'': $\mathrm{min}\|\vec{x}\|^2$

use QR Decomposition (\S\ref{sec:qr_decomposition}) to find:
\[
  A^T = QR
\]
for the ``tall'' transpose of $A$ giving:
\[
  A = R^T Q^T = \hat{R}^T \hat{Q}^T
\]
define $\vec{w}$ to be the ``expansion'' of $\vec{x}$ in the Columns of $Q$:
\begin{align*}
  \vec{x} & = Q\vec{w} \\
  \vec{x} & = [\hat{Q}\vec{u} + Q_N\vec{v}] [\vec{u} \ \vec{v}]^T \\
  \vec{x} & = \hat{Q}\vec{u} + Q_N\vec{v} \\
\end{align*}
then $A\vec{x} = \vec{b}$ can be written as:
\[
  \hat{R}^T\hat{Q}^T\hat{Q}\vec{u} + \hat{R}^T\hat{Q}^TQ_N\vec{v} = \vec{b}
\]
which simplifies (due to the Orthogonality of $Q$) to:
\[
  \hat{R}^T\vec{u} = \vec{b}
\]
which can be solved for $\vec{u}$ efficiently by Forward Substitution
(\S\ref{sec:forward_substitution}) since $\hat{R}$ is a Lower Triangular Matrix

since $\vec{v}$ has no constraints, it can be $\vec{0}$, so:
\[
  \vec{x} = \hat{Q}\hat{R}^{-T}\vec{b}
\]
and:
\[
  \|\vec{x}\|^2 = \|\vec{u}\|^2
\]



\subsubsection{Overdetermined System}\label{sec:overdetermined_system}

a System that has been \emph{Overconstrained}

more Equations than Unknowns; usually no exact Solution is possible

find the Solution that is ``closest''

Linear Systems (\S\ref{sec:linear_equation_system}): System Matrix $A$ is
Rectangular $m \times n$ with $m > n$, Unknown Vector $\vec{x}$ is an $n \times
1$ Column Vector, and RHS $\vec{b}$ is an $m \times 1$ Column Vector

Linear Least Squares (\S\ref{sec:linear_least_squares}) \emph{Normal Equations}
(\S\ref{sec:normal_equation}):

instead of finding an exact Solution, find the $\vec{x}$ that \emph{minimizes}
the (Euclidean) Norm $\|\vec{r}\|^2$ of the \emph{Residual}
(\S\ref{sec:regression_residual}) $\vec{r}$ to the desired RHS $\vec{b}$:
\[
  A\vec{x} - \vec{b} = \vec{r}
\]
e.g. find coefficients for a an equation when a number of noisy samples have
been recorded -- UC Math 352 - \url{https://www.youtube.com/watch?v=ZWGIchXVbho}

the Optimal Solution is then the Orthogonal Projection of $\vec{b}$ onto the
Column Space of $A$

Orthogonal Projection Matrix:
\[
  P = A(A^TA)^{-1}A^T
\]
so that $P\vec{b} = A\vec{x}$

and:
\[
  \vec{x} = (A^TA)^{-1}A^T\vec{b}
\]

Normal Equation:

$(A^TA)\vec{x} = A^T\vec{b}$



\subsubsection{Consistent System}\label{sec:consistent_system}

has a Solution



\subsubsection{Inconsistent System}\label{sec:inconsistent_system}

no Solution



% --------------------------------------------------------------------
\subsection{Solution Set}\label{sec:solution_set}
% --------------------------------------------------------------------

A \emph{Solution Set} is the Set of Values that Satisfy a given System of
Equations or Inequalities.

The \emph{Feasible Region} (\S\ref{sec:feasible_region}) of a Constrained
Optimization Problem (\S\ref{sec:constrained_optimization}) is the Solution Set
of the Constraints (\S\ref{sec:constraint}).

An \emph{Algebraic Variety} (\S\ref{sec:algebraic_variety}) can be defined as
the Solution Set of a System of Polynomial Equations
(\S\ref{sec:polynomial_equation_system})

Solution Sets are called Algebraic Sets (\S\ref{sec:algebraic_set}) if there
are no Inequalities.

Solution Sets over the Reals and with Inequalities are called Semialgebraic
Sets (\S\ref{sec:semialgebraic_set}).



% --------------------------------------------------------------------
\subsection{System of Polynomial Equations}
\label{sec:polynomial_equation_system}
% --------------------------------------------------------------------

Polynomial Equation (\S\ref{sec:polynomial_equation})

Algebraically Closed (\S\ref{sec:algebraically_closed}) Field
Extension (\S\ref{sec:field_extension})

An \emph{Algebraic Variety} (\S\ref{sec:algebraic_variety}) can be defined as
the Solution Set (\S\ref{sec:solution_set}) of a System of Polynomial Equations.



\subsubsection{System of Linear Equations}\label{sec:linear_equation_system}

\emph{Linear System (LS)}

Linear Equation (\S\ref{sec:linear_equation})

Forward Substitution (Lower Triangular Matrices
\S\ref{sec:forward_substitution})

Back Substitution (Upper Triangular Matrices \S\ref{sec:back_substitution})

a System of Linear Equations of the form:
\begin{align*}
  a_{11}x_1 + \cdots + a_{1n}x_n & = b_1 \\
  a_{21}x_1 + \cdots + a_{2n}x_n & = b_2 \\
                                 & \vdots \\
  a_{m1}x_1 + \cdots + a_{mn}x_n & = b_m
\end{align*}
is equivalent to the Matrix Equation:
\[
  \mathbf{A}\vec{x} = \vec{b}
\]

every Affine Set (\S\ref{sec:affine_set}) can be expressed as the Solution Set
of a System of Linear Equations

Algorithms:
\begin{itemize}
\item \emph{Direct Methods} (\S\ref{sec:direct_method}) for Solving
  Systems of Linear Equations using Matrix Decomposition
  (\S\ref{sec:matrix_decomposition}):
  \begin{itemize}
    \item Gaussian Elimination (\S\ref{sec:gaussian_elimination})
    \item LU Decomposition (\S\ref{sec:lu_decomposition})
    \item Cholesky Decomposition (\S\ref{sec:cholesky_decomposition})
    \item QR Decomposition (\S\ref{sec:qr_decomposition})
  \end{itemize}
\item \emph{Iterative Methods} (\S\ref{sec:iterative_method}), preferred
  for larger Systems:
  \begin{itemize}
    \item Jacobi Method (\S\ref{sec:jacobi_method})
    \item Gauss-Seidel Method (\S\ref{sec:gauss_seidel})
    \item Successive Over-relaxation
      (\S\ref{sec:successive_over_relaxation})
    \item Conjugate Gradient Method
      (\S\ref{sec:conjugate_gradient_method})
  \end{itemize}
\end{itemize}



\paragraph{Coefficient Matrix}\label{sec:coefficient_matrix}\hfill

Augmented Matrix

Reduced Row Echelon Form (\S\ref{sec:reduced_row_echelon})



\paragraph{Independent Equation}\label{sec:independent_equation}\hfill

\paragraph{Homogeneous System of Linear Equations}\hfill
\label{sec:homogeneous_linear_equation_system}

cf. Homogeneous Polynomial (\S\ref{sec:homogeneous_polynomial}), Homogeneous
Differential Equation (\S\ref{sec:homogeneous_differential_equation}),
Homogeneous Difference Equation (\S\ref{sec:homogeneous_differential_equation})



\subsubsection{System of Bilinear Equations}\label{sec:bilinear_equation_system}

each Equations written as a Bilinear Form (\S\ref{sec:bilinear_form})



\subsubsection{Cylindrical Algebraic Decomposition}
\label{sec:cylindrical_algebraic_decomposition}

Complexity: Double Exponential (\S\ref{sec:double_exponential})

Quantifier Elimination over Reals \fist Tarski-Seidenberg Theorem
(\S\ref{sec:tarski_seidenberg})



% --------------------------------------------------------------------
\subsection{System of Nonlinear Equations}\label{sec:nonlinear_equation_system}
% --------------------------------------------------------------------

Non-Linear System (NLS)

a Set of Simultaneous Equations in which the Unknowns or Unknown Functions
appear as Variables of a Polynomial of Degree higher than $1$, \emph{or} in the
Argument of a Function which is not a Polynomial of Degree $1$; i.e. a
Nonlinear System of Equations to be solved cannot be written as a Linear
Combination of Unknown Variables or Functions that appear in them

\fist Nonlinear Programming (\S\ref{sec:nonlinear_programming})

\fist Iterative Methods (\S\ref{sec:iterative_method})



% --------------------------------------------------------------------
\subsection{System of Difference Equations}
\label{sec:difference_equation_system}
% --------------------------------------------------------------------

or \emph{Matrix Difference Equation}

Difference Equations (\S\ref{sec:difference_equation})



% --------------------------------------------------------------------
\subsection{System of Ordinary Differential Equations}\label{sec:ode_system}
% --------------------------------------------------------------------

a number of ``coupled'' Differential Equations (\S\ref{sec:ode})

Any ODE of Order greater than one can be rewritten as a System of (Coupled)
First-order ODEs.

\fist Dynamical Systems (\S\ref{sec:dynamical_system}); Lorenz Equations
(\S\ref{sec:lorenz_system})

\fist the Jacobian Matrix (\S\ref{sec:jacobian}) for a System of DAEs
(\S\ref{sec:dae_system}) is a Singular (Non-invertible) Matrix
(\S\ref{sec:singular_matrix})



\subsubsection{System of Linear Differential Equations}
\label{sec:linear_ode_system}

a Holonomic Function (\S\ref{sec:holonomic_function}) is a Multivariable Smooth
Function that is a solution to a (System of) Homogeneous Linear Differential
Equation(s) with Polynomial Coefficients



\subsubsection{Autonomous System of Ordinary Differential Equations}
\label{sec:autonomous_ode_system}

a System of ODEs that does not ``explicitly'' depend on the Independent Variable

Autonomous Differential Equation (\S\ref{sec:autonomous_differential_equation})

cf. Autonomous Dynamical System (\S\ref{sec:autonomous_dynamical_system})

when the Variable is Time they are also called \emph{Time-Invariant Systems}
(\S\ref{sec:tiv_system})

(wiki): ``any Autonomous System can be transformed into a Dynamical System and,
using very weak assumptions, a Dynamical System can be transformed into an
Autonomous System''



% --------------------------------------------------------------------
\subsection{System of Partial Differential Equations}\label{sec:pde_system}
% --------------------------------------------------------------------

\subsubsection{System of Linear Partial Differential Equations}
\label{sec:linear_pde_system}

\fist Algebraic Analysis (\S\ref{sec:algebraic_analysis}) -- study of Systems of
Linear PDEs using Sheaf Theory (\S\ref{sec:sheaf_theory}) and Complex Analysis
(\S\ref{sec:complex_analysis})



% --------------------------------------------------------------------
\subsection{System of Differential-Algebraic Equations}\label{sec:dae_system}
% --------------------------------------------------------------------

(wiki):

a System of Equations that either contains Differential Equations
(\S\ref{sec:differential_equation}) and Algebraic (Polynomial) Equations
(\S\ref{sec:polynomial_equation}), or is equivalent to such a System

\fist Constraints of Multibody Systems (\S\ref{sec:multibody_system})

general form of System of Differential equations for Vector-valued Functions
$x$ in one Independent Variable $t$:
\[
  \vec{f}(\vec{x}'(t), \vec{x}(t), t) = 0
\]
where $\vec{x} : [a,b] \rightarrow \reals^n$ is a Vector of $n$ Dependent
Variables:
\[
  \vec{x}(t) = (x_1(t), \ldots, x_n(t))
\]
and the System has $n$ Equations:
\[
  \vec{f} = (\vec{f}_1, \ldots, \vec{f}_n) : \reals^{2n+1} \rightarrow \reals^n
\]

Solution consists of a search for Consistent Initial Values and a computed
Trajectory (\S\ref{sec:trajectory})

a DAE is not completely Solvable for the Derivatives of \emph{all} Components
of the Function $\vec{x}$ because they may not \emph{appear}, i.e. \emph{some}
Equations are \emph{Algebraic} (Polynomial)

compared to general Systems of ODEs (\S\ref{sec:ode_system}), the Jacobian
Matrix (\S\ref{sec:jacobian}) for a System of DAEs is a Singular
(Non-invertible) Matrix (\S\ref{sec:singular_matrix})

in practical terms, the Solution to a DAE System depends on the Derivatives of
the \emph{Input Signal} (\S\ref{sec:signal_flow}), and not just the ``Signal
itself'' as in the case of ODEs
(FIXME: clarify); e.g. see Systems with Hysteresis such as the Schmitt Trigger



% ====================================================================
\section{Algebraic Variety}\label{sec:algebraic_variety}
% ====================================================================

Dimension 2: Algebraic Surface (\S\ref{sec:algebraic_surface})

Scheme Theory (\S\ref{sec:scheme_theory}): Zariski Topology
(\S\ref{sec:zariski_topology}) allows Algebraic Varieties to be built by
``gluing together'' Affine Varieties (\S\ref{sec:affine_variety})

cf. Manifolds (\S\ref{sec:manifold}) built by ``gluing together'' Charts
(\S\ref{sec:chart}, Open Subsets of Real Affine Spaces
\S\ref{sec:affine_space}); Submanifolds (\S\ref{sec:submanifold})

\fist Projective Variety (Homogeneous Polynomials
\S\ref{sec:projective_variety})

Ground Field $K$

Affine Space (\S\ref{sec:affine_space}) $\mathrm{A}^n = K^n$

\fist $\mathbb{F}_1$-variety (\S\ref{sec:f1_variety})


(wiki):

An \emph{Algebraic Variety} is defined as the Solution Set
(\S\ref{sec:solution_set}) of a System of Polynomial Equations over the Real or
Complex Numbers.

\fist an \emph{Algebraic Manifold} (\S\ref{sec:algebraic_manifold}) is an
Algebraic Variety which is also an $m$-dimensional Manifold, i.e. every
sufficiently small Local Patch is Isomorphic to $k^m$ or equivalently the
Variety is \emph{Smooth} (free from Singular Points)

cf. Analytic Manifolds (\S\ref{sec:analytic_manifold})



% --------------------------------------------------------------------
\subsection{Singular Point}\label{sec:singular_point}
% --------------------------------------------------------------------

a Point of an Algebraic Variety where the Tangent Space is not ``regularly
defined'' (FIXME: clarify)

Analytic Manifolds (\S\ref{sec:analytic_manifold}) are not allowed to have
Singular Points

for Algebraic Varieties defined over the Reals, Singular Points are a
generalization of Local Non-flatness (\S\ref{sec:locally_flat})

a \emph{Node} is a Singular Point where the Hessian Matrix (Matrix of
Second-order Partial Derivatives) is Non-singular



% --------------------------------------------------------------------
\subsection{Motive}\label{sec:motive}
% --------------------------------------------------------------------

cf. Arithmetic Cryptography (\S\ref{sec:arithmetic_cryptography})

Motivic Homotopy Theory (\S\ref{sec:motivic_homotopy})



\subsubsection{Pure Motive}\label{sec:pure_motive}

\subsubsection{Mixed Motive}\label{sec:mixed_motive}



% --------------------------------------------------------------------
\subsection{Correspondence}\label{sec:variety_correspondence}
% --------------------------------------------------------------------

% --------------------------------------------------------------------
\subsection{Algebraic Manifold}\label{sec:algebraic_manifold}
% --------------------------------------------------------------------

an \emph{Algebraic Manifold} is an Algebraic Variety that is also an
$m$-dimensional Manifold, i.e. a \emph{Smooth} Algebraic Variety:
(free from Singular Points) every sufficiently small Local Patch is Isomorphic
to $K^m$ where $K$ is the Ground Field

generalization of the concept of \emph{Smooth Curves} and \emph{Smooth
  Surfaces} defined by Polynomials

cf. Smooth Manifold (\S\ref{sec:smooth_manifold}), Analytic Manifold
(\S\ref{sec:analytic_manifold})

\begin{itemize}
  \item the Unit Sphere defined as the Zero Set of the Polynomial
    $x^2 + y^2 + z^2 - 1$
  \item the Riemann Sphere (\S\ref{sec:riemann_sphere}) is a Complex Algebraic
    Manifold
  \item Elliptic Curves (\S\ref{sec:elliptic_curve})
\end{itemize}



\subsubsection{Nash Manifold}\label{sec:nash_manifold}

Algebraic Manifold with $\reals$ as the Ground Field



% --------------------------------------------------------------------
\subsection{Projective Variety}\label{sec:projective_variety}
% --------------------------------------------------------------------

is a Subset of some Projective $n$-space $\xspace{P}^n$ over an Algebraically
Closed Field $K$ that is the Zero-locus of some Finite Family of Homogenous
Polynomials (\S\ref{sec:homogeneous_polynomial}) of $n + 1$ Variables with
Coefficients in $K$ that generate a Prime Ideal that is the ``defining Ideal''
of the Variety (FIXME: clarify)

the Twisted Cubic (\S\ref{sec:twisted_cubic}) is the simplest example of a
Projective Variety that is not Linear or a Hypersurface



\subsubsection{Weighted Projective Space}\label{sec:weighted_projective_space}



% --------------------------------------------------------------------
\subsection{Irreducible Algebraic Variety}
\label{sec:irreducible_algebraic_variety}
% --------------------------------------------------------------------

not the Union of two smaller Sets that are Closed in the Zariski Topology
(\S\ref{sec:zariski_topology})



% --------------------------------------------------------------------
\subsection{Rational Variety}\label{sec:rational_variety}
% --------------------------------------------------------------------

Unirational



% --------------------------------------------------------------------
\subsection{Algebraic Set}\label{sec:algebraic_set}
% --------------------------------------------------------------------

a Non-irreducible Algebraic Variety

Solution Sets (\S\ref{sec:solution_set}) are called Algebraic Sets if there are
no Inequalities.

Solution Sets over the Reals and with Inequalities are called Semialgebraic
Sets (\S\ref{sec:semialgebraic_set}).

\emph{Hilbert's Nullstellensatz}: fundamental correspondence between Ideals
(\S\ref{sec:ring_ideal}) of Polynomial Rings (\S\ref{sec:polynomial_ring}) and
Algebraic Sets



\subsubsection{Affine Algebraic Set}\label{sec:affine_algebraic_set}

an Affine Algebraic Set (Affine Variety \S\ref{sec:algebraic_set}) is the
Intersection of the Zero Sets (\S\ref{sec:zero_set}) of a number of Polynomials
in a Polynomial Ring (\S\ref{sec:polynomial_ring}) $k[x_1,\ldots,x_n]$ over a
Field

(FIXME: same as affine algebraic set ???)
%FIXME: is this synonymous with algebraic set ???

%FIXME: synonymous with affine variety ???

Affine Set (\S\ref{sec:affine_set})

Affine Variety (\S\ref{sec:affine_variety})

Subset of an Affine Space (\S\ref{sec:affine_space}) that is the Zero Set of
Systems of Polynomials:

\[
  Z(S) = \{ p \in A^n | f(p) = 0 \forall_{f \in S} \} \subset A^n
\]



% --------------------------------------------------------------------
\subsection{Generic Property}\label{sec:generic_property}
% --------------------------------------------------------------------

\subsubsection{Generic Point}\label{sec:generic_point}

\fist General Position (\S\ref{sec:general_position})

\fist A \emph{Sober Space} (\S\ref{sec:sober_space}) is a Topological Space $X$
such that every Irreducible Closed Subset has a unique Generic Point.



% --------------------------------------------------------------------
\subsection{Function Field}\label{sec:function_field}
% --------------------------------------------------------------------

objects on $V$ which are interpreted as Rational Functions
(\S\ref{sec:rational_function}) on $V$

\fist Linear System of Divisors (\S\ref{sec:linear_system_of_divisors})



% --------------------------------------------------------------------
\subsection{Regular Map}\label{sec:regular_map}
% --------------------------------------------------------------------

a Regular Map between Complex Algebraic Varieties is a Holomorphic Map
(\S\ref{sec:holomorphic_function})

An Algebraic Group (or \emph{Group Variety} \S\ref{sec:algebraic_variety}) is a
Group that is an Algebraic Variety such that the Multiplication and Inversion
Operations are given by Regular Maps (\S\ref{sec:regular_map}) on the Variety.



\subsection{Regular Function}\label{sec:regular_function}

a Morphism from an Algebraic Variety to the Affine Line
(\S\ref{sec:affine_line})



% --------------------------------------------------------------------
\subsection{Analytic Variety}\label{sec:analytic_variety}
% --------------------------------------------------------------------

defined locally as the Set of common Solutions of several Equations involving
Analytic Functions (\S\ref{sec:analytic_function})

any Complex Manifold (\S\ref{sec:complex_manifold}) is an Analytic Variety



\subsubsection{Analytic Space}\label{sec:analytic_space}

locally the same as an Analytic Variety

%FIXME: move this section???



% --------------------------------------------------------------------
\subsection{Affine Variety}\label{sec:affine_variety}
% --------------------------------------------------------------------

an Affine Variety (Affine Algebraic Set \S\ref{sec:algebraic_set}) is the
Intersection of the Zero Sets (\S\ref{sec:zero_set}) of a number of Polynomials
in a Polynomial Ring (\S\ref{sec:polynomial_ring}) $k[x_1,\ldots,x_n]$ over a
Field

(FIXME: same as affine algebraic set ???)

\fist Diophantine Geometry (\S\ref{sec:diophantine_geometry}): Rational Points
(\S\ref{sec:rational_point})

(Eisenbud-Harris 2000):

\fist an Affine Scheme (\S\ref{sec:affine_scheme}) generalizes the relationship
between an Affine Varity and its \emph{Coordinate Ring}
(\S\ref{sec:coordinate_ring}):

Affine Schemes are Bijective with (Unital) Commutative Rings as Affine Varities
are to Finitely Generated, Nilpotent-free Rings over Algebraically Closed Field
$K$



\subsubsection{Coordinate Ring}\label{sec:coordinate_ring}

for an Affine Variety $X$ defined by a Prime Ideal $I$, the Quotient Ring
$k[x_1,\ldots,x_n]/I$ is called the \emph{Coordinate Ring} of $X$



% --------------------------------------------------------------------
\subsection{Toric Variety}\label{sec:toric_variety}
% --------------------------------------------------------------------

an Algebraic Variety containing an Algebraic Torus (\S\ref{sec:algebraic_torus})
as a Dense Subset

\fist Toric Geometry (\S\ref{sec:toric_geometry})



% --------------------------------------------------------------------
\subsection{Characteristic Set}\label{sec:characteristic_set}
% --------------------------------------------------------------------

\subsubsection{Regular Chain}\label{sec:regular_chain}



% --------------------------------------------------------------------
\subsection{Algebraic Surface}\label{sec:algebraic_surface}
% --------------------------------------------------------------------

Algebraic Variety of Dimension 2



% --------------------------------------------------------------------
\subsection{Zariski Tangent Space}\label{sec:zariski_space}
% --------------------------------------------------------------------

Tangent Space (Topology \S\ref{sec:tangent_space})

Tangent Space at a Point of an Algebraic Variety $V$ giving a Vector Space
(\S\ref{sec:vector_space}) $V$ with Dimension at least that of $V$

Points are called \emph{Singular Points} when the Dimension of the Tangent
Space is higher than $V$ at that Point, are called \emph{Non-singular Points}
when the Dimension of the Tangent Space is equal to that of $V$ at that Point



% --------------------------------------------------------------------
\subsection{Algebraic Group}\label{sec:algebraic_group}
% --------------------------------------------------------------------

(wiki):

An \emph{Algebraic Group} (or \emph{Group Variety}) is a Group that is an
Algebraic Variety such that the Multiplication and Inversion Operations are
given by Regular Maps (\S\ref{sec:regular_map}) on the Variety.

In Category Theoretic terms, an Algebraic Group is a Group Object
(\S\ref{sec:group_object}) in the Category of Algebraic Varieties.

\fist an Abelian Variety (\S\ref{sec:abelian_variety}) is a Projective Algebraic
Variety that is also an Algebraic Group

\begin{itemize}
  \item Finite Groups (\S\ref{sec:finite_group})
  \item General Linear Groups (\S\ref{sec:general_linear_group})
  \item Jet Groups (\S\ref{sec:jet_group})
  \item Elliptic Curves (\S\ref{sec:elliptic_curve})
\end{itemize}



\subsubsection{Isogeny}\label{sec:isogeny}

Morphism of Algebraic Groups that is Surjective and has a Finite Kernel

Post-quantum Cryptography: Supersingular Isogeny Diffie-Hellman (SIDH) key
exchange



\subsubsection{Linear Algebraic Group}\label{sec:linear_algebraic_group}

Subgroup of the Group of Invertible $n \times n$ Matrices defined by Polynomial
Equations

General Linear Group (\S\ref{sec:general_linear_group})



\subsubsection{Algebraic Torus}\label{sec:algebraic_torus}

an Toric Variety (\S\ref{sec:toric_variety}) is an Algebraic Variety containing
an Algebraic Torus as a Dense Subset



\subsubsection{Borel Subgroup}\label{sec:borel_subgroup}

\begin{itemize}
  \item the Group of Invertible Upper (or Lower) Triangular Matrices is a Borel
    Subgroup of the General Linear Group $GL_n$
\end{itemize}



% --------------------------------------------------------------------
\subsection{Projective Algebraic Variety}
\label{sec:projective_algebraic_variety}
% --------------------------------------------------------------------

\subsubsection{Abelian Variety}\label{sec:abelian_variety}

Projective Algebraic Variety that is also an Algebraic Group (Group Variety
\S\ref{sec:algebraic_group}), i.e. as a Group it is an Algebraic Variety
(\S\ref{sec:algebraic_variety})

as a Topological Group (\S\ref{sec:topological_group}), an Abelian Variety is a
Torus %FIXME

\fist Arithmetic Cryptography (\S\ref{sec:arithmetic_cryptography}) based on
the Discrete Logarithm Problem (\S\ref{sec:discrete_logarithm}) on Elliptic
Curves (\S\ref{sec:elliptic_curve}) or more general Abelian Varieties over
Finite Fields (\S\ref{sec:finite_field})



% --------------------------------------------------------------------
\subsection{Weil Conjectures}\label{sec:weil_conjectures}
% --------------------------------------------------------------------

Propositions on the Generating Functions (\S\ref{sec:generating_function})
derived from counting the number of Points on Algebraic Varieties over Finite
Fields (\S\ref{sec:finite_field})



% ====================================================================
\section{Real Algebraic Geometry}\label{sec:real_algebraic_geometry}
% ====================================================================

Real Analytic Geometry (\S\ref{sec:real_analytic_geometry})



% --------------------------------------------------------------------
\subsection{Semialgebraic Function}\label{sec:semialgebraic_function}
% --------------------------------------------------------------------

a Real Closed Ring (\S\ref{sec:real_closed_ring}) is a Commutative Ring $A$
that is a Subring of the Product of Real Closed Fields
(\S\ref{sec:real_closed}) which is Closed under Continuous Semi-algebraic
Functions defined over the Integers

the Real Closure of the Polynomial Ring (\S\ref{sec:polynomial_ring})
$\reals[T_1,\ldots,T_n]$ is the Ring of Continuous Semi-algebraic Functions
$\reals^n \rightarrow \reals$



% --------------------------------------------------------------------
\subsection{Semialgebraic Set}\label{sec:semialgebraic_set}
% --------------------------------------------------------------------

Decision Problem for the Existential Theory of the Reals (TODO) is equivalent
to the problem of testing whether a given Semialgebraic Set is Non-empty and is
$NP-hard$ lyingin $PSPACE$

Geometric Graph Theory

Solution Sets (\S\ref{sec:solution_set}) over the Reals and with Inequalities
are called Semialgebraic Sets.



% --------------------------------------------------------------------
\subsection{Semialgebraic Geometry}\label{sec:semialgebraic_geometry}
% --------------------------------------------------------------------



% ====================================================================
\section{Complex Algebraic Geometry}
\label{sec:complex_algebraic_geometry}
% ====================================================================

% ====================================================================
\section{Scheme Theory}\label{sec:scheme_theory}
% ====================================================================

generalization of Algebraic Geometry from Fields to other types of Rings

\fist cf. \emph{Global Analytic Geometry} (\S\ref{sec:global_analytic_geometry})
combines Non-archimedean (\S\ref{sec:nonarchimedean_analytic_geometry}) and
Archimedean Analytic Geometry (Part \ref{part:analytic_geometry}) and contains
Algebraic Geometry as a \emph{Sub-theory}; treats all Places
(\S\ref{sec:place}) ``on equal footing'' in contrast to Scheme Theory

\fist Monoid Schemes ($\mathbb{F}_1$-geometry \S\ref{sec:monoid_scheme})



% --------------------------------------------------------------------
\subsection{Zariski Topology}\label{sec:zariski_topology}
% --------------------------------------------------------------------

allows Algebraic Varieties to be built by ``gluing together'' Affine Varieties
 (\S\ref{sec:affine_variety})

cf. Manifolds (\S\ref{sec:manifold}) built by ``gluing together'' Charts
(\S\ref{sec:chart}, Open Subsets of Real Affine Spaces
\S\ref{sec:affine_space})

an Irreducible Algebraic Variety (\S\ref{sec:irreducible_algebraic_variety}) is
an Algebraic Variety that is not the Union of two smaller Sets that are Closed
in the Zariski Topology



% --------------------------------------------------------------------
\subsection{Scheme}\label{sec:scheme}
% --------------------------------------------------------------------

(Eisenbud-Harris 2000):

as Manifolds (\S\ref{sec:manifold}) are made by ``gluing together'' Open Balls
from Euclidean Space, \emph{Schemes} are made by ``gluing together'' Open Sets
called \emph{Affine Schemes} (\S\ref{sec:affine_scheme}); unlike Manifolds the
Open Neighborhoods of Points in Schemes are not necessarily Isomorphic

(wiki):

A \emph{Scheme} is a Topological Space together with Commutative Rings for all
its Open Sets which arise from ``gluing together'' \emph{Spectra} (Spaces of
Prime Ideals \S\ref{sec:ring_spectrum}) of Commutative Rings along their Open
Subsets, i.e. a Scheme is a \emph{Ringed Space} (\S\ref{sec:ringed_space})
which is locally a Spectrum of a Commutative Ring-- Schemes are Locally Ringed
Spaces (\S\ref{sec:locally_ringed_space}) obtained by ``gluing together''
Spectra of Commutative Rings

a Scheme is a Locally Ringed Space $X$ admitting a Covering by Open Sets $U_i$
such that each $U_i$ (as a Locally Ringed Space) is an Affine Scheme
(\S\ref{sec:affine_scheme})

\asterism

\fist Arithmetic Geometry (\S\ref{sec:arithmetic_geometry}): an Arithmetic
Surface (\S\ref{sec:arithmetic_surface}) $S$ over a Dedekind Domain
(\S\ref{sec:dedekind_domain}) $R$ is a Scheme with a Morphism $p : S \rightarrow
\Spec(R)$ with the Properties:
\begin{itemize}
  \item $S$ is Integral Normal, Excellent, Flat, and of Finite Type over $R$
  \item the Generic Fiber is a Non-singular, Connected Projective Curve over
    $\mathrm{Frac}(R)$
  \item for other $t \in \Spec(R)$:
    \[
      S \times_{\Spec(R)} \Spec(k_t)
    \]
    is a Union of Curves over $R / t$
\end{itemize}
(FIXME: xrefs)

\fist Arithmetic Cryptography (\S\ref{sec:arithmetic_cryptography}):
description of Public Key Cryptography Systems based on the use of the
Arithmetic Geometry (\S\ref{sec:arithmetic_geometry}) of Schemes (or Gobal
Analytic Spaces \S\ref{sec:global_analytic_space}) over $\ints$

Etale Topos (\S\ref{sec:etale_topos})

Locale (\S\ref{sec:locale})



\subsubsection{Affine Scheme}\label{sec:affine_scheme}

(Eisenbud-Harris 2000):

as Manifolds (\S\ref{sec:manifold}) are made by ``gluing together'' Open Balls
from Euclidean Space, \emph{Schemes} are made by ``gluing together'' Open Sets
called \emph{Affine Schemes}; unlike Manifolds the Open Neighborhoods of Points
in Schemes are not necessarily Isomorphic

Affine Schemes are Bijective with (Unital) Commutative Rings as Affine Varities
are to Finitely Generated, Nilpotent-free Rings over Algebraically Closed Field
$K$

an Affine Scheme is a Locally Ringed Space (\S\ref{sec:locally_ringed_space}) in
form form of the Spectrum (\S\ref{sec:ring_spectrum}) $\Spec(R)$ of a
Commutative Ring $R$

a Point $[p]$ of $\Spec(R)$ is a Prime Ideal (\S\ref{sec:prime_ideal}) of $R$
corresponding to the Prime $p$ of $R$

(wiki):

an Affine Scheme is a Locally Ringed Space (\S\ref{sec:locally_ringed_space})
that is Isomorphic to the Spectrum $Spec(R)$ of a Commutative Ring $R$;
$Spec(R)$ augmented with the Zariski Topology (\S\ref{sec:zariski_topology}) and
a Structure Sheaf (\S\ref{sec:structure_sheaf}) turns it into a Locally Ringed
Space

an Affine Scheme is a Scheme that is Representable
(\S\ref{sec:representable_functor}) as a Sheaf (\S\ref{sec:sheaf}) on the
Opposite Category $\cat{CRing}^{op}$ of Commutative Rings, \emph{a.k.a.} the
Category of Affine Schemes $\cat{Aff}$

(FIXME: clarify)

Noncommutative Geometry: Category of Noncommutative Affine Schemes
(\S\ref{sec:affine_scheme}) as Dual of the Category of Associative Unital Rings

a Biring (\S\ref{sec:biring}) is both a Ring and Co-ring, i.e. it is a
Commutative Ring Object in the Category of Affine Schemes



\subsubsection{Projective Scheme}\label{sec:projective_scheme}

\paragraph{Genus}\label{sec:scheme_genus}\hfill

whan a Projective Algebraic Scheme $X$ is an Algebraic Curve over the Field of
Complex Numbers and has no Singular Points, then the definitions of Arithmetic
Genus (\S\ref{sec:arithmetic_genus}) and Geometric Genus
(\S\ref{sec:geometric_genus}) agree and coincide with the Topological
Genus (\S\ref{sec:genus}) applied to the Riemann Surface of $X$ (i.e. its
Manifold of Complex Points)

\fist cf. Spinor Genus (Quadratic Forms \S\ref{sec:spinor_genus})

\begin{itemize}
  \item an Elliptic Curve (\S\ref{sec:elliptic_curve}) is a Smooth, Projective,
    Algebraic Curve of Genus $1$, with a given Rational Point
    (\S\ref{sec:rational_point}) on it
\end{itemize}



\subparagraph{Arithmetic Genus}\label{sec:arithmetic_genus}\hfill

\subparagraph{Geometric Genus}\label{sec:geometric_genus}\hfill

a Birational Invariant of Algebraic Varieties and Complex Manifolds



% --------------------------------------------------------------------
\subsection{Algebraic Space}\label{sec:algebraic_space}
% --------------------------------------------------------------------

generalization of Schemes



% --------------------------------------------------------------------
\subsection{Algebraic Fundamental Group}\label{sec:algebraic_fundamental_group}
% --------------------------------------------------------------------

or \emph{\'Etale Fundamental Group}

cf. Etale Topos, Etale Topology

cf. Fundamental Group (Algebraic Topology \S\ref{sec:fundamental_group}) of a
Topological Space (\S\ref{sec:topological_space})



% --------------------------------------------------------------------
\subsection{Moduli Space}\label{sec:moduli_space}
% --------------------------------------------------------------------

a kind of Space with Points representing Geometric Objects
(\S\ref{sec:geometric_object}) of some kind

Spaces of \emph{Parameters} rather than Spaces of \emph{Objects}

cf. Phase Space (\S\ref{sec:phase_space})

Fine Moduli Space

Coarse Moduli Space

a Configuration Space (\S\ref{sec:configuration_space}) is a type of (Fine)
Moduli Space



% --------------------------------------------------------------------
\subsection{Formal Scheme}\label{sec:formal_scheme}
% --------------------------------------------------------------------

cf. Formal Derivative (\S\ref{sec:formal_derivative})



% ====================================================================
\section{Arithmetic Geometry}\label{sec:arithmetic_geometry}
% ====================================================================

or \emph{Arithmetic Algebraic Geometry}

Zero Sets over the Field of Rationals $\rats$ (\S\ref{sec:rational})

cf. Number Theory (\S\ref{sec:number_theory})

Fermat's Last Theorem

some J.S. Milne books in PDF format:
\url{http://www.jmilne.org/math/Books/index.html}



% --------------------------------------------------------------------
\subsection{Diophantine Geometry}\label{sec:diophantine_geometry}
% --------------------------------------------------------------------

\subsubsection{Rational Point}\label{sec:rational_point}

Solution of a Set of Polynomial Equations in a given Field



\subsubsection{Arithmetic Curve}\label{sec:arithmetic_curve}

%FIXME

\subsubsection{Arithmetic Surface}\label{sec:arithmetic_surface}

(wiki):

An \emph{Arithmetic Surface} $S$ over a Dedekind Domain
(\S\ref{sec:dedekind_domain}) $R$ is a Scheme with a Morphism $p : S \rightarrow
\Spec(R)$ with the Properties:
\begin{itemize}
  \item $S$ is Integral Normal, Excellent, Flat, and of Finite Type over $R$
  \item the Generic Fiber is a Non-singular, Connected Projective Curve over
    $\Frac(R)$ (\S\ref{sec:fraction_field})
  \item for other $t \in \Spec(R)$:
    \[
      S \times_{\Spec(R)} \Spec(k_t)
    \]
    is a Union of Curves over $R / t$
\end{itemize}
(FIXME: xrefs)

an Arithmetic Surface is a Geometric Object having one conventional Dimension,
and the other Dimension provided by the \emph{Infinitude of the Primes}

when $R$ is the Ring of Integers, $\ints$, the Prime Ideal Spectrum
$\Spec(\ints)$ is seen as analogous to a Line



% --------------------------------------------------------------------
\subsection{Arithmetic Variety}\label{sec:arithmetic_variety}
% --------------------------------------------------------------------

Global Analytic Geometry (\S\ref{sec:global_analytic_geometry}): definition of
a Hodge Theory (\S\ref{sec:hodge_theory}) for Arithmetic Varieties
(\S\ref{sec:arithmetic_variety})



% --------------------------------------------------------------------
\subsection{Global Analytic Geometry}\label{sec:global_analytic_geometry}
% --------------------------------------------------------------------

\url{https://ncatlab.org/nlab/show/Global+analytic+geometry}

combines Non-archimedean (\S\ref{sec:nonarchimedean_analytic_geometry}) and
Archimedean Analytic Geometry and contains Algebraic Geometry as a Sub-theory

\fist treats all Places (\S\ref{sec:place}) ``on equal footing'' in contrast to
Scheme Theory (\S\ref{sec:scheme_theory})



\subsubsection{Global Analytic Space}\label{sec:global_analytic_space}

\fist Arithmetic Cryptography (\S\ref{sec:arithmetic_cryptography})



\subsubsection{Global Hodge Theory}\label{sec:global_hodge_theory}

definition of a Hodge Theory (\S\ref{sec:hodge_theory}) for Arithmetic
Varieties (\S\ref{sec:arithmetic_variety})



% --------------------------------------------------------------------
\subsection{Arithmetic Cryptography}\label{sec:arithmetic_cryptography}
% --------------------------------------------------------------------

\url{https://ncatlab.org/nlab/show/arithmetic+cryptography}

description of \emph{Public Key Cryptography Systems} based on the use of the
Arithmetic Geometry of Schemes (\S\ref{sec:scheme}), or Global Analytic Spaces
(\S\ref{sec:global_analytic_space}), over $\ints$

example Systems:
\begin{enumerate}
  \item Systems based on the fact that it is very difficult (computationally)
    to Factorize a Natural Number into a Product of two large Prime Numbers
    (see Semiprimes \S\ref{sec:semiprime})
  \item Systems based on the Discrete Logarithm Problem
    (\S\ref{sec:discrete_logarithm}) on Elliptic Curves
    (\S\ref{sec:elliptic_curve}) or more general Abelian Varieties
    (\S\ref{sec:abelian_variety}) over Finite Fields (\S\ref{sec:finite_field})
\end{enumerate}
(2) has the advantage relative to (1) by having shorter Keys without
compromising security

basic idea: use a Finite Family $X$ of Polynomials with Integer Coefficients
$P_1, \ldots, P_m \in \ints[X_1, \ldots, X_n]$--or more generally a
Quasi-projective Scheme (\S\ref{sec:projective_scheme}) $X$ of Finite Type over
$\ints$, or a Global Analytic Space $X$ over a Banach Ring
(\S\ref{sec:banach_algebra})--\emph{Encoded} in a Finite Number of Integers
(the Coefficients and Degrees of the corresponding Polynomials), together with
some ``additional data'' (such as a way to ``cut a part of the associated
Motive'' \S\ref{sec:motive}) to define a \emph{Public Key Cryptosystem}
(FIXME: clarify)

the aim is to define a ``good'' Geometric Cohomology Theory
(\S\ref{sec:cohomology_theory}) for Global Analytic Spaces based on Analyic
methods and Differential Calculus that would allow a definition of Public Key
Cryptography Systems based on the ``datum'' of a Global Analytic Space $X$ and
of (e.g.) a Part $M$ of the associated (or possible absolute) ``Rational
Motive'' $M(X)$; the definition of ``such methods'' would not give any new
``Attacks'' to the ``previous ones'' (???) but may give a larger class of
Public Keys that may allow the use of ``shorter ones''
(FIXME: clarify)

it is a hard computational task to determine if two Cohomology Classes are
Equal in the $\ell$-adic Setting (FIXME: explain)

Higher-dimensional Arithmetic Cryptography-- give an Algebraic Cohomology Class
$[c]$ and Compute $[d] = n.[c]$, Public Key is given by the Pair $([c],[d])$
and the Message is the Number $n$; another ``product type'' approach to the
definition of a Public Key is given two Prime Cohomology Classes $[c]$ and
$[d]$ (Classes of Irreducible Subvarieties of a given Codimension) and compute
their Product Class $[e] = [c].[d]$ in the Cohomology Ring, one may try given
$[e]$ to get back $[c]$ and $[d]$



% ====================================================================
\section{Derived Algebraic Geometry}\label{sec:intersection_theory}
% ====================================================================

or \emph{Spectral Algebraic Geometry}

generalization of Algebraic Geometry replacing Commutative Rings with Ring
Spectra in Algebraic Topology where higher Homotopy accounts for
Non-discreteness (Tor) of the Structure Sheaf
(FIXME: Tor ???)
%TODO xref

Derived Category (\S\ref{sec:derived_category})



% --------------------------------------------------------------------
\subsection{Derived Scheme}\label{sec:derived_scheme}
% --------------------------------------------------------------------

a Homotopy-theoretic (\S\ref{sec:homotopy_theory}) generalization of a Scheme
(\S\ref{sec:scheme})

a \emph{Derived Scheme} is a Pair $(X, \mathcal{O})$ of a Topological Space $X$
and a Sheaf $\mathcal{O}$ of Commutative Ring Spectra on $X$ such that:
\begin{itemize}
\item the Pair $(X, \pi_0\mathcal{O})$ is a Scheme
\end{itemize}
and:
\begin{itemize}
\item $\pi_k\mathcal{O}$ is a Quasi-coherent $\pi_0\mathcal{O}$-module
\end{itemize}

over a Field of Chracteristic Zero, Derived Schemes are equivalent to
Differential Graded Schemes (\S\ref{sec:differential_graded_scheme})



% --------------------------------------------------------------------
\subsection{Intersection Theory}\label{sec:intersection_theory}
% --------------------------------------------------------------------

in Algebraic Geometry: Subvarieties are \emph{Intersected} on an
Algebraic Variety

in Algebraic Topology: Intersections within ``the'' Cohomology Ring
%FIXME

\fist cf. Intersection Homology (\S\ref{sec:intersection_homology})

\fist Motivic Homotopy Theory (\S\ref{sec:motivic_homotopy})



% --------------------------------------------------------------------
\subsection{Enumerative Geometry}\label{sec:enumerative_geometry}
% --------------------------------------------------------------------



% ====================================================================
\section{$\mathbb{F}_1$-geometry}\label{sec:f1_geometry}
% ====================================================================

\emph{Non-additive Geometry} or \emph{Absolute Geometry}

$\mathbb{F}_1$ -- Field of Characteristic $1$

Set Theory (Part \ref{part:set_theory}) as Linear Algebra (Part
\ref{part:linear_algebra}) over the ``Field with one Element''

2018 - Lorscheid - \emph{$\mathbb{F}_1$ for Everyone}

Algebraic Geometry over hypothetical Field $\mathbb{F}_1$ with one Element

applications: Cyclic Homology (\S\ref{sec:cyclic_homology}), Tropical Geometry
(\S\ref{sec:tropical_geometry})

Tits57 -- Projective Geometries over Finite Fields have meaningful analogue for
$1$ one element

in Quantum Groups, $\mathbb{F}_q$ may be considered as $q \rightarrow 1$ where
$q$ occurs as a Complex parameter

the General Linear Group (\S\ref{sec:general_linear_group}) of Invertible
Matrices with Coefficients in $\mathbf{F}_q$ converges towards the Symmetric
Group $S_n$ (\S\ref{sec:symmetric_group}) on $n$ Elements as $q \rightarrow 1$

\emph{Geometry} -- a collection of various Homogeneous Spaces for a fixed Group
$G$ together with an Incidence Relation

limit of $G = GL(n,\mathbb{F}_q)$ and the Grassmanians (\S\ref{sec:grassmanian})
$Gr(k,\mathbb{F}^n_q)$ for various $n$; the Group for the limit Geometry as
$q \rightarrow 1$ is the Symmetric Group $S_n$ on $n$ Elements



% --------------------------------------------------------------------
\subsection{$\mathbb{F}_1$-variety}\label{sec:f1_variety}
% --------------------------------------------------------------------

\subsubsection{Monoid Scheme}\label{sec:monoid_scheme}

Dietmar05 -- $\mathbb{F}_1$-schemes or \emph{Monoid Schemes}

Commutative Monoids with Zero, i.e. an Absorbing Element $0$ such that $0 \cdot
a = 0$ for all $a \in A$

note that every Ring is a Monoid when omitting its Addition



% ====================================================================
\section{Universal Algebraic Geometry}\label{sec:universal_geometry}
% ====================================================================

% ====================================================================
\section{Computational Algebraic Geometry}
\label{sec:computational_algebraic_geometry}
% ====================================================================

% --------------------------------------------------------------------
\subsection{Gr\"obner Basis}\label{sec:grobner_basis}\hfill
% --------------------------------------------------------------------

%FIXME: move section ???



% ====================================================================
\section{Anabelian Geometry}\label{sec:anabelian_geometry}
% ====================================================================

% ====================================================================
\section{Tropical Geometry}\label{sec:tropical_geometry}
% ====================================================================

piece-wise Linear or ``skeletonized'' Algebraic Geometry

\fist $\mathbb{F}_1$-geometry



% ====================================================================
\section{Toric Geometry}\label{sec:toric_geometry}
% ====================================================================

Toric Varieties (\S\ref{sec:toric_variety})

2011 - Millan, Dickenstein, Shiu, Conradi - \emph{Chemical Reaction Systems with
  Toric Steady States}

\url{https://johncarlosbaez.wordpress.com/2018/07/03/toric-geometry-in-reaction-networks/}
- \emph{Toric Geometry in Reaction Networks}
