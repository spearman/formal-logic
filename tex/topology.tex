%%%%%%%%%%%%%%%%%%%%%%%%%%%%%%%%%%%%%%%%%%%%%%%%%%%%%%%%%%%%%%%%%%%%%%
%%%%%%%%%%%%%%%%%%%%%%%%%%%%%%%%%%%%%%%%%%%%%%%%%%%%%%%%%%%%%%%%%%%%%%
\part{Topology}\label{part:topology}\cite{lc11}
%%%%%%%%%%%%%%%%%%%%%%%%%%%%%%%%%%%%%%%%%%%%%%%%%%%%%%%%%%%%%%%%%%%%%%
%%%%%%%%%%%%%%%%%%%%%%%%%%%%%%%%%%%%%%%%%%%%%%%%%%%%%%%%%%%%%%%%%%%%%%

% ====================================================================
\section{Metric Space}\label{sec:metric_space}
% ====================================================================

A \emph{Metric Space} is a Set $M$ for which a \emph{Metric}
(\ref{sec:metric}) is defined for all Elements of that Set.

A Metric Space $M$ can then be defined as the pair:
\[
  (M,d)
\]
where $M$ is a Set of Elements and $d$ is a Distance Function.

Subsets of a Metric Space may be \emph{Open}
(\S\ref{sec:open_set}), \emph{Closed} (\S\ref{sec:closed_set}),
both or neither.

Denotational Semantics (\S\ref{sec:denotational_semantics})

Domain Theory (\S\ref{sec:domain_theory})



% --------------------------------------------------------------------
\subsection{Metric}\label{sec:metric}
% --------------------------------------------------------------------

A \emph{Metric} is a \emph{Distance Function} defined between all
Elements or Points of a Metric Space.

A Distance Function $d$ has the form:
\[
  d: M \times M \rightarrow \mathbb{R}^{+}
\]
with the following conditions:
\begin{enumerate}
\item $d(x_1, x_2) \geq 0$ (\emph{Non-negativity} or \emph{Separation
  Axiom})
\item $d(x_1, x_2) = 0 \leftrightarrow x_1 = x_2$ (\emph{Identity of
  Indiscernables} or \emph{Coincidence Axiom})
\item $d(x_1, x_2) = d(x_2, x_1)$ (\emph{Symmetry})
\item $d(x_1, x_3) \leq d(x_1, x_2) + d(x_2, x_3)$
  (\emph{Subadditivity} or \emph{Triangle Inequality})
\end{enumerate}
Two Metrics, $d_1$ and $d_2$, in a Metric Space, $M$, are
\emph{Metrically Equivalent}, $d_1 \sim d_2$, if for Topologies
$\tau_1$ Induced by $d_1$ and $\tau_2$ Induced by $d_2$, $\tau_1 =
\tau_2$. A sufficient condition for Metric Equivalence is given by:
\[
  \exists k_1, k_2 > 0 : \forall x \in M, \forall r > 0,
  B^{d1}_{rk_1}(x) \subseteq B^{d2}_r (x) \subseteq B^{d1}_{rk_2}(x)
\]

\emph{Euclidean Metric}:
\[
  d: \mathbb{R}^n \times \mathbb{R}^n \rightarrow \mathbb{R}
\]\[
  (\mathbf{p},\mathbf{q}) \mapsto \sqrt{\sum_{i=1}^{n}(q_i - p_i)^2}
\]

\emph{Discrete Metric}:
\[
  d: X \times X \rightarrow \mathbb{R}^{+}
\]\[
  (\mathbf{p},\mathbf{q}) \mapsto \left\{
  \begin{array}{l l}
    0: \mathbf{p} = \mathbf{q}\\
    1: \mathbf{p} \neq \mathbf{q}
  \end{array}\right.
\]

\emph{Max Metric}:
\[
  d: \mathbb{R}^n \times \mathbb{R}^n \rightarrow \mathbb{R}
\]\[
  (\mathbf{p},\mathbf{q}) \mapsto max_{1 \leq i \leq n}\{|q_i - p_i|\}
\]

\emph{Taxicab Metric}:
\[
  d: \mathbb{R}^n \times \mathbb{R}^n \rightarrow \mathbb{R}
\]\[
  (\mathbf{p},\mathbf{q}) \mapsto \sum_{i=1}^{n}|q_i - p_i|
\]



% --------------------------------------------------------------------
\subsection{Ball}\label{sec:ball}
% --------------------------------------------------------------------

A Metric \emph{Ball} is defined for a Point $p$ in a Metric Space
$(M,d)$ as the set of all Points (including $p$) within a given Radius
$r > 0$ as determined by the Distance Function of the Metric Space:
\[
  B_r(p) = {x \in M | d(x,p) < r }
\]
The above is termed an \emph{Open Ball} because it does not include
the points where $d(x,p) = r$. Such a Ball including these additional
Points is called a \emph{Closed Ball}.



\subsubsection{Unit Ball}\label{sec:unit_ball}

Unit Sphere (\S\ref{sec:unit_sphere})



\subsubsection{Open Ball}\label{sec:open_ball}

Open $n$-ball $B^n$

Homeomorphic to Cartesian Space (\S\ref{sec:cartesian_space})
$\reals^n$



\subsubsection{Closed Ball}\label{sec:closed_ball}

Closed $n$-ball $D^n$



% --------------------------------------------------------------------
\subsection{Open Set}\label{sec:open_set}
% --------------------------------------------------------------------

An \emph{Open Set} is a Subset of a Metric Space, defined in terms of
\emph{Open Balls} (\S\ref{sec:ball}). For a Metric Space $(M,d)$, the
Set $U \subseteq M$ is \emph{Open} if
\[
  \forall x \in U, \exists r > 0 : B_r(x) \subseteq U
\]
where $B_r(x)$ is an Open Ball centered on Point $x$.

Open Sets of a Metric Space $(M,d)$ have the following three
properties:
\begin{enumerate}
\item $\varnothing, M$ are Open Sets
\item If $U, V \in M$ are Open, then $U \cap V$ is Open in $M$
\item If $\{ U_i \}_{i \in I}$ are Open, then $\bigcup_{i \in I}
  U_i$ is an Open Set
\end{enumerate}
\fist Note that the Intersection of an arbitrary collection of
Open Sets is not necessarily an Open Set.



\subsubsection{Borel Set}\label{sec:borel_set}

Borel Algebra (\S\ref{sec:borel_algebra})

Borel Hierarchy (\S\ref{sec:borel_hierarchy})



% --------------------------------------------------------------------
\subsection{Closed Set}\label{sec:closed_set}
% --------------------------------------------------------------------

A Subset, $X \subseteq M$, of a Metric Space, $M$, is a \emph{Closed
  Set} if the Relative Complement (\ref{sec:relative_complement}), $M
\backslash X$, is an Open Set. Likewise, the Complement of an Open Set
is a Closed Set.

Properties of Closed Sets:
\begin{enumerate}
  \item Given Topological Space $(X, \tau)$, $\varnothing$ and $X$ are
    Closed
  \item If $C, D$ are Closed in $X$, then $C \cup D$ is Closed in $X$
  \item If $\{ U_i \}_{i \in I}$ are Closed, then $\bigcap_{i \in I}
  U_i$ is Closed
\end{enumerate}

If $C$ is a Closed Set and $f$ is a Continuous Function
(\S\ref{sec:continuous_function}), $f^{-1}(C)$ is a Closed Set.

\fist Note that the Complement of a Closed Set is not
necessarily an Open Set, and a Set may be both Open and Closed
(\emph{Clopen}) or neither Open nor Closed.

In the Discrete Topology, every Open Set is also a Closed Set. In the
Trivial Topology both the Empty Set and the entire Set are both Open
and Closed Sets.



\subsubsection{Perfect Set}\label{sec:perfect_set}

a Closed Set with no \emph{Isolated Points}
(\S\ref{sec:isolated_point})



% --------------------------------------------------------------------
\subsection{Continuous Function}\label{sec:continuous_function}
% --------------------------------------------------------------------

A Function $f : X \rightarrow Y$ between Metric Spaces $(X,d)$ and
$(Y,d')$ is \emph{Continuous} at a Point $c$ if:
\[
  \forall \epsilon > 0, \exists \delta > 0 :
  f (B_{\delta}(c)) \subseteq B_{\epsilon}(f(c))
\]
$f$ is a \emph{Continuous Function} if it satisfies the above for all
$c$. All Continuous Functions are Sequentially Continuous
(\S\ref{sec:sequentially_continuous}).

Alternatively, a Function $f: X \rightarrow Y$ is Continuous between
Metric Spaces if and only if for all Open Sets $V \subseteq Y$,
$f^{-1}(v)$ is an Open Set in $X$.

Real-valued Continuous Function (\S\ref{sec:real_continuous})

Local (Point)

Continuous on Closed and Bounded implies Uniformly Continuous
(\S\ref{sec:uniform_continuity})



\subsubsection{Uniform Continuity}\label{sec:uniform_continuity}

Global (Set)



\subsubsection{Absolute Continuity}\label{sec:absolute_contunuity}



% --------------------------------------------------------------------
\subsection{Isometry}\label{sec:isometry}
% --------------------------------------------------------------------

Distance-preserving Injective Map between Metric Spaces



\subsubsection{Isometry Group}\label{sec:isometry_group}

Set of all Endomorphic Bijective Isometries with Composition of
Isometries as the Group Operation and the Identity Function as the
Identity Element

Every Isometry Group is a Subgroup (\S\ref{sec:subgroup}) of Isometries

Symmetry Group (\S\ref{sec:symmetry_group}) of a Space is a Subgroup
of the Isometry Group of a Space



% --------------------------------------------------------------------
\subsection{Contraction Map}\label{sec:contraction_map}
% --------------------------------------------------------------------

Metric Space $(M,d)$

\emph{Contraction Map} (or \emph{Contraction} or \emph{Contractor}) is
a Function $f$ from $M$ to itself such that there is some Non-negative
Real Number $0 \leq k < 1$ such that $\forall x,y \in M$:
\[
  d(f(x),f(y)) \leq k d(x,y)
\]



\subsubsection{Banch Fixed-point Theorem}\label{sec:banach_fixedpoint}

\fist See application in the Topological Semantics
(\S\ref{sec:topological_semantics}) of Programming Languages



% --------------------------------------------------------------------
\subsection{Complete Metric Space}\label{sec:complete_metric_space}
% --------------------------------------------------------------------

Metric Space $(M,d)$ is \emph{Complete} if every Cauchy Sequence
(\S\ref{sec:cauchy_sequence}) of Elements in $M$ has a Limit in $M$

Partial Traces (\S\ref{sec:partial_trace})



% --------------------------------------------------------------------
\subsection{Ultrametric Space}\label{sec:ultrametric_space}
% --------------------------------------------------------------------

Triangle Inequality (\S\ref{sec:triangle_inequality}) is replaced
with:
\[
  d(x,z) \leq max\{d(x,y),d(y,z)\}
\]

Functional Reactive Programming (FRP) (\S\ref{sec:frp}): Decoupled
Functions form Contraction Maps (\S\ref{sec:contraction_map}) in an
Ultrametric Space of Functions



% ====================================================================
\section{Topology}\label{sec:topology}
% ====================================================================

Given a Set $X$, a \emph{Topology} $\tau$ on $X$ is a Family
(\S\ref{sec:family}) of Subsets of $X$ (the Open Sets
\S\ref{sec:open_set}) with the Properties:
\begin{enumerate}
  \item $\varnothing \in \tau$, $X \in \tau$
  \item $A,B \in \tau \Rightarrow A \cap B \in \tau$
  \item $\{A_i\}_{i \in I} \subseteq \tau \Rightarrow \bigcup_{i \in
  I} A_i \in \tau$
\end{enumerate}
$\tau$ is therefore a Subset of the Power Set of $X$:
\[
  \tau \subseteq \pow(X)
\]



% --------------------------------------------------------------------
\subsection{Discrete Topology}\label{sec:discrete_topology}
% --------------------------------------------------------------------

% --------------------------------------------------------------------
\subsection{Compact-open Topology}\label{sec:compact_open}
% --------------------------------------------------------------------

Defined on Set of Continuous Maps (\S\ref{sec:continuous_map}) between
two Topological Spaces (\S\ref{sec:topological_space}).



% --------------------------------------------------------------------
\subsection{Cofinite Topology}\label{sec:cofinite_topology}
% --------------------------------------------------------------------

\emph{Cofinite Topology} (or \emph{Finite Complement Topology})



% --------------------------------------------------------------------
\subsection{Cocountable Topology}\label{sec:cocountable_topology}
% --------------------------------------------------------------------

\emph{Cocountable Topology}



% --------------------------------------------------------------------
\subsection{Order Topology}\label{sec:order_topology}
% --------------------------------------------------------------------



% ====================================================================
\section{Topological Space}\label{sec:topological_space}
% ====================================================================

A \emph{Topological Space} is a pair consisting of a Metric Space,
$M$, and a Toplogy, $\tau$ on that Metric Space:
\[
  (M,\tau)
\]

Every Metric Space gives a Topology, but Topologies may exist for
which there is no definable Metric Space. See Metrizable Spaces
(\S\ref{sec:metrizable_space}).

Given a Metric Space $M$, the following Topologies may be described:
\begin{description}
\item[Trivial Topology] $\tau = \{\varnothing, M\}$ (Open Sets under
  any Metric)

\item[Discrete Topology] $\tau = \pow(M)$ (Open Sets under
  Discrete Metric)
\end{description}
The Discrete Metric Induces the Discrete Topology. There is no Metric
that Induces the Trivial Topology.

By the \emph{Homotopy Hypothesis}, $\inf$-groupoids
(\S\ref{sec:infinity_groupoid}) are Spaces.

\fist See also Topological Semantics
(\S\ref{sec:topological_semantics}) for Denotational Semantics based
on Topological Spaces



% --------------------------------------------------------------------
\subsection{Finite Topological Space}
\label{sec:finite_topological_space}
% --------------------------------------------------------------------

\subsubsection{Sierpi\'nski Space}\label{sec:sierpinski_space}

(or \emph{Connected Two-point Set})

smallest Topological Space that is neither Trivial nor Discrete

$(S,\tau_S)$

Point Set $S = \{0,1\}$

Open Sets $\tau_S = \{\varnothing, \{1\}, \{0,1\}\}$

Closed Sets $\{\varnothing, \{0\}, \{0,1\}\}$

For a Topological Space $(X,\tau_X)$ and Open Subset $U \in \tau_X$,
define the \emph{Characteristic Map} $\chi_U : X \rightarrow S$ of $U$ to be:
\[
  \chi_U (x) =
  \begin{cases}
    1  & \text{if}\; x \in U \\
    0  & \text{otherwise} \\
  \end{cases}
\]

$\cat{Top}((X,\tau_X),(S,\tau_S)) \cong \tau_X$

Ring Structure on $S$: $\vee$, $\wedge$



% --------------------------------------------------------------------
\subsection{Closure}\label{sec:closure}
% --------------------------------------------------------------------

Kuratowski Closure

Closure Operator



% --------------------------------------------------------------------
\subsection{Metrizable Space}\label{sec:metrizable_space}
% --------------------------------------------------------------------

Topological Space $(X, \tau)$ is a \emph{Metrizable Space} if there
exists a Metric $d : X \times X \rightarrow [0, \infty)$ such that the
  Topology Induced by $d$ is $\tau$.



\subsubsection{Metrization Theorem}\label{sec:metrization_theorem}



% --------------------------------------------------------------------
\subsection{Uniform Space}\label{sec:uniform_space}
% --------------------------------------------------------------------

% --------------------------------------------------------------------
\subsection{Attaching Space}\label{sec:attaching_space}
% --------------------------------------------------------------------

\subsubsection{Attaching Map}\label{sec:attaching_map}

\subsubsection{Mapping Cone}\label{sec:mapping_cone}



% --------------------------------------------------------------------
\subsection{Contractible Space}\label{sec:contractible_space}
% --------------------------------------------------------------------

A Topological Space $X$ is \emph{Contractible} if the Identity Map on
$X$ is Null-homotopic (\S\ref{sec:null_homotopy}).

Contractible CW Complex (\S\ref{sec:contractible_cwcomplex})



% --------------------------------------------------------------------
\subsection{Quotient Space}\label{sec:quotient_space}
% --------------------------------------------------------------------

Topological Space $(X, \tau_X)$, Equivalence Relation $\sim$ on $X$

\emph{Quotient Space} $(X/\sim, \tau_{X/\sim})$ is the Set of Equivalence
Classes of Elements of $X$:
\[
  X / \sim = \{ [x] | x \in X \}
\]
with Quotient Topology (\S\ref{sec:quotient_topology}) $\tau_{X/\sim}$:
\[
  \tau_{X/\sim} = \{ U \subseteq (X/\sim) |
    U = (\bigcup_{[a] \in U} [a]) \in \tau_X \}
\]
or equivalently, for a Surjective Function $q : X \rightarrow (X /
\sim)$ sending Points in $X$ to the Equivalence Class containing it:
\[
  \tau_{X/\sim} = \{ U \subseteq (X/\sim) | q^{-1}(U) \in \tau_X \}
\]
that is, the Sets with an Open Preimage under $q$.

In a Category of Spaces (e.g. $\cat{Top}$ of Topological Spaces or
$\cat{Loc}$ of Locales), a Quotient Space is a Quotient Object.



\subsubsection{Mapping Cylinder}\label{sec:mapping_cylinder}

$f : X \rightarrow Y$

\emph{Mapping Cylinder} $M_f = (X \times I) \sqcup Y$



% --------------------------------------------------------------------
\subsection{Compactly Generated Space}\label{sec:compactly_generated}
% --------------------------------------------------------------------

(or \emph{$k$-space}, ``kompakt'')

$\cat{CGTop}$

$\cat{CGHaus}$



% --------------------------------------------------------------------
\subsection{Separation Axioms}\label{sec:separation_axioms}
% --------------------------------------------------------------------

% --------------------------------------------------------------------
\subsection{Hausdorff Space}\label{sec:hausdorff_space}
% --------------------------------------------------------------------

or \emph{Separated Space}

distinct Points have Disjoint Neighborhoods (\S\ref{sec:neighborhood})



\subsubsection{Compactly Generated Hausdorff Space}
\label{sec:compact_hausdorff}



% --------------------------------------------------------------------
\subsection{Topologist's Sine Curve}\label{sec:topologists_sine}
% --------------------------------------------------------------------

% --------------------------------------------------------------------
\subsection{Cover}\label{sec:topological_cover}
% --------------------------------------------------------------------

A \emph{Cover} (or \emph{Covering}) of a Topological Space $(X, \tau)$
is an Indexed Family of Sets $C = \{ U_i : i \in I \}$ such that their
Union contains $X$:
\[
  X \subseteq \bigcup_{i \in I} U_i
\]
A Cover can also be defined for an arbitrary Subset of $X$, $Y
\subseteq X$:
\[
  Y \subseteq \bigcup_{i \in I} U_i
\]



\subsubsection{Covering Map}\label{sec:covering_map}

(\emph{Projection})

Covering Space $C$

Base Space $X$

Continuous (\S\ref{sec:continuous_map}) Surjective Map $p : C
\rightarrow X$ such that for every $x \in X$ there is an Open
Neighborhood (\S\ref{sec:open_neighborhood}) $U$ of $x$ with
$p^{-1}(U)$ a Disjoint Union of Open Sets in $C$, each Homeomorphic to
$U$ by $p$.

Trivial Covering



\paragraph{Fiber}\label{sec:point_fiber}\hfill

Inverse Image of Point $x \in X$

necessarily a Discrete Space



\subsubsection{Covering Space}\label{sec:covering_space}

(\emph{Total Space})



\paragraph{Universal Cover}\label{sec:universal_cover}\hfill

$\reals$ is the Universal Cover of the Unit Circle $S^1$



\subsubsection{Refinement}\label{sec:refinement}



% --------------------------------------------------------------------
\subsection{Base}\label{sec:topological_base}
% --------------------------------------------------------------------

A \emph{Base}, $B$, is a Subset of a Topology, $\tau$, in a Metric
Space, $(M,\tau)$, such that:
\[
  \forall U \in \tau, \exists \{B_i\}_{i \in I} \subseteq B :
  \bigcup_{i \in I}B_i = U
\]
Properties:
\begin{enumerate}
  \item $B$ is a Covering (\S\ref{sec:topological_cover}) of $M$, as
    stated by:
\[
  M \subseteq \bigcup_{i \in I} B_i
\]

  \item
\[
  \forall B_1, B_2 \in B, \forall x \in B_1 \cap B_2,
  \exists B_3 \in B : x \in B_3 \wedge B_3 \subseteq B_1 \cap B_2
\]

\end{enumerate}
An example of a Base is the Set of all Open Balls in a Metric Space.

A Base is not necessarily Unique for a given Topology. Adding Elements
to a Base results in another Base.



\subsubsection{Subbase}\label{sec:subbase}

A \emph{Subbase}, $S$, is a Subset of a Topology, $\tau$, in a Metric
Space, $(M,\tau)$, such that the Set:
\[
  S \subseteq \tau : \{ \bigcap_{j \in J} S_j : |J| < \infty \}
\]
is a Base for $\tau$.

There is no unique Subbase for a given Topology but there is a unique
Topology for a given Subbase.



% --------------------------------------------------------------------
\subsection{Subspace Topology}\label{sec:subspace_topology}
% --------------------------------------------------------------------

Topological Space $(X,\tau)$, Subset $S \subseteq X$, \emph{Subspace
  Topology} (also \emph{Relative Topology}, \emph{Induced Topology},
or \emph{Trace Topology}) is defined as:
\[
  \tau_S = \{ S \cap U | U \in \tau \}
\]
A Subset of $S$ is Open in $\tau_S$ if and only if it is the
Intersection of $S$ with an element of $\tau$.

A Subspace Topology may alternatively be given as a \emph{Topological
  Embedding} (\S\ref{sec:topological_embedding}).

Dual to Quotient Topology (\S\ref{sec:quotient_topology})

\emph{Open Subspace}

\emph{Closed Subspace}



% --------------------------------------------------------------------
\subsection{Quotient Topology}\label{sec:quotient_topology}
% --------------------------------------------------------------------

Quotient Space (\S\ref{sec:quotient_space})

Final Topology on the Quotient Space with respect to map $q : X
\rightarrow X / \sim$



% --------------------------------------------------------------------
\subsection{Initial Topology}\label{sec:initial_topology}
% --------------------------------------------------------------------

\subsubsection{Weak Topology}\label{sec:weak_topology}



% --------------------------------------------------------------------
\subsection{Final Topology}\label{sec:final_topology}
% --------------------------------------------------------------------

(also \emph{Strong Topology}, \emph{Colimit Topology}, or
\emph{Inductive Topology})



% --------------------------------------------------------------------
\subsection{Countability Axioms}\label{sec:countability_axioms}
% --------------------------------------------------------------------

A Topological Space, $(M,\tau)$, is \emph{First Countable} or $1c$ if
$\forall X \in M$, there exists a Countable Neighborhood Base
(\S\ref{sec:neighborhood_base}). First Countable Topological Spaces is
a narrower Class of Topological Spaces where Functions Preserve Limits
of Sequences. %FIXME

And the Topological Space is \emph{Second Countable} or $2c$ if $\tau$
has a Countable Base.
\[
  2c \rightarrow 1c
\]

All Metric Spaces are $1c$.



% --------------------------------------------------------------------
\subsection{Compact Space}\label{sec:compact_space}
% --------------------------------------------------------------------

\subsubsection{Tychonoff's Theorem}\label{sec:tychonoffs_theorem}

Compactness Theorem (Model Theory \S\ref{sec:compactness})



% --------------------------------------------------------------------
\subsection{Betti Number}\label{sec:betti_number}
% --------------------------------------------------------------------

Euler Characteristic (\S\ref{sec:euler_characteristic})



% ====================================================================
\section{General Topology}\label{sec:general_topology}
% ====================================================================

\emph{General Topology} (or \emph{Point-set Topology})

A \emph{Topology}, $\tau$, is a collection of Subsets called
\emph{Open Sets} (\S\ref{sec:open_set}) of a \emph{Metric Space}
(\S\ref{sec:metric_space}), $M$, subject to the following Inductive
definition:
\begin{enumerate}
\item $\varnothing \in \tau, M \in \tau$
\item $U,V \in \tau \rightarrow U \cap V \in \tau$
\item $\{U_i\}_{i \in I} \subseteq \tau \rightarrow \bigcup_{i \in I}
  U_i \in \tau$
\end{enumerate}

An equivalent definition is possible in terms of \emph{Closed Sets}
(\S\ref{sec:closed_set}).



% --------------------------------------------------------------------
\subsection{Point}\label{sec:topological_point}
% --------------------------------------------------------------------

\emph{Point}

Two Points that are within the same Open Set (they have exactly the
same Neighborhoods) of a Topology are said to be \emph{Topologically
  Indistinguishable} (\S\ref{sec:topologically_distinguishable}).



\subsubsection{Isolated Point}\label{sec:isolated_point}

\subsubsection{Topologically Distinguishable}
\label{sec:topologically_distinguishable}

Two Points of a Topological Space $X$ are \emph{Topologically
  Indistinguishable} if they have exactly the same Neighborhoods
(\S\ref{sec:neighborhood}).



% --------------------------------------------------------------------
\subsection{Coarseness}\label{sec:coarseness}
% --------------------------------------------------------------------

Given two Topologies, $\tau_1 \subset \tau_2$, $\tau_1$ is
\emph{Coarse} relative to $\tau_2$, and $\tau_2$ is \emph{Fine}
relative to $\tau_1$.

Coarser Topologies will have more Topologically Indistinguishable
Points.

The Coarsest Topology possible is the Trivial Topology $\{
\varnothing, X \}$.

The Coarsest Topology that contains a given collection of Open Sets $S
\subseteq \pow(X)$ is the Topology Generated by taking $S$ to
be a Subbase (\S\ref{sec:subbase}) of $X$.



% --------------------------------------------------------------------
\subsection{Interior}\label{sec:interior}
% --------------------------------------------------------------------

For a Subset, $V$, of a Topology, $\tau$, a Point $x \in V$ is in the
\emph{Interior} of $V$, $V^{\circ}$, if there is a \emph{Neighborhood}
(\S\ref{sec:neighborhood}) of $x$, $N \subset V$. Inductively, for
$\{A_i\}_{i \in I} \subseteq \tau \wedge \forall i, A_i \subseteq V$:
\[
  V^{\circ} = \bigcup_{i \in I} A_i \subseteq \tau
\]
Equivalently, the Interior of $V$, $V^o$, is every Open Set within $V$
and is itself an Open Set:
\[
  V^o = \bigcup_{i \in I} \{ A_i : A_i \subseteq \tau \wedge A_i
  \subseteq V \}
\]



\subsubsection{Interior Operator}\label{sec:interior_operator}

The Closure Operator is the Dual of the Interior Operator.



% --------------------------------------------------------------------
\subsection{Neighborhood}\label{sec:neighborhood}
% --------------------------------------------------------------------

A \emph{Neighborhood} of a Point $x$ in a Topological Space $(M,\tau)$
is a Subset $V \subseteq M$ such that:
\[
  \exists U \in \tau : U \subseteq V \wedge x \in U
\]
That is, $x$ is in the Interior of $V$ and $V$ is not necessarily
itself Open (see Open Neighborhoods \S\ref{sec:open_neighborhood}).

Every Subset of the Discrete Topology is a Neighborhood.

For the Trivial Topology, the only Neighborhood is the entire Space.



\subsubsection{Neighborhood System}\label{sec:neighborhood_system}

Given a Point $x$ in any Topological Space, a \emph{Neighbordhood
  System} (or \emph{Neighborhood Filter}), $\mathcal{V}(x)$, is the
Set of all Neighborhoods of $x$.



\subsubsection{Neighborhood Base}\label{sec:neighborhood_base}

Given a Neighborhood System, $\mathcal{V}(x)$, a \emph{Neighborhood
  Base} for $x$ is defined as a Subset of the Neighborhood System,
$\beta(x) \subseteq \mathcal{V}(x)$, such that:
\[
  \forall v \in V(x), \exists b \in \beta(x) : b \subseteq v
\]
For a Neighborhood Base $\beta(x) = \{ U \in B : x \in U \}$ where $B
\subseteq \pow(X)$ in Topological Space $(X,\tau)$, $B$ is a
Base for $\tau$ if and only if $\beta(x)$ is a Neighborhood Base for
all $x \in X$.



\subsubsection{Open Neighborhood}\label{sec:open_neighborhood}

\subsubsection{Uniform Neighborhood}\label{sec:uniform_neighborhood}



% --------------------------------------------------------------------
\subsection{Sequence}\label{sec:sequence_topology}
% --------------------------------------------------------------------

A \emph{Sequence} is a Net (\S\ref{sec:net}) where the Directed Set is
the Natural Numbers:
\[
  (x_n) : \mathbb{N} \rightarrow (X,\tau)
\]

Sequence (\S\ref{sec:sequence})



\subsubsection{Limit}\label{sec:limit_topology}

\subsubsection{Convergence}\label{sec:convergence_topology}

A Sequence $(x_n) : \mathbb{N} \rightarrow X$
\emph{Converges} in a Metric Space $(X,d)$, if:
\[
  \lim_{n \rightarrow \infty} d (x_n, x) = 0
\]

For a Topological Space, $(X,\tau)$, the Sequence $(x_n)$ Converges to
$x$ if:
\[
  \forall A \in \mathcal{V}(x), \exists n \in \mathbb{N}
  : m \geq n \Rightarrow x_m \in A
\]
There may possibly be more than one or infinite Limits in a general
Topological Space.

In the Trivial Topology $(X, \{\varnothing, X\})$, all Sequences
Converge to every $x \in X$.



\subsubsection{Residuality}\label{sec:reside}

In a Topological Space $(X, \tau)$, a Sequence is \emph{Residually}
(or \emph{Eventually}) in an arbitrary Subset $Y \subseteq X$ if:
\[
  \exists m \in \mathbb{N} : n \geq m \Rightarrow x_n \in Y
\]
Such a Sequence \emph{Resides} in $Y$.



\subsubsection{Frequentness}\label{sec:frequent}

In a Topological Space $(X, \tau)$, a Sequence is \emph{Frequently}
in an arbitrary Subset $Y \subseteq X$ if there is a Subsequence that
is always in that Set:
\[
  (\exists c : \mathbb{N} \rightarrow \mathbb{N})
  : (\forall n \in \mathbb{N}) x_{c(n)} \in Y
\]



\subsubsection{Accumulation}\label{sec:accumulation}

A Point $x$ in a Topological Space $(X, \tau)$ is an
\emph{Accumulation Point} (or \emph{Cluster Point}) of a Sequence
$(x_n)$ if $(x_n)$ is Frequently in every Neighborhood in the
Neighborhood Space of $x$.



\subsubsection{Subsequence}\label{sec:subsequence_topology}

Given a Sequence $(x_n) : \mathbb{N} \rightarrow X$, a
\emph{Subsequence} is given by the Composition of $(x_n)$ with a
Strictly Increasing Function on $c : \mathbb{N} \rightarrow
\mathbb{N}$:
\[
  (x_n) \circ c :
  \mathbb{N} \xrightarrow{c} \mathbb{N} \xrightarrow{(x_n)} \mathbb{N}
\]



\subsubsection{Sequentially Continuous Function}
\label{sec:sequentially_continuous}

A Function $f : X \rightarrow Y$ between Topological Spaces $(X,
\tau)$ and $(Y, \sigma)$ is \emph{Sequentially Continuous} at a Point
$x \in X$ if for any Sequence $(x_n)$ in $X$:
\[
  \lim_{n \rightarrow \infty} x_n = x
  \Rightarrow \lim_{n \rightarrow \infty} f(x_n) = f(x)
\]
and Sequentially Continuous in general if Sequentially Continuous at
every Point $x \in X$.

All Continuous Functions (\S\ref{sec:continuous_function}) are
Sequentially Continuous. If the Domain $X$ is First Countable
(\S\ref{sec:countability_axioms}), then any Sequentially Continuous
Function $f : X \rightarrow Y$ is also a Continuous Function.



% --------------------------------------------------------------------
\subsection{Limit Point}\label{sec:limit_point}
% --------------------------------------------------------------------

Limit Point of $D \in \reals$ is $l \in D$ such that:
\[
  \exists a_n \in D : a_n \neq l \wedge \lim a_n = l
\]

Limit Points of $(0,1)$ are $[0,1]$

A Discrete Set has no Limit Points. %FIXME



\subsubsection{Derived Set}\label{sec:derived_set}



% --------------------------------------------------------------------
\subsection{Density}\label{sec:density}
% --------------------------------------------------------------------

% --------------------------------------------------------------------
\subsection{Map}\label{sec:topology_map}
% --------------------------------------------------------------------

A \emph{Map} is a Function between Topological Spaces.



% --------------------------------------------------------------------
\subsection{Open \& Closed Map}\label{sec:open_closed_map}
% --------------------------------------------------------------------

\emph{Open Map} is a Function between Topological Spaces mapping Open
Sets to Open Sets.

\emph{Closed Map} is a Function between Topological Spaces mapping Closed
Sets to Closed Sets.

Every Homeomorphism (\S\ref{sec:homeomorphism}) is an Open Map, a
Closed Map, and a Continuous Function.



% --------------------------------------------------------------------
\subsection{Continuous Map}\label{sec:continuous_map}
% --------------------------------------------------------------------

A \emph{Continuous Map} $f : X \rightarrow Y$ between Topological
Spaces $(X,\tau_1)$ and $(Y,\tau_2)$ is given by:
\[
  \forall V \in \tau_2, f^{-1}(V) \in \tau_1
\]

\begin{enumerate}
  \item if $f$ is a Surjection, then $f$ is a Quotient Map
    (\S\ref{sec:quotient_map})
  \item if $f$ is a Injection, then $f$ is a Topological Embedding
    (\S\ref{sec:topological_embedding})
  \item if $f$ is a Bijection, then $f$ is a Homeomorphism
    (\S\ref{sec:homeomorphism})
\end{enumerate}

\begin{enumerate}
  \item Any Constant Function is Continuous
  \item Given two Continuous Functions, $f : X \rightarrow Y$ and $g
    : Y \rightarrow Z$, the Function $g \circ f : X \rightarrow Z$ is
    Continuous
  \item Given $f : (X, \tau) \rightarrow (Y, \sigma)$, $f$ is
    Continuous if $\tau = \pow(X)$ (Discrete Topology) or
    $\sigma = \{\varnothing, Y\}$ (Trivial Topology)
\end{enumerate}

Net Continuity (\S\ref{sec:net_continuity})

Given Topological Spaces $(X, \tau_X)$, $(Y, \tau_Y)$, and $(Z,
\tau_Z)$, where $f: X \rightarrow Y$ and $g: Y \rightarrow Z$ are
Continuous Maps, then the Composition $g \circ f : X \rightarrow Z$ is
Continuous.



\subsubsection{Quotient Map}\label{sec:quotient_map}

Quotient Space (\S\ref{sec:quotient_space})



\subsubsection{Topological Embedding}\label{sec:topological_embedding}

Topological Space $(X,\tau)$, Subset $S \subseteq X$

A Subspace Topology (\S\ref{sec:subspace_topology}) is the Coarsest
(\S\ref{sec:coarseness}) Topology such that the Inclusion Map
(\S\ref{sec:inclusion_map}) $\iota : S \hookrightarrow X$ is
Continuous (\S\ref{sec:continuous_map}). $iota$ is then called a
\emph{Topological Embedding} and $S$ is Homeomorphic
(\S\ref{sec:homeomorphism}) to its Image in $X$.



\subsubsection{Homeomorphism}\label{sec:homeomorphism}

\emph{Homeomorphism} between two Topological Spaces $(X, \tau)$ and
$(Y, \sigma)$ is is a Continuous Map $f : X \rightarrow Y$ that has a
Continuous Inverse.



% --------------------------------------------------------------------
\subsection{Product Topology}\label{sec:product_topology}
% --------------------------------------------------------------------

% --------------------------------------------------------------------
\subsection{Disjoint Union Topology}\label{sec:disjoint_union_topology}
% --------------------------------------------------------------------

(or \emph{Topological Sum})

Coproduct (\S\ref{sec:coproduct})

Dual to Product Topology (\S\ref{sec:product_topology})



% --------------------------------------------------------------------
\subsection{Net}\label{sec:net}
% --------------------------------------------------------------------

A \emph{Net} is a Function from a Directed Set
(\S\ref{sec:directed_set}) into a Topological Space:
\[
  (x_\alpha) : D \rightarrow (X, \tau)
\]
A Net is \emph{Eventually} in a Set $A \in X$ if:
\[
  \exists \alpha \in D
  : \forall \beta \in D, \beta \geq \alpha \wedge x_\beta \in A
\]

A Net is \emph{Frequently} in a Set $A \in X$ if:
\[
  \forall \alpha \in D, \exists \beta \in D
  : \alpha \leq \beta \wedge (x_\beta) \in A
\]

A Sequence (\S\ref{sec:sequence}) is a Net where the Directed Set is
the Natural Numbers $\mathbb{N}$.



\subsubsection{Net Convergence}\label{sec:net_convergence}

A Net $(x_\alpha)$ \emph{Converges} to a Point $x$ if for any
Neighborhood $U \in \mathcal{V}(x)$ in the Neighborhood System of $x$,
$(x_\alpha)$ is Eventually in $U$.



\subsubsection{Subnet}\label{sec:subnet}

A Net $(y_\beta) : B \rightarrow X$ is a Subnet of a Net $(x_\alpha) :
A \rightarrow X$ if there is a Monotone
(\S\ref{sec:monotonic_function}), Cofinal (\S\ref{sec:cofinal_map})
Total Function:
\[
  h : B \rightarrow A
\]
that is, with the Properties:
\begin{enumerate}
  \item Monotonicity:
  $\beta_1 \leq \beta_2 \Rightarrow h(\beta_1) \leq h(\beta_2)$
  \item Cofinality:
   $\forall \alpha \in A, \exists \beta \in B : \alpha \leq h(\beta)$
\end{enumerate}



\subsubsection{Cluster Point}\label{sec:cluster_point}

A Point $x$ in a Topological Space $(X,\tau)$ is a \emph{Cluster
  Point} of a Net $(x_\alpha)$ if $(x_\alpha)$ is Frequently in every
Neighborhood $U \in \mathcal{V}(x)$ of the Neighborhood System of $x$.
Alternatively, $x$ is a Cluster Point of a Net if there exists some
Subnet that Converges to $x$.



\subsubsection{Continuity}\label{sec:net_continuity}

Given two Topological Spaces $(X,\tau)$ and $(Y,\sigma)$, a Function
$f : X \rightarrow Y$ is \emph{Continuous}
(\S\ref{sec:continuous_map}) at a Point $x \in X$ if any only if for
any Convergent Net $(x_\alpha)$:
\[
  \lim (x_\alpha) = x \Rightarrow \lim f(x_\alpha) = f(x)
\]



% --------------------------------------------------------------------
\subsection{Path}\label{sec:path} \cite{hatcher02}
% --------------------------------------------------------------------

Continuous Map (\S\ref{sec:continuous_map}) $f : I \rightarrow X$ from
Unit Interval (\S\ref{sec:unit_interval}) $I$ to Space $X$.

Constant Path %FIXME

\fist See also \emph{Path} (Graph Theory \S\ref{sec:graph_path})



\subsubsection{Path Product}\label{sec:path_product}

$f,g : I \rightarrow X$ where $f(1) = g(0)$, \emph{Product Path} (or
\emph{Composition}) $f \cdot g$:
\[
  f \cdot g (s) =
  \begin{cases}
    f(2s)   & \quad 0 \leq s \leq \sfrac{1}{2} \\
    g(2s-1) & \quad \sfrac{1}{2} \leq s \leq 1 \\
  \end{cases}
\]



\subsubsection{Inverse Path}\label{sec:inverse_path}

$f^{-1}(s) = f(1-s)$



\subsubsection{Path Homotopy}\label{sec:path_homotopy}

\emph{Homotopic Paths} $f_0 \simeq f_1$ with \emph{Homotopy}
(\S\ref{sec:homotopy}) $f_t$

Family of Paths $f_t : I \rightarrow X$ for $t \in [0,1]$ such that:
\begin{enumerate}
  \item $f_t(0) = x_0$ and $f_t(1) = x_1$
  \item Map $F : I \times I \rightarrow X$ with $F(s,t) = f_t(s)$ is
    Continuous
\end{enumerate}

Linear Homotopy in $\reals^n$ (or any Convex Subspace
\S\ref{sec:convex_subspace} $X \subset \reals^n$):
\[
  f_t(s) = (1 - t) f_0(s) + t f_1(s)
\]



\paragraph{Homotopy Class}\label{sec:homotopy_class}\hfill

Equivalence Class $[f]$

Morphisms in the Category $\cat{Toph}$; this Category is not
Concretizable



\subsubsection{Loop}\label{sec:loop}

Path with equal Initial and Terminal Points called the
\emph{Basepoint}: Continuous Function
(\S\ref{sec:continuous_function}) $f$ from Unit Interval $[0,1]$ to
Topological Space $X$ where $f(0) = f(1) = x_0 \in X$

Two Loops $f_m$ and $f_n$ with the same Initial and Terminal Points
can be combined into a new Loop $f_m + f_n = B_{m+n}$.
\cite{hatcher02}

Loop Space (\S\ref{sec:loop_space})

Fundamental Group (\S\ref{sec:fundamental_group})

Link (\S\ref{sec:link})

\fist See also \emph{Loop} (Graph Theory \S\ref{sec:graph_loop})



% --------------------------------------------------------------------
\subsection{Pointed Space}\label{sec:pointed_space}
% --------------------------------------------------------------------

\subsubsection{Wedge Sum}\label{sec:wedge_sum}

For $X$, $Y$ Pointed Spaces, the \emph{Wedge Sum} (or \emph{One-point
  Union}):
\[
  X \vee Y = (X \amalg Y) / \sim
\]

Coproduct (\S\ref{sec:coproduct})



\paragraph{Rose}\label{sec:rose}\hfill

\paragraph{Sphere Boquet}\label{sec:sphere_boquet}\hfill



\subsubsection{Loop Space}\label{sec:loop_space}

Loop (\S\ref{sec:loop})

$\Omega X$

A$_\infty$-operad (\S\ref{sec:a_infinity_operad})



\paragraph{Free Loop Space}\label{sec:free_loop_space}\hfill

Space of Maps from $S^1$ to $X$ with Compact-open Topology
(\S\ref{sec:compact_open})



\paragraph{Fundamental Group}\label{sec:fundamental_group}\hfill

Algebraic Image of a Space from the Loops (\S\ref{sec:loop}) in the
Space

For a Space $X$, Elements are Loops in $X$ starting and ending at a
Base Point $x_0 \in X$, where two Loops are the same Element if they
can be Continuously Deformed (\S\ref{sec:path_homotopy}) into
eachother within the Space $X$.

All Homotopy Classes (\S\ref{sec:path_homotopy}) $[f]$ of Loops $f : I
\rightarrow X$ at Basepoint $x_0$:
\[
  \pi_1(X,x_0)
\]
with Group Operation as the Path Product (\S\ref{sec:path_product}):
\[
  [f][g] = [f \cdot g]
\]

Homeomorphic Spaces have Isomorphic Fundamental Groups. Every Group
$G$ can be realized as the Fundamental Group of some Space $X_G$.
\cite{hatcher02}

For $X$ the Complement of a Circle $A$ the Fundamental Group is the
Infinite Cyclic Group (\S\ref{sec:infinite_cyclic}) with one
Generator: $\pi_1(S^1) \cong \ints$

For the Complement of two Unlinked Circles $A$ and $B$, the
Fundamental Group is the Non-abelian (\S\ref{sec:nonabelian_group})
Free Group (\S\ref{sec:free_group}) on two Generators.

For the Complement of two Linked Circles $A$ and $B$, the Fundamental
Group is the Abelian (\S\ref{sec:abelian_group}) Free Group on two
Generators. \cite{hatcher02} This generalizes to the Fundamental Group
of a Rose (\S\ref{sec:rose}) of $k$ Circles being equal to the Free
Group on a Set of $k$ Elements.

$n$-th Homotopy Group (\S\ref{sec:homotopy_group}) $\pi_n(X,x_0)$

For Continuous Map $f : X \rightarrow Y$, there is a Group
Homomorphism $f_* : \pi (X) \rightarrow \pi (Y)$



% --------------------------------------------------------------------
\subsection{Connected Space}\label{sec:connected_space}
% --------------------------------------------------------------------

\subsubsection{Connected Set}\label{sec:connected_set}

Interval (\S\ref{sec:interval})



\subsubsection{Path-connected}\label{sec:path_connected}

Any two Points can be joined by a Path (\S\ref{sec:path})



\subsubsection{Simply-connected}\label{sec:simply_connected}



% --------------------------------------------------------------------
\subsection{Presheaf}\label{sec:presheaf}
% --------------------------------------------------------------------

Presheaf Category (\S\ref{sec:presheaf_category})



\subsubsection{Sheaf}\label{sec:sheaf}

\emph{Locality}, \emph{Gluing}

Sheave (\S\ref{sec:sheave})



% ====================================================================
\section{Simplex}\label{sec:simplex}
% ====================================================================

\emph{Topological Simplex}

$n$-simplex $\Delta^n$

Homeomorphic to Closed $n$-ball $D^n$ (\S\ref{sec:closed_ball})



% ====================================================================
\section{Kolmogorov Classification}\label{sec:kolmogorov_classification}
% ====================================================================



% ====================================================================
\section{Homotopy Theory}\label{sec:homotopy_theory}
% ====================================================================

Homotopy Type Theory (\S\ref{sec:hott})



% --------------------------------------------------------------------
\subsection{Homotopy}\label{sec:homotopy}
% --------------------------------------------------------------------

$H : X \times [0,1] \rightarrow Y$

\emph{Continuous Deformation}



\subsubsection{Classifying Space}\label{sec:classifying_space}

Dependent Types (\S\ref{sec:dependent_type})



\paragraph{Classifying Morphism}\label{sec:classifying_morphism}\hfill

Characteristic Morphism (\S\ref{sec:characteristic_morphism}) for
Subobjects; Classifying Space is the Subobject Classifier
(\S\ref{sec:subobject_classifier})

Substitution (\S\ref{sec:substitution}) of Terms in Dependent Types
can be Interpreted as Composition of the Morphism Interpreting the
Term with the Classifying Morphism Interpreting the Dependent Type



% --------------------------------------------------------------------
\subsection{Homotopic Function}\label{sec:homotopic_function}
% --------------------------------------------------------------------

Continuous Functions (\S\ref{sec:continuous_function})



\subsubsection{Null-homotopy}\label{sec:null_homotopy}

A Function $f$ is \emph{Null-homotopic} if it is Homotopic to a
Constant Function.

A Space $X$ is Contractible (\S\ref{sec:contractible_space}) if and
only if the Identity Map $id_X$ (always a Homotopy Equivalence
\S\ref{sec:homotopy_equivalence})) is Null-homotopic.



% --------------------------------------------------------------------
\subsection{Homotopy Equivalence}\label{sec:homotopy_equivalence}
% --------------------------------------------------------------------

Two Topological Spaces $X$ and $Y$ are \emph{Homotopy Equivalent} and
have the same \emph{Homotopy Type} (\S\ref{sec:homotopy_type}) if
there are Continuous Maps $f : X \rightarrow Y$ and $g : Y \rightarrow
X$ called \emph{Homotopy Equivalences} such that $g \circ f$ is
Homotopic (\S\ref{sec:homotopic_function}) to the Identity Map $id_X$
and $f \circ g$ is Homotopic to $id_Y$. Every Homeomorphism
(\S\ref{sec:homeomorphism}) is a Homotopy Equivalence, but not every
Homotopy Equivalence is a Homeomorphism.

Identity Functions are always Homotopy Equivalences. %FIXME

$X$ and $Y$ are Homotopy Equivalent if and only if there exists a
third Topological Space $Z$ containing both $X$ and $Y$ as Deformation
Retracts (\S\ref{sec:deformation_retraction})



\subsubsection{Homotopy Level}\label{sec:homotopy_level}

Truncation (\S\ref{sec:truncation})

\emph{$h$-level}

\begin{tabular}{l l l l}
$h$-level 0   & $(-2)$-groupoid & (Contractible Type, $\top$, Trivial)
  & \\
$h$-level 1   & $(-1)$-groupoid & (Proposition)
  & $h$-proposition (\S\ref{sec:h_proposition}) \\
$h$-level 2   & $0$-groupoid    & (Set)
  & $h$-set (\S\ref{sec:h_set}) \\
$h$-level 3   & $1$-groupoid    & (Groupoid)
  & $h$-groupoid (\S\ref{sec:h_groupoid}) \\
\end{tabular}



\subsubsection{Homotopy Type}\label{sec:homotopy_type}

Homotopy Type of a Point is called \emph{Contractible}
(\S\ref{sec:contractible_space}), by requiring that the Identity Map
of the Space is Null-homotopic (\S\ref{sec:null_homotopy}).



\subsubsection{Homotopy $0$-type}\label{sec:homotopy_0type}

$h$-set (\S\ref{sec:h_set})



\subsubsection{Homotopy $1$-type}\label{sec:homotopy_1type}

\emph{Homotopy $1$-type}

Groupoids (\S\ref{sec:groupoid})

$h$-groupoid (\S\ref{sec:h_groupoid})



% --------------------------------------------------------------------
\subsection{Homotopy Extension Property}\label{sec:homotopy_extension}
% --------------------------------------------------------------------

for Subspace $A \subset X$, if $(X,A)$ has the Homotpy Extension
Property then the Inclusion Map (\S\ref{sec:inclusion_map}) $i : A
\rightarrow X$ is a Cofibration (\S\ref{sec:cofibration})



% --------------------------------------------------------------------
\subsection{Fibration}\label{sec:fibration}
% --------------------------------------------------------------------

Homotopy Lifting Property



\subsubsection{Cofibration}\label{sec:cofibration}

Any Cofibration can be treated as having the Homotopy Extension
Property (\S\ref{sec:homotopy_extension})



\subsubsection{Codomain Fibration}\label{sec:codomain_fibration}

\subsubsection{Fiber Bundle}\label{sec:fiber_bundle}

Hopf Fibration



% --------------------------------------------------------------------
\subsection{Bundle}\label{sec:bundle}
% --------------------------------------------------------------------

Generalization of Fiber Bundle (\S\ref{sec:fiber_bundle})



\subsubsection{Global Section}\label{sec:global_section}

Section (\S\ref{sec:section}) of a Bundle



% --------------------------------------------------------------------
\subsection{Rational Homotopy Theory}\label{sec:rational_homotopy}
% --------------------------------------------------------------------

\subsubsection{Rational Space}\label{sec:rational_space}

\subsubsection{Sullivan Model}\label{sec:sullivan_model}



% ====================================================================
\section{Continuum Theory}\label{sec:continuum_theory}
% ====================================================================

% ====================================================================
\section{Algebraic Topology}\label{sec:algebraic_topology}
% ====================================================================

Combinatorial Topology

``Algebraic Images'' of Topological Spaces via ``Functors'': Images of
Spaces and Images of Maps, Continuous Maps
(\S\ref{sec:continuous_map}) between Spaces
(\S\ref{sec:topological_space}) are ``Projected'' onto Homomorphisms
(\S\ref{sec:homomorphism}) between their Algebraic
Images.\cite{hatcher02}



% --------------------------------------------------------------------
\subsection{Cell}\label{sec:topology_cell}
% --------------------------------------------------------------------

An $n$-dimensional Closed Cell is the Image of an $n$-dimensional
Closed Ball (\S\ref{sec:ball}) under an Attaching Map
(\S\ref{sec:attaching_map})



% --------------------------------------------------------------------
\subsection{CW Complex}\label{sec:cw_complex}
% --------------------------------------------------------------------

\emph{CW Complex} (\emph{Closure-finite Weak Topology})

A CW Complex is a Cell Complex (\S\ref{sec:cell_complex}) in
$\cat{Top}$ with respect to Generating Cofibrations in the Standard
Model Structure (\S\ref{sec:model_structure}) on Topological Spaces.

The Geometric Realization (\S\ref{sec:geometric_realization}) of any
Simplicial Set (\S\ref{sec:simplicial_set}) or $n$-groupoid
(\S\ref{sec:n_groupoid}), etc. is a CW Complex.

Point: $0$-cell

Open Arc: $1$-cell

Open Disc: $2$-cell

$n$-skeleton

$n$-disc

$n$-sphere $S^n$: Two Cells: $e^0$ and $e^n$, attached by Constant Map
$S^{n-1} \rightarrow e^0$ (equivalently $S^n$ regarded as Quotient
Space (\S\ref{sec:quotient_space}) $D^n/ \partial D^n$

Weak Topology (\S\ref{sec:weak_topology})

Topological Space $X$ constructed Inductively: \cite{hatcher02}
\begin{enumerate}
  \item Discrete Set $X^0$ of $0$-cells of $X$
  \item (Inductive Step) form $n$-skeleton $X^n$ from $X^{n-1}$ by
    attaching $n$-cells $e^n_\alpha$ with Maps $\varphi_\alpha :
    S^{n-1} \rightarrow X^{n-1}$: therefore $X^n$ is the Quotient
    Space (\S\ref{sec:quotient_space}) of the Disjoint Union of
    $X^{n-1}$ with a collection of $n$-discs $D^n_\alpha$ under
    identifications $x \sim \varphi_\alpha(x)$ for $x \in \partial
    D^n_\alpha$:
    \[
      X^{n-1}\amalg_\alpha D^n_\alpha
    \]
    and:
    \[
      X^n = X^{n-1}\amalg_\alpha e^n_\alpha
    \]
    where each $e^n_\alpha$ is an Open $n$-disc.
\end{enumerate}

1-dimensional Cell Complex $X = X^1$ is a \emph{Graph}

Product (\S\ref{sec:cwcomplex_product})

Quotient (\S\ref{sec:cwcomplex_quotient}): Suspension
(\S\ref{sec:suspension}), Join (\S\ref{sec:join}), Wedge Sum
(\S\ref{sec:wedge_sum}), Smash Product (\S\ref{sec:smash_product})



\subsubsection{Euler Characteristic}\label{sec:euler_characteristic}

$\chi = k_0 - k_1 + k_2 - k_3 + \ldots$ where $k_n$ denotes the number
of Cells of Dimension $n$ in the Complex.

Betti Number (\S\ref{sec:betti_number})

depends only on the Homotopy Type (\S\ref{sec:homotopy_type}) of a
CW Complex \cite{hatcher02}



\subsubsection{Characteristic Map}\label{sec:characteristic_map}

\subsubsection{Subcomplex}\label{sec:subcomplex}

Collapsing (\S\ref{sec:collapsible_cwcomplex}) a Contractible
(\S\ref{sec:contractible_cwcomplex}) Subcomplex is a Homotopy
Equivalence (\S\ref{sec:homotopy_equivalence})



\paragraph{CW Pair}\label{sec:cw_pair}\hfill

\emph{CW Pair}



\subsubsection{CW Complex Product}\label{sec:cwcomplex_product}

Product



\subsubsection{CW Complex Quotient}\label{sec:cwcomplex_quotient}

Quotient - CW Pair (\S\ref{sec:cw_pair})



\paragraph{Suspension}\label{sec:suspension}\hfill

\subparagraph{Reduced Suspension}\label{sec:reduced_suspension}\hfill



\paragraph{Join}\label{sec:join}\hfill

Simplex %FIXME hatcher02



\paragraph{Smash Product}\label{sec:smash_product}\hfill



\subsubsection{Contractible CW Complex}
\label{sec:contractible_cwcomplex}

Contractible Space (\S\ref{sec:contractible_space})



\subsubsection{Collapsible CW Complex}
\label{sec:collapsible_cwcomplex}

Collapse (\S\ref{sec:collapse})



\paragraph{Collapse}\label{sec:collapse}\hfill



\subsubsection{Simplicial Complex}\label{sec:simplicial_complex}

Chain Complex (\S\ref{sec:chain_complex})



\paragraph{Abstract Simplical Complex}\label{sec:abstract_complex}\hfill



% --------------------------------------------------------------------
\subsection{Retraction}\label{sec:subspace_retraction}
% --------------------------------------------------------------------

For a Topological Space $(X, \tau)$ and Subspace $A$ of $(X,\tau)$, a
Continuous Map (\S\ref{sec:continuous_map}) $r : X \rightarrow A$ is a
\emph{Retraction} if the Restriction of $r$ to $A$ is the Identity Map
on $A$.

$\iota : A \hookrightarrow X$ (\S\ref{sec:inclusion_map})

$r \circ \iota = id_A$

Topological analog to Projection Operators % FIXME xref
\cite{hatcher02}



\subsubsection{Deformation Retraction}\label{sec:deformation_retraction}

Continuous Map $f : X \times [0,1] \rightarrow X$

Deformation Retraction $f_t : X \rightarrow X$ is a Special case of
Homotopy (\S\ref{sec:homotopy}): a Deformation Retraction of $X$ onto
Subspace $A$ is a Homotopy from $id_X$ to a Retraction of $X$ to $A$,
$r : X \rightarrow X$, such that $r(X) = A$ and $r | A = id_A$.
\cite{hatcher02}



% --------------------------------------------------------------------
\subsection{Homotopy Group}\label{sec:homotopy_group}
% --------------------------------------------------------------------

Fundamental Group (\S\ref{sec:fundamental_group}) $\pi_1(X,x_0)$

$n$-th Homotopy Group $\pi_n(X,x_0)$

$I^n$



% --------------------------------------------------------------------
\subsection{Homology Theory}\label{sec:homology_theory}
% --------------------------------------------------------------------

Homological Algebra (\S\ref{sec:homological_algebra})



\subsubsection{Homology}\label{sec:homology}

\subsubsection{Cohomology}\label{sec:cohomology}

\subsubsection{Hochschild Homology}\label{sec:hochschild_homology}

End (\S\ref{sec:end})



% --------------------------------------------------------------------
\subsection{Teichm\"uller Space}\label{sec:teichmuller_space}
% --------------------------------------------------------------------

Riemann Surface (\S\ref{sec:riemann_surface})

\emph{Inter-universal Teichm\"uller Theory} % FIXME



% ====================================================================
\section{Geometric Topology}\label{sec:geometric_topology}
% ====================================================================

\emph{Geometry} $(X,G)$: Simply-connected Space $X$, Transitive Group $G$,



% --------------------------------------------------------------------
\subsection{Manifold}\label{sec:manifold}
% --------------------------------------------------------------------

$n$-sphere

$n$-disc

1-dimensional:

\begin{itemize}
  \item Unit Circle, $\mathbb{R}/\mathbb{Z}$
\end{itemize}

2-dimensional:

\begin{description}
  \item [Spherical]: 2-sphere ($\mathsf{S}^2$)
  \item [Euclidean]: Torus ($\mathsf{T}^2$),
  $\mathbb{R}^2/\mathbb{Z}^2$
  \item [Hyperbolic]: $\mathsf{H}^2$
\end{description}

3-dimensional:

\begin{description}
  \item [Spherical]: 3-sphere ($\mathsf{S}^3$)
  \item [Euclidean]: 3-Torus ($\mathsf{T}^3$)
  \item [Hyperbolic]: Siefert-Weber Dodecahedral Space, Gieseking
  Manifold
  \item [$S^2 \times E^1$]
  \item [$S^2 \times S^1$]
  \item [$H^2 \times E^1$]: $\mathsf{S}^3 - 3_1$
  \item [$Twisted H^2 \times E^1$]
  \item [Nilgeometry]
  \item [Solvegeometry]
\end{description}



\subsubsection{Submanifold}\label{sec:submanifold}

\subsubsection{Oriented Manifold}\label{sec:oriented_manifold}

Oriented $n$-manifolds form a Commutative Semigroup
(\S\ref{sec:semigroup}) with the Connected Sum
(\S\ref{sec:connected_sum}) as the Semigroup Operation and the
$n$-sphere as the Identity Element.



\paragraph{Genus}\label{sec:genus}\hfill

An Orientable Surface $M_g$ of Genus $g$ can be constructed from a
Polygon with $4g$ sides by identifying pairs of Edges.



\paragraph{Simplicial Volume}\label{sec:simplicial_volume}\hfill

\emph{Simplicial Volume} (or \emph{Gromov Norm})

The Set of Volumes of Hyperbolic 3-manifolds is a Closed, Well-ordered
Subset of $\mathbb{R}$ of Order Type $\omega^\omega$. For each Volume
the Set of 3-manifolds with that Volume is Finite.



\subsubsection{Geometrization Conjecture}
\label{sec:geometrization_conjecture}

Thurston



\subsubsection{Connected Sum}\label{sec:connected_sum}

$M_1 \# M_2$

Semigroup Operation for the Semigroup of Oriented $n$-manifolds
(\S\ref{sec:oriented_manifold}) with the $n$-sphere as the Identity
Element.

Reversible in 3 Dimensions

Prime 3-manifold

\emph{Prime Decomposition Theorem for 3-manifolds} (Kneser): Any
Compact, Orientable 3-manifold is the Connected Sum of a Unique (up to
Homeomorphism) collection of Prime 3-manifolds.



\subsubsection{Prime Manifold}\label{sec:prime_manifold}



\subsubsection{Dehn Surgery}\label{sec:dehn_surgery}

Any Oriented, Closed 3-manifold can be obtained from any other
Oriented, Closed 3-manifold by removing a collection of Disjoint,
Solid Tori (Dehn Drilling) and gluing them back in (Dehn Filling) by a
different identification.



\subsubsection{Differentiable Manifold}
\label{sec:differentiable_manifold}

\subsubsection{Smooth Manifold}\label{sec:smooth_manifold}

\paragraph{Diffeomorphism}\label{sec:diffeomorphism}\hfill

A \emph{Diffeomorphism} is an Isomorphism of Smooth Manifolds.



\subsubsection{Analytic Manifold}\label{sec:analytic_manifold}

\subsubsection{Complex Manifold}\label{sec:complex_manifold}

\subsubsection{Compact Manifold}\label{sec:compact_manifold}

\paragraph{Cobordism}\label{sec:cobordism}\hfill



% --------------------------------------------------------------------
\subsection{Gauge Theory}\label{sec:gauge_theory}
% --------------------------------------------------------------------

%FIXME field theory



\subsubsection{Lagrangian}\label{sec:lagrangian}



% --------------------------------------------------------------------
\subsection{JSJ Decomposition}\label{sec:jsj_decomposition}
% --------------------------------------------------------------------

\emph{Jaco-Shalen-Johannson Decomposition}

Essential Embedded 2-Torus



% --------------------------------------------------------------------
\subsection{Zariski Geometry}\label{sec:zariski_geometry}
% --------------------------------------------------------------------

% --------------------------------------------------------------------
\subsection{Simplical Homology}\label{sec:simplical_homology}
% --------------------------------------------------------------------

% --------------------------------------------------------------------
\subsection{Hodge Theory}\label{sec:hodge_theory}
% --------------------------------------------------------------------

\subsubsection{De Rham Cohomology}\label{sec:derham_cohomology}



% ====================================================================
\section{Knot Theory} \label{sec:knot_theory}
% ====================================================================

Dowker Notation

Conway Notation



Hyperbolic Knot

Torus Knot

Satellite Knot



Unknot

Trefoil Knot

Figure-eight Knot

Hopf Link

Borromean Rings



% --------------------------------------------------------------------
\subsection{Knot} \label{sec:knot}
% --------------------------------------------------------------------

Link with a single component %FIXME



% --------------------------------------------------------------------
\subsection{Link} \label{sec:link}
% --------------------------------------------------------------------

Loop (\S\ref{sec:loop})



\subsubsection{Whitehead Link} \label{sec:whitehead_link}

\subsubsection{Concordance} \label{sec:concordance}



% --------------------------------------------------------------------
\subsection{Ambient Isotopy} \label{sec:ambient_isotopy}
% --------------------------------------------------------------------

% --------------------------------------------------------------------
\subsection{Quandle} \label{sec:quandle}
% --------------------------------------------------------------------

% --------------------------------------------------------------------
\subsection{Torus Knot} \label{sec:torus_knot}
% --------------------------------------------------------------------



% ====================================================================
\section{Braid Theory} \label{sec:braid_theory}
% ====================================================================

% ====================================================================
\section{Differential Topology}\label{sec:differential_topology}
% ====================================================================

% --------------------------------------------------------------------
\subsection{General Position}\label{sec:general_position}
% --------------------------------------------------------------------

% --------------------------------------------------------------------
\subsection{Symplectic Topology}\label{sec:symplectic_topology}
% --------------------------------------------------------------------



% ====================================================================
\section{Singularity Theory}\label{sec:singularity_theory}
% ====================================================================

% ====================================================================
\section{Shape Theory}\label{sec:shape_theory}
% ====================================================================

% ====================================================================
\section{Descent Theory}\label{sec:descent_theory}
% ====================================================================

Beck's Monadicity Theorem (\S\ref{sec:monadic_functor})



% --------------------------------------------------------------------
\subsection{Descent}\label{sec:descent}
% --------------------------------------------------------------------

% --------------------------------------------------------------------
\subsection{Fibred Category}\label{sec:fibred_category}
% --------------------------------------------------------------------



% ====================================================================
\section{Pointless Topology}\label{sec:pointless_topology}
% ====================================================================

Complete Heyting Algebra (\S\ref{sec:complete_heyting})

Objects of $\cat{CHey}$, $\cat{Loc}$, and $\cat{Frm}$ are Complete
Lattices (\S\ref{sec:complete_lattice}) satisfying the Infinite
Distributive Law (\S\ref{sec:infinite_distributive}).

Rings (\S\ref{sec:ring}): Frames (\S\ref{sec:frame})

Schemes (\S\ref{sec:scheme}): Locale (\S\ref{sec:locale})



% --------------------------------------------------------------------
\subsection{Frame}\label{sec:frame}
% --------------------------------------------------------------------

Complete Lattice (\S\ref{sec:complete_lattice})

$\cat{Frm}$ Morphisms are Monotone Functions (Limit-)preserving Finite
Meets and arbitrary Joins

A Morphism of Frames that also preserves Implication is a Homomorphism
of Complete Heyting Algebras (\S\ref{sec:complete_heyting}).

Contravariant Functor $O : \cat{Top} \rightarrow \cat{Frm}$

Adjoint $|-|_{top} : \cat{Frm} \rightarrow \cat{Top}$ %FIXME



% --------------------------------------------------------------------
\subsection{Locale}\label{sec:locale}
% --------------------------------------------------------------------

Topos Theory (\S\ref{sec:topos_theory}) (???)

Topological Spaces (\S\ref{sec:topological_space})

$\cat{Top}$

Ring (\S\ref{sec:ring})

Sierpi\'nski Space (\S\ref{sec:sierpinski_space}) $S$

Topological Space $X = (M_X,\tau_X)$

$Map(X,S) \simeq \tau_X$



% ====================================================================
\section{Arithmetic Topology}\label{sec:arithmetic_topology}
% ====================================================================

Algebraic Number Theory (\S\ref{sec:algebraic_number_theory})
