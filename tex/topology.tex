%%%%%%%%%%%%%%%%%%%%%%%%%%%%%%%%%%%%%%%%%%%%%%%%%%%%%%%%%%%%%%%%%%%%%%
%%%%%%%%%%%%%%%%%%%%%%%%%%%%%%%%%%%%%%%%%%%%%%%%%%%%%%%%%%%%%%%%%%%%%%
\part{Topology}\label{sec:topology}\cite{lc11}
%%%%%%%%%%%%%%%%%%%%%%%%%%%%%%%%%%%%%%%%%%%%%%%%%%%%%%%%%%%%%%%%%%%%%%
%%%%%%%%%%%%%%%%%%%%%%%%%%%%%%%%%%%%%%%%%%%%%%%%%%%%%%%%%%%%%%%%%%%%%%

A \emph{Topology}, $\tau$, is a collection of Subsets called
\emph{Open Sets} (\S\ref{sec:open_set}) of a \emph{Metric Space}
(\S\ref{sec:metric_space}), $M$, subject to the following Inductive
definition:
\begin{enumerate}
\item $\varnothing \in \tau, M \in \tau$
\item $U,V \in \tau \rightarrow U \cap V \in \tau$
\item $\{U_i\}_{i \in I} \subseteq \tau \rightarrow \bigcup_{i \in I}
  U_i \in \tau$
\end{enumerate}

An equivalent definition is possible in terms of \emph{Closed Sets}
(\S\ref{sec:closed_set}).

Given two Topologies, $\tau_1 \subset \tau_2$, $\tau_1$ is
\emph{Coarse} relative to $\tau_2$, and $\tau_2$ is \emph{Fine}
relative to $\tau_1$.



% ====================================================================
\section{Metric Space}\label{sec:metric_space}
% ====================================================================

A \emph{Metric Space} is a Set $M$ for which a \emph{Metric}
(\ref{sec:metric}) is defined for all Elements of that Set.

A Metric Space $M$ can then be defined as the pair:
\[
    (M,d)
\]
where $M$ is a Set of Elements and $d$ is a Distance Function.

Subsets of a Metric Space may be \emph{Open}
(\S\ref{sec:open_set}), \emph{Closed} (\S\ref{sec:closed_set}),
both or neither.



% --------------------------------------------------------------------
\subsection{Metric}\label{sec:metric}
% --------------------------------------------------------------------

A \emph{Metric} is a \emph{Distance Function} defined between all
Elements or Points of a Metric Space.

A Distance Function $d$ has the form:
\[
    d: M \times M \rightarrow \mathbb{R}^{+}
\]
with the following conditions:
\begin{enumerate}
\item $d(x_1, x_2) \geq 0$ (\emph{Non-negativity} or \emph{Separation
  Axiom})
\item $d(x_1, x_2) = 0 \leftrightarrow x_1 = x_2$ (\emph{Identity of
  Indiscernables} or \emph{Coincidence Axiom})
\item $d(x_1, x_2) = d(x_2, x_1)$ (\emph{Symmetry})
\item $d(x_1, x_3) \leq d(x_1, x_2) + d(x_2, x_3)$
  (\emph{Subadditivity} or \emph{Triangle Inequality})
\end{enumerate}
Two Metrics, $d_1$ and $d_2$, in a Metric Space, $M$, are
\emph{Metrically Equivalent}, $d_1 \sim d_2$, if for $\tau_1$ Induced
by $d_1$ and $\tau_2$ Induced by $d_2$, $\tau_1 = \tau_2$. A
sufficient condition for Metric Equivalence is given by:
\[
    \exists k_1, k_2 > 0 : \forall x \in M, \forall r > 0,
    B^{d1}_{rk_1}(x) \subseteq B^{d2}_r (x) \subseteq B^{d1}_{rk_2}(x)
\]

\emph{Euclidean Metric}:
\[
    d: \mathbb{R}^n \times \mathbb{R}^n \rightarrow \mathbb{R}
\]\[
    (\mathbf{p},\mathbf{q}) \mapsto \sqrt{\sum_{i=1}^{n}(q_i - p_i)^2}
\]

\emph{Discrete Metric}:
\[
    d: X \times X \rightarrow \mathbb{R}^{+}
\]\[
    (\mathbf{p},\mathbf{q}) \mapsto \left\{
    \begin{array}{l l}
        0: \mathbf{p} = \mathbf{q}\\
        1: \mathbf{p} \neq \mathbf{q}
    \end{array}\right.
\]

\emph{Max Metric}:
\[
    d: \mathbb{R}^n \times \mathbb{R}^n \rightarrow \mathbb{R}
\]\[
    (\mathbf{p},\mathbf{q}) \mapsto max_{1 \leq i \leq n}\{|q_i - p_i|\}
\]

\emph{Taxicab Metric}:
\[
    d: \mathbb{R}^n \times \mathbb{R}^n \rightarrow \mathbb{R}
\]\[
    (\mathbf{p},\mathbf{q}) \mapsto \sum_{i=1}^{n}|q_i - p_i|
\]



% --------------------------------------------------------------------
\subsection{Ball}\label{sec:metric_ball}
% --------------------------------------------------------------------

A Metric \emph{Ball} is defined for a Point $p$ in a Metric Space
$(M,d)$ as the set of all Points (including $p$) within a given Radius
$r > 0$ as determined by the Distance Function of the Metric Space:
\[
    B_r(p) = {x \in M | d(x,p) < r }
\]
The above is termed an \emph{Open Ball} because it does not include
the points where $d(x,p) = r$. Such a Ball including these additional
Points is called a \emph{Closed Ball}.



% --------------------------------------------------------------------
\subsection{Open Set}\label{sec:open_set}
% --------------------------------------------------------------------

An \emph{Open Set} is a Subset of a Metric Space, defined in terms of
\emph{Open Balls} (\S\ref{sec:metric_ball}). For a Metric Space
$(M,d)$, the Set $U \subseteq M$ is \emph{Open} if
\[
    \forall x \in U, \exists r > 0 : B_r(x) \subseteq U
\]
where $B_r(x)$ is an Open Ball centered on Point $x$.

Open Sets of a Metric Space $(M,d)$ have the following three
properties:
\begin{enumerate}
\item $\varnothing, M$ are Open Sets
\item If $U, V \in M$ are Open, then $U \cap V$ is Open in $M$
\item If $\{ U_i \}_{i \in I}$ are Open, then $\bigcup_{i \in I}
  U_i$ is an Open Set
\end{enumerate}
Note that an arbitrary Intersection of Open Sets is not necessarily an
Open Set.



% --------------------------------------------------------------------
\subsection{Closed Set}\label{sec:closed_set}
% --------------------------------------------------------------------

A Subset, $X \subseteq M$, of a Metric Space, $M$, is a \emph{Closed
  Set} if the Relative Complement (\ref{sec:relative_complement}), $M
\backslash X$, is an Open Set.

The three Properties of Closed Sets:
\begin{enumerate}
\item Given Topological Space $(X, \tau)$, if $\varnothing, X \in
  \tau$ then $\varnothing$ and $X$ are Closed
\item If $C, D$ are Closed in $X$, then $C \cup D$ is Closed in $X$
\item If $\{ U_i \}_{i \in I}$ are Closed, then $\bigcap_{i \in I}
  U_i$ is Closed
\end{enumerate}

If $C$ is a Closed Set and $f$ is a Continuous Function
(\S\ref{sec:topological_continuity}), $f^{-1}(C)$ is a Closed Set.



\subsubsection{Perfect Set}

a Closed Set with no \emph{Isolated Points}
(\S\ref{sec:isolated_point})



% ====================================================================
\section{Topological Space}\label{sec:topological_space}
% ====================================================================

A \emph{Topological Space} is a pair consisting of a Metric Space,
$M$, and a Toplogy, $\tau$ on that Metric Space:
\[
    (M,\tau)
\]

Every Metric Space gives a Topology, but Topologies may exist for
which there is no definable Metric Space. %FIXME explain

Given a Metric Space $M$, the following Topologies may be described:
\begin{description}
\item[Trivial Topology] $\tau = \{\varnothing, M\}$ (Open Sets under
  any Metric)

\item[Discrete Topology] $\tau = \mathcal{P}(M)$ (Open Sets under
  Discrete Metric)
\end{description}

By the \emph{Homotopy Hypothesis}, $\inf$-groupoids
(\S\ref{sec:infinity_groupoid}) are Spaces.



% --------------------------------------------------------------------
\subsection{Base}\label{sec:topological_base}
% --------------------------------------------------------------------

A \emph{Base}, $B$, is a Subset of a Topology, $\tau$, in a Metric
Space, $(M,\tau)$, such that:
\[
    \forall U \in \tau, \exists \{B_i\}_{i \in I} \subseteq B :
    \bigcup_{i \in I}B_i = U
\]
Properties:
\begin{enumerate}
    \item $B$ is a \emph{Covering} (\S\ref{sec:cover}) of $M$, as
      stated by:
\[
    M \subseteq \bigcup_{i \in I} B_i
\]

    \item
\[
    \forall B_1, B_2 \in B, \forall x \in B_1 \cap B_2,
    \exists B_3 \in B : x \in B_3 \wedge B_3 \subseteq B_1 \cap B_2
\]

\end{enumerate}
An example of a Base is the Set of all Open Balls in a Metric Space.

A Base is not necessarily Unique for a given Topology. Adding Elements
to a Base results in another Base.



\subsubsection{Subbase}

A \emph{Subbase}, $S$, is a Subset of a Topology, $\tau$, in a Metric
Space, $(M,\tau)$, such that the Set:
\[
    S \subseteq \tau : \{ \bigcap_{j \in J} S_j : |J| < \infty \}
\]
is a Base for $\tau$.

There is no unique Subbase for a given Topology but there is a unique
Topology for a given Subbase.



% --------------------------------------------------------------------
\subsection{Cover}\label{sec:cover}
% --------------------------------------------------------------------

\subsubsection{Refinement}\label{sec:refinement}



% --------------------------------------------------------------------
\subsection{Countability Axioms}\label{sec:countability_axioms}
% --------------------------------------------------------------------

A Topological Space, $(M,\tau)$, is \emph{First Countable} or $1c$ if
$\forall X \in M$, there exists a Countable Neighborhood Base
(\S\ref{sec:neighborhood_base}).

And the Topological Space is \emph{Second Countable} or $2c$ if $\tau$
has a Countable Base.
\[
    2c \rightarrow 1c
\]

All Metric Spaces are $1c$.



% --------------------------------------------------------------------
\subsection{Compact Space}\label{sec:compact_space}
% --------------------------------------------------------------------

\subsubsection{Tychonoff's Theorem}\label{sec:tychonoffs_theorem}

Compactness Theorem (Model Theory \S\ref{sec:compactness})



% ====================================================================
\section{Point-set Topology}\label{sec:point_set}
% ====================================================================

% --------------------------------------------------------------------
\subsection{Point}\label{sec:topological_point}
% --------------------------------------------------------------------

\emph{Point}

Two Points that are within the same Open Set of a Topology are said to
be \emph{Topologically Indistinguishable}.



\subsubsection{Isolated Point}\label{sec:isolated_point}



% --------------------------------------------------------------------
\subsection{Interior}\label{sec:interior}
% --------------------------------------------------------------------

For a Subset, $V$, of a Topology, $\tau$, a Point $x \in V$ is in the
\emph{Interior} of $V$, $V^{\circ}$, if there is a \emph{Neighborhood}
of $x$, $N \subset V$. Inductively, for $\{A_i\}_{i \in I} \subseteq
\tau \wedge \forall i, A_i \subseteq V$:
\[
    V^{\circ} = \bigcup_{i \in I} A_i \subseteq \tau
\]
Equivalently, the Interior of $V$ is every Open Set within $V$ and is
itself an Open Set.



\subsubsection{Neighborhood}\label{sec:neighborhood}

A \emph{Neighborhood} of a Point $x$ in a Topological Space $(M,\tau)$
is a Set $V \subseteq M$ such that:
\[
    \exists U \in \tau : U \subseteq V \wedge x \in U
\]
That is, $x$ is in the Interior of $V$.



\subsubsection{Neighborhood System}\label{sec:neighborhood_system}

Given a Point $x$ in any Topological Space, a \emph{Neighbordhood
  System} (or \emph{Neighborhood Filter}), $\mathcal{V}(x)$, is the
Set of all Neighborhoods of $x$.



\subsubsection{Neighborhood Base}\label{sec:neighborhood_base}

Given a Neighborhood System, $V(x)$, a \emph{Neighborhood Base} for
$x$ is defined as a Subset of the Neighborhood System, $\mathcal{B}(x)
\subseteq \mathcal{V}(x)$, such that:
\[
    \forall v \in V(x), \exists b \in \mathcal{B}(x) : b \subseteq v
\]



% --------------------------------------------------------------------
\subsection{Continuity}\label{sec:topological_continuity}
% --------------------------------------------------------------------

\emph{Continuity} of a Function $f : X \rightarrow Y$ between Metric
Spaces $(X,d)$ and $(Y,d')$ is defined at a Point $c$
if
\[
    \forall \epsilon > 0, \exists \delta > 0 :
    f (B_{\delta}(c)) \subseteq B_{\epsilon}(f(c))
\]
and likewise a \emph{Continuous Function} satisfies the above for all
$c$.

Alternatively, a Function $f: X \rightarrow Y$ is Continuous between
Metric Spaces if and only if for all Open Sets $V \subseteq Y$,
$f^{-1}(v)$ is an Open Set in $X$.

Continuity of a Function, $f : X \rightarrow Y$, between Topological
Spaces, $(X,\tau_1)$ and $(Y,\tau_2)$, is given by:
\[
    \forall V \in \tau_2, f^{-1}(V) \in \tau_1
\]
The following are Properties of Continuous Functions between
Topological Spaces:
\begin{enumerate}
    \item Any Constant Function is Continuous
    \item Given two Continuous Functions, $f : X \rightarrow Y$ and $g
      : Y \rightarrow Z$, the Function $g \circ f : X \rightarrow Z$ is
      Continuous
    \item Given $f : (X, \tau)) \rightarrow (Y, \sigma)$, $f$ is
      Continuous if $\tau = \mathcal{P}(X)$ (Discrete Topology) or
      $\sigma = \{\varnothing, Y\}$ (Trivial Topology)
\end{enumerate}



% --------------------------------------------------------------------
\subsection{Connectedness}\label{sec:connectedness}
% --------------------------------------------------------------------



% ====================================================================
\section{Separation Axioms}\label{sec:separation_axioms}
% ====================================================================



% ====================================================================
\section{Algebraic Topology}\label{sec:algebraic_topology}
% ====================================================================

% --------------------------------------------------------------------
\subsection{Homology Theory}\label{sec:homology_theory}
% --------------------------------------------------------------------



% ====================================================================
\section{Geometric Topology}\label{sec:geometric_topology}
% ====================================================================

% --------------------------------------------------------------------
\subsection{Manifold}\label{sec:manifold}
% --------------------------------------------------------------------

% --------------------------------------------------------------------
\subsection{Zariski Geometry}\label{sec:zariski_geometry}
% --------------------------------------------------------------------



% ====================================================================
\section{Differential Topology}\label{sec:differential_topology}
% ====================================================================

% --------------------------------------------------------------------
\subsection{Symplectic Topology}\label{sec:symplectic_topology}
% --------------------------------------------------------------------



% ====================================================================
\section{Singularity Theory}\label{sec:singularity_theory}
% ====================================================================



% ====================================================================
\section{Pointless Topology}\label{sec:pointless_topology}
% ====================================================================
