%%%%%%%%%%%%%%%%%%%%%%%%%%%%%%%%%%%%%%%%%%%%%%%%%%%%%%%%%%%%%%%%%%%%%%%%%%%%%%%%
%%%%%%%%%%%%%%%%%%%%%%%%%%%%%%%%%%%%%%%%%%%%%%%%%%%%%%%%%%%%%%%%%%%%%%%%%%%%%%%%
\part{Topology}\label{part:topology}\cite{lc11}
%%%%%%%%%%%%%%%%%%%%%%%%%%%%%%%%%%%%%%%%%%%%%%%%%%%%%%%%%%%%%%%%%%%%%%%%%%%%%%%%
%%%%%%%%%%%%%%%%%%%%%%%%%%%%%%%%%%%%%%%%%%%%%%%%%%%%%%%%%%%%%%%%%%%%%%%%%%%%%%%%

\emph{analysis situs}

(wiki):

\emph{Distinction between Geometry and Topology}:
\begin{itemize}
\item Geometry has Local (Infinitesimal) structure; Continuous Moduli
  (\S\ref{sec:moduli_space}); Metric Spaces (\S\ref{sec:metric_space})
\item Topology only has Global structure; Discrete Moduli; Topological Spaces
  (\S\ref{sec:topological_space})
\end{itemize}

Symplectic Geometry and Topology is a ``boundary case''-- by Darboux's Theorem
(\S\ref{sec:darbouxs_theorem}) a Symplectic Manifold has no \emph{Local}
structure (Topological), but the Space of Symplectic Structures on a Manifold
form a Continuous Moduli (Geometrical); up to Isotopy, the Space of Symplectic
Structures is Discrete (FIXME: xrefs)

see also:
\begin{itemize}
  \item Topological Mixing (Ergodic Theory \S\ref{sec:topological_mixing})
\end{itemize}



% ==============================================================================
\section{Topology}\label{sec:topology}
% ==============================================================================

Given a Set $X$, a \emph{Topology} $\tau$ on $X$ is a Family
(\S\ref{sec:family}) of Subsets of $X$--the Open Sets
(\S\ref{sec:open_set})--with the Properties:
\begin{enumerate}
  \item $\varnothing \in \tau$, $X \in \tau$
  \item $A,B \in \tau \Rightarrow A \cap B \in \tau$
  \item $\{A_i\}_{i \in I} \subseteq \tau \Rightarrow \bigcup_{i \in
  I} A_i \in \tau$
\end{enumerate}
The Topology $\tau$ is therefore a Subset of the Power Set of $X$:
\[
  \tau \subseteq \pow(X)
\]

\fist on Finite or Countably Infinite Sets, every \emph{$\sigma$-algebra}
(\S\ref{sec:sigma_algebra}) is a Topology;
Topologies however do not require Complements and Topologies only require all
Finite Intersections instead of all \emph{Countable} Intersections



% ------------------------------------------------------------------------------
\subsection{Open Set}\label{sec:open_set}
% ------------------------------------------------------------------------------

An \emph{Open Set} is a Subset of a Metric Space, defined in terms of
\emph{Open Balls} (\S\ref{sec:ball}). For a Metric Space $(M,d)$, the
Set $U \subseteq M$ is \emph{Open} if
\[
  \forall x \in U, \exists r > 0 : B_r(x) \subseteq U
\]
where $B_r(x)$ is an Open Ball centered on Point $x$.

Open Sets of a Metric Space $(M,d)$ have the following three
properties:
\begin{enumerate}
\item $\varnothing, M$ are Open Sets
\item If $U, V \in M$ are Open, then $U \cap V$ is Open in $M$
\item If $\{ U_i \}_{i \in I}$ are Open, then $\bigcup_{i \in I}
  U_i$ is an Open Set
\end{enumerate}
\fist Note that the Intersection of an arbitrary collection of
Open Sets is not necessarily an Open Set.

Given a Set $X$, a \emph{Topology} (\S\ref{sec:topology}) $\tau$ on $X$ is a
Family (\S\ref{sec:family}) of Subsets of $X$--the Open Sets--with the
Properties:
\begin{enumerate}
  \item $\varnothing \in \tau$, $X \in \tau$
  \item $A,B \in \tau \Rightarrow A \cap B \in \tau$
  \item $\{A_i\}_{i \in I} \subseteq \tau \Rightarrow \bigcup_{i \in
  I} A_i \in \tau$
\end{enumerate}
The Topology $\tau$ is therefore a Subset of the Power Set of $X$:
\[
  \tau \subseteq \pow(X)
\]



\subsection{Open Map}\label{sec:open_map}

a Map from Open Sets to Open Sets

an Open Continuous Surjective Map is necessarily a Quotient Map
(\S\ref{sec:quotient_map})



\subsubsection{Borel Set}\label{sec:borel_set}

(wiki):

a \emph{Borel Set} is any Set in a Topological Space that can be formed from
\emph{Open Sets} through Countable Union, Countable Intersection, and Relative
Complement

for a Topological Space $X$, the collection of all Borel Sets on $X$ forms a
$\sigma$-algebra (\S\ref{sec:sigma_algebra}) called the \emph{Borel Algebra}
(\S\ref{sec:borel_algebra}) on $X$, which is the \emph{smallest}
$\sigma$-algebra containing all Open Sets (or equivalently all Closed Sets)

the \emph{Borel Space} (\S\ref{sec:measurable_space}) of a Topological Space $X$
is the pair $(X,B)$ where $B$ is the $\sigma$-algebra of Borel Sets of $X$; cf.
Standard Borel Space (\S\ref{sec:standard_borel_space}) of a Polish Space

any Measure defined on Borel Sets is called a Borel Measure
(\S\ref{sec:borel_measure})

Measurable Sets (\S\ref{sec:measurable_set}) on the Real Line are ``iterated''
Countable Unions and Intersections of Borel Sets

\fist Borel Hierarchy (\S\ref{sec:borel_hierarchy})



\subsubsection{Baire Set}\label{sec:baire_set}

a Baire Set is a Set whose Characteristic Function (Indicator Function
\S\ref{sec:indicator_function}) is a Baire Function (\S\ref{sec:baire_function})



% ------------------------------------------------------------------------------
\subsection{Closed Set}\label{sec:closed_set}
% ------------------------------------------------------------------------------

A Subset, $X \subseteq M$, of a Metric Space, $M$, is a \emph{Closed
  Set} if the Relative Complement (\ref{sec:relative_complement}), $M
\backslash X$, is an Open Set. Likewise, the Complement of an Open Set
is a Closed Set.

A Set is Closed if and only if it contains all its Limit Points
(\S\ref{sec:limit_point}).

Properties of Closed Sets:
\begin{enumerate}
  \item Given Topological Space $(X, \tau)$, $\varnothing$ and $X$ are
    Closed
  \item If $C, D$ are Closed in $X$, then $C \cup D$ is Closed in $X$
  \item If $\{ U_i \}_{i \in I}$ are Closed, then $\bigcap_{i \in I}
  U_i$ is Closed
\end{enumerate}

If $C$ is a Closed Set and $f$ is a Continuous Function
(\S\ref{sec:continuous_function}), $f^{-1}(C)$ is a Closed Set.

\fist Note that the Complement of a Closed Set is not
necessarily an Open Set, and a Set may be both Open and Closed
(\emph{Clopen}) or neither Open nor Closed.

In the Discrete Topology, every Open Set is also a Closed Set. In the
Trivial Topology both the Empty Set and the entire Set are both Open
and Closed Sets.

\fist cf. a Closed Manifold (\S\ref{sec:closed_manifold}) is a Manifold that is
Compact (\S\ref{sec:compact_space}) and \emph{without} a Boundary

\fist \emph{Hyperconnected Space} (\S\ref{sec:hyperconnected_space}) is a
Topological Space that cannot be written as the Union of two Proper Closed Sets
(either Disjoint or Non-disjoint).



\subsection{Closed Map}\label{sec:closed_map}

a Map from Closed Sets to Closed Sets

a Closed Continuous Surjective Map is necessarily a Quotient Map
(\S\ref{sec:quotient_map})



\subsubsection{Perfect Set}\label{sec:perfect_set}

a Closed Set with no \emph{Isolated Points}
(\S\ref{sec:isolated_point})



% ------------------------------------------------------------------------------
\subsection{Clopen Set}\label{sec:clopen_set}
% ------------------------------------------------------------------------------

% ------------------------------------------------------------------------------
\subsection{Almost Open Set}\label{sec:almost_open}
% ------------------------------------------------------------------------------

or \emph{Baire Property}

a Subset $A$ of a Topological Space is \emph{Almost Open} if it differs from an
Open Set by a Meagre Set (\S\ref{sec:meagre_set})



% ------------------------------------------------------------------------------
\subsection{Digraph Topology}\label{sec:digraph_topology}
% ------------------------------------------------------------------------------

(wolfram): a \emph{Digraph Topology} is an Unlabeled Transitive Digraph
(\S\ref{sec:transitive_digraph})



% ------------------------------------------------------------------------------
\subsection{Trivial Topology}\label{sec:trivial_topology}
% ------------------------------------------------------------------------------

or \emph{Indiscrete Topology}

the Trivial Topology is the Topology with the least possible number of Open
Sets and is the Coarsest (\S\ref{sec:coarseness}) possible Topology



% ------------------------------------------------------------------------------
\subsection{Discrete Topology}\label{sec:discrete_topology}
% ------------------------------------------------------------------------------

Discrete Metric

Discrete Space (\S\ref{sec:discrete_space}) -- any Function from a Discrete
Space to another Topological Space is Continuous (\S\ref{sec:continuous_map})

Finest (\S\ref{sec:coarseness}) possible Topology that can be given on a Set,
i.e. all Subsets are Open Sets (and therefore Closed Sets, so ``Clopen'' Sets)
and in particular each Singleton is an Open Set

A Topological Group (\S\ref{sec:topological_group}) with a Discrete Topology is
a \emph{Discrete Group} (\S\ref{sec:discrete_group}), and any Group can be
considered as a Topological Group by giving it the Discrete Topology.



% ------------------------------------------------------------------------------
\subsection{Compact-open Topology}\label{sec:compact_open}
% ------------------------------------------------------------------------------

Defined on Set of Continuous Maps (\S\ref{sec:continuous_map}) between two
Topological Spaces (\S\ref{sec:topological_space}).

Compact Convergence (\S\ref{sec:compact_convergence})

a \emph{Normal Family} (\S\ref{sec:normal_family}) is a Pre-compact Subset of
the Space of Continuous Functions with respect to the Compact-open Topology



% ------------------------------------------------------------------------------
\subsection{Cofinite Topology}\label{sec:cofinite_topology}
% ------------------------------------------------------------------------------

\emph{Cofinite Topology} (or \emph{Finite Complement Topology})



% ------------------------------------------------------------------------------
\subsection{Cocountable Topology}\label{sec:cocountable_topology}
% ------------------------------------------------------------------------------

\emph{Cocountable Topology}



% ------------------------------------------------------------------------------
\subsection{Order Topology}\label{sec:order_topology}
% ------------------------------------------------------------------------------

% ------------------------------------------------------------------------------
\subsection{Uniform Norm Topology}\label{sec:uniform_norm_topology}
% ------------------------------------------------------------------------------

$\infty$-norm (``Max Norm'' \S\ref{sec:p_norm})

the Set of Nowhere-differentiable (\S\ref{sec:nowhere_differentiable})
Real-valued Functions on the Closed Interval $[0,1]$ is Comeagre
(\S\ref{sec:comeagre_set}) in the Vector Space $C([0,1]; \reals)$ of Continuous
Real-valued Functions on $[0,1]$ with the Topology of Uniform Convergence



% ==============================================================================
\section{Topological Space}\label{sec:topological_space}
% ==============================================================================

A \emph{Topological Space} is a pair consisting of a Set, $X$, and a
Toplogy, $\tau$ on that Set:
\[
  (X,\tau)
\]

roughly, a Topological Space is a \emph{Geometric Object}
(\S\ref{sec:geometric_object})

Every Metric Space (\S\ref{sec:metric_space}) gives a Topology, but Topologies
may exist for which there is no definable Metric Space. See Metrizable Spaces
(\S\ref{sec:metrizable_space}).

Given a Metric Space $M$, the following Topologies may be described:
\begin{description}
\item[Trivial Topology] $\tau = \{\varnothing, M\}$ (Open Sets under
  any Metric)

\item[Discrete Topology] $\tau = \pow(M)$ (Open Sets under
  Discrete Metric)
\end{description}
The Discrete Metric Induces the Discrete Topology. There is no Metric
that Induces the Trivial Topology.

By the \emph{Homotopy Hypothesis}, $\inf$-groupoids
(\S\ref{sec:infinity_groupoid}) are Spaces.

\fist Topological Manifold (\S\ref{sec:topological_manifold}): a Topological
Space which \emph{Locally} resembles Real $n$-dimensional Space

A \emph{Continuous Group Action} (\S\ref{sec:continuous_group_action}) on a
Topological Space $X$ is a Group Action of a Topological Group
(\S\ref{sec:topological_group}) $G$ that is a \emph{Continuous Map}
(\S\ref{sec:continuous_map}).

\fist See also Topological Semantics
(\S\ref{sec:topological_semantics}) for Denotational Semantics based
on Topological Spaces

\emph{Stone Duality} (\S\ref{sec:stone_duality}) -- Dualities between
Topological Spaces and Partially Ordered Sets (\S\ref{sec:poset})

the Category of Sets is equivalent to the Category of Discrete Topological
Spaces
(\url{https://math.stackexchange.com/questions/3133963/is-the-theory-of-the-category-of-topological-spaces-computable})

(Tao10):
the notion of a Topological Space, $(X,\tau)$, and Continuous Functions
(\S\ref{sec:continuous_function}) is similar to that of a \emph{Measurable
  Space} (\S\ref{sec:measurable_space}), $(X,\sigma)$, and Measurable Functions
(\S\ref{sec:measurable_function})
\begin{itemize}
  \item both $\tau$ and $\sigma$ contain both $\varnothing$ and $X$
  \item $\sigma$ is closed under Complement and Countable Unions (and
    by implication, Countable Intersections)
  \item $\tau$ is closed under Finite Intersections and Finite or Infinite
    Unions
\end{itemize}
The Open Sets $\tau$ of a Topological Space \emph{generate} a $\sigma$-algebra,
known as the \emph{Borel Algebra} (\S\ref{sec:borel_algebra}), that is, the
smallest $\sigma$-algebra containing all Open Sets (or equivalently, all Closed
Sets)

\fist Topological Support (\S\ref{sec:topological_support})

\fist a Random Field (\S\ref{sec:random_field}) is a Stochastic Process
generalized from a Integer or Real-valued ``Time'' Index Sets, to Index Sets of
general Topological Spaces

\asterism

\begin{itemize}
  \item Baire Space (\S\ref{sec:baire_space}) -- Topological Space such that
    every Intersection of a Countable collection of Open Dense Sets is also
    Dense; e.g. Locally Compact Hausdorff Spaces, Complete Metric Spaces
\end{itemize}



% ------------------------------------------------------------------------------
\subsection{Finite Topological Space}\label{sec:finite_topological_space}
% ------------------------------------------------------------------------------

the number of possible Topologies on a Finite Set is equal to the number of
Preorders (\S\ref{sec:preorder}) on the Set

the number of possible Distinguishable ($T_0$) Topologies
(\S\ref{sec:distinguishable_space}) on a Finite Set is equal to the number of
Partial Orders (\S\ref{sec:partial_order}) on the Set



\subsubsection{Sierpi\'nski Space}\label{sec:sierpinski_space}

(or \emph{Connected Two-point Set})

Point Set $\{ \bot, \top \}$ with Open Sets:
\[
  \{ \varnothing, \{\top\}, \{\bot, \top\} \}
\]

smallest Topological Space that is neither Trivial nor Discrete

$(S,\tau_S)$

Point Set $S = \{0,1\}$

Open Sets $\tau_S = \{\varnothing, \{1\}, \{0,1\}\}$

Closed Sets $\{\varnothing, \{0\}, \{0,1\}\}$

For a Topological Space $(X,\tau_X)$ and Open Subset $U \in \tau_X$,
define the \emph{Characteristic Map} $\chi_U : X \rightarrow S$ of $U$ to be:
\[
  \chi_U (x) =
  \begin{cases}
    1  & \text{if}\; x \in U \\
    0  & \text{otherwise} \\
  \end{cases}
\]

$\cat{Top}((X,\tau_X),(S,\tau_S)) \cong \tau_X$

Ring Structure on $S$: $\vee$, $\wedge$



% ------------------------------------------------------------------------------
\subsection{Discrete Space}\label{sec:discrete_space}
% ------------------------------------------------------------------------------

Topological Space with the Discrete Topology (\S\ref{sec:discrete_topology})

any Function from a Discrete Space to another Topological Space is Continuous
(\S\ref{sec:continuous_map})

\begin{itemize}
  \item $O(1)$ -- the $1$-dimensional Orthogonal Group
    (\S\ref{sec:orthogonal_group}), a two-point Discrete Space, Isomorphic to
    the $0$-sphere $S^0$
\end{itemize}



% ------------------------------------------------------------------------------
\subsection{Metrizable Space}\label{sec:metrizable_space}
% ------------------------------------------------------------------------------

Topological Space $(X, \tau)$ is a \emph{Metrizable Space} if there
exists a Metric $d : X \times X \rightarrow [0, \infty)$ such that the
  Topology Induced by $d$ is $\tau$.



\subsubsection{Metrization Theorem}\label{sec:metrization_theorem}



% ------------------------------------------------------------------------------
\subsection{Uniform Space}\label{sec:uniform_space}
% ------------------------------------------------------------------------------

formal notion of Relative Closeness, Closeness of Points

generalizes Metric Spaces (\S\ref{sec:metric_space}) and Topological
Groups (\S\ref{sec:topological_group})

Mathematical Analysis (Part \ref{part:mathematical_analysis})



\subsubsection{Topological Vector Space}\label{sec:topological_vectorspace}

or \emph{Linear Topological Space}

Functional Analysis (\S\ref{sec:functional_analysis})

Uniform Space with the Algebraic concept of a Vector Space
(\S\ref{sec:vector_space})

Elements are Functions or Linear Operators acting on Topological
Vector Spaces

Hilbert Space (\S\ref{sec:hilbert_space})

Banach Space (\S\ref{sec:banach_space})

a Linear Map (\S\ref{sec:linear_transformation}) between Topological Vector
Spaces may be Continuous (\S\ref{sec:continuous_map}); if the Domain
and Codomain are the same it is a Continuous Linear Operator
(\S\ref{sec:continuous_linear})



\paragraph{Unbounded Linear Operator}\label{sec:unbounded_linear_operator}\hfill

%FIXME: move this section ???

the \emph{Spectrum} (\S\ref{sec:spectrum}) of an Unbounded Linear Operator
$T : X \to X$ is the Set of Elements:
\[
  \{ \lambda \in X | T - \lambda I \text{is not Invertible} \}
\]
where $I$ is the Identity Operator



\subparagraph{Bounded Linear Operator}\label{sec:bounded_linear_operator}\hfill

a Linear Transformation $L$ on Toplogical Vector Spaces
(\S\ref{sec:topological_vector_space}) such that the Ratio of the Norm of $L(v)$
to $v$ is Bounded above by the same number (the \emph{Operator Norm}) for all
Non-zero Vectors, i.e. $\exists M \geq 0 ||L(v)|| \leq M||v||$

a Linear Transformation between two Normed Vector Spaces
(\S\ref{sec:normed_vectorspace}) is a Bounded Linear Operator if and
only if it is a Continuous Linear Operator
(\S\ref{sec:continuous_linear})

\begin{itemize}
  \item Orthogonal Projection (\S\ref{sec:orthogonal_projection})
  \item Hermitian Adjoint (Adjoint Operator \S\ref{sec:adjoint_operator})
\end{itemize}



\paragraph{Continuous Dual Space}\label{sec:continuous_dual_space}
\hfill

the \emph{Continuous Dual Space} of a Topological Vector Space is the Subspace
of the the (Algebraic) Dual Space (\S\ref{sec:dual_space}) corresponding to the
\emph{Continuous Linear Functionals} (Continuous Linear Forms
\S\ref{sec:linear_form})



\paragraph{Topological Tensor Product}\label{sec:topological_tensor}
\hfill

Well-behaved theory of Tensor Products for Hilbert Spaces
(\S\ref{sec:hilbert_tensor}), Nuclear Spaces
(\S\ref{sec:nuclear_space})



\paragraph{Nuclear Space}\label{sec:nuclear_space}\hfill

has many properties of Finite-dimensional Vector Spaces
(\S\ref{sec:vector_space})

all Finite-dimensional Vector Spaces are Nuclear Spaces

every Nuclear Space is a Schwartz Space (\S\ref{sec:schwartz_space})

example Nuclear Space: Set of Smooth Functions
(\S\ref{sec:smooth_function}) on a Compact Manifold
(\S\ref{sec:compact_manifold})

\emph{Bochner-Minlos Theorem} -- for Nuclear Space
$A = \bigcap_{k=0}^\infty H_k$ of $H_k$ Hilbert Spaces, the Bochner-Minlos
Theorem guarantees the existence of a Probability measure with Characteristic
Function $e^{-\frac{1}{2}|y|_{H_0}^2}$, that is, the existence of the ``Gaussian
  Measure'' on the Dual Space called the \emph{White Noise Measure}; when $A$ is
  ``the'' Schwartz Space the corresponding Random Element is a Random
  Distribution (TODO: xrefs)



% ------------------------------------------------------------------------------
\subsection{Topological Dimension}\label{sec:topological_dimension}
% ------------------------------------------------------------------------------

(Mandelbrot82) \fist Fractal Dimension (\S\ref{sec:fractal_dimension}) --
Non-topological aspect of Form (Fractal Form); Hausdorff Dimension
(\S\ref{sec:hausdorff_dimension})




\subsubsection{Lebesgue Covering Dimension}\label{sec:lebesgue_dimension}

or just ``\emph{Topological Dimension}''

(Mandelbrot 1975): defines a ``Fractal'' (\S\ref{sec:fractal}) as an object
whose Hausdorff Dimension (\S\ref{sec:hausdorff_dimension}) is greater than its
Topological Dimension (Lebesgue Covering Dimension
\S\ref{sec:lebesgue_dimension}); note this requirement is not met by Fractal
Space-filling Curves, e.g. the Hilbert Curve



\subsubsection{Inductive Dimension}\label{sec:inductive_dimension}

agrees with Hausdorff Dimension (\S\ref{sec:hausdorff_dimension}) of a Metric
Space



% ------------------------------------------------------------------------------
\subsection{Attaching Space}\label{sec:attaching_space}
% ------------------------------------------------------------------------------

\subsubsection{Attaching Map}\label{sec:attaching_map}

\subsubsection{Mapping Cone}\label{sec:mapping_cone}



% ------------------------------------------------------------------------------
\subsection{Contractible Space}\label{sec:contractible_space}
% ------------------------------------------------------------------------------

A Topological Space $X$ is \emph{Contractible} if the Identity Map on
$X$ is Null-homotopic (\S\ref{sec:null_homotopy}).

Contractible CW Complex (\S\ref{sec:contractible_cwcomplex})



% ------------------------------------------------------------------------------
\subsection{Quotient Space}\label{sec:quotient_space}
% ------------------------------------------------------------------------------

Topological Space $(X, \tau_X)$, Equivalence Relation $\sim$ on $X$

\emph{Quotient Space} $(X/\sim, \tau_{X/\sim})$ is the Set of Equivalence
Classes of Elements of $X$:
\[
  X / \sim = \{ [x] | x \in X \}
\]
with Quotient Topology (\S\ref{sec:quotient_topology}) $\tau_{X/\sim}$:
\[
  \tau_{X/\sim} = \{ U \subseteq (X/\sim) |
    U = (\bigcup_{[a] \in U} [a]) \in \tau_X \}
\]
or equivalently, for a Surjective Function $q : X \rightarrow (X / \sim)$
sending Points in $X$ to the Equivalence Class containing it:
\[
  \tau_{X/\sim} = \{ U \subseteq (X/\sim) | q^{-1}(U) \in \tau_X \}
\]
that is, the Sets with an Open Preimage under $q$.

In a Category of Spaces (e.g. $\cat{Top}$ of Topological Spaces or $\cat{Loc}$
of Locales), a Quotient Space is a Quotient Object
(\S\ref{sec:quotient_object}).

A Quotient Space $X/~$ together with Quotient Map $q : X \rightarrow X/~$ is
characterized by a \emph{Universal Property}:

\emph{If $g : X \rightarrow Z$ is a Continuous Map such that $a ~ b$ Implies $g(a)
  = g(b)$ for all $a$ and $b$ in $X$, then there exists a unique Continuous Map
  $f : X/~ \rightarrow Z$ such that $g = f \circ q$ and $g$ is said to
  ``descend'' to the Quotient.}



\subsubsection{Quotient Map}\label{sec:quotient_map}

For a Continuous Surjection $q : X \rightarrow Y$, two sufficient criterea for
$q$ to be a Quotient Map are that $q$ is \emph{Open} (\S\ref{sec:open_map}) or
that $q$ is \emph{Closed} (\S\ref{sec:closed_map}), but they are not
\emph{necessary} criterea.



\subsubsection{Mapping Cylinder}\label{sec:mapping_cylinder}

$f : X \rightarrow Y$

\emph{Mapping Cylinder} $M_f = (X \times I) \sqcup Y$



\subsubsection{Orbit Space}\label{sec:orbit_space}

An \emph{Orbit Space} (or \emph{Space of Covariants}) is the Set of all
\emph{Orbits} (\S\ref{sec:orbit}) of a Set $X$ under the Action of $G$, denoted
$X / G$, also called the \emph{Quotient} of the Action

the $n$th Un-ordered Configuration Space (\S\ref{sec:configuration_space}) is
the Orbit Space of the Group Action of the Symmetric Group
(\S\ref{sec:symmetric_group}) $S_n$ on the Points of $Conf_n(X)$



\subsubsection{Lens Space}\label{sec:lens_space}



% ------------------------------------------------------------------------------
\subsection{Polish Space}\label{sec:polish_space}
% ------------------------------------------------------------------------------

(wiki):

a Separable Completely Metrizable Topological Space, i.e. a Space Homeomorphic
to a Complete Metric Space (\S\ref{sec:complete_metric_space}) that has a
Countable Dense Subset (i.e. Separable \S\ref{sec:separable_space})

\fist Descriptive Set Theory (\S\ref{sec:descriptive_set_theory})

examples:
\begin{itemize}
  \item the Real Line (\S\ref{sec:real_line})
  \item any Separable Banach Space (\S\ref{sec:banach_space})
  \item Cantor Space (\S\ref{sec:cantor_space})
  \item Baire Space (\S\ref{sec:baire_space})
\end{itemize}

between any two Uncountable Polish Spaces there is a Borel Isomorphism
(Bijection preserving Borel Structure), and every Uncountable Polish Space has
the Cardinality of the Continuum

generalizations (TODO xrefs):
\begin{itemize}
  \item Lusin Spaces
  \item Suslin Spaces
  \item Radon Spaces
\end{itemize}



\subsubsection{Standard Borel Space}\label{sec:standard_borel_space}

the \emph{Borel Space} (\S\ref{sec:measurable_space}), i.e. Set of all Borel
Sets (\S\ref{sec:borel_set}), of a Polish Space

a Measurable Space (Borel Space) $(X, \Sigma)$ is ``Standard Borel'' if there
exists a Metric on $X$ which makes it a Complete Separable Metric Space such
that $\Sigma$ is a Borel $\sigma$-algebra (\S\ref{sec:borel_algebra})

\textbf{Kuratowski's Theorem} \emph{
  A Polish Space $X$ as a Borel Space is Borel Isomorphic to one of:
  \begin{enumerate}
    \item $\reals$
    \item $\ints$
    \item a Finite Space
  \end{enumerate}
}

it follows that a Standard Borel Space is characterized up to Isomorphism by its
Cardinality, and any Uncountable Standard Borel Space has Cardinality of the
Continuum

a Measurable Function between Borel Spaces is called a \emph{Borel Function}



\subsubsection{Cantor Space}\label{sec:cantor_space}

\emph{Cantor Space}, $2^{\omega}$, is the Set of all Infinite
Sequences of $0$s and $1$s.



\subsubsection{Baire Space}\label{sec:baire_space}

Topological Space such that every Intersection of a Countable Collection of Open
Dense Sets is also Dense

e.g. Locally Compact Hausdorff Spaces, Complete Metric Spaces

every Banach Space(\S\ref{sec:banach_space}) is a Baire Space

$\omega^{\omega}$ or $\mathcal{N}$ -- the Set of all Infinite Sequences of
Natural Numbers

\emph{Baire Category Theorem}



% ------------------------------------------------------------------------------
\subsection{Configuration Space}\label{sec:configuration_space}
% ------------------------------------------------------------------------------

\emph{Fadell's Configuration Space} -- Space of pairwise distinct Points

for Topological Space $X$, the \emph{$n$th (Ordered) Configuration Space} of
$X$, $Conf_n(X)$ is the Set of $n$-tuples of pairwise distinct Points in $X$:
\[
  Conf_n(X) := \prod^n X - \{(x_1, x_2, \ldots, x_n) \in X^n
    \| x_i = x_j \text{ for some } i \neq j \}
\]
generally given the Subspace Topology from the Inclusion of $Conf_n(X)$ into
$X^n$

\emph{$n$th Un-ordered Configuration Space}:
\[
  UConf_n(X) := Conf_n(X)/S_n
\]
is the Orbit Space (\S\ref{sec:orbit_space}) of the Group Action of the
Symmetric Group (\S\ref{sec:symmetric_group}) $S_n$ on the Points of $Conf_n(X)$

cf. \emph{State Space} in Physics used to describe the \emph{State} of a
\emph{System} as a single \emph{Point} in a High-dimensional Space: a
\emph{Configuration Space} equivalent to the particular case of several
non-colliding particles

(FIXME: what about non-colliding rigid bodies ???)

the Configuration Space (\S\ref{sec:configuration_space}) of Distinct,
Un-ordered Points of a Manifold is also a Manifold and the Configuration Space
of \emph{not necessarily distinct} Unordered Points is an Orbifold
(\S\ref{sec:orbifold})

a Configuration Space is a type of Classifying Space
(\S\ref{sec:classifying_space}) or (Fine) Moduli Space
(\S\ref{sec:moduli_space})

the Homotopy Type (\S\ref{sec:homotopy_type}) of Configuration Spaces is not
Homotopy Invariant, i.e. for any two distinct values of $m$, the Spaces
$Conf_n(\reals^m)$ are not Homotopy Equivalent-- $Conf_n(\reals)$ is not
Connected, $Conf_n(\reals^2)$ is an Eilenberg-Maclane Space
(\S\ref{sec:eilenberg_maclane_space}) of type $K(\pi,1)$ and $Conf_n(\reals^m)$
is \emph{Simply Connected} for $3 \leq m$

Homotopy Invariance for Configuration Spaces of Simply Connected Closed
Manifolds remains an open question in general

for any Graph $\Gamma$, $Conf_n(\Gamma)$ is an Eilenberg-Maclane Space of Type
$K(\pi,1)$ and Strong Deformation Retracts into a Subspace of Dimension
$b(\Gamma)$ where $b(\Gamma)$ is the number of Vertices of Degree at least 3,
and $Uconf_n(\Gamma)$ and $Conf_n(\Gamma)$ Deformation Retract to
Non-positively Curved Cubical Compexes of Dimension at most $min(n,b(\gamma))$



\subsubsection{Ran Space}\label{sec:ran_space}

Space of all Un-ordered Configuration Spaces



% ------------------------------------------------------------------------------
\subsection{Compactly Generated Space}\label{sec:compactly_generated}
% ------------------------------------------------------------------------------

(or \emph{$k$-space}, ``kompakt'')

$\cat{CGTop}$

$\cat{CGHaus}$



% ------------------------------------------------------------------------------
\subsection{Topologist's Sine Curve}\label{sec:topologists_sine}
% ------------------------------------------------------------------------------

% ------------------------------------------------------------------------------
\subsection{Topological Invariant}\label{sec:topological_invariant}
% ------------------------------------------------------------------------------

a Property of a Topological Space preserved under Homeomorphisms
(Topological Isomorphism \S\ref{sec:homeomorphism}), i.e. Contiuous Functions
between Topological Spaces that have Continuous Inverses



% ------------------------------------------------------------------------------
\subsection{Cover}\label{sec:topological_cover}
% ------------------------------------------------------------------------------

A \emph{Cover} (or \emph{Covering}) of a Topological Space $(X, \tau)$
is an Indexed Family of Sets $C = \{ U_i : i \in I \}$ such that their
Union contains $X$:
\[
  X \subseteq \bigcup_{i \in I} U_i
\]
A Cover can also be defined for an arbitrary Subset of $X$, $Y
\subseteq X$:
\[
  Y \subseteq \bigcup_{i \in I} U_i
\]

FIXME: this section requires some cleanup to clearly define ``covers''
vs ``covering spaces''

Universal Covers (\S\ref{sec:universal_cover}) : Homotopy Theory ::
Maximal Abelian Covers (\S\ref{sec:maximal_abelian_cover}) : Homology
Theory

\begin{itemize}
  \item a Null Set (\S\ref{sec:null_set}) $N \subset \reals$ is a Set that can
    be Covered by a Countable Union of Intervals of arbitrarily small total
    length
  \item ...
\end{itemize}



\subsubsection{Covering Map}\label{sec:covering_map}

(\emph{Projection})

Covering Space $C$

Base Space $X$

Continuous (\S\ref{sec:continuous_map}) Surjective Map $p : C
\rightarrow X$ such that for every $x \in X$ there is an Open
Neighborhood (\S\ref{sec:open_neighborhood}) $U$ of $x$ with
$p^{-1}(U)$ a Disjoint Union of Open Sets in $C$, each Homeomorphic to
$U$ by $p$.

Trivial Covering



\paragraph{Fiber}\label{sec:point_fiber}\hfill

Inverse Image of Point $x \in X$

necessarily a Discrete Space



\paragraph{Monodromy}\label{sec:monodromy}\hfill

\fist the associated Holonomy (\S\ref{sec:holonomy}) for Flat Connections
(\S\ref{sec:flat_connection})



\subsubsection{Deck Transformation}\label{sec:deck_transformation}

A \emph{Deck Transformation} (or \emph{Covering Transformation} or
\emph{Automorphism}) of a Cover $p : C \rightarrow X$ is a
Homeomorphism $f : C \rightarrow C$ such that $p \circ f = p$

the Set of all Deck Transformations of $p$ forms a Group under
Composition, the \emph{Deck Transformation Group} $Aut(p)$



\subsubsection{Open Cover}\label{sec:open_cover}

\paragraph{Nerve}\label{sec:covering_nerve}\hfill

(wiki): of an Open Covering is a construction of an Abstract Simplicial Complex
from an Open Covering of a Topological Space

\emph{Nerve Theorem}

cf. Nerve (Quasicategories \S\ref{sec:nerve})



\subsubsection{Abelian Cover}\label{sec:abelian_cover}

A Cover is an \emph{Abelian Cover} if its Group of Deck
Transformations is Abelian



\paragraph{Maximal Abelian Cover}\label{sec:maximal_abelian_cover}\hfill

$\overline{X}$


\subsubsection{Covering Space}\label{sec:covering_space}

(\emph{Total Space})



\subsubsection{Universal Cover}\label{sec:universal_cover}\hfill

\emph{Universal Cover} $\tilde{X}$ is a Covering Space of $X$ that
Covers all other Covering Spaces of $X$

$\reals$ is the Universal Cover of the Unit Circle $S^1$




\subsubsection{Refinement}\label{sec:refinement}



% ------------------------------------------------------------------------------
\subsection{Base}\label{sec:topological_base}
% ------------------------------------------------------------------------------

A \emph{Base}, $B$, is a Subset of a Topology, $\tau$, in a Metric
Space, $(M,\tau)$, such that:
\[
  \forall U \in \tau, \exists \{B_i\}_{i \in I} \subseteq B :
  \bigcup_{i \in I}B_i = U
\]
Properties:
\begin{enumerate}
  \item $B$ is a Covering (\S\ref{sec:topological_cover}) of $M$, as
    stated by:
\[
  M \subseteq \bigcup_{i \in I} B_i
\]

  \item
\[
  \forall B_1, B_2 \in B, \forall x \in B_1 \cap B_2,
  \exists B_3 \in B : x \in B_3 \wedge B_3 \subseteq B_1 \cap B_2
\]

\end{enumerate}
An example of a Base is the Set of all Open Balls in a Metric Space.

A Base is not necessarily Unique for a given Topology. Adding Elements
to a Base results in another Base.



\subsubsection{Subbase}\label{sec:subbase}

A \emph{Subbase}, $S$, is a Subset of a Topology, $\tau$, in a Metric
Space, $(M,\tau)$, such that the Set:
\[
  S \subseteq \tau : \{ \bigcap_{j \in J} S_j : |J| < \infty \}
\]
is a Base for $\tau$.

There is no unique Subbase for a given Topology but there is a unique
Topology for a given Subbase.



% ------------------------------------------------------------------------------
\subsection{Subspace Topology}\label{sec:subspace_topology}
% ------------------------------------------------------------------------------

Topological Space $(X,\tau)$, Subset $S \subseteq X$, \emph{Subspace
  Topology} (also \emph{Relative Topology}, \emph{Induced Topology},
or \emph{Trace Topology}) is defined as:
\[
  \tau_S = \{ S \cap U | U \in \tau \}
\]
A Subset of $S$ is Open in $\tau_S$ if and only if it is the
Intersection of $S$ with an element of $\tau$.

A Subspace Topology may alternatively be given as a \emph{Topological
  Embedding} (\S\ref{sec:topological_embedding}).

An \emph{Embedded Submanifold} (\S\ref{sec:embedded_submanifold}) is an
Immersed Submanifold (\S\ref{sec:immersed_submanifold}) for which the Image of
the Inclusion Map has the Subspace Topology as the Submanifold Topology.

Dual to Quotient Topology (\S\ref{sec:quotient_topology})

\emph{Open Subspace}

\emph{Closed Subspace}

special case of Initial Topology (\S\ref{sec:initial_topology})

\fist Subspace (\S\ref{sec:subspace})

\fist Linear Subspace (\S\ref{sec:linear_subspace})



% ------------------------------------------------------------------------------
\subsection{Quotient Topology}\label{sec:quotient_topology}
% ------------------------------------------------------------------------------

Quotient Space (\S\ref{sec:quotient_space})

Final Topology on the Quotient Space with respect to map $q : X
\rightarrow X / \sim$



% ------------------------------------------------------------------------------
\subsection{Vietoris Topology}\label{sec:vietoris_topology}
% ------------------------------------------------------------------------------

\fist Power Domains (\S\ref{sec:power_domain})



% ------------------------------------------------------------------------------
\subsection{Fell Topology}\label{sec:fell_topology}
% ------------------------------------------------------------------------------

% ------------------------------------------------------------------------------
\subsection{Countability Axioms}\label{sec:countability_axioms}
% ------------------------------------------------------------------------------

A Topological Space, $(M,\tau)$, is \emph{First Countable} or $1c$ if
$\forall X \in M$, there exists a Countable Neighborhood Base
(\S\ref{sec:neighborhood_base}). First Countable Topological Spaces is
a narrower Class of Topological Spaces where Functions Preserve Limits
of Sequences. %FIXME

And the Topological Space is \emph{Second Countable} or $2c$ if $\tau$
has a Countable Base.
\[
  2c \rightarrow 1c
\]

All Metric Spaces are $1c$.



% ------------------------------------------------------------------------------
\subsection{Compact Space}\label{sec:compact_space}
% ------------------------------------------------------------------------------

\emph{Compactness}

\begin{enumerate}
\item \emph{Closed} (\S\ref{sec:closed_set}): contains all its Limit Points
\item \emph{Bounded} (\S\ref{sec:bounded_subset}): all Points lie within a Ball
  of Finite Radius
\end{enumerate}

generalizes the notion of a Subset of a Euclidean Space being Closed
(i.e. containing all its Limit Points) and Bounded (i.e. contained in
a Finite Interval)

``Topologically Finite''

a Finite Space (\S\ref{sec:finite_space}) is Trivially Compact

\fist a Closed Manifold (\S\ref{sec:closed_manifold}) is a Compact Manifold
without a Boundary and if no Boundary is possible then any Compact Manifold
is a Closed Manifold (FIXME: clarify)

Lebesgue Covering Theorem (TODO) -- Simplicial Approximation Theorem (TODO:
xref)



\subsubsection{Finite Space}\label{sec:finite_space}

Trivially Compact

\begin{itemize}
  \item Sierpinski Space (\S\ref{sec:sierpinski_space})
\end{itemize}



\subsubsection{Locally Compact Space}\label{sec:locally_compact}

a Topological Space $X$ with the Local Property (\S\ref{sec:local_property})
that that every Point $x$ of $X$ has a Compact Neighborhood, i.e. there exists
an Open Set $U$ and Compact Set $K$ such that $x \in U \subseteq K$

every Compact Hausdorff Space (\S\ref{sec:separated_space}) is also Locally
Compact



\subsubsection{Sequentially Compact}\label{sec:sequentially_compact}

\subsubsection{Tychonoff's Theorem}\label{sec:tychonoffs_theorem}

Compactness Theorem (Model Theory \S\ref{sec:compactness})



\subsection{Pre-compact Subspace}\label{sec:precompact}

or \emph{Relatively Compact Subspace} is a Subset of a Topological Space with a
Closure (\S\ref{sec:topological_closure}) that is Compact

\begin{itemize}
  \item Normal Family (\S\ref{sec:normal_family}) -- a Pre-compact Subset of the
    Space of Continuous Functions (\S\ref{sec:continuous_function})
\end{itemize}



% ------------------------------------------------------------------------------
\subsection{Cocompact Space}\label{sec:cocompact_space}
% ------------------------------------------------------------------------------

Wallpaper Group (\S\ref{sec:wallpaper_group})

Crystallographic Groups (\S\ref{sec:crystallographic_group})



% ------------------------------------------------------------------------------
\subsection{Betti Number}\label{sec:betti_number}
% ------------------------------------------------------------------------------

Euler Characteristic (\S\ref{sec:euler_characteristic})



% ------------------------------------------------------------------------------
\subsection{Fundamental Domain}\label{sec:fundamental_domain}
% ------------------------------------------------------------------------------

for Group Action (\S\ref{sec:group_action}) of Group $G$ on a Topological Space
$X$ by Homeomorphisms, a \emph{Fundamental Domain} for the Action is a Set $D$
of representatives for the Orbits (\S\ref{sec:orbit})

Uniform Tilings (\S\ref{sec:uniform_tiling})



% ------------------------------------------------------------------------------
\subsection{Suslin Property}\label{sec:suslin_property}
% ------------------------------------------------------------------------------

Property that every collection of Non-empty Disjoint Open Sets of a Topological
Space is at most Countable



% ==============================================================================
\section{Point-set Topology}\label{sec:pointset_topology}
% ==============================================================================

or \emph{General Topology}

A \emph{Topology}, $\tau$, is a collection of Subsets called
\emph{Open Sets} (\S\ref{sec:open_set}) of a \emph{Metric Space}
(\S\ref{sec:metric_space}), $M$, subject to the following Inductive
definition:
\begin{enumerate}
\item $\varnothing \in \tau, M \in \tau$
\item $U,V \in \tau \rightarrow U \cap V \in \tau$
\item $\{U_i\}_{i \in I} \subseteq \tau \rightarrow \bigcup_{i \in I}
  U_i \in \tau$
\end{enumerate}

An equivalent definition is possible in terms of \emph{Closed Sets}
(\S\ref{sec:closed_set}).

\fist cf. Pointless Topology (\S\ref{sec:pointless_topology})



% ------------------------------------------------------------------------------
\subsection{Point}\label{sec:topological_point}
% ------------------------------------------------------------------------------

cf. \emph{Point} (Geometric Primitive \S\ref{sec:point})

Two Points that are within the same Open Set (they have exactly the
same Neighborhoods) of a Topology are said to be \emph{Topologically
  Indistinguishable} (\S\ref{sec:topologically_distinguishable}).



\subsubsection{Isolated Point}\label{sec:isolated_point}

\subsubsection{Topologically Distinguishable}
\label{sec:topologically_distinguishable}

Two Points of a Topological Space $X$ are \emph{Topologically
  Indistinguishable} if they have exactly the same Neighborhoods
(\S\ref{sec:neighborhood}).

\fist Distinguishable Space ($T_0$ \S\ref{sec:distinguishable_space})



% ------------------------------------------------------------------------------
\subsection{Coarseness}\label{sec:coarseness}
% ------------------------------------------------------------------------------

Given two Topologies, $\tau_1 \subset \tau_2$, $\tau_1$ is
\emph{Coarse} relative to $\tau_2$, and $\tau_2$ is \emph{Fine}
relative to $\tau_1$.

Coarser Topologies will have more Topologically Indistinguishable
Points.

The Coarsest Topology possible is the Trivial Topology
(\S\ref{sec:trivial_topology}) $\{ \varnothing, X \}$.

The Coarsest Topology that contains a given collection of Open Sets $S
\subseteq \pow(X)$ is the Topology Generated by taking $S$ to
be a Subbase (\S\ref{sec:subbase}) of $X$.

The Finest Topology possible on a Set is the Discrete Topology
(\S\ref{sec:discrete_topology})-- all Subsets are Open Sets, in particular each
Singleton is an Open Set



\subsubsection{Initial Topology}\label{sec:initial_topology}

The \emph{Initial Topology} for a Set $X$ with respect to a Familiy of
Functions on $X$ is the \emph{Coarsest} Topology on $X$ that makes those
Functions Continuous (\S\ref{sec:continuous_function})

(FIXME: continuous maps ???)

Subspace Topology (\S\ref{sec:subspace_topology})

Product Topology (\S\ref{sec:product_topology})



\paragraph{Weak Topology}\label{sec:weak_topology}

(FIXME: synonym for initial topology ???)



\subsubsection{Final Topology}\label{sec:final_topology}

(also \emph{Strong Topology}, \emph{Colimit Topology}, or
\emph{Inductive Topology})

The \emph{Final Topology} for a Set $X$ with respect to a Familiy of
Functions on $X$ is the \emph{Finest} Topology on $X$ that makes those
Functions Continuous (\S\ref{sec:continuous_function})

(FIXME: continuous maps ???)



% ------------------------------------------------------------------------------
\subsection{Neighborhood}\label{sec:neighborhood}
% ------------------------------------------------------------------------------

A \emph{Neighborhood} of a Point $x$ in a Topological Space $(M,\tau)$
is a Subset $V \subseteq M$ such that:
\[
  \exists U \in \tau : U \subseteq V \wedge x \in U
\]
That is, $x$ is in the Interior of $V$ and $V$ is not necessarily
itself Open (see Open Neighborhoods \S\ref{sec:open_neighborhood}).

Every Subset of the Discrete Topology is a Neighborhood.

For the Trivial Topology, the only Neighborhood is the entire Space.



\subsubsection{Neighborhood System}\label{sec:neighborhood_system}

Given a Point $x$ in any Topological Space, a \emph{Neighbordhood
  System} (or \emph{Neighborhood Filter}), $\mathcal{V}(x)$, is the
Set of all Neighborhoods of $x$.



\subsubsection{Neighborhood Base}\label{sec:neighborhood_base}

Given a Neighborhood System, $\mathcal{V}(x)$, a \emph{Neighborhood
  Base} for $x$ is defined as a Subset of the Neighborhood System,
$\beta(x) \subseteq \mathcal{V}(x)$, such that:
\[
  \forall v \in V(x), \exists b \in \beta(x) : b \subseteq v
\]
For a Neighborhood Base $\beta(x) = \{ U \in B : x \in U \}$ where $B
\subseteq \pow(X)$ in Topological Space $(X,\tau)$, $B$ is a
Base for $\tau$ if and only if $\beta(x)$ is a Neighborhood Base for
all $x \in X$.



\subsubsection{Open Neighborhood}\label{sec:open_neighborhood}

Open Neighborhoods in Differentiable Manifolds
(\S\ref{sec:differentiable_manifold}) are Vector Spaces of Differentials
(\S\ref{sec:differential})



\subsubsection{Uniform Neighborhood}\label{sec:uniform_neighborhood}

\paragraph{$r$-Neighborhood}\label{sec:r_neighborhood}\hfill

An \emph{$r$-neighborhood} $S_r$ for $r > 0$ of a Set $S$ is the Set of all
Points in a Topological Space $X$ that are less than distance $r$ from $S$, or
equivalently the Union of all Open Balls of Radius $r$ centered at Points of
$S$:
\[
  S_r = \bigcup_{p \in S} B_r (p)
\]
It follows that an $r$-neighborhood is a Uniform Neighborhood.

Any $r$-Neighborhood of $0 \in \reals$ such that $r \leq 1$ is Closed under
Multiplication.



\paragraph{Epsilon Neighborhood}\label{sec:epsilon_neighborhood}\hfill

An \emph{Epsilon Neighborhood} $N_\epsilon(p)$ of a Point $p \in \reals$ is an
$r$-neighborhood with $r = \epsilon << 1$ arbitrarily small.

$N_\epsilon(0)$ is a Commutative Semigroup (\S\ref{sec:commutative_semigroup})
under Multiplication



\subsubsection{Limit Point}\label{sec:limit_point}

(wiki):

A \emph{Limit Point} of a Set $S$ in a Topological Space $X$ is a Point $x$
that can be ``approximated'' by Points of $S$ in the sense that every
Neighborhood of $x$ with respect to the Topology on $X$ \emph{also} contains a
Point of $S$ \emph{other} than $x$ itself. Note that a Limit Point of the Set
$S$ does not have to be an Element of $S$.

A Set is Closed (\S\ref{sec:closed_set}) if and only if it contains all its
Limit Points.

Limit Point of $D \in \reals$ is $l \in D$ such that:
\[
  \exists a_n \in D : a_n \neq l \wedge \lim a_n = l
\]

Limit Points of $(0,1)$ are $[0,1]$

A Discrete Set has no Limit Points. %FIXME



\subsubsection{Closure}\label{sec:topological_closure}

\emph{Topological Closure}

(wiki):

The \emph{Closure} of a Subset $S$ of Points in a Topological Space is the Set
of all Points in $S$ together with all Limit Points of $S$. Equivalently, the
Closure of $S$ is the Union of $S$ and its \emph{Boundary}
(\S\ref{sec:boundary}); the Points in the Closure that are not part of the
Boundary are the \emph{Interior} Points of $S$

Kuratowski Closure

Closure Operator

\begin{itemize}
  \item a \emph{Pre-compact Subspace} (\S\ref{sec:precompact}) is a Subset of a
    Topological Space with a Closure that is Compact
\end{itemize}



\subsubsection{Derived Set}\label{sec:derived_set}

\subsubsection{Local Property}\label{sec:local_property}

(wiki):

a Property (\S\ref{sec:property}) of a Topological Space is said to be exhibited
\emph{Locally} in one of the two different senses:
\begin{itemize}
  \item (Weaker) each Point of the Topological Space has a Neighborhood
    exhibiting the Property
\end{itemize}
or:
\begin{itemize}
  \item (Stronger) each Point of the Topological Space has a Neighborhood Base
    of Sets exhibiting the Property
\end{itemize}

examples:
\begin{itemize}
  \item Locally Invertible Function (\S\ref{sec:locally_invertible})
  \item Locally Compact Spaces (\S\ref{sec:locally_compact})
  \item Locally Connected Space (\S\ref{sec:locally_connected})
  \item Locally Flat (\S\ref{sec:locally_flat})
  \item ... MORE
\end{itemize}



% ------------------------------------------------------------------------------
\subsection{Interior}\label{sec:interior}
% ------------------------------------------------------------------------------

For a Subset, $V$, of a Topology, $\tau$, a Point $x \in V$ is in the
\emph{Interior} of $V$, $V^{\circ}$, if there is a \emph{Neighborhood}
(\S\ref{sec:neighborhood}) of $x$, $N \subset V$. Inductively, for
$\{A_i\}_{i \in I} \subseteq \tau \wedge \forall i, A_i \subseteq V$:
\[
  V^{\circ} = \bigcup_{i \in I} A_i \subseteq \tau
\]
Equivalently, the Interior of $V$, $V^o$, is every Open Set within $V$
and is itself an Open Set:
\[
  V^o = \bigcup_{i \in I} \{ A_i : A_i \subseteq \tau \wedge A_i
  \subseteq V \}
\]

\fist the \emph{Boundary} (\S\ref{sec:boundary}) of a Subset $S$ of a
Topological Space $X$ is the Set of Points in the \emph{Closure}
(\S\ref{sec:topological_closure}) of $S$ which do not belong to the
Interior of $S$; equivalently it is the Set of Points which can be
``approached'' from both $S$ and from ``outside'' $S$

\fist a Hyperconnected Space (\S\ref{sec:hyperconnected_space}) is a
Topological Space such that the Interior of every Proper
Closed Set is Empty



\subsubsection{Interior Operator}\label{sec:interior_operator}

The Closure Operator is the Dual of the Interior Operator.



% ------------------------------------------------------------------------------
\subsection{Boundary}\label{sec:boundary}
% ------------------------------------------------------------------------------

(wiki):

\emph{Boundary of a Subset} $S$ of a Topological Space $X$ is the Set of Points
which can be ``approached'' from both $S$ and from ``outside'' $S$; equivalently
it is the Set of Points in the \emph{Closure} (\S\ref{sec:topological_closure})
of $S$ which do not belong to the \emph{Interior} (\S\ref{sec:interior}) of $S$

\emph{Boundary of an $n$-manifold} $M$ is an $(n-1)$-manifold $\partial M$
without Boundary; a Manifold with Boundary has every Interior Point a
Neighborhood Homeomorphic to the Open $n$-ball, and every Boundary Point has a
Neighborhood Homeomorphic to the ``Half'' $n$-ball

\fist a Closed Manifold (\S\ref{sec:closed_manifold}) is a Manifold that is
Compact (\S\ref{sec:compact_space}) and \emph{without} a Boundary

\emph{Boundary of a Boundary}

\fist the Derivative (\S\ref{sec:derivative}) is the ``opposite'' of ``the''
Boundary --\url{https://www.youtube.com/watch?v=2ptFnIj71SM}



% ------------------------------------------------------------------------------
\subsection{Density}\label{sec:density}
% ------------------------------------------------------------------------------

Dense Subset

minimum Cardinality of a Dense Subset

Real Numbers with usual Topology have the Rational Numbers as a Countable Dense
Subset and the Irrational Numbers as an Uncountably Dense Subset

\fist a Hyperconnected Space (\S\ref{sec:hyperconnected_space}) is a
Topological Space such that every (Nonempty) Open Set is Dense in $X$

\fist cf. \emph{Polytope Density} (\S\ref{sec:polytope_density}) --
generalization of ``Winding Number'' (TODO: xref) from two to higher Dimensions



% ------------------------------------------------------------------------------
\subsection{Map}\label{sec:topology_map}
% ------------------------------------------------------------------------------

A \emph{Map} is a Function between Topological Spaces.



\subsubsection{Germ}\label{sec:germ}

\subsubsection{Operator}\label{sec:space_operator}

a Mapping from a Space to itself

cf. Operator (Logic \S\ref{sec:operator})

\begin{itemize}
  \item Linear Operator (\S\ref{sec:linear_operator}) -- a Module Operator that
    preserves Addition and Scalar Multiplication
\end{itemize}



% ------------------------------------------------------------------------------
\subsection{Open \& Closed Map}\label{sec:open_closed_map}
% ------------------------------------------------------------------------------

\emph{Open Map} is a Function between Topological Spaces mapping Open
Sets to Open Sets.

\emph{Closed Map} is a Function between Topological Spaces mapping Closed
Sets to Closed Sets.

Every Homeomorphism (\S\ref{sec:homeomorphism}) is an Open Map, a
Closed Map, and a Continuous Function.



% ------------------------------------------------------------------------------
\subsection{Continuous Map}\label{sec:continuous_map}
% ------------------------------------------------------------------------------

A \emph{Continuous Map} $f : X \rightarrow Y$ between Topological Spaces
$(X,\tau_1)$ and $(Y,\tau_2)$ is given by:
\[
  \forall V \in \tau_2, f^{-1}(V) \in \tau_1
\]

\fist cf. Continuous Functions (\S\ref{sec:continuous_function}) between Metric
Spaces

\begin{enumerate}
  \item if $f$ is a Surjection, then $f$ is a Quotient Map
    (\S\ref{sec:quotient_map})
  \item if $f$ is a Injection, then $f$ is a Topological Embedding
    (\S\ref{sec:topological_embedding})
  \item if $f$ is a Bijection, then $f$ is a Homeomorphism
    (\S\ref{sec:homeomorphism})
\end{enumerate}

\begin{enumerate}
  \item Any Constant Function is Continuous
  \item Given two Continuous Functions, $f : X \rightarrow Y$ and $g
    : Y \rightarrow Z$, the Function $g \circ f : X \rightarrow Z$ is
    Continuous
  \item Given $f : (X, \tau) \rightarrow (Y, \sigma)$, $f$ is
    Continuous if $\tau = \pow(X)$ (Discrete Topology) or
    $\sigma = \{\varnothing, Y\}$ (Trivial Topology)
\end{enumerate}

Net Continuity (\S\ref{sec:net_continuity})

Given Topological Spaces $(X, \tau_X)$, $(Y, \tau_Y)$, and $(Z,
\tau_Z)$, where $f: X \rightarrow Y$ and $g: Y \rightarrow Z$ are
Continuous Maps, then the Composition $g \circ f : X \rightarrow Z$ is
Continuous.

any Function from a Discrete Space (\S\ref{sec:discrete_space}) to another
Topological Space is Continuous

A Group with Topology $\tau$ such that the Group \emph{Binary Operation} and
\emph{Inverse Function} are Continuous Maps with respect to $\tau$, is a
\emph{Topological Group} (\S\ref{sec:topological_group}), and any Group can be
considered a Topological Group by giving it the Discrete Topology (a Discrete
Group \S\ref{sec:discrete_group})

``all Computable Functions are Continuous''--
(\url{http://math.andrej.com/2006/03/27/sometimes-all-functions-are-continuous/})



\subsubsection{Topological Embedding}\label{sec:topological_embedding}

Topological Space $(X,\tau)$, Subset $S \subseteq X$

A Subspace Topology (\S\ref{sec:subspace_topology}) is the Coarsest
(\S\ref{sec:coarseness}) Topology such that the Inclusion Map
(\S\ref{sec:inclusion_map}) $\iota : S \hookrightarrow X$ is
Continuous (\S\ref{sec:continuous_map}). $iota$ is then called a
\emph{Topological Embedding} and $S$ is Homeomorphic
(\S\ref{sec:homeomorphism}) to its Image in $X$.

An \emph{Embedded Submanifold} (\S\ref{sec:embedded_submanifold}) is an
Immersed Submanifold (\S\ref{sec:immersed_submanifold}) for which the Inclusion
Map is a Topological Embedding.



\subsubsection{Homeomorphism}\label{sec:homeomorphism}

\emph{Homeomorphism} (\emph{Topological Isomorphism} or \emph{Bi-continuous
  Function}) between two Topological Spaces $(X, \tau)$ and $(Y, \sigma)$ is is
a Continuous Map $f : X \rightarrow Y$ that has a Continuous Inverse

a Homeomorphism preserves Topological Invariants
(\S\ref{sec:topological_invariant})

2014 - Olah - \emph{Neural Networks, Manifolds, and Topology} -
\url{http://colah.github.io/posts/2014-03-NN-Manifolds-Topology/} -- tanh Layers
in a Deep Neural Network with $N \times N$ Non-singular Weight Matrices are
Homeomorphisms

Homeomorphism Group (FIXME)

a Homeomorphism $\varphi$ from an Open Subset $U$ of a Topolgoical Space $M$ to
an Open Subset of Euclidean Space is called a \emph{Chart} (\S\ref{sec:chart})



\paragraph{Invariance of Domain}\label{sec:domain_invariance}\hfill

or Invariance of Dimension

as a consequence, $\reals^n$ cannot be Homeomorphic to $\reals^m$ if $m \neq n$,
and no Non-empty Open Subset of $\reals^n$ can be Homeomorphic to any Open
Subset of $\reals^m$



% ------------------------------------------------------------------------------
\subsection{Product Space}\label{sec:product_space}
% ------------------------------------------------------------------------------

Cartesian Product of a Family of Topological Spaces equipped with a natural
Product Topology

cf. Phase Space (\S\ref{sec:phase_space})

starting witht he Standard Topology on $\reals$ and defining a Product Topology
on $n$ copies of $\reals$ gives the ordinary Euclidean Topology on $\reals^n$

the Cantor Set is Homeomorphic to the Product of Countably many copies of the
Discrete Space $\{0,1\}$

the Space of Irrational Numbers is Homeomorphic to the Product of countably many
copies of the Natural Numbers, each with the Discrete Topology



\subsubsection{Product Topology}\label{sec:product_topology}

natural Topology of a Product Space

special case of Initial Topology (\S\ref{sec:initial_topology})

Pointwise Convergence (\S\ref{sec:pointwise_convergence}): Convergence in the
Product Topology on teh Space $Y^X$ where $X$ is the Domain and $Y$ is the
Codomain



\subsubsection{Box Topology}\label{sec:box_topology}

alternative Topology of a Product Space that agrees with the Product Topology
when the Product is over a Finite number of Spaces



\subsubsection{Cylinder Set}\label{sec:cylinder_set}



% ------------------------------------------------------------------------------
\subsection{Disjoint Union Topology}\label{sec:disjoint_union_topology}
% ------------------------------------------------------------------------------

(or \emph{Topological Sum})

Coproduct (\S\ref{sec:coproduct})

Dual to Product Topology (\S\ref{sec:product_topology})



% ------------------------------------------------------------------------------
\subsection{Sequence}\label{sec:sequence_topology}
% ------------------------------------------------------------------------------

A \emph{Sequence} is a Net (\S\ref{sec:net}) where the Directed Set is
the Natural Numbers:
\[
  (x_n) : \mathbb{N} \rightarrow (X,\tau)
\]

cf. Sequences (Countable Totally Ordered Multisets \S\ref{sec:sequence})



\subsubsection{Convergence}\label{sec:convergence}

A Sequence $(x_n) : \mathbb{N} \rightarrow X$ \emph{Converges} in a Metric
Space $(X,d)$, if:
\[
  \lim_{n \rightarrow \infty} d (x_n, x) = 0
\]

For a Topological Space, $(X,\tau)$, the Sequence $(x_n)$ Converges to
$x$ if:
\[
  \forall A \in \mathcal{V}(x), \exists n \in \mathbb{N}
  : m \geq n \Rightarrow x_m \in A
\]
There may possibly be more than one or infinite Limits in a general
Topological Space.

In the Trivial Topology $(X, \{\varnothing, X\})$, all Sequences
Converge to every $x \in X$.



\paragraph{Pointwise Convergence}\label{sec:pointwise_convergence}\hfill

Convergence in the Product Topology on the Space $X^Y$ where $X$ is the Domain
and $Y$ is the Codomain



\paragraph{Uniform Convergence}\label{sec:uniform_convergence}\hfill

Convergence in the Uniform Norm Topology (\S\ref{sec:uniform_norm_topology})

(wiki):

\emph{Weierstrass Approximation Theorem}: every Continuous Function
(\S\ref{sec:continuous_function}) defined on a Closed Interval can be Uniformly
Approximated to arbitrary closeness by a Polynomial Function
(\S\ref{sec:polynomial_function})

the Set of Nowhere-differentiable (\S\ref{sec:nowhere_differentiable})
Real-valued Functions on the Closed Interval $[0,1]$ is Comeagre
(\S\ref{sec:comeagre_set}) in the Vector Space $C([0,1]; \reals)$ of Continuous
Real-valued Functions on $[0,1]$ with the Topology of Uniform Convergence

if a Sequence of Analytic Functions Converges Uniformly in a region $S$ of the
Complex Plane then the Limit is Analytic in $S$-- this demonstrates that the
Complex Functions are more ``well-behaved'' than the Real Functions since the
Uniform Limit of Analytic Functions on a Real Interval do not need to be
Differentiable



\paragraph{Compact Convergence}\label{sec:compact_convergence}\hfill

Uniform Convergence on Compact Sets

cf. Universal Approximation Theorem in Machine Learning

Compact-open Topology (\S\ref{sec:compact_open})



\subsubsection{Residuality}\label{sec:reside}

In a Topological Space $(X, \tau)$, a Sequence is \emph{Residually}
(or \emph{Eventually}) in an arbitrary Subset $Y \subseteq X$ if:
\[
  \exists m \in \mathbb{N} : n \geq m \Rightarrow x_n \in Y
\]
Such a Sequence \emph{Resides} in $Y$.



\subsubsection{Frequentness}\label{sec:frequent}

In a Topological Space $(X, \tau)$, a Sequence is \emph{Frequently}
in an arbitrary Subset $Y \subseteq X$ if there is a Subsequence that
is always in that Set:
\[
  (\exists c : \mathbb{N} \rightarrow \mathbb{N})
  : (\forall n \in \mathbb{N}) x_{c(n)} \in Y
\]



\subsubsection{Accumulation}\label{sec:accumulation}

A Point $x$ in a Topological Space $(X, \tau)$ is an
\emph{Accumulation Point} (or \emph{Cluster Point}) of a Sequence
$(x_n)$ if $(x_n)$ is Frequently in every Neighborhood in the
Neighborhood Space of $x$.



\subsubsection{Subsequence}\label{sec:subsequence_topology}

Given a Sequence $(x_n) : \mathbb{N} \rightarrow X$, a
\emph{Subsequence} is given by the Composition of $(x_n)$ with a
Strictly Increasing Function on $c : \mathbb{N} \rightarrow
\mathbb{N}$:
\[
  (x_n) \circ c :
  \mathbb{N} \xrightarrow{c} \mathbb{N} \xrightarrow{(x_n)} \mathbb{N}
\]



\subsubsection{Exact Sequence}\label{sec:exact_sequence}

\subsubsection{Sequentially Continuous Function}
\label{sec:sequentially_continuous}

A Function $f : X \rightarrow Y$ between Topological Spaces $(X,
\tau)$ and $(Y, \sigma)$ is \emph{Sequentially Continuous} at a Point
$x \in X$ if for any Sequence $(x_n)$ in $X$:
\[
  \lim_{n \rightarrow \infty} x_n = x
  \Rightarrow \lim_{n \rightarrow \infty} f(x_n) = f(x)
\]
and Sequentially Continuous in general if Sequentially Continuous at
every Point $x \in X$.

All Continuous Functions (\S\ref{sec:continuous_function}) are
Sequentially Continuous. If the Domain $X$ is First Countable
(\S\ref{sec:countability_axioms}), then any Sequentially Continuous
Function $f : X \rightarrow Y$ is also a Continuous Function.



% ------------------------------------------------------------------------------
\subsection{Net}\label{sec:net}
% ------------------------------------------------------------------------------

A \emph{Net} is a Function from a Directed Set (\S\ref{sec:directed_set}) into
a Topological Space:
\[
  (x_\alpha) : D \rightarrow (X, \tau)
\]
A Net is \emph{Eventually} in a Set $A \in X$ if:
\[
  \exists \alpha \in D
  : \forall \beta \in D, \beta \geq \alpha \wedge x_\beta \in A
\]

A Net is \emph{Frequently} in a Set $A \in X$ if:
\[
  \forall \alpha \in D, \exists \beta \in D
  : \alpha \leq \beta \wedge (x_\beta) \in A
\]

A Sequence (\S\ref{sec:sequence}) is a Net where the Directed Set is the
Natural Numbers $\mathbb{N}$.

\fist not to be confused with Polytope Nets (\S\ref{sec:polytope_net})



\subsubsection{Net Convergence}\label{sec:net_convergence}

A Net $(x_\alpha)$ \emph{Converges} to a Point $x$ if for any
Neighborhood $U \in \mathcal{V}(x)$ in the Neighborhood System of $x$,
$(x_\alpha)$ is Eventually in $U$.



\subsubsection{Limit}\label{sec:net_limit}

(wiki):

For a Net $(x_\alpha)$ that is a Net from a Directed Set $A$

Examples:

\begin{itemize}
  \item Sequence Limits (\S\ref{sec:sequence_limit})
  \item Function Limits (\S\ref{sec:function_limit})
  \item Limits of Nets of Riemann Sums in Riemann Integrals
    (\S\ref{sec:riemann_integral})-- the Directed Set is the Set of Partitions
    of Interval of Integration, Partially Ordered by Inclusion
\end{itemize}



\subsubsection{Subnet}\label{sec:subnet}

A Net $(y_\beta) : B \rightarrow X$ is a Subnet of a Net $(x_\alpha) :
A \rightarrow X$ if there is a Monotone
(\S\ref{sec:monotonic_function}), Cofinal (\S\ref{sec:cofinal_map})
Total Function:
\[
  h : B \rightarrow A
\]
that is, with the Properties:
\begin{enumerate}
  \item Monotonicity:
  $\beta_1 \leq \beta_2 \Rightarrow h(\beta_1) \leq h(\beta_2)$
  \item Cofinality:
   $\forall \alpha \in A, \exists \beta \in B : \alpha \leq h(\beta)$
\end{enumerate}



\subsubsection{Cluster Point}\label{sec:cluster_point}

A Point $x$ in a Topological Space $(X,\tau)$ is a \emph{Cluster
  Point} of a Net $(x_\alpha)$ if $(x_\alpha)$ is Frequently in every
Neighborhood $U \in \mathcal{V}(x)$ of the Neighborhood System of $x$.
Alternatively, $x$ is a Cluster Point of a Net if there exists some
Subnet that Converges to $x$.



\subsubsection{Continuity}\label{sec:net_continuity}

Given two Topological Spaces $(X,\tau)$ and $(Y,\sigma)$, a Function
$f : X \rightarrow Y$ is \emph{Continuous}
(\S\ref{sec:continuous_map}) at a Point $x \in X$ if any only if for
any Convergent Net $(x_\alpha)$:
\[
  \lim (x_\alpha) = x \Rightarrow \lim f(x_\alpha) = f(x)
\]



% ------------------------------------------------------------------------------
\subsection{Path}\label{sec:path} \cite{hatcher02}
% ------------------------------------------------------------------------------

Continuous Map (\S\ref{sec:continuous_map}) $f : I \rightarrow X$ from
Unit Interval (\S\ref{sec:unit_interval}) $I$ to Space $X$.

Constant Path %FIXME

\fist See also \emph{Path} (Graph Theory \S\ref{sec:graph_path})



\subsubsection{Path Product}\label{sec:path_product}

$f,g : I \rightarrow X$ where $f(1) = g(0)$, \emph{Product Path} (or
\emph{Composition}) $f \cdot g$:
\[
  f \cdot g (s) =
  \begin{cases}
    f(2s)   & \quad 0 \leq s \leq \sfrac{1}{2} \\
    g(2s-1) & \quad \sfrac{1}{2} \leq s \leq 1 \\
  \end{cases}
\]



\subsubsection{Inverse Path}\label{sec:inverse_path}

$f^{-1}(s) = f(1-s)$



\subsubsection{Path Homotopy}\label{sec:path_homotopy}

\emph{Homotopic Paths} $f_0 \simeq f_1$ with \emph{Homotopy}
(\S\ref{sec:homotopy}) $f_t$

Family of Paths $f_t : I \rightarrow X$ for $t \in [0,1]$ such that:
\begin{enumerate}
  \item $f_t(0) = x_0$ and $f_t(1) = x_1$
  \item Map $F : I \times I \rightarrow X$ with $F(s,t) = f_t(s)$ is
    Continuous
\end{enumerate}

Linear Homotopy in $\reals^n$ (or any Convex Subspace
\S\ref{sec:convex_subspace} $X \subset \reals^n$):
\[
  f_t(s) = (1 - t) f_0(s) + t f_1(s)
\]



\paragraph{Homotopy Class}\label{sec:homotopy_class}\hfill

Equivalence Class $[f]$

Morphisms in the Category $\cat{Toph}$; this Category is not
Concretizable



\subsubsection{Loop}\label{sec:loop}

Path with equal Initial and Terminal Points called the
\emph{Basepoint}: Continuous Function
(\S\ref{sec:continuous_function}) $f$ from Unit Interval $[0,1]$ to
Topological Space $X$ where $f(0) = f(1) = x_0 \in X$

Two Loops $f_m$ and $f_n$ with the same Initial and Terminal Points
can be combined into a new Loop $f_m + f_n = B_{m+n}$.
\cite{hatcher02}

Loop Space (\S\ref{sec:loop_space})

Fundamental Group (\S\ref{sec:fundamental_group})

Link (\S\ref{sec:link})

\fist See also \emph{Loop} (Graph Theory \S\ref{sec:graph_loop})



\paragraph{Simple Closed Curve}\label{sec:simple_closed_curve}\hfill

or \emph{Jordan Curve}

Non-self-intersecting Continuous Loop in the Cartesian Plane $\reals^2$
(\S\ref{sec:cartesian_plane})

\fist Plane Curve (\S\ref{sec:plane_curve})

Winding Number (\S\ref{sec:winding_number})

\emph{Jordan Curve Theorem}: Every Jordan Curve divides the Plane into an
Interior region Bounded by the Curve, and an Exterior Region containing all the
Exterior Points, such that every Continuous Path connecting the two Regions
Intersects with the Loop.


MAE 5790 Lec. 8 - \url{https://www.youtube.com/watch?v=O2fcpxLT5wk}:

\fist Dynamical Systems (\S\ref{sec:dynamical_system})

``Index'' of a Simple Closed Curve in a Continuous Vector Field
(\S\ref{sec:vector_field}) \fist cf. Winding Number
(\S\ref{sec:winding_number})



% ------------------------------------------------------------------------------
\subsection{Pointed Space}\label{sec:pointed_space}
% ------------------------------------------------------------------------------

\fist Fundamental Group (Algebraic Topology \S\ref{sec:fundamental_group})



\subsubsection{Wedge Sum}\label{sec:wedge_sum}

For $X$, $Y$ Pointed Spaces, the \emph{Wedge Sum} (or \emph{One-point
  Union}):
\[
  X \vee Y = (X \amalg Y) / \sim
\]

Coproduct (\S\ref{sec:coproduct})



\paragraph{Rose}\label{sec:rose}\hfill

\paragraph{Sphere Boquet}\label{sec:sphere_boquet}\hfill



\subsubsection{Loop Space}\label{sec:loop_space}

Loop (\S\ref{sec:loop})

$\Omega X$

A$_\infty$-operad (\S\ref{sec:a_infinity_operad})



\paragraph{Free Loop Space}\label{sec:free_loop_space}\hfill

Space of Maps from $S^1$ to $X$ with Compact-open Topology
(\S\ref{sec:compact_open})



% ------------------------------------------------------------------------------
\subsection{Connected Space}\label{sec:connected_space}
% ------------------------------------------------------------------------------

\fist Connected Sum (\S\ref{sec:connected_sum})

a \emph{Connected Component} is a Maximal Connected Subset (Ordered by
Inclusion)

\fist cf. Connected Component (Graph Theory \S\ref{sec:connected_component})

\begin{itemize}
  \item for any Manifold $M$ with $n$ Connected Components, the $0$-th de Rham
    Cohomology Group (\S\ref{sec:derham_complex}) $H_{\mathrm{dR}}^0(M)$ is
    Isomorphic to $\reals^n$:
    \[
      H_{\mathrm{dR}}^0(M) \cong \reals^n
    \]
    following from the fact that any Smooth Function on $M$ with Zero Derivative
    (Locally Constant) is Constant on each of the Connected Components of $M$
\end{itemize}



\subsubsection{Connected Set}\label{sec:connected_set}

Interval (\S\ref{sec:interval})



\subsubsection{Path-connected}\label{sec:path_connected}

Any two Points can be joined by a Path (\S\ref{sec:path})



\subsubsection{Simply-connected}\label{sec:simply_connected}

\emph{Uniformization Theorem}: every Simply Connected Riemann Surface
(\S\ref{sec:riemann_surface}) is Conformally Equivalent to either the Open Unit
Disk (Hyperbolic Geometry), the Complex Plane (Euclidean Geometry), or the
Riemann Sphere (Spherical Geometry)

Uniformization Theorem is a generalization of the Riemann Mapping Theorem (TODO)

\fist cf. \emph{Thurston's Geometrization Conjecture}: analog for Closed
3-manifolds



\subsubsection{Locally Connected Space}\label{sec:locally_connected}

a Topological Space is \emph{Locally Connected} if it has the Local Property
(\S\ref{sec:local_property}) that every Point admits a Neighborhood Basis
consisting entirely of Open, Connected Sets



\subsubsection{Totally Disconnected Space}\label{sec:totally_disconnected}

\begin{itemize}
  \item the question of whether a given Iterated Function System
    (\S\ref{sec:ifs}) is Totally Disconnected is \emph{Undecidable} (Dube 1993)
\end{itemize}



% ------------------------------------------------------------------------------
\subsection{Hyperconnected Space}\label{sec:hyperconnected_space}
% ------------------------------------------------------------------------------

(wiki):

A \emph{Hyperconnected Space} (or \emph{Irreducible Space}) is a Topological
Space that cannot be written as the Union of two Proper Closed Sets (either
Disjoint or Non-disjoint).

equivalent conditions, Satisfaction of any one of which implies that a Space
$X$ is Hyperconnected:
\begin{itemize}
  \item no two (Nonempty) Open Sets are Disjoint
  \item $X$ cannot be written as the Union of two Proper Closed Sets
  \item every (Nonempty) Open Set is \emph{Dense} (\S\ref{sec:density}) in $X$
  \item the Interior (\S\ref{sec:interior}) of every Proper Closed Set is Empty
\end{itemize}



\subsubsection{Irreducible Set}\label{sec:irreducible_set}

(wiki):

An \emph{Irreducible Set} is a Subset of a Topological Space for which the
Subspace Topology is Irreducible (Hyperconnected).



% ==============================================================================
\section{Pointless Topology}\label{sec:pointless_topology}
% ==============================================================================

Complete Heyting Algebra (\S\ref{sec:complete_heyting_algebra})

Objects of $\cat{CHey}$, $\cat{Loc}$, and $\cat{Frm}$ are Complete
Lattices (\S\ref{sec:complete_lattice}) satisfying the Infinite
Distributive Law (\S\ref{sec:infinite_distributive}).

Rings (\S\ref{sec:ring}): Frames (\S\ref{sec:frame})

Schemes (\S\ref{sec:scheme}): Locale (\S\ref{sec:locale})



% ------------------------------------------------------------------------------
\subsection{Frame}\label{sec:frame}
% ------------------------------------------------------------------------------

Complete Lattice (\S\ref{sec:complete_lattice})

$\cat{Frm}$ Morphisms are Monotone Functions (Limit-)preserving Finite
Meets and arbitrary Joins

A Morphism of Frames that also preserves Implication is a Homomorphism
of Complete Heyting Algebras (\S\ref{sec:complete_heyting_algebra}).

Contravariant Functor $O : \cat{Top} \rightarrow \cat{Frm}$

Adjoint $|-|_{top} : \cat{Frm} \rightarrow \cat{Top}$ %FIXME

classical Stone Duality (\S\ref{sec:stone_duality}): Duality between the
Category $\cat{Sob}$ of Sober Spaces (\S\ref{sec:sober_space}) with Continuous
Functions and the Category $\cat{SFrm}$ of Spatial Frames (Complete Heyting
Algebras \S\ref{sec:complete_heyting_algebra}) and Frame Homomorphisms



% ------------------------------------------------------------------------------
\subsection{Locale}\label{sec:locale}
% ------------------------------------------------------------------------------

nLab:

like a Topological Space (\S\ref{sec:topological_space}) that may or may not
have ``enough Points''; contains \emph{Open Subspaces} but there may or may not
be enough Points to ``distinguish'' between Subspaces; an Open Subspace of a
Locale may be regarded as ``conveying'' a Bounded amount of Information about
the \emph{hypothetical} Points it contains (FIXME: clarify)

examples:
\begin{itemize}
  \item \emph{Localic Real Line} (\S\ref{sec:localic_real_line}) -- Locale of
    all Surjections from the Discrete Space $\nats$ to the Real Line $\reals$
  \item Cantor Set (\S\ref{sec:cantor_set})
  \item ... MORE?
\end{itemize}

Topos Theory (\S\ref{sec:topos_theory}) (???)

$\cat{Top}$

Ring (\S\ref{sec:ring})

Sierpi\'nski Space (\S\ref{sec:sierpinski_space}) $S$

Topological Space $X = (M_X,\tau_X)$

$Map(X,S) \simeq \tau_X$



% ==============================================================================
\section{Chu Space}\label{sec:chu_space}
% ==============================================================================

(wiki):

Models of Concurrent Computation in Automata Theory to express ``Branching
Time'' and Concurrency

Chu Spaces correspond to \emph{Wavefunctions} (\S\ref{sec:wave_function}) as
Vectors of Hilbert Space (\S\ref{sec:hilbert_space})



% ==============================================================================
\section{Kolmogorov Classification}\label{sec:kolmogorov_classification}
% ==============================================================================

% ------------------------------------------------------------------------------
\subsection{Countability Axiom}\label{sec:countability_axiom}
% ------------------------------------------------------------------------------

\subsubsection{Separation Axioms}\label{sec:separation_axioms}



% ------------------------------------------------------------------------------
\subsection{Distinguishable Space}\label{sec:distinguishable_space}
% ------------------------------------------------------------------------------

$\xspace{T}_0$

the number of possible Distinguishable Topologies on a Finite Set is equal to
the number of Partial Orders (\S\ref{sec:partial_order}) on the Set



\subsubsection{Sober Space}\label{sec:sober_space}

A \emph{Sober Space} is a Topological Space $X$ such that every Irreducible
Closed Subset (\S\ref{sec:irreducible_set}) of $X$ is the Topological Closure
(\S\ref{sec:topological_closure}) of exactly one Point of $X$, i.e. every
Irreducible Closed Subset has a unique Generic Point
(\S\ref{sec:generic_point}).

classical Stone Duality (\S\ref{sec:stone_duality}): Duality between the
Category $\cat{Sob}$ of Sober Spaces (\S\ref{sec:sober_space}) with Continuous
Functions and the Category $\cat{SFrm}$ of Spatial Frames (Complete Heyting
Algebras \S\ref{sec:complete_heyting_algebra}) and Frame Homomorphisms



% ------------------------------------------------------------------------------
\subsection{Accessible Space}\label{sec:accessible_space}
% ------------------------------------------------------------------------------

$\xspace{T}_1$

or \emph{Tychonoff Space}

an Accessible Space is a Completely Regular Separated (Hausdorff) Space
(\S\ref{sec:separated_space})

every Locally Euclidean Space (\S\ref{sec:locally_euclidean}) is Accessible



% ------------------------------------------------------------------------------
\subsection{Separated Space}\label{sec:separated_space}
% ------------------------------------------------------------------------------

$\xspace{T}_2$

or \emph{Hausdorff Space}

distinct Points have Disjoint Neighborhoods (\S\ref{sec:neighborhood})

a Locally Euclidean (\S\ref{sec:locally_euclidean}) Separated Space is called a
\emph{Topological Manifold} (\S\ref{sec:topological_manifold})

every Compact (\S\ref{sec:compact_space}) Hausdorff Space is also Locally
Compact (\S\ref{sec:locally_compact}), examples:
\begin{itemize}
  \item the Real Line, Unit Interval $[0,1]$, Epsilon Neighborhoods
  \item the Cantor Set (\S\ref{sec:cantor_set})
  \item Hilbert Cube (\S\ref{sec:hilbert_cube})
  \item the Stone Space (Profinite Space \S\ref{sec:stone_space})
    $\xspace{S}(B)$ of a Boolean Algebra $B$ -- a Compact Totally Disconnected
    (\S\ref{sec:totally_disconnected}) Hausdorff Space
\end{itemize}

an Accessible (Tychonoff) Space (\S\ref{sec:accessible_space}) is a Completely
Regular Separated (Hausdorff) Space



\subsubsection{Compactly Generated Separated Space}\label{sec:compact_separated}

Compactly Generated Hausdorff Space



\subsubsection{Hilbert Cube}\label{sec:hilbert_cube}

Topological Product of Intervals $[0, 1/n]$ for $n \in \nats^+$, i.e. a Cuboid
(\S\ref{sec:cuboid}) of Countably Infinite Dimension, Topologically
indistinguishable from the Unit Cube of Countably Infinite Dimension

a (Locally) Compact (\S\ref{sec:locally_compact}) Hausdorff Space



% ------------------------------------------------------------------------------
\subsection{Regular Space}\label{sec:regular_space}
% ------------------------------------------------------------------------------

$\xspace{T}_3$ -- Regular Hausdorff (\S\ref{sec:separated_space})



\subsubsection{Completely Regular Space}\label{sec:completely_regular_space}

an Accessible (Tychonoff) Space (\S\ref{sec:accessible_space}) is a Completely
Regular Separated (Hausdorff) Space (\S\ref{sec:separated_space})



% ------------------------------------------------------------------------------
\subsection{Normal Space}\label{sec:normal_space}
% ------------------------------------------------------------------------------

$\xspace{T}_4$ -- Normal Hausdorff

$\xspace{T}_5$ -- Completely Normal

$\xspace{T}_6$ -- Perfectly Normal



% ------------------------------------------------------------------------------
\subsection{Separable Space}\label{sec:separable_space}
% ------------------------------------------------------------------------------



% ==============================================================================
\section{Metric Space}\label{sec:metric_space}
% ==============================================================================

\fist Metric Geometry (\S\ref{sec:metric_geometry})

A \emph{Metric Space} is a Set $M$ for which a \emph{Metric}
(\S\ref{sec:metric}) is defined for all Elements of that Set.

A Metric Space $M$ can then be defined as the pair:
\[
  (M,d)
\]
where $M$ is a Set of Elements and $d$ is a Distance Function.

Subsets of a Metric Space may be \emph{Open} (\S\ref{sec:open_set}),
\emph{Closed} (\S\ref{sec:closed_set}), both or neither.

Abstract Axiomatic Structure needed to define a Theory of \emph{Differentiation}
(\S\ref{sec:differential_calculus})

cf. \emph{Measure Spaces} (\S\ref{sec:measure_space}) -- Abstract Axiomatic
Structure needed to define a Theory of \emph{Integration}
(\S\ref{sec:integral_calculus})

Denotational Semantics (\S\ref{sec:denotational_semantics})

Domain Theory (\S\ref{sec:domain_theory})

(wiki): Linear Inverse Problems (Functional Analysis
\S\ref{sec:linear_inverse_problem}) are represented by mappings between Metric
Spaces

any Normed Vector Space (\S\ref{sec:normed_vectorspace}) is a Metric Space by
defining $d(x,y) = \|y-x\|$



% ------------------------------------------------------------------------------
\subsection{Metric}\label{sec:metric}
% ------------------------------------------------------------------------------

A \emph{Metric} is a \emph{Distance Function} defined between all
Elements or Points of a Metric Space.

cf. Norm (\S\ref{sec:norm}), $p$-norm (\S\ref{sec:p_norm})

A Distance Function $d$ has the form:
\[
  d: M \times M \rightarrow \mathbb{R}^{+}
\]
with the following conditions:
\begin{enumerate}
\item $d(x_1, x_2) \geq 0$ (\emph{Non-negativity} or \emph{Separation
  Axiom})
\item $d(x_1, x_2) = 0 \leftrightarrow x_1 = x_2$ (\emph{Identity of
  Indiscernables} or \emph{Coincidence Axiom})
\item $d(x_1, x_2) = d(x_2, x_1)$ (\emph{Symmetry})
\item $d(x_1, x_3) \leq d(x_1, x_2) + d(x_2, x_3)$
  (\emph{Subadditivity} or \emph{Triangle Inequality})
\end{enumerate}
Two Metrics, $d_1$ and $d_2$, in a Metric Space, $M$, are
\emph{Metrically Equivalent}, $d_1 \sim d_2$, if for Topologies
$\tau_1$ Induced by $d_1$ and $\tau_2$ Induced by $d_2$, $\tau_1 =
\tau_2$. A sufficient condition for Metric Equivalence is given by:
\[
  \exists k_1, k_2 > 0 : \forall x \in M, \forall r > 0,
  B^{d1}_{rk_1}(x) \subseteq B^{d2}_r (x) \subseteq B^{d1}_{rk_2}(x)
\]

\emph{Euclidean Metric} ($2$-norm):
\[
  d: \mathbb{R}^n \times \mathbb{R}^n \rightarrow \mathbb{R}
\]\[
  (\mathbf{p},\mathbf{q}) \mapsto \sqrt{\sum_{i=1}^{n}(q_i - p_i)^2}
\]

\emph{Discrete Metric}:
\[
  d: X \times X \rightarrow \mathbb{R}^{+}
\]\[
  (\mathbf{p},\mathbf{q}) \mapsto \left\{
  \begin{array}{l l}
    0: \mathbf{p} = \mathbf{q}\\
    1: \mathbf{p} \neq \mathbf{q}
  \end{array}\right.
\]

\emph{Max Metric} ($\infty$-norm):
\[
  d: \mathbb{R}^n \times \mathbb{R}^n \rightarrow \mathbb{R}
\]\[
  (\mathbf{p},\mathbf{q}) \mapsto max_{1 \leq i \leq n}\{|q_i - p_i|\}
\]

\emph{Taxicab Metric} ($1$-norm):
\[
  d: \mathbb{R}^n \times \mathbb{R}^n \rightarrow \mathbb{R}
\]\[
  (\mathbf{p},\mathbf{q}) \mapsto \sum_{i=1}^{n}|q_i - p_i|
\]

\fist Hamming Distance (Hamming Spaces \S\ref{sec:hamming_distance})



\subsubsection{Metric Derivative}\label{sec:metric_derivative}

(wiki): generalized notion of ``speed'' (Absolute Velocity; cf. conventional
Derivative \S\ref{sec:derivative}) from Vector Spaces, with both notion of
Distance and Direction, to Metric Spaces, with only a notion of Distance but not
Direction



% ------------------------------------------------------------------------------
\subsection{Ball}\label{sec:ball}
% ------------------------------------------------------------------------------

A Metric \emph{Ball} is defined for a Point $p$ in a Metric Space
$(M,d)$ as the set of all Points (including $p$) within a given Radius
$r > 0$ as determined by the Distance Function of the Metric Space:
\[
  B_r(p) = {x \in M | d(x,p) < r }
\]
The above is termed an \emph{Open Ball} because it does not include
the points where $d(x,p) = r$. Such a Ball including these additional
Points is called a \emph{Closed Ball}.

a Subset of a Metric Space is \emph{Bounded} (\S\ref{sec:bounded_subset}) if it
is contained in a Ball of Finite Radius



\subsubsection{Unit Ball}\label{sec:unit_ball}

Unit Sphere (\S\ref{sec:unit_sphere})



\subsubsection{Open Ball}\label{sec:open_ball}

Open $n$-ball $B^n$

Homeomorphic to Cartesian Space (\S\ref{sec:cartesian_space})
$\reals^n$



\subsubsection{Closed Ball}\label{sec:closed_ball}

Closed $n$-ball $D^n$



% ------------------------------------------------------------------------------
\subsection{Bounded Subset}\label{sec:bounded_subset}
% ------------------------------------------------------------------------------

a Subset of a Metric Space is \emph{Bounded} if it is contained in a Ball of
Finite Radius



% ------------------------------------------------------------------------------
\subsection{Bounded Space}\label{sec:bounded_space}
% ------------------------------------------------------------------------------

a Space is Compact (\S\ref{sec:compact_space}) if it is Bounded
(\S\ref{sec:bounded_space}) and Closed (\S\ref{sec:closed_set})



% ------------------------------------------------------------------------------
\subsection{Extremum}\label{sec:extremum}
% ------------------------------------------------------------------------------

\fist Critical Points (Differentiable Functions \S\ref{sec:critical_point})



\subsubsection{Maximum}\label{sec:maximum}

Upper Bound (\S\ref{sec:upper_bound})

Least Upper Bound (\S\ref{sec:least_upperbound})

Local Maxima

Second Derivative Test (\S\ref{sec:second_derivative_test})

Multivariable -- Gradient (\S\ref{sec:gradient}) equal to the Zero Vector:
$\nabla{f} = \vec{0}$

\fist \emph{Saddle Points} (\S\ref{sec:saddle_point}) are specific to
Multivariable Functions



\paragraph{Argmax}\label{sec:argmax}\hfill



\subsubsection{Minimum}\label{sec:minimum}

Lower Bound (\S\ref{sec:lower_bound})

Greatest Lower Bound (\S\ref{sec:greatest_lowerbound})

Local Minima

Second Derivative Test (\S\ref{sec:second_derivative_test})

Multivariable -- Gradient (\S\ref{sec:gradient}) equal to the Zero Vector:
$\nabla{f} = \vec{0}$

\fist \emph{Saddle Points} (\S\ref{sec:saddle_point}) are specific to
Multivariable Functions

Convex Functions (\S\ref{sec:convex_function}) have the property that Local
Minimum is necessarily a Global Minimum



\paragraph{Argmin}\label{sec:argmin}\hfill



% ------------------------------------------------------------------------------
\subsection{Continuous Function}\label{sec:continuous_function}
% ------------------------------------------------------------------------------

A Function $f : X \rightarrow Y$ between Metric Spaces $(X,d)$ and $(Y,d')$ is
\emph{Continuous} at a Point $c$ if:
\[
  \forall \epsilon > 0, \exists \delta > 0 :
  f (B_{\delta}(c)) \subseteq B_{\epsilon}(f(c))
\]
$f$ is a \emph{Continuous Function} if it satisfies the above for all $c$. All
Continuous Functions are Sequentially Continuous
(\S\ref{sec:sequentially_continuous}).

Alternatively, a Function $f: X \rightarrow Y$ is Continuous between Metric
Spaces if and only if for all Open Sets $V \subseteq Y$, $f^{-1}(v)$ is an Open
Set in $X$.

\fist cf. Continuous Maps (\S\ref{sec:continuous_map}) between Topological
Spaces

\fist cf. \emph{Measurable Functions} (\S\ref{sec:measurable_function}) -- a
Function between Measurable Spaces (\S\ref{sec:measurable_space}) is Measurable
if the Preimage of any Measurable Set is Measurable

a Continuous Function with a Continuous Inverse Function is called a
\emph{Homeomorphism}

Continuous Functions form the Differentiability Class
(\S\ref{sec:differentiability_class}) $\mathcal{C}^0$

Real-valued Continuous Function (\S\ref{sec:real_continuous})

Continuously Differentiable (\S\ref{sec:continuously_differentiable})

Local (Point)

Continuous on Closed and Bounded implies Uniformly Continuous
(\S\ref{sec:uniform_continuity})

the Fundamental Theorem of Calculus (\S\ref{sec:fundamental_calculus_theorem})
treats of Real-valued Continuous Functions

a Bounded Function (\S\ref{sec:bounded_function}) on a Compact Interval $[a,b]$
is Riemann Integrable (\S\ref{sec:integrable_function}) if and only if it is
Continuous ``Almost Everywhere'', i.e. Set of Points of Discontinuity has
Lebesgue Measure Zero (\S\ref{sec:lebesgue_measure})

an Open or Closed Continuous Surjective Map is necessarily a Quotient Map
(\S\ref{sec:quotient_map})

\emph{Weierstrass Approximation Theorem}
(\S\ref{sec:weierstrass_approximation}): every Continuous Function defined on a
Closed Interval can be Uniformly Approximated (\S\ref{sec:uniform_convergence})
to arbitrary closeness by a \emph{Polynomial Function}
(\S\ref{sec:polynomial_function})

\emph{Stone-Weierstrass Theorem}: the Set of all Continuous Functions on a
Closed Interval is the Uniform Closure (cf. Uniform Norm \S\ref{sec:p_norm}) of
the Set of Polynomials on the Interval

Continuous Functions $f : \reals^n \rightarrow \reals$ form a Filtration
(\S\ref{sec:filtration}):
\[
  \cdots \subset C^n(\reals^k) \subset \cdots \subset C^1(R^k) \subset
    C^0(\reals^k)
\]

\begin{itemize}
  \item Normal Family (\S\ref{sec:normal_family}) -- a Pre-compact Subset
    (\S\ref{sec:precompact}) of the Space of Continuous Functions with respect
    to the Compact-open Topology (\S\ref{sec:compact_open})
\end{itemize}



\subsubsection{Uniform Continuity}\label{sec:uniform_continuity}

Global (Set)

\begin{itemize}
  \item Singular Functions (\S\ref{sec:singular_function})
    --FIXME: correct or only cantor function ???
\end{itemize}



\subsubsection{$\alpha$-H\"older Continuity}\label{sec:holder_continuity}

\emph{H\"older Condition}:
\[
  |f(x) - f(y)| \leq C \|x - y\|^\alpha
\]

\emph{H\"older Exponent}



\subsubsection{Absolute Continuity}\label{sec:absolute_continuity}

\subsubsection{Lipschitz Continuity}\label{sec:lipschitz_continuity}

\subsubsection{Baire Function}\label{sec:baire_function}

a Baire Set (\S\ref{sec:baire_set}) is a Set whose Characteristic Function
(Indicator Function \S\ref{sec:indicator_function}) is a Baire Function

%FIXME: move section ?



\subsubsection{Discontinuous Function}\label{sec:discontinuous_function}

\paragraph{Discontinuity}\label{sec:discontinuity}\hfill

\fist Singularity (\S\ref{sec:singularity})

Type I: Jump Discontinuity, Removable Discontinuity

Type II: Infinite Discontinuity, Essential Singularity



\paragraph{Unit Step Function}\label{sec:unit_step_function}\hfill

%FIXME: move this section ???

\emph{Heaviside Step Function} or \emph{Heaviside Theta}

Anti-derivative of the Dirac Delta (Unit Impulse \S\ref{sec:dirac_delta})

Step Function (\S\ref{sec:step_function})

$u_c(t)$

$H$

\begin{itemize}
  \item a Perceptron (\S\ref{sec:perceptron}) is an Artificial Neuron
    (\S\ref{sec:ann}) using the Unit Step Function as the Activation Function
\end{itemize}



\paragraph{Rectangular Function}\label{sec:rectangular_function}\hfill

%FIXME: move this section ???

or \emph{Unit Pulse}

cf. Dirac Delta Function (Unit Impulse Function \S\ref{sec:dirac_delta})

the Fourier Transform of the Rectangular Function is the Sinc Function
(\S\ref{sec:sinc_function})



\subsubsection{Topological Curve}\label{sec:topological_curve}

A \emph{Topological Curve} is the Image of a Continuous Function from a Real
Interval (\S\ref{sec:real_interval}) into a Topological Space

a \emph{Curve} is a Topological Space that is locally Homeomorphic to a Line
(\S\ref{sec:algebraic_line})

\begin{itemize}
  \item Simple Closed Curve (\S\ref{sec:simple_closed_curve})
  \item Plane Curve (\S\ref{sec:plane_curve})
  \item Space-filling Curve (\S\ref{sec:space_filling_curve})
  \item Fractal Curve (\S\ref{sec:fractal_curve})
  \item Differentiable Curve (\S\ref{sec:differentiable_curve})
  \item Parametric Curve (\S\ref{sec:parametric_curve})
  \item Integral Curve (\S\ref{sec:integral_curve})
\end{itemize}



% ------------------------------------------------------------------------------
\subsection{Locally Invertible Map}\label{sec:locally_invertible}
% ------------------------------------------------------------------------------

Locally (\S\ref{sec:local_property})

Invertible (\S\ref{sec:inverse_function})

Curvilinear Coordinates (\S\ref{sec:curvilinear_coordinates}) can be derived
from a Locally Invertible Transformation of a Set of Cartesian Coordinates; cf.
Affine Coordinates (\S\ref{sec:affine_coordinates}) can be derived from a Linear
Transformation (\S\ref{sec:linear_transformation}) of a Set of Cartesian
Coordinates

--FIXME: is this equivalent to the condition that a transformation $F$ has
components that are Invertible, Real-valued with Continuous $2$nd Partial
Derivatives everywhere ???
\url{https://www.youtube.com/watch?v=XtpVVcKXfnA}



% ------------------------------------------------------------------------------
\subsection{Isometry}\label{sec:isometry}
% ------------------------------------------------------------------------------

Distance-preserving Injective Map between Metric Spaces

a Similarity Transform (\S\ref{sec:similarity_transformation}) with Scaling
Factor $r = 1$ is a \emph{Euclidean Isometry} (or \emph{Rigid Transformation}
\S\ref{sec:rigid_transformation})

Isometries of Euclidean Space (\S\ref{sec:euclidean_space}):
\begin{itemize}
  \item Point Reflection (Central Inversion) -- an Affine Transformation
  \item TODO
  ...
\end{itemize}

Isometries in Hilbert Space (\S\ref{sec:hilbert_space}):
\begin{itemize}
  \item Unitary Transformations (\S\ref{sec:unitary_transformation})
  \item ...
\end{itemize}

A \emph{Motion} (\S\ref{sec:motion}) is a Surjective Isometry of a Metric
Space.

Crystallographic Groups (\S\ref{sec:crystallographic_group}) are Cocompact
(\S\ref{sec:cocompact_space}), Discrete Subgroups (\S\ref{sec:discrete_group})
of the Isometries of some Euclidean Space (\S\ref{sec:euclidean_space})



\subsubsection{Motion}\label{sec:motion}

%FIXME: move to geometry ???

A \emph{Motion} is a Surjective Isometry of a Metric Space.



\subsubsection{Isometry Group}\label{sec:isometry_group}

``Rigid Geometry''

Automorphism Group (\S\ref{sec:automorphism_group}) $G$ of a Space $X$

Symmetry Group (\S\ref{sec:symmetry_group}) of a Space is a Subgroup
of the Isometry Group of a Space %FIXME: ???

Set of all Endomorphic Bijective Isometries with Composition of
Isometries as the Group Operation and the Identity Function as the
Identity Element

Every Isometry Group is a Subgroup (\S\ref{sec:subgroup}) of Isometries

Euclidean Group (\S\ref{sec:euclidean_group}) $E(n)$ -- Group of all
Isometries $ISO(n)$ of $n$-dimensional Euclidean Space
(\S\ref{sec:euclidean_space})

Frieze Groups (\S\ref{sec:frieze_group}) and Wallpaper Groups
(\S\ref{sec:wallpaper_group}) are Discrete Subgroups
(\S\ref{sec:discrete_group}) of the Isometry Group of the Euclidean Plane
(\S\ref{sec:euclidean_plane})



\paragraph{Discrete Isometry Group}\label{sec:discrete_isometry_group}\hfill

cf. Discrete Group (\S\ref{sec:discrete_group})

a Discrete Symmetry Group (\S\ref{sec:discrete_symmetry_group}) is a Symmetry
Group that is a Discrete Isometry Group



% ------------------------------------------------------------------------------
\subsection{Dilation}\label{sec:dilation}
% ------------------------------------------------------------------------------

a Function $f$ from a Metric Space to itself that Satisfies:
\[
  d(f(x), f(y)) = rd(x, y)
\]

a \emph{Similarity} (\S\ref{sec:similarity_transformation}) is a Dilation of
Euclidean Space

the Group of Translations and Homotheties forms a \emph{Dilation Group}
(\S\ref{sec:dilation_group})



% ------------------------------------------------------------------------------
\subsection{Contraction Map}\label{sec:contraction_map}
% ------------------------------------------------------------------------------

Metric Space $(M,d)$

\emph{Contraction Map} (or \emph{Contraction} or \emph{Contractor}) is a
Function $f$ from $M$ to itself such that there is some Non-negative Real Number
$0 \leq k < 1$ such that $\forall x,y \in M$:
\[
  d(f(x),f(y)) \leq k d(x,y)
\]

\fist an \emph{Iterated Function System} (IFS \S\ref{sec:ifs}) is a Finite Set
of Contraction Mappings on a Complete Metric Space
(\S\ref{sec:complete_metric_space})



\subsubsection{Banach Fixed-point Theorem}\label{sec:banach_fixedpoint}

\fist See application in the Topological Semantics
(\S\ref{sec:topological_semantics}) of Programming Languages



% ------------------------------------------------------------------------------
\subsection{Complete Metric Space}\label{sec:complete_metric_space}
% ------------------------------------------------------------------------------

Metric Space $(M,d)$ is \emph{Complete} if every Cauchy Sequence
(\S\ref{sec:cauchy_sequence}) of Elements in $M$ has a Limit in $M$

Partial Traces (\S\ref{sec:partial_trace})

a Complete Normed Vector Space (\S\ref{sec:normed_vectorspace}) is a Banach
Space (\S\ref{sec:banach_space})

a Complete Metric Space with an Inner Product (\S\ref{sec:inner_product}) is a
Hilbert Space (\S\ref{sec:hilbert_space})

a \emph{Polish Space} (\S\ref{sec:polish_space}) is a Space that is Homeomorphic
to a Complete Metric Space

\fist an \emph{Iterated Function System} (\S\ref{sec:ifs}) is a Finite Set of
Contraction Mappings (\S\ref{sec:contraction_map}) on a Complete Metric Space

\begin{itemize}
  \item Banach Space -- a Complete Normed Vector Space
    (\S\ref{sec:normed_vectorspace})
  \item ...
\end{itemize}



% ------------------------------------------------------------------------------
\subsection{Compact Metric Space}\label{sec:compact_metric_space}
% ------------------------------------------------------------------------------

$2^{\aleph_0}$ is the largest number of Connected Components a Compact Metric
Space can have
--\url{https://golem.ph.utexas.edu/category/2017/11/topology_puzzles.html}



% ------------------------------------------------------------------------------
\subsection{Ultrametric Space}\label{sec:ultrametric_space}
% ------------------------------------------------------------------------------

Triangle Inequality (\S\ref{sec:triangle_inequality}) is replaced
with:
\[
  d(x,z) \leq max\{d(x,y),d(y,z)\}
\]

Functional Reactive Programming (FRP) (\S\ref{sec:frp}): Decoupled
Functions form Contraction Maps (\S\ref{sec:contraction_map}) in an
Ultrametric Space of Functions



% ------------------------------------------------------------------------------
\subsection{Magnitude}\label{sec:magnitude}
% ------------------------------------------------------------------------------

2016 - Leinster - A Survey of Magnitude (n-Category Cafe) %FIXME

general definition of the Magnitude of an Enriched Category %FIXME

Lawvere: Metric Spaces are usefully viewed as a certain kind of
Enriched Category

Finite Metric Space $A$

$Z_A$ -- Square Matrix whose Rows and Columns are Indexed by Points of
$A$ with entries $Z_A(a,b) = e^{-d(a,b)}$

for Invertible $Z_A$, the \emph{Magnitude} of $A$ is:
\[
  |A| = \sum_{a,b \in A} Z^{-1}_A (a,b)
\]

Magnitude is Well-defined for every Finite Subset of Euclidean Space;
$Z_A$ is Positive Definite when $A \subseteq \reals^n$

can be (canonically) extended from Finite Space to Compact Spaces as
long as they are ``Positive Definite'', i.e. the Matrix $Z_A$ is
Positive Definite for every Finite Subset $A$; this includes all
Compact Subspaces of $\reals^n$



\subsubsection{Magnitude Function}\label{sec:magnitude_function}

Volume, Minkowski Dimension



% ------------------------------------------------------------------------------
\subsection{Metric Space Cardinality}\label{sec:metric_cardinality}
% ------------------------------------------------------------------------------

\url{https://golem.ph.utexas.edu/category/2008/11/entropy_diversity_and_cardinal_1.html}

\emph{Weighing}



% ==============================================================================
\section{Homotopy Theory}\label{sec:homotopy_theory}
% ==============================================================================

Homotopical Algebra (\S\ref{sec:homotopical_algebra})

Homotopy Type Theory (\S\ref{sec:hott})

\url{https://ncatlab.org/nlab/show/homotopy}

``Higher Groupoid Theory'' (\S\ref{sec:groupoid})

Simplicial Set (\S\ref{sec:simplicial_set}) -- a purely Combinatorial
construction designed to capture the notion of a ``Well-behaved'' Topological
Space for use in Homotopy Theory (wiki)

\fist an $\infty$-groupoid (\S\ref{sec:infinity_groupoid}) is an Abstract
Homotopical Model for Topological Spaces (\S\ref{sec:topological_space}): by the
\emph{Homotopy Hypothesis}, $\infty$-groupoids are \emph{Spaces}

Eilenberg-MacLane Space (\S\ref{sec:eilenberg_maclane_space})

see also:
\begin{itemize}
  \item Derived Algebraic Geometry (\S\ref{sec:derived_algebraic_geometry}) --
    generalization of Algebraic Geometry replacing Commutative Rings with Ring
    Spectra in Algebraic Topology where higher Homotopy accounts for
    Non-discreteness (Tor) of the Structure Sheaf (FIXME: Tor ???)
  \item Derived Scheme (\S\ref{sec:derived_scheme}) -- a Homotopy-theoretic
    generalization of a Scheme (\S\ref{sec:scheme})
\end{itemize}



% ------------------------------------------------------------------------------
\subsection{Homotopy}\label{sec:homotopy}
% ------------------------------------------------------------------------------

$H : X \times [0,1] \rightarrow Y$

\emph{Continuous Deformation}

Homology Groups (\S\ref{sec:homology_group}) are a ``Functorial'' Homotopy
Invariant (FIXME: clarify)



\subsubsection{Classifying Space}\label{sec:classifying_space}

Dependent Types (\S\ref{sec:dependent_type})

a Configuration Space (\S\ref{sec:configuration_space}) is a type of
Classifying Space



\paragraph{Classifying Morphism}\label{sec:classifying_morphism}\hfill

Characteristic Morphism (\S\ref{sec:characteristic_morphism}) for
Subobjects; Classifying Space is the Subobject Classifier
(\S\ref{sec:subobject_classifier})

Substitution (\S\ref{sec:substitution}) of Terms in Dependent Types
can be Interpreted as Composition of the Morphism Interpreting the
Term with the Classifying Morphism Interpreting the Dependent Type



% ------------------------------------------------------------------------------
\subsection{Homotopic Function}\label{sec:homotopic_function}
% ------------------------------------------------------------------------------

Continuous Functions (\S\ref{sec:continuous_function})



\subsubsection{Null-homotopy}\label{sec:null_homotopy}

A Function $f$ is \emph{Null-homotopic} if it is Homotopic to a
Constant Function.

A Space $X$ is Contractible (\S\ref{sec:contractible_space}) if and
only if the Identity Map $id_X$ (always a Homotopy Equivalence
\S\ref{sec:homotopy_equivalence})) is Null-homotopic.



% ------------------------------------------------------------------------------
\subsection{Homotopy Equivalence}\label{sec:homotopy_equivalence}
% ------------------------------------------------------------------------------

Two Topological Spaces $X$ and $Y$ are \emph{Homotopy Equivalent} and
have the same \emph{Homotopy Type} (\S\ref{sec:homotopy_type}) if
there are Continuous Maps $f : X \rightarrow Y$ and $g : Y \rightarrow
X$ called \emph{Homotopy Equivalences} such that $g \circ f$ is
Homotopic (\S\ref{sec:homotopic_function}) to the Identity Map $id_X$
and $f \circ g$ is Homotopic to $id_Y$. Every Homeomorphism
(\S\ref{sec:homeomorphism}) is a Homotopy Equivalence, but not every
Homotopy Equivalence is a Homeomorphism.

Identity Functions are always Homotopy Equivalences. %FIXME

$X$ and $Y$ are Homotopy Equivalent if and only if there exists a
third Topological Space $Z$ containing both $X$ and $Y$ as Deformation
Retracts (\S\ref{sec:deformation_retraction})



\subsubsection{Homotopy Level}\label{sec:homotopy_level}

Truncation (\S\ref{sec:truncation})

\emph{$h$-level}

\begin{tabular}{l l l l}
$h$-level 0   & $(-2)$-groupoid & (Contractible Type, $\top$, Trivial)
  & \\
$h$-level 1   & $(-1)$-groupoid & (Proposition)
  & $h$-proposition (\S\ref{sec:h_proposition}) \\
$h$-level 2   & $0$-groupoid    & (Set)
  & $h$-set (\S\ref{sec:h_set}) \\
$h$-level 3   & $1$-groupoid    & (Groupoid)
  & $h$-groupoid (\S\ref{sec:h_groupoid}) \\
\end{tabular}



\subsubsection{Homotopy Type}\label{sec:homotopy_type}

Homotopy Type of a Point is called \emph{Contractible}
(\S\ref{sec:contractible_space}), by requiring that the Identity Map
of the Space is Null-homotopic (\S\ref{sec:null_homotopy}).

the Homotopy Type of Configuration Spaces (\S\ref{sec:configuration_space}) is
not Homotopy Invariant

if two Topological Spaces $X$ and $Y$ are the same Homotopy Type then their
Homology Groups (\S\ref{sec:homology_group}) are equal:
\[
  H_n(X) = H_n(Y)
\]
so Homology Groups are Topological Invariants



\subsubsection{Homotopy $0$-type}\label{sec:homotopy_0type}

$h$-set (\S\ref{sec:h_set})



\subsubsection{Homotopy $1$-type}\label{sec:homotopy_1type}

\emph{Homotopy $1$-type}

Groupoids (\S\ref{sec:groupoid})

$h$-groupoid (\S\ref{sec:h_groupoid})



% ------------------------------------------------------------------------------
\subsection{Fundamental Groupoid}\label{sec:fundamental_groupoid}
% ------------------------------------------------------------------------------

cf. Fundamental Group (\S\ref{sec:fundamental_group})

$\Pi_1(X)$

Groupoid (\S\ref{sec:groupoid}) with Objects as Points of $X$ and Morphisms as
Paths in $X$, identified up to endpoint-preserving Homotopy
(\S\ref{sec:homotopy})



\subsubsection{Fundamental $\infty$-groupoid}
\label{sec:fundamental_infinity_groupoid}

%FIXME: move this section ???

the \emph{Fundamental $\infty$-groupoid}, $\Pi_\infty(X)$, of a Topological
Space $X$ is the $\infty$-groupoid (\S\ref{sec:infinity_groupoid}) whose
$k$-morphisms are the $k$-dimensional Paths in $X$



% ------------------------------------------------------------------------------
\subsection{Homotopy Extension Property}\label{sec:homotopy_extension}
% ------------------------------------------------------------------------------

for Subspace $A \subset X$, if $(X,A)$ has the Homotpy Extension
Property then the Inclusion Map (\S\ref{sec:inclusion_map}) $i : A
\rightarrow X$ is a Cofibration (\S\ref{sec:cofibration})



% ------------------------------------------------------------------------------
\subsection{Fibration}\label{sec:fibration}
% ------------------------------------------------------------------------------

Homotopy Lifting Property

``a Fibration is a Functor that has well-behaved Inverse Images'' (TODO:
explain)



\subsubsection{Cofibration}\label{sec:cofibration}

Any Cofibration can be treated as having the Homotopy Extension
Property (\S\ref{sec:homotopy_extension})



\subsubsection{Codomain Fibration}\label{sec:codomain_fibration}

\subsubsection{Fiber Bundle}\label{sec:fiber_bundle}

A \emph{Fiber Bundle} a Space that is Locally a Product Space
(\S\ref{sec:product_space}) but Globally may have a different Topological
Structure, defined as a structure:
\[
  (E, B, \pi, F)
\]
of Topological Spaces $E$ (the \emph{Total Space}), $B$ (the \emph{Base
  Space}), and $F$ (the \emph{Fiber}) and a Continuous Surjection $\pi : E
\rightarrow B$ called the \emph{Projection} or \emph{Submersion} of the Bundle
satisfying a \emph{Local Triviality Condition}:

For every $x \in E$ there is an Open Neighborhood $U \subset B$ of $\pi(x)$
called the \emph{Trivializing Neighborhood} such that there is a Homeomorphism
$\varphi : \pi^{-1}(U) \rightarrow U \times F$ such that $\pi$ ``agrees'' with
the Projection onto the ``first factor'' --*FIXME*: clarify and explain

The case where $E$ is just $B \times F$ is a \emph{Trivial Bundle}
(\S\ref{sec:trivial_bundle})

\fist a \emph{Bundle} (\S\ref{sec:bundle}) is a generalization of a Fiber Bundle
dropping the condition of a Local Product Structure (FIXME: clarify)

\fist Bundle Object (\S\ref{sec:bundle_object})

\fist Fiber (Set Theory \S\ref{sec:fiber})

\url{https://johncarlosbaez.wordpress.com/2020/01/07/topos-theory-part-2/}:
every Sheaf (\S\ref{sec:sheaf}) comes from a Bundle, and every Sheaf (or
Presheaf) gives rise to a Bundle

Hopf Fibration

examples:

Vector Space (\S\ref{sec:vector_space}) as a Fiber Bundle on a Set of Basis
Vectors

Vector Bundles (\S\ref{sec:vector_bundle}) have Vector Spaces for Fibers;
includes Tangent Bundles (\S\ref{sec:tangent_bundle}) and Cotangent Bundles
(\S\ref{sec:cotangent_bundle})

%FIXME: cleanup



\paragraph{Bundle Map}\label{sec:bundle_map}\hfill

Pushforward (Differential of Smooth Maps \S\ref{sec:pushforward}) induces a
Bundle Map, specifically a Vector Bundle Homomorphism



\paragraph{Hopf Fibration}\label{sec:hopf_fibration}\hfill

mapping from Unit $3$-sphere (\S\ref{sec:unit_glome}) in the Two-dimensional
State Space $\comps^2$ (FIXME: clarify) to the Bloch Sphere
(\S\ref{sec:bloch_sphere}, i.e. $2$-sphere \S\ref{sec:unit_sphere}) is the
\emph{Hopf Fibration}



% ------------------------------------------------------------------------------
\subsection{Eilenberg-MacLane Space}\label{sec:eilenberg_maclane_space}
% ------------------------------------------------------------------------------

an \emph{Eilenberg-MacLane Space} is a Topological Space with a single
Nontrivial Homotopy Group (\S\ref{sec:homotopy_group})

building block for Homotopy Theory

the Configuration Space (\S\ref{sec:configuration_space}) $Conf_n(\reals^2)$
is an Eilenberg-MacLane Space of Type $K(\pi,1)$

for any Graph $\Gamma$, $Conf_n(\Gamma)$ is an Eilenberg-Maclane Space of Type
$K(\pi,1)$ and Strong Deformation Retracts into a Subspace of Dimension
$b(\Gamma)$ where $b(\Gamma)$ is the number of Vertices of Degree at least 3,
and $Uconf_n(\Gamma)$ and $Conf_n(\Gamma)$ Deformation Retract to
Non-positively Curved Cubical Compexes of Dimension at most $min(n,b(\gamma))$



% ------------------------------------------------------------------------------
\subsection{Stable Homotopy Theory}\label{sec:stable_homotopy}
% ------------------------------------------------------------------------------

Operations of Looping and Delooping are Equivalences

Topological Spectrum (Homotopical Algebra \S\ref{sec:topological_spectrum})



% ------------------------------------------------------------------------------
\subsection{Rational Homotopy Theory}\label{sec:rational_homotopy}
% ------------------------------------------------------------------------------

\subsubsection{Rational Space}\label{sec:rational_space}

\subsubsection{Sullivan Model}\label{sec:sullivan_model}



% ------------------------------------------------------------------------------
\subsection{Motivic Homotopy Theory}\label{sec:motivic_homotopy}
% ------------------------------------------------------------------------------

$\mathbb{A}^1$ Homotopy Theory -- Affine Line $\mathbb{A}^1$

\fist Intersection Theory (\S\ref{sec:intersection_theory})

Derived Category (\S\ref{sec:derived_category})

Mixed Motive (\S\ref{sec:mixed_motive})



% ------------------------------------------------------------------------------
\subsection{Synthetic Homotopy Theory}\label{sec:synthetic_homotopy_theory}
% ------------------------------------------------------------------------------

use Types in HoTT to talk about $\infty$-groupoids

\fist Cohesive HoTT (\S\ref{sec:cohesive_hott})

\url{https://www.youtube.com/watch?v=uLLbSAYd3yI} - Dan Licata - A Fibrational
Framework for Modal Simple Type Theories

\emph{Cohesion}

Axiomatic Cohesion (Lawvere)

Differential Cohesion - \url{https://www.youtube.com/watch?v=uEZXHPdwvJU}

Frameworks, Doctrines, Theories, Models

\emph{Doctrine} (2-theories in a Framework) -- Type Constructors/Logical
Connectives; Semantic setting specifies Categorical structure of Models
(2-category where Models are 1-morphisms)

\emph{Theory} (in a Doctrine) -- Signature/Axioms using
Constructors/Connectives; Semantically specifies an ``internal something'' in a
Category with the Doctrine's structure

\emph{Model} (of a Theory) -- Functor from the Free Category to ``that'' (FIXME:
clarify); Syntatically presented as an implementation of the Signature by some
other Types/Terms in a Theory (FIXME: clarify)

Theory of a Monadic Modality in a Doctrine

\emph{Framework} (3-theory) --

Fibrational Frameworks

Mode Theory (\S\ref{sec:mode_theory})

\url{https://www.youtube.com/watch?v=ACGjJDarEc4} - Felix Wellen - \emph{The
  Shape Modality in Real Cohesive HoTT and Covering Spaces}

\emph{Shape Modality} $\int$

Cohesive Topos (\S\ref{sec:cohesive_topos})

Fundamental $\infty$-groupoid of a Space



% ==============================================================================
\section{Algebraic Topology}\label{sec:algebraic_topology}
% ==============================================================================

(wiki): previously \emph{Combinatorial Topology} when Topological Invariants of
Spaces (e.g. Betti Numbers) were regarded as derived from ``\emph{Combinatorial
  Decompositions}'' of Spaces, e.g. as Decomposition into Simplicial Complexes

\emph{Simplicial Approximation Theorem}
(\S\ref{sec:simplicial_approximation_theorem})

``Algebraic Images'' of Topological Spaces via ``Functors'': Images of
Spaces and Images of Maps, Continuous Maps
(\S\ref{sec:continuous_map}) between Spaces
(\S\ref{sec:topological_space}) are ``Projected'' onto Homomorphisms
(\S\ref{sec:homomorphism}) between their Algebraic
Images.\cite{hatcher02}

Bott-Tu1982 - \emph{Differential Forms in Algebraic Topology}

``Equations are summaries of reversible processes, a purely Algebraic concept of
space emerges in which reversible processes can be treated as \emph{Paths}
(Algebraic Topology)''
(John Baez)

\url{http://people.maths.ox.ac.uk/nanda/cat/} - Lecture notes on Computational
Algebraic Topology



% ------------------------------------------------------------------------------
\subsection{Simplicial Approximation Theorem}
\label{sec:simplicial_approximation_theorem}
% ------------------------------------------------------------------------------

(Brouwer)

(wiki): justifies the ``reduction to Combinatorial terms'', after sufficient
subdivision of Simplicial Complexes (\S\ref{sec:simplicial_complex}), of the
``treatment'' of general Continuous Mappings (\S\ref{sec:continuous_map})
--FIXME: clarify

Lebesgue Covering Theorem (TODO: xref)

showed that the Topological ``effect'' on Homology Groups
(\S\ref{sec:homology_group}) of Continuous Maps in a given case could be
expressed in a \emph{Finitary} way



% ------------------------------------------------------------------------------
\subsection{Winding Number}\label{sec:winding_number}
% ------------------------------------------------------------------------------

``Topological Index Theory''

Simple Closed Curve (\S\ref{sec:simple_closed_curve})



\subsubsection{Degree Theory}\label{sec:degree_theory}

generalization of Winding Number to higher dimensions

cf. Fixed-point Theorem (\S\ref{sec:fixedpoint_theorem})

Complementarity Problem (\S\ref{sec:complementarity_theory}) applications



% ------------------------------------------------------------------------------
\subsection{Fundamental Group}\label{sec:fundamental_group}
% ------------------------------------------------------------------------------

Algebraic Image of a Space from the Loops (\S\ref{sec:loop}) in the Space

cf. Algebraic Fundamental Group (\S\ref{sec:algebraic_fundamental_group}) of
Schemes

\fist Fundamental Groupoid (Homotopy Theory \S\ref{sec:fundamental_groupoid})

For a Space $X$, Elements are Loops in $X$ starting and ending at a Base Point
$x_0 \in X$, where two Loops are the same Element if they can be Continuously
Deformed (\S\ref{sec:path_homotopy}) into eachother within the Space $X$.

All Homotopy Classes (\S\ref{sec:path_homotopy}) $[f]$ of Loops $f : I
\rightarrow X$ at Basepoint $x_0$:
\[
  \pi_1(X,x_0)
\]
with Group Operation as the Path Product (\S\ref{sec:path_product}):
\[
  [f][g] = [f \cdot g]
\]

Homeomorphic Spaces have Isomorphic Fundamental Groups. Every Group $G$ can be
realized as the Fundamental Group of some Space $X_G$. \cite{hatcher02}

For $X$ the Complement of a Circle $A$ the Fundamental Group is the Infinite
Cyclic Group (\S\ref{sec:infinite_cyclic}) with one Generator: $\pi_1(S^1) \cong
\ints$

For the Complement of two Unlinked Circles $A$ and $B$, the Fundamental Group is
the Non-abelian (\S\ref{sec:noncommutative_group}) Free Group
(\S\ref{sec:free_group}) on two Generators.

For the Complement of two Linked Circles $A$ and $B$, the Fundamental Group is
the Abelian (\S\ref{sec:commutative_group}) Free Group on two Generators.
\cite{hatcher02} This generalizes to the Fundamental Group of a Rose
(\S\ref{sec:rose}) of $k$ Circles being equal to the Free Group on a Set of $k$
Elements.

$n$-th Homotopy Group (\S\ref{sec:homotopy_group}) $\pi_n(X,x_0)$

For Continuous Map $f : X \rightarrow Y$, there is a Group Homomorphism $f_* :
\pi (X) \rightarrow \pi (Y)$


\emph{Riemann-Hilbert Correspondence for Regular Singular Connections} -- TODO

\asterism

\url{https://math.stackexchange.com/a/1770950}:

$\pi_1(SO(2), x) \simeq \ints$

$\pi_1(SO(n), x) \simeq \ints_2$ for $n > 2$



% ------------------------------------------------------------------------------
\subsection{Homogeneous Space}\label{sec:homogeneous_space}
% ------------------------------------------------------------------------------

A \emph{Homogeneous Space} for a Group $G$ is a Topological Space $X$ on which
$G$ Acts Transitively (\S\ref{sec:transitive_action}); the Elements of $G$ are
called the \emph{Symmetries} of the Space $X$.

Homogeneous Spaces are by definition endowed with a Transitive Group Action

Symmetry Group (\S\ref{sec:symmetry_group})

Special cases when $G$ is the Automorphism Group
(\S\ref{sec:automorphism_group}) of $X$:
\begin{itemize}
  \item Homeomorphism Group (\S\ref{sec:homeomorphism}) -- Topology
  \item Isometry Group (\S\ref{sec:isometry_group}) -- ``Rigid'' Geometry
  \item Diffeomorphism Group (\S\ref{sec:diffeomorphism}) -- Differential
    Geometry
\end{itemize}

Analytic Geometry (Part \ref{part:analytic_geometry}): Euclidean Space
(\S\ref{sec:euclidean_space}), Affine Space (\S\ref{sec:affine_space}), and
Projective Space (\S\ref{sec:projective_space}) are Homogeneous Spaces for their
respective Symmetry Groups (\S\ref{sec:symmetry_group})

Topological Groups (\S\ref{sec:topological_group})



\subsubsection{Principal Homogeneous Space}
\label{sec:principal_homogeneous_space}

or \emph{Torsor}

the Transitive Group Action of a Principal Homogeneous Space is Free

an Affine Space (\S\ref{sec:affine_space}) is a Principal Homogeneous Space for
its Symmetry Groups (\S\ref{sec:symmetry_group}), i.e. for the Action of the
Additive Group of a Vector Space



\subsubsection{Coset Space}\label{sec:coset_space}



% ------------------------------------------------------------------------------
\subsection{Triangulation}\label{sec:triangulation}
% ------------------------------------------------------------------------------

Barycentric Subdivision



% ------------------------------------------------------------------------------
\subsection{Cell}\label{sec:topology_cell}
% ------------------------------------------------------------------------------

An $n$-dimensional Closed Cell is the Image of an $n$-dimensional
Closed Ball (\S\ref{sec:ball}) under an Attaching Map
(\S\ref{sec:attaching_map})



% ------------------------------------------------------------------------------
\subsection{CW Complex}\label{sec:cw_complex}
% ------------------------------------------------------------------------------

\emph{CW Complex} (\emph{Closure-finite Weak Topology})

A CW Complex is a Cell Complex (\S\ref{sec:cell_complex}) in $\cat{Top}$ with
respect to Generating Cofibrations in the Standard Model Structure
(\S\ref{sec:model_structure}) on Topological Spaces.

The Geometric Realization (\S\ref{sec:geometric_realization}) of any Simplicial
Set (\S\ref{sec:simplicial_set}) or $n$-groupoid (\S\ref{sec:n_groupoid}), etc.
is a CW Complex.

Point: $0$-cell

Open Arc: $1$-cell

Open Disc: $2$-cell

$n$-skeleton

$n$-disc

$n$-sphere $S^n$: Two Cells: $e^0$ and $e^n$, attached by Constant Map
$S^{n-1} \rightarrow e^0$ (equivalently $S^n$ regarded as Quotient
Space (\S\ref{sec:quotient_space}) $D^n/ \partial D^n$

Weak Topology (\S\ref{sec:weak_topology})

Topological Space $X$ constructed Inductively: \cite{hatcher02}
\begin{enumerate}
  \item Discrete Set $X^0$ of $0$-cells of $X$
  \item (Inductive Step) form $n$-skeleton $X^n$ from $X^{n-1}$ by
    attaching $n$-cells $e^n_\alpha$ with Maps $\varphi_\alpha :
    S^{n-1} \rightarrow X^{n-1}$: therefore $X^n$ is the Quotient
    Space (\S\ref{sec:quotient_space}) of the Disjoint Union of
    $X^{n-1}$ with a collection of $n$-discs $D^n_\alpha$ under
    identifications $x \sim \varphi_\alpha(x)$ for $x \in \partial
    D^n_\alpha$:
    \[
      X^{n-1}\amalg_\alpha D^n_\alpha
    \]
    and:
    \[
      X^n = X^{n-1}\amalg_\alpha e^n_\alpha
    \]
    where each $e^n_\alpha$ is an Open $n$-disc.
\end{enumerate}

1-dimensional Cell Complex $X = X^1$ is a \emph{Graph}

Product (\S\ref{sec:cwcomplex_product})

Quotient (\S\ref{sec:cwcomplex_quotient}): Suspension
(\S\ref{sec:suspension}), Join (\S\ref{sec:join}), Wedge Sum
(\S\ref{sec:wedge_sum}), Smash Product (\S\ref{sec:smash_product})



\subsubsection{Subcomplex}\label{sec:subcomplex}

Collapsing (\S\ref{sec:collapsible_cwcomplex}) a Contractible
(\S\ref{sec:contractible_cwcomplex}) Subcomplex is a Homotopy
Equivalence (\S\ref{sec:homotopy_equivalence})



\paragraph{CW Pair}\label{sec:cw_pair}\hfill

\emph{CW Pair}



\subsubsection{CW Complex Product}\label{sec:cwcomplex_product}

Product



\subsubsection{CW Complex Quotient}\label{sec:cwcomplex_quotient}

Quotient - CW Pair (\S\ref{sec:cw_pair})



\paragraph{Suspension}\label{sec:suspension}\hfill

\subparagraph{Reduced Suspension}\label{sec:reduced_suspension}\hfill



\paragraph{Join}\label{sec:join}\hfill

Simplex %FIXME hatcher02



\paragraph{Smash Product}\label{sec:smash_product}\hfill



\subsubsection{$k$-chain}\label{sec:k_chain}

in Simplicial Complexes (\S\ref{sec:simplicial_complex}), $k$-chains are
combinations of $k$-simplices

in Cubical Complexes (\S\ref{sec:cubical_complex}), $k$-chains are
combinations of $k$-cubes

Elements of a Homology Group (\S\ref{sec:homology_group}) are Equivalence
Classes of Chains

generally a Differential $k$-form (\S\ref{sec:differential_form}) can be
Integrated (\S\ref{sec:integral}) over $k$-dimensional Chains; for $k=0$ this is
just Evaluation of a Function at Points



\subsubsection{Contractible CW Complex}
\label{sec:contractible_cwcomplex}

Contractible Space (\S\ref{sec:contractible_space})



\subsubsection{Collapsible CW Complex}
\label{sec:collapsible_cwcomplex}

Collapse (\S\ref{sec:collapse})



\paragraph{Collapse}\label{sec:collapse}\hfill



\subsubsection{Simplicial Complex}\label{sec:simplicial_complex}

a Set composed of $n$-dimensional Simplices (\S\ref{sec:simplex})

\fist cf. Simplicial Set (\S\ref{sec:simplicial_set}) -- a Contravariant Functor
from the Simplex Category fo the Category of Sets

Chain Complex (\S\ref{sec:chain_complex})

Simplicial Homology (\S\ref{sec:simplicial_homology})

$\mathbb{F}_1$-geometry (\S\ref{sec:f1_geometry}) -- analogy between Symmetries
in Projective Geometry (\S\ref{sec:projective_geometry}) and Combinatorics of
Simplicial Complexes (Tits 1957)

\emph{Simplicial Approximation Theorem} (Brouwer
\S\ref{sec:simplicial_approximation_theorem})



\paragraph{Abstract Simplical Complex}\label{sec:abstract_complex}\hfill

Combinatorial description of a Simplicial Complex, consisting of a Family of
Non-empty Finite Sets Closed under the Operation of taking Non-empty Subsets

\fist \emph{Independence Systems} (Matroid Theory
\S\ref{sec:independence_system})



\subsubsection{Cubical Complex}\label{sec:cubical_complex}



% ------------------------------------------------------------------------------
\subsection{Retraction}\label{sec:subspace_retraction}
% ------------------------------------------------------------------------------

For a Topological Space $(X, \tau)$ and Subspace $A$ of $(X,\tau)$, a
Continuous Map (\S\ref{sec:continuous_map}) $r : X \rightarrow A$ is a
\emph{Retraction} if the Restriction of $r$ to $A$ is the Identity Map
on $A$.

cf. Projection (\S\ref{sec:projection})

$\iota : A \hookrightarrow X$ (\S\ref{sec:inclusion_map})

$r \circ \iota = id_A$

Topological analog to Projection Operators % FIXME xref
\cite{hatcher02}



\subsubsection{Deformation Retraction}\label{sec:deformation_retraction}

Continuous Map $f : X \times [0,1] \rightarrow X$

Deformation Retraction $f_t : X \rightarrow X$ is a Special case of
Homotopy (\S\ref{sec:homotopy}): a Deformation Retraction of $X$ onto
Subspace $A$ is a Homotopy from $id_X$ to a Retraction of $X$ to $A$,
$r : X \rightarrow X$, such that $r(X) = A$ and $r | A = id_A$.
\cite{hatcher02}



% ------------------------------------------------------------------------------
\subsection{Homotopy Group}\label{sec:homotopy_group}
% ------------------------------------------------------------------------------

Fundamental Group (\S\ref{sec:fundamental_group}) $\pi_1(X,x_0)$

$n$-th Homotopy Group $\pi_n(X,x_0)$

$I^n$

an \emph{Eilenberg-MacLane Space} (\S\ref{sec:eilenberg_maclane_space}) is a
Topological Space with a single Nontrivial Homotopy Group



\subsubsection{Aspherical Space}\label{sec:aspherical_space}

a Topological Space with all Homotopy Groups $\pi_n(X)$ equal to $0$ when $n >
1$



% ------------------------------------------------------------------------------
\subsection{Homology Theory}\label{sec:homology_theory}
% ------------------------------------------------------------------------------

\fist Homological Algebra (\S\ref{sec:homological_algebra}) -- Homology in the
``abstract setting''

cf. Bordism (Equivalence Class of Cobordisms \S\ref{sec:cobordism})

(wiki):

a particular type of mathematical object such as a Topological Space or a
Group may have one ore more associated ``\emph{Homology Theories}''

when the underlying object has a ``geometric'' interpretation like Topological
Spaces, then the \emph{$n$th Homology Group} (\S\ref{sec:homology_group})
represents the ``behavior'' unique to Dimension $n$

2021 - Kahle -
\emph{Configuration spaces of particles: homological solid, liquid, and gas}
-- \url{https://www.youtube.com/watch?v=fwOMI6lTJ0Y}



\subsubsection{Euler Characteristic}\label{sec:euler_characteristic}

(wiki):

The \emph{Euler Characteristic} is a Topological Invariant describing the
``shape'' or ``structure'' of a Topological Space regardless of the way that it
is ``bent''

generalization of Cardinality (\S\ref{sec:cardinality}) to admit
Negative Integer values

\fist See also Homotopy Cardinality (\S\ref{sec:homotopy_cardinality})
for Cardinality generalized to admit Positive Real values

by the Classification Theorem for Closed Surfaces, a Closed Surface is
determined up to Homeomorphism by its Euler Characterstic and whether or not it
is Orientable

$\chi = k_0 - k_1 + k_2 - k_3 + \ldots$ where $k_n$ denotes the number
of Cells of Dimension $n$ in the Complex.

Betti Number (\S\ref{sec:betti_number})

depends only on the Homotopy Type (\S\ref{sec:homotopy_type}) of a
CW Complex \cite{hatcher02}

Dyck's Surface $P \# P \# P$



\paragraph{Euler-Schanuel Characteristic}\label{sec:euler_schanuel}\hfill

(open/closed intervals) %FIXME baez counting.pdf

$\chi(X) = rank (H^0(X)) - rank (H^1(X)) + rank (H^2(X)) - \ldots$

agrees with ordinary Euler Characteristic on Compact Spaces
(\S\ref{sec:compact_space})

in general is defined using Compactly Supported Cohomology and is
defined for Cohomologically Finite Spaces (i.e. those for which the
Sum converges) %FIXME

$\chi(\reals) = -1$

$\chi(\comps) = 1$



\paragraph{Characteristic Map}\label{sec:characteristic_map}\hfill

%FIXME



\subsubsection{Homology}\label{sec:homology}

(wiki):

a Chain Complex (\S\ref{sec:chain_complex}) is Algebraic Structure consisting
of a Sequence of Abelian Groups or Modules and a Sequence of Homomorphisms
between consecutive Groups (Modules) such that the Image of each Homomorphism
is included in the Kernel of the next

\emph{Homology} describes how these Images are included in Kernels

the Singular Chain Complex (\S\ref{sec:singular_chain_complex}) of a
Topological Space $X$ is constructed using Continuous Maps from a Simplex
(\S\ref{sec:simplex}) to $X$, and the Homomorphisms of the Chain Complex
capture how the maps ``restrict to the boundary of the Simplex''; the Homology
of this Chain Complex (FIXME: clarify)



\paragraph{Simplicial Homology}\label{sec:simplicial_homology}\hfill

the Chain Complex (\S\ref{sec:chain_complex}) defining the Simplicial Homology
of a Finite Simplicial Complex is an example of a Bounded Chain Complex



\paragraph{Homology Group}\label{sec:homology_group}\hfill

$H_n(X)$ \emph{Homology Groups} of Topological Space $X$

if two Topological Spaces $X$ and $Y$ are the same Homotopy Type
(\S\ref{sec:homotopy_type}) then:
\[
  H_n(X) = H_n(Y)
\]
so Homology Groups are Topological Invariants

in Category Theory the Homology Group becomes a Functor from the Category of
Topological Spaces to the Category of Graded Abelian Groups

(wiki):

a particular type of mathematical object such as a Topological Space or a
Group may have one ore more associated ``\emph{Homology Theories}''

when the underlying object has a ``geometric'' interpretation like Topological
Spaces, then the \emph{$n$th Homology Group} (\S\ref{sec:homology_group})
represents the ``behavior'' unique to Dimension $n$

Elements of a Homology Group are Equivalence
Classes of Chains (\S\ref{sec:k_chain})

Homology Groups are a ``Functorial'' Homotopy Invariant (\S\ref{sec:homotopy})
--FIXME: clarify



\paragraph{Singular Homology}\label{sec:singular_homology}\hfill

(wiki):

counts for each Dimension $n$ the number of $n$-dimensional ``holes'' of a
Space

constructed by taking Maps of the standard $n$-simplex (\S\ref{sec:simplex}) to
a Topological Space and composing them into Formal Sums
(\S\ref{sec:formal_sum}) called Singular Chains and the Boundary Operation
(\S\ref{sec:boundary_operator}) maps each $n$-dimensional Simplex to its
$(n-1)$-dimensional Boundary inducing the \emph{Singular Chain Complex}
(\S\ref{sec:singular_chain_complex})

the Singular Homology is then the Homology of the Chain Complex and the
resulting Homology Groups are the same for all Homotopically Equivalent Spaces
(\S\ref{sec:homotopy_equivalence})



\paragraph{Intersection Homology}\label{sec:intersection_homology}\hfill

analogue of Singular Homology

\fist cf. Intersection Theory (\S\ref{sec:intersection_theory})



\paragraph{Cyclic Homology}\label{sec:cyclic_homology}\hfill

Noncommutative Geometry (\S\ref{sec:noncommutative_geometry})

$\mathbb{F}_1$-geometry (\S\ref{sec:f1_geometry})



\paragraph{Khovanov Homology}\label{sec:khovanov_homology}\hfill

\fist Knot Polynomials (\S\ref{sec:knot_polynomial})



\subsubsection{Cohomology}\label{sec:cohomology}

Cochain Complex (\S\ref{sec:cochain_complex})

Topological Spectrum (\S\ref{sec:topological_spectrum}) --
represents a Generalized Cohomology Theory



\paragraph{Cohomology Theory}\label{sec:cohomology_theory}\hfill

Sheaf Cohomology (\S\ref{sec:sheaf_cohomology})



\subparagraph{Cohomology Group}\label{sec:cohomology_group}\hfill

\fist Hodge Theory (\S\ref{sec:hodge_theory}) uses Partial Differential
Equations (\S\ref{sec:pde}) to study the Cohomology Groups of Smooth Manifolds
(\S\ref{sec:smooth_manifold})



\subparagraph{Cohomology Ring}\label{sec:cohomology_ring}\hfill

to any Topological Space $X$ may be associated its Integral Cohomology Ring
formed by the Cohomology Grups of $X$ together with the Cup product serving as
Ring Multiplication

Ring structure in Cohomology provides foundation for Characteristic Classes of
Fiber Bundles, Intersection Theory on Manifolds and Algebraic Varieties,
Schubert Calculus (FIXME: xrefs)



\subparagraph{Elliptic Cohomology}\label{sec:elliptic_cohomology}\hfill

\subparagraph{Generalized Cohomology Theory}
\label{sec:generalized_cohomology_theory}\hfill

Topological Spectrum (\S\ref{sec:topological_spectrum})



\subsubsection{Hochschild Homology}\label{sec:hochschild_homology}

End (\S\ref{sec:end})



\paragraph{Magnitude Homology}\label{sec:magnitude_homology}\hfill

\url{https://golem.ph.utexas.edu/category/2017/11/magnitude_homology_is_hochschi.html}

\url{https://golem.ph.utexas.edu/category/2018/03/magnitude_homology_reading_sem.html}



% ------------------------------------------------------------------------------
\subsection{Sheaf Theory}\label{sec:sheaf_theory}
% ------------------------------------------------------------------------------

\fist Algebraic Analysis (\S\ref{sec:algebraic_analysis}) -- study of Systems of
Linear PDEs (\S\ref{sec:linear_pde_system}) using Sheaf Theory and Complex
Analysis (\S\ref{sec:complex_analysis})

\fist $D$-modules (\S\ref{sec:d_module})

\url{https://julesh.com/2018/02/27/the-rising-sea-in-applied-mathematics/}:

``to study a Differential Equation, you study a Topos of Sheaves of
Differentiable Functions on its Solution Manifold''



\subsubsection{Presheaf}\label{sec:presheaf}

\url{https://johncarlosbaez.wordpress.com/2020/01/05/topos-theory-part-1/}:

the Open Sets $\tau$ of a Topological Space $(X, \tau)$ is a Poset with
Inclusion as the Partial Ordering, and therefore is also a Category

a \emph{Presheaf} is a Contravariant Functor $F : \tau^{op} \rightarrow
\cat{Set}$ assigning to each Open Set $U$ a Set $F U$, and being a Contravariant
allows \emph{Restriction} of Elements of $F U$ to \emph{smaller} Open Sets

Presheaf Category (\S\ref{sec:presheaf_category})

\begin{itemize}
  \item Combinatorial Species (\S\ref{sec:combinatorial_species})
\end{itemize}



\subsubsection{Sheaf}\label{sec:sheaf}

a ``tool'' for systematically tracking locally defined ``\emph{Data}'' attached
to the Open Sets of a Topological Space

\emph{Locality}, \emph{Gluing}

cf. Fiber Bundle (\S\ref{sec:fiber_bundle})

\url{https://johncarlosbaez.wordpress.com/2020/01/05/topos-theory-part-1/}

(Algebraic Geometry, Part \ref{part:algebraic_geometry}): associate to a Space
the Commutative Ring of Functions on that Space; e.g. for $X$ a Compact
Hausdorff Space the Ring $C(X)$ of all Continuous Real-valued Functions on $X$
can be used to ``recover'' $X$ from the Ring

for situations where there aren't enough ``everywhere-defined'' Functions of the
kind being considered on a Space, instead consider Functions defined on
\emph{Open Sets}

\url{https://johncarlosbaez.wordpress.com/2020/01/07/topos-theory-part-2/}:
every Sheaf comes from a Bundle (\S\ref{sec:fiber_bundle}), and every Sheaf (or
Presheaf) gives rise to a Bundle



\paragraph{Structure Sheaf}\label{sec:structure_sheaf}\hfill

a Pair $(X,\mathcal{O}_X)$ of a Topological Space $X$ and a Structure Sheaf
$\mathcal{O}_X$ of Rings on $X$ is a \emph{Ringed Space}
(\S\ref{sec:ringed_space});
when the ``Stalks'' (FIXME: ???) of the Structure Sheaf are Local Rings
(\S\ref{sec:local_ring}) then $(X,\mathcal{O}_X)$ is a \emph{Locally Ringed
  Space} (\S\ref{sec:locally_ringed_space})



\paragraph{Perverse Sheaf}\label{sec:perverse_sheaf}\hfill



\subsubsection{Ringed Space}\label{sec:ringed_space}

a Sheaf of Rings

a Pair $(X,\mathcal{O}_X)$ of a Topological Space $X$ and a Structure Sheaf
(\S\ref{sec:structure_sheaf}) $\mathcal{O}_X$ of Rings on $X$

a Topological Space with a collection of Commutative Rings
(\S\ref{sec:commutative_ring}) with elements as Functions on each Open Set of
the Space; equivalently a Family of Commutative Rings parameterized by Open
Subsets of a Topological Space

a \emph{Scheme} (\S\ref{sec:scheme}) is a Ringed Space that is locally a
Spectrum (\S\ref{sec:spectrum}) of a Commutative Ring



\paragraph{Locally Ringed Space}\label{sec:locally_ringed_space}\hfill

when the ``Stalks'' (FIXME: ???) of the Structure Sheaf $\mathcal{O}_X$ are
Local Rings (\S\ref{sec:local_ring}) then $(X,\mathcal{O}_X)$ is a Locally
Ringed Space

\fist Schemes (\S\ref{sec:scheme}) are Locally Ringed Spaces obtained by
``gluing together'' Spectra (\S\ref{sec:ring_spectrum}) of Commutative Rings

\fist \emph{Affine Scheme} (\S\ref{sec:affine_scheme}) -- a Spectrum augmented
with the Zariski Topology (\S\ref{sec:zariski_topology}) and a Structure Sheaf
(\S\ref{sec:structure_sheaf}), turning it into a Locally Ringed Space

a Scheme is a Locally Ringed Space $X$ admitting a Covering by Open Sets $U_i$
such that each $U_i$ (as a Locally Ringed Space) is an Affine Scheme

an Affine Scheme is a Locally Ringed Space that is Isomorphic to the Spectrum
$Spec(R)$ of a Commutative Ring $R$



\subsubsection{Sheaf Cohomology}\label{sec:sheaf_cohomology}

\fist Cohomology Theory (\S\ref{sec:cohomology_theory})



\paragraph{Local System}\label{sec:local_system}\hfill

Local Coefficients

half-way stage between Homology Theory or Cohomology Theory and general Sheaf
Cohomology



% ==============================================================================
\section{Geometric Topology}\label{sec:geometric_topology}
% ==============================================================================

\emph{Geometry} $(X,G)$: Simply-connected Space $X$, Transitive Group $G$,

\textbf{Low-dimensional Topology} -- $4$ or fewer Dimensions:
\begin{itemize}
  \item Knot Theory (\S\ref{sec:knot_theory})
\end{itemize}



% ------------------------------------------------------------------------------
\subsection{Locally Euclidean Topological Space}\label{sec:locally_euclidean}
% ------------------------------------------------------------------------------

every Point has a Neighborhood that is Homeomorphic to Real $n$-space $\reals^n$

every Locally Euclidean Space is Accessible ($T_0$ or Kolmogorov
\S\ref{sec:accessible_space})

although Euclidean Space is Separated, the Hausdorff Property is not Local, so a
Locally Euclidean Space is not necessarily Separated



% ------------------------------------------------------------------------------
\subsection{Topological Manifold}\label{sec:topological_manifold}
% ------------------------------------------------------------------------------

a \emph{Topological Manifold} is a Locally Euclidean Separated Space ($T_2$ or
Hausdorff Space \S\ref{sec:separated_space})

every Manifold (\S\ref{sec:manifold}) has an ``underlying'' Topological Manifold
obtained by ``forgetting'' any additional structure the Manifold has

%FIXME: merge with sec:manifold ???

a Presentation of a Topological Manifold is a Second-countable Hausdorff Space
that is Locally Homeomorphic to a Linear Space by a collection called an
\emph{Atlas} (\S\ref{sec:atlas}) of Homeomorphisms called \emph{Charts}
(\S\ref{sec:chart}); and a Topological Manifold is such a Space with an
Equivalence Class of Atlases.

(FIXME: xref)

\fist any Manifold without a Boundary is \emph{trivially} an Orbifold
(\S\ref{sec:orbifold}).

the Configuration Space (\S\ref{sec:configuration_space}) of Distinct,
Un-ordered Points of a Manifold is also a Manifold

\fist Lie Groups (\S\ref{sec:lie_group}): Manifolds with Group Structure



\subsubsection{Orientable Manifold}\label{sec:orientable_manifold}

\fist Orientable Surface (\S\ref{sec:orientable_surface})



% ------------------------------------------------------------------------------
\subsection{Topologically Stratified Space}\label{sec:topologically_stratified}
% ------------------------------------------------------------------------------

Strata -- Topological Manifolds that are required to ``fit together'' in a
certain way

TODO



% ------------------------------------------------------------------------------
\subsection{Coordinate Chart}\label{sec:chart}
% ------------------------------------------------------------------------------

A \emph{Chart} (or \emph{Coordinate Chart}, \emph{Coordinate Patch},
\emph{Coordinate Map}, or \emph{Local Frame}), $(U, \varphi)$ of a Topological
Space $M$ is a Homeomorphism (\S\ref{sec:homeomorphism}) $\varphi$ from an Open
Subset $U$ of $M$ to an Open Subset of some ``simple Space''.

An collection of Charts which Covers a Manifold is called an \emph{Atlas}
(\S\ref{sec:atlas}).

Open Subsets of Real Affine Spaces (\S\ref{sec:affine_space})

Manifold built by ``gluing together'' Charts

cf. Algebraic Varieties (\S\ref{sec:algebraic_variety}) built by ``gluing
together'' Affine Varities (\S\ref{sec:affine_variety})

\fist cf. Reference Frame (\S\ref{sec:reference_frame})



% ------------------------------------------------------------------------------
\subsection{Atlas}\label{sec:atlas}
% ------------------------------------------------------------------------------

An \emph{Atlas} is a Set of Charts which ``Covers'' (\S\ref{sec:cover}) a
Manifold.

\fist cf. Local Coordinates (\S\ref{sec:local_coordinate})

(wiki):

For a general Topological Space $M$, an Atlas is a collection $\{(U_\alpha,
\varphi_\alpha) \|\ \alpha \in A \}$ of Charts on $M$ Indexed
(\S\ref{sec:index_set}) by a Countable Set $A$ such that
$\bigcup_{\alpha\in{A}} U_\alpha = M$.

If the Codomain of each Chart is the $n$-dimensional Euclidean Space, then $M$
is said to be an \emph{$n$-dimensional Manifold} (\S\ref{sec:manifold}).

\emph{Boundary Conditions} -- forms a Submanifold ???



\subsubsection{Transition Map}\label{sec:transition_map}

\subsubsection{Smooth Atlas}\label{sec:smooth_atlas}

a \emph{Smooth Atlas} for a Topological



\subsubsection{Smooth Structure}\label{sec:smooth_structure}

a \emph{Smooth Structure} on a Manifold $M$ is a Collection of \emph{Smoothly
  Equivalent Smooth Atlases}

(ncat):

A canonical Smooth Structure on the $n$-dimensional Cartesian Space
(\S\ref{sec:cartesian_space}) $\reals^n$ makes it a Smooth Manifold
(\S\ref{sec:smooth_manifold}). For all $n$, the Open $n$-ball with standard
Smooth Structure is Diffeomorphic to the Cartesian Space $\reals^n$ with its
standard Smooth Structure.

Thm. \emph{For $n \in \nats$ with $n \neq 4$, there is a Unique (up to
  Isomorphism) Smooth Structure on Cartesian Space $\reals^n$}.

Thm. \emph{On $\reals^4$ there exist Exotic Smooth Structures}.



\paragraph{Exotic Smooth Structure}\label{sec:exotic_smooth_structure}\hfill

$\reals^4$



% ------------------------------------------------------------------------------
\subsection{Manifold}\label{sec:manifold}
% ------------------------------------------------------------------------------

%FIXME: "manifold" is any topological manifold possibly with additional
% structure

Manifold built by ``gluing together'' Coordinate Charts (\S\ref{sec:chart}, Open
Subsets of Affine Spaces \S\ref{sec:affine_space}).

every Manifold has an ``underlying'' Topological Manifold
(\S\ref{sec:topological_manifold}) obtained by ``forgetting'' any additional
structure the Manifold has

(Mathview - \emph{Manifolds} -
\url{https://www.youtube.com/watch?v=jvotAfSUxWU}) -- qualitative definition: a
Manifold $M$ is a Topological Space with Points labelled by \emph{Coordinates}
(\S\ref{sec:coordinate_system}); Points of $M$ are not Vectors in the
traditional sense--Manifolds may have \emph{Curvature}
(\S\ref{sec:curvature})--but Points in $M$ are Vectors in their Epsilon
Neighborhoods.

cf. Algebraic Varieties (\S\ref{sec:algebraic_variety}) built by ``gluing
together'' Affine Varities (\S\ref{sec:affine_variety})

\fist a \emph{Smooth Scheme} (\S\ref{sec:scheme}) over a Field is well
approximated by Affine Space near any Point (cf. Manifolds)

\fist cf. \emph{Schemes} (\S\ref{sec:scheme} -- (Eisenbud-Harris 2000):

as Manifolds are made by ``gluing together'' Open Balls from Euclidean Space,
\emph{Schemes} are made by ``gluing together'' Open Sets called \emph{Affine
  Schemes} (\S\ref{sec:affine_scheme}); unlike Manifolds the Open Neighborhoods
of Points in Schemes are not necessarily Isomorphic

(wiki):

Manifolds do not need to be Connected (\S\ref{sec:connected_space}) or Closed
(\S\ref{sec:closed_manifold})

a Manifold is never Countable unless the Dimension of the Manifold is $0$

a Vector Field (\S\ref{sec:vector_field}) attaches to every Point of a Manifold
a Vector from the Tangent Space (\S\ref{sec:tangent_space}) at that Point

families of Manifolds:
\begin{itemize}
  \item Real Coordinate Spaces (\S\ref{sec:real_coordinate_space})
  \item $n$-spheres (\S\ref{sec:n_sphere})
  \item $n$-tori (\S\ref{sec:n_torus})
  \item Real Projective Spaces (\S\ref{sec:real_projective_space})
  \item Complex Projective Spaces (\S\ref{sec:complex_projective_space})
  \item Quaternionic Projective Spaces
  \item Grasmann Manifolds
  \item Flag Manifolds
  \item Stiefel Manifolds
  \item Lie Groups
  \item ... MORE?
\end{itemize}
%FIXME xrefs

$n$-disc ??? %FIXME

1-dimensional (Knot Theory \S\ref{sec:knot_theory}) -- up to Homeomorphism,
there are two Connected $1$-manifolds without Boundary (\S\ref{sec:boundary}):

\begin{itemize}
  \item Unit Interval, $(0,1)$
  \item Unit Circle, $\mathbb{R}/\mathbb{Z}$
\end{itemize}

2-dimensional (Suraces \S\ref{sec:surface}):

\begin{description}
  \item [Spherical]: 2-sphere ($\mathsf{S}^2$)
  \item [Euclidean]: Torus ($\mathsf{T}^2$),
  $\mathbb{R}^2/\mathbb{Z}^2$
  \item [Hyperbolic]: $\mathsf{H}^2$
\end{description}

3-dimensional (3-manifolds \S\ref{sec:three_manifold}):

\begin{description}
  \item [Spherical]: 3-sphere ($\mathsf{S}^3$)
  \item [Euclidean]: 3-Torus ($\mathsf{T}^3$)
  \item [Hyperbolic]: Siefert-Weber Dodecahedral Space, Gieseking
  Manifold
  \item [$S^2 \times E^1$]
  \item [$S^2 \times S^1$]
  \item [$H^2 \times E^1$]: $\mathsf{S}^3 - 3_1$
  \item [$Twisted H^2 \times E^1$]
  \item [Nilgeometry]
  \item [Solvegeometry]
\end{description}

4-manifolds (\S\ref{sec:four_manifold})

\asterism

2014 - Olah - \emph{Neural Networks, Manifolds, and Topology} -
\url{http://colah.github.io/posts/2014-03-NN-Manifolds-Topology/} --

Deep Neural Networks (DNNs \S\ref{sec:dnn})

\emph{Manifold Hypothesis}



\subsubsection{Submanifold}\label{sec:submanifold}

$S \subset M$

Inclusion Map $S \rightarrow M$

Ambient Manifold

cf. Algebraic Variety (\S\ref{sec:algebraic_variety})



\paragraph{Locally Flat}\label{sec:locally_flat}\hfill

Property of a Submanifold in a Topological Manifold of larger Dimension

for Algebraic Varieties (\S\ref{sec:algebraic_variety}) defined over the Reals,
Singular Points (\S\ref{sec:singular_point}) are a generalization of Local
Non-flatness



\paragraph{Immersed Submanifold}\label{sec:immersed_submanifold}\hfill

Image $S$ of an Immersion Map $f : N \rightarrow M$; need not be a Submanifold
as a Subset and Immersion Map does not need to be Injective, i.e. allows for
Self-intersections

\emph{Injective Immersed Submanifold}: Injective Immersion where the Image
Subset $S$ has a Topology and Differential Structure such that $S$ is a
Manifold and the Inclusion $f : N \rightarrow M$ is an Diffeomorphism
(\S\ref{sec:diffeomorphism})

\fist Lie Subgroups (\S\ref{sec:lie_subgroup}) are naturally Immersed
Submanifolds



\subparagraph{Embedded Submanifold}\label{sec:embedded_submanifold}\hfill

An \emph{Embedded Submanifold} (or \emph{Regular Submanifold}) is an Immersed
Submanifold for which the Inclusion Map is a Topological Embedding
(\S\ref{sec:topological_embedding}), i.e. the Image of the Inclusion Map
has the Subspace Topology (\S\ref{sec:subspace_topology}) for the Submanifold
Topology.



\subsubsection{Oriented Manifold}\label{sec:oriented_manifold}

Oriented $n$-manifolds form a Commutative
Semigroup (\S\ref{sec:semigroup}) with the Connected Sum
(\S\ref{sec:connected_sum}) as the Semigroup Operation and the $n$-sphere as
the Identity Element.

cf. Orientable Surfaces (\S\ref{sec:orientable_surface})

\fist Currents (\S\ref{sec:current}) -- generalization of concept of Oriented
Manifolds possibly with Boundary



\paragraph{Simplicial Volume}\label{sec:simplicial_volume}\hfill

\emph{Simplicial Volume} (or \emph{Gromov Norm})

The Set of Volumes of Hyperbolic 3-manifolds is a Closed, Well-ordered
Subset of $\mathbb{R}$ of Order Type $\omega^\omega$. For each Volume
the Set of 3-manifolds with that Volume is Finite.



\subsubsection{Geometrization Conjecture}
\label{sec:geometrization_conjecture}

\emph{Thurston's Geometrization Conjecture}: every Closed 3-manifold can be
decomposed in a canonical way into pieces that each have one of eight types of
Geometric structure (Model Geometries):
\begin{enumerate}
  \item $S^3$ -- Spherical
  \item $E^3$ -- Euclidean
  \item $H^3$ -- Hyperbolic
  \item $S^2 \times \reals$
  \item $H^2 \times \reals$
  \item Universal Cover of $SL(2, \reals)$ (Group of $2 \times 2$ Real Matrices
    with Determinant $1$)
  \item Geometry of the Heisenberg Group (TODO: xref;
    One-dimensional Quantum Mechanical Systems)) of $3 \times 3$ Upper
    Triangular Matrices with $1$s along the Diagonal
    \fist Nilmanifold (TODO)
  \item Geometry of the Identity Component of the Group of Maps from
    $2$-dimensional Minkowski Space to itself that are either Isometries or
    multiply the Metric by $-1$
    \fist Solvmanifold (TODO)
\end{enumerate}

analog of the Uniformization Theorem for Two-dimensional Surfaces that states
every Simply-connected Riemann Surface can be given one of either Euclidean,
Spherical, or Hyperbolic Geometries; every Surface without Boundary has a
Geometric structure consisting of a \emph{Metric} with Constant Curvature



\subsubsection{Connected Sum}\label{sec:connected_sum}

$M_1 \# M_2$

Semigroup Operation for the Semigroup of Oriented $n$-manifolds
(\S\ref{sec:oriented_manifold}) with the $n$-sphere as the Identity Element.

Reversible in 3 Dimensions

Prime 3-manifold

\emph{Prime Decomposition Theorem for 3-manifolds} (Kneser): Any Compact,
Orientable 3-manifold is the Connected Sum of a Unique (up to Homeomorphism)
collection of Prime 3-manifolds.



\subsubsection{Prime Manifold}\label{sec:prime_manifold}

\subsubsection{Dehn Surgery}\label{sec:dehn_surgery}

Any Oriented, Closed 3-manifold can be obtained from any other
Oriented, Closed 3-manifold by removing a collection of Disjoint,
Solid Tori (Dehn Drilling) and gluing them back in (Dehn Filling) by a
different identification.



\subsubsection{Compact Manifold}\label{sec:compact_manifold}

When $V$ is a Real or Complex Vector Space, the Grassmanians
(\S\ref{sec:grassmanian}) of $V$--parameterizations of the Linear Subspaces
(\S\ref{sec:linear_subspace}) of $V$--are Compact Smooth Manifolds
(\S\ref{sec:smooth_manifold}).



\paragraph{Closed Manifold}\label{sec:closed_manifold}\hfill

Compact Manifold without a Boundary; if no Boundary is possible then any
Compact Manofld is a Closed Manifold (FIXME: clarify)

FIXME: is this opposite from the notion of ``closed set'' ???

\emph{Classification Theorem of Closed Surfaces}: Any Connected
(\S\ref{sec:connected_space}) Closed Surface is Homeomorphic to a member of one
of the three families:
\begin{enumerate}
  \item the Sphere (\S\ref{sec:n_sphere})
  \item the Connected Sum of $1 \leq g$ Tori (\S\ref{sec:torus})
  \item the Connected Sum of $1 \leq k$ Real Projective Planes
    (\S\ref{sec:real_projective_plane})
\end{enumerate}
\fist Connected Sum (\S\ref{sec:connected_sum})

it follows that a Closed Surface is determined up to Homeomorphism by its Euler
Characteristic and whether it is Orientable or not

$\cat{Cob}$ (or $\cat{nCob}$) -- Dagger Compact Category
(\S\ref{sec:dagger_compact}) of Compact Manifolds and Cobordisms
(\S\ref{sec:cobordism}) with Disjoint Union as Tensor and ``reversal'' (FIXME:
???) of Closed Manifolds as Dagger



\paragraph{Cobordism}\label{sec:cobordism}\hfill

(wiki):

the term \emph{Bordism} is sometimes used for the Equivalence Classes of
Cobordisms as distinct from individual Cobordisms

cf. Homology (\S\ref{sec:homology_theory})

a Cobordism $(W; M, N)$ is a kind of Cospan (\S\ref{sec:cospan})
$M \rightarrow W \leftarrow N$

$\cat{Cob}$

a Topological Quantum Field Theory can be defined as a Functor from $\cat{nCob}$
into $\cat{FdHilb}$

Equivalence Relation on the Class of Compact Manifolds of the same Dimension:
two Manifolds of the same Dimension are \emph{Cobordant} if their Disjoint
Union is the Boundary of a Compact Manifold of one Dimension higher

it is not possible to classify Manifolds up to Diffeo- or Homeo-morphism in
Dimensions $\geq 4$--because the Word Problem for Groups cannot be solved--but
they can be classified up to Cobordism

%FIXME: explain



\subsubsection{K\"ahler Manifold}\label{sec:kahler_manifold}

can be viewed as either:

\begin{itemize}
\item a Hermitian Manifold (\S\ref{sec:hermitian_manifold}) with a
  Closed Hermitian Form called the \emph{K\"ahler Metric} %FIXME
\item $(K,\omega)$ -- a Symplectic Manifold
  (\S\ref{sec:symplectic_manifold}) equipped with an Integrable
  Almost-Complex Structure, Compatible with the Symplectic Form %FIXME
\end{itemize}

\url{https://johncarlosbaez.wordpress.com/2018/12/01/geometric-quantization-part-1/}:

the Projective Space $PH$ of Quantum States (1-dimensional Subspaces of a
Hilbert Space $H$) is the simplest example of a K\"ahler Manifold equipped with
a \emph{Holomorphic Hermitian Line Bundle} whose Curvature is the Imaginary Part
of the K\"ahler Structure



\paragraph{Calabi-Yau Manifold}\label{sec:calabi_yau_manifold}



% ------------------------------------------------------------------------------
\subsection{Orbifold}\label{sec:orbifold}
% ------------------------------------------------------------------------------

generalization of Manifold--
instead of being ``Locally Modeled'' (???) on Open Subsets of $\reals^n$, an
\emph{Orbifold} is Locally Modeled on \emph{Quotients} of Open Subsets of
$\reals^n$ by \emph{Finite Group Actions}

\begin{itemize}
  \item any Manifold without a Boundary is \emph{trivially} an Orbifold
\end{itemize}

a Topological Space called the \emph{Underlying Space} that locally looks like
the Quotient Space of Euclidean Space under the Linear Action of a Finite Group
(FIXME: clarify) with an \emph{Orbifold Structure}

the Configuration Space (\S\ref{sec:configuration_space}) of not-necessarily
distinct Unordered Points of a Manifold is an Orbifold



% ------------------------------------------------------------------------------
\subsection{Knot Theory} \label{sec:knot_theory}
% ------------------------------------------------------------------------------

2010 - Meredith, Snyder - \emph{Knots as processes: a new kind of invariant} --
\fist Process Calculus (\S\ref{sec:process_calculus})

Dowker Notation

Conway Notation



Hyperbolic Knot

Torus Knot

Satellite Knot



Unknot

Trefoil Knot

Figure-eight Knot

Hopf Link

Borromean Rings

\asterism

all $n$-dimensional Manifolds can be Untangled in $2n + 2$ Dimensions



\subsubsection{Knot} \label{sec:knot}

Link with a single component %FIXME

\url{https://knotplot.com/} -- images of Knots and Links



\subsubsection{Link} \label{sec:link}

Loop (\S\ref{sec:loop})



\paragraph{Whitehead Link} \label{sec:whitehead_link}\hfill

\paragraph{Concordance} \label{sec:concordance}\hfill



\subsubsection{Ambient Isotopy} \label{sec:ambient_isotopy}

\subsubsection{Quandle} \label{sec:quandle}

\paragraph{Rack}\label{sec:rack}\hfill



\subsubsection{Torus Knot} \label{sec:torus_knot}

\subsubsection{Knot Polynomial} \label{sec:knot_polynomial}

\fist Khovanov Homology (\S\ref{sec:khovanov_homology})



\paragraph{Jones Polynomial} \label{sec:jones_polynomial}\hfill

Planar Algebra (TODO)

\asterism

\emph{Firewalls, AdS/CFT, and the Complexity of States and Unitaries: A Computer
  Science Perspective}
(video lectures)
-
\url{https://video.ias.edu/PiTP/2018/0719-ScottAaronson}

Topological Quantum Field Theory: any Quantum Computation can be expressed as
approximating the value of the Jones Polynomial of some Knot



\subsubsection{Braid Theory} \label{sec:braid_theory}



% ------------------------------------------------------------------------------
\subsection{Surface}\label{sec:surface}
% ------------------------------------------------------------------------------

$2$-manifold



\subsubsection{Genus}\label{sec:genus}

An Orientable Surface (\S\ref{sec:orientable_surface}) $M_g$ of Genus $g$ can
be constructed from a Polygon with $4g$ sides by identifying pairs of Edges.

\fist Arithmetic Genus (\S\ref{sec:arithmetic_genus}), Geometric Genus
(\S\ref{sec:geometric_genus}), Projective Scheme Genus
(\S\ref{sec:scheme_genus}), Quadratic Form Genus (\S\ref{sec:quadratic_genus}),
Spinor Genus (\S\ref{sec:spinor_genus})



\subsubsection{Closed Surface}\label{sec:closed_surface}

Compact \emph{without} Boundary

\textbf{Classification Theorem of Closed Surfaces}: \emph{
  Any Connected Closed Surface is Homeomorphic to a member of the three
  families:
  \begin{itemize}
    \item Sphere
    \item a Connected Sum of $g$ Tori for $g \geq 1$
    \item a Connected Sum of $k$ Real Projective Planes for $k \geq 1$
  \end{itemize}
}
families (1.) and (2.) are Orientable and may be regarded as a single family
with the SPhere as the Connected Sum of $0$ Tori

it follows that a Closed Surface is determined up to Homeomorphism by its Euler
Characterstic and whether or not it is Orientable

Homeomorphism Classes of Closed Surfaces form a \emph{Commutative Monoid} with
Identity as the Sphere and the Projective Plane and Torus Generates the Monoid
with a single Relation $P \# P \# P = P \# T$



\subsubsection{Orientable Surface}\label{sec:orientable_surface}\hfill

cf. Orientable Manifolds (\S\ref{sec:orientable_manifold})

An Orientable Surface $M_g$ of Genus (\S\ref{sec:genus}) $g$ can
be constructed from a Polygon with $4g$ sides by identifying pairs of Edges.

by the Classification Theorem for Closed Surfaces, a Closed Surface is
determined up to Homeomorphism by its Euler Characterstic and whether or not it
is Orientable



\subsubsection{Gaussian Surface}\label{sec:gaussian_surface}\hfill

\subsubsection{Teichm\"uller Space}\label{sec:teichmuller_space}

Riemann Surface (\S\ref{sec:riemann_surface})

\emph{Inter-universal Teichm\"uller Theory} % FIXME



% ------------------------------------------------------------------------------
\subsection{3-manifold}\label{sec:three_manifold}
% ------------------------------------------------------------------------------

\emph{Thurston's Geometrization Conjecture}
(\S\ref{sec:geometrization_conjecture}): every Closed 3-manifold can be
decomposed in a canonical way into pieces that each have one of eight types of
Geometric structure

analog of the Uniformization Theorem for Two-dimensional Surfaces that states
every Simply-connected Riemann Surface can be given one of either Euclidean,
Spherical, or Hyperbolic Geometries



% ------------------------------------------------------------------------------
\subsection{4-manifold}\label{sec:four_manifold}
% ------------------------------------------------------------------------------

% ------------------------------------------------------------------------------
\subsection{Hypersurface}\label{sec:hypersurface}
% ------------------------------------------------------------------------------

%FIXME: move this section ???

Manifold of Dimension $n-1$ in an Ambient Space of Dimension $n$

generalization of Hyperplane (\S\ref{sec:hyperplane})

every Hypersurface defines a Characteristic Foliation: 1D Foliation
(\S\ref{sec:foliation})

a Hypersurface of ``\emph{Contact Type}''; Liouville Vector Field (Canonical
Vector Field on a Tangent Bundle \S\ref{sec:liouville_vector_field}) \fist
Contact Geometry (\S\ref{sec:contact_geometry})

a \emph{Contact Structure} (\S\ref{sec:contact_structure}) is a Smooth Field
$\xi$ of Hyperplanes which is a Subbundle of the Tangent Bundle of an
Odd-dimensional Manifold: $\xi \subseteq T X$

%TODO



\subsubsection{Polar Hypersurface}\label{sec:polar_hypersurface}

Algebraic Geometry

Projective Algebraic Hypersurfaces

%FIXME



% ------------------------------------------------------------------------------
\subsection{Gauge Theory}\label{sec:gauge_theory}
% ------------------------------------------------------------------------------

%FIXME field theory

Causal Perturbation Theory (\S\ref{sec:causal_perturbation})



\subsubsection{Lagrangian}\label{sec:lagrangian}

\fist Lagrange Multipliers (Constrained Optimization
\S\ref{sec:lagrange_multiplier})

$\mathcal{L}(x,y,\lambda) = f(x,y) - \lambda (C(x,y) - c)$

where $f(x,y)$ is the Real-valued Multivariable Function to Maximize, $\lambda$
is the Lagrange Multiplier and $C(x,y) = c$ is a Constraint on the Real-valued
Multivariable Function $C(x,y)$ is solved by:
\[
  \nabla\mathcal{L} = 0
\]
i.e. the Point $x,y$ such that the Gradient of $\mathcal{L}(x,y,\lambda)$ is
Zero, subject to the Constraint $C(x,y) = c$

\emph{Hamilton's Principal Function} in the Hamilton-Jacobi Equation
(\S\ref{sec:hamilton_jacobi}):
\[
  S(q,t) = \int^{(q,t)} \mathcal{L} dt
\]
where $\mathcal{L}$ is the Lagrangian of the System is equal to the Classical
Action (\S\ref{sec:trajectory_action}) of the System



\subsubsection{Higher Gauge Theory}\label{sec:higher_gauge_theory}

\url{https://ncatlab.org/nlab/show/higher+gauge+field}

\url{https://golem.ph.utexas.edu/category/2018/02/physics_and_2groups.html}



% ------------------------------------------------------------------------------
\subsection{JSJ Decomposition}\label{sec:jsj_decomposition}
% ------------------------------------------------------------------------------

\emph{Jaco-Shalen-Johannson Decomposition}

Essential Embedded 2-Torus



% ------------------------------------------------------------------------------
\subsection{Zariski Geometry}\label{sec:zariski_geometry}
% ------------------------------------------------------------------------------

% ------------------------------------------------------------------------------
\subsection{Simplical Homology}\label{sec:simplical_homology}
% ------------------------------------------------------------------------------

% ------------------------------------------------------------------------------
\subsection{Hodge Theory}\label{sec:hodge_theory}
% ------------------------------------------------------------------------------

uses Partial Differential Equations (\S\ref{sec:pde}) to study the Cohomology
Groups (\S\ref{sec:cohomology_group}) of a Smooth Manifold
(\S\ref{sec:smooth_manifold}), $M$

\fist Global Analytic Geometry (\S\ref{sec:global_analytic_geometry}), ``Global
Hodge Theories'' (\S\ref{sec:global_hodge_theory}): definition of a Hodge
Theory for Arithmetic Varieties (\S\ref{sec:arithmetic_variety})



\subsubsection{De Rham Cohomology}\label{sec:derham_cohomology}



% ==============================================================================
\section{Differential Topology}\label{sec:differential_topology}
% ==============================================================================

Differentiable Functions (\S\ref{sec:differentiable_function}) on Differentiable
Manifolds (\S\ref{sec:differentiable_manifold})

\fist Differential Geometry (\S\ref{sec:differential_geometry})

\fist Differential Calculus (\S\ref{sec:differential_calculus})

General Position (\S\ref{sec:general_position})

Generic Intersections: Transversality (\S\ref{sec:transversality})



% ------------------------------------------------------------------------------
\subsection{Differentiable Manifold}\label{sec:differentiable_manifold}
% ------------------------------------------------------------------------------

A \emph{Differentiable Manifold} (or \emph{Differential Manifold}) is a
Topological Manifold together with an Equivalence Class of Atlases whose
Transition Maps are all Differentiable.

A \emph{$C^k$-manifold} is a Topological Manifold with an Atlas having
Transition Maps that are all $k$-times Continuously Differentiable.

\fist Differential Geometry (\S\ref{sec:differential_geometry})

Currents (Geometric Measure Theory \S\ref{sec:current})

\asterism

Mathview - Manifolds - \url{https://www.youtube.com/watch?v=jvotAfSUxWU}
--
Connected, Hausdorff, Paracompact Topological Space -- gives well-behaved Points
and Neighborhoods;
Epsilon Neighborhoods constructed from Coordinates in
$\reals^n$ are Vector Spaces of \emph{Differentials} (\S\ref{sec:differential});
an Epsilon Neighborhood normally is not a Vector Space (i.e. not Closed under
Vector Addition), but Differentials (i.e. arbitrarily small Positive Real
Numbers) are:

Thm. \emph{The Coordinate Differentials $\mathrm{d}x^n$ form a Vector Space in
  Open Neighborhood $O_\epsilon \subset \reals^n$ over the Field of Real
  Numbers.}

Thm. \emph{The Vector Space of Differentials in $O_\epsilon(\hat{x}_p) \subset
  \reals^n$ is Isomorphic to the Vector Space of Differentials in
  $O_\epsilon(\hat{p}) \subset M$.}

implies Points of $M$ are Elements of of a Local Linear (Vector) Space



\subsubsection{Differentiable Structure}\label{sec:differentiable_structure}

or \emph{Differential Structure} on a Set $M$ makes $M$ into a Differentiable
Manifold

$\reals^n$ has a unique Differentiable Structure \emph{unless} $n = 4$;
$\reals^4$ has Uncountably many Differentiable Structures-- cf. Exotic
$\reals^4$, Exotic Spheres; Exotic Smooth Structure
(\S\ref{sec:exotic_smooth_structure})



\subsubsection{Affine Manifold}\label{sec:affine_manifold}

cf. Hyperplane (\S\ref{sec:hyperplane}), Affine Hyperplane
(\S\ref{sec:affine_hyperplane})



\subsubsection{Metric Tensor}\label{sec:metric_tensor}

Covariant (\S\ref{sec:vector_covariance}) and Contravariant
(\S\ref{sec:vector_contravariance}) Vector Components of Non-cartesian
Coordinates can be converted into one another using the Metric Tensor

a Metric Tensor $g : V \times V \rightarrow K$ that is a Symmetric Bilinear Form
(\S\ref{sec:symmetric_bilinear}) allows Covectors to be identified with Vectors,
i.e. a Vector $v$ uniquely determines a Covector $\alpha$ and conversely each
Covector $\alpha$ determines a unique Vector $v$:
\[
  \alpha (w) = g (v,w)
\]
for all Vectors $w$
(FIXME: explain)

the Covariant Derivative of the Metric Tensor is Zero (FIXME: clarify)

Dual Vector Space (\S\ref{sec:dual_space})

Metric Tensor is a Non-degenerate Symmetric Bilinear Form
(\S\ref{sec:symmetric_bilinear}) on each Tangent Space that varies Smoothly from
Point to Point; Inner Product (\S\ref{sec:inner_product}) --FIXME: clarify

the Symplectic Form (\S\ref{sec:symplectic_form}) in Symplectic Geometry
(\S\ref{sec:symplectic_geometry}) plays the role analagous to Metric Tensor in
Riemannian Geometry (\S\ref{sec:riemannian_geometry})



\paragraph{Positive Definite Metric Tensor}
\label{sec:positive_definite_metric_tensor}\hfill

Positive Definite Quadratic Form (\S\ref{sec:definite_quadratic})

A Manifold with a Positive Definite Metric Tensor is known as a
Riemannian Manifold (\S\ref{sec:riemannian_manifold})



\subsubsection{Submersion}\label{sec:submersion}

A \emph{Submersion} is a Differentiable Function
(\S\ref{sec:differentiable_function}) between Differentiable Manifolds
whose Derivative is everywhere Surjective.



\paragraph{Fibered Manifold}\label{sec:fibered_manifold}\hfill



\subsubsection{Immersion}\label{sec:immersion}

An Immersion is a Differentiable Function
(\S\ref{sec:differentiable_function}) between Differentiable Manifolds
whose Derivative is everywhere Injective.



\subsubsection{Tangent Space}\label{sec:tangent_space}

$T_x M$ -- the Tangent Space of $M$ at $x$, i.e. the Fiber (\S\ref{sec:fiber})
of the Tangent Bundle (\S\ref{sec:tangent_bundle}) $T M$ over $x$

(wiki): a Real Vector Space attached to each point $x$ of a Differentiable
Manifold which contains the possible ``directions'' one can Tangentially pass
through $x$

the Elements of the Tangent Space at $x$ are called the \emph{Tangent Vectors}
at $x$-- generalization of the notion of a \emph{Bound Vector}
(\S\ref{sec:bound_vector}) in a Euclidean Space

the Dimension of the Tangent Space at every Point of a Connected Manifold
(\S\ref{sec:connected_space}) is the same as that of the Manifold itself

\fist cf. Parallel Transport (\S\ref{sec:parallel_transport})

\fist Algebraic Varieties (Algebraic Geometry \S\ref{sec:algebraic_variety}):
Zariski Tangent Space (\S\ref{sec:zariski_space})

a \emph{Vector Field} (\S\ref{sec:vector_field}) attaches at every Point of a
Manifold a Vector from the Tangent Space at that Point in a \emph{Smooth}
(\S\ref{sec:smooth_function}) manner manner and such a Vector Field defines a
generalized Ordinary Differential Equation (ODE \S\ref{sec:ode}) on a Manifold
where a Solution to such an Equation is a Diefferentiable Curve on the Manifold
with Derivative at any Point equal to the Tangent Vector attached to that Point
by the Vector Field

\fist an Affine Connection (\S\ref{sec:affine_connection}) \emph{Connects}
nearby Tangent Spaces permitting Tangent Vector Fields
(\S\ref{sec:vector_field}) to be Differentiated (\S\ref{sec:derivative}) as if
they were Functions on the Manifold with Values in a fixed Vector Space

a Subbundle (\S\ref{sec:subbundle}) of the Tangent Bundle
(\S\ref{sec:tangent_bundle}) of a Smooth manifold is called a
\emph{Distribution} (\S\ref{sec:tangent_bundle_distribution}) of Tangent
Vectors

Metric Tensor (\S\ref{sec:metric_tensor}) is a Non-degenerate Symmetric Bilinear
Form (\S\ref{sec:symmetric_bilinear}) on each Tangent Space that varies Smoothly
from Point to Point



\paragraph{Tangent Bundle}\label{sec:tangent_bundle}\hfill

the \emph{Tangent Bundle}, $T M$ of a Differentiable Manifold $M$ is the
Disjoint Union of the Tangent Spaces of $M$. The Tangent Bundle equipped with a
natural Topology is the prototypical example of a Vector Bundle
(\S\ref{sec:vector_bundle}), i.e. a Fiber Bundle with Fibers that are Vector
Spaces.

(wiki):

the Dimension of $T M$ is twice the Dimension of $M$

the Tangent Bundle ``comes equipped'' with a ``natural'' Topology that is
\emph{not} the Disjoint Union Topology (\S\ref{sec:disjoint_union_topology}),
and with a Smooth Structure (\S\ref{sec:smooth_structure}), making it into a
Manifold itself

the Category of Smooth Vector Bundles (\S\ref{sec:vector_bundle}) is a Bundle
Object (\S\ref{sec:bundle_object}) over the Category of Smooth Manifolds in
$\cat{Cat}$ (the Category of Small Categories) and the Functor taking each
Manifold to its Tangent Bundle (\S\ref{sec:tangent_bundle}) is a Section of the
Bundle Object (i.e. the Category of Smooth Vector Bundles)

(nlab):

\begin{itemize}
  \item $\pi : T X \twoheadrightarrow X$ -- the ``natural'' Projection defining
    the \emph{Tangent Bundle} of a ``sufficiently Differentiable'' Space $X$
  \item $T_x X$ -- the Fiber of $T X$ over $x$, i.e. the \emph{Tangent Space} of
    $X$ at Point $x$
  \item $v \in T_x X$ -- a \emph{Tangent Vector} on $X$ at $x$
  \item $\sigma : X \rightarrow T X$ -- a Section of $T X$, i.e. a
    \emph{Tangent Vector Field}
\end{itemize}

provides the Domain and Range for the Derivative of a \emph{Smooth Function}

definition of \emph{Differentiation} (\S\ref{sec:derivative}) as an
\emph{(Endo-)functor} on the Category of Smooth Manifolds and Smooth Maps:
\[
  \mathrm{d} : \cat{Diff} \rightarrow \cat{Diff}
\]
sending:
\begin{itemize}
  \item Smooth Manifolds $X$ to Tangent Bundles $T X$, with Points of $T X$
    being Ordered Pairs $(x, v)$ where $v$ is a Tangent Vector at $x$, i.e. an
    (Augmented) Derivation (\S\ref{sec:augmented_derivation}) $v : C^\infty(X)
    \rightarrow \reals$ on the Algebra of Smooth Functions, Augmented by
    Evaluation $\mathrm{ev}_x : C^\infty(X) \rightarrow \reals$ at $x$
    (FIXME: clarify)
  \item Smooth Functions $f : X \rightarrow Y$ to Derivatives (FIXME: nlab calls
    this here a ``Differential'' of $f$ but that seems to be inconsistent with
    the terminology used elsewhere) $\diffy{f} : TX \rightarrow TY$; if $\gamma
    : [-1,1] \rightarrow X$ is a Path in $X$ representing a Vector
    $v \in T_x{X}$ then $(\diffy{f})(v) \in T_{f(x)}Y$ is the Vector represented
    by the Path $[-1, 1] \xrightarrow{\gamma} X \xrightarrow{f} Y$
\end{itemize}

the Chain Rule (\S\ref{sec:chain_rule}) is a statement of the Functoriality of
Differentiation on $\cat{Diff}$

a Tangent Bundle comes equipped with a ``natural'' Topology (FIXME: clarify)
and a Smooth Structure (\S\ref{sec:smooth_structure}), making it a Manifold
itself

TODO: Differential $\diffy{f}$ as a Morphism of Tangent Bundles

the Dimension of $T M$ is twice that of $M$

a Vector Field (\S\ref{sec:vector_field}) on a Manifold $M$ is a Smooth Map $V
: M \rightarrow T M$, i.e. a \emph{Cross Section} (\S\ref{sec:cross_section})
of the Tangent Bundle

the Set of all Vector Fields on $M$ is denoted $\Gamma(TM)$ and has the
structure of a Module over the Commutative Algebra of Smooth Functions on $M$,
$C^\infty(M)$

\emph{Canonical Vector Field} (Liouville Vector Field
\S\ref{sec:liouville_vector_field}) on $TM$:
\[
  V : TM \rightarrow TTM
\]
as the Diagonal Map on the Tangent Space at each Point

the Cotangent Bundle (\S\ref{sec:cotangent_bundle}) is the Dual Bundle
(\S\ref{sec:dual_bundle}) of the Tangent Bundle

a Riemannian Metric (\S\ref{sec:riemannian_metric}) or Symplectic Form
(\S\ref{sec:symplectic_manifold}) gives rise to a Natural Isomorphism between
the Tangent Space and the Cotangent Space at a Point

by definition a Manifold $M$ is \emph{Parallelizable} if and only if the
Tangent Bundle is a Trivial Bundle (\S\ref{sec:trivial_bundle})

\emph{Frobenius' Theorem} (\S\ref{sec:frobenius_theorem}): the Subbundle
(\S\ref{sec:subbundle}) of the Tangent Bundle of a Manifold is Integrable (or
Involutive) if and only if it arises from a \emph{Regular Foliation}
(\S\ref{sec:foliation})

Differential (\S\ref{sec:differentiable_function}) $df$ as a Morphism of
Tangent Bundles $df : T\reals \rightarrow T\reals$ (FIXME: explain)



\subparagraph{Foliation}\label{sec:foliation}\hfill

\emph{Leaves} of a Foliation consist of Integrable Subbundles of the Tangent
Bundle

Locally, a Foliation looks like a decomposition of the Manifold as Union of
Parallel Submanifolds of smaller Dimension

\emph{Frobenius' Theorem} (\S\ref{sec:frobenius_theorem}): the Subbundle
(\S\ref{sec:subbundle}) of the Tangent Bundle of a Manifold is Integrable (or
Involutive) if and only if it arises from a \emph{Regular Foliation}

every Hypersurface (\S\ref{sec:hypersurface}) defines a 1D Characteristic
Foliation



\subparagraph{Distribution}\label{sec:tangent_bundle_distribution}\hfill

a Subbundle (\S\ref{sec:subbundle}) of the Tangent Bundle of a Smooth manifold
is called a \emph{Distribution} of Tangent Vectors (\S\ref{sec:tangent_space})

\fist Contact Geometry (\S\ref{sec:contact_geometry}): Geometric (Differentiable
\S\ref{sec:differentiable_manifold}) Structure on Smooth Manifolds given by a
Hyperplane Distribution in the Tangent Bundle satisfying the \emph{Complete
  Non-integrability} condition

2013 - Tanedo - \emph{Notes on non-holonomic constraints} -
\url{https://www.physics.uci.edu/~tanedo/files/teaching/P3318S13/Sec_05_nonholonomic.pdf}

a Tangent Bundle Distribution is a manifestation of a Non-holonomic Constraint
(\S\ref{sec:nonholonomic_constraint})



\subparagraph{Liouville Vector Field}\label{sec:liouville_vector_field}\hfill

cf. Hypersurfaces (\S\ref{sec:hypersurface}) of ``Contact Type''; \fist Contact
Geometry (\S\ref{sec:contact_geometry})

or \emph{Canonical Vector Field} on a Tangent Bundle $TM$:
\[
  V : TM \rightarrow TTM
\]
as the Diagonal Map on the Tangent Space at each Point



\subsubsection{Cotangent Space}\label{sec:cotangent_space}

(Algebraic) Dual Space (\S\ref{sec:dual_space}) $T^*$ of the Tangent Space $T$
at $x$

a direct definition of a Cotangent Space without reference to a Tangent Space
is formulated in terms of Equivalence Classes of Smooth Functions on $M$:
informally two Smooth Functions $f$ and $g$ are Equivalent at a Point $x$ if
they have the same ``\emph{First-order Behavior}'' at $x$, analogous to their
Linear Taylor Polynomials (\S\ref{sec:taylor_polynomial}), where First-order
Behavior is defined as Equivalent if and only if the Derivative of the Function
$f-g$ \emph{vanishes} at $x$ --the Cotangent Space then consists of all the
possible ``\emph{First-order Behaviors}'' of a Function near $x$

Elements are called \emph{Cotangent Vectors} or \emph{Tangent Covectors}

all Cotangent Spaces on a Connected Manifold have the same Dimension as the
Manifold itself

a Riemannian Metric (\S\ref{sec:riemannian_metric}) or Symplectic Form
(\S\ref{sec:symplectic_manifold}) gives rise to a Natural Isomorphism between
the Tangent Space and the Cotangent Space at a Point



\paragraph{Cotangent Bundle}\label{sec:cotangent_bundle}\hfill

Vector Bundle (\S\ref{sec:vector_bundle}) of all the Cotangent Spaces
(\S\ref{sec:cotangent_space}) at every Point in a Smooth Manifold, forming a
new Differentiable Manifold with twice the Dimension of the underlying Manifold

Dual Bundle (\S\ref{sec:dual_bundle}) to Tangent Bundle
(\S\ref{sec:tangent_bundle})

\fist Differential Forms (\S\ref{sec:differential_form}) of Multivariable
Calculus -- (nlab): a \emph{Differential Form} (or \emph{Exterior Differential
  Form}) on a Generalized Smooth Space $X$ is a Section (\S\ref{sec:section}) of
the Exterior Algebra (\S\ref{sec:exterior_algebra}) of the Cotangent Bundle over
$X$

the Exterior Derivative (\S\ref{sec:exterior_derivative}) of a Tautological
$1$-form (\S\ref{sec:tautological_1form}) defined on the Cotangent Bundle $T *
Q$ of a Manifold $Q$ defines a Symplectic Form (\S\ref{sec:symplectic_form})
giving $T * Q$ the Structure of a Symplectic Manifold
(\S\ref{sec:symplectic_manifold})

Canonical Coordinates (\S\ref{sec:canonical_coordinate}) in Hamiltonian
Mechanics (\S\ref{sec:hamiltonian_system}) can be generalized to definition of
Coordinates on the Phase Space (\S\ref{sec:phase_space}) as a Cotangent Bundle
of a Manifold



\subsubsection{Diffeology}\label{sec:diffeology}

\emph{Diffeological Space}



% ------------------------------------------------------------------------------
\subsection{Smooth Manifold}\label{sec:smooth_manifold}
% ------------------------------------------------------------------------------

A \emph{Smooth Manifold} (or \emph{$C^\infty$-manifold}) is a Differentiable
Manifold for whih all Transition Maps are \emph{Smooth} (Derivatives of
all Orders exist, i.e. a $C^k$-manifold for all $k$).

\fist cf. Algebraic Manifolds (\S\ref{sec:algebraic_manifold})-- Smooth
Algebraic Varieties that are also Manifolds

the Category of Smooth Manifolds and Smooth Maps is $\cat{Diff}$

A canonical Smooth Structure (\S\ref{sec:smooth_structure}) on the
$n$-dimensional Cartesian Space (\S\ref{sec:cartesian_space}) $\reals^n$ makes
it a Smooth Manifold.

When $V$ is a Real or Complex Vector Space, the Grassmanians
(\S\ref{sec:grassmanian}) of $V$--parameterizations of the Linear Subspaces
(\S\ref{sec:linear_subspace}) of $V$--are Compact (\S\ref{sec:compact_manifold})
Smooth Manifolds.

\fist Hodge Theory (\S\ref{sec:hodge_theory}) uses Partial Differential
Equations (\S\ref{sec:pde}) to study the Cohomology Groups
(\S\ref{sec:cohomology_group}) of a Smooth Manifold, $M$

\fist The \emph{Holonomy} (\S\ref{sec:holonomy} of a Connection
(\S\ref{sec:connection}) on a Smooth Manifold measures the extent to which
Parallel Transport (\S\ref{sec:parallel_transport}) around a Closed Loop fails
to preserve the ``Geometrical Data'' being Transported.

(nlab):

$\cat{Diff}$ -- the Category of Smooth Manifolds and Smooth Maps
(\S\ref{sec:smooth_function})

the \emph{Chain Rule} (\S\ref{sec:chain_rule}) is the statement that
\emph{Differentiation} (\S\ref{sec:derivative}) $\mathrm{d} : \cat{Diff}
\rightarrow \cat{Diff}$ is a \emph{(Endo-)Functor} on $\cat{Diff}$

for two Smooth Functions $f : X \rightarrow Y$ and $g : Y \rightarrow Z$, then:
\[
  \diffy{(g \circ f)} : TX \xrightarrow{\diffy{f}} TY \xrightarrow{\diffy{f}} TZ
\]



\subsubsection{Diffeomorphism}\label{sec:diffeomorphism}

A \emph{Diffeomorphism} is an Isomorphism of Smooth Manifolds.

Diffeomorphism Group (FIXME)

it is not possible to classify Manifolds up to Diffeo- or Homeo-morphism in
Dimensions $\geq 4$--because the Word Problem for Groups cannot be solved--but
they can be classified up to Cobordism

%FIXME: explain



\paragraph{General Covariance}\label{sec:general_covariance}\hfill

Coordinate-free (\S\ref{sec:coordinate_free}) treatments of ``Physical
Theories'' is a corollary of the \emph{Principle of General Covariance} (TODO)



\subsubsection{Pushforward}\label{sec:pushforward}

the Derivative (\S\ref{sec:derivative}), or sometimes ``Total Derivative'' or
``Differential'' of a Smooth Map

induces a Bundle Map (\S\ref{sec:bundle_map})



\subsubsection{Riemannian Manifold}\label{sec:riemannian_manifold}

%FIXME move to sec:riemannian_geometry?

or \emph{(Smooth) Reimannian Space}

Differentiable Manifold with a Positive Definite Metric Tensor
(\S\ref{sec:positive_definite_metric_tensor}) called the
\emph{Riemannian Metric}

Riemannian Geometry (\S\ref{sec:riemannian_geometry}), Sectional Curvature
(\S\ref{sec:sectional_curvature})

Smooth Manifold $M$

the Local Linear Space becomes an Inner Product Space with a Local Metric Tensor
(FIXME: explain)

Inner Product $g_p$ on Tangent Space $T_pM$ at each point $p$ that
``varies smoothly'' in that if $X$ and $Y$ are Vector Fields on
$M$ then $p \mapsto g_p(X(p),Y(p))$ is a Smooth Function
(FIXME: clarify)

Family $g_p$ of Inner Products: \emph{Riemannian Metric Tensor}

$(M,g)$

Euclidean Space (\S\ref{sec:euclidean_space}) $\reals^n$ as the
``Model'' Riemannian Manifold

cf. Minkowski Space (\S\ref{sec:minkowski_space}) $\reals^{n-1,1}$
with the Flat Minkowski Metric (\S\ref{sec:minkowski_metric}) as the
``Model'' Lorentzian Manifold (\S\ref{sec:lorentzian_manifold})

cf. Hermitian Manifold (\S\ref{sec:hermitian_manifold}): Complex
analogue of a Riemannian Manifold

\fist Statistical Manifolds (\S\ref{sec:statistical_manifold}):
Riemannian Manifold with points taken from Probability Distributions
for a Statistical Model and Fisher Information Metric
(\S\ref{sec:fisher_metric}) as the Riemannian Metric \fist Information Geometry
(\S\ref{sec:information_geometry})

\textbf{Nash Embedding Theorem}: \emph{Every Riemannian Manifold can be
  Isometrically Embedded into some Euclidean Space}

Isometric -- preserving Path Lengths



\paragraph{Riemannian Metric}\label{sec:riemannian_metric}\hfill

for a Differentiable Manifold $M$ of Dimension $n$, a \emph{Riemannian
  Metric} is a Family of Positive Definite
(\S\ref{sec:definite_quadratic}) Inner Products
(\S\ref{sec:inner_product}):
\[
  g_p : T_p M \times T_p M \rightarrow \reals, p \in M
\]
such that for any two Differentiable Vector Fields $X$, $Y$ on $M$:
\[
  p \mapsto g_p (X(p), Y(p))
\]
defines a Smooth Function (\S\ref{sec:smooth_function}) $M \rightarrow
\reals$

equivalently a Riemannian Metric $g$ is a Symmetric $(0,2)$-tensor
that is Positive Definite ($0 < g(X,X)$) for all non-zero Tangent
Vectors %FIXME

a Riemannian Metric or Symplectic Form (\S\ref{sec:symplectic_manifold}) gives
rise to a Natural Isomorphism between the Tangent Space
(\S\ref{sec:tangent_space}) and the Cotangent Space
(\S\ref{sec:cotangent_space}) at a Point

the Fisher Information Metric (\S\ref{sec:fisher_metric}) is the Riemannian
Metric for Statistical Manifolds (\S\ref{sec:statistical_manifold})



\paragraph{Scalar Curvature}\label{sec:scalar_curvature}\hfill

simplest Curvature on a Riemannian Manifold



\subparagraph{Ricci Curvature Tensor}\label{sec:ricci_curvature}\hfill



\paragraph{Riemannian Curvature Tensor}\label{sec:riemannian_curvature}\hfill

\paragraph{Space Form}\label{sec:space_form}\hfill

a Complete Riemannian Manifold $M$ of Constant Sectional Curvature
(\S\ref{sec:sectional_curvature}) $k$

(wiki):

by rescaling the Metric, there are three possible cases
\begin{itemize}
  \item $-1$ Curvature -- Hyperbolic Manifold (\S\ref{sec:hyperbolic_manifold}),
    i.e. Hyperbolic Space (\S\ref{sec:hyperbolic_space})
  \item $0$ Curvature -- Euclidean Space (\S\ref{sec:euclidean_space})
  \item $+1$ Curvature -- Elliptic Manifold (\S\ref{sec:elliptic_manifold}),
    i.e. $n$-sphere (\S\ref{sec:n_sphere})
\end{itemize}

generalized Crystallography

\emph{Space Form Problem} -- conjecture: ``Any two Compact Aspherical
(\S\ref{sec:aspherical_space}) Riemannian Manifolds with Isomorphic Fundamental
Groups (\S\ref{sec:fundamental_group}) are Homeomorphic
(\S\ref{sec:homeomorphism}).''



\subparagraph{Elliptic Manifold}\label{sec:elliptic_manifold}\hfill

a Complete Riemannian $n$-manifold with constant Sectional Curvature
(\S\ref{sec:sectional_curvature}) $+1$

$n$-sphere (\S\ref{sec:n_sphere})

2002 - Gadgil - \emph{Contact Structures on Elliptic 3-manifolds}
-- Contact Structure (\S\ref{sec:contact_structure})



\subparagraph{Hyperbolic Manifold}\label{sec:hyperbolic_manifold}\hfill

a Complete Riemannian $n$-manifold with constant Sectional Curvature
(\S\ref{sec:sectional_curvature}) $-1$

Pseudosphere (\S\ref{sec:pseudosphere})



\subsubsection{Pseudo-Riemannian Manifold}
\label{sec:pseudo_riemannian}

Metric Tensor required to be \emph{Nondegenerate}
(\S\ref{sec:degenerate_bilinear_form}) but need not be Positive
Definite (\S\ref{sec:positive_definite_metric_tensor})

$(M,g)$

Differentiable Manifold $M$

Non-degenerate, Smooth, Symmetric Metric Tensor $g$



\paragraph{Lorentzian Manifold}\label{sec:lorentzian_manifold}\hfill

General Relativity: Spacetime as a 4-dimensional Lorentzian Manifold

one Dimension has a ``sign opposite'' to that of the rest

\emph{Causal Structure} -- Tangent Vectors can be classified as:
\begin{itemize}
  \item Timelike
  \item Null
  \item Spacelike
\end{itemize}

Minkowski Space (\S\ref{sec:minkowski_space}) $\reals^{n-1,1}$
with the Flat Minkowski Metric (\S\ref{sec:minkowski_metric}) as the
``Model'' Lorentzian Manifold

cf. Euclidean Space (\S\ref{sec:euclidean_space}) $\reals^n$ as the
``Model'' Riemannian Manifold (\S\ref{sec:riemannian_manifold})



\paragraph{Conformal Manifold}\label{sec:conformal_manifold}\hfill

a \emph{Conformal Manifold} is a Pseudo-Riemannian Manifold equipped with an
Equivalence Class of Metric Tensors in which any two Metrics $g$ and $h$ are
Equivalent if and only if $h = \lambda^2 g$ where $\lambda$ is a Real-valued
Smooth Function

\fist Conformal Geometry (\S\ref{sec:conformal_geometry})



\subsubsection{Symplectic Manifold}\label{sec:symplectic_manifold}

A \emph{Symplectic Manifold} is a Smooth Manifold equipped with a Closed
Non-degenerate Differential 2-form $\omega$ called the \emph{Symplectic Form}
(\S\ref{sec:symplectic_form})

the Symplectic Form in Symplectic Geometry (\S\ref{sec:symplectic_geometry})
plays the role analagous to Metric Tensor (\S\ref{sec:metric_tensor}) in
Riemannian Geometry (\S\ref{sec:riemannian_geometry})

a Riemannian Metric (\S\ref{sec:riemannian_metric}) or Symplectic Form gives
rise to a Natural Isomorphism between the Tangent Space
(\S\ref{sec:tangent_space}) and the Cotangent Space
(\S\ref{sec:cotangent_space}) at a Point

the Exterior Derivative (\S\ref{sec:exterior_derivative}) of a Tautological
$1$-form (\S\ref{sec:tautological_1form}) defined on the Cotangent Bundle
(\S\ref{sec:cotangent_bundle}) $T *
Q$ of a Manifold $Q$ defines a Symplectic Form (\S\ref{sec:symplectic_form})
giving $T * Q$ the Structure of a Symplectic Manifold

Darboux's Theorem (TODO)



\paragraph{Lagrangian Submanifold}\label{sec:lagrangian_submanifold}\hfill

\fist Holonomic Modules (\S\ref{sec:holonomic_module}): the Characteristic
Variety (FIXME: xref) $Ch(M)$ of any $D$-module $M$, when seen as a Subvariety
of the Cotangent Bundle $T^*X$ of $X$, is an \emph{Involutive Variety} (FIXME:
xref), i.e. the Module is Holonomic if and only if $Ch(M)$ is a Lagrangian
Submanifold



\subsubsection{Contact Manifold}\label{sec:contact_manifold}

\fist Contact Geometry (\S\ref{sec:contact_geometry})

\url{http://www.map.mpim-bonn.mpg.de/Contact_manifold}:

A \emph{Contact Manifold} is a Pair $(M,\xi)$ of an Odd-dimensional Smooth
Manifold $M$ with a Contact Structure (\S\ref{sec:contact_structure}) $\xi$.



\subsubsection{Jacobi Manifold}\label{sec:jacobi_manifold}

\url{https://golem.ph.utexas.edu/category/2019/02/jacobi_manifolds.html}



\paragraph{Poisson Manifold}\label{sec:poisson_manifold}\hfill



% ------------------------------------------------------------------------------
\subsection{Analytic Manifold}\label{sec:analytic_manifold}
% ------------------------------------------------------------------------------

An \emph{Analytic Manifold} or \emph{$C^\omega$-manifold} is a Smooth Manifold
with the additional condition that each Transition Map is \emph{Analytic}
(\S\ref{sec:analytic_function}), i.e. the Taylor Series Expansion
(\S\ref{sec:taylor_series}) is Absolutely Convergent and equals the Function on
some Open Ball.

cf. Algebraic Manifolds (\S\ref{sec:algebraic_manifold})

examples:
\begin{itemize}
  \item the Real Unit Interval $[0,1]$
\end{itemize}



\subsubsection{Complex Manifold}\label{sec:complex_manifold}

Riemann Surface (\S\ref{sec:riemann_surface}): One-dimensional Complex
Manifold

%FIXME: move to differential geometry?

Complex Surface (\S\ref{sec:complex_surface})

any Complex Manifold is an Analytic Variety (\S\ref{sec:analytic_variety})

cf. Complex Analytic Variety (Complex Analytic Space
\S\ref{sec:complex_variety}) -- generalization of Complex Manifold allowing
\emph{Singularities} (\S\ref{sec:singularity})



\paragraph{Hermitian Manifold}\label{sec:hermitian_manifold}\hfill

Complex analogue of Riemannian Manifold (\S\ref{sec:riemannian_manifold})

a K\"ahler Manifold (\S\ref{sec:kahler_manifold}) is a Hermitian
Manifold with a Closed Hermitian Form called the \emph{K\"ahler
  Metric} %FIXME



% ------------------------------------------------------------------------------
\subsection{Symplectic Topology}\label{sec:symplectic_topology}
% ------------------------------------------------------------------------------

Symplectic Manifold (\S\ref{sec:symplectic_manifold})



% ------------------------------------------------------------------------------
\subsection{Morse Theory}\label{sec:morse_theory}
% ------------------------------------------------------------------------------

2021 - Kahle -
\emph{Configuration spaces of particles: homological solid, liquid, and gas}
-- \url{https://www.youtube.com/watch?v=fwOMI6lTJ0Y}

Discrete Morse Theory



% ==============================================================================
\section{Singularity Theory}\label{sec:singularity_theory}
% ==============================================================================

\fist Singularity (\S\ref{sec:singularity}), Singular Points
(\S\ref{sec:singular_point})



% ==============================================================================
\section{Shape Theory}\label{sec:shape_theory}
% ==============================================================================

% ==============================================================================
\section{Descent Theory}\label{sec:descent_theory}
% ==============================================================================

Beck's Monadicity Theorem (\S\ref{sec:monadic_functor})



% ------------------------------------------------------------------------------
\subsection{Descent}\label{sec:descent}
% ------------------------------------------------------------------------------

% ------------------------------------------------------------------------------
\subsection{Fibred Category}\label{sec:fibred_category}
% ------------------------------------------------------------------------------



% ==============================================================================
\section{Continuum Theory}\label{sec:continuum_theory}
% ==============================================================================

% ==============================================================================
\section{Arithmetic Topology}\label{sec:arithmetic_topology}
% ==============================================================================

Algebraic Number Theory (\S\ref{sec:algebraic_number_theory})



% ==============================================================================
\section{Synthetic Topology}\label{sec:synthetic_topology}
% ==============================================================================

\fist Shulman
