%%%%%%%%%%%%%%%%%%%%%%%%%%%%%%%%%%%%%%%%%%%%%%%%%%%%%%%%%%%%%%%%%%%%%%
%%%%%%%%%%%%%%%%%%%%%%%%%%%%%%%%%%%%%%%%%%%%%%%%%%%%%%%%%%%%%%%%%%%%%%
\part{Formal Semantics}\label{part:formal_semantics}
%%%%%%%%%%%%%%%%%%%%%%%%%%%%%%%%%%%%%%%%%%%%%%%%%%%%%%%%%%%%%%%%%%%%%%
%%%%%%%%%%%%%%%%%%%%%%%%%%%%%%%%%%%%%%%%%%%%%%%%%%%%%%%%%%%%%%%%%%%%%%

\emph{Formal Semantics} is the general study of Interpretations
(\S\ref{sec:interpretation}) of Formal Languages (Part
\ref{sec:formal_language}).



% ====================================================================
\section{Interpretation}\label{sec:interpretation}
% ====================================================================

An \emph{Interpretation} of a Formal Language is an assignment of
Meanings (\S\ref{sec:meaning}) to Symbols.

The Logical Symbols (\S\ref{sec:logical_constant}) are given the same
Meaning in every Interpretation, while the Meanings of Non-logical
Symbols (\S\ref{sec:nonlogical_constant}) are given by an Interpretation
Function (\S\ref{sec:interpretation_function}).

A \emph{Truth Functional Interpretation}

In Higher-order Logic (\S\ref{sec:higherorder_logic}) Quantifiers may
be Interpreted differently depending on the Semantics given: a
\emph{Full Semantics} Interprets Quantifiers as Ranging over all
possible Objects of a given Type (\S\ref{sec:type}), while a
\emph{Henkin Semantics} requires an separate Domain for each Type of
Higher-order Variable to Range over.



% --------------------------------------------------------------------
\subsection{Semantic Consequence}\label{sec:semantic_consequence}
% --------------------------------------------------------------------

\emph{Semantic Consequence} of a Theory (\S\ref{sec:formal_theory})
$\mathcal{T}$ of a Formal System $\mathcal{S}$:
\[
  \mathcal{T} \vDash_{\mathcal{S}} \varphi
\]
is defined for Sentences $\varphi$ such that any Interpretation
$\mathcal{I}$ that Satisfies (\S\ref{sec:satisfaction}) $\mathcal{T}$
also Satisfies $\varphi$:
\[
  \mathcal{I} \models \mathcal{T}
  \Rightarrow \mathcal{I} \models \varphi
\]
That is, $\varphi$ is True in all Models of $\mathcal{T}$.

A Tautology (\S\ref{sec:tautology}) is expressed as:
\[
    \vDash {\varphi}
\]
where a Sentence $\varphi$ is the Semantic Consequence of the Empty
Set.



% --------------------------------------------------------------------
\subsection{Valuation}\label{sec:valuation}
% --------------------------------------------------------------------

A \emph{Valuation} is the assignment of a Truth-value
(\S\ref{sec:truth_value}) to each Sentence (\S\ref{sec:sentence}) in a
Formal Language that follows a \emph{Truth Schema} (or \emph{T-schema}
\S\ref{sec:t_schema}).

In Propositional Logic a Valuation results from the Truth Assignment
(Interpretation Function \S\ref{sec:interpretation_function}) giving
Truth-values to each Propositional Variable which induces Truth-values
for all Propositional Formulas according to the Truth Functional
Interpretation of the Logical Constants of the Language.

In First-order Logic a Structure (\S\ref{sec:structure}) gives the
Domain of Discourse over which Quantifiers Range and an Interpretation
Function for Function and Predicate Symbols results in unique
assignment of Truth-values to all Sentences of the Language.

A Valuation $v : L \rightarrow \{true, false\}$ on a Formula $\varphi
\in L$ of a Language $L$ may be denoted $v(\varphi) = [[\varphi]]_v$.



\subsubsection{Supervaluation}\label{sec:supervaluation}

A \emph{Supervaluation} is a Valuation that does not assign
Truth-values to Vacuous Truths.

A Supervaluation $V(\varphi)$ is Undefined if there are exactly two
Valuations $v$ and $v'$ such that $v(\varphi) = true$ and $v'(\varphi)
= false$.

Valid Sentences still have Truth-values even if the constituent Atomic
Formulas do not and under Supervaluationism may be called
\emph{Supertrue}. That is, a Supervaluation $V(\varphi)$ is Supertrue
if and only if $v(\varphi) = true$ for every Valuation $v$. The dual
notion of \emph{Superfalse} is defined for a Supervaluation $V(\psi)$
such that $v(\psi) = false$ for every Valuation $v$.



% --------------------------------------------------------------------
\subsection{Semantic Validity}\label{sec:semantic_validity}
% --------------------------------------------------------------------

An Inference (\S\ref{sec:logical_inference}) is \emph{Semantically
  Valid} if all Interpretations that Satisfy the Premises also Satisfy
the Conclusion. \cite{gamut91}



% --------------------------------------------------------------------
\subsection{Intended Interpretation}\label{sec:intended_interpretation}
% --------------------------------------------------------------------

Standard Model (\S\ref{sec:standard_model})



% --------------------------------------------------------------------
\subsection{Herbrand Interpretation}\label{sec:herbrand_interpretation}
% --------------------------------------------------------------------



% ====================================================================
\section{Semantic Truth}\label{sec:semantic_truth}
% ====================================================================

\emph{Semantic Truth} is a conception of Logical Truth
(\S\ref{sec:logical_truth}) as a Property (\S\ref{sec:property}) of
Sentences (\S\ref{sec:sentence}).

The two related conceptions of Truth are the \emph{Correspondence
  Theory} (\S\ref{sec:correspondence_truth}) and \emph{Deflationary
  Theory} (\S\ref{sec:deflationary_truth}).

Briefly, the Undefinability Theorem
(\S\ref{sec:undefinability_theorem}) results in a \emph{Material
  Adequacy Condition} (called \emph{Convention T}) that any Theory of
Truth must entail:
\[
    \forall P (\mathrm{True}(S) \leftrightarrow P)
\]
where $S$ is the name of the Sentence $P$ in the Metalanguage which is
an Interpretation of $P$ in the Object Language. This is the
\emph{T-schema} (\S\ref{sec:t_schema}) used in \emph{Tarski's Semantic
  Theory of Truth} to Inductively define Truth, expressed as a
First-order Sentence. When a Modal Logic is based on the T-schema it
is said to give rise to \emph{T-Theory}. Tarski's Semantic Theory of
Truth is used as the definition for Truth in \emph{Model Theory}
(Part \ref{part:model_theory}).

An example sentence conforming to Convention T in Natural Language
where the Object Language is German and the Metalanguage is English:
\begin{description}
  \item ``\emph{Der Schnee ist wei\ss} is True if and only if snow is
    white''.
\end{description}
Here the right side of the Biconditional ('snow is white') is the
\emph{Truth-condition} of the left side.

Tarski considered this definition of Truth to be a type of
Correspondence Theory.

\emph{Dialetheism}



% --------------------------------------------------------------------
\subsection{T-schema}\label{sec:t_schema}
% --------------------------------------------------------------------

\emph{T-schema} (or \emph{Equivalence Schema}) is used to give an
Inductive definition of Truth.



% --------------------------------------------------------------------
\subsection{Correspondence Theory}\label{sec:correspondence_truth}
% --------------------------------------------------------------------

The \emph{Correspondence Theory of Truth} defines Truth of a Statement
by its relation and Correspondence with the world.



% --------------------------------------------------------------------
\subsection{Coherence Theory}\label{sec:coherence_theory}
% --------------------------------------------------------------------

The \emph{Coherence Theory of Truth} defines Truth of a Statement by
its relation to other Statements.



% --------------------------------------------------------------------
\subsection{Deflationary Theory}\label{sec:deflationary_truth}
% --------------------------------------------------------------------

A \emph{Deflationary Theory of Truth} is one that states that
ascribing Truth to a Statement does not attribute a property of Truth
to any such Statement in one of a number of different ways below.



\subsubsection{Redundancy Theory}\label{sec:redundancy_theory}

The \emph{Redundancy Theory of Truth} states that the Predicate of
Truth is Redundant in that it is equal to the Statement it is applied
to.\cite{ramsey27} Essentially, Truth is a \emph{periphrasis} of the
Sentence it is applied to.



\paragraph{Disappearance Theory}\label{sec:disappearance_theory}
\hfill \\

A \emph{Disappearance Theory of Truth} states that Truth is both
Redundant and there is no such Property of Truth. A.J. Ayer is known
for this Theory.



\subsubsection{Performative Theory}\label{sec:performative_theory}

The \emph{Performative Theory of Truth} is a Deflationary Theory that
sees the Predicate of Truth as a signal of agreement with the
Statement, for such reasons as arriving at Consensus or such others.



\subsubsection{Disquotational Theory}\label{sec:disquotational_theory}

The \emph{Disquotational Theory of Truth} is a Deflationary
interpretation of Tarski's definition of Truth by W.V.O. Quine. It
states the Truth predicate has the effect of \emph{Dereferencing}
Sentences (removing the quotation marks). So
\[
    S \leftrightarrow True(``S``)
\]
that is, $S$ is equivalent to \emph{``$S$'' is true}.

The effect of adding \emph{is True} to an Assertion is then to convert
the Use of the Assertion to a Mention.



\subsubsection{Prosententialism}\label{sec:prosententialism}

\emph{Prosententialism} denies that ``is true'' is a Predicate and is
instead a \emph{Prosentence} (the Sentential analog to
\emph{Pronouns}) that stands in for another Sentence.



\subsubsection{Minimalist Theory}\label{sec:minimalist_theory}

\emph{Minimalism} defines Truth as a \emph{Metalinguistic} property
and that only Propositoins are Truth-bearing.



% --------------------------------------------------------------------
\subsection{Normative Theory}\label{sec:normative_theory}
% --------------------------------------------------------------------

A \emph{Normative Theory of Truth} states that Truth is the Normative
goal of Assertion.



% --------------------------------------------------------------------
\subsection{Undefinability Theorem}\label{sec:undefinability_theorem}
% --------------------------------------------------------------------

\emph{Tarski's Undefinability Theorem} \cite{tarski36} uses the same
techniques as G\"odel's Incompleteness Theorems to show that Truth
cannot be defined in an Object Language.



% ====================================================================
\section{Model-theoretic Semantics}\label{sec:model_semantics}
% ====================================================================

% --------------------------------------------------------------------
\subsection{Tarskian Semantics}\label{sec:tarski_semantics}
% --------------------------------------------------------------------

\emph{Semantic Theory of Truth}



% --------------------------------------------------------------------
\subsection{Truth-conditional Semantics}
\label{sec:truth_conditional_semantics}
% --------------------------------------------------------------------

\subsubsection{Truth-condition}\label{sec:truth_condition}



% --------------------------------------------------------------------
\subsection{Frame Semantics}\label{sec:frame_semantics}
% --------------------------------------------------------------------

\emph{Frame Semantics} is the extension of Model Theory to
Non-classical Logic Systems, beginning with Modal Logic
(\S\ref{sec:modal_logic}).



\subsubsection{Kripke Semantics}\label{sec:kripke_semantics}

\emph{Kripke Semantics} (or \emph{Relational Semantics})

\begin{description}
\item [Kripke Frame] (\S\ref{sec:kripke_frame}) $\langle W,R \rangle$
  where $W$ is a Non-empty Set of \emph{Nodes} (\emph{Worlds}) and $R$
  is a Binary Relation called the \emph{Accessibility Relation}
\item [Kripke Model] $\langle W,R,\Vdash \rangle$ where $\Vdash$ is a
  \emph{Forcing Relation} for Nodes of $W$
\end{description}
The Accessbility Relation is described by Modal Axioms that specify on
what basis a Statement is True in \emph{any} Possible World.

The Forcing Relation $\Vdash$ (read as Satisfies or \emph{Forces}
(\S\ref{sec:forcing})) has the following properties:
\begin{itemize}
\item $w \Vdash \neg A$ if and only if $w \nVdash A$
\item $w \Vdash A \rightarrow B$ if and only if $w \nVdash A$ or $w
  \Vdash B$
\item $w \Vdash \square A$ if and only if $u \Vdash A$ for all $u$
  such that $w R u$
\end{itemize}
\emph{Validity} of a Proposition is defined for
\begin{itemize}
\item Model $\langle W,R, \Vdash \rangle$ if $\forall w \in W,
  w \Vdash A$
\item Frame $\langle W,R \rangle$ if Valid in Model $\langle W,R,
  \Vdash \rangle$ for all choices of $\Vdash$
\item Class of Frames or Models, $C$, if Valid for all Frames or
  Models of the Class
\end{itemize}
$Thm(C)$ is defined as the Set of all Formulas Valid
(\S\ref{sec:validity}) in $C$. For a Set of Formulas $X$,
$Mod(X)$ is defined as the Class of all Frames which Validate every
Formula in $X$.

Kripke Models are a special case of Labelled State Transition Systems
(\S\ref{sec:state_transition_system}).

A Modal Theory, $T$, is \emph{Sound} (\S\ref{sec:soundness}) with
respect to a Class of Frames, $C$, if $T \subseteq Thm(C)$. $T$ is
\emph{Complete} with respect to $C$ if $T \supseteq Thm(C)$.

For any $C$, $Thm(C)$ is a \emph{Normal Modal Logic}
(\S\ref{sec:alethic_logic}). A Normal Modal Logic is said to
\emph{Correspond} to $C$ if $C = Mod(L)$.

The \emph{Canonical Model} of a Modal Theory $T$ is a Kripke Model
$\langle W,R, \Vdash \rangle$ where $W$ is the Set of all Maximally
Consistent Sets for $T$ (\S\ref{sec:formal_theory}) and:
\begin{itemize}
\item $XRY$ if and only if for all Formulae $A$, if $\square A
  \in X$ then $A \in Y$
\item $X\Vdash A$ if and only if $A \in X$
\end{itemize}

%FIXME

\emph{Unravelling}

\emph{Filtration}



\paragraph{Kripke Frame}\label{sec:kripke_frame}

\subparagraph{General Frame}\label{sec:general_frame}
\hfill \\

Kripke Frame with additional structure used to Model Modal
(\S\ref{sec:modal_logic}) and Intermediate Logic
(\S\ref{sec:intermediate_logic}).



\paragraph{Kripke Model}\label{sec:kripke_model}

\paragraph{Possible World Semantics}\label{sec:possible_world}
\hfill \\

\subparagraph{Possibility}\label{sec:possibility}
\hfill \\

Logical Possibility

Nomological Possibility

Temporal Possibility

\subparagraph{Subjunctive}\label{sec:subjunctive}

\emph{Counterfactual}



\paragraph{Montague Grammar}\label{sec:montague_grammar}

\paragraph{Type-logical Semantics}\label{sec:typelogical_semantics}

\paragraph{Glue Semantics}\label{sec:glue_semantics}



\subsubsection{Neighborhood Semantics}
\label{sec:neighborhood_semantics}

\emph{Neighborhood Frame} $\langle W, N \rangle$



\subsubsection{General Frame Semantics}
\label{sec:general_frame_semantics}

A \emph{Modal General Frame} is defined as a triple $\mathbf{F} =
\langle W,R,V \rangle$ where $\langle W,R \rangle$ is a Kripke Frame
and $V$ is a Set of Subsets of $W$ closed under Intersection, Union,
Complement, and the Operation $\square$ defined as
\[
    \square A = \{x \in W; \forall y \in W ( x R y \rightarrow y \in A ) \}
\]



\subsubsection{Intuitionistic Semantics}
\label{sec:intuitionistic_semantics}

The Semantics of Intuitionistic Logic
(\S\ref{sec:intuitionistic_logic}) is definable as Kripke Semantics
(\S\ref{sec:kripke_semantics}) or \emph{Heyting Algebra Semantics}.



\paragraph{Heyting Algebra Semantics}\label{sec:heyting_semantics}
\hfill \\

Instead of assigning Valuations from a Boolean Algebra, Intuitionistic
Semantics uses Values from a \emph{Heyting Algebra}
(\S\ref{sec:heyting_algebra}). A Formula is Valid if and only if it
receives the Value of the Top Element for any Valuation.



\paragraph{Brouwer-Heyting-Kolmogorov Interpretation}
\label{sec:brouwer_heyting_kolmogorov}
\hfill \\

\emph{BHK Interpretation}

Realizability (\S\ref{sec:realizability})

Realizability Models (\S\ref{sec:realizability_model})



\subsubsection{Kripke-Joyal Semantics}\label{sec:kripke_joyal}

\subsubsection{Boolean-valued Model}\label{sec:boolean_model}

\emph{Boolean-valued Models} are related to Heyting Algebras and
Intuitionistic Logic and is equivalent to the method of Forcing.

%FIXME ref complete boolean algebra
Instead of limiting Formulas in a System $S$ to True or False, they
may be assigned values from a fixed \emph{Complete Boolean Algebra}.



% ====================================================================
\section{Proof-theoretic Semantics}\label{sec:proof_semantics}
% ====================================================================

\emph{Proof-theoretic Semantics} is based on the Propositions and
Logical Connectives of Systems of Inference (Part
\ref{sec:formal_system}). See also \emph{Semantic Tableau}
(\S\ref{sec:tableau_calculus}).

Meaning Explanation (\S\ref{sec:meaning_explanation})



% --------------------------------------------------------------------
\subsection{Logical Harmony} \label{sec:logical_harmony}
% --------------------------------------------------------------------

\emph{Logical Harmony} refers to constraints required between
Introduction and Elimination Rules.



% ====================================================================
\section{Algebraic Semantics}\label{sec:algebraic_semantics}
% ====================================================================

% --------------------------------------------------------------------
\subsection{Boolean Algebra}\label{sec:boolean_algebra}
% --------------------------------------------------------------------

$Ult(B) \cong Hom_\mathbf{BA}(B,\mathbf{2})$ Ultrafilters
(\S\ref{sec:ultrafilter})



\subsubsection{Truth-table}\label{sec:truth_table}

\subsubsection{Two-element Boolean Algebra}\label{sec:twoelement_boolean}

Propositional Logic (\S\ref{sec:propositional_logic})

Syntactically, every Boolean Term corresponds to a Propositional
Formula



\subsubsection{Boolean Algebra with Operators}\label{sec:boolean_with_operators}

Propositional Modal Logic (\S\ref{sec:modal_logic})



\subsubsection{Monadic Boolean Algebra}\label{sec:monadic_boolean}

Modal Logic $\mathrm{S5}$ (\S\ref{sec:modal_logic})

Monadic Predicate Logic (\S\ref{sec:monadic_firstorder})



\subsubsection{Complete Boolean Algebra}\label{sec:complete_boolean}

First-order Logic (\S\ref{sec:firstorder_logic})



% --------------------------------------------------------------------
\subsection{Heyting Algebra}\label{sec:heyting_algebra}
% --------------------------------------------------------------------

A \emph{Heyting Algebra} is a Poset with:
\begin{enumerate}
  \item Finite Meets $1$ and $p \wedge q$
  \item Finite Joins $0$ and $p \vee q$
  \item Exponentials $a \Rightarrow b$ for each $a,b$ such that $a
    \wedge b \leq c$ if and only if $a \leq b \Rightarrow c$
\end{enumerate}
A Heyting Algebra is a Distributive Lattice
(\S\ref{sec:distributive_lattice}), but not every Distributive Lattice
is a Heyting Algebra.

A Boolean Algebra is a Heyting Algebra with a Complementary Negation,
that is, in a Heyting Algebra it is not the case that (in a Preorder):
\[
  \top \leq A \vee \overline{A}
\]
which implies that there are Undecidable Propositions.

Cartesian Closed Preorder (\S\ref{sec:cartesian_preorder})

Algebraic equivalent of Intuitionistic Propositional Logic
(\S\ref{sec:intuitionistic_logic})

Theorem (Intuitionistic Propositional Logic does not refute the Law of
the Excluded Middle)\cite{harper12}:
\[
  \forall A, \neg (\neg (A \vee \neg A))
\]



\subsubsection{Complete Heyting Algebra}\label{sec:complete_heyting}

Heyting Algebra that is a Complete Lattice
(\S\ref{sec:complete_lattice}) with the Infinite Distributive Law
(\S\ref{sec:infinite_distributive})

Locale (\S\ref{sec:locale})

Frame (\S\ref{sec:frame})

a Homomorphism of Heyting Algebras is a Morphism of Frames that also
preserves Implication

$\cat{CHey}$ Morphisms are Homomorphisms of Complete Heyting Algebras



% --------------------------------------------------------------------
\subsection{Lindenbaum-Tarski Algebra}\label{sec:lindenbaum_tarski}
% --------------------------------------------------------------------

\cite{awodey06}
For an Intuitionistic Propositional Calculus $\mathcal{L}$ with
Propositional Formulas $p,q$, a \emph{Lindenbaum-Tarski Algebra} is a
Heyting Algebra $H(\mathcal{L})$ with Equivalence Classes:
\[
  [p] = [q] \Leftrightarrow p \dashv \vdash q
\]
and Ordering Relation:
\[
  [p] \leq [q] \Leftrightarrow p \vdash q
\]
and Induced Operations:
\[
  1 = [\top]
\]\[
  0 = [\bot]
\]\[
  [p] \wedge [q] = [p \wedge q]
\]\[
  [p] \vee [q] = [p \vee q]
\]\[
  [p] \Rightarrow [q] = [p \Rightarrow q]
\]
A Formula $p$ is Provable in $\mathcal{L}$ just in case $[p] = 1$.

Propositional Logic (\S\ref{sec:propositional_logic})



% --------------------------------------------------------------------
\subsection{MV-algebra}\label{sec:mv_algebra}
% --------------------------------------------------------------------

Lukasiewicz Logic (\S\ref{sec:lukasiewicz_logic})



% --------------------------------------------------------------------
\subsection{Cylindric Algebra}\label{sec:cylindric_algebra}
% --------------------------------------------------------------------

First-order Logic with Equality (\S\ref{sec:firstorder_equality})



% --------------------------------------------------------------------
\subsection{Polyadic Algebra}\label{sec:polyadic_algebra}
% --------------------------------------------------------------------

First-order Logic without Equality
(\S\ref{sec:firstorder_no_equality})



% --------------------------------------------------------------------
\subsection{Predicate Functor Logic}\label{sec:pfl}
% --------------------------------------------------------------------

First-order Logic (\S\ref{sec:firstorder_logic})

\emph{Predicate Functor Logic} allows the expression of First-order
Logic (\S\ref{sec:firstorder_logic}) Algebraically without Quantified
Variables by using \emph{Predicate Functors} that Operate on Terms to
yield Terms.



% --------------------------------------------------------------------
\subsection{Modal Algebra}\label{sec:modal_algebra}
% --------------------------------------------------------------------

Normal Modal Logic $\mathrm{K}$ (\S\ref{sec:normal_modal})



% --------------------------------------------------------------------
\subsection{Interior Algebra}\label{sec:interior_algebra}
% --------------------------------------------------------------------

Topology (\S\ref{sec:topology})

Modal Logic $\mathrm{S4}$ (\S\ref{sec:modal_logic})



% --------------------------------------------------------------------
\subsection{Relation Algebra}\label{sec:relation_algebra}
% --------------------------------------------------------------------

Set Theory (Part \S\ref{sec:set_theory})



% ====================================================================
\section{Categorical Semantics}\label{sec:categorical_semantics}
% ====================================================================

Categorical Logic (\S\ref{sec:categorical_logic})



% ====================================================================
\section{Operational Semantics}\label{sec:operational_semantics}
% ====================================================================

\emph{Operational Semantics} are used in the Interpretation of
Programming Languages (Formalized Idealized Interpreter).

\emph{Concurrency}

Reduction Relation (\S\ref{sec:reduction_relation})



% --------------------------------------------------------------------
\subsection{Structural Semantics}\label{sec:structural_semantics}
% --------------------------------------------------------------------

Structural Equivalence (\S\ref{sec:structural_equality})



\subsubsection{Reduction Semantics}\label{sec:reduction_semantics}

Parallel Composition (\S\ref{sec:parallel_composition})



% --------------------------------------------------------------------
\subsection{Natural Semantics}\label{sec:natural_semantics}
% --------------------------------------------------------------------



% ====================================================================
\subsection{Truth-value Semantics}\label{sec:truthvalue_semantics}
% ====================================================================

\emph{Truth-value Semantics} is also called the \emph{Substitution
  Interpretation for Quantifiers} or \emph{Substitutional
  Quantification}.



% ====================================================================
\section{Composition Semantics}\label{sec:composition_semantics}
% ====================================================================

% --------------------------------------------------------------------
\subsection{Linear Semantics}\label{sec:linear_semantics}
% --------------------------------------------------------------------

\subsubsection{Game Semantics}\label{sec:game_semantics}

\emph{Game Semantics} studies the \emph{Dialogical} properties of
Semantics. It is used in connection with Intuitionistic Logic
(\S\ref{sec:intuitionistic_logic}) and Linear Logic
(\S\ref{sec:linear_logic}).

\emph{Refute}

Geometry of Interaction (\S\ref{sec:geometry_of_interaction})



% --------------------------------------------------------------------
\subsection{Independence-friendly Semantics}
\label{sec:independence_semantics}
% --------------------------------------------------------------------

Semantics of Independence-friendly Logics
(\S\ref{sec:independence_logic}) is defined by Game Semantics where
players have Imperfect Information.



% --------------------------------------------------------------------
\subsection{Denotational Semantics}\label{sec:denotational_semantics}
% --------------------------------------------------------------------

Expressions in a Programming Language are interpreted as Mathematics
Objects called \emph{Denotations}.

A Denotational Semantics for $\lambda$-calculus may be given by a
B\"om Tree (\S\ref{sec:bohm_tree}).

A Denotational Semantics of a Programming Language $L$ with Types as
Objects and Procedures as Morphisms, $\mathbf{C}(L)$, can be given as
a Functor into a Scott Domain $\mathbf{D}$: \cite{awodey06}
\[
  S : \mathbf{C}(L) \rightarrow \mathbf{D}
\]



% ====================================================================
\section{Probabilistic Semantics}\label{sec:probabilistic_semantics}
% ====================================================================

% ====================================================================
\section{Axiomatic Semantics}\label{sec:axiomatic_semantics}
% ====================================================================

\emph{Axiomatic Semantics} equates Meaning of a Sentence with the
Logical Formulas that describe it; that is what can be proven about it
in some System of Logic.

Axiomatic Semantics are used as a Formal Semantics of Programming
Languages.



% ====================================================================
\section{Applicative Universal Grammar}
\label{sec:applicative_universal_grammar}
% ====================================================================

% ====================================================================
\section{Semiotics}\label{sec:semiotics}
% ====================================================================

Saussurean Semiology (Sign/Syntax, Signal/Semantics)
\S\ref{sec:dyadic_sign},

[Signifier, Signified (Intension), Referent (Extension)]

Peircean (Sign, Object, Interpretant) \S\ref{sec:triadic_sign}

Morphism $\leftrightarrow$ Representation
(\S\ref{sec:representation_theory})



% --------------------------------------------------------------------
\subsection{Syntactics}\label{sec:syntactics}
% --------------------------------------------------------------------

Formal Language Theory (Part \ref{part:formal_language})

Formal Grammar (Syntax \S\ref{sec:formal_grammar})



\subsubsection{Signifier}\label{sec:signifier}

\emph{Sign} \emph{Symbol}



\paragraph{Morpheme}\label{sec:morpheme}

\paragraph{Dyadic Sign}\label{sec:dyadic_sign}

\paragraph{Triadic Sign}\label{sec:triadic_sign}



\subsubsection{Tagmeme}\label{sec:tagmeme}

\subsubsection{Clause}\label{sec:grammatical_clause}

smallest Grammatical Unit able to express a complete Proposition
(\S\ref{sec:proposition})

cf. \emph{Clause} \S\ref{sec:clause} (Logic)



\paragraph{Independent Clause}\label{sec:independent_clause}

\paragraph{Dependent Clause}\label{sec:dependent_clause}\hfill \\

\emph{Subordinate Clause}

Subjunctive (\S\ref{sec:subjunctive})

\subparagraph{Content Clause}\label{sec:content_clause}\hfill \\



\subsubsection{Modality}\label{sec:modality}

Sign-type



\subsubsection{Inscription}\label{sec:inscription}

An \emph{Inscription} is a concrete occurence of a Word



\subsubsection{Use-mention}\label{sec:use_mention}

A Word is \emph{Used} when it is used to Refer (\S\ref{sec:reference})
to something.

A Word is \emph{Mentioned} when it is used to Refer to the Signifier
itself, and is differentiated from being Use by italics or single
quotes.



\paragraph{Quasi-quote}\label{sec:quasi_quote}
\hfill \\

Substitutional Quantification (\S\ref{sec:truthvalue_semantics})



% --------------------------------------------------------------------
\subsection{Semantics}\label{sec:semantics}
% --------------------------------------------------------------------

\subsubsection{Extension}\label{sec:extension}\cite{chalmers02}

The Extension of a Term (\S\ref{sec:term}) is its Referent
(\S\ref{sec:referent}).

The Extension of a Sentence (\S\ref{sec:sentence}) is a Truth-value
(\S\ref{sec:truth_value})

The Extension of a Property (\S\ref{sec:property}) is a Truth Function
(\S\ref{sec:truth_function}) or a Relation (\S\ref{sec:relation}).

Expressions sharing the same Extension are called \emph{Co-extensive}.



\subsubsection{Intension}\label{sec:intension}\cite{chalmers02}

An \emph{Intension} may be generally defined as an Algorithm
(\S\ref{sec:algorithm}) or something to be \emph{Evaluated}, the
result of which gives the \emph{Extension} (\S\ref{sec:extension}).
% FIXME

Any Property (\S\ref{sec:property}) or quality Connoted
(\S\ref{sec:connotation}) by a Symbol or Expression.

\emph{Epistemic Intension}: Possibility (\S\ref{sec:possibility})
$\rightarrow$ Extension (\S\ref{sec:extension}), where the Possibility
is given as a Description (\S\ref{sec:description}).

\emph{Subjunctive Intension}: Counterfactual Possibility $\rightarrow$
Extension

\emph{Metaphysical Intension} %FIXME

\emph{Contextual Intension}

In Epistemic Intensions, Names (\S\ref{sec:rigid_designator}) and
Descriptions (\S\ref{sec:description}) have the same Epistemic
Intensions and Extensions, but in Subjunctive Intensions they may have
different Subjunctive Intensions.

\emph{Primary Intension}: Sense (\S\ref{sec:sense}), \emph{A
  posteriori}

\emph{Secondary Intension}: Reference (\S\ref{sec:reference}),
\emph{Necessary}

\emph{Type Meaning}

\emph{Token Meaning}



\paragraph{Comprehension}\label{sec:comprehension}
\hfill \\

All Intensions of an Object



\paragraph{Two-dimensional Intension}\label{sec:twodimensional_intension}
\hfill \\

\emph{Two-dimensional Intension}: $(V,W) \rightarrow$ Extensions,
where $V$ is an Epistemic Possibility and $W$ is a Metaphysical
Possibility.



\subsubsection{Meaning}\label{sec:meaning}

The \emph{Meaning} of Symbols and Expressions in a Formal Language is
assigned by an Interpretation (\S\ref{sec:interpretation}).



\paragraph{Sememe}\label{sec:sememe}
\hfill \\

Unit of Meaning (\S\ref{sec:meaning}), correlative to a Morpheme
(\S\ref{sec:morpheme})

Denotational (\S\ref{sec:denotation}):

\begin{enumerate}
  \item Primary Denotation
  \item Secondary Denotation
\end{enumerate}

Connotational (\S\ref{sec:connotation}):

\begin{enumerate}
  \item High Position
  \item Emotive
  \item Evaluative
\end{enumerate}



\subparagraph{Seme}\label{sec:seme}
\hfill \\

A single characteristic of a Sememe



\subparagraph{Episememe}\label{sec:episememe}
\hfill \\

Tagmeme (\S\ref{sec:tagmeme})



\paragraph{Sense}\label{sec:sense}
\cite{chalmers02}

Intension (\S\ref{sec:intension})

Relevant Condition on Extension (\S\ref{sec:extension}), cf.
Description (\S\ref{sec:description})

The Terms $a$ and $b$ have different Senses if and only if $a = b$ is
Non-trivial (i.e. not knowable \emph{A priori}). Two expressions have
different Senses if and only if a statement of their Co-extensiveness
(\S\ref{sec:extension}) is Non-trivial.

\subparagraph{Polysemy}\label{sec:polysemy}



\paragraph{Reference}\label{sec:reference}
\hfill \\

Extension (\S\ref{sec:extension})

Statements with the same Extension are called \emph{Co-extensive}

A Word used to Refer to something is said to be \emph{Used}
(\S\ref{sec:use_mention}).

\emph{Secondary Reference}

\subparagraph{Referent}\label{sec:referent}
\hfill \\

\emph{Co-referential}

\subparagraph{Absent Referent}\label{sec:absent_referent}

\subparagraph{Self-reference}\label{sec:self_reference}



\paragraph{Causal Theory}\label{sec:causal_reference}

\subparagraph{Semantic Externalism}\label{sec:semantic_externalism}



\subsubsection{Denotation}\label{sec:denotation}

A \emph{Denotation} is an \emph{Extension} (\S\ref{sec:extension}) of
a Term (\S\ref{sec:term})

Denotational Semantics (\S\ref{sec:denotational_semantics}



\paragraph{Rigid Designator}\label{sec:rigid_designator}
\hfill \\

same Denotation in all Possible Worlds (\S\ref{sec:possible_world})

\subparagraph{Name}\label{sec:name}
\hfill \\

A Computation (\S\ref{sec:computation_model}) is given a Name using a
Fixed Point Operator: $fix x is e$
%FIXME xref cite harper blog ``recursion''



\paragraph{Non-rigid Designator}\label{sec:nonrigid_designator}

\subparagraph{Description}\label{sec:description}
\cite{chalmers02}

Epistemic Condition on Extension (\S\ref{sec:extension}), cf. Sense
(\S\ref{sec:sense}), Intension (\S\ref{sec:intension})

A Description $D$ is \emph{Canonical} or \emph{Epistemically Complete}
if there is no $S$ such that both $D \wedge S$ and $D \wedge \neg S$
are Epistemically Possible, and that $D$ uses only Semantically
Neutral Expressions with Indexicals (\S\ref{sec:indexical}).

\subparagraph{Definite Description}\label{sec:definite_description}

\subparagraph{Indefinite Description}\label{sec:indefinite_description}



\subsubsection{Connotation}\label{sec:connotation}

\emph{Connotation} is a Second-order Denotation giving the
\emph{Intension} (\S\ref{sec:intension}) of an Object. The
\emph{Comprehension} (\S\ref{sec:comprehension}) is the collection of
all Intensions for an Object.



\subsubsection{Definition}\label{sec:definition}

\emph{Extensional Definition}

\emph{Intensional}

\emph{Identity} \emph{Determinable} \emph{Determinate}

\emph{Stipulative Definition}

\emph{Precising Definition}

\emph{Persuasive Definition}



\paragraph{Well-defined}\label{sec:well_defined}
\hfill \\

An Expression is \emph{Well-defined} if its Definition assigns a
unique Interpretation or Value.

A Function (\S\ref{sec:function}) is Well-defined if changing the
input representation while leaving the input Value unchanged gives the
same result.



\paragraph{Recursive Definition}\label{sec:recursive_definition}
\hfill \\

\emph{Recursive} = \emph{Inductive}

\begin{enumerate}
    \item Base case
    \item Inductive clause
    \item Extremal clause
\end{enumerate}

\emph{Structural Recursion}



\paragraph{Circular Definition}\label{sec:circular_definition}
\hfill \\

\emph{Homoiconic} (\S\ref{sec:homoiconicity})



\paragraph{Impredicative Definition}\label{sec:impredicative_definition}
\hfill \\

\emph{Self-referencing}



\subsubsection{Semiosis}\label{sec:semiosis}

\subsubsection{Semasiology}\label{sec:semasiology}

\subsubsection{Onamasiology}\label{sec:onamasiology}

Synchronic

Diachronic

\paragraph{Metonym}\label{sec:metonym}



% --------------------------------------------------------------------
\subsection{Pragmatics}\label{sec:pragmatics}
% --------------------------------------------------------------------

\subsubsection{Interpretant}\label{sec:interpretant}

\subsubsection{Indexical}\label{sec:indexical}

\subsubsection{Deixis}\label{sec:deixis}

\paragraph{Demonstrative}\label{sec:demonstrative}



% --------------------------------------------------------------------
\subsection{Computational Semiotics}\label{sec:computational_semiotics}
% --------------------------------------------------------------------

\subsubsection{Algebraic Semiotics}\label{sec:algebraic_semiotics}
