%%%%%%%%%%%%%%%%%%%%%%%%%%%%%%%%%%%%%%%%%%%%%%%%%%%%%%%%%%%%%%%%%%%%%%%%%%%%%%%%
%%%%%%%%%%%%%%%%%%%%%%%%%%%%%%%%%%%%%%%%%%%%%%%%%%%%%%%%%%%%%%%%%%%%%%%%%%%%%%%%
\part{Formal Semantics}\label{part:formal_semantics}
%%%%%%%%%%%%%%%%%%%%%%%%%%%%%%%%%%%%%%%%%%%%%%%%%%%%%%%%%%%%%%%%%%%%%%%%%%%%%%%%
%%%%%%%%%%%%%%%%%%%%%%%%%%%%%%%%%%%%%%%%%%%%%%%%%%%%%%%%%%%%%%%%%%%%%%%%%%%%%%%%

\emph{Formal Semantics} is the general study of Interpretations
(\S\ref{sec:interpretation}) of Formal Languages (Part
\ref{sec:formal_language}).

cf. Per Martin-L\"of - \emph{On the Meanings of the Logical Constants and
  Justification of Logical Laws}

2021 - Hutton - \emph{It's Easy As 1,2,3} -
\url{http://www.cs.nott.ac.uk/~pszgmh/123.pdf}



% ==============================================================================
\section{Interpretation}\label{sec:interpretation}
% ==============================================================================

An \emph{Interpretation} of a Formal Language is an assignment of
Meanings (\S\ref{sec:meaning}) to Symbols.

The Logical Symbols (\S\ref{sec:logical_constant}) are given the same
Meaning in every Interpretation, while the Meanings of Non-logical
Symbols (\S\ref{sec:nonlogical_constant}) are given by an Interpretation
Function (\S\ref{sec:interpretation_function}).

A \emph{Truth Functional Interpretation}

In Higher-order Logic (\S\ref{sec:higherorder_logic}) Quantifiers may
be Interpreted differently depending on the Semantics given: a
\emph{Full Semantics} Interprets Quantifiers as Ranging over all
possible Objects of a given Type (\S\ref{sec:type}), while a
\emph{Henkin Semantics} requires an separate Domain for each Type of
Higher-order Variable to Range over.

cf. ``\emph{Hermeneutics}'' -- theory and methodology of Interpretation



% ------------------------------------------------------------------------------
\subsection{Semantic Consequence}\label{sec:semantic_consequence}
% ------------------------------------------------------------------------------

\emph{Semantic Consequence} of a Theory (\S\ref{sec:formal_theory})
$\mathcal{T}$ of a Formal System $\mathcal{S}$:
\[
  \mathcal{T} \vDash_{\mathcal{S}} \varphi
\]
is defined for Sentences $\varphi$ such that any Interpretation
$\mathcal{I}$ that Satisfies (\S\ref{sec:satisfaction}) $\mathcal{T}$
also Satisfies $\varphi$:
\[
  \mathcal{I} \models \mathcal{T}
  \Rightarrow \mathcal{I} \models \varphi
\]
That is, $\varphi$ is True in all Models of $\mathcal{T}$.

A Tautology (\S\ref{sec:tautology}) is expressed as:
\[
    \vDash {\varphi}
\]
where a Sentence $\varphi$ is the Semantic Consequence of the Empty
Set.



% ------------------------------------------------------------------------------
\subsection{Valuation}\label{sec:valuation}
% ------------------------------------------------------------------------------

A \emph{Valuation} is the assignment of a Truth-value
(\S\ref{sec:truth_value}) to each Sentence (\S\ref{sec:sentence}) in a
Formal Language that follows a \emph{Truth Schema} (or \emph{T-schema}
\S\ref{sec:t_schema}).

In Propositional Logic a Valuation results from the Truth Assignment
(Interpretation Function \S\ref{sec:interpretation_function}) giving
Truth-values to each Propositional Variable which induces Truth-values
for all Propositional Formulas according to the Truth Functional
Interpretation of the Logical Constants of the Language.

In First-order Logic a Structure (\S\ref{sec:structure}) gives the
Domain of Discourse over which Quantifiers Range and an Interpretation
Function for Function and Predicate Symbols results in unique
assignment of Truth-values to all Sentences of the Language.

A Valuation $v : L \rightarrow \{true, false\}$ on a Formula $\varphi
\in L$ of a Language $L$ may be denoted $v(\varphi) = [[\varphi]]_v$.



\subsubsection{Supervaluation}\label{sec:supervaluation}

A \emph{Supervaluation} is a Valuation that does not assign
Truth-values to Vacuous Truths.

A Supervaluation $V(\varphi)$ is Undefined if there are exactly two
Valuations $v$ and $v'$ such that $v(\varphi) = true$ and $v'(\varphi)
= false$.

Valid Sentences still have Truth-values even if the constituent Atomic
Formulas do not and under Supervaluationism may be called
\emph{Supertrue}. That is, a Supervaluation $V(\varphi)$ is Supertrue
if and only if $v(\varphi) = true$ for every Valuation $v$. The dual
notion of \emph{Superfalse} is defined for a Supervaluation $V(\psi)$
such that $v(\psi) = false$ for every Valuation $v$.



% ------------------------------------------------------------------------------
\subsection{Semantic Validity}\label{sec:semantic_validity}
% ------------------------------------------------------------------------------

An Inference (\S\ref{sec:logical_inference}) is \emph{Semantically
  Valid} if all Interpretations that Satisfy the Premises also Satisfy
the Conclusion. \cite{gamut91}



% ------------------------------------------------------------------------------
\subsection{Intended Interpretation}\label{sec:intended_interpretation}
% ------------------------------------------------------------------------------

Standard Model (\S\ref{sec:standard_model})



% ------------------------------------------------------------------------------
\subsection{Herbrand Interpretation}\label{sec:herbrand_interpretation}
% ------------------------------------------------------------------------------



% ==============================================================================
\section{Semantic Truth}\label{sec:semantic_truth}
% ==============================================================================

\emph{Semantic Truth} is a conception of Logical Truth
(\S\ref{sec:logical_truth}) as a Property (\S\ref{sec:property}) of
Sentences (\S\ref{sec:sentence}).

The two related conceptions of Truth are the \emph{Correspondence
  Theory} (\S\ref{sec:correspondence_truth}) and \emph{Deflationary
  Theory} (\S\ref{sec:deflationary_truth}).

Briefly, the Undefinability Theorem
(\S\ref{sec:undefinability_theorem}) results in a \emph{Material
  Adequacy Condition} (called \emph{Convention T}) that any Theory of
Truth must entail:
\[
    \forall P (\mathrm{True}(S) \leftrightarrow P)
\]
where $S$ is the name of the Sentence $P$ in the Metalanguage which is
an Interpretation of $P$ in the Object Language. This is the
\emph{T-schema} (\S\ref{sec:t_schema}) used in \emph{Tarski's Semantic
  Theory of Truth} to Inductively define Truth, expressed as a
First-order Sentence. When a Modal Logic is based on the T-schema it
is said to give rise to \emph{T-Theory}. Tarski's Semantic Theory of
Truth is used as the definition for Truth in \emph{Model Theory}
(Part \ref{part:model_theory}).

An example sentence conforming to Convention T in Natural Language
where the Object Language is German and the Metalanguage is English:
\begin{description}
  \item ``\emph{Der Schnee ist wei\ss} is True if and only if snow is
    white''.
\end{description}
Here the right side of the Biconditional ('snow is white') is the
\emph{Truth-condition} of the left side.

Tarski considered this definition of Truth to be a type of
Correspondence Theory.

\emph{Dialetheism}



% ------------------------------------------------------------------------------
\subsection{T-schema}\label{sec:t_schema}
% ------------------------------------------------------------------------------

\emph{T-schema} (or \emph{Equivalence Schema}) is used to give an
Inductive definition of Truth.



% ------------------------------------------------------------------------------
\subsection{Correspondence Theory}\label{sec:correspondence_truth}
% ------------------------------------------------------------------------------

The \emph{Correspondence Theory of Truth} defines Truth of a Statement
by its relation and Correspondence with the world.



% ------------------------------------------------------------------------------
\subsection{Coherence Theory}\label{sec:coherence_theory}
% ------------------------------------------------------------------------------

The \emph{Coherence Theory of Truth} defines Truth of a Statement by its
relation to other Statements.

\fist Coherence (\S\ref{sec:coherence})



% ------------------------------------------------------------------------------
\subsection{Deflationary Theory}\label{sec:deflationary_truth}
% ------------------------------------------------------------------------------

A \emph{Deflationary Theory of Truth} is one that states that
ascribing Truth to a Statement does not attribute a property of Truth
to any such Statement in one of a number of different ways below.



\subsubsection{Redundancy Theory}\label{sec:redundancy_theory}

The \emph{Redundancy Theory of Truth} states that the Predicate of
Truth is Redundant in that it is equal to the Statement it is applied
to.\cite{ramsey27} Essentially, Truth is a \emph{periphrasis} of the
Sentence it is applied to.



\paragraph{Disappearance Theory}\label{sec:disappearance_theory}\hfill

A \emph{Disappearance Theory of Truth} states that Truth is both
Redundant and there is no such Property of Truth. A.J. Ayer is known
for this Theory.



\subsubsection{Performative Theory}\label{sec:performative_theory}

The \emph{Performative Theory of Truth} is a Deflationary Theory that
sees the Predicate of Truth as a signal of agreement with the
Statement, for such reasons as arriving at Consensus or such others.



\subsubsection{Disquotational Theory}\label{sec:disquotational_theory}

The \emph{Disquotational Theory of Truth} is a Deflationary
interpretation of Tarski's definition of Truth by W.V.O. Quine. It
states the Truth predicate has the effect of \emph{Dereferencing}
Sentences (removing the quotation marks). So
\[
    S \leftrightarrow True(``S``)
\]
that is, $S$ is equivalent to \emph{``$S$'' is true}.

The effect of adding \emph{is True} to an Assertion is then to convert
the Use of the Assertion to a Mention.



\subsubsection{Prosententialism}\label{sec:prosententialism}

\emph{Prosententialism} denies that ``is true'' is a Predicate and is
instead a \emph{Prosentence} (the Sentential analog to
\emph{Pronouns}) that stands in for another Sentence.



\subsubsection{Minimalist Theory}\label{sec:minimalist_theory}

\emph{Minimalism} defines Truth as a \emph{Metalinguistic} property
and that only Propositoins are Truth-bearing.



% ------------------------------------------------------------------------------
\subsection{Normative Theory}\label{sec:normative_theory}
% ------------------------------------------------------------------------------

A \emph{Normative Theory of Truth} states that Truth is the Normative
goal of Assertion.



% ------------------------------------------------------------------------------
\subsection{Undefinability Theorem}\label{sec:undefinability_theorem}
% ------------------------------------------------------------------------------

\emph{Tarski's Undefinability Theorem} \cite{tarski36} uses the same
techniques as G\"odel's Incompleteness Theorems to show that Truth
cannot be defined in an Object Language.



% ==============================================================================
\section{Model-theoretic Semantics}\label{sec:model_semantics}
% ==============================================================================

% ------------------------------------------------------------------------------
\subsection{Tarskian Semantics}\label{sec:tarski_semantics}
% ------------------------------------------------------------------------------

\emph{Semantic Theory of Truth}



% ------------------------------------------------------------------------------
\subsection{Truth-conditional Semantics}
\label{sec:truth_conditional_semantics}
% ------------------------------------------------------------------------------

\subsubsection{Truth-condition}\label{sec:truth_condition}



% ------------------------------------------------------------------------------
\subsection{Frame Semantics}\label{sec:frame_semantics}
% ------------------------------------------------------------------------------

\emph{Frame Semantics} is the extension of Model Theory to
Non-classical Logic Systems, beginning with Modal Logic
(\S\ref{sec:modal_logic}).



\subsubsection{Kripke Semantics}\label{sec:kripke_semantics}

\emph{Kripke Semantics} (or \emph{Relational Semantics})

\begin{description}
\item [Kripke Frame] (\S\ref{sec:kripke_frame}) $\langle W,R \rangle$
  where $W$ is a Non-empty Set of \emph{Nodes} (\emph{Worlds}) and $R$
  is a Binary Relation called the \emph{Accessibility Relation}
\item [Kripke Model] $\langle W,R,\Vdash \rangle$ where $\Vdash$ is a
  \emph{Forcing Relation} for Nodes of $W$
\end{description}
The Accessbility Relation is described by Modal Axioms that specify on
what basis a Statement is True in \emph{any} Possible World.

The Forcing Relation $\Vdash$ (read as Satisfies or \emph{Forces}
(\S\ref{sec:forcing})) has the following properties:
\begin{itemize}
\item $w \Vdash \neg A$ if and only if $w \nVdash A$
\item $w \Vdash A \rightarrow B$ if and only if $w \nVdash A$ or $w
  \Vdash B$
\item $w \Vdash \square A$ if and only if $u \Vdash A$ for all $u$
  such that $w R u$
\end{itemize}
\emph{Validity} of a Proposition is defined for
\begin{itemize}
\item Model $\langle W,R, \Vdash \rangle$ if $\forall w \in W,
  w \Vdash A$
\item Frame $\langle W,R \rangle$ if Valid in Model $\langle W,R,
  \Vdash \rangle$ for all choices of $\Vdash$
\item Class of Frames or Models, $C$, if Valid for all Frames or
  Models of the Class
\end{itemize}
$Thm(C)$ is defined as the Set of all Formulas Valid
(\S\ref{sec:validity}) in $C$. For a Set of Formulas $X$,
$Mod(X)$ is defined as the Class of all Frames which Validate every
Formula in $X$.

Kripke Models are a special case of Labelled State Transition Systems
(\S\ref{sec:state_transition}).

A Modal Theory, $T$, is \emph{Sound} (\S\ref{sec:soundness}) with
respect to a Class of Frames, $C$, if $T \subseteq Thm(C)$. $T$ is
\emph{Complete} with respect to $C$ if $T \supseteq Thm(C)$.

For any $C$, $Thm(C)$ is a \emph{Normal Modal Logic}
(\S\ref{sec:alethic_logic}). A Normal Modal Logic is said to
\emph{Correspond} to $C$ if $C = Mod(L)$.

The \emph{Canonical Model} of a Modal Theory $T$ is a Kripke Model
$\langle W,R, \Vdash \rangle$ where $W$ is the Set of all Maximally
Consistent Sets for $T$ (\S\ref{sec:formal_theory}) and:
\begin{itemize}
\item $XRY$ if and only if for all Formulae $A$, if $\square A
  \in X$ then $A \in Y$
\item $X\Vdash A$ if and only if $A \in X$
\end{itemize}

%FIXME

\emph{Unravelling}

\emph{Filtration}



\paragraph{Kripke Frame}\label{sec:kripke_frame}\hfill

2019 - Greco, Liang, Moortgat, Palmigiano - \emph{Vector Spaces as Kripke
  Frames}: a Sound Semantics for Associative, Commutative, Unital Lambek
Calculus (\S\ref{sec:lambek_grammar}) can be based on Vector Spaces by
interpreting Fusion as the Tensor Product of Vector Spaces



\subparagraph{General Frame}\label{sec:general_frame}\hfill

Kripke Frame with additional structure used to Model Modal
(\S\ref{sec:modal_logic}) and Intermediate Logic
(\S\ref{sec:intermediate_logic}).

\emph{J\'onsson-Tarski Duality} --
generalization of Stone Representation Theorem for Boolean Algebras
(\S\ref{sec:boolean_algebra}) to Modal Algebras (\S\ref{sec:modal_algebra});
each Modal Algebra can be Represented (\S\ref{sec:representation}) as the
Algebra of Admissible Sets (\S\ref{sec:admissible_set} in a Modal General Frame;
$(\cdots)^+$ and $(\cdots)_+$ are Contravariant Functors between the Category of
General Frames and the Category of Modal Algebras



\paragraph{Kripke Model}\label{sec:kripke_model}\hfill

\paragraph{Possible World Semantics}\label{sec:possible_world}\hfill

cf. Modal Probabilistic Logic (\S\ref{sec:probabilistic_logic}) --
(\url{https://plato.stanford.edu/entries/logic-probability/#ModProLog})

\fist Barcan Formula (\S\ref{sec:barcan_formula}); \emph{Actualism} -- ``there
are no \emph{merely} Possible individuals'': in terms of Possible World
Semantics (\S\ref{sec:possible_world}), the Barcan Formula implies that all
objects which ``exist'' in any Possible World that is ``\emph{Accessible}'' to
the ``Actual World'', exist in the Actual World; cf. \emph{Possibilism} (there
are some entities that are \emph{merely possible})



\subparagraph{Possibility}\label{sec:possibility}\hfill

Logical Possibility

Nomological Possibility

Temporal Possibility

\subparagraph{Subjunctive}\label{sec:subjunctive}\hfill

\emph{Counterfactual}



\paragraph{Montague Grammar}\label{sec:montague_grammar}\hfill

\paragraph{Type-logical Semantics}\label{sec:typelogical_semantics}\hfill

\paragraph{Glue Semantics}\label{sec:glue_semantics}\hfill



\subsubsection{Neighborhood Semantics}
\label{sec:neighborhood_semantics}

\emph{Neighborhood Frame} $\langle W, N \rangle$



\subsubsection{General Frame Semantics}
\label{sec:general_frame_semantics}

A \emph{Modal General Frame} is defined as a triple $\mathbf{F} =
\langle W,R,V \rangle$ where $\langle W,R \rangle$ is a Kripke Frame
and $V$ is a Set of Subsets of $W$ closed under Intersection, Union,
Complement, and the Operation $\square$ defined as
\[
    \square A = \{x \in W; \forall y \in W ( x R y \rightarrow y \in A ) \}
\]



\subsubsection{Intuitionistic Semantics}
\label{sec:intuitionistic_semantics}

The Semantics of Intuitionistic Logic
(\S\ref{sec:intuitionistic_logic}) is definable as Kripke Semantics
(\S\ref{sec:kripke_semantics}) or \emph{Heyting Algebra Semantics}.



\paragraph{Heyting Algebra Semantics}\label{sec:heyting_semantics}\hfill

Instead of assigning Valuations from a Boolean Algebra, Intuitionistic
Semantics uses Values from a \emph{Heyting Algebra}
(\S\ref{sec:heyting_algebra}). A Formula is Valid if and only if it
receives the Value of the Top Element for any Valuation.



\paragraph{Brouwer-Heyting-Kolmogorov Interpretation}\hfill
\label{sec:brouwer_heyting_kolmogorov}

\emph{BHK Interpretation}

Realizability (\S\ref{sec:realizability})

Realizability Models (\S\ref{sec:realizability_model})



\subsubsection{Kripke-Joyal Semantics}\label{sec:kripke_joyal}

\subsubsection{Boolean-valued Model}\label{sec:boolean_model}

\emph{Boolean-valued Models} are related to Heyting Algebras and
Intuitionistic Logic and is equivalent to the method of Forcing.

%FIXME ref complete boolean algebra
Instead of limiting Formulas in a System $S$ to True or False, they
may be assigned values from a fixed \emph{Complete Boolean Algebra}.



% ==============================================================================
\section{Proof-theoretic Semantics}\label{sec:proof_semantics}
% ==============================================================================

\emph{Proof-theoretic Semantics} is based on the Propositions and
Logical Connectives of Systems of Inference (Part
\ref{sec:formal_system}). See also \emph{Semantic Tableau}
(\S\ref{sec:tableau_calculus}).

Meaning Explanation (\S\ref{sec:meaning_explanation})



% ------------------------------------------------------------------------------
\subsection{Logical Harmony} \label{sec:logical_harmony}
% ------------------------------------------------------------------------------

\emph{Logical Harmony} refers to constraints required between
Introduction and Elimination Rules.



% ==============================================================================
\section{Algebraic Semantics}\label{sec:algebraic_semantics}
% ==============================================================================

\url{https://golem.ph.utexas.edu/category/2018/05/linear_logic_for_constructive.html}:

$\vee$ and $\wedge$ can be regarded as Axioms on a Poset, with $\wedge$ as
Binary Meet (GLB) and $\vee$ as Join (LUB) and the laws of Classical Logic
require the Poset to be a Boolean Algebra; a Proof in Propositional Logic shows
an Equation that must hold in all Boolean Algebras

a Heyting Algebra is a Lattice that when considered as a Thin Category is
Cartesian Closed, i.e. in addition to $\vee$ and $\wedge$, it has Implication,
$\rightarrow$; any Boolean Algebra is a Heyting Algebra with $(P \rightarrow Q)
\equiv (\neg{P}\vee{Q})$, but not conversely, e.g. the Open-set Lattice of a
Topological Space is a Heyting Algebra but not generally a Boolean Algebra; a
Proof in Intuitionistic Logic shows an Equation that must hold in all Heyting
Algebras

Intuitionistic Linear Logic with Tensor Product $\otimes$ and Internal-hom
$\multimap$ corresponds to Closed Symmetric Monoidal Lattices instead of
Cartesian Closed; Star-autonomous Lattice is a Closed Symmetric Monoidal
Lattice with an Object $\bot$ (not necessarily the Bottom Element) such that $P
\equiv (P \multimap \bot) \multimap \bot$, which also has a Cotensor Product
with Internal-hom definable in terms of the Cotensor Product, $(P \multimap Q)
\equiv (P^\bot \parr Q)$, with corresponding Logic as Classical Linear Logic
having two Conjunctions $\wedge$, $\otimes$ (Additive and Multiplicative,
resp.) and two Disjunctions $\vee$, $\parr$ (Additive and Multiplicative,
resp.); Closed Symmetric Monoidal Lattices include all Heyting Algebras

Meaning Explanation for Classical Linear Logic:
\begin{itemize}
  \item a Proof of $P \wedge Q$ is a Proof of $P$ together with a Proof of $Q$
    and a Refutation of $P \wedge Q$ is either a Refutation of $P$ or a
    Refutation of $Q$
  \item a Proof of $P \vee Q$ is a Proof of $P$ or a Proof of $Q$ and a
    Refuration of $P \vee Q$ is a Refutation of $P$ together with a Refutation
    of $Q$
  \item a Proof of $P^\bot$ is a Refutation of $P$ and a Refutation of $P^\bot$
    is a Proof of $P$
  \item a Proof of $P \otimes Q$ is a Proof of $P$ together with a Proof of $Q$
    and a Refutation of $P \otimes Q$ is a Construction of a Refutation of $Q$
    from any Proof of $P$ and a Construction of a Refutation of $P$ from any
    Proof of $Q$
  \item a Proof of $P \parr Q$ is a Construction of a Proof of $Q$ from any
    Refutation of $P$ and a Construction of a Proof of $P$ from any Refutation
    of $Q$, and a Refutation of $P \parr Q$ is a Refutation of $P$ together
    with a Refutation of $Q$
\end{itemize}

(FIXME: clarify ``Construction'')

\begin{itemize}
\item the Variety (\S\ref{sec:variety}) of all Modal Algebras
  (\S\ref{sec:modal_algebra}) is the equivalent Algebraic Semantics of the Modal
  $K$ Logic (\S\ref{sec:modal_logic}), in the sense of Abstract Algebraic Logic
  (\S\ref{sec:abstract_algebraic_logic})
\end{itemize}


% ------------------------------------------------------------------------------
\subsection{Boolean Algebra}\label{sec:boolean_algebra}
% ------------------------------------------------------------------------------

(wiki):

a Power Set (\S\ref{sec:powerset}) with the Operations of Union, Intersection,
and Complement forms a Boolean Algebra

any Finite Boolean Algebra is Isomorphic to the Boolean Algebra of the Power Set
of a Finite Set

(see Stone's Representation Theorem with regards to Infinite Boolean Algebras)

the Propositional Calculi of Classical Mechanics are Boolean Algebras (Birkhoff,
Von Neumann36)

\fist Boolean Differential Calculus (\S\ref{sec:bdc})

$Ult(B) \cong Hom_\mathbf{BA}(B,\mathbf{2})$ Ultrafilters
(\S\ref{sec:ultrafilter})

any Control Category (\ref{sec:control_category}) in which $\parr$ is
Bifunctorial is equivalent to a Boolean Algebra \cite{selinger01}

for every Boolean Algebra $B$, the \emph{Stone Space} (Profinite Space
\S\ref{sec:stone_space}) $\xspace{S}(B)$ is a Compact Totally Disconnected
Separated (Hausdorff) Space (\S\ref{sec:separated_space}), and conversely given
any Topological Space $X$, the collection of Clopen Subsets
(\S\ref{sec:clopen_set}) of $X$ is a Boolean Algebra

Representation Theory (\S\ref{sec:representation_theory}) of Boolean Algebras
(\S\ref{sec:boolean_algebra}) -- every Boolean Algebra can be Represented as a
Field of Sets (Set Algebra \S\ref{sec:set_algebra})

every Complemented (\S\ref{sec:complemented_lattice}) Distributive Lattice
(\S\ref{sec:distributive_lattice}) has a unique Orthocomplementation and is a
Boolean Algebra

(Birkhoff35): any Finite-dimensional Complemented Modular Lattice
(\S\ref{sec:modular_lattice}) is the Direct Product of a Finite number of
Abstract Projective Geometries (\S\ref{sec:projective_geometry}) and a Finite
Boolean Algebra, and such a
Lattice is a single Projective Geometry if and only if it is Irreducible
(FIXME: correct ?; see Birkhoff, Von Neumann36 \S 12)

\emph{Stone Representation Theorem} -- every Boolean Algebra $B$ is Isomorphic
to a certain Field of Sets, viz. the Algebra of Clopen Subsets of its Stone
Space $\xspace{S}(B)$ sending an Element $b \in B$ to the Set of all
Ultrafilters that contain $b$; Category Theoretic statement: there is a Duality
between the Category of Boolean Algebras, $\cat{BA}$, and the Category of Stone
Spaces, additionally implying that each Homomorphism from a Boolean Algebra $A$
to a Boolean Algebra $B$ corresponds in a ``natural way'' to a \emph{Continuous
  Function} (\S\ref{sec:continuous_function}) from $S(B)$ to $S(A)$, i.e. there
is a Contravariant Functor (\S\ref{sec:contravariant_functor}) that gives an
Equivalence between Categories (a Nontrivial Duality of Categories)

\fist Representation Theory (\S\ref{sec:representation_theory})

\fist Stone Representation Theorem for Boolean Algebras is a special case of
\emph{Stone Duality} (\S\ref{sec:stone_duality}) for Dualities between
Topological Spaces (\S\ref{sec:topological_space}) and Partially Ordered Sets
(\S\ref{sec:poset})

\emph{Loomis-Sikorski Theorem} -- a Stone-type Duality between Countably
Complete Boolean Algebras (``\emph{Abstract $\sigma$-algebras}''
\S\ref{sec:sigma_algebra}) and Measurable Spaces (\S\ref{sec:measurable_space})

\fist \emph{J\'onsson-Tarski Duality} --
generalization of Stone Representation Theorem to Modal Algebras
(\S\ref{sec:modal_algebra}) and General Frames (\S\ref{sec:general_frame})



\subsubsection{Truth-table}\label{sec:truth_table}

\subsubsection{Two-element Boolean Algebra}\label{sec:twoelement_boolean}

Propositional Logic (\S\ref{sec:propositional_logic})

Syntactically, every Boolean Term corresponds to a Propositional
Formula



\subsubsection{Boolean Algebra with Operators}\label{sec:boolean_with_operators}

Propositional Modal Logic (\S\ref{sec:modal_logic})



\subsubsection{Monadic Boolean Algebra}\label{sec:monadic_boolean}

Modal Logic $\mathrm{S5}$ (\S\ref{sec:modal_logic})

Monadic Predicate Logic (\S\ref{sec:monadic_firstorder})



\subsubsection{Complete Boolean Algebra}\label{sec:complete_boolean}

First-order Logic (\S\ref{sec:firstorder_logic})



% ------------------------------------------------------------------------------
\subsection{Conditional Event Algebra}\label{sec:conditional_event_algebra}
% ------------------------------------------------------------------------------

% ------------------------------------------------------------------------------
\subsection{Heyting Algebra}\label{sec:heyting_algebra}
% ------------------------------------------------------------------------------

A \emph{Heyting Algebra} is a Poset with:
\begin{enumerate}
  \item Finite Meets $1$ and $p \wedge q$
  \item Finite Joins $0$ and $p \vee q$
  \item Exponentials $a \Rightarrow b$ for each $a,b$ such that $a
    \wedge b \leq c$ if and only if $a \leq b \Rightarrow c$
\end{enumerate}
A Heyting Algebra is a Distributive Lattice
(\S\ref{sec:distributive_lattice}), but not every Distributive Lattice
is a Heyting Algebra. A Lattice is Distributive if and only if it is Isomorphic
to a \emph{Lattice of Sets} (Closed under Set Union and Intersection)

A Boolean Algebra is a Heyting Algebra with a Complementary Negation,
that is, in a Heyting Algebra it is not the case that (in a Preorder):
\[
  \top \leq A \vee \overline{A}
\]
which implies that there are Undecidable Propositions.

Cartesian Closed Preorder (\S\ref{sec:cartesian_preorder})

Algebraic equivalent of Intuitionistic Propositional Logic
(\S\ref{sec:intuitionistic_logic})

\emph{Topological Models} for Intuitionistic Logic in terms of Heyting Algebras
(Tarsky,Stone); any Heyting Algebra can be ``turned into a Space''
\url{http://comonad.com/reader/2018/computational-quadrinitarianism-curious-correspondences-go-cubical/}

Theorem (Intuitionistic Propositional Logic does not refute the Law of
the Excluded Middle)\cite{harper12}:
\[
  \forall A, \neg (\neg (A \vee \neg A))
\]



\subsubsection{Complete Heyting Algebra}\label{sec:complete_heyting_algebra}

Heyting Algebra that is a Complete Lattice
(\S\ref{sec:complete_lattice}) with the Infinite Distributive Law
(\S\ref{sec:infinite_distributive})

Frame (\S\ref{sec:frame})

Locale (\S\ref{sec:locale}) -- Opposite Category of Frames; a generalization of
Topological Spaces

A Homomorphism of Heyting Algebras is a Morphism of Frames that also
preserves Implication.

Complete Heyting Algebras are Objects in:
\begin{itemize}
  \item $\cat{CHey}$ -- Morphisms are Homomorphisms of Complete
    Heyting Algebras
  \item $\cat{Loc}$
  \item $\cat{Frm}$ -- Opposite Category of $\cat{Loc}$; Morphisms are
    (necessarily Monotone) Functions that Preserve Finite Meets and
    arbitrary Joins.
\end{itemize}

classical Stone Duality (\S\ref{sec:stone_duality}): Duality between the
Category $\cat{Sob}$ of Sober Spaces (\S\ref{sec:sober_space}) with Continuous
Functions and the Category $\cat{SFrm}$ of Spatial Frames (Complete Heyting
Algebras) and Frame Homomorphisms

examples:
\begin{itemize}
  \item Lindembaum-Tarski Algebras (\S\ref{sec:lindenbaum_tarski}) -- abstracted
    to Abstract Algebraic Logic (\S\ref{sec:abstract_algebraic_logic})
\end{itemize}



% ------------------------------------------------------------------------------
\subsection{Modal Algebra}\label{sec:modal_algebra}
% ------------------------------------------------------------------------------

(wiki):

Algebraic Structure (\S\ref{sec:algebraic_structure}) $\langle{A, \wedge, \vee,
  -, 0, 1, \square}\rangle$ where $\langle{A, \wedge, \vee, -, 0, 1}\rangle$ is
a Boolean Algebra and $\square$ is a Unary Operation on $A$ Satisfying
$\square{1} = 1$ and $\square(x \wedge y) = \square x \wedge \square y$ for all
$x, y \in A$

provides Models of Propositional Modal Logic (\S\ref{sec:modal_logic}) in the
same way that Boolean Algebras (\S\ref{sec:boolean_algebra}) are Models of
Classical Logic (\S\ref{sec:classical_logic})

\emph{J\'onsson-Tarski Duality} -- generalization of Stone Representation
Theorem for Boolean Algebras to Modal Algebras; each Modal Algebra can be
Represented (\S\ref{sec:representation}) as the Algebra of Admissible Sets
(\S\ref{sec:admissible_set} in a Modal General Frame
(\S\ref{sec:general_frame}); $(\cdots)^+$ and $(\cdots)_+$ are Contravariant
Functors between the Category of General Frames and the Category of Modal
Algebras

the Variety (\S\ref{sec:variety}) of all Modal Algebras
(\S\ref{sec:modal_algebra}) is the equivalent Algebraic Semantics
(\S\ref{sec:algebraic_semantics}) of the Normal Modal Logic $\mathrm{K}$
(\S\ref{sec:normal_modal}), in the sense of Abstract Algebraic Logic
(\S\ref{sec:abstract_algebraic_logic}), and the Lattice (\S\ref{sec:lattice}) if
its Subvarieties is ``Dually Isomorphic'' (FIXME: clarify) to the Lattice of
Normal Modal $\mathrm{K}$ Logics



\subsubsection{Interior Algebra}\label{sec:interior_algebra}

Interior Algebras are to Topology (Part \S\ref{part:topology}) and Modal
$\mathsf{S4}$ Logic (\S\ref{sec:modal_logic}) what Boolean Algebras
(\S\ref{sec:boolean_algebra}) are to Set Theory (Part \ref{part:set_theory}) and
ordinary Propositional Logic (\S\ref{sec:propositional_logic})

Interior Algebras are a Variety (\S\ref{sec:variety}) of Modal Algebras

Dual to Closure Algebra (\S\ref{sec:closure_algebra})



\subsubsection{Closure Algebra}\label{sec:closure_algebra}

McKinsey, Tarski ``Algebra of Topology''-- describe Topological Spaces in
Algebraic (Axiomatic) Terms
\url{http://comonad.com/reader/2018/computational-quadrinitarianism-curious-correspondences-go-cubical/}

Semantics of S4 Modal Logic (\S\ref{sec:modal_logic})

Dual to Interior Algebra (\S\ref{sec:interior_algebra})



% ------------------------------------------------------------------------------
\subsection{MV-algebra}\label{sec:mv_algebra}
% ------------------------------------------------------------------------------

Lukasiewicz Logic (\S\ref{sec:lukasiewicz_logic})



% ------------------------------------------------------------------------------
\subsection{Cylindric Algebra}\label{sec:cylindric_algebra}
% ------------------------------------------------------------------------------

First-order Logic with Equality (\S\ref{sec:firstorder_equality})



% ------------------------------------------------------------------------------
\subsection{Polyadic Algebra}\label{sec:polyadic_algebra}
% ------------------------------------------------------------------------------

First-order Logic without Equality
(\S\ref{sec:firstorder_no_equality})



% ------------------------------------------------------------------------------
\subsection{Relation Algebra}\label{sec:relation_algebra}
% ------------------------------------------------------------------------------

Set Theory (Part \S\ref{part:set_theory})



% ==============================================================================
\section{Categorical Semantics}\label{sec:categorical_semantics}
% ==============================================================================

1963 - Lawvere - \emph{Functorial Semantics of Algebraic Theories} -- Lawvere
Theories (\S\ref{sec:lawvere_theory}) give Theories of Structures with $n$-ary
Operations $f : X^n \rightarrow X$

1965 - MacLane - \emph{Categorical Algebra} -- PROPs (PROducts and Permutations
Categories \S\ref{sec:prop_category}) give Theories of Structures with
Operations $f : X^{\otimes m} \rightarrow X^{\otimes n}$ (e.g. Vector Spaces)

\fist Categorical Logic (\S\ref{sec:categorical_logic})

nLab:

Interpretation of Formal Logic (Part \ref{part:formal_logic}) as a
Formal Language for talking about the collection of Monomorphisms into
a given Object of a given Category, i.e. the Subobject Poset
(\S\ref{sec:subobject_poset}) of that Object.

Interpretation of Type Theory (Part \ref{part:type_theory}, Dependent
Type Theory \S\ref{sec:dependent_type_theory}) as being a Formal
Language for talking about Slice Categories
(\S\ref{sec:slice_category}), i.e. all Morphisms into a given Object.

A Mode Theory (\S\ref{sec:mode_theory}) specifying Context Descriptors
and Structural Properties is analyzed as a $2$-dimensional Cartesian
Multicategory (\S\ref{sec:cartesian_multicategory})

A given Formal Theory (\S\ref{sec:formal_theory}) or Type Theory has a
Categorical Semantics if there is a Category such that the given
Theory is that of its Slice Categories, i.e. the Internal Logic
(\S\ref{sec:internal_logic}) of that Category.


\asterism


(Caramello - Introduction to Categorical Logic) %FIXME cite

Interpreting $\Sigma$-structure $\struct{M}$ in a Category $\cat{C}$
with Finite Products

Sorts, Function Symbols, Relation Symbols

Terms $t_1 = t_2$ -- Equalizers

Conjunction $\varphi \wedge \psi$ -- Pullbacks

$\top \vdash \varphi \vee \neg\varphi$ doesn't work in an arbitrary
Category (Topos) (Caramello - Introduction to Categorical Logic) --
Boolean Category/Topos?

a $\Sigma$-structure $\struct{M}$ is a Model of a Theory $\class{T}$
over Signature $\Sigma$ if all the Axioms of $\class{T}$ are Satsified
in $\struct{M}$, i.e. if the Interpretation of the Antecedent is
contained as a Subobject in the Interpretation of the Consequent

%FIXME

$\Sigma-str(\cat{C})$, $\class{T}-mod(\cat{C})$

Soundness, Completeness

$F : \cat{C} \rightarrow \cat{D}$

$\cat{C}_{\class{T}}$ -- Syntactic Category
(\S\ref{sec:syntactic_category})

$\class{T}-mod(\cat{C}) \simeq Hom(\cat{C}_{\class{T}},\cat{C})$
($\simeq$ is Equivalence of Categories
\S\ref{sec:category_equivalence})

$\struct{M} \rightarrow F(\struct{M})$

$\class{T}-mod(\cat{C}) \rightarrow \class{T}-mod(\cat{D})$

Algebraic Functor: Finite Product Preserving

Horn Functor: Algebraic andPreserves Finitary Disjunctions (Cartesian
Categories?)

Regular Functor: Horn and Image Preserving

Coherent Functor: Regular and Preserves Finite Unions of Subobjects

Geometric Functor: Coherent and Preserves Infinitary Disjunctions (?)
%FIXME

\emph{A Categorical Semantics for Causal Structure} -
\url{https://golem.ph.utexas.edu/category/2018/01/a_categorical_semantics_for_ca.html}
- $*$-autonomous extension of a Compact Closed Category



% ------------------------------------------------------------------------------
\subsection{Categorical Model}\label{sec:categorical_model}
% ------------------------------------------------------------------------------

or \emph{Semantic Universe}

Interaction Category (\S\ref{sec:interaction_category})



% ------------------------------------------------------------------------------
\subsection{Functorial Semantics}\label{sec:functorial_semantics}
% ------------------------------------------------------------------------------

\emph{Functorial Semantics of Algebraic Theories} -- (Lawvere63)

Categories with specifica extra Structure as ``Theories'' with
Structure-preserving Functors into other such Categories as ``Models''

Syntactic Expressions are Morphisms in some Category $\cat{X}$ and
Meanings are Morphisms in some other Category $\cat{Y}$ with a Functor
$F : \cat{X} \rightarrow \cat{Y}$ sending Syntax to Semantics



% ==============================================================================
\section{Operational Semantics}\label{sec:operational_semantics}
% ==============================================================================

\emph{Operational Semantics} are used in the Interpretation of
Programming Languages (Formalized Idealized Interpreter).

\emph{Structured Operational Semantics} (or \emph{Small-step Semantics})

\emph{Natural Semantics} (or \emph{Big-step Semantics})

\emph{Concurrency}

Reduction Relation (\S\ref{sec:reduction_relation})

Meaning of a Program is a Transition Function on a Virtual Machine

Labelled Transition System (\S\ref{sec:labelled_transition})

\fist Cf. Denotational Semantics (\S\ref{sec:denotational_semantics});
Denotational Semantics are less ``sensitive'' to changes in the
Presentation of a Language (Bakker-Vink96) %FIXME

PLT Redex: (Racket) Domain-specific Language for Specifying and
Debugging Operational Semantics

Meredith-Stay17 - \emph{Name-free Combinators for Concurrency} --
Combinator Calculus with no Bound Names into which Asynchronous
$\pi$-calculus has a Faithful Embedding with Multisorted Lawvere
Theories Enriched over Graphs as the Operational Semantics

Stay-Meredith17 \emph{Representing Operational Semantics with Enriched
Lawvere Theories}

\url{https://golem.ph.utexas.edu/category/2017/06/gphenriched_lawvere_theories_f.html}:

\url{https://golem.ph.utexas.edu/category/2017/06/eliminating_binders_for_easier.html}

Denotational Semantics: $\lambda$-calculus, Operational Semantics:
Turing Machines

Operational Semantics of a Programming Language:
\begin{enumerate}
  \item ``layout'' of the State of the ``Computer'' -- for each
    ``kind'' of Data in the description of the State, there is a
    \emph{Sort}, e.g.:
    \begin{itemize}
      \item in $\lambda$-calculus there is a Sort for Variables and a
        Sort for Terms and the Term is the entire State of the
        Computer
      \item a Turing Machine a Tape, a State Transition Table, Current
        State, and position of Read/Write Head on the Tape
      \item in a modern Programming Language the State is more complex
        with Stack, Heap, etc.
    \end{itemize}
  \item ``building up'' of the State using \emph{Term Constructors}
  \item \emph{Structural Congruence} (\S\ref{sec:structural_congruence}) --
    ``rearrangements'' of State that are ``ignored'', e.g.:
    \begin{itemize}
      \item in $\lambda$-calculus two Terms are the same if they only
        differ in the choice of Bound Variables
      \item in $\pi$-calculus the order in which Processes are listed
        makes no difference
    \end{itemize}
  \item \emph{Reduction Rules} describe how the State is allowed to
    change, e.g.:
    \begin{itemize}
      \item in $\lambda$-calculus the State only changes via
        $\beta$-reduction (Applying Functions to their Arguments, i.e.
        Substituting the Argument of a Function for the Bound Variable)
      \item in a Turing Machine each State may change in five ways
        (change Bit to 0 or 1, then move left or right; or halt)
      \item in $\pi$-calculus there may be multiple Transitions out of a
        State: if a Process is listening on a Channel for two Messages,
        either one may be processed first
    \end{itemize}
  \item \emph{Contexts} under which Reduction Rules apply, e.g.:
    \begin{itemize}
      \item in Weak-head Normal Form $\lambda$-calculus no Reduction
        happens under Abstraction, i.e. if a Term $t$ Reduces to $t'$,
        $\lambda x.t$ \emph{does not} Reduce to $\lambda x.t'$
    \end{itemize}
\end{enumerate}

\emph{Context Constructors}: Unary Morphisms for marking Reduction
Contexts

Multi-sorted $\cat{Gph}$-enriched Lawvere Theory:
\begin{enumerate}
  \item Generating Object for each Sort
  \item Generating Morphism for each Term Constructor and
    \emph{Context Constructor}
  \item Equation between Morphisms encoding Structural Congruence and
    \emph{Context Propagation}
  \item an Edge for each Rewrite in its appropriate Context
\end{enumerate}

Interpretation in $\cat{Gph}$:
\begin{enumerate}
  \item Sorts map to Graphs
  \item Term Constructors map to Graph Homomorphisms
  \item Equations map to Equations
  \item Rewrites map to ``Graph Transformations'' (genrealization of
    Natural Transformation to $\cat{Gph}$)
\end{enumerate}

system where given a Formal Semantics for a Language, one can
Algorithmically derive a Type System ``tailored'' to the Language



% ------------------------------------------------------------------------------
\subsection{Structural Semantics}\label{sec:structural_semantics}
% ------------------------------------------------------------------------------

Structural Equivalence (\S\ref{sec:structural_equality})

Plotkin



\subsubsection{Reduction Semantics}\label{sec:reduction_semantics}

Parallel Composition (\S\ref{sec:parallel_composition})



% ------------------------------------------------------------------------------
\subsection{Natural Semantics}\label{sec:natural_semantics}
% ------------------------------------------------------------------------------



% ==============================================================================
\subsection{Truth-value Semantics}\label{sec:truthvalue_semantics}
% ==============================================================================

\emph{Truth-value Semantics} is also called the \emph{Substitution
  Interpretation for Quantifiers} or \emph{Substitutional
  Quantification}.



% ==============================================================================
\section{Composition Semantics}\label{sec:composition_semantics}
% ==============================================================================

% ------------------------------------------------------------------------------
\subsection{Linear Semantics}\label{sec:linear_semantics}
% ------------------------------------------------------------------------------

\subsubsection{Game Semantics}\label{sec:game_semantics}

\emph{Game Semantics} studies the \emph{Dialogical} properties of
Semantics. It is used in connection with Intuitionistic Logic
(\S\ref{sec:intuitionistic_logic}) and Linear Logic
(\S\ref{sec:linear_logic}).

\emph{Refute}


Game Semantics for Linear Logic (\S\ref{sec:linear_logic}):
Interaction Geometry (\S\ref{sec:interaction_geometry}) Semantics of
Linear Logic (\S\ref{sec:linear_logic}) characterizes Linear Logic in
terms of Linear Algebra

1997 - Abramsky - \emph{Game Semantics for Programming Languages}

2002 - Abramsky - \emph{Games in the Semantics of Programming Languages}

2012 - Melli\'es - \emph{Game semantics in string diagrams}

2019 - Ghica - \emph{The far side of the cube: An elementary introduction to
  game semantics}



% ------------------------------------------------------------------------------
\subsection{Independence-friendly Semantics}
\label{sec:independence_semantics}
% ------------------------------------------------------------------------------

Semantics of Independence-friendly Logics
(\S\ref{sec:independence_logic}) is defined by Game Semantics where
players have Imperfect Information.



% ------------------------------------------------------------------------------
\subsection{Denotational Semantics}\label{sec:denotational_semantics}
% ------------------------------------------------------------------------------

Expressions in a Programming Language are interpreted as Mathematical
Objects called \emph{Denotations}.

A Denotational Semantics for $\lambda$-calculus may be given by a
B\"om Tree (\S\ref{sec:bohm_tree}).

A Denotational Semantics of a Programming Language $L$ with Types as
Objects and Procedures as Morphisms, $\mathbf{C}(L)$, can be given as
a Functor into a Scott Domain $\mathbf{D}$: \cite{awodey06}
\[
  S : \mathbf{C}(L) \rightarrow \mathbf{D}
\]

Functional Computation: Categories of ``Sets'' (Domains) to Interpret
Types, certain Functions between Sets to Interpret Programs

Canonical Formal Calculus as $\lambda$-calculus
(\S\ref{sec:untyped_lambda})

Domain Theory (\S\ref{sec:domain_theory})

Metric Spaces (\S\ref{sec:metric_space})

Mapping of an Initial Basis to a new Simplified Program ??? %FIXME

Tokens?

\fist Cf. Operational Semantics (\S\ref{sec:operational_semantics});
Denotational Semantics are less ``sensitive'' to changes in the
Presentation of a Language (Bakker-Vink96) %FIXME

\fist Monads (\S\ref{sec:monad}) introduced by Moggi (FIXME: xref) as a way to
structure Denotational Semantics



\subsubsection{Predicate Transformer Semantics}
\label{sec:predicate_transformer_semantics}

Edsger Dijkstra -- Guarded Command Language (GCL) -- (1975
\emph{Guarded commands, non-determinacy and formal derivation of
  programs}) -- Program Verification using Hoare Logic
(\S\ref{sec:hoare_logic})

$\mono{rustproof}$ -- Rust ``verification condition generator'' using
Predicate Transformer Semantics



\subsubsection{Resumption}\label{sec:resumption}

(David A. Schmidt - \emph{Denotational Semantics: A Methodology for
  Language Development}) %FIXME

Concurrency: Concurrent System as a collection of
Resumption-computations \cite{filinski99}

Continuation (\S\ref{sec:continuation})

Monad of $\bar{e}$-resumptions: \cite{filinski99} %FIXME
\[
  \mathsf{T}\alpha = \mu \beta. 1 \xrightarrow{\bar{e}} (\alpha + \beta)
\]



\subsubsection{Topological Semantics}\label{sec:topological_semantics}

(Bakker-Vink96): Denotational Semantics based on Topological Spaces

Function Space Metric Space %FIXME

Banach Fixed-point Theorem (\S\ref{sec:banach_fixedpoint})

2020 - Bazerman - \emph{A Localic Approach to Dependency, Conflict and
  Concurrency} -
\url{https://johncarlosbaez.wordpress.com/2020/04/28/a-localic-approach-to-dependency-conflict-and-concurrency/}



\subsubsection{Fixpoint Semantics}\label{sec:fixpoint_semantics}

Georg Nees - \emph{Fixpoint Semantics and Semiotics} -
\url{https://gjoncas.github.io/posts/2022-02-12-fixpoint-semantics-and-semiotics.html}

Semiotics (\S\ref{sec:semiotics})



% ==============================================================================
\section{Computational Semantics}\label{sec:computational_semantics}
% ==============================================================================

cf. Computational Semiotics (\S\ref{sec:computational_semiotics})

(FIXME: the following describes a sort of ``computational semantics'' defined by
a set of rewrite rules in a computational model, which is a different sense than
the natural language semantics usually meant by ``computational semantics'')

(Stay-Meredith16):

Milner92 \cite{milner92} -- now ``standard'' presentation of a Computational
Calculus (\S\ref{sec:computation_model}):
\begin{itemize}
  \item Grammar (\S\ref{sec:formal_grammar}) describing the primary Data Type
    over which \emph{Computations} are carried out
  \item Structural Equivalence (\S\ref{sec:structural_equality}) used to erase
    Syntactic differences that are ``irrelevant'' to Computation
  \item Set of Rewrite Rules (\S\ref{sec:abstract_rewrite}) describing how to
    ``realize'' Computation through Operations on Data Structures
\end{itemize}
generalizing the Generators and Relations Presentations
(\S\ref{sec:presentation}) of Universal Algebra (\S\ref{sec:universal_algebra})
with the Grammar replacing the Generators as the ``free construction'' and the
Structural Equivalence replacing the Relations, and the Rewrite Rules providing
the Computational Semantics (\S\ref{sec:computational_semantics})



% ==============================================================================
\section{Situation Semantics}\label{sec:situation_semantics}
% ==============================================================================

Situation Theory (\S\ref{sec:situation_theory})

Non-well-founded Set Theory (\S\ref{sec:non_wellfounded})

Situation Theory : Situation Semantics
  :: Type Theory : Montague Semantics



% ------------------------------------------------------------------------------
\subsection{Situation Theory}\label{sec:situation_theory}
% ------------------------------------------------------------------------------



% ==============================================================================
\section{Probabilistic Semantics}\label{sec:probabilistic_semantics}
% ==============================================================================

\fist \emph{Probabilistic Logic} (\S\ref{sec:probabilistic_logic})

Probabilistic Semantics for Classical Languages: Probabilistic Consequence
Relation yields \emph{Probability Preserving} (\S\ref{sec:probability}) (dually,
\emph{Uncertainty Propagating} \S\ref{sec:uncertainty}) Deductive Validity
(\S\ref{sec:validity}) rather than Truth Preserving
(\S\ref{sec:truth_preservation}); the Premises of a Valid Argument can be
Uncertain as well as the Conclusion



% ==============================================================================
\section{Path Semantics}\label{sec:path_semantics}
% ==============================================================================

Programming Abstractions are ``Compressed Rules'' over a
``Super-exponential Geometric Space'' of Meaning

\url{https://github.com/advancedresearch/path_semantics/blob/master/sequences.md}



% ==============================================================================
\section{Axiomatic Semantics}\label{sec:axiomatic_semantics}
% ==============================================================================

\emph{Axiomatic Semantics} equates Meaning of a Sentence with the
Logical Formulas that describe it; that is what can be proven about it
in some System of Logic.

Axiomatic Semantics are used as a Formal Semantics of Programming
Languages.



% ==============================================================================
\section{Applicative Universal Grammar}
\label{sec:applicative_universal_grammar}
% ==============================================================================

% ==============================================================================
\section{Semiotics}\label{sec:semiotics}
% ==============================================================================

Saussurean Semiology (Sign/Syntax, Signal/Semantics)
\S\ref{sec:dyadic_sign},

[Signifier, Signified (Intension), Referent (Extension)]

Peircean (Sign, Object, Interpretant) \S\ref{sec:triadic_sign}

Morphism $\leftrightarrow$ Representation (\S\ref{sec:representation_theory})

cf. \emph{Interactive Design}

Georg Nees - \emph{Fixpoint Semantics and Semiotics} -
\url{https://gjoncas.github.io/posts/2022-02-12-fixpoint-semantics-and-semiotics.html}

Fixpoint Semantics (\S\ref{sec:fixpoint_semantics})



% ------------------------------------------------------------------------------
\subsection{Syntactics}\label{sec:syntactics}
% ------------------------------------------------------------------------------

Formal Language Theory (Part \ref{part:formal_language})

Formal Grammar (Syntax \S\ref{sec:formal_grammar})



\subsubsection{Signifier}\label{sec:signifier}

\emph{Sign} \emph{Symbol}



\paragraph{Morpheme}\label{sec:morpheme}\hfill

\paragraph{Dyadic Sign}\label{sec:dyadic_sign}\hfill

\paragraph{Triadic Sign}\label{sec:triadic_sign}\hfill



\subsubsection{Tagmeme}\label{sec:tagmeme}

\subsubsection{Clause}\label{sec:grammatical_clause}

smallest Grammatical Unit able to express a complete Proposition
(\S\ref{sec:proposition})

cf. \emph{Clause} \S\ref{sec:clause} (Logic)



\paragraph{Independent Clause}\label{sec:independent_clause}\hfill

\paragraph{Dependent Clause}\label{sec:dependent_clause}\hfill \\\hfill

\emph{Subordinate Clause}

Subjunctive (\S\ref{sec:subjunctive})

\subparagraph{Content Clause}\label{sec:content_clause}\hfill \\\hfill



\subsubsection{Modality}\label{sec:syntax_modality}

Sign-type

material means of conveying Meaning

Pierce: the ``Truth Value'' of a Sign

cf. ``Medium''

%FIXME xref with modal logic?



\subsubsection{Inscription}\label{sec:inscription}

An \emph{Inscription} is a concrete occurence of a Word



\subsubsection{Use-mention}\label{sec:use_mention}

A Word is \emph{Used} when it is used to Refer (\S\ref{sec:reference})
to something.

A Word is \emph{Mentioned} when it is used to Refer to the Signifier
itself, and is differentiated from being Use by italics or single
quotes.



\paragraph{Quasi-quote}\label{sec:quasi_quote}\hfill

Substitutional Quantification (\S\ref{sec:truthvalue_semantics})



% ------------------------------------------------------------------------------
\subsection{Semantics}\label{sec:semantics}
% ------------------------------------------------------------------------------

\subsubsection{Extension}\label{sec:extension}\cite{chalmers02}

The Extension of a Term (\S\ref{sec:term}) is its Referent
(\S\ref{sec:referent}).

The Extension of a Sentence (\S\ref{sec:sentence}) is a Truth-value
(\S\ref{sec:truth_value})

The Extension of a Property (\S\ref{sec:property}) is a Truth Function
(\S\ref{sec:truth_function}) or a Relation (Predicate
\S\ref{sec:predicate}, Set Relation \S\ref{sec:set_relation}).

Expressions sharing the same Extension are called \emph{Co-extensive}.



\subsubsection{Intension}\label{sec:intension}\cite{chalmers02}

An \emph{Intension} may be generally defined as an Algorithm
(\S\ref{sec:algorithm}) or something to be \emph{Evaluated}, the result of which
gives the \emph{Extension} (\S\ref{sec:extension}).
% FIXME

Any Property (\S\ref{sec:property}) or quality Connoted
(\S\ref{sec:connotation}) by a Symbol or Expression.

\emph{Epistemic Intension}: Possibility (\S\ref{sec:possibility}) $\rightarrow$
Extension (\S\ref{sec:extension}), where the Possibility is given as a
Description (\S\ref{sec:description}).

\emph{Subjunctive Intension}: Counterfactual Possibility $\rightarrow$ Extension

\emph{Metaphysical Intension} %FIXME

\emph{Contextual Intension}

In Epistemic Intensions, Names (\S\ref{sec:rigid_designator}) and Descriptions
(\S\ref{sec:description}) have the same Epistemic Intensions and Extensions, but
in Subjunctive Intensions they may have different Subjunctive Intensions.

\emph{Primary Intension}: Sense (\S\ref{sec:sense}), \emph{A
  posteriori}

\emph{Secondary Intension}: Reference (\S\ref{sec:reference}),
\emph{Necessary}

\emph{Type Meaning}

\emph{Token Meaning}



\paragraph{Comprehension}\label{sec:comprehension}\hfill

All Intensions of an Object



\paragraph{Two-dimensional Intension}\label{sec:twodimensional_intension}\hfill

\emph{Two-dimensional Intension}: $(V,W) \rightarrow$ Extensions,
where $V$ is an Epistemic Possibility and $W$ is a Metaphysical
Possibility.



\subsubsection{Meaning}\label{sec:meaning}

The \emph{Meaning} of Symbols and Expressions in a Formal Language is
assigned by an Interpretation (\S\ref{sec:interpretation}).

\emph{Formalism} -- the manipulation of \emph{Representations}
(\S\ref{sec:model_representation}) without regard to \emph{Meaning};
Representations are chosen such as to faithfully ``encapsulate'' significant
features of the Meaning (Dix2003)



\paragraph{Principle of Compositionality}\label{sec:compositionality}
\hfill

\emph{Principle of Compositionality} (or \emph{Frege's Principle}):
Meaning of an Expression determined by Meanings of its constituent
Expressions together with the Syntactic Rules for combining them

Interpretation of a Language is given by a Homomorphism between an
Algebra of Syntactic Representations and an Algebra of Semantic
Objects



\paragraph{Sememe}\label{sec:sememe}\hfill

Unit of Meaning (\S\ref{sec:meaning}), correlative to a Morpheme
(\S\ref{sec:morpheme})

Denotational (\S\ref{sec:denotation}):

\begin{enumerate}
  \item Primary Denotation
  \item Secondary Denotation
\end{enumerate}

Connotational (\S\ref{sec:connotation}):

\begin{enumerate}
  \item High Position
  \item Emotive
  \item Evaluative
\end{enumerate}



\subparagraph{Seme}\label{sec:seme}\hfill

A single characteristic of a Sememe



\subparagraph{Episememe}\label{sec:episememe}\hfill

Tagmeme (\S\ref{sec:tagmeme})



\paragraph{Sense}\label{sec:sense}\hfill
\cite{chalmers02}

Intension (\S\ref{sec:intension})

Relevant Condition on Extension (\S\ref{sec:extension}), cf.
Description (\S\ref{sec:description})

The Terms $a$ and $b$ have different Senses if and only if $a = b$ is
Non-trivial (i.e. not knowable \emph{A priori}). Two expressions have
different Senses if and only if a statement of their Co-extensiveness
(\S\ref{sec:extension}) is Non-trivial.

\subparagraph{Polysemy}\label{sec:polysemy}\hfill



\paragraph{Reference}\label{sec:reference}\hfill

Extension (\S\ref{sec:extension})

Statements with the same Extension are called \emph{Co-extensive}

A Word used to Refer to something is said to be \emph{Used}
(\S\ref{sec:use_mention}).

\emph{Secondary Reference}

\subparagraph{Referent}\label{sec:referent}\hfill

\emph{Co-referential}

\subparagraph{Absent Referent}\label{sec:absent_referent}\hfill

\subparagraph{Self-reference}\label{sec:self_reference}\hfill



\paragraph{Causal Theory}\label{sec:causal_reference}\hfill

\subparagraph{Semantic Externalism}\label{sec:semantic_externalism}\hfill



\subsubsection{Denotation}\label{sec:denotation}

A \emph{Denotation} is an \emph{Extension} (\S\ref{sec:extension}) of
a Term (\S\ref{sec:term})

Denotational Semantics (\S\ref{sec:denotational_semantics}



\paragraph{Rigid Designator}\label{sec:rigid_designator}\hfill

same Denotation in all Possible Worlds (\S\ref{sec:possible_world})

cf. \emph{Named Entity} in Information Extraction



\subparagraph{Name}\label{sec:name}\hfill

A Computation (\S\ref{sec:computation_model}) is given a Name using a
Fixed Point Operator: $fix x is e$
%FIXME xref cite harper blog ``recursion''



\paragraph{Non-rigid Designator}\label{sec:nonrigid_designator}\hfill

\subparagraph{Description}\label{sec:description}\hfill
\cite{chalmers02}

Epistemic Condition on Extension (\S\ref{sec:extension}), cf. Sense
(\S\ref{sec:sense}), Intension (\S\ref{sec:intension})

A Description $D$ is \emph{Canonical} or \emph{Epistemically Complete}
if there is no $S$ such that both $D \wedge S$ and $D \wedge \neg S$
are Epistemically Possible, and that $D$ uses only Semantically
Neutral Expressions with Indexicals (\S\ref{sec:indexical}).

\subparagraph{Definite Description}\label{sec:definite_description}\hfill

\subparagraph{Indefinite Description}\label{sec:indefinite_description}\hfill



\subsubsection{Connotation}\label{sec:connotation}

\emph{Connotation} is a Second-order Denotation giving the
\emph{Intension} (\S\ref{sec:intension}) of an Object. The
\emph{Comprehension} (\S\ref{sec:comprehension}) is the collection of
all Intensions for an Object.



\subsubsection{Definition}\label{sec:definition}

\emph{Extensional Definition}

\emph{Intensional}

\emph{Identity} \emph{Determinable} \emph{Determinate}

\emph{Stipulative Definition}

\emph{Precising Definition}

\emph{Persuasive Definition}



\paragraph{Well-defined}\label{sec:well_defined}\hfill

An Expression is \emph{Well-defined} if its Definition assigns a
unique Interpretation or Value.

A Function (or Decision Problem \S\ref{sec:decision_problem}) is
Well-defined if changing the input representation while leaving the
input Value unchanged gives the same result. %FIXME



\paragraph{Recursive Definition}\label{sec:recursive_definition}\hfill

\emph{Recursive} = \emph{Inductive}

\begin{enumerate}
    \item Base case
    \item Inductive clause
    \item Extremal clause
\end{enumerate}

\emph{Structural Recursion}

Recursively Defined Functions will have the Partiality Effect
(\S\ref{sec:partiality_effect}) in an Effect System
(\S\ref{sec:effect_system}, see Layered Monad
\S\ref{sec:layered_monad})



\paragraph{Circular Definition}\label{sec:circular_definition}\hfill

\emph{Homoiconic} (\S\ref{sec:homoiconicity})



\paragraph{Impredicative Definition}\label{sec:impredicative_definition}\hfill

\emph{Self-referencing}



\subsubsection{Semiosis}\label{sec:semiosis}

\subsubsection{Semasiology}\label{sec:semasiology}

\subsubsection{Onamasiology}\label{sec:onamasiology}

Synchronic

Diachronic

\paragraph{Metonym}\label{sec:metonym}\hfill



% ------------------------------------------------------------------------------
\subsection{Pragmatics}\label{sec:pragmatics}
% ------------------------------------------------------------------------------

(Corfield 2018): Side-effects (\S\ref{sec:effect}) are to Programming Languages
what Pragmatics are to Natural Languages (Marsik, Amblard 2016)



\subsubsection{Interpretant}\label{sec:interpretant}

\subsubsection{Indexical}\label{sec:indexical}

\subsubsection{Deixis}\label{sec:deixis}

\paragraph{Demonstrative}\label{sec:demonstrative}\hfill



% ------------------------------------------------------------------------------
\subsection{Computational Semiotics}\label{sec:computational_semiotics}
% ------------------------------------------------------------------------------

cf. Computational Semantics (\S\ref{sec:computational_semantics})



\subsubsection{Algebraic Semiotics}\label{sec:algebraic_semiotics}
