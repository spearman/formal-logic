%%%%%%%%%%%%%%%%%%%%%%%%%%%%%%%%%%%%%%%%%%%%%%%%%%%%%%%%%%%%%%%%%%%%%%
%%%%%%%%%%%%%%%%%%%%%%%%%%%%%%%%%%%%%%%%%%%%%%%%%%%%%%%%%%%%%%%%%%%%%%
\part{Formal Semantics}\label{sec:formal_semantics}
%%%%%%%%%%%%%%%%%%%%%%%%%%%%%%%%%%%%%%%%%%%%%%%%%%%%%%%%%%%%%%%%%%%%%%
%%%%%%%%%%%%%%%%%%%%%%%%%%%%%%%%%%%%%%%%%%%%%%%%%%%%%%%%%%%%%%%%%%%%%%

\emph{Formal Semantics} is the general study of \emph{Interpretations}
(\S\ref{subsec:interpretation}) of \emph{Formal Languages} (Part
\ref{sec:formal_language}).



% ====================================================================
\section{Soundness}\label{sec:soundness}
% ====================================================================

An Argument in a System of Logic is \emph{Sound} if and only if the
Argument is Valid and all of its Premises are True. The Logical System
itself has the Soundness Property if and only if its Inference Rules
(\S\ref{subsec:inference_rules}) Prove only Valid Formulas under
Semantic Interpretation. This usually amounts to the simple
requirement that the Axioms are Valid and the Inference Rules preserve
Validity.



% ====================================================================
\section{Truth}\label{sec:semantic_truth}
% ====================================================================

\emph{Tarski's Undefinability Theorem} \cite{tarski36} uses the same
techniques as G\"odel's Incompleteness Theorems to show that Truth
cannot be defined in an Object Language
(\S\ref{sec:metalanguage}). The two related conceptions of Truth are
the \emph{Correspondence Theory} (\S\ref{subsec:correspondence_truth})
and \emph{Deflationary Theory} (\S\ref{subsec:deflationary_truth}).

Briefly, the Undefinability Theorem results in a \emph{Material
  Adequacy Condition} (called \emph{Convention T}) that any Theory of
Truth must entail:
\[
    \forall P (\mathrm{True}(S) \leftrightarrow P)
\]
where $S$ is the name of the Sentence $P$ in the Metalanguage which is
an Interpretation of $P$ in the Object Language. This is the
\emph{T-Schema} used in \emph{Tarski's Semantic Theory of Truth} to
Inductively define Truth, expressed as a First-order Sentence. When a
Modal Logic is based on the T-Schema it is said to give rise to
\emph{T-Theory}. Tarski's Semantic Theory of Truth is used as the
definition for Truth in \emph{Model Theory}
(\S\ref{sec:model_theory}).

An example sentence conforming to Convention T in Natural Language
where the Object Language is German and the Metalanguage is English:
\begin{description}
\item ``\emph{Der Schnee ist wei\ss} is True if and only if snow is
  white''.
\end{description}
Here the right side of the Biconditional ('snow is white') is the
\emph{Truth-Condition} of the left side.

Tarski considered this definition of Truth to be a type of
Correspondence Theory.

\emph{Dialetheism}

% --------------------------------------------------------------------
\subsection{Correspondence Theory}\label{subsec:correspondence_truth}
% --------------------------------------------------------------------

The \emph{Correspondence Theory of Truth} defines Truth of a Statement
by its relation and Correspondence with the world.

% --------------------------------------------------------------------
\subsection{Coherence Theory}
% --------------------------------------------------------------------

The \emph{Coherence Theory of Truth} defines Truth of a Statement by
its relation to other Statements.

% --------------------------------------------------------------------
\subsection{Deflationary Theory}\label{subsec:deflationary_truth}
% --------------------------------------------------------------------

A \emph{Deflationary Theory of Truth} is one that states that
ascribing Truth to a Statement does not attribute a property of Truth
to any such Statement in one of a number of different ways below.

% --------------------------------------------------------------------
\subsubsection{Redundancy Theory}
% --------------------------------------------------------------------

The \emph{Redundancy Theory of Truth} states that the Predicate of
Truth is Redundant in that it is equal to the Statement it is applied
to.\cite{ramsey27} Essentially, Truth is a \emph{periphrasis} of the
Sentence it is applied to.

% --------------------------------------------------------------------
\paragraph{Disappearance Theory}
% --------------------------------------------------------------------
\hfill \\ A \emph{Disappearance Theory of Truth} states that Truth is
both Redundant and there is no such Property of Truth. A.J. Ayer is
known for this Theory.

% --------------------------------------------------------------------
\subsubsection{Performative Theory}
% --------------------------------------------------------------------

The \emph{Performative Theory of Truth} is a Deflationary Theory that
sees the Predicate of Truth as a signal of agreement with the
Statement, for such reasons as arriving at Consensus or such others.

% --------------------------------------------------------------------
\subsubsection{Disquotational}
% --------------------------------------------------------------------

The \emph{Disquotational Theory of Truth} is a Deflationary
interpretation of Tarski's definition of Truth by W.V.O. Quine. It
states the Truth predicate has the effect of \emph{Dereferencing}
Sentences (removing the quotation marks). So
\[
    S \leftrightarrow True(``S``)
\]
that is, $S$ is equivalent to \emph{``$S$'' is true}.

The effect of adding \emph{is True} to an Assertion is then to convert
the Use of the Assertion to a Mention.

% --------------------------------------------------------------------
\subsubsection{Prosententialism}
% --------------------------------------------------------------------

\emph{Prosententialism} denies that ``is true'' is a Predicate and is
instead a \emph{Prosentence} (the Sentential analog to
\emph{Pronouns}) that stands in for another Sentence.

% --------------------------------------------------------------------
\subsubsection{Minimal}
% --------------------------------------------------------------------

\emph{Minimalism} defines Truth as a \emph{Metalinguistic} property
and that only Propositoins are Truth-bearing.

% --------------------------------------------------------------------
\subsection{Normative}
% --------------------------------------------------------------------

A \emph{Normative Theory of Truth} states that Truth is the Normative
goal of Assertion.



% ====================================================================
\section{Definition}\label{sec:semantic_definition}
% ====================================================================

\emph{Intension}

\emph{Extension}



% ====================================================================
\section{Structure}\label{sec:mathematical_structure}
% ====================================================================

A \emph{Mathematical Structure} is composed of an arbitrary Set called
a \emph{Carrier Set} or \emph{Underlying Set} (or Domain or Universe)
with one or more \emph{Operators}. Allowing \emph{Infinitary
  Operators} leads to a Theory of \emph{Complete Lattices}.

Formal definition of a Structure:
\[
    \mathcal{A} = (A, \sigma, I)
\]
with Domain $A$, Signature $\sigma$, and \emph{Interpretation
  Function} $I$. The Domain of a Structure $\mathcal{A}$ may also be
written as $|\mathcal{A}|$.

The \emph{Signature} or \emph{Vocabulary} of a Structure is the Set of
Operators (Functions and Relations) that characterize it. The
Signature is a synonym for the \emph{Type} of the Structure
(Schematically represented by $\Omega$), and can be written as an
ordered sequence of Natural Numbers representing the arity of the
Operators. The arity, $n$, of a particular Operator symbol, $s$, may
be written $n=ar(s)$. Sometimes the Signature is given as a triple
\[
    (O,F,P)
\]
where $O$ are Constants, $F$ are Functions, and $P$ are Predicates
(Relations).

A Structure with no Relation Symbols is an \emph{Algebra}
(\S\ref{subsec:universal_algebra}). A Structure with no Functions may
be used as the basis for a \emph{Relational Model}
(\S\ref{sec:relational_model}) in \emph{Database Theory}.

A \emph{Reduct} of a Structure is created by omitting certain
Operations and Relations from the Signature. The converse is
\emph{Expansion}.



% --------------------------------------------------------------------
\subsection{Interpretation}\label{subsec:interpretation}
% --------------------------------------------------------------------

In Model Theory an Interpretation fixes the \emph{Domain of Discourse}
for expressions of Symbolic Logic (Part \ref{sec:symbolic_logic});
that is it assigns the Sets and Relations that Variables can range
over.

Roughly, an Interpretation of a Formal Language is an assignment of
\emph{Meanings} to Symbols and \emph{Truth-Conditions}
(\S\ref{sec:semantic_truth}) to Sentences. An Interpretation of
First-order Logic Maps Terms to Individuals in the Universe and
Propositions to Truth Values.

The Interpretation Function in a Mathematical Structure maps Function
and Relation Symbols of the Signature to actual Functions and
Relations on the Domain:
\[
    f^{\mathcal{A}} = I (f)
\]
\[
    R^{\mathcal{A}} = I (R) \subseteq A^{ar(R)}
\]
A Constant (Nullary) Symbol is identified with an Element of the
Domain:
\[
    I(c) \in A
\]

Thus the Interpretation Function is the \emph{Extension}
(\S\ref{subsec:set_property}) of the Symbols and Strings of Symbols of
the Object Language.

\emph{Intended Interpretation}, \emph{Standard Model}



% --------------------------------------------------------------------
\subsection{Semantic Consequence}\label{subsec:semantic_consequence}
% --------------------------------------------------------------------

\emph{Semantic Consequence} is written as
\[
    T \vDash_{\mathcal{S}} \varphi
\]
where $T$ is a Theory of a Formal System $\mathcal{S}$ and $\varphi$
is a Formula that is the Semantic Consequence of that Theory. This
Relation can only be True if there exists a Structure which
\emph{Satisfies} (\S\ref{subsec:satisfaction}) both $T$ and
$\varphi$. A \emph{Tautology} is expressed as
\[
    \vDash {\varphi}
\]
where a Formula $\varphi$ is the Semantic Consequence of the Empty
Set.



\subsubsection{Valuation}\label{subsec:model_valuation}

A \emph{Valuation} is the assignment of Values to Variables of a
Formula.

\emph{Supervaluation}



% --------------------------------------------------------------------
\subsection{Substructure}\label{subsec:model_substructure}
% --------------------------------------------------------------------

A Structure $\mathcal{A}$ is an \emph{Induced Substructure} of
Structure $\mathcal{B}$ when
\begin{itemize}
\item $\sigma(\mathcal{A}) = \sigma(\mathcal{B})$
\item $A \subseteq B$
\item $I_{\mathcal{A}}=I_{\mathcal{B}}$
\end{itemize}
denoted by the notation $\mathcal{A} \subseteq \mathcal{B}$ where
$\mathcal{B}$ is called the \emph{Extension} or \emph{Superstructure}
of $\mathcal{A}$.

A Substructure $A$ is an \emph{Elementary Substructure} of $B$ if $A$
and $B$ both \emph{Satisfy} (\S\ref{subsec:satisfaction}) the same
Sentences. Here $B$ would be an \emph{Elementary Extension} of $A$.

When a Structure is applied as a \emph{Model}
(\S\ref{sec:model_theory}) of a particular Theory
(\S\ref{subsec:formal_theory}), if no extensions of that Structure
result in Theories that are Consistent, that Theory is termed
\emph{Complete}. A Theory $T$ is called \emph{Model Complete}
(\S\ref{subsec:model_completion}) if every Substructure of a Model of
$T$ is itself a Model of $T$.

Induced Substructures (and \emph{Closed Subsets} described in the next
section) on a Structure form a \emph{Lattice}.



\subsubsection{Closed Subsets}

A Subset of a Domain is a \emph{Closed Subset} if it is closed under
the Operators of the Structure. For any Subset, $B$, of a Domain,
$|\mathcal{A}|$, there is a \emph{smallest Closed Subset} of
$|\mathcal{A}|$ that contains $B$ called the \emph{Hull} of $B$
denoted by $\langle B \rangle$ or $\langle B \rangle_{\mathcal{A}}$,
which is said to be \emph{generated} by $B$. $\langle \rangle$ is the
\emph{Finitary Closure Operator} (\S\ref{subsec:finitary_closure}).



\subsubsection{Embedding}\label{subsec:sigma_embedding}

% FIXME Generalized embedding, instances in specific domains

An \emph{$\sigma$-Embedding} of two $\sigma$-Structures $\mathcal{A}$
and $\mathcal{B}$ is given by an Injective Map $h: A \hookrightarrow
B$ (the ``hooked arrow'' is used to indicate the Map is an Embedding)
where
\begin{itemize}
\item for every $f_n \in \sigma$ and $a_1, \ldots, a_n \in A^n$,
  $h(f_{n}^A(a_1,\ldots,a_n)) = f_{n}^B(h(a_1),\ldots,h(a_n))$
\item for every $R_n \in \sigma$ and $a_1, \ldots, a_n \in A^n$, $A
  \vDash R(a_1, \ldots, a_n) \leftrightarrow B \vDash R(h(a_1),
  \ldots, h(a_n))$
\end{itemize}
Such an Embedding is an \emph{Elementary Embedding} if $h(A)$ is an
Elementary Substructure (\S\ref{subsec:model_substructure}) of $B$.



% ====================================================================
\section{Algebraic Logic}
% ====================================================================

\emph{Algebraic Logic} is the reasoning arising from the manipulation
of Equations with Free Variables. Algebraic Logic deals with
\emph{Algebraic Semantics} of Classes of Algebras which are the
specification of Semantics based on \emph{Abstract Algebra}
(\S\ref{sec:abstract_algebra}). This allows the matching of Logical
Systems with Structures that Model (\S\ref{sec:model_theory}) them.


\emph{Lindenbaum-Tarski}


% --------------------------------------------------------------------
\subsection{Abstract Algebraic Logic}
% --------------------------------------------------------------------

\emph{Abstract Algebraic Hierarchy} (also called the \emph{Leibniz
  Hierarchy})



\subsubsection{Leibniz Operator}\label{subsec:leibniz_operator}



% --------------------------------------------------------------------
\subsection{Term Algebra}\label{subsec:term_algebra}
% --------------------------------------------------------------------

A \emph{Term Algebra} (also termed \emph{Absolutely Free Algebra} or
\emph{Anarchic Algebra}) is an Algebraic Structure freely generated
over a given Signature. In Category Theory a Term Algebra is an
\emph{Initial Algebra} for the Category of all Algebras with a given
Signature.

A \emph{Free Algebra}, $\mathbf{A}$, is defined by a Set of \emph{Free
  Generators}, $S$, and a Type Signature, $\rho$, which Generate an
Underlying Set, $A$. If $\psi : S \rightarrow A$ is a Function,
$\mathbf{A}$ may be represented by the Free Algebra $(A,\psi)$ if for
every Algebra $\mathbf{B}$ of type $\rho$ with Function $\tau : S
\rightarrow B$, there exists a unique Homomorphism $\sigma : A
\rightarrow B$ such that $\sigma\psi = \tau$.

% --------------------------------------------------------------------
\subsubsection{Herbrand Universe}\label{subsec:herbrand_universe}
% --------------------------------------------------------------------
\hfill \\
A \emph{Herbrand Universe} is a Structure in Logic generated by a Set
of Clauses over a Set of Constant and Function Symbols. This results
in the Herbrand Universe being composed of all Ground Terms (Terms
without Variables).

A \emph{Herbrand Base} is the set of all \emph{Ground Atoms} (Atomic
Formulas (\S\ref{subsec:formation_rules}) in which only Ground Terms
appear).

% --------------------------------------------------------------------
\subsubsection{Quotient Algebra}\label{subsec:quotient_algebra}
% --------------------------------------------------------------------

For an Algebra $\mathbf{A}$ with Underlying Set $A$, the
\emph{Quotient Set}, $A / E$ is the Partitioning of $A$ into
Equivalence Classes by a \emph{Congruence Relation}
(\S\ref{subsec:congruence_relation}) $E$. Since the Operators are
Compatible with the Equivalence Classes of the Quotient Set, these
Classes are \emph{Quotient Algebras}.



% --------------------------------------------------------------------
\subsection{Universal Algebra}\label{subsec:universal_algebra}
% --------------------------------------------------------------------

%FIXME: ref Complete Lattices
Universal Algebra is the study of \emph{Algebraic Structure} (as
opposed to specific instances of Algebraic Systems). Universal Algebra
together with \emph{Category Theory} (Part \ref{sec:category_theory})
makes up \emph{Abstract Algebra} (\S\ref{sec:abstract_algebra}). An
Algebraic Structure differs from a general \emph{Mathematical
  Structure} in that its Signature consists of only Function Symbols
and no Relation Symbols.

An Algebra may be limited by Axioms of \emph{Equational Laws} (eg. the
Associative Axiom).



\subsubsection{Variety}\label{subsec:model_variety}

%FIXME ref Fields, Homomorphism, Subalgebra, Direct Product
A \emph{Variety} is a \emph{Class} of Algebras defined only by Axioms
that are Identities satisfied by a given Signature
(\S\ref{sec:formal_system}). This is equivalent to saying a Variety
is the Class of Algebraic Structures with the same Signature that is
closed under \emph{Homomorphic Images}, \emph{Subalgebras}, and
\emph{Direct Products}; a result known as the \emph{HSP Theorem} or
\emph{Birkhoff's Theorem}\cite{birkhoff35}. This rules out Logical
Connectives, Existential Quantification, and all Relations besides
Equality (thus excluding the Class of \emph{Fields}) and Identities
being implicitly Universally Quantified over the Domain.

Algebraic Structures in a Variety are Quotient Algebras
(\S\ref{subsec:quotient_algebra}) generated by the Set of Identities
on the Term Algebra generated from the Signature and Underlying Set.

A \emph{Subvariety} is a Subclass of a Variety with the same Signature
(eg. the Class of \emph{Abelian Groups} is a Subvariety of the Class
of \emph{Groups}). Classes of Finite Algebras (Algebras with a finite
Underlying Set) are sometimes called \emph{Pseudovarieties}.

An example of a Variety with Signature $\Omega = (2)$ is the Class of
all \emph{Semigroups} with an equation defining the Associative Law:
\[
    x(yz) = (xy)z
\]

%FIXME: ref homomorphism
A Homomorphism between two Algebras $A$ and $B$ is a function $h: A
\rightarrow B$ defined for $n$-ary Operations:
\[
\forall f_A \in A, f_B \in B, h(f_A(x_1, ..., x_n)) = f_B(h(x_1), ...,
h(x_n))
\]

A Subalgebra of an Algebra, $A$, is a Subset of $A$ that is closed
under all the operations of $A$.

The Product of a set of Algebraic Structures is the \emph{Cartesian
  Product} of the Sets with the Operations defined coordinatewise.

% --------------------------------------------------------------------
\subsection{Elementary Class}\label{subsec:elementary_class}
% --------------------------------------------------------------------

A Class of Structures, $K$, with Signature $\sigma$ is an
\emph{Elementary Class} if there is a First-order Theory, $T$, with
Signature $\sigma$ such that $K$ contains all Models of $T$.
Expressed with the \emph{Satisfaction Relation}
(\S\ref{subsec:satisfaction}):
\[
    \mathcal{M} \in \mathcal{E}_T \leftrightarrow \mathcal{M} \vDash T
\]
where $\mathcal{E}_T$ is an Elementary Class, $\mathcal{M}$ is a
Model, and $T$ is a Theory.

If $T$ has only a single Sentence, then $K$ is a \emph{Basic
  Elementary Class}. The Reduct (\S\ref{sec:mathematical_structure})
of an Elementary Class is a \emph{Pseudoelementary Class}.

Elementary Classes are termed \emph{Axiomatizable in First-Order
  Logic} (or simply \emph{Axiomatizable} when implicitly First-Order).

The notion of \emph{Strength} of Formal Systems is defined in terms of
Elementary Clases. A Logic $\alpha$ is equal to another Logic $\beta$
when every Elementary Class in $\beta$ is an Elementary Class in
$\alpha$.

% --------------------------------------------------------------------
\subsection{Ultraproducts}\label{subsec:ultraproducts}
% --------------------------------------------------------------------



% ====================================================================
\section{Model Theory}\label{sec:model_theory}
% ====================================================================

\emph{Model Theory}, Formal Language (Part \ref{sec:formal_language}),
and Formal Logic (\S\ref{sec:formal_system}) together compose the
study of \emph{Metalogic}. One definition of Model Theory is the
combination of Formal Logic with \emph{Universal Algebra}
(\S\ref{subsec:universal_algebra}). An alternative view of Model
Theory equates it with \emph{Algebraic Geometry}
(Part \ref{sec:algebraic_geometry}). The broadest definition of Model
Theory includes four divisions: Classical Model Theory, Model Theory
of Groups and Fields, Geometric Model Theory, and Computable Model
Theory (\S\ref{subsec:computable_model_theory}).

The Language of Model Theory is has two Kinds of Mathematical Object:
\emph{Sets} and \emph{Relations}. As such Model Theory forms the
\emph{Domain of Discourse} for \emph{Predicate Logic}. % FIXME xref

\emph{Models} are \emph{Interpretations}
(\S\ref{subsec:interpretation}) of Theories
(\S\ref{subsec:formal_theory}) in a Formal Language. That is, an
Interpretation is a Model if it assigns Truth values to the Sentences
of a Theory. Model Theory uses Tarski's Semantic Theory of Truth
(\S\ref{sec:semantic_truth}) as the definition of Truth. Model Theory
also forms the foundation of \emph{Formal (Truth-conditional)
  Semantics}-- a reduction of the Meaning of Assertions in Natural
Languages to their Truth-conditions.

% --------------------------------------------------------------------
\subsection{Satisfaction}\label{subsec:satisfaction}
% --------------------------------------------------------------------

When defining a Theory as a set of Sentences in a Formal Language, a
\emph{Model} is an \emph{Interpretation} that \emph{Satisfies} the
Sentences of that Theory. For a Formula $\phi$ and a Structure
$\mathcal{M}$, a \emph{Satisfaction Relation} is denoted:
\[
    \mathcal{M} \vDash \phi
\]
For $\mathcal{M}$ to be a Model of a Theory, $T$, it is required that:
\begin{itemize}
\item The Language of $\mathcal{M}$ is the same as the Language of $T$
\item Every Sentence in $T$ is Satisfied by $\mathcal{M}$
\end{itemize}
%FIXME ref completeness theorem
By the Completeness Theorem a Consistent Theory is Satisfiable, that
is, a Theory has a Model if and only if it is Consistent. The
Compactness Theorem (\S\ref{subsec:firstorder_properties}) implies
that a Theory has a Model if and only if every Finite Subset of the
Sentences in that Theory also have Models.



% --------------------------------------------------------------------
\subsection{Compactness Theorem}
% --------------------------------------------------------------------

%FIXME

% --------------------------------------------------------------------
\subsection{Quantifier Elimination}
% --------------------------------------------------------------------

Within a Theory $T$, if every First-order Formula $\varphi(x_1,
\ldots, x_n)$ with Quantifiers is equivalent to a First-order Formula
$\psi(x_1, \ldots, x_n)$ without Quantifiers, $T$ is said to have the
property of \emph{Quantifier Elimination}. A Theory without Quantifier
Elimination may be made to have it by adding Symbols to its Signature.

% --------------------------------------------------------------------
\subsection{Model Completion}\label{subsec:model_completion}
% --------------------------------------------------------------------

A First-order Theory $T$ is called \emph{Model Complete} if every
Embedding (\S\ref{subsec:sigma_embedding}) of Models of $T$ is an
Elementary Embedding.

A Theory $T^*$ is a \emph{Companion} of another Theory $T$ if every
Model of $T$ can be Embedded in a Model of $T^*$ and likewise every
Model of $T^*$ can be Embedded in a Model of $T$. A \emph{Model
  Companion} is a \emph{Companion} of a Theory that is \emph{Model
  Complete}.

%FIXME ref Amalgamation Property
A \emph{Model Completion} is a Model Companion $T^*$ of a Model $T$
that has the \emph{Amalgamation Property}. This means that every Model
of $T$ can be uniquiely Embedded in a Model of $T^*$.

% --------------------------------------------------------------------
\subsection{Categoricity}
% --------------------------------------------------------------------

%FIXME ref Cardinal, Lowenheim-Skolem
A Theory is termed \emph{Categorical} if all its Models are
Isomorphic. With this definition and the L\"owenheim-Skolem Theorem it
follows that any First-order Theory with a Model of infinite
Cardinality can't be Categorical.

For a Cardinal $\kappa$, a Theory $T$ is \emph{$\kappa$-Categorical}
if any two Models of $T$ of Cardinality $\kappa$ are Isomorphic to one
another. By \emph{Morley's Categoricity Theorem}\cite{morley65} if a
First-order Theory in a Countable Language is Categorical in an
Uncountable Cardinal $\kappa$, then it is Categorical in all
Uncountable Cardinalities. There are three possible cases for
$\kappa$-Categoricity:
\begin{description}
\item[Totally Categorical] $\kappa$-Categorical for all Infinite
  Cardinals
\item[Uncountably Categorical] $\kappa$-Categorical if and only if
  $\kappa$ is an Uncountable Cardinal
\item[Countably Categorical] $\kappa$-Categorical if and only if
  $\kappa$ is a Countable Cardinal
\end{description}
The special case of $\kappa = \aleph_0$ is called
\emph{$\omega$-Categorical}.

% --------------------------------------------------------------------
\subsection{Interpretability}
% --------------------------------------------------------------------

Given two Structures, $M$ and $N$, an \emph{Interpretation} of $M$ in
$N$ is a pair $(n,f)$ where
\begin{itemize}
    \item $n \in \mathbb{N}$
    \item $f:f_{dom} \subset N^n \rightarrow M$ such that the
      $f^k$-preimage of every set $X \subseteq M^k$ definable in $M$
      by a First-order Formula is definable in $N$ by a First-order
      Formula
\end{itemize}

Two Structures are \emph{Bi-interpretable} if they can be interpreted
in each other. This can be used to define an Equivalence Relation
between Structures.

% --------------------------------------------------------------------
\subsection{Abstract Model Theory}
% --------------------------------------------------------------------

\subsubsection{Abstract Logic}

An \emph{Abstract Logic} is a Formal System that consists of a Class
of Sentences with a Satisfaction Relation
(\S\ref{subsec:satisfaction}).

%FIXME compactness, lowenheim-skolem
\emph{Lindstr\"om's Theorem} states that First-order Logic is the
Strongest (\S\ref{subsec:elementary_class}) Logic which has both
Countable Compactness and the Downward L\"owenheim-Skolem Property.

\subsubsection{Institutional Model Theory}

\emph{Institutional Model Theory} generalizes First-order Model Theory
to arbitrary Logical Systems formalized as \emph{Institutions}
(\S\ref{subsec:institution_theory}).

% --------------------------------------------------------------------
\subsection{Finite Model Theory}
% --------------------------------------------------------------------

\emph{Finite Model Theory} (FMT) is a restriction of Model Theory to
Interpretations of Finite Structures.

A Finite Structure can always be described by a single First-order
Sentence. An example structure of $n$ Elements:
\[
    \exists x_1 \cdots \exists x_n ( \varphi_1 \wedge \cdots \wedge
    \varphi_m )
\]
This may be extended to a Finite number of Structures:
\[
    \exists x_1 \cdots \exists x_n ( \varphi_1 \wedge \cdots \wedge
    \varphi_m )
    \vee
    \cdots
    \vee
    \exists x_1 \cdots \exists x_p ( \psi_1 \wedge \cdots \wedge
    \psi_q )
\]
Note the difference here with Infinite First-order Model Theory in
which a Model cannot be uniquely determined by a set of First-order
Sentences because of the Compactness Theorem (For every Infinite Model
a Non-isomorphic Model exists).

The ability of a \emph{Property} (\S\ref{subsec:set_property}) $P$ to
be expressed in First-order Logic may be determined by whether two
Structures $A \in P$ and $B \notin P$ satisfy all the same First-order
Sentences:
\[
    A \vDash \alpha \leftrightarrow B \vDash \alpha
\]

\subsubsection{Finite Model Property}

A System of Logic $S$ has the \emph{Finite Model Property} if there is
a Class of Models $\mathrm{M}$ such that any non-Theorem of $S$ is
Falsified by some Finite Model in $\mathrm{M}$. If $fmp(S)$, $A$ is a
$S$-theorem if and only if $A$ is a Theorem of the Theory of Finite
Models of $S$.

If $S$ is Finitely Axiomatizable (\S\ref{subsec:axiomatization}) and
$fmp(S)$ then it is Decidable (\S\ref{sec:computable_function}).



% --------------------------------------------------------------------
\subsection{Computable Model Theory}\label{subsec:computable_model_theory}
% --------------------------------------------------------------------

\cite{harizanov98}

% --------------------------------------------------------------------
\subsection{Geometric Model Theory}
% --------------------------------------------------------------------

% --------------------------------------------------------------------
\subsubsection{Classification Theory}
% --------------------------------------------------------------------

\emph{Classification Theory} is the division of Theories based on
their \emph{Stability} which is the ability of the Models of the
Theory to be \emph{Classified}.

% --------------------------------------------------------------------
\subsubsection{Types}
% --------------------------------------------------------------------

An \emph{$n$-type} of a Model, $\mathcal{M}$, over a (possibly empty)
Subset of Constants, $A \in M$, is a set of Formulas,
$p(x_1,\ldots,x_n) = p(\mathbf{x})$, with at most $n$ Free Variables
in the Language $L(A)$, formed by adding the members of $A$ to the
Language of $\mathcal{M}$:
\[
    L(A) = L \cup \{ c_a : a \in A \}
\]
such that for every Finite Subset $p_0(\mathbf{x}) \subseteq
p(\mathbf{x})$ there exist Elements $b_1,\ldots,b_n \in M$ with
$\mathcal{M} \vDash p_0(b_1,\ldots,b_n)$.

A \emph{Complete Type} is \emph{Maximal}
(\S\ref{subsec:formal_theory})) under Inclusion such that $\forall
\phi(\mathbf{x}) \in L(A,\mathbf{x})$ either $\phi(\mathbf{x}) \in
p(\mathbf{x})$ or $\neg \phi(\mathbf{x}) \in p(\mathbf{x})$. A
non-Complete type is called a \emph{Partial Type}.

An $n$-type is \emph{Realized} in $\mathcal{M}$ if there is an Element
$\mathbf{b} \in M^n$ such that $\mathcal{M} \vDash
p(\mathbf{b})$. This is guaranteed by the Compactness Theorem
(\S\ref{subsec:firstorder_properties}) in either $\mathcal{M}$ or an
Elementary Extension (\S\ref{subsec:model_substructure}) of
$\mathcal{M}$. This is denoted by $tp_{n}^{\mathcal{M}}(\mathbf{b}/A)$
which is read as ``the Complete Type of $\mathbf{b}$ over $A$''.

A Type $p(\mathbf{x})$ is \emph{Isolated} by a Formula
$\varphi(\mathcal{x})$ if $\forall \psi(\mathbf{x}) \in
p(\mathbf{x})$, $\varphi (\mathbf{x}) \rightarrow
\psi(\mathbf{x})$. Isolated Types are Realized in every Elementary
Substructure or Extension.

\paragraph{Saturation}\label{subsec:model_saturation}\hfill
\\

A Model $\mathcal{M}$ is \emph{$\kappa$-saturated} (where $\kappa$ is
a Cardinal number) if for all $A \subseteq M$ of Cardinality $<
\kappa$, $M$ Realizes all Complete Types over $A$. A Model is
\emph{Saturated} if it is $|M|$-saturated where $|M|$ is the
Cardinality of $M$.

\subsubsection{Stability}\label{subsec:model_stability}

A Theory $T$ is \emph{$\kappa$-stable} for an Infinite Cardinal $\kappa$
if for every set $A$ such that $|A| = \kappa$, the Set of Complete
Types over $A$ has Cardinality $\kappa$. Theories are Classified with
the following terms:
\begin{description}
\item [Stable] $\kappa$-stable for some Infinite Cardinal $\kappa$
\item [Unstable] not $\kappa$-stable for all Infinite Cardinals $\kappa$
\item [Superstable] $\kappa$-stable for all sufficiently large
  Cardinals $\kappa$
\item [Totally Transcendental] \emph{Morley Rank}\cite{morley65} less
  than $\infty$
\end{description}



% ====================================================================
\section{Frame Semantics}\label{sec:frame_semantics}
% ====================================================================

\emph{Frame Semantics} is the extension of Model Theory to
Non-classical Logic Systems, beginning with Modal Logic
(\S\ref{subsec:modal_logic}).

% --------------------------------------------------------------------
\subsection{Kripke Semantics}\label{subsec:kripke_semantics}
% --------------------------------------------------------------------

\begin{description}
\item [Kripke Frame] $\langle W,R \rangle$ where $W$ is a Non-empty
  Set of \emph{Nodes} (\emph{Worlds}) and $R$ is a Binary Relation
  called the \emph{Accessibility Relation}
\item [Kripke Model] $\langle W,R,\Vdash \rangle$ where $\Vdash$ is a
  \emph{Forcing Relation} for Nodes of $W$
\end{description}
Accessbility Relation % FIXME describe accessibility

The Forcing Relation $\Vdash$ (read as Satisfies or \emph{Forces}
(\S\ref{subsec:forcing})) has the following properties:
\begin{itemize}
\item $w \Vdash \neg A$ if and only if $w \nVdash A$
\item $w \Vdash A \rightarrow B$ if and only if $w \nVdash A$ or $w
  \Vdash B$
\item $w \Vdash \square A$ if and only if $u \Vdash A$ for all $u$
  such that $w R u$
\end{itemize}
\emph{Validity} of a Proposition is defined for
\begin{itemize}
\item Model $\langle W,R, \Vdash \rangle$ if $\forall w \in W,
  w \Vdash A$
\item Frame $\langle W,R \rangle$ if Valid in Model $\langle W,R,
  \Vdash \rangle$ for all choices of $\Vdash$
\item Class of Frames or Models, $C$, if Valid for all Frames or
  Models of the Class
\end{itemize}
$Thm(C)$ is defined as the Set of all Formulas Valid
(\S\ref{sec:logic_terminology}) in $C$. For a Set of Formulas $X$,
$Mod(X)$ is defined as the Class of all Frames which Validate every
Formula in $X$.

Kripke Models are a special case of Labelled State Transition Systems
(\S\ref{subsec:state_transition_system}).

A Modal Theory, $T$, is \emph{Sound} (\S\ref{sec:soundness}) with
respect to a Class of Frames, $C$, if $T \subseteq Thm(C)$. $T$ is
\emph{Complete} with respect to $C$ if $T \supseteq Thm(C)$.

For any $C$, $Thm(C)$ is a \emph{Normal Modal Logic}
(\S\ref{subsec:alethic_logic}). A Normal Modal Logic is said to
\emph{Correspond} to $C$ if $C = Mod(L)$.

The \emph{Canonical Model} of a Modal Theory $T$ is a Kripke Model
$\langle W,R, \Vdash \rangle$ where $W$ is the Set of all Maximally
Consistent Sets for $T$ (\S\ref{subsec:formal_theory}) and:
\begin{itemize}
\item $XRY$ if and only if for all Formulae $A$, if $\square A
  \in X$ then $A \in Y$
\item $X\Vdash A$ if and only if $A \in X$
\end{itemize}

%FIXME

\emph{Unravelling}

\emph{Filtration}

% --------------------------------------------------------------------
\subsection{General Frame Semantics}
% --------------------------------------------------------------------

A \emph{Modal General Frame} is defined as a triple $\mathbf{F} =
\langle W,R,V \rangle$ where $\langle W,R \rangle$ is a Kripke Frame
and $V$ is a Set of Subsets of $W$ closed under Intersection, Union,
Complement, and the Operation $\square$ defined as
\[
    \square A = \{x \in W; \forall y \in W ( x R y \rightarrow y \in A ) \}
\]

% --------------------------------------------------------------------
\subsection{Intuitionistic Semantics}
% --------------------------------------------------------------------

The Semantics of Intuitionistic Logic
(\S\ref{subsec:intuitionistic_logic}) is definable as Kripke Semantics
(\S\ref{subsec:kripke_semantics}) or \emph{Heyting Algebra Semantics}.

\subsubsection{Heyting Algebra Semantics}\label{subsec:heyting_semantics}

Instead of assigning Valuations from a Boolean Algebra, Intuitionistic
Semantics uses Values from a \emph{Heyting Algebra}
(\S\ref{subsec:heyting_algebra}). A Formula is Valid if and only if it
receives the Value of the Top Element for any Valuation.

% --------------------------------------------------------------------
\subsection{Kripke-Joyal Semantics}
% --------------------------------------------------------------------

% --------------------------------------------------------------------
\subsection{Boolean-valued Model}
% --------------------------------------------------------------------

\emph{Boolean-valued Models} are related to Heyting Algebras and
Intuitionistic Logic and is equivalent to the method of Forcing.

%FIXME ref complete boolean algebra
Instead of limiting Formulas in a System $S$ to True or False, they
may be assigned values from a fixed \emph{Complete Boolean Algebra}.



% ====================================================================
\section{Proof-theoretic Semantics}
% ====================================================================

\emph{Proof-theoretic Semantics} is based on the Propositions and
Logical Connectives of Systems of Inference (Part
\ref{sec:formal_system}). See also \emph{Semantic Tableau}
(\S\ref{subsec:tableau_calculus}).

% --------------------------------------------------------------------
\subsection{Logical Harmony} \label{subsec:logical_harmony}
% --------------------------------------------------------------------

\emph{Logical Harmony} refers to constraints required between
Introduction and Elimination Rules.



% ====================================================================
\section{Probabilistic Semantics}
% ====================================================================

% --------------------------------------------------------------------
\subsection{Truth-value Semantics}
% --------------------------------------------------------------------

\emph{Truth-value Semantics} is also called the \emph{Substitution
  Interpretation} for Quantifiers or \emph{Substitutional
  Quantification}.



% ====================================================================
\section{Composition Semantics}
% ====================================================================

% --------------------------------------------------------------------
\subsection{Linear Semantics}
% --------------------------------------------------------------------

% --------------------------------------------------------------------
\subsubsection{Game Semantics}
% --------------------------------------------------------------------

\emph{Game Semantics} studies the \emph{Dialogical} properties of
Semantics. It is used in connection with Intuitionistic Logic
(\S\ref{subsec:intuitionistic_logic}) and Linear Logic
(\S\ref{subsec:linear_logic}).

% --------------------------------------------------------------------
\subsection{Independence-friendly Semantics}
% --------------------------------------------------------------------

Semantics of Independence-friendly Logics
(\S\ref{subsec:independence_logic}) is defined by Game Semantics where
players have Imperfect Information.

% --------------------------------------------------------------------
\subsection{Denotational Semantics}
% --------------------------------------------------------------------

Expressions in a Programming Language are interpreted as Mathematics
Objects called \emph{Denotations}.



% ====================================================================
\section{Operational Semantics}
% ====================================================================

\emph{Operational Semantics} are used in the interpretation of
Programming Languages.

\emph{Concurrency}

% --------------------------------------------------------------------
\subsection{Structural Semantics}
% --------------------------------------------------------------------

% --------------------------------------------------------------------
\subsubsection{Reduction Semantics}
% --------------------------------------------------------------------

% --------------------------------------------------------------------
\subsection{Natural Semantics}
% --------------------------------------------------------------------



% ====================================================================
\section{Axiomatic Semantics}
% ====================================================================

\emph{Axiomatic Semantics} equates Meaning of a Sentence with the
Logical Formulas that describe it; that is what can be proven about it
in some System of Logic.

Axiomatic Semantics are used as a Formal Semantics of Programming
Languages.



% ====================================================================
\section{Untyped $\lambda$-Calculus}\label{sec:untyped_lambda}
% ====================================================================

\emph{Untyped $\lambda$-Calculus} describes a Semantics for
\emph{Computable Functions} (\S\ref{sec:computable_function}).

\emph{Lambda Terms}

\emph{Free Variables}

\emph{Anonymous Functions}

\emph{Lambda Abstraction}

\emph{Application}

% --------------------------------------------------------------------
\subsection{Computation Rule}\label{subsec:computation_rule}
% --------------------------------------------------------------------

\[
    (\lambda x.t)(u) :\equiv t[u/x]
\]

% --------------------------------------------------------------------
\subsection{Alpha Equivalence}\label{subsec:alpha_equivalent}
% --------------------------------------------------------------------

% --------------------------------------------------------------------
\subsection{Capture-avoiding Substitution}\label{subsec:capture_avoiding}
% --------------------------------------------------------------------

\emph{Capture-avoiding Substitution}

% --------------------------------------------------------------------
\subsection{Beta Reduction}\label{subsec:beta_reduction}
% --------------------------------------------------------------------

% --------------------------------------------------------------------
\subsection{Eta Conversion}\label{subsec:eta_conversion}
% --------------------------------------------------------------------

% --------------------------------------------------------------------
\subsection{Normalization \& Confluence}\label{subsec:normalization_confluence}
% --------------------------------------------------------------------



% ====================================================================
\section{Inner Model Theory}\label{sec:inner_model_theory}
% ====================================================================
