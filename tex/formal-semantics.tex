%%%%%%%%%%%%%%%%%%%%%%%%%%%%%%%%%%%%%%%%%%%%%%%%%%%%%%%%%%%%%%%%%%%%%%
%%%%%%%%%%%%%%%%%%%%%%%%%%%%%%%%%%%%%%%%%%%%%%%%%%%%%%%%%%%%%%%%%%%%%%
\part{Formal Semantics}\label{sec:formal_semantics}
%%%%%%%%%%%%%%%%%%%%%%%%%%%%%%%%%%%%%%%%%%%%%%%%%%%%%%%%%%%%%%%%%%%%%%
%%%%%%%%%%%%%%%%%%%%%%%%%%%%%%%%%%%%%%%%%%%%%%%%%%%%%%%%%%%%%%%%%%%%%%

\emph{Formal Semantics} is the general study of \emph{Interpretations}
(\S\ref{sec:interpretation}) of \emph{Formal Languages} (Part
\ref{sec:formal_language}).



% ====================================================================
\section{Soundness}\label{sec:soundness}
% ====================================================================

An Argument in a System of Logic is \emph{Sound} if and only if the
Argument is Valid and all of its Premises are True. The Logical System
itself has the Soundness Property if and only if its Inference Rules
(\S\ref{sec:inference_rule}) Prove only Valid Formulas under
Semantic Interpretation. This usually amounts to the simple
requirement that the Axioms are Valid and the Inference Rules preserve
Validity.



% ====================================================================
\section{Truth}\label{sec:semantic_truth}
% ====================================================================

\emph{Tarski's Undefinability Theorem} \cite{tarski36} uses the same
techniques as G\"odel's Incompleteness Theorems to show that Truth
cannot be defined in an Object Language
(\S\ref{sec:metalanguage}). The two related conceptions of Truth are
the \emph{Correspondence Theory} (\S\ref{sec:correspondence_truth})
and \emph{Deflationary Theory} (\S\ref{sec:deflationary_truth}).

Briefly, the Undefinability Theorem results in a \emph{Material
  Adequacy Condition} (called \emph{Convention T}) that any Theory of
Truth must entail:
\[
    \forall P (\mathrm{True}(S) \leftrightarrow P)
\]
where $S$ is the name of the Sentence $P$ in the Metalanguage which is
an Interpretation of $P$ in the Object Language. This is the
\emph{T-Schema} used in \emph{Tarski's Semantic Theory of Truth} to
Inductively define Truth, expressed as a First-order Sentence. When a
Modal Logic is based on the T-Schema it is said to give rise to
\emph{T-Theory}. Tarski's Semantic Theory of Truth is used as the
definition for Truth in \emph{Model Theory}
(\S\ref{sec:model_theory}).

An example sentence conforming to Convention T in Natural Language
where the Object Language is German and the Metalanguage is English:
\begin{description}
\item ``\emph{Der Schnee ist wei\ss} is True if and only if snow is
  white''.
\end{description}
Here the right side of the Biconditional ('snow is white') is the
\emph{Truth-Condition} of the left side.

Tarski considered this definition of Truth to be a type of
Correspondence Theory.

\emph{Dialetheism}

% --------------------------------------------------------------------
\subsection{Correspondence Theory}\label{sec:correspondence_truth}
% --------------------------------------------------------------------

The \emph{Correspondence Theory of Truth} defines Truth of a Statement
by its relation and Correspondence with the world.

% --------------------------------------------------------------------
\subsection{Coherence Theory}
% --------------------------------------------------------------------

The \emph{Coherence Theory of Truth} defines Truth of a Statement by
its relation to other Statements.

% --------------------------------------------------------------------
\subsection{Deflationary Theory}\label{sec:deflationary_truth}
% --------------------------------------------------------------------

A \emph{Deflationary Theory of Truth} is one that states that
ascribing Truth to a Statement does not attribute a property of Truth
to any such Statement in one of a number of different ways below.

% --------------------------------------------------------------------
\subsubsection{Redundancy Theory}
% --------------------------------------------------------------------

The \emph{Redundancy Theory of Truth} states that the Predicate of
Truth is Redundant in that it is equal to the Statement it is applied
to.\cite{ramsey27} Essentially, Truth is a \emph{periphrasis} of the
Sentence it is applied to.

% --------------------------------------------------------------------
\paragraph{Disappearance Theory}
% --------------------------------------------------------------------
\hfill \\ A \emph{Disappearance Theory of Truth} states that Truth is
both Redundant and there is no such Property of Truth. A.J. Ayer is
known for this Theory.

% --------------------------------------------------------------------
\subsubsection{Performative Theory}
% --------------------------------------------------------------------

The \emph{Performative Theory of Truth} is a Deflationary Theory that
sees the Predicate of Truth as a signal of agreement with the
Statement, for such reasons as arriving at Consensus or such others.

% --------------------------------------------------------------------
\subsubsection{Disquotational}
% --------------------------------------------------------------------

The \emph{Disquotational Theory of Truth} is a Deflationary
interpretation of Tarski's definition of Truth by W.V.O. Quine. It
states the Truth predicate has the effect of \emph{Dereferencing}
Sentences (removing the quotation marks). So
\[
    S \leftrightarrow True(``S``)
\]
that is, $S$ is equivalent to \emph{``$S$'' is true}.

The effect of adding \emph{is True} to an Assertion is then to convert
the Use of the Assertion to a Mention.

% --------------------------------------------------------------------
\subsubsection{Prosententialism}
% --------------------------------------------------------------------

\emph{Prosententialism} denies that ``is true'' is a Predicate and is
instead a \emph{Prosentence} (the Sentential analog to
\emph{Pronouns}) that stands in for another Sentence.

% --------------------------------------------------------------------
\subsubsection{Minimal}
% --------------------------------------------------------------------

\emph{Minimalism} defines Truth as a \emph{Metalinguistic} property
and that only Propositoins are Truth-bearing.

% --------------------------------------------------------------------
\subsection{Normative}
% --------------------------------------------------------------------

A \emph{Normative Theory of Truth} states that Truth is the Normative
goal of Assertion.



% ====================================================================
\section{Definition}\label{sec:semantic_definition}
% ====================================================================

\emph{Intension}

\emph{Extension}



% --------------------------------------------------------------------
\subsection{Semantic Consequence}\label{sec:semantic_consequence}
% --------------------------------------------------------------------

\emph{Semantic Consequence} is written as
\[
    T \vDash_{\mathcal{S}} \varphi
\]
where $T$ is a Theory of a Formal System $\mathcal{S}$ and $\varphi$
is a Formula that is the Semantic Consequence of that Theory. This
Relation can only be True if there exists a Structure which
\emph{Satisfies} (\S\ref{sec:satisfaction}) both $T$ and
$\varphi$. A \emph{Tautology} is expressed as
\[
    \vDash {\varphi}
\]
where a Formula $\varphi$ is the Semantic Consequence of the Empty
Set.



\subsubsection{Valuation}\label{sec:model_valuation}

A \emph{Valuation} is the assignment of Values to Variables of a
Formula.

\emph{Supervaluation}



% --------------------------------------------------------------------
\subsection{Substructure}\label{sec:model_substructure}
% --------------------------------------------------------------------

A Structure $\mathcal{A}$ is an \emph{Induced Substructure} of
Structure $\mathcal{B}$ when
\begin{itemize}
\item $\sigma(\mathcal{A}) = \sigma(\mathcal{B})$
\item $A \subseteq B$
\item $I_{\mathcal{A}}=I_{\mathcal{B}}$
\end{itemize}
denoted by the notation $\mathcal{A} \subseteq \mathcal{B}$ where
$\mathcal{B}$ is called the \emph{Extension} or \emph{Superstructure}
of $\mathcal{A}$.

A Substructure $A$ is an \emph{Elementary Substructure} of $B$ if $A$
and $B$ both \emph{Satisfy} (\S\ref{sec:satisfaction}) the same
Sentences. Here $B$ would be an \emph{Elementary Extension} of $A$.

When a Structure is applied as a \emph{Model}
(\S\ref{sec:model_theory}) of a particular Theory
(\S\ref{sec:formal_theory}), if no extensions of that Structure
result in Theories that are Consistent, that Theory is termed
\emph{Complete}. A Theory $T$ is called \emph{Model Complete}
(\S\ref{sec:model_completion}) if every Substructure of a Model of
$T$ is itself a Model of $T$.

Induced Substructures (and \emph{Closed Subsets} described in the next
section) on a Structure form a \emph{Lattice}.



\subsubsection{Closed Subsets}

A Subset of a Domain is a \emph{Closed Subset} if it is closed under
the Operators of the Structure. For any Subset, $B$, of a Domain,
$|\mathcal{A}|$, there is a \emph{smallest Closed Subset} of
$|\mathcal{A}|$ that contains $B$ called the \emph{Hull} of $B$
denoted by $\langle B \rangle$ or $\langle B \rangle_{\mathcal{A}}$,
which is said to be \emph{generated} by $B$. $\langle \rangle$ is the
\emph{Finitary Closure Operator} (\S\ref{sec:finitary_closure}).



\subsubsection{Embedding}\label{sec:sigma_embedding}

% FIXME Generalized embedding, instances in specific domains

An \emph{$\sigma$-Embedding} of two $\sigma$-Structures $\mathcal{A}$
and $\mathcal{B}$ is given by an Injective Map $h: A \hookrightarrow
B$ (the ``hooked arrow'' is used to indicate the Map is an Embedding)
where
\begin{itemize}
\item for every $f_n \in \sigma$ and $a_1, \ldots, a_n \in A^n$,
  $h(f_{n}^A(a_1,\ldots,a_n)) = f_{n}^B(h(a_1),\ldots,h(a_n))$
\item for every $R_n \in \sigma$ and $a_1, \ldots, a_n \in A^n$, $A
  \vDash R(a_1, \ldots, a_n) \leftrightarrow B \vDash R(h(a_1),
  \ldots, h(a_n))$
\end{itemize}
Such an Embedding is an \emph{Elementary Embedding} if $h(A)$ is an
Elementary Substructure (\S\ref{sec:model_substructure}) of $B$.



% ====================================================================
\section{Algebraic Logic}
% ====================================================================

\emph{Algebraic Logic} is the reasoning arising from the manipulation
of Equations with Free Variables. Algebraic Logic deals with
\emph{Algebraic Semantics} of Classes of Algebras which are the
specification of Semantics based on \emph{Abstract Algebra}
(\S\ref{sec:abstract_algebra}). This allows the matching of Logical
Systems with Structures that Model (\S\ref{sec:model_theory}) them.


\emph{Lindenbaum-Tarski}


% --------------------------------------------------------------------
\subsection{Abstract Algebraic Logic}
% --------------------------------------------------------------------

\emph{Abstract Algebraic Hierarchy} (also called the \emph{Leibniz
  Hierarchy})



\subsubsection{Leibniz Operator}\label{sec:leibniz_operator}



% --------------------------------------------------------------------
\subsection{Term Algebra}\label{sec:term_algebra}
% --------------------------------------------------------------------

A \emph{Term Algebra} (also termed \emph{Absolutely Free Algebra} or
\emph{Anarchic Algebra}) is an Algebraic Structure Freely Generated
(\S\ref{sec:free_object}) over a given Signature
(\S\ref{sec:signature}).

Given a Signature $\sigma$ and a Set of Variables $X$, the Term
Algebra of $\sigma$ with Basis (\S\ref{sec:free_module}) $X$ is the
Structure (\S\ref{sec:structure}) $A$ with Domain consisting of all
the Terms (\S\ref{sec:term}) of $\sigma$ with Variables taken from
$X$, such that:
\[
  f^A(\bar{t}) = f(\bar{t})
\] and: \[
  R^A = \varnothing
\]
for Function Symbols $f$ and Relation Symbols $R$ in $\sigma$, and
$\bar{t}$ $n$-tuples of Elements in $|A|$ (i.e. Terms). The Signature
of $A$, $\sigma'$, is the Subset of $\sigma$ without any Relation
Symbols.

In Category Theory a Term Algebra is an Initial Algebra
(\S\ref{sec:initial_algebra}) for the Category of all Algebras with a
given Signature.



\subsubsection{Herbrand Universe}\label{sec:herbrand_universe}

A \emph{Herbrand Universe} is the Freely Generated Structure over the
Function Symbols of a Signature, resulting in all Ground Terms
(\S\ref{sec:term}) over that Signature.

A \emph{Herbrand Base} is the set of all Ground Atoms
(\S\ref{sec:atomic_formula}) that can be formed from the Predicate
Symbols of the Signature.

\emph{Herbrand Structure}



% --------------------------------------------------------------------
\subsection{Quotient Algebra}\label{sec:quotient_algebra}
% --------------------------------------------------------------------

\emph{Quotient Algebra} (or \emph{Factor Algebra})

For an Algebra $\mathbf{A}$ with Underlying Set $A$, the
\emph{Quotient Set}, $A / E$ is the Partitioning of $A$ into
Equivalence Classes by a \emph{Congruence Relation}
(\S\ref{sec:congruence_relation}) $E$. Since the Operators are
Compatible with the Equivalence Classes of the Quotient Set, these
Classes are \emph{Quotient Algebras}.



% --------------------------------------------------------------------
\subsection{Universal Algebra}\label{sec:universal_algebra}
% --------------------------------------------------------------------

%FIXME: ref Complete Lattices
Universal Algebra is the study of \emph{Algebraic Structure} (as
opposed to specific instances of Algebraic Systems). Universal Algebra
together with \emph{Category Theory} (Part \ref{sec:category_theory})
makes up \emph{Abstract Algebra} (\S\ref{sec:abstract_algebra}). An
Algebraic Structure differs from a general \emph{Mathematical
  Structure} in that its Signature consists of only Function Symbols
and no Relation Symbols.

An Algebra may be limited by Axioms of \emph{Equational Laws} (eg. the
Associative Axiom).



\subsubsection{Congruence}\label{sec:general_congruence}



\subsubsection{Kernel}\label{sec:general_kernel}

For Algebraic Structures $A$ and $B$, and Homomorphism $f: A
\rightarrow B$, the \emph{Kernel} of $f$ is defined as:
\[
    ker(f) = \{ (a,a') \in A \times A : f(a) = f(a') \}
\]
$ker(f)$ is a Congruence Relation on $A$ and $f$ is Injective if and
only if $ker(f) = \{(a,a) : a \in A\}$.

The Quotient Algebra $A/ker(f)$ is Isomorphic to the Image of $f$
(which is a Subalgebra of $B$, see First Isomorphism Theorem
(\S\ref{sec:isomorphism_theorem}).



\subsubsection{Free Object}\label{sec:free_object}



\subsubsection{Variety}\label{sec:model_variety}

%FIXME ref Fields, Homomorphism, Subalgebra, Direct Product
A \emph{Variety} is a \emph{Class} of Algebras defined only by Axioms
that are Identities satisfied by a given Signature
(\S\ref{sec:formal_system}). This is equivalent to saying a Variety
is the Class of Algebraic Structures with the same Signature that is
closed under \emph{Homomorphic Images}, \emph{Subalgebras}, and
\emph{Direct Products}; a result known as the \emph{HSP Theorem} or
\emph{Birkhoff's Theorem}\cite{birkhoff35}. This rules out Logical
Connectives, Existential Quantification, and all Relations besides
Equality (thus excluding the Class of \emph{Fields}) and Identities
being implicitly Universally Quantified over the Domain.

Algebraic Structures in a Variety are Quotient Algebras
(\S\ref{sec:quotient_algebra}) generated by the Set of Identities
on the Term Algebra generated from the Signature and Underlying Set.

A \emph{Subvariety} is a Subclass of a Variety with the same Signature
(eg. the Class of \emph{Abelian Groups} is a Subvariety of the Class
of \emph{Groups}). Classes of Finite Algebras (Algebras with a finite
Underlying Set) are sometimes called \emph{Pseudovarieties}.

An example of a Variety with Signature $\Omega = (2)$ is the Class of
all \emph{Semigroups} with an equation defining the Associative Law:
\[
    x(yz) = (xy)z
\]

%FIXME: ref homomorphism
A Homomorphism between two Algebras $A$ and $B$ is a function $h: A
\rightarrow B$ defined for $n$-ary Operations:
\[
\forall f_A \in A, f_B \in B, h(f_A(x_1, ..., x_n)) = f_B(h(x_1), ...,
h(x_n))
\]

A Subalgebra of an Algebra, $A$, is a Subset of $A$ that is closed
under all the operations of $A$.

The Product of a set of Algebraic Structures is the \emph{Cartesian
  Product} of the Sets with the Operations defined coordinatewise.

% --------------------------------------------------------------------
\subsection{Elementary Class}\label{sec:elementary_class}
% --------------------------------------------------------------------

A Class of Structures, $K$, with Signature $\sigma$ is an
\emph{Elementary Class} if there is a First-order Theory, $T$, with
Signature $\sigma$ such that $K$ contains all Models of $T$.
Expressed with the \emph{Satisfaction Relation}
(\S\ref{sec:satisfaction}):
\[
    \mathcal{M} \in \mathcal{E}_T \leftrightarrow \mathcal{M} \vDash T
\]
where $\mathcal{E}_T$ is an Elementary Class, $\mathcal{M}$ is a
Model, and $T$ is a Theory.

If $T$ has only a single Sentence, then $K$ is a \emph{Basic
  Elementary Class}. The Reduct (\S\ref{sec:structure}) of an
Elementary Class is a \emph{Pseudoelementary Class}.

Elementary Classes are termed \emph{Axiomatizable in First-Order
  Logic} (or simply \emph{Axiomatizable} when implicitly First-Order).

The notion of \emph{Strength} of Formal Systems is defined in terms of
Elementary Clases. A Logic $\alpha$ is equal to another Logic $\beta$
when every Elementary Class in $\beta$ is an Elementary Class in
$\alpha$.

% --------------------------------------------------------------------
\subsection{Ultraproducts}\label{sec:ultraproducts}
% --------------------------------------------------------------------



% ====================================================================
\section{Frame Semantics}\label{sec:frame_semantics}
% ====================================================================

\emph{Frame Semantics} is the extension of Model Theory to
Non-classical Logic Systems, beginning with Modal Logic
(\S\ref{sec:modal_logic}).

% --------------------------------------------------------------------
\subsection{Kripke Semantics}\label{sec:kripke_semantics}
% --------------------------------------------------------------------

\begin{description}
\item [Kripke Frame] $\langle W,R \rangle$ where $W$ is a Non-empty
  Set of \emph{Nodes} (\emph{Worlds}) and $R$ is a Binary Relation
  called the \emph{Accessibility Relation}
\item [Kripke Model] $\langle W,R,\Vdash \rangle$ where $\Vdash$ is a
  \emph{Forcing Relation} for Nodes of $W$
\end{description}
Accessbility Relation % FIXME describe accessibility

The Forcing Relation $\Vdash$ (read as Satisfies or \emph{Forces}
(\S\ref{sec:forcing})) has the following properties:
\begin{itemize}
\item $w \Vdash \neg A$ if and only if $w \nVdash A$
\item $w \Vdash A \rightarrow B$ if and only if $w \nVdash A$ or $w
  \Vdash B$
\item $w \Vdash \square A$ if and only if $u \Vdash A$ for all $u$
  such that $w R u$
\end{itemize}
\emph{Validity} of a Proposition is defined for
\begin{itemize}
\item Model $\langle W,R, \Vdash \rangle$ if $\forall w \in W,
  w \Vdash A$
\item Frame $\langle W,R \rangle$ if Valid in Model $\langle W,R,
  \Vdash \rangle$ for all choices of $\Vdash$
\item Class of Frames or Models, $C$, if Valid for all Frames or
  Models of the Class
\end{itemize}
$Thm(C)$ is defined as the Set of all Formulas Valid
(\S\ref{sec:validity}) in $C$. For a Set of Formulas $X$,
$Mod(X)$ is defined as the Class of all Frames which Validate every
Formula in $X$.

Kripke Models are a special case of Labelled State Transition Systems
(\S\ref{sec:state_transition_system}).

A Modal Theory, $T$, is \emph{Sound} (\S\ref{sec:soundness}) with
respect to a Class of Frames, $C$, if $T \subseteq Thm(C)$. $T$ is
\emph{Complete} with respect to $C$ if $T \supseteq Thm(C)$.

For any $C$, $Thm(C)$ is a \emph{Normal Modal Logic}
(\S\ref{sec:alethic_logic}). A Normal Modal Logic is said to
\emph{Correspond} to $C$ if $C = Mod(L)$.

The \emph{Canonical Model} of a Modal Theory $T$ is a Kripke Model
$\langle W,R, \Vdash \rangle$ where $W$ is the Set of all Maximally
Consistent Sets for $T$ (\S\ref{sec:formal_theory}) and:
\begin{itemize}
\item $XRY$ if and only if for all Formulae $A$, if $\square A
  \in X$ then $A \in Y$
\item $X\Vdash A$ if and only if $A \in X$
\end{itemize}

%FIXME

\emph{Unravelling}

\emph{Filtration}

% --------------------------------------------------------------------
\subsection{General Frame Semantics}
% --------------------------------------------------------------------

A \emph{Modal General Frame} is defined as a triple $\mathbf{F} =
\langle W,R,V \rangle$ where $\langle W,R \rangle$ is a Kripke Frame
and $V$ is a Set of Subsets of $W$ closed under Intersection, Union,
Complement, and the Operation $\square$ defined as
\[
    \square A = \{x \in W; \forall y \in W ( x R y \rightarrow y \in A ) \}
\]

% --------------------------------------------------------------------
\subsection{Intuitionistic Semantics}
% --------------------------------------------------------------------

The Semantics of Intuitionistic Logic
(\S\ref{sec:intuitionistic_logic}) is definable as Kripke Semantics
(\S\ref{sec:kripke_semantics}) or \emph{Heyting Algebra Semantics}.

\subsubsection{Heyting Algebra Semantics}\label{sec:heyting_semantics}

Instead of assigning Valuations from a Boolean Algebra, Intuitionistic
Semantics uses Values from a \emph{Heyting Algebra}
(\S\ref{sec:heyting_algebra}). A Formula is Valid if and only if it
receives the Value of the Top Element for any Valuation.

% --------------------------------------------------------------------
\subsection{Kripke-Joyal Semantics}
% --------------------------------------------------------------------

% --------------------------------------------------------------------
\subsection{Boolean-valued Model}
% --------------------------------------------------------------------

\emph{Boolean-valued Models} are related to Heyting Algebras and
Intuitionistic Logic and is equivalent to the method of Forcing.

%FIXME ref complete boolean algebra
Instead of limiting Formulas in a System $S$ to True or False, they
may be assigned values from a fixed \emph{Complete Boolean Algebra}.



% ====================================================================
\section{Proof-theoretic Semantics}
% ====================================================================

\emph{Proof-theoretic Semantics} is based on the Propositions and
Logical Connectives of Systems of Inference (Part
\ref{sec:formal_system}). See also \emph{Semantic Tableau}
(\S\ref{sec:tableau_calculus}).



% --------------------------------------------------------------------
\subsection{Logical Harmony} \label{sec:logical_harmony}
% --------------------------------------------------------------------

\emph{Logical Harmony} refers to constraints required between
Introduction and Elimination Rules.



% ====================================================================
\section{Probabilistic Semantics}
% ====================================================================

% --------------------------------------------------------------------
\subsection{Truth-value Semantics}
% --------------------------------------------------------------------

\emph{Truth-value Semantics} is also called the \emph{Substitution
  Interpretation} for Quantifiers or \emph{Substitutional
  Quantification}.



% ====================================================================
\section{Composition Semantics}
% ====================================================================

% --------------------------------------------------------------------
\subsection{Linear Semantics}
% --------------------------------------------------------------------

\subsubsection{Game Semantics}

\emph{Game Semantics} studies the \emph{Dialogical} properties of
Semantics. It is used in connection with Intuitionistic Logic
(\S\ref{sec:intuitionistic_logic}) and Linear Logic
(\S\ref{sec:linear_logic}).

\emph{Refute}



% --------------------------------------------------------------------
\subsection{Independence-friendly Semantics}
% --------------------------------------------------------------------

Semantics of Independence-friendly Logics
(\S\ref{sec:independence_logic}) is defined by Game Semantics where
players have Imperfect Information.

% --------------------------------------------------------------------
\subsection{Denotational Semantics}
% --------------------------------------------------------------------

Expressions in a Programming Language are interpreted as Mathematics
Objects called \emph{Denotations}.



% ====================================================================
\section{Operational Semantics}\label{sec:operational_semantics}
% ====================================================================

\emph{Operational Semantics} are used in the interpretation of
Programming Languages.

\emph{Concurrency}

% --------------------------------------------------------------------
\subsection{Structural Semantics}
% --------------------------------------------------------------------

% --------------------------------------------------------------------
\subsubsection{Reduction Semantics}
% --------------------------------------------------------------------

% --------------------------------------------------------------------
\subsection{Natural Semantics}
% --------------------------------------------------------------------



% ====================================================================
\section{Axiomatic Semantics}
% ====================================================================

\emph{Axiomatic Semantics} equates Meaning of a Sentence with the
Logical Formulas that describe it; that is what can be proven about it
in some System of Logic.

Axiomatic Semantics are used as a Formal Semantics of Programming
Languages.



% ====================================================================
\section{Untyped $\lambda$-Calculus}\label{sec:untyped_lambda}
% ====================================================================

\emph{Untyped $\lambda$-Calculus} describes a Semantics for
\emph{Computable Functions} (\S\ref{sec:computable_function}).

\emph{Lambda Terms}

\emph{Free Variables}

\emph{Anonymous Functions}

\emph{Lambda Abstraction}

\emph{Application}

\emph{Church-Rosser Theorem}



% --------------------------------------------------------------------
\subsection{Computation Rule}\label{sec:computation_rule}
% --------------------------------------------------------------------

\[
    (\lambda x.t)(u) :\equiv t[u/x]
\]

% --------------------------------------------------------------------
\subsection{Alpha Equivalence}\label{sec:alpha_equivalent}
% --------------------------------------------------------------------

% --------------------------------------------------------------------
\subsection{Capture-avoiding Substitution}\label{sec:capture_avoiding}
% --------------------------------------------------------------------

\emph{Capture-avoiding Substitution}

% --------------------------------------------------------------------
\subsection{Beta Reduction}\label{sec:beta_reduction}
% --------------------------------------------------------------------

% --------------------------------------------------------------------
\subsection{Eta Conversion}\label{sec:eta_conversion}
% --------------------------------------------------------------------

% --------------------------------------------------------------------
\subsection{Normalization \& Confluence}\label{sec:normalization_confluence}
% --------------------------------------------------------------------



% --------------------------------------------------------------------
\subsection{Deductive $\lambda$-Calclulus}
% --------------------------------------------------------------------
