%%%%%%%%%%%%%%%%%%%%%%%%%%%%%%%%%%%%%%%%%%%%%%%%%%%%%%%%%%%%%%%%%%%%%%
%%%%%%%%%%%%%%%%%%%%%%%%%%%%%%%%%%%%%%%%%%%%%%%%%%%%%%%%%%%%%%%%%%%%%%
\part{Appendix}\label{sec:appendix}
%%%%%%%%%%%%%%%%%%%%%%%%%%%%%%%%%%%%%%%%%%%%%%%%%%%%%%%%%%%%%%%%%%%%%%
%%%%%%%%%%%%%%%%%%%%%%%%%%%%%%%%%%%%%%%%%%%%%%%%%%%%%%%%%%%%%%%%%%%%%%

% ====================================================================
\section{Mathematical Object}\label{sec:mathematical_object}
% ====================================================================

Model Theory: \emph{Sets} and \emph{Relations}



% ====================================================================
\section{Logical Consequence}\label{sec:logical_consequence}
% ====================================================================

\emph{Logical Consequence} (\emph{Entailment}):
\[
    S \in \mathbf{L}, D \subset \mathbf{L}, S \leftrightarrow D
\]
\begin{itemize}
    \item 1. Logical Form (\S\ref{sec:normal_form})
    \item 2. A Priori
    \item 3. Modality (\S\ref{sec:modal_logic})
\end{itemize}

\emph{Syntactic Consequence}

\emph{Semantic Consequence}

\emph{Material Consequence} (\emph{Implication})

\emph{Validity}



% ====================================================================
\section{Inference}\label{sec:mathematical_inference}
% ====================================================================

% --------------------------------------------------------------------
\subsection{Deductive Inference}
% --------------------------------------------------------------------

\emph{Mathematical Induction} as an Inference Rule is the bottom-up,
\emph{Implicative} process where a \emph{Base Case} is shown to extend
to the more general by means of Implication (the Inductive step). Note
that Mathematical Induction is not \emph{Inductive Reasoning} which is
an empirical or probabilistic Inference and not a form of Deduction.
The \emph{Principle of Mathematical Induction}
(\S\ref{sec:induction_principle}) is an application of Mathematical
Induction to the \emph{Natural Numbers} (\S\ref{sec:natural_number}).



% --------------------------------------------------------------------
\subsection{Inductive Inference}
% --------------------------------------------------------------------

\emph{Universal Inductive Inference}




% ====================================================================
\section{Misc}
% ====================================================================

\emph{Note}: this section contains odds and ends, needs to be
organized and cross-referenced

\emph{Holism}, \emph{Reductionism}

\emph{Abstract}, \emph{Concrete}

\emph{Implicate}, \emph{Explicate}

\emph{A Priori}, \emph{A Posteriori}

\emph{Induction}, \emph{Deduction} (\emph{Entailment})

\emph{Analytic}, \emph{Synthetic}

\emph{Intensional}, \emph{Extensional}

\emph{Constructivism}:
\begin{itemize}
    \item \emph{Intuitionism}, \emph{Intuitionistic Logic}
    \item \emph{Finitism}
    \item \emph{Constructive Recursion}
    \item \emph{Constructive Analysis}
    \item \emph{Constructive Set Theory}
    \item \emph{Topos Theory}
    \item \emph{Realizability}
\end{itemize}
