%%%%%%%%%%%%%%%%%%%%%%%%%%%%%%%%%%%%%%%%%%%%%%%%%%%%%%%%%%%%%%%%%%%%%%
%%%%%%%%%%%%%%%%%%%%%%%%%%%%%%%%%%%%%%%%%%%%%%%%%%%%%%%%%%%%%%%%%%%%%%
\part{Appendix}\label{part:appendix}
%%%%%%%%%%%%%%%%%%%%%%%%%%%%%%%%%%%%%%%%%%%%%%%%%%%%%%%%%%%%%%%%%%%%%%
%%%%%%%%%%%%%%%%%%%%%%%%%%%%%%%%%%%%%%%%%%%%%%%%%%%%%%%%%%%%%%%%%%%%%%

% ====================================================================
\section{Notation}\label{sec:notation}
% ====================================================================

\begin{itemize}
  \item Metavariables $\alpha, \beta, \gamma, \ldots$
  \item Formulas $\Phi, \Psi, \Xi, \ldots$
  \item Variables $x, y, z, \ldots$
  \item Individuals, Functions $a, b, c, \ldots$
  \item Sets, Relations $A, B, C, \ldots$
  \item Functors $F, G, H, \ldots$
  \item Natural Transformations $\Delta, \Theta, \Xi, \ldots$
  \item Proper Classes $\class{A}, \class{B}, \class{C}, \ldots$
  \item Structures, Algebras $\struct{A}, \struct{B},
    \struct{C}, \ldots$
  \item Theories $\thy{T}, \thy{U}, \thy{V}, \ldots$
  \item Spaces $\xspace{A}, \xspace{B}, \xspace{C}, \ldots$
  \item Vectors $\vect{a}, \vect{b}, \vect{c}, \ldots$
  \item Categories, Matrices $\mat{A}, \mat{B}, \mat{C}, \ldots$
\end{itemize}



% ====================================================================
\section{Primitive Notion}\label{sec:primitive_notion}
% ====================================================================

% ====================================================================
\section{Mathematical Object}\label{sec:mathematical_object}
\cite{laycock10}
% ====================================================================

A \emph{Mathematical Object} is an \emph{Abstract Object}

\emph{Entity}, \emph{Thing}, \emph{Logical Subject} (Singular
Reference \S\ref{sec:reference})

Object (Category Theory \S\ref{sec:category_object}): the Category
Theory view equates Objects with Identity Morphisms
(\S\ref{sec:identity_morphism})

Nominalism (Antithetical) \S\ref{sec:nominalism}

Trope (Philosophy) \S\ref{sec:trope_theory}

Comprehension (\S\ref{sec:comprehension})

Bundle Theory (alternatives):
\begin{enumerate}
  \item Set of Properties: \emph{Co-instantiation}
  \item Temporal Sequence of Sets of Properties
  \item Logical Constructions of References to Properties
\end{enumerate}

Term (\S\ref{sec:term}) (Russell)

\emph{Purely Formal Concept} (Wittgenstein)

\emph{Universal Applicability} (Leibniz, Kant)

Abstract/Concrete

Universal/Particular

An \emph{Ontic} Interpretation equates Abstract/Concrete and
Universal/Particular as the same Dichotomy.

A \emph{Semantic} Interpretation equates Abstract names with
Attributes (Properties) and Concrete names with ``Things''. (Mill)

Model Theory: \emph{Sets} and \emph{Relations}



% --------------------------------------------------------------------
\subsection{Unity}\label{sec:unity}
% --------------------------------------------------------------------

\subsubsection{Restricted Unity}\label{sec:restricted_unity}

\subsubsection{Unrestricted Unity}\label{sec:unrestricted_unity}



% --------------------------------------------------------------------
\subsection{Universal}\label{sec:universal}
% --------------------------------------------------------------------

\subsubsection{Attribute}\label{sec:attribute}

\emph{Attribute} (Singular Abstraction): lacking independent existence

An Attribute gives rise to an \emph{Abstract Singular Term} as the name
of the Property, e.g. the Unary Predicate $R$ for ``is red'' gives
rise to the Abstract Singular Term ``redness''. \cite{laycock10}



\subsubsection{Kind}\label{sec:universal_kind}

\emph{Immanent Universal}



\paragraph{Species}\label{sec:species}\hfill

\emph{Count Noun}

Singular or Plural



\paragraph{Substance}\label{sec:substance}\hfill

\emph{Non-count Noun}

\emph{Mass Reference}, \emph{Natural Kind}

Neither Singular nor Plural



% --------------------------------------------------------------------
\subsection{Unique}\label{sec:unique}
% --------------------------------------------------------------------

nCatLab: generalized ``the'' -- something Characterized Uniquely up to
Unique Coherent Isomorphism %FIXME



% --------------------------------------------------------------------
\subsection{Invariance}\label{sec:invariance}
% --------------------------------------------------------------------

\subsubsection{Symmetry}\label{sec:symmetry}

Mathematical Object is \emph{Symmetric} with respect to a Mathematical
Operator (\S\ref{sec:operator}) if the Operation Applied to the Object
preserves some Property (\S\ref{sec:property}) of the Object.

Set of Operations preserving a given Property of the Object forms a
Group %FIXME ???

Automorphism: Structure Preserving Symmetry
(\S\ref{sec:structure_symmetry})



% --------------------------------------------------------------------
\subsection{Abstract Structure}\label{sec:abstract_structure}
% --------------------------------------------------------------------

% FIXME

Algebraic Structure (\S\ref{sec:algebraic_structure})

Structure (Category Theory) %FIXME

Recognizable Structure vs. Combinatorics (Part
\ref{part:combinatorics}) (Combinatorics requiring special arguments)

Structure (Bourbaki): Topological Space
(\S\ref{sec:topological_space}), Abstract Algebra
(\S\ref{sec:algebraic_structure}), Concrete Category
(\S\ref{sec:concrete_category})

Topological $\Rightarrow$ Infinitary Operations

Category Theory: Implicit

Algebraic Structures: Explicit

Representation (\S\ref{sec:representation_theory}): Instatiation
(\S\ref{sec:instantiation}), Implementation



\subsubsection{Hierarchy}\label{sec:hierarchy}

\paragraph{Subsumption Relation}\label{sec:subsumption_relation}\hfill

``Hyponym-Hypernym'' (``Is-A'') Relation

% FIXME Is-A relation is apparently a larger category of relations

Supertype-Subtype

Category Theory (Part \ref{part:category_theory})

Subobject (\S\ref{sec:subobject})



\paragraph{Tree Structure}\label{sec:tree_structure}\hfill

Tree (Order Theory \S\ref{sec:tree})

Tree Graph (\S\ref{sec:tree_graph})



% ====================================================================
\section{Map}\label{sec:map}
% ====================================================================

A \emph{Map} (or \emph{Mapping}) is a Relation (Set Relation
\S\ref{sec:set_relation}) or \emph{Morphism} (\S\ref{sec:morphism})
that is either a \emph{Function} (Set Function
\S\ref{sec:set_function}) or \emph{Multimap} (\S\ref{sec:multimap}),
depending on whether the Mapping is Determinate or not.

Operator (\S\ref{sec:operator})

A Map or Mapping is a general Term that can mean either a Function or
a Morphism, sometimes in contrast to a general Function in that a
``Map'' may be a Function of a specific type, e.g. a Continuous
Function in Topology.

% FIXME graphs?
Maps refer to the Extensional Definition of Functions (a Subset of the
Cartesian Product of the Domain and Codomain), while the underlying
Function may or may not be Extensionally Defined.

Decision Problem (\S\ref{sec:decision_problem})

Computable Function (\S\ref{sec:computable_function})



% ====================================================================
\section{Subject}\label{sec:subject}
% ====================================================================

Subject (as in \emph{Subjective})

For Logical Subject see \emph{Reference} \S\ref{sec:reference}



% --------------------------------------------------------------------
\subsection{Epistemology}\label{sec:epistemology}
\cite{chalmers02}
% --------------------------------------------------------------------

\emph{Sufficient Information}

\emph{Sufficient Reasoning}

\emph{Epistemic Dependence}: Sense $\rightarrow$ Extension

\emph{Locating Information} (Indexical \S\ref{sec:indexical})

\emph{Centered World}



\subsubsection{Verification}\label{sec:verification}



% ====================================================================
\section{Equality}\label{sec:equality}\cite{baez15}
% ====================================================================

Definitional Equality (Metalanguage): $a \equiv b$

Convertibility: $a \leftrightarrow b$

Identity: $a = b$

Equivalence Relation: $a \sim b$

Extensional/Intensional Equality $a \simeq b$

Discernibility (\S\ref{sec:discernibility})

Homotopy



\begin{description}

\item[\emph{Indiscernibility of Identicals}]: Identical objects have
the same Properties
\begin{itemize}
    \item Second-order Axiom:
    \[
        \forall x \forall y
        [ x = y \Rightarrow \forall P [ P(x) \Leftrightarrow P(y) ]]
    \]
    \item First-order Axiom Schema, where $\Phi'$ is $\Phi$ with free
      occurences of $x$ in $\Phi$ replaced with $y$:
    \[
        x = y \Rightarrow [\Phi \Rightarrow \Phi']
    \]
\end{itemize}

\item[\emph{Identity of Indiscernibles}]: objects with the same Properties
are Identical
\begin{itemize}
    \item Second-order Axiom:
    \[
        \forall x \forall y
        [ \forall P [ P(x) \Leftrightarrow P(y) ] \Rightarrow x = y
    \]
    \item First-order Axiom Schema, where $\Phi'$ is $\Phi$ with free
      occurences of $x$ in $\Phi$ replaced with $y$:
    \[
        [\Phi \Rightarrow \Phi'] \Rightarrow x = y
    \]
\end{itemize}

\item[\emph{Leibniz Law}]: combination of Indiscernibility of
  Identicals with Identity of Indiscernibles
\begin{itemize}
    \item Second-order Axiom:
    \[
        \forall x \forall y
        [ \forall P [ P(x) \Leftrightarrow P(y) ] \Leftrightarrow x = y
    \]
    \item Dual Second-order Axiom (cf. \emph{Axiom of Extensionality}
      \S\ref{sec:extensionality_axiom}):
    \[
        \forall P \forall Q
        [ \forall x [P(x) \Leftrightarrow Q(x)] \Leftrightarrow P = Q ]
    \]
    \item First-order Axiom Schema:
    \[
        x = y \Leftrightarrow [\Phi \Rightarrow \Phi']
    \]
\end{itemize}


\end{description}



% --------------------------------------------------------------------
\subsection{Structural Equality}\label{sec:structural_equality}
% --------------------------------------------------------------------

Structural Equality (\emph{Syntactic Equality}): Terms correspond to
same Tree (??? Concrete or Abstract)



% --------------------------------------------------------------------
\subsection{Observational Equality}\label{sec:observational_equality}
% --------------------------------------------------------------------



% ====================================================================
\section{Dichotomy}\label{sec:dichotomy}
% ====================================================================

List of Dichotomies:

\emph{Abstract}, \emph{Concrete}

\emph{Universal}, \emph{Particular}

\emph{Sense}, \emph{Reference}

\emph{Holism}, \emph{Reductionism}

\emph{Type}, \emph{Token}

\emph{Implicate}, \emph{Explicate}

\emph{A Priori}, \emph{A Posteriori}

\emph{Induction}, \emph{Deduction} (\emph{Entailment})

\emph{Analytic}, \emph{Synthetic}

\emph{Impredicative}, \emph{Predicative}

\emph{Intensional}, \emph{Extensional}

\emph{De Re}, \emph{De Dicto}

\emph{Real}, \emph{Apparent}

\emph{Use}, \emph{Mention}

\emph{Consumer}, \emph{Producer}

\emph{Constructivism}:
\begin{itemize}
    \item \emph{Intuitionism}, \emph{Intuitionistic Logic}
    \item \emph{Finitism}
    \item \emph{Constructive Recursion}
    \item \emph{Constructive Analysis}
    \item \emph{Constructive Set Theory}
    \item \emph{Topos Theory}
    \item \emph{Realizability}
\end{itemize}

\fist See also \emph{Adjoint Modality} (\S\ref{sec:adjoint_modality})
which is identified with Hegelian ``Unity of Opposites''



% ====================================================================
\section{Survey Table}
% ====================================================================

\begin{tabularx}{\textwidth}{| X | X | X |}
    \hline
    Decidability            & Automata Theory (AT)  & Formal Language (FL) \\
    \hline
    Syntactic Consequence   & Proof Theory (PT)     & Formal System (FS): FL + Deductive System (DS) \\
    \hline
    Semantic Consequence    & Model Theory (MT)     & Logical System (LS): FS + Interpretation (I) \\
    \hline
    Homomorphism            & Abstract Algebra (AA) & Algebraic Structure (AS): LS + Signature ($\sigma$) \\
    \hline
    Homotopy                & Topology (T)          & \\
    \hline
    Natural Transformation  & Category Theory (CT)  & Category ($\mathbf{C}$): AS + Homset (Hom) \\
    \hline
    Predicativity           & Type Theory (TT)      & Type System (TS): $\mathbf{C}$ + Typing Judgement\\
    \hline
\end{tabularx}



% ====================================================================
\section{Paradox}\label{sec:paradox}
% ====================================================================

% --------------------------------------------------------------------
\subsection{Logical Paradox}\label{sec:logical_paradox}
\cite{curry77}
% --------------------------------------------------------------------

\subsubsection{Russel's Paradox}\label{sec:russels_paradox}

\emph{Barber Pseudoparadox}

\emph{Catalogue Pseudoparadox}



\subsubsection{Burali-Forti Paradox}\label{sec:baruliforti_paradox}

\subsubsection{Cantor Paradox}\label{sec:cantor_paradox}



% --------------------------------------------------------------------
\subsection{Semantic Paradox}\label{sec:semantic_paradox}
% --------------------------------------------------------------------

\subsubsection{Liar Paradox}\label{sec:liar_paradox}

\subsubsection{Richard Paradox}\label{sec:richard_paradox}

\subsubsection{Berry Paradox}\label{sec:berry_paradox}

\subsubsection{Grelling Paradox}\label{sec:grelling_paradox}



% ====================================================================
\section{Mathematical Logic}\label{sec:mathematical_logic}
\cite{curry77}
% ====================================================================

% --------------------------------------------------------------------
\subsection{Formalism}\label{sec:formalism}
% --------------------------------------------------------------------

\subsubsection{Nominalism}\label{sec:nominalism}

\begin{enumerate}
  \item denial of Universals (\S\ref{sec:universal})
  \item denial of Abstract Objects (\S\ref{sec:mathematical_object})
\end{enumerate}



\paragraph{Trope Theory}\label{sec:trope_theory}\hfill



% --------------------------------------------------------------------
\subsection{Contensivism}\label{sec:contensivism}
% --------------------------------------------------------------------

\subsubsection{Platonism}\label{sec:platonism}

\subsubsection{Critical Contensivism}\label{sec:critical_contensivism}

\paragraph{Intuitionism}\label{sec:intuitionism}\hfill



% ====================================================================
\section{Space}\label{sec:space}
% ====================================================================

Hierarchy:

Inner Product Space \newline
$\subset$ Normed Vector Space \newline
$\subset$ Metric Space \newline
$\subset$ Topological Space

Joyal: %FIXME

Space $X$

$R$ Ring Object in a Category of Spaces

$X \xrightarrow{\quad f \quad} R$

Algebraic Geometry (Part \ref{part:algebraic_geometry})

$Hom(X,R) : Spaces \rightarrow Rings$

Spectrum (???): $Rings \rightarrow Spaces$



% ====================================================================
\section{Modularity}\label{sec:modularity}
% ====================================================================

% ====================================================================
\section{Complication}\label{sec:complication}
% ====================================================================

``Complexification'' (Goertzel)
