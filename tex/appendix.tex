%%%%%%%%%%%%%%%%%%%%%%%%%%%%%%%%%%%%%%%%%%%%%%%%%%%%%%%%%%%%%%%%%%%%%%
%%%%%%%%%%%%%%%%%%%%%%%%%%%%%%%%%%%%%%%%%%%%%%%%%%%%%%%%%%%%%%%%%%%%%%
\part{Appendix}\label{sec:appendix}
%%%%%%%%%%%%%%%%%%%%%%%%%%%%%%%%%%%%%%%%%%%%%%%%%%%%%%%%%%%%%%%%%%%%%%
%%%%%%%%%%%%%%%%%%%%%%%%%%%%%%%%%%%%%%%%%%%%%%%%%%%%%%%%%%%%%%%%%%%%%%

% ====================================================================
\section{Mathematical Object}\label{sec:mathematical_object}
\cite{laycock10}
% ====================================================================

\emph{Entity}, \emph{Thing}, \emph{Logical Subject} (Singular
Reference \S\ref{sec:reference})

Nominalism (Antithetical)

Trope (Philosophy)

Comprehension (\S\ref{sec:comprehension})

Bundle Theory (alternatives):
\begin{enumerate}
  \item Set of Properties: \emph{Co-instantiation}
  \item Temporal Sequence of Sets of Properties
  \item Logical Constructions of References to Properties
\end{enumerate}

Term (\S\ref{sec:term}) (Russell)

\emph{Purely Formal Concept} (Wittgenstein)

\emph{Universal Applicability} (Leibniz, Kant)

Model Theory: \emph{Sets} and \emph{Relations}

Abstract/Concrete

Universal/Particular

An \emph{Ontic} Interpretation equates Abstract/Concrete and
Universal/Particular as the same Dichotomy.

A \emph{Semantic} Interpretation equates Abstract names with
Attributes (Properties) and Concrete names with ``Things''. (Mill)



% --------------------------------------------------------------------
\subsection{Unity}\label{sec:unity}
% --------------------------------------------------------------------

\subsubsection{Restricted Unity}\label{sec:restricted_unity}

\subsubsection{Unrestricted Unity}\label{sec:unrestricted_unity}




% --------------------------------------------------------------------
\subsection{Universal}\label{sec:universal}
% --------------------------------------------------------------------

\subsubsection{Attribute}\label{sec:attribute}

\emph{Attribute} (Singular Abstraction): lacking independent existence

An Attribute gives rise to an \emph{Abstract Singular Term} as the name
of the Property, e.g. the Unary Predicate $R$ for ``is red'' gives
rise to the Abstract Singular Term ``redness''. \cite{laycock10}



\subsubsection{Kind}\label{sec:kind}

\emph{Immanent Universal}



\paragraph{Species}\label{sec:species}
\hfill \\

\emph{Count Noun}

Singular or Plural



\paragraph{Substance}\label{sec:substance}
\hfill \\

\emph{Non-count Noun}

\emph{Mass Reference}

Neither Singular nor Plural



% ====================================================================
\section{Subject}\label{sec:subject}
% ====================================================================

Subject (as in \emph{Subjective})

For Logical Subject see \emph{Reference} \S\ref{sec:reference}



% ====================================================================
\section{Map}\label{sec:map}
% ====================================================================

A \emph{Map} or \emph{Mapping} is a general Term that can mean either
a Function or a Morphism, sometimes in contrast to a general Function
in that a ``Map'' may be a Function of a specific type, e.g. a
Continuous Function in Topology.



% ====================================================================
\section{Equality}\label{sec:equality}\cite{baez15}
% ====================================================================

Discernibility (\S\ref{sec:discernibility})

\begin{description}

\item[\emph{Indiscernibility of Identicals}]: Identical objects have
the same Properties
\begin{itemize}
    \item Second-order Axiom:
    \[
        \forall x \forall y
        [ x = y \Rightarrow \forall P [ P(x) \Leftrightarrow P(y) ]]
    \]
    \item First-order Axiom Schema, where $\Phi'$ is $\Phi$ with free
      occurences of $x$ in $\Phi$ replaced with $y$:
    \[
        x = y \Rightarrow [\Phi \Rightarrow \Phi']
    \]
\end{itemize}

\item[\emph{Identity of Indiscernibles}]: objects with the same Properties
are Identical
\begin{itemize}
    \item Second-order Axiom:
    \[
        \forall x \forall y
        [ \forall P [ P(x) \Leftrightarrow P(y) ] \Rightarrow x = y
    \]
    \item First-order Axiom Schema, where $\Phi'$ is $\Phi$ with free
      occurences of $x$ in $\Phi$ replaced with $y$:
    \[
        [\Phi \Rightarrow \Phi'] \Rightarrow x = y
    \]
\end{itemize}

\item[\emph{Leibniz Law}]: combination of Indiscernibility of
  Identicals with Identity of Indiscernibles
\begin{itemize}
    \item Second-order Axiom:
    \[
        \forall x \forall y
        [ \forall P [ P(x) \Leftrightarrow P(y) ] \Leftrightarrow x = y
    \]
    \item Dual Second-order Axiom (cf. \emph{Axiom of Extensionality}
      \S\ref{sec:extensionality}):
    \[
        \forall P \forall Q
        [ \forall x [P(x) \Leftrightarrow Q(x)] \Leftrightarrow P = Q ]
    \]
    \item First-order Axiom Schema:
    \[
        x = y \Leftrightarrow [\Phi \Rightarrow \Phi']
    \]
\end{itemize}


\end{description}



% ====================================================================
\section{Dichotomy}\label{sec:dichotomy}
% ====================================================================

List of Dichotomies:

\emph{Abstract}, \emph{Concrete}

\emph{Universal}, \emph{Particular}

\emph{Sense}, \emph{Reference}

\emph{Holism}, \emph{Reductionism}

\emph{Type}, \emph{Token}

\emph{Implicate}, \emph{Explicate}

\emph{A Priori}, \emph{A Posteriori}

\emph{Induction}, \emph{Deduction} (\emph{Entailment})

\emph{Analytic}, \emph{Synthetic}

\emph{Impredicative}, \emph{Predicative}

\emph{Intensional}, \emph{Extensional}

\emph{De Re}, \emph{De Dicto}

\emph{Real}, \emph{Apparent}

\emph{Use}, \emph{Mention}

\emph{Constructivism}:
\begin{itemize}
    \item \emph{Intuitionism}, \emph{Intuitionistic Logic}
    \item \emph{Finitism}
    \item \emph{Constructive Recursion}
    \item \emph{Constructive Analysis}
    \item \emph{Constructive Set Theory}
    \item \emph{Topos Theory}
    \item \emph{Realizability}
\end{itemize}



% ====================================================================
\section{Survey Table}
% ====================================================================

\begin{tabularx}{\textwidth}{| X | X | X |}
    \hline
    Decidability            & Automata Theory (AT)  & Formal Language (FL) \\
    \hline
    Syntactic Consequence   & Proof Theory (PT)     & Formal System (FS): FL + Deductive System (DS) \\
    \hline
    Semantic Consequence    & Model Theory (MT)     & Logical System (LS): FS + Interpretation (I) \\
    \hline
    Homomorphism            & Abstract Algebra (AA) & Algebraic Structure (AS): LS + Signature ($\sigma$) \\
    \hline
    Homotopy                & Topology (T)          & \\
    \hline
    Natural Transformation  & Category Theory (CT)  & Category ($\mathbf{C}$): AS + Homset (Hom) \\
    \hline
    Predicativity           & Type Theory (TT)      & Type System (TS): $\mathbf{C}$ + Typing Judgement\\
    \hline
\end{tabularx}



% ====================================================================
\section{Paradox}\label{sec:paradox}
% ====================================================================

% --------------------------------------------------------------------
\subsection{Logical Paradox}\label{sec:logical_paradox}
\cite{curry77}
% --------------------------------------------------------------------

\subsubsection{Russel's Paradox}\label{sec:russels_paradox}

\emph{Barber Pseudoparadox}

\emph{Catalogue Pseudoparadox}



\subsubsection{Burali-Forti Paradox}\label{sec:baruliforti_paradox}

\subsubsection{Cantor Paradox}\label{sec:cantor_paradox}



% --------------------------------------------------------------------
\subsection{Semantic Paradox}\label{sec:semantic_paradox}
% --------------------------------------------------------------------

\subsubsection{Liar Paradox}\label{sec:liar_paradox}

\subsubsection{Richard Paradox}\label{sec:richard_paradox}

\subsubsection{Berry Paradox}\label{sec:berry_paradox}

\subsubsection{Grelling Paradox}\label{sec:grelling_paradox}



% ====================================================================
\section{Mathematical Logic}\label{sec:mathematical_logic}
\cite{curry77}
% ====================================================================

% --------------------------------------------------------------------
\subsection{Formalism}\label{sec:formalism}
% --------------------------------------------------------------------

\subsubsection{Nominalism}\label{sec:nominalism}



% --------------------------------------------------------------------
\subsection{Contensivism}\label{sec:contensivism}
% --------------------------------------------------------------------

\subsubsection{Platonism}\label{sec:platonism}

\subsubsection{Critical Contensivism}\label{sec:critical_contensivism}

\paragraph{Intuitionism}\label{sec:intuitionism}
