%%%%%%%%%%%%%%%%%%%%%%%%%%%%%%%%%%%%%%%%%%%%%%%%%%%%%%%%%%%%%%%%%%%%%%
%%%%%%%%%%%%%%%%%%%%%%%%%%%%%%%%%%%%%%%%%%%%%%%%%%%%%%%%%%%%%%%%%%%%%%
\part{Appendix}\label{sec:appendix}
%%%%%%%%%%%%%%%%%%%%%%%%%%%%%%%%%%%%%%%%%%%%%%%%%%%%%%%%%%%%%%%%%%%%%%
%%%%%%%%%%%%%%%%%%%%%%%%%%%%%%%%%%%%%%%%%%%%%%%%%%%%%%%%%%%%%%%%%%%%%%

% ====================================================================
\section{Mathematical Object}\label{sec:mathematical_object}
% ====================================================================

Model Theory: \emph{Sets} and \emph{Relations}



\subsection{Variable}\label{sec:variable}



% ====================================================================
\section{Consequence}\label{sec:consequence}
% ====================================================================

% --------------------------------------------------------------------
\subsection{Logical Consequence}\label{sec:logical_consequence}
% --------------------------------------------------------------------

\emph{Logical Consequence} (\emph{Entailment}):
\[
    S \in \mathbf{L}, D \subset \mathbf{L}, S \leftrightarrow D
\]
\begin{itemize}
    \item 1. Logical Form (\S\ref{sec:normal_form})
    \item 2. A Priori
    \item 3. Modality (\S\ref{sec:modal_logic})
\end{itemize}

\emph{Syntactic Consequence}

\emph{Semantic Consequence}

\emph{Validity}



% --------------------------------------------------------------------
\subsection{Material Consequence}\label{sec:material_consequence}
% --------------------------------------------------------------------

\emph{Material Consequence} (\emph{Implication})



% ====================================================================
\section{Inference}\label{sec:mathematical_inference}
% ====================================================================

% --------------------------------------------------------------------
\subsection{Deductive Inference}
% --------------------------------------------------------------------

\emph{Mathematical Induction} as an Inference Rule is the bottom-up,
\emph{Implicative} process where a \emph{Base Case} is shown to extend
to the more general by means of Implication (the Inductive step). Note
that Mathematical Induction is not \emph{Inductive Reasoning} which is
an empirical or probabilistic Inference and not a form of Deduction.
The \emph{Principle of Mathematical Induction}
(\S\ref{sec:induction_principle}) is an application of Mathematical
Induction to the \emph{Natural Numbers} (\S\ref{sec:natural_number}).



% --------------------------------------------------------------------
\subsection{Inductive Inference}
% --------------------------------------------------------------------

\emph{Universal Inductive Inference}




% ====================================================================
\section{Equality}\label{sec:equality}
% ====================================================================

\begin{description}
\item[\emph{Indiscernibility of Identicals}]: Identical objects have
the same Properties (Second-order Axiom or First-order Axiom Schema)

\item[\emph{Identity of Indiscernibles}]: objects with the same Properties
are Identical (Second-order Axiom or First-order Axiom Schema)
\end{description}



% ====================================================================
\section{Misc}
% ====================================================================

\emph{Note}: this section contains odds and ends, needs to be
organized and cross-referenced

\emph{Holism}, \emph{Reductionism}

\emph{Abstract}, \emph{Concrete}

\emph{Implicate}, \emph{Explicate}

\emph{A Priori}, \emph{A Posteriori}

\emph{Induction}, \emph{Deduction} (\emph{Entailment})

\emph{Analytic}, \emph{Synthetic}

\emph{Impredicative}, \emph{Predicative}

\emph{Intensional}, \emph{Extensional}

\emph{Real}, \emph{Apparent}

\emph{Constructivism}:
\begin{itemize}
    \item \emph{Intuitionism}, \emph{Intuitionistic Logic}
    \item \emph{Finitism}
    \item \emph{Constructive Recursion}
    \item \emph{Constructive Analysis}
    \item \emph{Constructive Set Theory}
    \item \emph{Topos Theory}
    \item \emph{Realizability}
\end{itemize}



% ====================================================================
\section{Survey Table}
% ====================================================================

\begin{tabularx}{\textwidth}{| X | X | X |}
    \hline
    Decidability            & Automata Theory (AT)  & Formal Language (FL) \\
    \hline
    Syntactic Consequence   & Proof Theory (PT)     & Formal System (FS): FL + Deductive System (DS) \\
    \hline
    Semantic Consequence    & Model Theory (MT)     & Logical System (LS): FS + Interpretation (I) \\
    \hline
    Homomorphism            & Abstract Algebra (AA) & Algebraic Structure (AS): LS + Signature ($\sigma$) \\
    \hline
    Homotopy                & Topology (T)          & \\
    \hline
    Natural Transformation  & Category Theory (CT)  & Category ($\mathbf{C}$): AS + Homset (Hom) \\
    \hline
    Predicativity           & Type Theory (TT)      & Type System (TS): $\mathbf{C}$ + Typing Judgement\\
    \hline
\end{tabularx}
