%%%%%%%%%%%%%%%%%%%%%%%%%%%%%%%%%%%%%%%%%%%%%%%%%%%%%%%%%%%%%%%%%%%%%%%%%%%%%%%%
%%%%%%%%%%%%%%%%%%%%%%%%%%%%%%%%%%%%%%%%%%%%%%%%%%%%%%%%%%%%%%%%%%%%%%%%%%%%%%%%
\part{Appendix}\label{part:appendix}
%%%%%%%%%%%%%%%%%%%%%%%%%%%%%%%%%%%%%%%%%%%%%%%%%%%%%%%%%%%%%%%%%%%%%%%%%%%%%%%%
%%%%%%%%%%%%%%%%%%%%%%%%%%%%%%%%%%%%%%%%%%%%%%%%%%%%%%%%%%%%%%%%%%%%%%%%%%%%%%%%

% ==============================================================================
\section{Notation}\label{sec:notation}
% ==============================================================================

\begin{itemize}
  \item Metavariables $\alpha, \beta, \gamma, \ldots$
  \item Formulas $\Phi, \Psi, \Xi, \ldots$
  \item Variables $x, y, z, \ldots$
  \item Individuals, Functions $a, b, c, \ldots$
  \item Sets, Relations $A, B, C, \ldots$
  \item Functors $F, G, H, \ldots$
  \item Natural Transformations $\Delta, \Theta, \Xi, \ldots$
  \item Proper Classes $\class{A}, \class{B}, \class{C}, \ldots$
  \item Structures, Algebras $\struct{A}, \struct{B},
    \struct{C}, \ldots$
  \item Theories $\thy{T}, \thy{U}, \thy{V}, \ldots$
  \item Spaces $\xspace{A}, \xspace{B}, \xspace{C}, \ldots$
  \item Vectors $\vect{a}, \vect{b}, \vect{c}, \ldots$
  \item Categories, Matrices $\mat{A}, \mat{B}, \mat{C}, \ldots$
\end{itemize}

\fist \emph{Notes on Notation} -  \url{https://github.com/hypotext/notation}



% ==============================================================================
\section{Primitive Notion}\label{sec:primitive_notion}
% ==============================================================================

Geometric Primitives (\S\ref{sec:geometric_primitive}):
\begin{itemize}
  \item Geometric Objects:
    \begin{itemize}
      \item Point
      \item Line
      \item Plane
    \end{itemize}
  \item Geometric Relations:
    \begin{itemize}
      \item Incidence
      \item Intermediacy
    \end{itemize}
\end{itemize}

cf. Per Martin-L\"of - \emph{On the Meanings of the Logical Constants and
  Justification of Logical Laws}



% ==============================================================================
\section{Mathematical Object}\label{sec:mathematical_object}
\cite{laycock10}
% ==============================================================================

A \emph{Mathematical Object} is an \emph{Abstract Object}

\emph{Entity}, \emph{Thing}, \emph{Logical Subject} (Singular
Reference \S\ref{sec:reference})

Object (Category Theory \S\ref{sec:category_object}): the Category
Theory view equates Objects with Identity Morphisms
(\S\ref{sec:identity_morphism})

Nominalism (Antithetical) \S\ref{sec:nominalism}

Trope (Philosophy) \S\ref{sec:trope_theory}

Comprehension (\S\ref{sec:comprehension})

Bundle Theory (alternatives):
\begin{enumerate}
  \item Set of Properties: \emph{Co-instantiation}
  \item Temporal Sequence of Sets of Properties
  \item Logical Constructions of References to Properties
\end{enumerate}

Term (\S\ref{sec:term}) (Russell)

\emph{Purely Formal Concept} (Wittgenstein)

\emph{Universal Applicability} (Leibniz, Kant)

Abstract/Concrete

Universal/Particular

An \emph{Ontic} Interpretation equates Abstract/Concrete and
Universal/Particular as the same Dichotomy.

A \emph{Semantic} Interpretation equates Abstract names with
Attributes (Properties) and Concrete names with ``Things''. (Mill)

Model Theory: \emph{Sets} and \emph{Relations}



% ------------------------------------------------------------------------------
\subsection{Unity}\label{sec:unity}
% ------------------------------------------------------------------------------

\subsubsection{Restricted Unity}\label{sec:restricted_unity}

\subsubsection{Unrestricted Unity}\label{sec:unrestricted_unity}



% ------------------------------------------------------------------------------
\subsection{Universal}\label{sec:universal}
% ------------------------------------------------------------------------------

\subsubsection{Attribute}\label{sec:attribute}

\emph{Attribute} (Singular Abstraction): lacking independent existence

An Attribute gives rise to an \emph{Abstract Singular Term} as the name
of the Property, e.g. the Unary Predicate $R$ for ``is red'' gives
rise to the Abstract Singular Term ``redness''. \cite{laycock10}

\fist cf. Attribute (Relational Database \S\ref{sec:database_attribute})



\subsubsection{Kind}\label{sec:universal_kind}

\emph{Immanent Universal}



\paragraph{Species}\label{sec:species}\hfill

\emph{Count Noun}

Singular or Plural



\paragraph{Substance}\label{sec:substance}\hfill

\emph{Non-count Noun}

\emph{Mass Reference}, \emph{Natural Kind}

Neither Singular nor Plural



% ------------------------------------------------------------------------------
\subsection{Unique}\label{sec:unique}
% ------------------------------------------------------------------------------

nLab: generalized ``the'' -- something Characterized Uniquely up to
Unique Coherent Isomorphism %FIXME



% ------------------------------------------------------------------------------
\subsection{Exceptional}\label{sec:exceptional}
% ------------------------------------------------------------------------------

\fist Singularity (\S\ref{sec:singularity}) -- Point at which an Exceptional Set
fails to be ``well-behaved''

(wiki):

\emph{Exceptional Object}

Classifications (\S\ref{sec:classification_theorem}) tend to result in a number
of (Infinite) Series of Objects, and a Finite number of \emph{Exceptions} that
do not fit into any Series, and such \emph{Exceptional Objects} are often
related to Exceptional Objects in other branches of Mathematics

\begin{itemize}
  \item Exceptional Jordan Algebra (\S\ref{sec:exceptional_jordan_algebra})
  \item ...
\end{itemize}

\emph{Exceptional Isomorphisms}



% ------------------------------------------------------------------------------
\subsection{Invariance}\label{sec:invariance}
% ------------------------------------------------------------------------------

\subsubsection{Symmetry}\label{sec:symmetry}

Mathematical Object is \emph{Symmetric} with respect to a Mathematical
Operator (\S\ref{sec:operator}) if the Operation Applied to the Object
preserves some Property (\S\ref{sec:property}) of the Object.

Set of Operations preserving a given Property of the Object forms a
Group %FIXME ???

Automorphism: Structure Preserving Symmetry
(\S\ref{sec:structure_symmetry})

When a Group $G$ Acts \emph{Transitively} on $X$ if $X$ is Non-empty and if for
each pair $x,y \in X, \exists g \in G : g \cdot x = y$.

When a Group $G$ Acts Transitively (\S\ref{sec:transitive_action}) on a
Topological Space $X$, $X$ is called the \emph{Homogeneous Space}
(\S\ref{sec:homogeneous_space}) of $G$, and the Elements of $G$ are called the
\emph{Symmetries} of $X$.



% ------------------------------------------------------------------------------
\subsection{Duality}\label{sec:duality}
% ------------------------------------------------------------------------------

(wiki): \emph{Translation} of ``Concepts, Theorems, or Mathematical
Structures'' into other Concepts, Theorems, or Mathematical Structures in a
one-to-one fashion, usually by means of an \emph{Involution}
(\S\ref{sec:involution}).

cf. Stone Duality (Algebraic Logic \S\ref{sec:stone_duality})

\url{https://terrytao.wordpress.com/2010/06/25/the-uncertainty-principle/}:

\emph{Internal vs. External Duality}

a Mathematical Object $X$ can either be described:
\begin{itemize}
\item \emph{Internally} (or in \emph{Physical Space} or \emph{Locally}) by
  describing what $X$ ``consists of'', or what kind of Maps exist \emph{into}
  $X$
\end{itemize}
or:
\begin{itemize}
\item \emph{Externally} (or in \emph{Frequency Space} or \emph{Globally}) by
  describing what $X$ ``globally interacts or resonates with'', or what kind of
  Maps exist \emph{out of} $X$
\end{itemize}
and often these two perspectives on an Object $X$ are \emph{Dual} to eachother:
\begin{itemize}
\item an Operation on $X$ may transform it one way in Physical Space, but in a
  Dual way in Frequency Space, with the Frequency Space ``description'' often
  being an ``Inversion'' of the Physical Space description
\end{itemize}

some Fundamental ``Duality'' Theorems:
\begin{itemize}
  \item \emph{Semantic-Syntactic Duality} (G\"odel Completeness Theorem) --
    a Mathematical Theory can be described:
    \begin{itemize}
      \item Internally or \emph{Syntactically} by its \emph{Axioms and Theorems}
      \item Externally or \emph{Semantically} by its \emph{Models}
    \end{itemize}
  \item \emph{Group Duality} (Cayley's Theorem) -- a Group $G$ can be
    described:
    \begin{itemize}
      \item Internally by \emph{Presentations} (lists of \emph{Generators}
        together with Relations between them)
      \item Externally by \emph{Representations} (``realizations'' of the Group
        in some more ``concrete'' Group of Transformations)
    \end{itemize}

  \item \emph{Stone/Gelfand Duality} -- TODO

  \item \emph{Intrinsic/Extrinsic Duality} -- TODO

  \item \emph{Vector Space Duality} -- TODO

  \item \emph{Ideal/Variety Duality} -- TODO

  \item \emph{Convex Duality} (Farkas Lemma) -- a Closed, Bounded Convex Boxy
    $K$ in a Vector Space $V$ can be described:
    \begin{itemize}
      \item Internally by listing a Set of Points defining the \emph{Convex
        Hull} of $K$
      \item Externally by listing a Set of (Irreducible) Linear Inequalities
        that ``cut out'' $K$
    \end{itemize}

  \item \emph{Hilbert Space Duality} -- TODO

  \item \emph{Fourier Duality} -- TODO

  \item \emph{Uncertainty Principle} -- TODO

  \item \emph{Projective Plane Duality} (\S\ref{sec:plane_duality})

  \item ... MORE
\end{itemize}

TODO: xrefs



% ------------------------------------------------------------------------------
\subsection{Abstraction}\label{sec:abstraction}
% ------------------------------------------------------------------------------

\subsubsection{Concept}\label{sec:concept}

\subsubsection{Instantiation}\label{sec:instantiation}

\fist cf. Reification (\S\ref{sec:reification})



% ------------------------------------------------------------------------------
\subsection{Abstract Structure}\label{sec:abstract_structure}
% ------------------------------------------------------------------------------

% FIXME

Algebraic Structure (\S\ref{sec:algebraic_structure})

Structure (Category Theory) %FIXME

Recognizable Structure vs. Combinatorics (Part
\ref{part:combinatorics}) (Combinatorics requiring special arguments)

Structure (Bourbaki): Topological Space
(\S\ref{sec:topological_space}), Abstract Algebra
(\S\ref{sec:algebraic_structure}), Concrete Category
(\S\ref{sec:concrete_category})

Topological $\Rightarrow$ Infinitary Operations

Category Theory: Implicit

Algebraic Structures: Explicit

Representation (\S\ref{sec:representation_theory}): Instatiation
(\S\ref{sec:instantiation}), Implementation



\subsubsection{Heterarchy}\label{sec:heterarchy}

\subsubsection{Hierarchy}\label{sec:hierarchy}

\paragraph{Subsumption Relation}\label{sec:subsumption_relation}\hfill

``Hyponym-Hypernym'' (``Is-A'') Relation

% FIXME Is-A relation is apparently a larger category of relations

Supertype-Subtype

Category Theory (Part \ref{part:category_theory})

Subobject (\S\ref{sec:subobject})



\paragraph{Tree Structure}\label{sec:tree_structure}\hfill

Tree (Order Theory \S\ref{sec:tree})

Tree Graph (\S\ref{sec:tree_graph})



\paragraph{Strange Loop}\label{sec:strange_loop}\hfill



% ==============================================================================
\section{Map}\label{sec:map}
% ==============================================================================

A \emph{Map} (or \emph{Mapping}) is a Relation (Set Relation
\S\ref{sec:set_relation}) or \emph{Morphism} (\S\ref{sec:morphism})
that is either a \emph{Function} (Set Function
\S\ref{sec:set_function}) or \emph{Multimap} (\S\ref{sec:multimap}),
depending on whether the Mapping is Determinate or not.

Operator (\S\ref{sec:operator})

A Map or Mapping is a general Term that can mean either a Function or
a Morphism, sometimes in contrast to a general Function in that a
``Map'' may be a Function of a specific type, e.g. a Continuous
Function in Topology.

% FIXME graphs?
Maps refer to the Extensional Definition of Functions (a Subset of the
Cartesian Product of the Domain and Codomain), while the underlying
Function may or may not be Extensionally Defined.

Decision Problem (\S\ref{sec:decision_problem})

Computable Function (\S\ref{sec:computable_function})



% ==============================================================================
\section{Subject}\label{sec:subject}
% ==============================================================================

Subject (as in \emph{Subjective})

For Logical Subject see \emph{Reference} \S\ref{sec:reference}



% ------------------------------------------------------------------------------
\subsection{Epistemology}\label{sec:epistemology}
\cite{chalmers02}
% ------------------------------------------------------------------------------

\emph{Sufficient Information}

\emph{Sufficient Reasoning}

\emph{Epistemic Dependence}: Sense $\rightarrow$ Extension

\emph{Locating Information} (Indexical \S\ref{sec:indexical})

\emph{Centered World}



\subsubsection{Verification}\label{sec:verification}



% ==============================================================================
\section{Equality}\label{sec:equality}\cite{baez15}
% ==============================================================================

Definitional Equality (Metalanguage): $a \equiv b$

Convertibility: $a \leftrightarrow b$

Identity: $a = b$

Equivalence Relation: $a \sim b$

Extensional/Intensional Equality $a \simeq b$

Discernibility (\S\ref{sec:discernibility})

Homotopy



\begin{description}

\item[\emph{Indiscernibility of Identicals}]: Identical objects have
the same Properties
\begin{itemize}
    \item Second-order Axiom:
    \[
        \forall x \forall y
        [ x = y \Rightarrow \forall P [ P(x) \Leftrightarrow P(y) ]]
    \]
    \item First-order Axiom Schema, where $\Phi'$ is $\Phi$ with free
      occurences of $x$ in $\Phi$ replaced with $y$:
    \[
        x = y \Rightarrow [\Phi \Rightarrow \Phi']
    \]
\end{itemize}

\item[\emph{Identity of Indiscernibles}]: objects with the same Properties
are Identical
\begin{itemize}
    \item Second-order Axiom:
    \[
        \forall x \forall y
        [ \forall P [ P(x) \Leftrightarrow P(y) ] \Rightarrow x = y
    \]
    \item First-order Axiom Schema, where $\Phi'$ is $\Phi$ with free
      occurences of $x$ in $\Phi$ replaced with $y$:
    \[
        [\Phi \Rightarrow \Phi'] \Rightarrow x = y
    \]
\end{itemize}

\item[\emph{Leibniz Law}]: combination of Indiscernibility of
  Identicals with Identity of Indiscernibles
\begin{itemize}
    \item Second-order Axiom:
    \[
        \forall x \forall y
        [ \forall P [ P(x) \Leftrightarrow P(y) ] \Leftrightarrow x = y
    \]
    \item Dual Second-order Axiom (cf. \emph{Axiom of Extensionality}
      \S\ref{sec:extensionality_axiom}):
    \[
        \forall P \forall Q
        [ \forall x [P(x) \Leftrightarrow Q(x)] \Leftrightarrow P = Q ]
    \]
    \item First-order Axiom Schema:
    \[
        x = y \Leftrightarrow [\Phi \Rightarrow \Phi']
    \]
\end{itemize}


\end{description}



% ------------------------------------------------------------------------------
\subsection{Intensional Equality}\label{sec:intensional_equality}
% ------------------------------------------------------------------------------

(or \emph{Definitional Equality})



% ------------------------------------------------------------------------------
\subsection{Extensional Equality}\label{sec:extensional_equality}
% ------------------------------------------------------------------------------

(or \emph{Observational Equality})

Observational Equality: %FIXME same as extensional?

Weak Bisimulation (\S\ref{sec:bisimulation})

Observational Type Theory (\S\ref{sec:observational_type})



% ------------------------------------------------------------------------------
\subsection{Structural Equality}\label{sec:structural_equality}
% ------------------------------------------------------------------------------

Structural Equality (\emph{Syntactic Equality}): Terms correspond to
same Tree (??? Concrete or Abstract)



% ==============================================================================
\section{Dichotomy}\label{sec:dichotomy}
% ==============================================================================

List of Dichotomies:

\emph{Abstract}, \emph{Concrete}

\emph{Continuous}, \emph{Discrete} %FIXME order?

\emph{Universal}, \emph{Particular}

\emph{Sense}, \emph{Reference}

\emph{Holism}, \emph{Reductionism}

\emph{Type}, \emph{Token}

\emph{Implicate}, \emph{Explicate}

\emph{A Priori}, \emph{A Posteriori}

\emph{Induction}, \emph{Deduction} (\emph{Entailment})

\emph{Analytic}, \emph{Synthetic}

\emph{Impredicative}, \emph{Predicative}

\emph{Intensional}, \emph{Extensional}

\emph{De Re}, \emph{De Dicto}

\emph{Real}, \emph{Apparent}

\emph{Use}, \emph{Mention}

\emph{Consumer}, \emph{Producer}

\emph{Constructivism}:
\begin{itemize}
    \item \emph{Intuitionism}, \emph{Intuitionistic Logic}
    \item \emph{Finitism}
    \item \emph{Constructive Recursion}
    \item \emph{Constructive Analysis}
    \item \emph{Constructive Set Theory}
    \item \emph{Topos Theory}
    \item \emph{Realizability}
\end{itemize}

\fist See also \emph{Adjoint Modality} (\S\ref{sec:adjoint_modality})
which is identified with Hegelian ``Unity of Opposites''



% ==============================================================================
\section{Survey Table}
% ==============================================================================

\begin{tabularx}{\textwidth}{| X | X | X |}
    \hline
    Decidability            & Automata Theory (AT)  & Formal Language (FL) \\
    \hline
    Syntactic Consequence   & Proof Theory (PT)     & Formal System (FS): FL + Deductive System (DS) \\
    \hline
    Semantic Consequence    & Model Theory (MT)     & Logical System (LS): FS + Interpretation (I) \\
    \hline
    Homomorphism            & Abstract Algebra (AA) & Algebraic Structure (AS): LS + Signature ($\sigma$) \\
    \hline
    Homotopy                & Topology (T)          & \\
    \hline
    Natural Transformation  & Category Theory (CT)  & Category ($\mathbf{C}$): AS + Homset (Hom) \\
    \hline
    Predicativity           & Type Theory (TT)      & Type System (TS): $\mathbf{C}$ + Typing Judgement \\
    \hline
\end{tabularx}



% ==============================================================================
\section{Paradox}\label{sec:paradox}
% ==============================================================================

% ------------------------------------------------------------------------------
\subsection{Logical Paradox}\label{sec:logical_paradox}
\cite{curry77}
% ------------------------------------------------------------------------------

\subsubsection{Russell's Paradox}\label{sec:russells_paradox}

\emph{Barber Pseudoparadox}

\emph{Catalogue Pseudoparadox}

in Type Theory: Girard's Paradox (\S\ref{sec:girards_paradox})



\subsubsection{Burali-Forti Paradox}\label{sec:baruliforti_paradox}

\subsubsection{Cantor Paradox}\label{sec:cantor_paradox}

\subsubsection{Curry's Paradox}\label{sec:currys_paradox}

Rules used to construct the Proof:
\begin{itemize}
  \item Rule of Assumption
  \item Contraction
  \item Modus Ponens
\end{itemize}

Curry's Paradox arises when $\lambda$-calculus
(\S\ref{sec:untyped_lambda}) is embedded in a Logic with Modus Ponens
(Binary Implication Operator)



% ------------------------------------------------------------------------------
\subsection{Semantic Paradox}\label{sec:semantic_paradox}
% ------------------------------------------------------------------------------

\subsubsection{Liar Paradox}\label{sec:liar_paradox}

\subsubsection{Richard Paradox}\label{sec:richard_paradox}

\url{http://fexpr.blogspot.com/2013/07/bypassing-no-go-theorems.html}:

``concerns the use of an Expression in some Class to Designate an
Object that by definition cannot be Designated by Expressions of that
Class''

e.g. the 21-syllable sentence ``the least natural number not nameable
in fewer than twenty-two syllables''

a form of Richard Paradox is facilitated in $\lambda$-calculus
(\S\ref{sec:untyped_lambda}) by ``first-class status'' of Functions



\subsubsection{Berry Paradox}\label{sec:berry_paradox}

\subsubsection{Grelling Paradox}\label{sec:grelling_paradox}



% ==============================================================================
\section{Reasoning}\label{sec:reasoning}
% ==============================================================================

\begin{itemize}
  \item Diagrammatic (Visual) Reasoning
  \item Linguistic (Formal ???) Reasoning
  \item Algebraic (Equational) Reasoning
\end{itemize}

TODO: xrefs



% ==============================================================================
\section{Mathematical Logic}\label{sec:mathematical_logic}
\cite{curry77}
% ==============================================================================

% ------------------------------------------------------------------------------
\subsection{Formalism}\label{sec:formalism}
% ------------------------------------------------------------------------------

\subsubsection{Nominalism}\label{sec:nominalism}

\begin{enumerate}
  \item denial of Universals (\S\ref{sec:universal})
  \item denial of Abstract Objects (\S\ref{sec:mathematical_object})
\end{enumerate}



\paragraph{Trope Theory}\label{sec:trope_theory}\hfill



% ------------------------------------------------------------------------------
\subsection{Contensivism}\label{sec:contensivism}
% ------------------------------------------------------------------------------

\subsubsection{Platonism}\label{sec:platonism}

\subsubsection{Critical Contensivism}\label{sec:critical_contensivism}

\paragraph{Intuitionism}\label{sec:intuitionism}\hfill



% ==============================================================================
\section{Mathematical Expression}\label{sec:mathematical_expression}
% ==============================================================================

Mathematical Expression $\supset$
Analytic Expression     $\supset$
Closed-form Expression  $\supset$
Algebraic Expression    $\supset$
Polynomial Expressions  $\supset$
Arithmetic Expression

\begin{itemize}
  \item Arithmetic Expression (\S\ref{sec:arithmetic_expression})
  \item Polynomial Expression (\S\ref{sec:polynomial_expression})
  \item Algebraic Expression (\S\ref{sec:algebraic_expression})
  \item Closed-form Expression (\S\ref{sec:closed_form_expression})
  \item Analytic Expression (\S\ref{sec:analytic_expression})
\end{itemize}

\emph{Tarski-Seidenerg Theorem} (\S\ref{sec:tarski_seidenberg}): the First-order
Theory (\S\ref{sec:firstorder_theory}) of the Real Field is \emph{Decidable}:
every Formula constructed from Polynomial Equations and Inequalities by the
Logical Connectives $\vee$, $\wedge$, $\neg$, and Quantifiers $\forall$,
$\exists$ is equivalent to a Formula without Quantifiers; Euclidean Geometry
without the ability to measure Angles is also a Model of the Real Field Axioms
and therefore also decidable

(Laczkovich03) refinement of \emph{Richardson's Theorem}
(\S\ref{sec:richardsons_theorem}) the use of $\pi$ can be removed and the use
of Composition reduced:

given an Expression $A(x)$ in the Ring generated by the Integers, $x$, $\sin
x^n$, and $\sin(x \sin x^n)$, the question whether $A(x) > 0$ for some $x$ and
whether $A(x) = 0$ for some $x$ are \emph{Unsolvable}



% ==============================================================================
\section{Space}\label{sec:space}
% ==============================================================================

\begin{itemize}
  \item Inner Product Space \newline
    $\subset$ Normed Vector Space \newline
    $\subset$ Metric Space \newline
    $\subset$ Topological Space
\end{itemize}

Synthetic Geometry (Part \ref{part:synthetic_geometry}) -- Axiomatizations
describe Properties of Points in a Space (Elements of a Set)

Antisets (\S\ref{sec:antiset}): Sets of \emph{Abstract Spaces}

Joyal: %FIXME

Space $X$

$R$ Ring Object in a Category of Spaces

$X \xrightarrow{\quad f \quad} R$

Algebraic Geometry (Part \ref{part:algebraic_geometry})

$Hom(X,R) : Spaces \rightarrow Rings$

Spectrum (???): $Rings \rightarrow Spaces$

\begin{itemize}
  \item Phase Space (Systems Theory \S\ref{sec:phase_space})
  \item Moduli Space (Algebraic Geometry \S\ref{sec:moduli_space})
\end{itemize}



% ==============================================================================
\section{Modularity}\label{sec:modularity}
% ==============================================================================

% ==============================================================================
\section{Complication}\label{sec:complication}
% ==============================================================================

``Complexification'' (Goertzel)



% ==============================================================================
\section{Anti-induction}\label{sec:anti_induction}
% ==============================================================================

e.g. \emph{stock markets} can be said to behave Anti-inductively
