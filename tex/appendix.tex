%%%%%%%%%%%%%%%%%%%%%%%%%%%%%%%%%%%%%%%%%%%%%%%%%%%%%%%%%%%%%%%%%%%%%%
%%%%%%%%%%%%%%%%%%%%%%%%%%%%%%%%%%%%%%%%%%%%%%%%%%%%%%%%%%%%%%%%%%%%%%
\part{Appendix}\label{sec:appendix}
%%%%%%%%%%%%%%%%%%%%%%%%%%%%%%%%%%%%%%%%%%%%%%%%%%%%%%%%%%%%%%%%%%%%%%
%%%%%%%%%%%%%%%%%%%%%%%%%%%%%%%%%%%%%%%%%%%%%%%%%%%%%%%%%%%%%%%%%%%%%%

% ====================================================================
\section{Mathematical Object}\label{sec:mathematical_object}
% ====================================================================

Model Theory: \emph{Sets} and \emph{Relations}



% ====================================================================
\section{Equality}\label{sec:equality}\cite{baez15}
% ====================================================================

\begin{description}

\item[\emph{Indiscernibility of Identicals}]: Identical objects have
the same Properties
\begin{itemize}
    \item Second-order Axiom:
    \[
        \forall x \forall y
        [ x = y \Rightarrow \forall P [ P(x) \Leftrightarrow P(y) ]]
    \]
    \item First-order Axiom Schema, where $\Phi'$ is $\Phi$ with free
      occurences of $x$ in $\Phi$ replaced with $y$:
    \[
        x = y \Rightarrow [\Phi \Rightarrow \Phi']
    \]
\end{itemize}

\item[\emph{Identity of Indiscernibles}]: objects with the same Properties
are Identical
\begin{itemize}
    \item Second-order Axiom:
    \[
        \forall x \forall y
        [ \forall P [ P(x) \Leftrightarrow P(y) ] \Rightarrow x = y
    \]
    \item First-order Axiom Schema, where $\Phi'$ is $\Phi$ with free
      occurences of $x$ in $\Phi$ replaced with $y$:
    \[
        [\Phi \Rightarrow \Phi'] \Rightarrow x = y
    \]
\end{itemize}

\item[\emph{Leibniz Law}]: combination of Indiscernibility of
  Identicals with Identity of Indiscernibles
\begin{itemize}
    \item Second-order Axiom:
    \[
        \forall x \forall y
        [ \forall P [ P(x) \Leftrightarrow P(y) ] \Leftrightarrow x = y
    \]
    \item Dual Second-order Axiom (cf. \emph{Axiom of Extensionality}
      \S\ref{sec:extensionality}):
    \[
        \forall P \forall Q
        [ \forall x [P(x) \Leftrightarrow Q(x)] \Leftrightarrow P = Q ]
    \]
    \item First-order Axiom Schema:
    \[
        x = y \Leftrightarrow [\Phi \Rightarrow \Phi']
    \]
\end{itemize}


\end{description}



% --------------------------------------------------------------------
\subsection{Discernibility}\label{sec:discernibility}
\cite{ladyman-linnebo-pettigrew11}
% --------------------------------------------------------------------

\emph{Trivially Discernible}



\subsubsection{Absolute Discernibility}\label{sec:absolute_discernibility}

Two objects are \emph{Absolutely Discernible} (or \emph{Monadically
  Discernible}) when one object has any kind of Property that the
other lacks.



\subsubsection{Relative Discernibility}\label{sec:relative_discernibility}

Two objects are \emph{Relatively Discernible} when the first stands in
an Asymmetric Binary Relation to the second, i.e. for objects $a$ and
$b$ and Relation $R$, it is the case that $aRb$ but not the case that
$bRa$.



\subsubsection{Weak Discernibility}\label{sec:weak_discernibility}

Two objects are \emph{Weakly Discernible} when the first stands in an
Irreflexive Binary Relation to the second, i.e. for objects $a$ and
$b$ and Relation $R$, it is the case that $aRb$ but not the case that
$aRa$.

\paragraph{Very Weak Discernibility}\label{sec:very_weak_discernibility}

\paragraph{Hilbert-Bernays Discernibility}
\label{sec:hilbert_bernays_discernibility}

\emph{First-order Discernibility}


\subsubsection{Intrinsic Discernibility}\label{sec:intrinsic_discernibility}

% FIXME ref intrinsic property

Two objects are \emph{Intrinsically Discernible} when one object has
an \emph{Intrinsic Property} that the other lacks.



\subsubsection{Strong Indiscernibility}\label{sec:strong_indiscernibility}

Two objects are \emph{Strongly Indiscernible} (or \emph{Utterly
  Indiscernible}) when they are not Discernible in any way, especially
not Weakly Discernible (\S\ref{sec:weak_discernibility}).



% ====================================================================
\section{Misc}
% ====================================================================

\emph{Note}: this section contains odds and ends, needs to be
organized and cross-referenced

\emph{Holism}, \emph{Reductionism}

\emph{Abstract}, \emph{Concrete}

\emph{Type}, \emph{Token}

\emph{Implicate}, \emph{Explicate}

\emph{A Priori}, \emph{A Posteriori}

\emph{Induction}, \emph{Deduction} (\emph{Entailment})

\emph{Analytic}, \emph{Synthetic}

\emph{Impredicative}, \emph{Predicative}

\emph{Intensional}, \emph{Extensional}

\emph{De Re}, \emph{De Dicto}

\emph{Real}, \emph{Apparent}

\emph{Constructivism}:
\begin{itemize}
    \item \emph{Intuitionism}, \emph{Intuitionistic Logic}
    \item \emph{Finitism}
    \item \emph{Constructive Recursion}
    \item \emph{Constructive Analysis}
    \item \emph{Constructive Set Theory}
    \item \emph{Topos Theory}
    \item \emph{Realizability}
\end{itemize}



% ====================================================================
\section{Survey Table}
% ====================================================================

\begin{tabularx}{\textwidth}{| X | X | X |}
    \hline
    Decidability            & Automata Theory (AT)  & Formal Language (FL) \\
    \hline
    Syntactic Consequence   & Proof Theory (PT)     & Formal System (FS): FL + Deductive System (DS) \\
    \hline
    Semantic Consequence    & Model Theory (MT)     & Logical System (LS): FS + Interpretation (I) \\
    \hline
    Homomorphism            & Abstract Algebra (AA) & Algebraic Structure (AS): LS + Signature ($\sigma$) \\
    \hline
    Homotopy                & Topology (T)          & \\
    \hline
    Natural Transformation  & Category Theory (CT)  & Category ($\mathbf{C}$): AS + Homset (Hom) \\
    \hline
    Predicativity           & Type Theory (TT)      & Type System (TS): $\mathbf{C}$ + Typing Judgement\\
    \hline
\end{tabularx}
