%%%%%%%%%%%%%%%%%%%%%%%%%%%%%%%%%%%%%%%%%%%%%%%%%%%%%%%%%%%%%%%%%%%%%%
%%%%%%%%%%%%%%%%%%%%%%%%%%%%%%%%%%%%%%%%%%%%%%%%%%%%%%%%%%%%%%%%%%%%%%
\part{Proof Theory}\label{sec:proof_theory}
%%%%%%%%%%%%%%%%%%%%%%%%%%%%%%%%%%%%%%%%%%%%%%%%%%%%%%%%%%%%%%%%%%%%%%
%%%%%%%%%%%%%%%%%%%%%%%%%%%%%%%%%%%%%%%%%%%%%%%%%%%%%%%%%%%%%%%%%%%%%%

\emph{Proof Theory} is the study of Syntactic Consequence
(\S\ref{sec:syntactic_consequence}) by means of representing Deductive
Arguments (\S\ref{sec:logical_argument}) as Formal Proofs
(\S\ref{sec:formal_proof}).



% ====================================================================
\section{Formal System}\label{sec:formal_system}
% ====================================================================

A \emph{Formal System} (also \emph{Logical Calculus} or \emph{System
  of Inference}) is a Formal Language (\S\ref{sec:formal_language})
combined with a Deductive Apparatus (\S\ref{sec:deductive_apparatus})
in order to Derive Theorems (\S\ref{sec:formal_proof}).

In the context of a Formal System, Expressions that belong to the
underlying Formal Language are called ``\emph{Well-formed Formulas}''
(\emph{Formulas}) or just \emph{Formulas} (\S\ref{sec:formula}). The
actual Syntax will be determined within the bounds of the particular
Logical System (\S\ref{sec:logical_system}) under consideration.



% --------------------------------------------------------------------
\subsection{Deductive Apparatus}\label{sec:deductive_apparatus}
% --------------------------------------------------------------------

A \emph{Deductive Apparatus} (or \emph{Deductive System}) is a Set of
zero or more Formulas called \emph{Axioms} (\S\ref{sec:axiom}) and one or
more Functions on Formulas called \emph{Inference Rules}
(\S\ref{sec:inference_rule}) that can be used to Derive Theorems
(\S\ref{sec:formal_proof}).

Deduction (\S\ref{sec:deductive_inference}) proper is the top-down,
Reductive (\S\ref{sec:abstract_rewrite}) process that starts with
general Axioms and Reduces to the specific Theorem that is being
proved by repeated application of Inference Rules.

A Deductive System gives a precise account of Syntactic Consequence
(\S\ref{sec:syntactic_consequence}) with respect to a Formal Language.
A Formal System is termed \emph{Effective} (i.e. Recursively
Definable) if the Set of Axioms and the Set of Inference Rules are
Decidable or Semi-decidable (\S\ref{sec:decidable_set}). The notion of
``Theorem'', however, is not in general Effective.



% --------------------------------------------------------------------
\subsection{Inference Rule}\label{sec:inference_rule}
% --------------------------------------------------------------------

An \emph{Inference Rule} (or \emph{Transformation Rule}) is a Function
which takes an input Set of Formulas called \emph{Premises} and returns an
output Set of Formulas called the \emph{Conclusion}. An alternative
formulation for Inference Rules is as a \emph{Deducibility Relation},
$\vdash$, that holds between zero or more Premises and one or more
Conclusions.

Taken individually, both Premises and Conclusions are Propositions
(\S\ref{sec:proposition}), that is, bearers of Truth-value
(\S\ref{sec:truth_value}). The Conclusion relies on the truth of the
Premises; if the Premises are left Unsatisfied
(\S\ref{sec:satisfaction}), then the Derivation is \emph{Hypothetical}
and the Premises \emph{Hypotheses}.

Inference Rules may be identified as redundant in two senses:
\begin{enumerate}
  \item An \emph{Admissible Rule} is one which does not change the Set
    of Theorems in a Formal System when it is added.
  \item A \emph{Derivable Rule} is a case of an Admissible Rule that
    has been Derived from existing rules.
\end{enumerate}



\subsubsection{Rule Schema}\label{sec:rule_schema}



% --------------------------------------------------------------------
\subsection{Axiom}\label{sec:axiom}
% --------------------------------------------------------------------

\emph{Axioms} are special cases of Inference Rules which have no
Premises, only a universally Valid (\S\ref{sec:validity}) Conclusion;
that is, a \emph{Logical Assertion} (\S\ref{sec:assertion}). Axioms
may be further differentiated from Rules by saying that Rules are
statements \emph{about} the System, and Axioms are statements
\emph{in} the System.

Axioms may be divided into two kinds: \emph{Logical Axioms} of a
Tautological sort, and \emph{Non-logical Axioms}, hereafter referred
to as \emph{Postulates}, that play the role of assumptions: defining
properties of the domain of the Theory (\S\ref{sec:formal_theory}) in
question.

Systems with a large number of Axioms and few other Inference Rules
are called \emph{Axiomatic Systems} (\S\ref{sec:axiomatic_system}).



\subsubsection{Logical Axiom}\label{sec:logical_axiom}

\subsubsection{Non-logical Axiom}\label{sec:nonlogical_axiom}

\emph{Non-logical Axiom} (or \emph{Postulate})



\subsubsection{Axiom Schema}\label{sec:axiom_schema}

An \emph{Axiom Schema} is a template for Axioms in which one or more
Schematic Variables (\S\ref{sec:metalanguage}) appear, standing for a
Subformula in the Object Language of the System. For a Language with
infinitely many Formulas, an Axiom Schema describes a countably infinite
number of Axioms. A System without Schema is termed \emph{Finitely
  Axiomatized}.



% --------------------------------------------------------------------
\subsection{Double-negation Translation}
\label{sec:double_negation_translation}
% --------------------------------------------------------------------

(or \emph{Negative Translation})



\subsubsection{G\"odel-Gentzen Translation}\label{sec:godel_gentzen}

First-order Logic

Intuitionistic Logic, Linear Logic



% ====================================================================
\section{Formal Proof} \label{sec:formal_proof}
% ====================================================================

A \emph{Formal Proof} (or \emph{Derivation}) is a Finite Sequence of
Formulas of a Formal System that are either Axioms (\S\ref{sec:axiom})
or follow from the preceding Formulas by Inference Rules
(\S\ref{sec:inference_rule}). A Formal Proof is said to provide
\emph{Justification} for the Theorem (\S\ref{sec:theorem}) thus
Derived.

A Formal Proof is \emph{Valid} (\S\ref{sec:validity})

A Formal Proof is \emph{Sound} (\S\ref{sec:soundness}) if it is Valid
and all the Premises are True.

For a Set of Formulas, $\Gamma$, in a Formal System, $\mathcal{S}$,
there is a Syntactic Consequence, $A$, if there is a Formal Proof of
$A$ from the Set $\Gamma$:
\[
  \Gamma \vdash_{\mathcal{S}} A
\]

Undischarged Assumptions correspond to Free Variables in a Typing
Derivation (\S\ref{sec:typing_derivation}), Discharged Assumptions to
Bound Variables


% --------------------------------------------------------------------
\subsection{Theorem}\label{sec:theorem}
% --------------------------------------------------------------------

The concluding Formula in the sequence is a \emph{Theorem}. A Theorem
is a Syntactic Consequence (\S\ref{sec:syntactic_consequence}) of the
preceding Formulas in the Proof. Therefore, a Theorem is a Judgement
(\S\ref{sec:judgement}), i.e. Formulas stand in relation to Theorems
as Propositions to Judgements. \cite{martinlof84}

A Theorem used in the course of Deriving a further Theorem is called a
\emph{Lemma}.



\subsubsection{Deduction Theorem}\label{sec:deduction_theorem}

If a Formula $B$ is Deducible from a Set of Assumptions $\Delta \cup
\{ A \}$ where $A$ is a Closed Formula, then the Implication $A
\rightarrow B$ is Deducible from $\Delta$:
\[
  \Delta \cup \{A\} \vdash B \Rightarrow \Delta \vdash A \rightarrow B
\]



% --------------------------------------------------------------------
\subsection{Analytic Proof}\label{sec:analytic_proof}
% --------------------------------------------------------------------

A \emph{Theoretic Analytic Proof} begins with an assumption and
proceeds to an accepted truth (an Axiom, or Contradiction as in
\S\ref{sec:tableau_calculus} Analytic Tableau).



% --------------------------------------------------------------------
\subsection{Synthetic Proof}\label{sec:synthetic_proof}
% --------------------------------------------------------------------

A \emph{Synthetic Proof} is the reverse of this process; beginning
with known truths and reasoning up to the desired Proof. A
\emph{Problematic Analytic Proof} is constructed from given conditions
that are to be satisfied.



% --------------------------------------------------------------------
\subsection{Constructive Proof}\label{sec:constructive_proof}
% --------------------------------------------------------------------

\subsubsection{Realizability}\label{sec:realizability}

Formalization of the BHK Interpretation
(\S\ref{sec:brouwer_heyting_kolmogorov}) of Intuitionistic Logic
(\S\ref{sec:intuitionistic_logic}).

Introduced as a way of forming Recursive Models (\S\ref{sec:model}) of
Intuitionistic Theories

Realizability Model (Type Theory \S\ref{sec:realizability_model})

Realizability Topos (\S\ref{sec:realizability_topos})

\HandRight\; Cf. Focalization (\S\ref{sec:focalization})



\paragraph{Modified Realizability}\label{sec:modified_realizability}



\subsubsection{Proof Mining}\label{sec:proof_mining}



% --------------------------------------------------------------------
\subsection{Non-constructive Proof}\label{sec:nonconstructive_proof}
% --------------------------------------------------------------------

\emph{Existence Theorem}

Probabilistic Method (\S\ref{sec:probabilistic_method})



% --------------------------------------------------------------------
\subsection{Proof Procedure}\label{sec:proof_procedure}
% --------------------------------------------------------------------

\subsubsection{Direct Proof}\label{sec:direct_proof}

\emph{Direct Implication}



\paragraph{Proof by Induction}\label{sec:induction_proof}
\hfill \\

Mathematical Induction (\S\ref{sec:mathematical_induction})



\paragraph{Proof by Exhaustion}\label{sec:exhaustion_proof}



\subsubsection{Indirect Proof}\label{sec:indirect_proof}

\paragraph{Proof by Contradiction}\label{sec:contradiction_proof}

\subparagraph{Proof by Infinite Descent}\label{sec:infinite_descent}

\paragraph{Proof by Contrapositive}\label{sec:contrapositive_proof}



\subsubsection{Reverse Mathematics}\label{sec:reverse_mathematics}



% --------------------------------------------------------------------
\subsection{Proof Equality}\label{sec:proof_equality}
\cite{harper12}
% --------------------------------------------------------------------

\begin{enumerate}
  \item Definitional Equality (Analytic Judgement
    \S\ref{sec:analytic_judgement}): $\Gamma \vdash a \equiv b : A$
    ``Equality of Sense'' (\S\ref{sec:sense}), corresponding to
    symbolic execution ($\beta$-reduction \S\ref{sec:beta_reduction})
  \item Denotational Equality (Synthetic Judgement
    \S\ref{sec:synthetic_judgement}): $\Gamma \vdash a = b : A$
    ``Equality of Reference'' (\S\ref{sec:reference}), reached by
    Definitional Equality combined with Mathematical Induction
  \item Weak (Homotopy \S\ref{sec:homotopy}) Equivalence (Synthetic
    Judgement): $\Gamma \vdash \alpha :: a \cong b : A$ ``Evidence for
    Equivalence'', such as Isomorphism (Bijection) of Sets, e.g. $2
    \cong 2 : Set$
\end{enumerate}



% --------------------------------------------------------------------
\subsection{Proof Simplification}\label{sec:proof_simplification}
% --------------------------------------------------------------------

Rewrite Rules (\S\ref{sec:abstract_rewrite})

Redex (LHS), Reduct (RHS)

Normal Form (\S\ref{sec:normalization})

Subformula Principle



% ====================================================================
\section{Formal Theory}\label{sec:formal_theory}
% ====================================================================

A \emph{Formal Theory} is a Set of Sentences (\S\ref{sec:sentence}) of
a Formal Language (\S\ref{sec:formal_language}).

For a Formal Language, $L$, a Formal Theory, $\mathcal{T}$, is a
Subset of a Conceptual Class of \emph{Elementary Expressions}
(\emph{Elementary Statements} or \emph{Well-formed Formulas}
\S\ref{sec:formula}), $\mathcal{E}$, consisting of \emph{Elementary
  Theorems} considered to be True. In a \emph{Deductive Theory}
(\S\ref{sec:deductive_theory}), $\mathcal{T}$ is an \emph{Inductive
  Class} such that some of the Elementary Theorems are taken to be
\emph{Axioms} and any Expressions Deducible by those Axioms are also
Elementary Theorems in $\mathcal{T}$.

Model Theory: a Structure (\S\ref{sec:structure}) is a Realization of
Types, Operations, and Relations in some Signature with the Empty
Theory (\S\ref{sec:empty_theory})

A Theory is \emph{Satisfiable} (\S\ref{sec:satisfiability}) if there
exists a Model (\S\ref{sec:structure}) for that Theory.

A Theory is \emph{Categorical} (\S\ref{sec:categoricity}) if it has
just one Model up to Isomorphism.

A Theory $\mathcal{T}$ is \emph{Sound} (\S\ref{sec:soundness}) when
every Theorem in $\mathcal{T}$ is Valid (\S\ref{sec:validity}). The
Soundness Property for Propositional Calculus
(\S\ref{sec:propositional_calculus}) can be expressed in terms of the
Syntactic (\S\ref{sec:syntactic_consequence}) and Semantic
(\S\ref{sec:semantic_consequence}) Consequence Relations:
\[
  \Gamma \vdash \varphi \Rightarrow \Gamma \vDash \varphi
\]

A Theory is \emph{Effective} when it is possible to Effectively
(\S\ref{sec:effective_method}) determine whether a Deduction is Valid
or not.

Elementary Theory (\S\ref{sec:elementary_theory})

Algebraic Theory (\S\ref{sec:algebraic_theory})



% --------------------------------------------------------------------
\subsection{Deductive Theory}\label{sec:deductive_theory}
% --------------------------------------------------------------------

In the context of a Deductive System, $\mathcal{S}$, a \emph{Deductive
  Theory} (or \emph{Inductive Class}), $T$, is a Formal Theory that
consists of all the Theorems that are Derivable in $\mathcal{S}$,
including any Axioms.



% --------------------------------------------------------------------
\subsection{Empty Theory}\label{sec:empty_theory}
% --------------------------------------------------------------------

No Axioms

Model Theory: a Structure (\S\ref{sec:structure}) is a Realization of
Types, Operations, and Relations in some Signature with the Empty
Theory



% --------------------------------------------------------------------
\subsection{Subtheory \& Extension}\label{sec:subtheory}
% --------------------------------------------------------------------

A Theory $S$ is a \emph{Subtheory} of a Theory $T$ if $S \subseteq T$.
$T$ is then a \emph{Supertheory} (or \emph{Extension}) of $S$.



% --------------------------------------------------------------------
\subsection{Conservative Extension}\label{sec:conservative_extension}
% --------------------------------------------------------------------

Given two Theories $T_1$ and $T_2$ with underlying Formal Languages
$L_1$ and $L_2$ respectively, $T_2$ is a \emph{Conservative Extension}
of $T_1$ if:
\begin{enumerate}
  \item $L_2$ extends $L_1$
  \item $T_1 \subseteq T_2$
  \item any Theorem of $T_2$ in $L_1$ is already a Theorem of $T_1$
\end{enumerate}
An Extension which is not Conservative may be called a \emph{Proper
  Extension}.



\subsubsection{Conservativity Theorem}\label{sec:conservativity_theorem}



% --------------------------------------------------------------------
\subsection{Consistency}\label{sec:consistency}
% --------------------------------------------------------------------

A Theory is \emph{Absolutely Consistent} (also \emph{Coherent} or
\emph{Non-trivial}) if not every Formula in the underlying Language is
a Theorem.

A Non-trivial Theory is \emph{Consistent} if it satisfies the
\emph{Principle of Explosion}, that is, Proof of a contradiction
Implies Proof of all other Formulas. Therefore, a Consistent Theory must
be free of contradictions.

By G\"odels Completeness Theorem (\S\ref{sec:completeness}), a
First-order Theory is Consistent if and only if it is Satisfiable.
Higher-order Logics may allow Syntactically Consistent Theories that
are not Satisfiable.

The Conservative Extension (\S\ref{sec:conservative_extension}) of a
Consistent Theory is Consistent.

See also \emph{Paraconsistency} (\S\ref{sec:paraconsistent_inference}).



\subsubsection{$\omega$-consistent}\label{sec:omega_consistent}

\subsubsection{Consistency Strength}\label{sec:consistency_strength}



% --------------------------------------------------------------------
\subsection{Completeness}\label{sec:completeness}
% --------------------------------------------------------------------

A Consistent (\S\ref{sec:consistency}) Theory is said to be
\emph{Complete} if for every Sentence in the Language of that Theory,
the Theory contains either that Sentence or its Negation. Such a Set
of Sentences is a \emph{Maximal Consistent Set}.

For a Theory $\mathcal{T}$ in Propositional
(\S\ref{sec:propositional_logic}) or First-order Logic
(\S\ref{sec:firstorder_logic}), Completeness may be defined as follows
with the Semantic (\S\ref{sec:semantic_consequence}) and Syntactic
(\S\ref{sec:syntactic_consequence}) Consequence Relations on a Set of
Sentences $\Gamma$ and Formulas $\varphi \in \mathcal{T}$:
\[
  \Gamma \vDash \varphi \Rightarrow \Gamma \vdash \varphi
\]

In Propositional and First-order Logic, Completeness and Soundness
(\S\ref{sec:soundness}) are equivalent Properties.



\subsubsection{Incompleteness Theorem}\label{sec:incompleteness_theorem}

\paragraph{Rosser's Trick}\label{sec:rossers_trick}



% --------------------------------------------------------------------
\subsection{Quantifier Elimination}\label{sec:quantifier_elimination}
% --------------------------------------------------------------------

Within a Theory $T$, if every First-order Formula $\varphi(x_1,
\ldots, x_n)$ with Quantifiers is equivalent to a First-order Formula
$\psi(x_1, \ldots, x_n)$ without Quantifiers, $T$ is said to have the
property of \emph{Quantifier Elimination}. A Theory without Quantifier
Elimination may be made to have it by adding Symbols to its Signature.



% --------------------------------------------------------------------
\subsection{Synthetic Theory}\label{sec:synthetic_theory}
% --------------------------------------------------------------------

% --------------------------------------------------------------------
\subsection{Craig's Interpolation Theorem}
\label{sec:interpolation_theorem}
% --------------------------------------------------------------------

Cut Elimination (\S\ref{sec:cut_elimination})

Resolution (\S\ref{sec:resolution})



% ====================================================================
\section{Proof Calculus}\label{sec:proof_calculus}
% ====================================================================

A \emph{Proof Calculus} is a family of Formal Systems
(\S\ref{sec:formal_system}) that use a common style of Formal
Inference for Inference Rules.



% --------------------------------------------------------------------
\subsection{Axiomatic System} \label{sec:axiomatic_system}
% --------------------------------------------------------------------

\emph{Axiomatic System}

In contrast to Sequent Calculus (\S\ref{sec:sequent_calculus}) or
Natural Deduction (\S\ref{sec:natural_deduction}), Axiomatic Systems
are more useful in the context of Model Theory (Part
\ref{part:model_theory}).



\subsubsection{Categorical}\label{sec:categorical}

An Axiomatic System is \emph{Categorical} (or \emph{Categorically
  Axiomatized}) if for any two Models (\S\ref{sec:model}) of the
System, are Isomorphic to each other.



\subsubsection{Hilbert Systems}\label{sec:hilbert_system}

\emph{Hilbert Systems} are characterized by having a large number of
Axiom Schema and few Inference Rules-- just Modus Ponens for
Propositional Logics and Universal Generalization for Predicate
Logic. In a Hilbert System, Judgements and Formulas are not
differentiated. A Theorem in a Hilbert System is the Concluding
Judgement in a Derivation.

A Hilbert System is differentiated from Systems of Natural Deduction
by not having any Rules that change the Context of a Formula.



% --------------------------------------------------------------------
\subsection{Structural Proof Theory}\label{sec:structural_proof}
% --------------------------------------------------------------------

\emph{Structural Proof Theory} studies Proof Calculi that support
Analytic Proof (\S\ref{sec:analytic_proof}); that is Proofs that are
\emph{Cut-free} (they do not use the \emph{Cut Rule}) or in Normal
Form (\S\ref{sec:normal_form}).



\subsubsection{Natural Deduction}\label{sec:natural_deduction}
\cite{prawitz65}

\emph{Natural Deduction} provides a generalization of Formal Proofs.
Contrasted with Axiomatic Systems (\S\ref{sec:axiomatic_system}),
Systems of Natural Deduction include many Inference Rules but few or
no Axioms. A Natural Deduction System allows Judgements with multiple
Antecedents and a single Succedent:
\[
  A_1,\ldots,A_n \vdash B
\]
Inference Rules in Natural Deduction have the general notation
\[
  {
    \frac{J_1 \quad J_2 \quad \cdots \quad J_n}
    {J}
  } name
\]
where the Rule with name $name$ has Premises of zero or more
Judgements $J_i$ and the Judgement $J$ is the Conclusion.

Inference Rules that introduce a Logical Connective in the Conclusion
are called \emph{Introduction Rules}. Example:
\[
  {
    \frac{A\;\mathrm{true} \quad B\;\mathrm{true}}
    {(A \wedge B)\;\mathrm{true}}
  } \wedge_I
\]
where $A$ and $B$ are Propositions.

Conversely, Inference Rules that remove Logical Connectives are
\emph{Elimination Rules}.
\[
  {
    \frac{A \wedge B\;\mathrm{true}}
    {A\;\mathrm{true}}
  } \wedge_E
\]

Subformula Principle

\emph{Hypothetical Derivations} (reasoning from \emph{Assumptions})
are required for Implication Introduction or Disjunction
Elimination. The general form of a Hypothetical Derivation with
Antecedents $D_i$ and Succedent $J$:
\[
  D_1 \quad D_2 \cdots D_n
\]\[
  \vdots
\]\[
  J
\]
Introduction Rules for Implication:
\[
  {
    \frac{}
    {A\;\mathrm{true}}
  } u
\]\[
  \vdots
\]\[
  {
    \frac{B\;\mathrm{true}}
    {A \rightarrow B\;\mathrm{true}}
  } \rightarrow_{I^u}
\]
The Premise $u$ here is considered \emph{discharged} by the Rule
$I^u$; that is the scope of $u$ does not extend past $I^u$.
Elimination Rule for Implication (Modus Ponens):
\[
  {
    \frac{A \rightarrow B\;\mathrm{true} \quad A\;\mathrm{true}}
    {B\;\mathrm{true}}
  } \rightarrow_{E}
\]
Disjunctive Elimination:
\[
  \frac{
  A \vee B\;\mathrm{true} \quad
  \begin{matrix}
    {
      \frac{}
      {A\;\mathrm{true}}
    }u \\
    \vdots \\
    C\;\mathrm{true}
  \end{matrix}
  \quad
  \begin{matrix}
    {
      \frac{}
      {B\;\mathrm{true}}
    }w \\
    \vdots \\
    C\;\mathrm{true}
  \end{matrix}
  }{ C\;\mathrm{true}}\wedge_{E^{u,w}}
\]
A Theory is \emph{Locally Consistent} (or \emph{Locally Reducible}) if
an Introduction of a Connective followed by its Elimination can be
equivalently Derived without these steps. The dual to Local
Consistency is \emph{Local Completeness} which states that Elimination
rules can decompose a Connective into the forms of its Introduction
Rule. These correspond to $\beta$-reduction and $\eta$-conversion in
$\lambda$-Calculus (\S\ref{sec:untyped_lambda}) where Propositions
are \emph{Types} and Proofs are \emph{Programs} (see
\emph{Curry-Howard Correspondence} \S\ref{sec:curry_howard}).

If an entire Derivation has only Eliminations followed by
Introductions, it is said to be in \emph{Normal Form}
(\S\ref{sec:normal_form}).

In a Formal Proof, the Judgements representing Antecedents are
presented as Rules with no Premises, named by a \emph{Proof Variable}
(from a countable set $V$ of variables):
\[
  \frac{}{J_1}u_1 \; \frac{}{J_2}u_2 \; \cdots \frac{}{J_n}u_n
\]\[
  \vdots
\]\[
  J
\]
where $u_i \in V$. Written in Sequent Notation:
\[
  u_1:J_1, u_2:J_2, \ldots, u_n:J_n \vdash J
\]
This convention is sometimes called \emph{Localized Hypotheses}. In
general, $\pi : A$ may be read ``$\pi$ is a proof of $A$''.



\subsubsection{Sequent Calculus}\label{sec:sequent_calculus}

In \emph{Sequent Calculus} (or \emph{Gentzen System}) a Formal Proof
is a Sequence of Sequents (\S\ref{sec:sequent}) where each
successive Sequent is Derivable from prior Sequents by Inference
Rules.

Theorems are Sequents of the form $\vdash B$ which are the Conclusion
of a Valid Proof.



\paragraph{Sequent}\label{sec:sequent}
\hfill \\

A \emph{Sequent} is a specific kind of Judgement
(\S\ref{sec:judgement}) of the form
\[
  \Gamma \vdash \Sigma
\]
where the \emph{Antecedent} (\S\ref{sec:antecedent}), $\Gamma$, is a
Conjunctive sequence of Formulas, and the \emph{Succedent}
(\S\ref{sec:succedent}), $\Sigma$, is a Disjunctive sequence of
Formulas. Together, Antecedents and Succedents are \emph{Cedents}.

In a general Sequent Calculus there may be any number of Formulas on
either side
\[
  A_1, \ldots, A_n \vdash B_1, \ldots, B_k
\]
is equivalent to
\[
  \vdash(A_1 \wedge \cdots \wedge A_n) \rightarrow
  (B_1 \vee \cdots \vee B_k)
\]
and the dual nature of Judgements and negation can be expressed by the
dual forms
\[
  \vdash \neg A_1 \vee \cdots \vee \neg A_n \vee B_1 \vee \cdots
  \vee B_k
\]
and
\[
  \vdash \neg(A_1 \wedge \cdots \wedge A_n \wedge \neg B_1 \wedge
  \cdots \wedge \neg B_k)
\]

If Sequents are defined as Sets or Multisets (\S\ref{sec:multiset})
instead of Sequences (that is, Unordered Sets), then the Permutation
Rule (\S\ref{sec:permutation_rule}) is obsolete, likewise the
Contraction Rule (\S\ref{sec:contraction_rule}) would be obsolete for
Sets instead of Sequences.



\paragraph{Antecedent}\label{sec:antecedent}

\paragraph{Succedent}\label{sec:succedent}

\paragraph{Context}\label{sec:sequent_context} \hfill \\

A Sequence of Cedents may be called a \emph{Context}, but \emph{the}
Context for a \emph{specific} Judgement is usually meant to be the
Antecedent.



\paragraph{Assertion}\label{sec:assertion} \hfill \\

A Sequent with no Antecedent ($\vdash \Sigma$) is an \emph{Assertion}
(or \emph{Logical Assertion}) and the Succedent is a \emph{Tautology}
(\S\ref{sec:tautology}).

One-sided Sequent



\paragraph{Contradiction}\label{sec:contradiction} \hfill \\

A Sequent with no Succedent ($\Gamma \vdash$) is a
\emph{Contradiction} meaning it proves falsity which is
Inconsistent.



\paragraph{Structural Rule}\label{sec:structural_rule} \hfill \\

The Extra-logical Operators, $\vdash$, and $,$ (comma), are called
\emph{Structural Operators} and Rules which change only Structural
Operators are \emph{Structural Rules} (as opposed to \emph{Logical
  Rules}). Logics lacking Structural Rules are called
\emph{Substructural Logics} (\S\ref{sec:substructural_logic}).



\subparagraph{Weakening Rule}\label{sec:weakening_rule} \hfill \\

\emph{Weakening} refers to a Rule that introduces arbitrary elements
to a Sequent.



\subparagraph{Contraction Rule}\label{sec:contraction_rule} \hfill \\

\emph{Contraction} refers to a Rule that removes
multiple occurences of some Element.



\subparagraph{Exchange Rule}\label{sec:exchange_rule}



\paragraph{Permutation Rule}\label{sec:permutation_rule} \hfill \\

\emph{Permutation} refers to the re-ordering of elements.



\paragraph{Cut Elimination}\label{sec:cut_elimination} \hfill \\

The Rule for \emph{Cut} is as follows:
\[
  \frac{
    \Gamma \vdash \Delta, A \quad A, \Sigma \vdash \Pi
  }{
    \Gamma, \Sigma \vdash \Delta, \Pi
  }(Cut)
\]
It states that when a Formula $A$ that can be Concluded can also be
used as a Premise, it can be \emph{cut} out and the Derivations joined
together. That is, wherever the Lemma $A$ occurs, it can be
substituted for the Proof of $A$. This means that the Cut Rule is an
Admissible Rule (\S\ref{sec:inference_rule}).

The \emph{Cut-elimination Theorem} states that any Judgement with a
Proof in Sequent Calculus that uses the Cut Rule may be expressed as a
\emph{Cut-free} Proof without using the Cut Rule. Usually,
demonstrating the existence of the Cut-elmination Theorem implies that
the System is Consistent since that would rule-out the possibility of
Proof of Contradiction.

Subformula Property %FIXME

Strong Normalization Property

Cut Elimination corresponds to Communication
(\S\ref{sec:communication}) in Process Calculus
(\S\ref{sec:process_calculus})

Craig's Interpolation Theorem (\S\ref{sec:interpolation_theorem})



\paragraph{$\mathbf{LK}$}\label{sec:lk} \hfill \\

Formalization of Classical Logic (\S\ref{sec:classical_logic})
(sound and complete in First-Order) with Sequents having zero or more
RHS Formulas. Allowing multiple RHS Formulas with a \emph{Right
  Contraction Rule} is equivalent to the admissibility of the
\emph{Law of the Excluded Middle}.



\paragraph{$\mathbf{LJ}$}\label{sec:lj} \hfill \\

Formalization of Intuitionistic Logic
(\S\ref{sec:intuitionistic_logic}) obtained from modifying
$\mathbf{LK}$ to allow at most one RHS Formula in a Sequent.

The Cut Rule for $\mathbf{LJ}$:
\[
  \frac{
    \Gamma \vdash A \quad \Pi, A \vdash B
  }{
    \Gamma, \Pi \vdash B
  }(Cut)
\]



\paragraph{Substructural Rule Sets}\label{sec:substructural_rule}
\hfill \\

A Substructural Logic (\S\ref{sec:substructural_logic}) lacking the
usual Structural Rules is usually weaker than $\mathbf{LK}$.

In Relevance Logic (\S\ref{sec:relevance_logic}), Weakening Rules
are not included on the grounds that introduced Formulas are not
\emph{Relevant}.

In Linear Logic (\S\ref{sec:linear_logic}), duplicate Formulas are
treated differently so Contraction and Weakening Rules are is absent
or controlled.



\subsubsection{Calculus of Structures}\label{sec:calculus_of_structures}

\emph{Calculus of Structures} is a Proof Calculus with \emph{Deep
  Inference} for studying Non-commutative Logic
(\S\ref{sec:noncommutative_logic}). Deep Inference is a
generalization of Structure to handle greater Structural complexity
\cite{schutte77}.



% --------------------------------------------------------------------
\subsection{Tableau Calculus}\label{sec:tableau_calculus}
% --------------------------------------------------------------------

\emph{Tableau Calculus} (or \emph{Method of Analytic Tableau}) is
commonly used as a Proof procedure for Modal Logics
(\S\ref{sec:modal_logic}). An Analytic Tableau is a tree with a
Formula at the root and a Subformula at each node. A specific Tableau
Calculus is a finite collection of Rules for breaking down Logical
Connectives into constituent parts. Rules can be expressed as Sets,
Multisets, Lists, or Trees of Formulas. If Sets of Formulas are used
at each node (\emph{Set-labeled Tableau}), they are taken in
Conjunction.

A \emph{Strongly Complete Tableau} is one in which every Formula in
every branch has been expanded.



\subsubsection{Refutation Tableau}\label{sec:refutation_tableau}

A \emph{Refutation Tableau} attempts to show that a negation of the
root Formula cannot be satisfied, thereby proving Logical Truth of the
Formula. Rules for handling Logical Connectives may produce a branch
in the tree and if a branch leads to a Contradiction, the branch is
closed and if all branches are closed the Proof is complete and the
root Formula is proved. Nodes on a single branch are considered in
Conjunction, Nodes on separate branches are considered Disjunctively.



\subsubsection{Destructive \& Non-destructive Tableau}
\label{sec:destructive_tableau}

\emph{Non-destructive Tableau Calculi} use Rules that only allow
addition of nodes, while \emph{Destructive Tableau Calculi} use Rules
that allow modification of existing nodes.



\subsubsection{Proof Confluence} \label{sec:proof_confluence}

\emph{Proof Confluence} is the property of a Tableau Calculus that a
closed Tableau (for an un-satisfiable set of Propositions) can always
be generated from an arbitrary partially constructed Tableau
regardless of which Rules are chosen at each application (if a choice
between Rules is available).



\subsubsection{Unification} \label{sec:tableau_unification}

A method of dealing with non-determinism of Rules involving Universal
Quantification (in First-order Tableau) is called \emph{Unification}.
This allows Free Variables to be substituted in the Rule for
Eliminating Universal Quantifiers, which can later be Unified by
choosing an appropriate Term to close the branch.



\subsubsection{Clause Tabeau} \label{sec:clause_tableau}

\emph{Clause Tableau} (Tableau Method applied to sets of Clauses) may
be used for increased efficiency.



\paragraph{Connection Tableau} \label{sec:connection_tableau}
\hfill \\

\emph{Connection Tableau} restrict expansion of Clause Tableau
branches (not the bare root) to contain only Literals that unify with
a Literal already on the branch (\emph{Weak Connectedness}) or a
Literal in the current leaf (\emph{Strong Connectedness}).



% --------------------------------------------------------------------
\subsection{Geometry of Interaction}
\label{sec:geometry_of_interaction}
% --------------------------------------------------------------------

Girard

PCF (\S\ref{sec:pcf})

Game Semantics (\S\ref{sec:game_semantics}) for Linear Logic
(\S\ref{sec:linear_logic})



\subsubsection{Proof Net} \label{sec:proof_net}

\cite{llwiki16}

Proof Nets can be seen as a Quotient of One-sided Sequent Calculus
Proofs (\S\ref{sec:sequent_calculus}) under Commutation of Rules



\subsubsection{Geometry of Synthesis} \label{sec:geometry_of_synthesis}

Bounded version of Geometry of Interaction



% --------------------------------------------------------------------
\subsection{Syllogistic Calculus} \label{sec:syllogistic_calculus}
% --------------------------------------------------------------------



% ====================================================================
\section{Automated Theorem Proving}\label{sec:atp}
% ====================================================================

% --------------------------------------------------------------------
\subsection{Unit Propagation}\label{sec:unit_propagation}
% --------------------------------------------------------------------

(or \emph{Boolean Constraint Propagation})

\emph{One-literal Rule} (OLR)

\emph{Unit Clause}



% --------------------------------------------------------------------
\subsection{Resolution}\label{sec:resolution}
% --------------------------------------------------------------------



% ====================================================================
\section{Ludics} \label{sec:ludics}
% ====================================================================

\cite{girard01}

Interactive Computation (??? Communication \S\ref{sec:communication}),
Game Semantics (\S\ref{sec:game_semantics})

Propositions: \emph{Localization} (\S\ref{sec:localization})

Two Interpretations of Propositions:

\begin{enumerate}
  \item Gentzen System (Sequent Calculus
    \S\ref{sec:sequent_calculus}): Meaning of a Proposition arises
    from Introduction and Elimination Rules; \emph{Focalization}
    \S\ref{sec:focalization}

  \item Brouwer-Heyting-Kolmogorov Interpretation
    (\S\ref{sec:brouwer_heyting_kolmogorov}): Fixed Computation System
    (\S\ref{sec:computation_model}); Meaning of a Proposition arises
    from its Realizability Interpretation (Constructive Proof
    \S\ref{sec:realizability}); visible behavior of Propositions
    (Extensional?)
\end{enumerate}

For Second-order Affine Linear Logic (\S\ref{sec:affine_logic}), given
a Computational System with Non-termination
(\S\ref{sec:rewrite_termination}) and Error Stops (???) as Effects
(\S\ref{sec:algebraic_effect}), Realizability and Focalization give
the same Meaning to Types.



% --------------------------------------------------------------------
\subsection{Localization}\label{sec:localization}
% --------------------------------------------------------------------

\emph{Loci} (or \emph{Localizations}) over a \emph{Base}

Abstract Syntax (\S\ref{sec:abstract_syntax}) for Memory Pointers in
Computer Science



% --------------------------------------------------------------------
\subsection{Focalization}\label{sec:focalization}
% --------------------------------------------------------------------

\emph{Focalization} (or \emph{Focusing})

\emph{Positive Propositions}: Meaning arises from Introduction Rules

\emph{Negative Propositions}: Meaning arises from Elimination Rules

Focused Calculi: possible to define Positive Connectives by
Introduction Rules only and Elimination Rules are forced, and the Dual
case for Negative Connectives and Elimination Rules



% --------------------------------------------------------------------
\subsection{Compound Connective}\label{sec:compound_connective}
% --------------------------------------------------------------------



% ====================================================================
\section{Metatheory} \label{sec:metatheory}
% ====================================================================

A \emph{Metatheory} is a Theory whose subject matter is some other
Theory.



% --------------------------------------------------------------------
\subsection{Syntactic Consequence}\label{sec:syntactic_consequence}
% --------------------------------------------------------------------

For a Formal System, $\mathcal{S}$, a Formula, $A$, is a \emph{Syntactic
  Consequence} of a possibly empty Set of Formulas, $\Gamma$, if there
is a Formal Proof (\S\ref{sec:formal_proof}) of $A$ from $\Gamma$ in
$\mathcal{S}$:
\[
  \Gamma \vdash_{\mathcal{S}} A
\]
where $\vdash_{\mathcal{S}}$ is the \emph{Syntactic Consequence
  Relation} on $\mathcal{S}$. This expresses that $A$ is
\emph{Provable} from $\Gamma$.



% --------------------------------------------------------------------
\subsection{Metatheorem}\label{sec:metatheorem}
% --------------------------------------------------------------------

A \emph{Metatheorem} is a Theorem proved within a Metatheory.



% --------------------------------------------------------------------
\subsection{Judgement}\label{sec:judgement}
% --------------------------------------------------------------------

A \emph{Judgement} is an inductively definable assertion in the
Metatheory of a Logical System. That is, one that includes
Extra-logical Symbols, namely that of Logical Consequence, $\vdash$,
and commas used in Sequents (\S\ref{sec:sequent}). In this way, Axioms
are Judgements, and a Formal Proof expresses a Judgement with the
Premises being a sequence of Judgements and the Conclusion also a
Judgement.



\subsubsection{Synthetic Judgement}\label{sec:synthetic_judgement}

\emph{Synthetic Judgement}: Proof Search



\subsubsection{Analytic Judgement}\label{sec:analytic_judgement}

\emph{Analytic Judgement}: Proof Checking
