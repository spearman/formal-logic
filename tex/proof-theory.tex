%%%%%%%%%%%%%%%%%%%%%%%%%%%%%%%%%%%%%%%%%%%%%%%%%%%%%%%%%%%%%%%%%%%%%%
%%%%%%%%%%%%%%%%%%%%%%%%%%%%%%%%%%%%%%%%%%%%%%%%%%%%%%%%%%%%%%%%%%%%%%
\part{Proof Theory}\label{sec:proof_theory}
%%%%%%%%%%%%%%%%%%%%%%%%%%%%%%%%%%%%%%%%%%%%%%%%%%%%%%%%%%%%%%%%%%%%%%
%%%%%%%%%%%%%%%%%%%%%%%%%%%%%%%%%%%%%%%%%%%%%%%%%%%%%%%%%%%%%%%%%%%%%%

\emph{Proof Theory} is the study of \emph{Syntactic Consequence}
(\S\ref{sec:syntactic_consequence}) by means of representing
\emph{Deductive Arguments} (\S\ref{sec:logical_argument}) as
\emph{Formal Proofs} (\S\ref{sec:formal_proof}).



% ====================================================================
\section{Formal System}\label{sec:formal_system}
% ====================================================================

A \emph{Formal System} (also \emph{Logical Calculus} or \emph{System
  of Inference}) is a Formal Language (\S\ref{sec:formal_language})
combined with a \emph{Deductive Apparatus}
(\S\ref{sec:deductive_apparatus}) in order to Derive \emph{Theorems}.
Here, Expressions (strings) of Symbols are usually referred to as
\emph{Forumulas}, and Expressions that belong to the underlying
Language are called ``\emph{Well-formed Formulas}'' (\emph{WFF}).



% ====================================================================
\section{Deductive Apparatus} \label{sec:deductive_apparatus}
% ====================================================================

A \emph{Deductive Apparatus} (or \emph{Deductive System}) is a Set of
zero or more WFFs called \emph{Axioms} (\S\ref{sec:axiom}) and one or
more Functions on WFFs called \emph{Inference Rules}
(\S\ref{sec:inference_rule}) that can be used to Derive
\emph{Theorems} (\S\ref{sec:formal_proof}).

Deduction (\S\ref{sec:deductive_inference}) proper is the top-down,
\emph{Reductive} (\S\ref{sec:abstract_rewrite}) process that starts
with general Axioms and Reduces to the specific Theorem that is being
proved.

A Deductive System gives a precise account of \emph{Syntactic
  Consequence} (\S\ref{sec:syntactic_consequence}) with respect to a
Formal Language. A Formal System is termed \emph{Effective} (i.e.
Recursively Definable) if the Set of Axioms and the Set of Inference
Rules are Decidable or Semi-decidable (\S\ref{sec:decidable_set}). The
notion of \emph{Theorem}, however, is not in general Effective.



% --------------------------------------------------------------------
\subsection{Syntactic Consequence}\label{sec:syntactic_consequence}
% --------------------------------------------------------------------

For a Formal System $\mathcal{S}$, a WFF $A$ is a \emph{Syntactic
  Consequence} of a possibly empty Set of Formulas $\Gamma$ if there
is a Formal Proof (\S\ref{sec:formal_proof}) of $A$ from $\Gamma$ in
$\mathcal{S}$:
\[
    \Gamma \vdash_{\mathcal{S}} A
\]
where $\vdash_{\mathcal{S}}$ is the \emph{Syntactic Consequence
  Relation} on $\mathcal{S}$.



% --------------------------------------------------------------------
\subsection{Inference Rule} \label{sec:inference_rule}
% --------------------------------------------------------------------

An \emph{Inference Rule} is a Function which takes an input Set of
WFFs called \emph{Premises} and returns an output Set of WFFs called
\emph{Conclusions}. An alternative formulation for Inference Rules is
as a \emph{Deducibility Relation}, $\vdash$, that holds between zero
or more Premises and one or more Conclusions.

Taken individually, both Premises and Conclusions are
\emph{Propositions}, that is, bearers of \emph{Truth-value}
(\S\ref{sec:logical_truth}). The Conclusion relies on the truth of the
Premises; if the Premises are left \emph{Unsatisfied}
(\S\ref{sec:satisfaction}), then the Derivation is \emph{Hypothetical}
and the Premises \emph{Hypotheses}.

Inference Rules may be identified as reduntant in two senses:
\begin{enumerate}
\item An \emph{Admissible Rule} is one which does not change the Set
  of \emph{Theorems} in a Formal System when it is added.
\item A \emph{Derivable Rule} is a case of an Admissible Rule that has
  been Derived from existing rules.
\end{enumerate}



% --------------------------------------------------------------------
\subsection{Axiom}\label{sec:axiom}
% --------------------------------------------------------------------

\emph{Axioms} are given as WFFs, the truth value of which are assumed
for the purpose of performing analysis within a Formal System. Axioms
are special cases of Inference Rules which have no Premises, only a
universally Valid Conclusion; that is, a \emph{Logical Assertion}
(\S\ref{sec:sequent_notation}). Axioms may be further differentiated
from Rules by saying that Rules are statements \emph{about} the
system, and Axioms are statements \emph{in} the system.

Axioms may be divided into two kinds: \emph{Logical Axioms} of a
tautological sort, and \emph{Non-logical Axioms}, hereafter referred
to as \emph{Postulates}, that play the role of assumptions: defining
properties of the domain of the \emph{Theory}
(\S\ref{sec:formal_theory}) in question.

An \emph{Axiom Schema} is a template for Axioms in which one or more
\emph{Schematic Variables} (\S\ref{sec:metalanguage} Metalanguage)
appear, standing for a Subformula in the Object Language of the
System. For a Language with infinitely many WFFs, an Axiom Schema
describes a countably infinite number of Axioms. A System without
Schema is termed \emph{Finitely Axiomatized}.



\subsubsection{Logical Axiom}\label{sec:logical_axiom}



\subsubsection{Non-logical Axiom}\label{sec:nonlogical_axiom}

\emph{Non-logical Axiom} (or \emph{Postulate})



% --------------------------------------------------------------------
\subsection{Formal Proof} \label{sec:formal_proof}
% --------------------------------------------------------------------

A \emph{Formal Proof} (or \emph{Derivation}) is a finite sequence of
Well-formed Formulas of a Formal System that are either Axioms of a
Logical System or follow from the preceding Formulas by an Inference
Rule. The concluding Formula in the sequence is a \emph{Theorem}. A
Theorem used in the course of Deriving a further Theorem is called a
\emph{Lemma}.

For a Formal System, $\mathcal{S}$, of a set of Formulas, $\Gamma$,
there is a \emph{Syntactic Consequence}, $A$, if there is a
Formal Proof of $A$ from the set $\Gamma$:
\[
    \Gamma \vdash_{\mathcal{S}} A
\]
It will suffice for now to say that Syntactic Consequence as
differentiated from \emph{Material Consequence} ($\rightarrow$)
should:

\begin{enumerate}
\item Rely on the Logical Form (\S\ref{sec:inference_rule}) of the
  Expressions
\item Be completely \emph{a priori}
\item Be \emph{Modal} (\S\ref{sec:modal_logic}) (i.e. Necessary)
\end{enumerate}
Another form of Logical Consequence, \emph{Semantic Consequence}
(\S\ref{sec:semantic_consequence}) will be described under
\emph{Model Theory} (Part \ref{sec:model_theory}).

An \emph{Theoretic Analytic Proof} begins with an assumption and
proceeds to an accepted truth (an Axiom, or contradiction as in
Analytic Tableau (\S\ref{sec:tableau_calculus}). A \emph{Synthetic
  Proof} is the reverse of this process; beginning with known truths
and reasoning up to the desired Proof. A \emph{Problematic Analytic
  Proof} is constructed from given conditions that are to be
satisfied.



\subsubsection{Sequent Notation}\label{sec:sequent_notation}

A \emph{Sequent} is a specific kind of Judgement of the form
\[\Gamma \vdash \Sigma \]
where the \emph{Antecedent}, $\Gamma$, is a Conjunctive sequence of
Formula, and the \emph{Succedent}, $\Sigma$, is a Disjunctive sequence
of Formulas. Together, Antecedents and Succedents are
\emph{Cedents}. The Extra-logical Operators, $\vdash$, and $,$
(comma), are called \emph{Structural Operators} and Rules which change
only Structural Operators are \emph{Structural Rules} (as opposed to
\emph{Logical Rules}. A sequence of Cedents may be called a
\emph{Context}, but \emph{the} Context for a specific Judgement is
usually meant to be the Antecedent.

\emph{Weakening} refers to a Rule that introduces arbitrary elements
to a Sequent. \emph{Contraction} refers to a Rule that removes
multiple occurences of some element and \emph{Permutation} refers to
the re-ordering of elements. Logics lacking Structural Rules are
\emph{Substructural Logics} (\S\ref{sec:substructural_logic}). If
Sequents are defined as Sets or Multisets instead of Sequences (that
is, unordered Sets), then the Permutation rule is obsolete, likewise
the Contraction Rule would be obsolete for Sets instead of Sequences.

In a general \emph{Sequent Calculus} there may be any
number of Formulas on either side
\[
    A_1, \ldots, A_n \vdash B_1, \ldots, B_k
\]
is equivalent to
\[
    \vdash(A_1 \wedge \cdots \wedge A_n) \rightarrow (B_1 \vee \cdots \vee B_k)
\]
and the dual nature of Judgements and negation can be expressed by the
dual forms
\[
    \vdash \neg A_1 \vee \cdots \vee \neg A_n \vee B_1 \vee \cdots
    \vee B_k
\]
and
\[
    \vdash \neg(A_1 \wedge \cdots \wedge A_n \wedge \neg B_1 \wedge
    \cdots \wedge \neg B_k)
\]

A Sequent with no Succedent ($\Gamma \vdash$) is a
\emph{Contradiction} meaning it proves falsity which is
Inconsistent. A Sequent with no Antecedent ($\vdash \Sigma$) is a
\emph{Logical Assertion} and the Succedent is a
\emph{Tautology}. Theorems are those of the form $\vdash B$ which are
the Conclusion of a Valid Proof.



\subsubsection{Constructive Proof}\label{sec:constructive_proof}

\emph{Existence Theorem}



% --------------------------------------------------------------------
\subsection{Formal Theory}\label{sec:formal_theory}
% --------------------------------------------------------------------

A \emph{Formal Theory} is a Set of Sentences in some Formal Language
(Part \ref{sec:formal_language}). In a Deductive System, a Theory has
certain Sentences which are Axioms, and any Sentences which are a
Logical Consequence are also Sentences of that Theory. That is, a
Theory $\mathcal{T}$ which is an \emph{Inductive Class}. The Sentences
which are considered for a particular Theory are drawn from a
\emph{Conceptual Class} of \emph{Elementary Statements}. A Formal
Theory is said to be \emph{Complete} if for every Sentence in the
Language of that Theory, the Theory contains either that Sentence or
its Negation. Such a Set of Sentences is a \emph{Maximal Consistent
  Set}.

\emph{Synthetic Theory}

A Formal Theory is \emph{Categorical} if it has just one \emph{Model}
(Part \ref{sec:model_theory}) up to Isomorphism.

%FIXME connections with stability theory



\subsubsection{Duality Principle}\label{sec:duality_principle}



% ====================================================================
\section{Proof Calculi}
% ====================================================================

\emph{Proof Calculi} are families of Formal Systems
(\S\ref{sec:formal_system}), specifying templates for forms of
\emph{Formal Inference} (Axioms and Inference Rules).



% --------------------------------------------------------------------
\subsection{Axiomatic Systems}
% --------------------------------------------------------------------

\subsubsection{Hilbert Systems} \label{sec:hilbert_systems}

\emph{Hilbert Systems} are characterized by having a large number of
Axiom Schema and few Inference Rules-- just Modus Ponens for
Propositional Logics and Universal Generalization for Predicate
Logic. In a Hilbert System, Judgements and Formulas are not
differentiated. A Theorem in a Hilbert System is the Concluding
Judgement in a Derivation.

A Hilbert System is differentiated from Systems of \emph{Natural
 Deduction} by not having any Rules that change the Context of a
Formula.

% --------------------------------------------------------------------
\subsection{Structural Proof Theory}
% --------------------------------------------------------------------

\emph{Structural Proof Theory} studies Proof Calculi that support
\emph{Analytic Proof}; that is Proofs that are \emph{Cut-free} (they
do not use the \emph{Cut Rule}) or in \emph{Normal Form}.

\subsubsection{Natural Deduction}\label{sec:natural_deduction} \hfill
\\
Systems of \emph{Natural Deduction}\cite{prawitz65}, contrasted with
Hilbert Systems, include many Inference Rules but few or no Axioms. A
Natural Deduction System allows Judgements with multiple Antecedents
and a single Succedent
\[
    A_1,\ldots,A_n \vdash B
\]
Inference Rules in Natural Deduction have the general notation
\[
    {
        \frac{J_1 \quad J_2 \quad \cdots \quad J_n}
        {J}
    } name
\]
where the Rule with name $name$ has Premises of zero or more
Judgements $J_i$ and the Judgement $J$ is the Conclusion.

Inference Rules that introduce a Logical Connective in the Conclusion
are called \emph{Introduction Rules}. Example
\[
    {
        \frac{A\;\mathrm{true} \quad B\;\mathrm{true}}
        {(A \wedge B)\;\mathrm{true}}
    } \wedge_I
\]
where $A$ and $B$ are Propositions.

Conversely, Inference Rules that remove Logical Connectives are
\emph{Elimination Rules}.
\[
    {
        \frac{A \wedge B\;\mathrm{true}}
        {A\;\mathrm{true}}
    } \wedge_E
\]

\emph{Hypothetical Derivations} (reasoning from \emph{Assumptions})
are required for Implication Introduction or Disjunction
Elimination. The general form of a Hypothetical Derivation with
Antecedents $D_i$ and Succedent $J$:
\[
    D_1 \quad D_2 \cdots D_n
\]\[
    \vdots
\]\[
    J
\]
Introduction Rules for Implication:
\[
    {
        \frac{}
        {A\;\mathrm{true}}
    } u
\]\[
    \vdots
\]\[
    {
        \frac{B\;\mathrm{true}}
        {A \rightarrow B\;\mathrm{true}}
    } \rightarrow_{I^u}
\]
The Premise $u$ here is considered \emph{discharged} by the Rule
$I^u$; that is the scope of $u$ does not extend past $I^u$.
Elimination Rule for Implication (Modus Ponens):
\[
    {
        \frac{A \rightarrow B\;\mathrm{true} \quad A\;\mathrm{true}}
        {B\;\mathrm{true}}
    } \rightarrow_{E}
\]
Disjunctive Elimination:
\[
    \frac{
    A \vee B\;\mathrm{true} \quad
    \begin{matrix}
        {
            \frac{}
            {A\;\mathrm{true}}
        }u \\
        \vdots \\
        C\;\mathrm{true}
    \end{matrix}
    \quad
    \begin{matrix}
        {
            \frac{}
            {B\;\mathrm{true}}
        }w \\
        \vdots \\
        C\;\mathrm{true}
    \end{matrix}
    }{ C\;\mathrm{true}}\wedge_{E^{u,w}}
\]
A Theory is \emph{Locally Consistent} (or \emph{Locally Reducible}) if
an Introduction of a Connective followed by its Elimination can be
equivalently Derived without these steps.  The dual to Local
Consistency is \emph{Local Completeness} which states that Elimination
rules can decompose a Connective into the forms of its Introduction
Rule. These correspond to $\beta$-reduction and $\eta$-conversion in
$\lambda$-Calculus (\S\ref{sec:lambda_calculus}) where Propositions
are \emph{Types} and Proofs are \emph{Programs}. If an entire
Derivation has only Eliminations followed by Introductions, it is said
to be in \emph{Normal Form}.

In a Formal Proof, the Judgements representing Antecedents are
presented as Rules with no Premises, named by a \emph{Proof Variable}
(from a countable set $V$ of variables):
\[
    \frac{}{J_1}u_1 \; \frac{}{J_2}u_2 \; \cdots \frac{}{J_n}u_n
\]\[
    \vdots
\]\[
    J
\]
where $u_i \in V$. Written in Sequent Notation:
\[
    u_1:J_1, u_2:J_2, \ldots, u_n:J_n \vdash J
\]
This convention is sometimes called \emph{Localized Hypotheses}. In
general, $\pi : A$ may be read ``$\pi$ is a proof of $A$''.

\subsubsection{Sequent Calculus}

In \emph{Sequent Calculus} a Formal Proof is a Sequence of Sequents
(\S\ref{sec:sequent_notation}) where each successive Sequent is
Derivable from prior Sequents by Inference Rules.

The Rule for \emph{Cut} is as follows:
\[
    \frac{
        \Gamma \vdash \Delta, A \quad A, \Sigma \vdash \Pi
    }{
        \Gamma, \Sigma \vdash \Delta, \Pi
    }(Cut)
\]
It states that when a Formula $A$ that can be Concluded can also be
used as a Premise, it can be \emph{cut} out and the Derivations joined
together. That is, wherever the Lemma $A$ occurs, it can be
substituted for the Proof of $A$. This means that the Cut Rule is an
Admissible Rule (\S\ref{sec:inference_rule}).

The \emph{Cut-elimination Theorem} states that any Judgement with a
Proof in Sequent Calculus that uses the Cut Rule may be expressed as a
\emph{Cut-free} Proof without using the Cut Rule. Usually,
demonstrating the existence of the Cut-elmination Theorem implies that
the System is Consistent since that would rule-out the possibility of
Proof of Contradiction.

\paragraph{$\mathbf{LK}$} \hfill \\

Formalization of Classical Logic (\S\ref{sec:classical_logic})
(sound and complete in First-Order) with Sequents having zero or more
RHS Formulas. Allowing multiple RHS Formulas with a \emph{Right
  Contraction Rule} is equivalent to the admissibility of the
\emph{Law of the Excluded Middle}.

\paragraph{$\mathbf{LJ}$} \hfill \\

Formalization of Intuitionistic Logic
(\S\ref{sec:intuitionistic_logic}) with Sequents having at most one
RHS Formula. The Cut Rule for $\mathbf{LJ}$:
\[
    \frac{
        \Gamma \vdash A \quad \Pi, A \vdash B
    }{
        \Gamma, \Pi \vdash B
    }(Cut)
\]

\paragraph{Substructural Rule Sets} \hfill \\

A Substructural Logic (\S\ref{sec:substructural_logic}) lacking the
usual Structural Rules is usually weaker than $\mathbf{LK}$.

In Relevance Logic (\S\ref{sec:relevance_logic}), Weakening Rules
are not included on the grounds that introduced Formulas are not
\emph{Relevant}.

In Linear Logic (\S\ref{sec:linear_logic}), duplicate Formulas are
treated differently so Contraction and Weakening Rules are is absent
or controlled.



% --------------------------------------------------------------------
\subsection{Calculus of Structures}
% --------------------------------------------------------------------

\emph{Calculus of Structures} is a Proof Calculus with \emph{Deep
  Inference} for studying Non-commutative Logic
(\S\ref{sec:noncommutative_logic}). Deep Inference is a
generalization of Structure to handle greater Structural complexity
\cite{schutte77}.



% --------------------------------------------------------------------
\subsection{Tableau Calculus}\label{sec:tableau_calculus}
% --------------------------------------------------------------------

\emph{Tableau Calculus} (or \emph{Method of Analytic Tableau}) is
commonly used as a Proof procedure for Modal Logics
(\S\ref{sec:modal_logic}). An Analytic Tableau is a tree with a
Formula at the root and a Subformula at each node. A specific Tableau
Calculus is a finite collection of Rules for breaking down Logical
Connectives into constituent parts. Rules can be expressed as Sets,
Multisets, Lists, or Trees of Formulas. If Sets of Formulas are used
at each node (\emph{Set-labeled Tableau}), they are taken in
Conjunction.

A \emph{Refutation Tableau} attempts to show that a negation of the
root Formula cannot be satisfied, thereby proving Logical Truth of the
Formula. Rules for handling Logical Connectives may produce a branch
in the tree and if a branch leads to a Contradiction, the branch is
closed and if all branches are closed the Proof is complete and the
root Formula is proved. Nodes on a single branch are considered in
Conjunction, Nodes on separate branches are considered Disjunctively.

\emph{Non-destructive Tableau Calculi} use Rules that only allow
addition of nodes, while \emph{Destructive Tableau Calculi} use Rules
that allow modification of existing nodes. \emph{Proof Confluence} is
the property of a Tableau Calculus that a closed Tableau (for an
un-satisfiable set of Propositions) can always be generated from an
arbitrary partially constructed Tableau regardless of which Rules are
chosen at each application (if a choice between Rules is available). A
\emph{Strongly Complete Tableau} is one in which every Formula in
every branch has been expanded.

A method of dealing with non-determinism of rules involving Universal
Quantification (in First-order Tableau) is called
\emph{Unification}. This allows Free Variables to be substituted in
the Rule for Eliminating Universal Quantifiers, which can later be
Unified by choosing an appropriate Term to close the branch.

\emph{Clause Tableau} (Tableau Method applied to sets of Clauses) may
be used for increased efficiency. \emph{Connection Tableau} restrict
expansion of Clause Tableau branches (not the bare root) to contain
only Literals that unify with a Literal already on the branch
(\emph{Weak Connectedness}) or a Literal in the current leaf
(\emph{Strong Connectedness}).



% ====================================================================
\section{Realizability} \label{sec:realizability}
% ====================================================================



% ====================================================================
\section{Reverse Mathematics} \label{sec:reverse_mathematics}
% ====================================================================



% ====================================================================
\section{Metatheory} \label{sec:metatheory}
% ====================================================================

Proof Theory is itself \emph{Metamathematical} in that it studies the
form of Logical Consequence within Logical Systems, and therefore is
expressed as \emph{Metatheory}.

A \emph{Judgement} is an inductively definable assertion in the
Metatheory of a Logical System. That is, one that includes
Extra-logical Symbols, namely that of Logical Consequence, $\vdash$,
and commas used in \emph{Sequents}
(\S\ref{sec:sequent_notation}). In this way, Axioms are Judgements,
and a \emph{Formal Proof} expresses a Judgement with the Premises
being a sequence of Judgements and the Conclusion also a Judgement.
