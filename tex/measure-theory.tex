%%%%%%%%%%%%%%%%%%%%%%%%%%%%%%%%%%%%%%%%%%%%%%%%%%%%%%%%%%%%%%%%%%%%%%%%%%%%%%%%
%%%%%%%%%%%%%%%%%%%%%%%%%%%%%%%%%%%%%%%%%%%%%%%%%%%%%%%%%%%%%%%%%%%%%%%%%%%%%%%%
\part{Measure Theory}\label{part:measure_theory}
%%%%%%%%%%%%%%%%%%%%%%%%%%%%%%%%%%%%%%%%%%%%%%%%%%%%%%%%%%%%%%%%%%%%%%%%%%%%%%%%
%%%%%%%%%%%%%%%%%%%%%%%%%%%%%%%%%%%%%%%%%%%%%%%%%%%%%%%%%%%%%%%%%%%%%%%%%%%%%%%%

\fist Geometric Measure Theory (\S\ref{sec:geometric_measure_theory})

\fist Probability Theory (Part \ref{part:probability_theory})

\fist Geometric Probability (``\emph{Continuous Combinatorics}''
\S\ref{sec:geometric_probability}): analogies between \emph{Counting} and
\emph{Measure} (\S\ref{sec:measure}) \fist Combinatorics (Part
\ref{part:combinatorics})

(Kolmogorov33)

\begin{itemize}
  \item analogy between the \emph{Measure} (\S\ref{sec:measure}) of a Set and
    the \emph{Probability} (\S\ref{sec:probability}) of an Event
  \item analogy between the \emph{Expectation} (\S\ref{sec:expected_value}) of a
    Random Variable (\S\ref{sec:random_variable}), and \emph{Lebesgue
      Integration} (\S\ref{sec:lebesgue_integral})
\end{itemize}

2010 - Tao - \emph{An Epsilon of Room: pages from year three of a mathematical
  blog}, Chapter 1. Real analysis:



% ==============================================================================
\section{$\sigma$-algebra}\label{sec:sigma_algebra}
% ==============================================================================

a \emph{$\sigma$-algebra} is a Set Algebra (Field of Sets
\S\ref{sec:set_algebra}) that is Closed under Countable Intersections and
Countable Unions and the corresponding Field of Sets is called a
\emph{Measurable Space} (\S\ref{sec:measurable_space})

a $\sigma$-algebra on a Set $X$ is a Non-empty Set $\mathcal{A} \subseteq 2^X$
that is Closed under Set Operations of Complement
(\S\ref{sec:absolute_complement}) and Countable Union (\S\ref{sec:set_union})

\begin{enumerate}
  \item $\varnothing \in \mathcal{A}$
  \item $X \in \mathcal{A}$
  \item $(E_i)_{i \geq 1} \in \mathcal{A} \Rightarrow
    \bigcap_{i \geq 1} E_i \in \mathcal{A}$
\end{enumerate}

cf. \emph{Field of Probabilities} (\S\ref{sec:probability}) (Kolmogorov33)

the effect of a $\sigma$-algebra is to restrict the Domain so that not every
Subset of a Sample Space needs to have a Probability which is crucial for
Probabilities to be defined on Uncountably Infinite Sets
(\url{https://plato.stanford.edu/entries/logic-probability/#ProSpa})

\fist on Finite or Countably Infinite Sets, every $\sigma$-algebra is a
\emph{Topology} (\S\ref{sec:topology}), and on every Uncountable Set there is a
$\sigma$-algebra that \emph{isn't} a Topology, viz. the
\emph{Countable-cocountable $\sigma$-algebra}
--\url{https://math.stackexchange.com/questions/51222/is-there-an-example-of-a-sigma-algebra-that-is-not-a-topology}

$\Sigma$ Subsets of $X$

$\{ \varnothing, X \}$

Collection of $\sigma$-algebras $\{ \Sigma_\alpha : \alpha \in \class{A} \}$

$\mathcal{F}$ Family of Subsets of $X$, $\sigma(\mathcal{F})$ is the
$\sigma$-algebra Generated by $\mathcal{F}$

$\sigma(\varnothing) = \{ \varnothing, X \}$



% ------------------------------------------------------------------------------
\subsection{$\sigma$-additivity}\label{sec:sigma_additivity}
% ------------------------------------------------------------------------------

\emph{Countable Additivity}

cf. \emph{Additivity} (Finite Additivity \S\ref{sec:additivity}):
$\sigma$-additivity implies Additivity

\fist $\sigma$-additivity Axiom (\S\ref{sec:probability_axioms}): the
Probability of a Countable Sequence of Disjoint Sets (Mutually Exclusive Events
\S\ref{sec:mutually_exclusive}) is equal to the Sum of the individual
Probabilities

cf. Lebesgue Measure (\S\ref{sec:lebesgue_measure})



% ------------------------------------------------------------------------------
\subsection{Borel Algebra}\label{sec:borel_algebra}
% ------------------------------------------------------------------------------

Collection of all Borel Sets (\S\ref{sec:borel_set}) on a Topological Space $X$
forms the \emph{Borel Algebra} on $X$ which is the smallest $\sigma$-algebra
containing all Open Sets (or equivalently all Closed Sets) of $X$

any Measure defined on the Borel Sets is called a Borel Measure
(\S\ref{sec:borel_measure})

on $\reals$: $\struct{B}_\reals$ is equal to the Minimal $\sigma$-algebra
Generated by the Closed Sets $E_2 = \{ [a,b] \;|\; a < b \}$.

Measurable Sets (\S\ref{sec:measurable_set}) on the Real Line are ``iterated''
Countable Unions and Intersections of Borel Sets



\subsubsection{Borel Hierarchy}\label{sec:borel_hierarchy}\hfill

\fist Descriptive Set Theory (\S\ref{sec:descriptive_set_theory})



% ------------------------------------------------------------------------------
\subsection{Measurable Space}\label{sec:measurable_space}
% ------------------------------------------------------------------------------

a Measurable Space is the Field of Sets underlying a $\sigma$-algebra

(wiki):

a \emph{Measurable Space} or \emph{Borel Space} is a Set $X$ together with a
$\sigma$-algebra $\sigma$ on the Set:
\[
  (X,\sigma)
\]

\fist A \emph{Measure Space} (\S\ref{sec:measure_space}) is a Measurable Space
equipped with a \emph{Measure} (\S\ref{sec:measure})
$\mu : \sigma \rightarrow \overline{\reals}$

\emph{Loomis-Sikorski Theorem} -- a Stone-type Duality
(\S\ref{sec:stone_duality}) between Countably Complete Boolean Algebras
(``\emph{Abstract $\sigma$-algebras}'' \S\ref{sec:boolean_algebra}) and
Measurable Spaces

(Tao10): the notion of a Measurable Space, $(X,\sigma)$, and Measurable
Functions (\S\ref{sec:measurable_function}) is similar to that of a
\emph{Topological Space} (\S\ref{sec:topological_space}), $(X,\tau)$, and
Continuous Functions (\S\ref{sec:continuous_function}):
\begin{itemize}
  \item both $\tau$ and $\sigma$ contain both $\varnothing$ and $X$
  \item $\tau$ is closed under Finite Intersections and Finite or Infinite
    Unions
  \item $\sigma$ is closed under Complement and Countable Unions (and,
    by implication, Countable Intersections)
\end{itemize}
The Open Sets $\tau$ of a Topological Space \emph{generate} a $\sigma$-algebra,
known as the \emph{Borel Algebra} (\S\ref{sec:borel_algebra}), that is, the
smallest $\sigma$-algebra containing all Open Sets (or equivalently, all Closed
Sets)



% ------------------------------------------------------------------------------
\subsection{Measurable Function}\label{sec:measurable_function}
% ------------------------------------------------------------------------------

\emph{A Categorical Look at Random Variables} -
\url{https://golem.ph.utexas.edu/category/2018/09/a_categorical_look_at_random_v.html}:

$\cat{Prob}$ -- Category with Objects as Probability Spaces
(\S\ref{sec:probability_space}) and Morphisms are ``Almost-everywhere-equality
Equivalence Classes'' of Measure-preserving Maps
(\S\ref{sec:measure_preserving_map})

the fundamental ``objects'' of Probability Theory are the \emph{Morphisms} of
$\cat{Prob}$ and these Morphisms are \emph{Random Variables}
(\S\ref{sec:random_variable})

a Random Variable is defined as a \emph{Measurable Map}:
\[
  X : \Omega \rightarrow E
\]
where $(\Omega,\mathbb{P})$ is a Probability Space and $E$ is an arbitrary
Measurable Space



% ==============================================================================
\section{Measure}\label{sec:measure}
% ==============================================================================

Set $X$

$(X,\Sigma)$ -- $\sigma$-algebra (\S\ref{sec:sigma_algebra}) on $X$

A \emph{Measure} on $(X,\Sigma)$:
\[
  \mu : \Sigma \rightarrow [0,\infty]
\]

\begin{enumerate}
  \item $\mu(\varnothing) = 0$

  \item For $(E_i)_{i \geq 1} \in \Sigma$ with $E_i \cap E_j =
    \varnothing$ when $i \neq j$, then:
    \[
      \mu (\bigcup_i E_i) = \sum_i \mu(E_i)
    \]
\end{enumerate}

$(X,\Sigma,\mu)$ is called a \emph{Measure Space} (\S\ref{sec:measure_space})
and a Measure Space with a Probability Measure
(\S\ref{sec:probability_measure}) is a \emph{Probability Space}
(\S\ref{sec:probability_space}); a Probability Measure must assign a value of
$1$ to the entire Probability Space.

Measurable

\fist cf. \emph{Valuation} (\S\ref{sec:domain_valuation}) -- a Borel Measure
always restricts to a Valuation

(Kolmogorov33) analogy between the Measure of a Set and the \emph{Probability}
(\S\ref{sec:probability}) of an Event

\emph{Countable Additivity}

in Integration Theory (\S\ref{sec:integral}), specifying a Measure allows the
definition of Integrals on Spaces more general than Subsets of Euclidean Space

\fist \emph{Unit Measure Axiom} (Kolmogorov Axioms
\S\ref{sec:probability_axioms})

\fist a \emph{Content} (\S\ref{sec:content}) is a Set Function like Measure that
need not be Countably Additive, but must only be Finitely Additive



% ------------------------------------------------------------------------------
\subsection{Measure Space}\label{sec:measure_space}
% ------------------------------------------------------------------------------

$(X,\Sigma,\mu)$

\fist cf. \emph{Measureable Space} (\S\ref{sec:measureable_space}) -- a Set
together $X$ with a $\sigma$-algebra $\Sigma$ on the Set: $(X,\Sigma)$

Abstract Axiomatic Structure needed to define a Theory of \emph{Integration}
(\S\ref{sec:integral_calculus})

cf. \emph{Metric Spaces} (\S\ref{sec:metric_space}) -- Abstract Axiomatic
Structure needed to define Theory of \emph{Differentiation}
(\S\ref{sec:differential_calculus})

A Measure Space with a Probability Measure (\S\ref{sec:probability_measure}) is
a \emph{Probability Space} (\S\ref{sec:probability_space}). A Probability
Measure must assign a value of $1$ to the entire Probability Space.

For $E,F \in \Sigma$:
\[
  \mu(E) = \mu(E \cap F) + \mu(E \cap F^c)
\]

For $E,F \in \Sigma$ such that $F \subset E$:
\[
  \mu(F) \leq \mu(E)
\]

Countable Additivity (or $\sigma$-additivity): %FIXME

For $E_i \in \Sigma$ (Countable Collections):
\[
  \mu(\bigcup_i E_i) \leq \sum_i \mu(E_i)
\]



\subsubsection{Null Set}\label{sec:null_set}\hfill

$M = (X, \Sigma, \mu)$

$S \subset X$ such that $\mu(S) = 0$

a Null Set $N \subset \reals$ is a Set that can be Covered (\S\ref{sec:cover})
by a Countable Union of Intervals of arbitrarily small total length

\begin{itemize}
  \item Non-random Infinite Binary Random Sequences form a maximal Constructive
    Null Set
  \item ...
\end{itemize}



% ------------------------------------------------------------------------------
\subsection{Distribution Function}\label{sec:distribution_function}
% ------------------------------------------------------------------------------

\fist Cumulative Distribution Function (Probability Theory \S\ref{sec:cdf})



% ------------------------------------------------------------------------------
\subsection{Topological Support}\label{sec:topological_support}
% ------------------------------------------------------------------------------

or \emph{Topological Support} or \emph{Spectrum} of a Measure $\mu$ on a
Measurable Topological Space $(\xspace{X}, Borel(\xspace{X}))$ is the largest
Closed Subset of $\xspace{X}$ for which every Open Neighborhood of every Point
of the Set has a Positive Measure

cf. Support of a Real-valued Function (\S\ref{sec:support})



% ------------------------------------------------------------------------------
\subsection{Measure-preserving Map}\label{sec:measure_preserving_map}
% ------------------------------------------------------------------------------

cf. Measure-preserving Dynamical Systems (\S\ref{sec:measure_preserving_system})

\url{https://golem.ph.utexas.edu/category/2018/09/a_categorical_look_at_random_v.html}:

$\cat{Prob}$ -- Category with Objects as Probability Spaces
(\S\ref{sec:probability_space}) and Morphisms are ``Almost-everywhere-equality
Equivalence Classes'' of Measure-preserving Maps

the fundamental ``objects'' of Probability Theory are the \emph{Morphisms} of
$\cat{Prob}$ and these Morphisms are \emph{Random Variables}
(\S\ref{sec:random_variable})



% ------------------------------------------------------------------------------
\subsection{$\sigma$-finite Measure}\label{sec:sigma_finite}
% ------------------------------------------------------------------------------

$(X,\Sigma,\mu)$ is \emph{$\sigma$-finite} if there exists a Sequence
$(E_i)_{i \geq 1}$ in $M$ such that $\bigcup_{i} E_i = X$ and
$\mu(E_i) \leq \infty$.



\subsubsection{Counting Measure}\label{sec:counting_measure}

$\mu(A) = |A|$ for $A \subset \nats$



\subsubsection{Lebesgue Measure}\label{sec:lebesgue_measure}

\emph{$n$-dimensional Volume}

Measure of Subsets of $\reals$

$m((a,b)) = b - a$

\emph{Lebesgue Measurable}

if the Axiom of Choice is accepted, according to the \emph{Vitali Theorem},
there are Uncountably many \emph{Vitali Sets} which are not Lebesgue Measurable

\emph{Solovay Model}: assuming the existence of an \emph{Inaccessible Cardinal}
(\S\ref{sec:inaccessible_cardinal}), a Model of ZF Set Theory without the Axiom
of Choice (but the Axiom of Countable choice holds) in which all Sets of Real
Numbers are Lebesgue Measurable

Lebesgue Outer Measure (\S\ref{sec:lebesgue_outer_measure})

cf. Lebesgue Integral (\S\ref{sec:lebesgue_integral})

a Bounded Function (\S\ref{sec:bounded_function}) on a Compact Interval $[a,b]$
is Riemann Integrable (\S\ref{sec:integrable_function}) if and only if it is
Continuous (\S\ref{sec:continuous_function}) ``Almost Everywhere'', i.e. Set of
Points of Discontinuity has Measure Zero



% ------------------------------------------------------------------------------
\subsection{Locally Finite Measure}\label{sec:locally_finite_measure}
% ------------------------------------------------------------------------------

% ------------------------------------------------------------------------------
\subsection{Borel Measure}\label{sec:borel_measure}
% ------------------------------------------------------------------------------

any Measure defined on the Borel Sets of a Borel Algebra
(\S\ref{sec:borel_algebra})

Moderate Measure



\subsubsection{Regular Borel Measure}\label{sec:regualr_borel}\hfill

\paragraph{Radon Measure}\label{sec:radon_measure}\hfill

both Regular and Locally Finite (\S\ref{sec:locally_finite_measure})



% ------------------------------------------------------------------------------
\subsection{Similarity Measure}\label{sec:similarity_measure}
% ------------------------------------------------------------------------------

\fist cf. Similarity Transformation (\S\ref{sec:similarity_transformation})

Cross-correlation (Functional Analysis \S\ref{sec:cross_correlation})



% ------------------------------------------------------------------------------
\subsection{Outer Measure}\label{sec:outer_measure}
% ------------------------------------------------------------------------------

\subsubsection{Lebesgue Outer Measure}\label{sec:lebesgue_outer_measure}

Lebesgue Measure (\S\ref{sec:lebesgue_measure})



\subsubsection{Hausdorff Measure}\label{sec:hausdorff_measure}

\fist Geometric Measure Theory (\S\ref{sec:geometric_measure_theory})



% ------------------------------------------------------------------------------
\subsection{Cylinder Set Measure}\label{sec:cylinder_set_measure}
% ------------------------------------------------------------------------------

kind of prototype for a Measure on Infinite-dimensional Vector Spaces
(\S\ref{sec:infinite_dimensional_vectorspace})



\subsubsection{Wiener Measure}\label{sec:wiener_measure}\hfill

the collection of Functions that are Differentiable at a single point of
$[0,1]$ has Wiener Measure $0$, even when taking Finite-dimensional ``slices''
of the Vector Space of Continuous Functions $C([0,1];\reals)$, in the sense
that the Nowhere-differentiable Functions (\S\ref{sec:nowhere_differentiable})
form a Prevalent Subset (\S\ref{sec:prevalent_set}) of $C([0,1]; \reals)$



% ==============================================================================
\section{Measurable Set}\label{sec:measurable_set}
% ==============================================================================

Measurable Sets on the Real Line are ``iterated'' Countable Unions and
Intersections of Borel Sets (\S\ref{sec:borel_set})

Solovay's Model: all Subsets of the Reals are Measurable (excluding Uncountable
Choice)



% ==============================================================================
\section{Non-measurable Set}\label{sec:nonmeasurable_set}
% ==============================================================================

% ==============================================================================
\section{Almost Everywhere}\label{sec:almost_everywhere}
% ==============================================================================

a Generic Property (\S\ref{sec:generic_property}) is one that holds Almost
Everywhere



% ==============================================================================
\section{Negligible Function}\label{sec:negligible_function}
% ==============================================================================

% ==============================================================================
\section{Negligible Set}\label{sec:negligible_set}
% ==============================================================================

Meagre Set (\S\ref{sec:meagre_set}): a Negligible Subset of a Topological Space



% ==============================================================================
\section{Generic Property}\label{sec:generic_measure_property}
% ==============================================================================

a Property that holds Almost Everywhere

opposite of a Negligible Set

cf. Generic Property (Algebraic Geometry \S\ref{sec:generic_property})



% ==============================================================================
\section{Prevalent Set}\label{sec:prevalent_set}
% ==============================================================================

the collection of Functions that are Differentiable at a single point of
$[0,1]$ has Wiener Measure (\S\ref{sec:wiener_measure}) $0$, even when taking
Finite-dimensional ``slices'' of $C([0,1];\reals)$, in the sense that the
Nowhere-differentiable Functions (\S\ref{sec:nowhere_differentiable}) form a
Prevalent Subset of $C([0,1]; \reals)$



% ==============================================================================
\section{Content}\label{sec:content}
% ==============================================================================

(wiki): a Set Function like Measure (\S\ref{sec:measure}) that need not be
Countably Additive, but must only be Finitely Additive
