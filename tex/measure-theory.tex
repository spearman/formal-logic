%%%%%%%%%%%%%%%%%%%%%%%%%%%%%%%%%%%%%%%%%%%%%%%%%%%%%%%%%%%%%%%%%%%%%%%%%%%%%%%%
%%%%%%%%%%%%%%%%%%%%%%%%%%%%%%%%%%%%%%%%%%%%%%%%%%%%%%%%%%%%%%%%%%%%%%%%%%%%%%%%
\part{Measure Theory}\label{part:measure_theory}
%%%%%%%%%%%%%%%%%%%%%%%%%%%%%%%%%%%%%%%%%%%%%%%%%%%%%%%%%%%%%%%%%%%%%%%%%%%%%%%%
%%%%%%%%%%%%%%%%%%%%%%%%%%%%%%%%%%%%%%%%%%%%%%%%%%%%%%%%%%%%%%%%%%%%%%%%%%%%%%%%

\fist Geometric Measure Theory (\S\ref{sec:gmt})

\fist Probability Theory (Part \ref{part:probability_theory})

\fist Geometric Probability (``\emph{Continuous Combinatorics}''
\S\ref{sec:geometric_probability}): analogies between \emph{Counting} and
\emph{Measure} (\S\ref{sec:measure}) \fist Combinatorics (Part
\ref{part:combinatorics})

(Kolmogorov33)

\begin{itemize}
  \item analogy between the \emph{Measure} (\S\ref{sec:measure}) of a Set and
    the \emph{Probability} (\S\ref{sec:probability}) of an Event
  \item analogy between the \emph{Expectation} (\S\ref{sec:expected_value}) of a
    Random Variable (\S\ref{sec:random_variable}), and \emph{Lebesgue
      Integration} (\S\ref{sec:lebesgue_integral})
\end{itemize}

1968 - Segal, Kunze - \emph{Integrals and Operators}; Real Analysis, Integration
Theory, Spectral Theory

2001 - Pollard - \emph{A User's Guide to Measure Theoretic Probability}

2010 - Tao - \emph{An Epsilon of Room: pages from year three of a mathematical
  blog}, Chapter 1. Real analysis:

2012 - Tao - \emph{An introduction to measure theory}



% ==============================================================================
\section{$\pi$-system}\label{sec:pi_system}
% ==============================================================================

a \emph{$\pi$-system} on a Set $\Omega$ is a collection of Subsets
$P \subseteq \Omega^2$ such that $P$ is Non-empty and
$A, B \in P \Rightarrow A \cap B \in P$



% ==============================================================================
\section{$\lambda$-system}\label{sec:lambda_system}
% ==============================================================================

a \emph{$\lambda$-system} or \emph{Dynkin System} on a Set $\Omega$ is a
collection of Subsets $D \subseteq \Omega^2$ such that:
\begin{itemize}
  \item $\Omega \in D$
  \item $A \in D \Rightarrow \bar{A} \in D$
  \item $A_1, A_2, A_3, \ldots \in D :
    \forall i \neq jA_i \cap A_j = \varnothing
    \Rightarrow \bigcup_{n=1}^\infty A_n \in D$
\end{itemize}



% ==============================================================================
\section{$\sigma$-algebra}\label{sec:sigma_algebra}
% ==============================================================================

a \emph{$\sigma$-algebra} is a Set Algebra (Field of Sets
\S\ref{sec:set_algebra}) that is Closed under Countable Intersections and
Countable Unions and the corresponding Field of Sets is called a
\emph{Measurable Space} (\S\ref{sec:measurable_space})

a $\sigma$-algebra on a Set $X$ is a Non-empty Set $\mathcal{A} \subseteq 2^X$
that is Closed under Set Operations of Complement
(\S\ref{sec:absolute_complement}) and Countable Union (\S\ref{sec:set_union})

\begin{enumerate}
  \item $\varnothing \in \mathcal{A}$
  \item $X \in \mathcal{A}$
  \item $(E_i)_{i \geq 1} \in \mathcal{A} \Rightarrow
    \bigcap_{i \geq 1} E_i \in \mathcal{A}$
\end{enumerate}

cf. \emph{Field of Probabilities} (\S\ref{sec:probability}) (Kolmogorov33)

a $\pi$-system (\S\ref{sec:pi_system}) that is also a $\lambda$-system
(\S\ref{sec:lambda_system}) is a $\sigma$-algebra

the effect of a $\sigma$-algebra is to restrict the Domain so that not every
Subset of a Sample Space needs to have a Probability which is crucial for
Probabilities to be defined on Uncountably Infinite Sets
(\url{https://plato.stanford.edu/entries/logic-probability/#ProSpa})

if a $\sigma$-algebra is Infinite, then it is Uncountably Infinite

\fist on Finite or Countably Infinite Sets, every $\sigma$-algebra is a
\emph{Topology} (\S\ref{sec:topology}), and on every Uncountable Set there is a
$\sigma$-algebra that \emph{isn't} a Topology, viz. the
\emph{Countable-cocountable $\sigma$-algebra}
--\url{https://math.stackexchange.com/questions/51222/is-there-an-example-of-a-sigma-algebra-that-is-not-a-topology}

Topologies however do not require Complements and Topologies only require all
Finite Intersections instead of all \emph{Countable} Intersections

$\Sigma$ Subsets of $X$

$\{ \varnothing, X \}$

Collection of $\sigma$-algebras $\{ \Sigma_\alpha : \alpha \in \class{A} \}$

$\mathcal{F}$ Family of Subsets of $X$, $\sigma(\mathcal{F})$ is the
$\sigma$-algebra Generated by $\mathcal{F}$

$\sigma(\varnothing) = \{ \varnothing, X \}$



% ------------------------------------------------------------------------------
\subsection{$\sigma$-additivity}\label{sec:sigma_additivity}
% ------------------------------------------------------------------------------

\emph{Countable Additivity}

cf. \emph{Additivity} (Finite Additivity \S\ref{sec:additivity}):
$\sigma$-additivity implies Additivity

\fist $\sigma$-additivity Axiom (\S\ref{sec:probability_axioms}): the
Probability of a Countable Sequence of Disjoint Sets (Mutually Exclusive Events
\S\ref{sec:mutually_exclusive}) is equal to the Sum of the individual
Probabilities

cf. Lebesgue Measure (\S\ref{sec:lebesgue_measure})



% ------------------------------------------------------------------------------
\subsection{Borel Algebra}\label{sec:borel_algebra}
% ------------------------------------------------------------------------------

Collection of all Borel Sets (\S\ref{sec:borel_set}) on a Topological Space $X$
forms the \emph{Borel Algebra} (or \emph{Borel $\sigma$-algebra}) $B$ on $X$,
which is the smallest $\sigma$-algebra containing all Open Sets (or equivalently
all Closed Sets) of $X$, and the Measurable Space (\S\ref{sec:measurable_space})
$(X,B)$ is called the \emph{Borel Space} of $X$

\fist cf. Standard Borel Space (\S\ref{sec:standard_borel_space}) -- the Borel
Space of a Polish Space (\S\ref{sec:polish_space}); a Measurable Space $(X,
\Sigma)$ is ``Standard Borel'' if there exists a Metric on $X$ which makes it a
Complete Separable Metric Space such that $\Sigma$ is a Borel $\sigma$-algebra

any Measure defined on the Borel Sets is called a Borel Measure
(\S\ref{sec:borel_measure})

on $\reals$: $\struct{B}_\reals$ is equal to the Minimal $\sigma$-algebra
Generated by the Closed Sets $E_2 = \{ [a,b] \;|\; a < b \}$.

Measurable Sets (\S\ref{sec:measurable_set}) on the Real Line are ``iterated''
Countable Unions and Intersections of Borel Sets



\subsubsection{Borel Hierarchy}\label{sec:borel_hierarchy}\hfill

\fist Descriptive Set Theory (\S\ref{sec:descriptive_set_theory})



% ==============================================================================
\section{Measurable Space}\label{sec:measurable_space}
% ==============================================================================

a Measurable Space is the Field of Sets underlying a $\sigma$-algebra

(wiki):

a \emph{Measurable Space} or \emph{Borel Space} is a Set $X$ together with a
$\sigma$-algebra $\Sigma$ on the Set:
\[
  (X,\Sigma)
\]
Members of $\Sigma$ are called \emph{Measurable Sets}
(\S\ref{sec:measurable_set}), and Functions between Measurable Spaces such that
the Preimages are Measurable are called \emph{Measurable Functions}
(\S\ref{sec:measurable_function})

\fist a \emph{Measure Space} (\S\ref{sec:measure_space}) is a Measurable Space
equipped with a \emph{Measure} (\S\ref{sec:measure})
$\mu : \sigma \rightarrow \overline{\reals}$

\fist cf. Standard Borel Space (\S\ref{sec:standard_borel_space}) -- the Borel
Space of a Polish Space (\S\ref{sec:polish_space}); a Measurable Space $(X,
\Sigma)$ is ``Standard Borel'' if there exists a Metric on $X$ which makes it a
Complete Separable Metric Space such that $\Sigma$ is a Borel $\sigma$-algebra
(\S\ref{sec:borel_algebra})

\emph{Loomis-Sikorski Theorem} -- a Stone-type Duality
(\S\ref{sec:stone_duality}) between Countably Complete Boolean Algebras
(``\emph{Abstract $\sigma$-algebras}'' \S\ref{sec:boolean_algebra}) and
Measurable Spaces

(Tao10): the notion of a Measurable Space, $(X,\sigma)$, and Measurable
Functions (\S\ref{sec:measurable_function}) is similar to that of a
\emph{Topological Space} (\S\ref{sec:topological_space}), $(X,\tau)$, and
Continuous Functions (\S\ref{sec:continuous_function}):
\begin{itemize}
  \item both $\tau$ and $\sigma$ contain both $\varnothing$ and $X$
  \item $\tau$ is closed under Finite Intersections and Finite or Infinite
    Unions
  \item $\sigma$ is closed under Complement and Countable Unions (and,
    by implication, Countable Intersections)
\end{itemize}
The Open Sets $\tau$ of a Topological Space \emph{generate} a $\sigma$-algebra,
known as the \emph{Borel Algebra} (\S\ref{sec:borel_algebra}), that is, the
smallest $\sigma$-algebra containing all Open Sets (or equivalently, all Closed
Sets)



% ------------------------------------------------------------------------------
\subsection{Measurable Set}\label{sec:measurable_set}
% ------------------------------------------------------------------------------

a Member of the $\sigma$-algebra of a Measurable Space

Measurable Sets on the Real Line are ``iterated'' Countable Unions and
Intersections of Borel Sets (\S\ref{sec:borel_set})

Solovay's Model: all Subsets of the Reals are Measurable (excluding Uncountable
Choice)

cf. Measurable Cardinals (\S\ref{sec:measurable_cardinal})



\subsubsection{Atom}\label{sec:atom}

a Measurable Set that has Positive Measure and contains no Set of smaller
Positive Measure

cf. Elementary (Atomic) Event (\S\ref{sec:elementary_event})

\emph{Diffuse Measure} (\S\ref{sec:diffuse_measure}) is a Measure which has no
Atoms



% ------------------------------------------------------------------------------
\subsection{Measurable Function}\label{sec:measurable_function}
% ------------------------------------------------------------------------------

a \emph{Measurable Function} is a Function between two Measurable Spaces such
that the Preimage of any Measurable Set is Measurable

for a Measure Space $(E, X, \mu)$ and Function $f : E \rightarrow \reals$, an
equivalent requirement is that the Pre-image of any Borel Subset
(\S\ref{sec:borel_set}) of $\reals$ is in $X$

\fist cf. Continuous Functions (\S\ref{sec:continuous_function}) -- a Function
between Metric Spaces (\S\ref{sec:metric_space}) is Continuous if the Preimage
of each Open Set is Open

Measurable Functions are closed under Algebraic Operations, and Point-wise
Sequential Limits:
\begin{align*}
  & \sup_{k\in\nats} f_k \\
  & {\lim \inf}_{k\in\nats} f_k \\
  & {\lim \sup}_{k\in\nats} f_k \\
\end{align*}
are Measurable if the original Sequence $(f_k)_k, k \in \nats$ consists of
Measurable Functions

\fist Lebesgue Integrals (\S\ref{sec:lebesgue_integral})

a Random Variable (\S\ref{sec:random_variable}) is defined as a Measurable
Function:
\[
  X : \Omega \rightarrow E
\]
where $(\Omega,\mathbb{P})$ is a Probability Space (i.e. a Measure Space with a
Probability Measure) and $E$ is an arbitrary Measurable Space called the
\emph{State Space}

\emph{A Categorical Look at Random Variables} -
\url{https://golem.ph.utexas.edu/category/2018/09/a_categorical_look_at_random_v.html}:

$\cat{Prob}$ -- Category with Objects as Probability Spaces
(\S\ref{sec:probability_space}) and Morphisms are ``Almost-everywhere-equality
Equivalence Classes'' of Measure-preserving Maps
(\S\ref{sec:measure_preserving_map})

the fundamental ``objects'' of Probability Theory are the \emph{Morphisms} of
$\cat{Prob}$ and these Morphisms are \emph{Random Variables}



\subsubsection{Slowly Varying Function}\label{sec:slowly_varying}

behavior at Inifinity is ``similar'' to the behavior of a Function Converging at
Infinity

cf. Power Law (Regular Variation \S\ref{sec:power_law})



\subsubsection{Regularly Varying Function}\label{sec:regularly_varying}

behavior at Infinity is similar to the behavior of a Power Law
(\S\ref{sec:power_law}) Function (cf. Polynomial
\S\ref{sec:polynomial_function})

Fat-tailed Distributions (\S\ref{sec:fat_tailed})



% ==============================================================================
\section{Measure}\label{sec:measure}
% ==============================================================================

A \emph{Measure} on a Measurable Space (\S\ref{sec:measurable_space})
$(X,\Sigma)$ is a Function:
\[
  \mu : \Sigma \rightarrow [0,\infty]
\]
Satisfying:
\begin{enumerate}
  \item $\forall E \in \Sigma, \mu(E) \geq 0$ -- \emph{Non-negativity}
  \item $\mu(\varnothing) = 0$ -- \emph{Null Empty Set}
  \item For $(E_i)_{i \geq 1} \in \Sigma$ with $E_i \cap E_j = \varnothing$
    when $i \neq j$, then:
    \[
      \mu (\bigcup_i E_i) = \sum_i \mu(E_i)
    \]
    -- \emph{Countable Additivity} (\emph{$\sigma$-additivity})
\end{enumerate}
Note that a \emph{Signed Measure} (\S\ref{sec:signed_measure}) does not require
Property (1.) of Non-negativity.

Measures on the Real Line are special cases of \emph{Distributions}
(\S\ref{sec:distribution})

there is a one-to-one correspondence between Measures and Integration Operators
(\S\ref{sec:integral_operator}); the Set $T(X)$ of Integration Operators on a
Set $X$ is a Functor $T : \cat{Set} \to \cat{Set}$ and is the Codensity monad of
the Inclusion $\cat{Ctbl} \hookrightarrow \cat{Set}$ of Countable Sets into the
Category of Sets
(\url{https://golem.ph.utexas.edu/category/2021/07/large_sets_11.html})
%FIXME: is this the correct notion of integration operator?

$(X,\Sigma,\mu)$ is called a \emph{Measure Space} (\S\ref{sec:measure_space})
and a Measure Space with a Probability Measure
(\S\ref{sec:probability_measure}) is a \emph{Probability Space}
(\S\ref{sec:probability_space}); a Probability Measure must assign a value of
$1$ to the entire Probability Space.

\fist Ergodic Theory (\S\ref{sec:ergodic_theory}): Measure-preserving
Endomorphisms of Probability Spaces

\fist cf. \emph{Valuation} (\S\ref{sec:domain_valuation}) -- a Borel Measure
(\S\ref{sec:borel_measure}) always restricts to a Valuation

(Kolmogorov33) analogy between the Measure of a Set and the \emph{Probability}
(\S\ref{sec:probability}) of an Event

\emph{Countable Additivity}

in Integration Theory (\S\ref{sec:integral}), specifying a Measure allows the
definition of Integrals on Spaces more general than Subsets of Euclidean Space

\fist \emph{Unit Measure Axiom} (Kolmogorov Axioms
\S\ref{sec:probability_axioms})

\fist a \emph{Content} (\S\ref{sec:content}) is a Set Function like Measure that
need not be Countably Additive, but must only be Finitely Additive

\fist Integral Geometry (\S\ref{sec:integral_geometry}): Theory of Measures on
``Geometrical Space'' Invariant under Symmetry Group
(\S\ref{sec:symmetry_group}) of that Space

2020 - Hedges - \emph{Probabilistic programming with continuations}:

loose equivalence between Functional Analysis and Measure Theory

\emph{Riesz Representation Theorem} (\S\ref{sec:riesz_representation}): every
Linear Operator that looks like the Integration Operator for some Measure
actually is, uniquely



% ------------------------------------------------------------------------------
\subsection{Null Set}\label{sec:null_set}
% ------------------------------------------------------------------------------

$M = (X, \Sigma, \mu)$

$S \subset X$ such that $\mu(S) = 0$

a Null Set $N \subset \reals$ is a Set that can be Covered (\S\ref{sec:cover})
by a Countable Union of Intervals of arbitrarily small total length

a Complete Measure (\S\ref{sec:complete_measure}) is a Measure for which every
Subset of a Null Set is Measurable with Measure $0$

\begin{itemize}
  \item all Countable Sets are Null Sets
  \item Non-random Infinite Binary Random Sequences
    (\S\ref{sec:random_sequence}) form a maximal Constructive Null Set
  \item ...
\end{itemize}



% ------------------------------------------------------------------------------
\subsection{Distribution Function}\label{sec:distribution_function}
% ------------------------------------------------------------------------------

(wiki)

for a Measure $\mu$ on the Real Numbers with the Borel $\sigma$-algebra, i.e.
the Measure Space $(\reals, \struct{B}_\reals, \mu)$, the Function:
\[
  F_\mu : \reals \rightarrow \reals \cup \{+\infty, -\infty\}
\]
defined as:
\[
  F_\mu(t) = \begin{cases}
     \mu((0,t]) & \text{if} t \geq 0 \\
    -\mu((t,0]) & \text{if} t < 0 \\
  \end{cases}
\]
is the \emph{Right-continuous Distribution Function} of the Measure $\mu$

for example, the Lebesgue Measure (\S\ref{sec:lebesgue_measure}) $\lambda$ has
$\lambda((0,t]) = t - 0 = t$ and $-\lambda((t,0]) = -(0 - t) = t$, so
$F_\lambda(t) = t$

generalization of Cumulative Distribution Functions (Probability Theory
\S\ref{sec:cdf}), which have the boundary conditions
$\lim_t\rightarrow\infty F(t) = 0$ and $\lim_{t\rightarrow\infty}F(t) = 1$



% ------------------------------------------------------------------------------
\subsection{Topological Support}\label{sec:topological_support}
% ------------------------------------------------------------------------------

or \emph{Topological Support} or \emph{Spectrum} of a Measure $\mu$ on a
Measurable Topological Space $(\xspace{X}, Borel(\xspace{X}))$ is the largest
Closed Subset of $\xspace{X}$ for which every Open Neighborhood of every Point
of the Set has a Positive Measure

cf. Support of a Real-valued Function (\S\ref{sec:support})



% ------------------------------------------------------------------------------
\subsection{Measure-preserving Map}\label{sec:measure_preserving_map}
% ------------------------------------------------------------------------------

cf. Measure-preserving Dynamical Systems (\S\ref{sec:measure_preserving_system})

\url{https://golem.ph.utexas.edu/category/2018/09/a_categorical_look_at_random_v.html}:

$\cat{Prob}$ -- Category with Objects as Probability Spaces
(\S\ref{sec:probability_space}) and Morphisms are ``Almost-everywhere-equality
Equivalence Classes'' of Measure-preserving Maps

the fundamental ``objects'' of Probability Theory are the \emph{Morphisms} of
$\cat{Prob}$ and these Morphisms are \emph{Random Variables}
(\S\ref{sec:random_variable})



% ------------------------------------------------------------------------------
\subsection{Measure Convergence}\label{sec:measure_convergence}
% ------------------------------------------------------------------------------

cf. Stochastic Convergence (\S\ref{sec:stochastic_convergence})



% ------------------------------------------------------------------------------
\subsection{Signed Measure}\label{sec:signed_measure}
% ------------------------------------------------------------------------------

or \emph{Charge}



% ------------------------------------------------------------------------------
\subsection{Diffuse Measure}\label{sec:diffuse_measure}
% ------------------------------------------------------------------------------

or \emph{Non-atomic Measure}

no Atoms (\S\ref{sec:atom})



% ------------------------------------------------------------------------------
\subsection{Complex Measure}\label{sec:complex_measure}
% ------------------------------------------------------------------------------

the Set of all Complex Measures on a Measure Space forms a Vector Space (the Sum
of two Complex Measures is a Complex Measure and the Product of a Complex
Measure by a Complex Number is a Complex Measure)



% ------------------------------------------------------------------------------
\subsection{Pushforward Measure}\label{sec:pushforward_measure}
% ------------------------------------------------------------------------------

or \emph{Image Measure}

(wiki):

for a Measurable Function $f : X_1 \rightarrow X_2$ between Measurable Spaces
$(X_1, \Sigma_1)$, $(X_2, \Sigma_2)$, the \emph{Pushforward} of a Measure
$\mu : \Sigma_1 \rightarrow [0, \infty]$ by $f$ is the Measure
$f_*(\mu) : \Sigma_2 \rightarrow [0, \infty]$ given by:
\[
  (f_*(\mu))(B) = \mu(f^{-1}(B))
\]
for $B \in \Sigma_2$

the Pushforward Measure is also denoted by $f_\sharp\mu$

\fist a Probability Distribution (\S\ref{sec:probability_distribution}) is a
Pushforward Measure of a Random Variable (\S\ref{sec:random_variable})



% ------------------------------------------------------------------------------
\subsection{Complete Measure}\label{sec:complete_measure}
% ------------------------------------------------------------------------------

a Measure for which every Subset of a Null Set (\S\ref{sec:null_set}) is
Measurable with Measure $0$

\textbf{Maharam's Theorem}: \emph{
  Every Complete Measure Space is decomposable into a Measure on the Continuum
  and a Finite or Countable Counting Measure.
}



% ------------------------------------------------------------------------------
\subsection{Inner Measure}\label{sec:inner_measure}
% ------------------------------------------------------------------------------

% ------------------------------------------------------------------------------
\subsection{Outer Measure}\label{sec:outer_measure}
% ------------------------------------------------------------------------------

(wiki):

an \emph{Outer Measure} (or \emph{Exterior Measure}) on a Set $X$ is a Function
on the Power Set of $X$, $\varphi : 2^X \rightarrow [0, \infty]$, Satisfying:
\begin{enumerate}
  \item $\varphi(\varnothing) = 0$ -- \emph{Null Empty Set}
  \item $A,B \in X, A \subseteq B \Rightarrow \varphi(A) \leq \varphi(B)$ --
    \emph{Monotonicity}
  \item for a Sequence $\{A_j\}, j \in \{1, \ldots, \infty\}$ of Subsets of $X$,
    which may or may not be pairwise Disjoint:
    \[
      \varphi\Big(\bigcup_{i=1}^\infty A_i\Big) \leq \sum_j\varphi(A_j)
    \]
    -- \emph{Countable Subadditivity}
\end{enumerate}

a Subset $E \subseteq X$ is \emph{$\varphi$-measurable} (or
\emph{Carath\'eodory-measurable} by $\varphi$) if and only if for every Subset
$A \subseteq X$:
\[
  \varphi(A) = \varphi(A \cap E) + \varphi(A \cap \bar{E})
\]

$\textbf{Thm.}$ \emph{
  The $\varphi$-measurable Sets form a $\sigma$-algebra and $\varphi$ restricted
  to the Measurable Sets is a Countably Additive Complete Measure.
}

Geometric Measure Theory (\S\ref{sec:gmt})

Hausdorff Dimension (\S\ref{sec:hausdorff_dimension})



\subsubsection{Lebesgue Outer Measure}\label{sec:lebesgue_outer_measure}

for a Subset $E \subseteq \reals$ and for Intervals $I = [a,b]$ with lengths
$\ell(I) = b-a$, the \emph{Lebesgue Outer Measure} $\lambda^*(E)$ is defined as:
\[
  \lambda^*(E) = \inf \Big\{ \sum_k^\infty \ell(I_k) : (I_k)_{k \in \nats}
    \text{is a Sequence of Intervals with Open Boundaries with}
    E \subseteq \bigcup_{k=1}^\infty I_k \Big\}
\]

the Lebesgue Outer Measure reduces the Subsets $E \subseteq \reals$ to its Outer
Measure by \emph{Coverage} by Sets of Open Intervals-- each such Set of
Intervals $I$ \emph{Covers} $E$ in the sense that when the Intervals are
combined by Union, they Contain $E$; the total length of any Covering Interval
can \emph{overestimate} the Measure of $E$, since the Intervals may include
Points which are not in $E$; the Lebesgue Outer Measure is the Greatest Lower
Bound (Infimum \S\ref{sec:greatest_lowerbound}) of the lengths from all possible
Sets of Intervals-- i.e. it is the total length of Interval Sets which ``fit''
$E$ most tightly and do not overlap

the \emph{Lebesgue Measure} (\S\ref{sec:lebesgue_measure}) is defined on the
Lebesgue $\sigma$-algebra which is the collection of Subsets $E$ Satisfying the
\emph{Carath\'eodory Criterion} for every $A \subseteq \reals$:
\[
  \lambda^*(A) = \lambda^*(A \cap E) + \lambda^*(A \cap \bar{E})
\]
and the Lebesgue Measure $\lambda$ is given by the Lebesgue outer Measure:
$\lambda(E) = \lambda^*(E)$

this further requirement ensures that using a Set $E$ to Partition an arbitrary
Set $A$ by Set Difference of $A$ and $E$ results in Sets whose Outer Measures
Sum to the original Set $A$, i.e. the Sets $E$ are \emph{Lebesgue Measurable}



\subsubsection{Hausdorff Measure}\label{sec:hausdorff_measure}

\fist Geometric Measure Theory (\S\ref{sec:gmt})



% ------------------------------------------------------------------------------
\subsection{$\sigma$-finite Measure}\label{sec:sigma_finite}
% ------------------------------------------------------------------------------

$(X,\Sigma,\mu)$ is \emph{$\sigma$-finite} if there exists a Sequence
$(E_i)_{i \geq 1}$ in $M$ such that $\bigcup_{i} E_i = X$ and
$\mu(E_i) \leq \infty$.



\subsubsection{Counting Measure}\label{sec:counting_measure}

$\mu(A) = |A|$ for $A \subset \nats$

\textbf{Maharam's Theorem}: \emph{
  Every Complete Measure Space is decomposable into a Measure on the Continuum
  and a Finite or Countable Counting Measure.
}

\begin{itemize}
  \item Point Process (\S\ref{sec:point_process}) -- a Random Element
    (\S\ref{sec:random_variable}) whose values are ``Point Patterns'' or
    Locally-finite Counting Measures on a Set $S$
\end{itemize}



\subsubsection{Lebesgue Measure}\label{sec:lebesgue_measure}

\emph{Volume}

(wiki):

\emph{$n$-dimensional Volume}; standard way of assigning a Measure to Subsets of
$n$-dimensional Euclidean Space

Measure of Subsets of $\reals$

$m((a,b)) = b - a$

the \emph{Lebesgue Measure} is defined on the Lebesgue $\sigma$-algebra which is
the collection of Subsets $E \subseteq \reals$ Satisfying the
\emph{Carath\'eodory Criterion} for every $A \subseteq \reals$:
\[
  \lambda^*(A) = \lambda^*(A \cap E) + \lambda^*(A \cap \bar{E})
\]
and the Lebesgue Measure $\lambda$ is given by the Lebesgue Outer Measure
(\S\ref{sec:lebesgue_outer_measure}): $\lambda(E) = \lambda^*(E)$

the Lebesgue Outer Measure reduces the Subsets $E \subseteq \reals$ to its Outer
Measure by \emph{Coverage} by Sets of Open Intervals-- each such Set of
Intervals $I$ \emph{Covers} $E$ in the sense that when the Intervals are
combined by Union, they Contain $E$; the total length of any Covering Interval
can \emph{overestimate} the Measure of $E$, since the Intervals may include
Points which are not in $E$; the Lebesgue Outer Measure is the Greatest Lower
Bound (Infimum \S\ref{sec:greatest_lowerbound}) of the lengths from all possible
Sets of Intervals-- i.e. it is the total length of Interval Sets which ``fit''
$E$ most tightly and do not overlap

the further requirements of the Lebesgue Measure ensures that using a Set $E$ to
Partition an arbitrary Set $A$ by Set Difference of $A$ and $E$ results in Sets
whose Outer Measures Sum to the original Set $A$, i.e. the Sets $E$ are
\emph{Lebesgue Measurable}

\fist cf. Lebesgue Integral (\S\ref{sec:lebesgue_integral})

construction by application of Carath\'eodory's Extension Theorem (TODO)

\textbf{Thm.} \emph{
  There is no analogue of Lebesgue Measure on Infinite-dimensional Banach Space
  (\S\ref{sec:banach_space}).
}

the Borel Measure (\S\ref{sec:lebesgue_measure}) agrees with the Lebesgue
Measure on the Sets for which it is defined, but there are more
Lebesgue-measurable Sets than there are Borel-measurable Sets

the existence of Sets that are not Lebesgue-measurable is a consequence of the
Axiom of Choice

if the Axiom of Choice is accepted, according to the \emph{Vitali Theorem},
there are Uncountably many \emph{Vitali Sets} which are not Lebesgue Measurable
\emph{Solovay Model}: assuming the existence of an \emph{Inaccessible Cardinal}
(\S\ref{sec:inaccessible_cardinal}), a Model of ZF Set Theory without the Axiom
of Choice (but the Axiom of Countable choice holds) in which all Sets of Real
Numbers are Lebesgue Measurable

a Bounded Function (\S\ref{sec:bounded_function}) on a Compact Interval $[a,b]$
is Riemann Integrable (\S\ref{sec:integrable_function}) if and only if it is
Continuous (\S\ref{sec:continuous_function}) ``Almost Everywhere'', i.e. Set of
Points of Discontinuity has Measure Zero

the Distribution Function (\S\ref{sec:distribution_function}) of the Lebesgue
Measure is just the Identity Function $F_\lambda(t) = t$ for all $t \in \reals$



\paragraph{Haar Measure}\label{sec:haar_measure}\hfill



\subsubsection{Dirac Measure}\label{sec:dirac_measure}

cf. Dirac Delta Function (\S\ref{sec:dirac_delta})

a Point Process (\S\ref{sec:point_process}) can be defined as a Random Measure
(\S\ref{sec:random_measure}) of the form $\mu = \sum_{n=1}^N \delta_{X_n}$ where
$\delta$ is the Dirac Measure



% ------------------------------------------------------------------------------
\subsection{Borel Measure}\label{sec:borel_measure}
% ------------------------------------------------------------------------------

any Measure defined on the Borel Sets of a Borel Algebra
(\S\ref{sec:borel_algebra})

the Borel Measure agrees with the Lebesgue Measure
(\S\ref{sec:lebesgue_measure}) on the Sets for which it is defined, but there
are more Lebesgue-measurable Sets than there are Borel-measurable Sets

the Borel Measure is Translation-invariant but not Complete

Moderate Measure

\fist a Borel Measure always restricts to a \emph{Valuation}
(\S\ref{sec:domain_valuation})



\subsubsection{Regular Borel Measure}\label{sec:regualr_borel}\hfill

\paragraph{Lebesgue-Stieltjes Measure}\label{sec:lebesgue_stieltjes}\hfill

Radon Integral (\S\ref{sec:radon_integral}) -- ordinary Lebesgue Integral with
respect to the Lebesgue-Stieltjes Measure

%FIXME: are the lebesgue-stieltjes measure and the radon measure the same ???



\paragraph{Radon Measure}\label{sec:radon_measure}\hfill

both Regular and Locally Finite (\S\ref{sec:locally_finite_measure})

\fist Rectifiable Sets (Geometric Measure Theory \S\ref{sec:rectifiable_set})

%FIXME: are the lebesgue-stieltjes measure and the radon measure the same ???



% ------------------------------------------------------------------------------
\subsection{Locally Finite Measure}\label{sec:locally_finite_measure}
% ------------------------------------------------------------------------------

% ------------------------------------------------------------------------------
\subsection{Similarity Measure}\label{sec:similarity_measure}
% ------------------------------------------------------------------------------

\fist cf. Similarity Transformation (\S\ref{sec:similarity_transformation})

Cross-correlation (Functional Analysis \S\ref{sec:cross_correlation})



% ------------------------------------------------------------------------------
\subsection{Cylinder Set Measure}\label{sec:cylinder_set_measure}
% ------------------------------------------------------------------------------

kind of prototype for a Measure on Infinite-dimensional Vector Spaces
(\S\ref{sec:infinite_dimensional_vectorspace})



\subsubsection{Wiener Measure}\label{sec:wiener_measure}\hfill

the collection of Functions that are Differentiable at a single point of
$[0,1]$ has Wiener Measure $0$, even when taking Finite-dimensional ``slices''
of the Vector Space of Continuous Functions $C([0,1];\reals)$, in the sense
that the Nowhere-differentiable Functions (\S\ref{sec:nowhere_differentiable})
form a Prevalent Subset (\S\ref{sec:prevalent_set}) of $C([0,1]; \reals)$



% ------------------------------------------------------------------------------
\subsection{Tightness}\label{sec:tightness}
% ------------------------------------------------------------------------------



% ==============================================================================
\section{Measure Space}\label{sec:measure_space}
% ==============================================================================

$(X,\Sigma,\mu)$

\fist cf. \emph{Measurable Space} (\S\ref{sec:measurable_space}) -- a Set
together $X$ with a $\sigma$-algebra $\Sigma$ on the Set: $(X,\Sigma)$

Abstract Axiomatic Structure needed to define a Theory of \emph{Integration}
(\S\ref{sec:integral_calculus})

cf. \emph{Metric Spaces} (\S\ref{sec:metric_space}) -- Abstract Axiomatic
Structure needed to define Theory of \emph{Differentiation}
(\S\ref{sec:differential_calculus})

A Measure Space with a Probability Measure (\S\ref{sec:probability_measure}) is
a \emph{Probability Space} (\S\ref{sec:probability_space}). A Probability
Measure is required to assign a value of $1$ to the entire Probability Space.

For $E,F \in \Sigma$:
\[
  \mu(E) = \mu(E \cap F) + \mu(E \cap F^c)
\]

For $E,F \in \Sigma$ such that $F \subset E$:
\[
  \mu(F) \leq \mu(E)
\]

Countable Additivity (or $\sigma$-additivity): %FIXME

For $E_i \in \Sigma$ (Countable Collections):
\[
  \mu(\bigcup_i E_i) \leq \sum_i \mu(E_i)
\]



% ==============================================================================
\section{Non-measurable Set}\label{sec:nonmeasurable_set}
% ==============================================================================

% ==============================================================================
\section{Almost Everywhere}\label{sec:almost_everywhere}
% ==============================================================================

a Generic Property (\S\ref{sec:generic_property}) is one that holds Almost
Everywhere



% ==============================================================================
\section{Negligible Function}\label{sec:negligible_function}
% ==============================================================================

% ==============================================================================
\section{Negligible Set}\label{sec:negligible_set}
% ==============================================================================

Meagre Set (\S\ref{sec:meagre_set}): a Negligible Subset of a Topological Space



% ==============================================================================
\section{Generic Property}\label{sec:generic_measure_property}
% ==============================================================================

a Property that holds Almost Everywhere

opposite of a Negligible Set

cf. Generic Property (Algebraic Geometry \S\ref{sec:generic_property})



% ==============================================================================
\section{Prevalent Set}\label{sec:prevalent_set}
% ==============================================================================

the collection of Functions that are Differentiable at a single point of
$[0,1]$ has Wiener Measure (\S\ref{sec:wiener_measure}) $0$, even when taking
Finite-dimensional ``slices'' of $C([0,1];\reals)$, in the sense that the
Nowhere-differentiable Functions (\S\ref{sec:nowhere_differentiable}) form a
Prevalent Subset of $C([0,1]; \reals)$



% ==============================================================================
\section{Content}\label{sec:content}
% ==============================================================================

(wiki): a Set Function like Measure (\S\ref{sec:measure}) that need not be
Countably Additive, but must only be Finitely Additive



% ==============================================================================
\section{Geometric Measure Theory}\label{sec:gmt}
% ==============================================================================

extends Differential Geometry (\S\ref{sec:differential_geometry}) to a larger
calss of Surfaces that are not necessarily Smooth

\fist Fractal Geometry (Part \ref{part:fractal_geometry})

Outer Measure (\S\ref{sec:outer_measure}),
Hausdorff Measure (\S\ref{sec:hausdorff_measure})

Plateau's Problem (Calculus of Variations \S\ref{sec:variational_calculus}) --
find the Minimal Surface with a given Boundary

Current (Differential Topology \S\ref{sec:current}) --
generalization of the concept of Oriented Manifolds, possibly with Boundary;
Functional (\S\ref{sec:functional}) on the Space of Compactly Supported
Differential $k$-forms (\S\ref{sec:differential_form}) on a Smooth Manifold $M$;
behaves like Schwartz Distributions (\S\ref{sec:schwartz_space}) on a Space of
Differential Forms, but in a ``geometric setting'' they can represent
Integration over a Submanifold, generalizing the Dirac Delta Function
(\S\ref{sec:dirac_delta}), or more generally Directional Derivatives of Delta
Functions spread out along Subsets of $M$



% ------------------------------------------------------------------------------
\subsection{Rectifiable Set}\label{sec:rectifiable_set}
% ------------------------------------------------------------------------------

``Piece-wise Smooth Set''

Radon Measure (\S\ref{sec:radon_measure})



% ------------------------------------------------------------------------------
\subsection{Caccioppoli Set}\label{sec:caccioppoli_set}
% ------------------------------------------------------------------------------

Set of Locally Finite Perimeter

a Set is Caccioppoli if its Characteristic Function (Indicator Function
\S\ref{sec:indicator_function}) is a Function of a Bounded Variation
(\S\ref{sec:bounded_variation})



% ------------------------------------------------------------------------------
\subsection{Current}\label{sec:current}
% ------------------------------------------------------------------------------

Differential Topology (\S\ref{sec:differential_topology})

generalization of concept of Oriented Manifolds (\S\ref{sec:oriented_manifold})
possibly with Boundary

Functional (\S\ref{sec:functional}) on the Space of Compactly Supported
Differential $k$-forms (\S\ref{sec:differential_form}) on a Smooth Manifold $M$

behaves like Schwartz Distributions (\S\ref{sec:schwartz_space}) on a Space of
Differential Forms, but in a ``geometric setting'' they can represent
Integration over a Submanifold, generalizing the Dirac Delta Function
(\S\ref{sec:dirac_delta}), or more generally Directional Derivatives of Delta
Functions spread out along Subsets of $M$

\fist Analytic solution to Orientable Plateau's Problem; Geometric Measure
Theory (\S\ref{sec:gmt})




\subsubsection{Integral Current}\label{sec:integral_current}

\paragraph{Flat Convergence}\label{sec:flat_convergence}\hfill



% ------------------------------------------------------------------------------
\subsection{Co-area Formula}\label{sec:coarea_formula}
% ------------------------------------------------------------------------------

% ------------------------------------------------------------------------------
\subsection{Isoperimetric Inequality}\label{sec:isoperimetric_inequality}
% ------------------------------------------------------------------------------

% ==============================================================================
\section{Transportation Theory}\label{sec:transportation_theory}
% ==============================================================================

%FIXME: move to optimization?
