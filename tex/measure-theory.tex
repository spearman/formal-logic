%%%%%%%%%%%%%%%%%%%%%%%%%%%%%%%%%%%%%%%%%%%%%%%%%%%%%%%%%%%%%%%%%%%%%%%%%%%%%%%%
%%%%%%%%%%%%%%%%%%%%%%%%%%%%%%%%%%%%%%%%%%%%%%%%%%%%%%%%%%%%%%%%%%%%%%%%%%%%%%%%
\part{Measure Theory}\label{part:measure_theory}
%%%%%%%%%%%%%%%%%%%%%%%%%%%%%%%%%%%%%%%%%%%%%%%%%%%%%%%%%%%%%%%%%%%%%%%%%%%%%%%%
%%%%%%%%%%%%%%%%%%%%%%%%%%%%%%%%%%%%%%%%%%%%%%%%%%%%%%%%%%%%%%%%%%%%%%%%%%%%%%%%

\fist Geometric Measure Theory (\S\ref{sec:geometric_measure_theory})

\fist Probability Theory (Part \ref{part:probability_theory})



% ==============================================================================
\section{$\sigma$-algebra}\label{sec:sigma_algebra}
% ==============================================================================

A \emph{$\sigma$-algebra} on a Set $X$ is a Non-empty Set $\mathcal{A}
\subseteq 2^X$ that is Closed under Set Operations of Complement
(\S\ref{sec:absolute_complement}) and Countable Union (\S\ref{sec:set_union}).

\begin{enumerate}
  \item $\varnothing \in \mathcal{A}$
  \item $X \in \mathcal{A}$
  \item $(E_i)_{i \geq 1} \in \mathcal{A} \Rightarrow
    \bigcap_{i \geq 1} E_i \in \mathcal{A}$
\end{enumerate}

$\Sigma$ Subsets of $X$

$\{ \varnothing, X \}$

Collection of $\sigma$-algebras $\{ \Sigma_\alpha : \alpha \in \class{A} \}$

$\mathcal{F}$ Family of Subsets of $X$, $\sigma(\mathcal{F})$ is the
$\sigma$-algebra Generated by $\mathcal{F}$

$\sigma(\varnothing) = \{ \varnothing, X \}$

\fist $\sigma$-additivity Axiom (\S\ref{sec:probability_axioms}): the
Probability of a Countable Sequence of Disjoint Sets (Mutually Exclusive Events
\S\ref{sec:mutually_exclusive}) is equal to the Sum of the individual
Probabilities



% ------------------------------------------------------------------------------
\subsection{Borel Algebra}\label{sec:borel_algebra}
% ------------------------------------------------------------------------------

Collection of all Borel Sets (\S\ref{sec:borel_set}) on a Topological Space $X$
forms the Borel Algebra on $X$ which is the smallest $\sigma$-algebra
containing all Open Sets (or equivalently all Closed Sets) of $X$

any Measure defined on the Borel Sets is called a Borel Measure
(\S\ref{sec:borel_measure})

on $\reals$: $\struct{B}_\reals$ is equal to the Minimal $\sigma$-algebra
Generated by the Closed Sets $E_2 = \{ [a,b] \;|\; a < b \}$.

Measurable Sets (\S\ref{sec:measurable_set}) on the Real Line are ``iterated''
Countable Unions and Intersections of Borel Sets



\subsubsection{Borel Hierarchy}\label{sec:borel_hierarchy}\hfill

\fist Descriptive Set Theory (\S\ref{sec:descriptive_set_theory})



% ==============================================================================
\section{Measure}\label{sec:measure}
% ==============================================================================

Set $X$

$(X,\Sigma)$ -- $\sigma$-algebra (\S\ref{sec:sigma_algebra}) on $X$

A \emph{Measure} on $(X,\Sigma)$:
\[
  \mu : \Sigma \rightarrow [0,\infty]
\]

\begin{enumerate}
  \item $\mu(\varnothing) = 0$

  \item For $(E_i)_{i \geq 1} \in \Sigma$ with $E_i \cap E_j =
    \varnothing$ when $i \neq j$, then:
    \[
      \mu (\bigcup_i E_i) = \sum_i \mu(E_i)
    \]
\end{enumerate}

$(X,\Sigma,\mu)$ is called a \emph{Measure Space} (\S\ref{sec:measure_space})
and a Measure Space with a Probability Measure
(\S\ref{sec:probability_measure}) is a \emph{Probability Space}
(\S\ref{sec:probability_space}); a Probability Measure must assign a value of
$1$ to the entire Probability Space.

Measurable

\emph{Countable Additivity}

\fist \emph{Unit Measure Axiom} (Kolmogorov Axioms
\S\ref{sec:probability_axioms})



% ------------------------------------------------------------------------------
\subsection{Measure Space}\label{sec:measure_space}
% ------------------------------------------------------------------------------

$(X,\Sigma,\mu)$

A Measure Space with a Probability Measure (\S\ref{sec:probability_measure}) is
a \emph{Probability Space} (\S\ref{sec:probability_space}). A Probability
Measure must assign a value of $1$ to the entire Probability Space.

For $E,F \in \Sigma$:
\[
  \mu(E) = \mu(E \cap F) + \mu(E \cap F^c)
\]

For $E,F \in \Sigma$ such that $F \subset E$:
\[
  \mu(F) \leq \mu(E)
\]

Countable Additivity (or $\sigma$-additivity): %FIXME

For $E_i \in \Sigma$ (Countable Collections):
\[
  \mu(\bigcup_i E_i) \leq \sum_i \mu(E_i)
\]



\subsubsection{Null Set}\label{sec:null_set}\hfill

$M = (X, \Sigma, \mu)$

$S \subset X$ such that $\mu(S) = 0$



% ------------------------------------------------------------------------------
\subsection{$\sigma$-finite Measure}\label{sec:sigma_finite}
% ------------------------------------------------------------------------------

$(X,\Sigma,\mu)$ is \emph{$\sigma$-finite} if there exists a Sequence
$(E_i)_{i \geq 1}$ in $M$ such that $\bigcup_{i} E_i = X$ and
$\mu(E_i) \leq \infty$.



\subsubsection{Counting Measure}\label{sec:counting_measure}\hfill

$\mu(A) = |A|$ for $A \subset \nats$



\subsubsection{Lebesgue Measure}\label{sec:lebesgue_measure}\hfill

Measure of Subsets of $\reals$

$m((a,b)) = b - a$

a Bounded Function (\S\ref{sec:bounded_function}) on a Compact Interval $[a,b]$
is Riemann Integrable (\S\ref{sec:integrable_function}) if and only if it is
Continuous (\S\ref{sec:continuous_function}) ``Almost Everywhere'', i.e. Set of
Points of Discontinuity has Measure Zero



% ------------------------------------------------------------------------------
\subsection{Locally Finite Measure}\label{sec:locally_finite_measure}
% ------------------------------------------------------------------------------

% ------------------------------------------------------------------------------
\subsection{Borel Measure}\label{sec:borel_measure}
% ------------------------------------------------------------------------------

any Measure defined on the Borel Sets of a Borel Algebra
(\S\ref{sec:borel_algebra})

Moderate Measure



\subsubsection{Regular Borel Measure}\label{sec:regualr_borel}\hfill

\paragraph{Radon Measure}\label{sec:radon_measure}\hfill

both Regular and Locally Finite (\S\ref{sec:locally_finite_measure})



% ------------------------------------------------------------------------------
\subsection{Probability Measure}\label{sec:probability_measure}
% ------------------------------------------------------------------------------

A Measure Space with a Probability Measure is a Probability Space
(\S\ref{sec:probability_space}). A Probability Measure must assign the Value
$1$ to the entire Probability Space

Probability Measure Function
(\S\ref{sec:probability_measure_function})



% ------------------------------------------------------------------------------
\subsection{Similarity Measure}\label{sec:similarity_measure}
% ------------------------------------------------------------------------------

\fist cf. Similarity Transformation (\S\ref{sec:similarity_transformation})

Cross-correlation (Functional Analysis \S\ref{sec:cross_correlation})



% ------------------------------------------------------------------------------
\subsection{Outer Measure}\label{sec:outer_measure}
% ------------------------------------------------------------------------------

\subsubsection{Hausdorff Measure}\label{sec:hausdorff_measure}\hfill

\fist Geometric Measure Theory (\S\ref{sec:geometric_measure_theory})



% ------------------------------------------------------------------------------
\subsection{Cylinder Set Measure}\label{sec:cylinder_set_measure}
% ------------------------------------------------------------------------------

kind of prototype for a Measure on Infinite-dimensional Vector Spaces
(\S\ref{sec:infinite_dimensional_vectorspace})



\subsubsection{Wiener Measure}\label{sec:wiener_measure}\hfill

the collection of Functions that are Differentiable at a single point of
$[0,1]$ has Wiener Measure $0$, even when taking Finite-dimensional ``slices''
of the Vector Space of Continuous Functions $C([0,1];\reals)$, in the sense
that the Nowhere-differentiable Functions (\S\ref{sec:nowhere_differentiable})
form a Prevalent Subset (\S\ref{sec:prevalent_set}) of $C([0,1]; \reals)$



% ==============================================================================
\section{Measurable Function}\label{sec:measurable_function}
% ==============================================================================

% ==============================================================================
\section{Measurable Set}\label{sec:measurable_set}
% ==============================================================================

Measurable Sets on the Real Line are ``iterated'' Countable Unions and
Intersections of Borel Sets (\S\ref{sec:borel_set})

Solovay's Model: all Subsets of the Reals are Measurable (excluding Uncountable
Choice)



% ==============================================================================
\section{Non-measurable Set}\label{sec:nonmeasurable_set}
% ==============================================================================

% ==============================================================================
\section{Almost Everywhere}\label{sec:almost_everywhere}
% ==============================================================================

a Generic Property (\S\ref{sec:generic_property}) is one that holds Almost
Everywhere



% ==============================================================================
\section{Negligible Function}\label{sec:negligible_function}
% ==============================================================================

% ==============================================================================
\section{Negligible Set}\label{sec:negligible_set}
% ==============================================================================

Meagre Set (\S\ref{sec:meagre_set}): a Negligible Subset of a Topological Space



% ==============================================================================
\section{Generic Property}\label{sec:generic_measure_property}
% ==============================================================================

a Property that holds Almost Everywhere

opposite of a Negligible Set

cf. Generic Property (Algebraic Geometry \S\ref{sec:generic_property})



% ==============================================================================
\section{Prevalent Set}\label{sec:prevalent_set}
% ==============================================================================

the collection of Functions that are Differentiable at a single point of
$[0,1]$ has Wiener Measure (\S\ref{sec:wiener_measure}) $0$, even when taking
Finite-dimensional ``slices'' of $C([0,1];\reals)$, in the sense that the
Nowhere-differentiable Functions (\S\ref{sec:nowhere_differentiable}) form a
Prevalent Subset of $C([0,1]; \reals)$
