%%%%%%%%%%%%%%%%%%%%%%%%%%%%%%%%%%%%%%%%%%%%%%%%%%%%%%%%%%%%%%%%%%%%%%
%%%%%%%%%%%%%%%%%%%%%%%%%%%%%%%%%%%%%%%%%%%%%%%%%%%%%%%%%%%%%%%%%%%%%%
\part{Synthetic Geometry}\label{part:synthetic_geometry}
%%%%%%%%%%%%%%%%%%%%%%%%%%%%%%%%%%%%%%%%%%%%%%%%%%%%%%%%%%%%%%%%%%%%%%
%%%%%%%%%%%%%%%%%%%%%%%%%%%%%%%%%%%%%%%%%%%%%%%%%%%%%%%%%%%%%%%%%%%%%%

``Axiomatic Geometry''



% ====================================================================
\section{Euclidean Geometry}\label{sec:euclidean_geometry}
% ====================================================================

$E^n$ -- $n$-dimensional Euclidean Space (Euclidean Space
\S\ref{sec:euclidean_space}) with Cartesian Coordinates is Modelled by
$\reals^n$ ($n$-dimensional Real Coordinate Space
\S\ref{sec:real_coordinate_space})

\fist cf. Homogenous Coordinates (\S\ref{sec:homogenous_coordinate})
in Projective Geometry (\S\ref{sec:projective_geometry})

Riemannian Manifold (\S\ref{sec:riemannian_manifold}) of Constant Vanishing
Sectional Curvature (\S\ref{sec:sectional_curvature})

\emph{Euclidean Group} (\S\ref{sec:euclidean_group}) -- Group of all
Isometries $ISO(n)$ or $E(n)$; makes Euclidean Geometry a case of
Klein Geometry (\S\ref{sec:klein_geometry})



% --------------------------------------------------------------------
\subsection{Half-space}\label{sec:half_space}
% --------------------------------------------------------------------

Hyperbolic Spaces (\S\ref{sec:hyperbolic_space})



% --------------------------------------------------------------------
\subsection{Convex Geometry}\label{sec:convex_geometry}
% --------------------------------------------------------------------

%FIXME:

Simplex



\subsubsection{Affine Hull}\label{sec:affine_hull}

\subsubsection{Convex Hull}\label{sec:convex_hull}

Vertex representation of a Convex Polytope (\S\ref{sec:convex_polytope})

a Convex Polyhedra (\S\ref{sec:convex_polyhedra}) is the Convex Hull of
finitely many Points not all lying in the same Plane



% --------------------------------------------------------------------
\subsection{Rigid Transformation}\label{sec:rigid_transformation}
% --------------------------------------------------------------------

A \emph{Rigid Transformation} or \emph{Euclidean Isometry} is an Isometry
(\S\ref{sec:isometry}) associated with the Euclidean Distance Metric of a
Euclidean Space.

The Euclidean Group (\S\ref{sec:euclidean_group}) $\mathrm{ISO}(n)$ is the
Symmetry Group of $n$-dimensional Euclidean Space with Rigid Transformations as
its Elements.

cf. Affine Transformations (\S\ref{sec:affine_transformation}), Affine Group
(\S\ref{sec:affine_group})



\subsubsection{Translation}\label{sec:translation}

Affine Transformation (\S\ref{sec:affine_transformation})

\fist Periodic Functions (\S\ref{sec:periodic_function}) are Functions with
Graphs that exhibit Translational Symmetry (\S\ref{sec:symmetry_group})



\subsubsection{Reflection}\label{sec:reflection}

Affine Transformation (\S\ref{sec:affine_transformation})

Isometry (\S\ref{sec:isometry})

Reflection Groups (\S\ref{sec:reflection_group}) are Discrete Groups Generated
by a Set of Reflections of a Finite-dimensional Euclidean Space
(\S\ref{sec:reflection})

The Coxeter Groups (\S\ref{sec:coxeter_group}) are precisely the Finite
Euclidean Reflection Groups



\subsubsection{Rotation}\label{sec:rotation}

Affine Transformation (\S\ref{sec:affine_transformation})



% --------------------------------------------------------------------
\subsection{Similarity Transformation}\label{sec:similarity_transformation}
% --------------------------------------------------------------------

in Projective Geometry (\S\ref{sec:projective_geometry}) a Similarity
Transformation fixes a given Elliptic Involution (FIXME: clarify)

in Projective Geometry a Homothety (\S\ref{sec:homothety}) is a Similarity
Transformation that leaves the Line at Infinity Pointwise Invariant

\fist cf. Similarity Measure (\S\ref{sec:similarity_measure})



% ====================================================================
\section{Higher-dimensional Geometry}\label{sec:higher_geometry}
% ====================================================================

% --------------------------------------------------------------------
\subsection{Polytope}\label{sec:polytope}
% --------------------------------------------------------------------

\begin{itemize}
  \item Polygon (\S\ref{sec:polygon}) -- 2D
  \item Polyhedra (\S\ref{sec:polyhedra}) -- 3D
  \item Polychoron (Polycell \S\ref{sec:polychoron}) -- 4D
\end{itemize}



\subsubsection{Convex Polytope}\label{sec:convex_polytope}

Vertex representation is a Convex Hull (\S\ref{sec:convex_hull})

may be defined as an intersection of a finite number of Half-spaces

\fist Polyhedral Combinatorics (\S\ref{sec:polyhedral_combinatorics})



\subsubsection{Uniform Polytope}\label{sec:uniform_polytope}

\paragraph{Regular Polytope}\label{sec:regular_polytope}\hfill

Rectification

Truncation (\S\ref{sec:truncation_operator})



\paragraph{Truncation}\label{sec:truncation_operator}\hfill

\begin{enumerate}
  \item Truncation -- 2D (Polygons) and higher
  \item Cantellation (Rectified Rectification) -- 3D (Polyhedra) and higher
\end{enumerate}



\subsubsection{Star Polytope}\label{sec:star_polytope}

\subsubsection{Simplicial Polytope}\label{sec:simplicial_polytope}

all Facets are Simplices

Deltahedron (\S\ref{sec:deltahedron})



\subsubsection{Net}\label{sec:polytope_net}

\fist not to be confused with Topological Nets (\S\ref{sec:net})

\emph{D\"urer's Conjecture}: does every Convex Polyhedron have a Net?

Ghomi (2014): every Convex Polyhedron admits a net after an Affine
Transformation

it is known that every Convex Uniform $4$-polytope can be cut along
Two-dimensional Faces shared by its Three-dimensional Facets and unfolded into
a single non-overlapping Polyhedron ``Net'' in $3$ Dimensions




% --------------------------------------------------------------------
\subsection{Polygon}\label{sec:polytope}
% --------------------------------------------------------------------

2D Polytope



% --------------------------------------------------------------------
\subsection{Polyhedra}\label{sec:polyhedra}
% --------------------------------------------------------------------

3D ``Solid''

Nets of Polyhedra (\S\ref{sec:polytope_net})

\emph{D\"urer's Conjecture}: does every Convex Polyhedron have a Net?

Ghomi (2014): every Convex Polyhedron admits a net after an Affine
Transformation



\subsubsection{Convex Polyhedra}\label{sec:convex_polyhedra}

Convex Hull (\S\ref{sec:convex_hull}) of finitely many Points not all on the
same Plane

\fist Polyhedral Combinatorics (\S\ref{sec:polyhedral_combinatorics})



\paragraph{Platonic Solid}\label{sec:platonic_solid}\hfill

\paragraph{Archimedian Solid}\label{sec:archimedian_solid}\hfill

\paragraph{Johnson Solid}\label{sec:johnson_solid}\hfill

Non-uniform Convex Polyhedron with Regular Polygons for faces

\begin{itemize}
  \item Pyramids
  \item Cupolae and Rotundae
  \item Elongated and Gyroelongated Pyramids
  \item Bipyramids, Elongated Bipyramids, Gyroelongated Bipyramids
  \item Elongated Cupolae and Elongated Rotundae
  \item Bicupolae
  \item Bianticupolae
  \item Cupola-rotundae and Birodunta
  \item Triangular Hebesphenorotunda -- only Johnson Solid with Faces of 3, 4,
    5, and 6 Sides
  \item ... MORE
\end{itemize}



\paragraph{Pyramid}\label{sec:pyramid}\hfill

\paragraph{Prism}\label{sec:prism}\hfill

\paragraph{Antiprism}\label{sec:antiprism}\hfill

\paragraph{Rotunda}\label{sec:rotunda}\hfill

\paragraph{Cupola}\label{sec:cupola}\hfill



\subsubsection{Deltahedron}\label{sec:deltahedron}

Simplicial Polytope (\S\ref{sec:simplicial_polytope})



\subsubsection{Uniform Polyhedra}\label{sec:uniform_polyhedra}

\fist Uniform Polytope (\S\ref{sec:uniform_polytope})



% --------------------------------------------------------------------
\subsection{Polychoron}\label{sec:polychoron}
% --------------------------------------------------------------------

A \emph{Polycell} or \emph{Polychoron} is a $4$-polytope.



\subsubsection{Uniform Polychoron}\label{sec:uniform_polychoron}

it is known that every Convex Uniform $4$-polytope can be cut along
Two-dimensional Faces shared by its Three-dimensional Facets and unfolded into
a single non-overlapping Polyhedron ``Net'' in $3$ Dimensions



% ====================================================================
\section{Non-euclidean Geometry}\label{sec:noneuclidean_geometry}
% ====================================================================

some applets for exploring Non-euclidean Geometries (listed on
\url{https://www.maa.org/press/periodicals/loci/rethinking-pythagoras-0}):

\url{http://www.cs.unm.edu/~joel/NonEuclid/NonEuclid.html}

\url{http://merganser.math.gvsu.edu/easel/applet.html}

\url{https://www.cinderella.de/tiki-index.php}

\url{http://homepages.gac.edu/~hvidsten/explorer/}



% --------------------------------------------------------------------
\subsection{Hyperbolic Geometry}\label{sec:hyperbolic_geometry}
% --------------------------------------------------------------------

Riemannian Manifold (\S\ref{sec:riemannian_manifold}) of Constant Negative
Sectional Curvature (\S\ref{sec:sectional_curvature})



\subsubsection{Hyperbolic Space}\label{sec:hyperbolic_space}

Homogenous Space (\S\ref{sec:homogenous_space}) with Constant Negative
Sectional Curvature (\S\ref{sec:sectional_curvature})

Upper Half-space (\S\ref{sec:half_space})



% --------------------------------------------------------------------
\subsection{Elliptic Geometry}\label{sec:elliptic_geometry}
% --------------------------------------------------------------------

Riemannian Manifold (\S\ref{sec:riemannian_manifold}) of Constant Positive
Sectional Curvature (\S\ref{sec:sectional_curvature})



% ====================================================================
\section{Affine Space}\label{sec:affine_space}
% ====================================================================

An \emph{Affine Space} generalizes properties of Euclidean Spaces
(\S\ref{sec:euclidean_space}) to be \emph{independent} of the concepts of
Distance (cf. Norm \S\ref{sec:norm}) and Measure of Angles (cf. Rotation
\S\ref{sec:rotation}), keeping only properties relating to
\emph{Parallelism}

Affine Space is a Homogenous Space (\S\ref{sec:homogenous_space}) for its
Symmetry Groups (\S\ref{sec:symmetry_groups})

Manifolds (\S\ref{sec:manifold}) built by ``gluing together'' Charts
(\S\ref{sec:chart}, Open Subsets of Real Affine Spaces)

cf. Algebraic Varieties (\S\ref{sec:algebraic_variety}) built by ``gluing
together'' Affine Varities (\S\ref{sec:affine_variety})



% --------------------------------------------------------------------
\subsection{Affine Transformation}\label{sec:affine_transformation}
% --------------------------------------------------------------------

(or \emph{Affine Map} or \emph{Affinity}) is a Function between Affine
Spaces which preserves Points, Straight Lines, and Planes, and Sets of
Parallel Lines remain Parallel

preserves Ratios of Distances betwen Points lying on a Straight Line

does not necessarily preserve Angles between Lines or Distances
between Points

for Affine Spaces $X$ and $Y$

\fist Affine Group (\S\ref{sec:affine_group})

\fist All Linear Transformations (\S\ref{sec:linear_transformation}) are
Affine, but not every Affine Transformation is Linear; Affine Transformations
are not required to preserve the Zero Point in a Linear Space (FIXME: clarify)

Affine Transformations:
\begin{itemize}
\item Translation (\S\ref{sec:translation})
\item Rotation (\S\ref{sec:rotation})
\item Reflection (\S\ref{sec:reflection})
\item Scaling
\item Homothety (Homogenous Dilation or Central Similarity)
\item Similarity Transformation (???)
\item Shear Mapping (???)
\end{itemize}
and combinations of the above in any combination and sequence

cf. Affine Logic (\S\ref{sec:affine_logic})



% --------------------------------------------------------------------
\subsection{Half-space}\label{sec:halfspace}
% --------------------------------------------------------------------

either of two parts into which a Hyperplane (\S\ref{sec:hyperplane}) divides an
Affine Space

\begin{itemize}
  \item Ray -- 1D
  \item Half-plane -- 2D
  \item Half-space -- 3D
\end{itemize}



% --------------------------------------------------------------------
\subsection{Homothety}\label{sec:homothety}
% --------------------------------------------------------------------

\emph{Scale Transformation}

Linear Transformation (\S\ref{sec:linear_transformation}) of an Affine Space
determined by a Homothety Center Point $S$ and a Nonzero Scalar $\lambda$
called the \emph{Ratio} fixing $S$ and sending any Point $M$ to a Point $N$
such that $\vec{SN}$ is on the same Line as $\vec{SM}$ but \emph{Scaled} by a
factor of $\lambda$:
\[
  M \mapsto S + \lambda\vec{SM}
\]

\fist Projective Geometry (\S\ref{sec:projective_geometry}): a Homothety is a
Similarity Transformation (\S\ref{sec:similarity_transformation}) that leaves
the Line at Infinity Pointwise Invariant

\fist Scale Invariance (\S\ref{sec:scale_invariance})



% ====================================================================
\section{Projective Geometry}\label{sec:projective_geometry}
% ====================================================================

\emph{Homogenous Coordinates} (or \emph{Projective Coordinates}
\S\ref{sec:homogenous_coordinate}) -- Points (including Points at Infinity) can
be represented using Finite Coordinates; allows for Affine Transformations
(\S\ref{sec:affine_transformation})

cf. Cartesian Coordinates (\S\ref{sec:cartesian_coordinate}) in
Euclidean Geometry (\S\ref{sec:euclidean_geometry})

``Subsumes'' Euclidean, Elliptic, and Hyperbolic Geometry: Projective
Plane (\S\ref{sec:projective_plane}) includes Euclidean, Elliptic, and
Hyperbolic Planes and its Symmetry Group contains their Symmetry
Groups

``all Projective Geometries are exercises in Linear Algebra'' (E.
Kmett interview -- source?) %FIXME

a Homothety (\S\ref{sec:homothety}) in Projective Geometry is a Similarity
Transformation (\S\ref{sec:similarity_transformation}) that leaves the Line at
Infinity Pointwise Invariant



% --------------------------------------------------------------------
\subsection{Incidence}\label{sec:incidence}
% --------------------------------------------------------------------

%FIXME: move this section ?

an \emph{Incidence Relation} is a Binary Relation between different types of
objects (FIXME: clarify)

\emph{Incidence Propositions}

\fist cf. Colinearity (TODO)

\fist Incidence Structure (Finite Geometry \S\ref{sec:incidence_structure})



% --------------------------------------------------------------------
\subsection{Projective Space}\label{sec:projective_space}
% --------------------------------------------------------------------

Homogenous Coordinates (\S\ref{sec:homogenous_coordinate})

Projective Space is a Homogenous Space (\S\ref{sec:homogenous_space}) for its
Symmetry Groups (\S\ref{sec:symmetry_groups})



% --------------------------------------------------------------------
\subsection{Homogenous Coordinate}\label{sec:homogenous_coordinate}
% --------------------------------------------------------------------

or \emph{Projective Coordinates}

Projective Geometry (\S\ref{sec:projective_geometry}), Projective
Space (\S\ref{sec:projective_space}) -- Points (including Points at
Infinity) can be represented using Finite Coordinates; allows for
Affine Transformations (\S\ref{sec:affine_transformation})

cf. Cartesian Coordinates (\S\ref{sec:cartesian_coordinate}) in
Euclidean Geometry (\S\ref{sec:euclidean_geometry})



% --------------------------------------------------------------------
\subsection{Projective Transformation}
\label{sec:projective_transformation}
% --------------------------------------------------------------------

a \emph{Projective Transformation} (or \emph{Homography}) is an
Isomorphism of Projective Spaces induced by an Isomorphism of the
Vector Spaces from which the Projective Spaces derive

Collineation



% --------------------------------------------------------------------
\subsection{Homogenous Function}\label{sec:homogenous_function}
% --------------------------------------------------------------------

Scale Invariance (\S\ref{sec:scale_invariance})



\subsubsection{Homogenous Polynomial}\label{sec:homogenous_polynomial}

Polynomial (\S\ref{sec:polynomial})



\paragraph{Projective Variety}\label{sec:projective_variety}\hfill

Algebraic Variety (\S\ref{sec:algebraic_variety})



% --------------------------------------------------------------------
\subsection{Projective Plane}\label{sec:projective_plane}
% --------------------------------------------------------------------

includes Euclidean, Elliptic, and Hyperbolic Planes

Symmetry Group contains Symmetry Groups for Euclidean, Elliptic, and
Hyperbolic Planes



% --------------------------------------------------------------------
\subsection{Duality}\label{sec:projective_duality}
% --------------------------------------------------------------------

different type of Symmetry %FIXME



% --------------------------------------------------------------------
\subsection{Conic Section}\label{sec:conic_section}
% --------------------------------------------------------------------

% --------------------------------------------------------------------
\subsection{Quadric}\label{sec:quadric}
% --------------------------------------------------------------------

% --------------------------------------------------------------------
\subsection{Fano Plane}\label{sec:fano_plane}
% --------------------------------------------------------------------

% --------------------------------------------------------------------
\subsection{Grassmanian}\label{sec:grassmanian}
% --------------------------------------------------------------------

\subsubsection{Real Projective Space}\label{sec:real_projective_space}

Compact (\S\ref{sec:compact_space}) Smooth Manifold
(\S\ref{sec:smooth_manifold})



\paragraph{Real Projective Plane}\label{sec:real_projective_plane}\hfill

(or \emph{Extended Euclidean Plane})



\subsubsection{Complex Projective Space}
\label{sec:complex_projective_space}



% ====================================================================
\section{Finite Geometry}\label{sec:finite_geometry}
% ====================================================================

\fist Discrete Geometry (\S\ref{sec:discrete_geometry})



% --------------------------------------------------------------------
\subsection{Incidence Structure}\label{sec:incidence_structure}
% --------------------------------------------------------------------

Incidence (\S\ref{sec:incidence})



% ====================================================================
\section{Affine Geometry}\label{sec:affine_geometry}
% ====================================================================

% ====================================================================
\section{Synthetic Differential Geometry}
\label{sec:synthetic_differential_geometry}
% ====================================================================

Differential Geometry (\S\ref{sec:differential_geometry}) formalized in the
language of Topos Theory (\S\ref{sec:topos_theory})



% ====================================================================
\section{Klein Geometry}\label{sec:klein_geometry}
% ====================================================================

%FIXME this is a larger categorization of geometries, possibly
%reorganize others under this heading

a \emph{Klein Geometry} is a pair $(G,H)$ of a Lie Group
(\S\ref{sec:lie_group}) $G$ and $H$ a Closed Lie Subgroup of $G$ such
that the (Left) Coset Space (\S\ref{sec:coset_space}) $G|H$ is
Connected (\S\ref{sec:connected_space})

$G|H$ is called the \emph{Space} of the Geometry and is a Smooth
Manifold (\S\ref{sec:smooth_manifold}) of Dimension:
\[
  dim(X) = dim(G) - dim(H)
\]

\emph{Erlangen Program} -- Projective Geometry as least restrictive
``unifying frame'' for other Geometries considered; from least to more
restrictive: Projective Geometry, Affine Geometry, Euclidean Geometry

examples: %FIXME

Projective Geometry (\S\ref{sec:projective_geometry})

Conformal Geometry (\S\ref{sec:conformal_geometry}) on a Sphere

Hyperbolic Geometry

Elliptic Geometry

Spherical Geometry

Affine Geometry

Euclidean Geometry



% ====================================================================
\section{Discrete Geometry}\label{sec:discrete_geometry}
% ====================================================================

or \emph{Combinatorial Geometry} \fist Combinatorics (Part
\ref{sec:combinatorics})



% --------------------------------------------------------------------
\subsection{Polyhedral Combinatorics}\label{sec:polyhedral_combinatorics}
% --------------------------------------------------------------------

Convex Polyhedra (\S\ref{sec:convex_polyhedra}) and Convex Polytopes
(\S\ref{sec:polytope})
