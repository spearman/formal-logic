%%%%%%%%%%%%%%%%%%%%%%%%%%%%%%%%%%%%%%%%%%%%%%%%%%%%%%%%%%%%%%%%%%%%%%
%%%%%%%%%%%%%%%%%%%%%%%%%%%%%%%%%%%%%%%%%%%%%%%%%%%%%%%%%%%%%%%%%%%%%%
\part{Synthetic Geometry}\label{part:synthetic_geometry}
%%%%%%%%%%%%%%%%%%%%%%%%%%%%%%%%%%%%%%%%%%%%%%%%%%%%%%%%%%%%%%%%%%%%%%
%%%%%%%%%%%%%%%%%%%%%%%%%%%%%%%%%%%%%%%%%%%%%%%%%%%%%%%%%%%%%%%%%%%%%%

%FIXME: this section needs to be made consistent with what is found under
%``analytic geometry''

``Axiomatic Geometry''



% ====================================================================
\section{Euclidean Geometry}\label{sec:euclidean_geometry}
% ====================================================================

$E^n$ -- $n$-dimensional Euclidean Space (Euclidean Space
\S\ref{sec:euclidean_space}) with Cartesian Coordinates is Modelled by
$\reals^n$ ($n$-dimensional Real Coordinate Space
\S\ref{sec:real_coordinate_space})

\fist cf. Homogeneous Coordinates (\S\ref{sec:homogeneous_coordinate})
in Projective Geometry (\S\ref{sec:projective_geometry})

Riemannian Manifold (\S\ref{sec:riemannian_manifold}) of Constant Vanishing
Sectional Curvature (\S\ref{sec:sectional_curvature})

\emph{Euclidean Group} (\S\ref{sec:euclidean_group}) -- Group of all
Isometries $ISO(n)$ or $E(n)$; makes Euclidean Geometry a case of
Klein Geometry (\S\ref{sec:klein_geometry})

Tarski-Seidenberg Theorem (\S\ref{sec:tarski_seidenberg}): Euclidean Geometry
without the ability to measure Angles is a Model of the Real Field Axioms
and therefore also Decidable



% --------------------------------------------------------------------
\subsection{Line}\label{sec:line}
% --------------------------------------------------------------------

A \emph{Line} (or \emph{Straight Line}) is a Primitive Notion
(\S\ref{sec:primitive_notion}) representing a Geometric Object with no
\emph{Curvature} (\S\ref{sec:curvature})

\fist cf. Line (Algebraic Curve \S\ref{sec:algebraic_line})



% --------------------------------------------------------------------
\subsection{Half-space}\label{sec:half_space}
% --------------------------------------------------------------------

Hyperbolic Spaces (\S\ref{sec:hyperbolic_space})



% --------------------------------------------------------------------
\subsection{Convex Geometry}\label{sec:convex_geometry}
% --------------------------------------------------------------------

%FIXME:

Simplex

Convex Set (\S\ref{sec:convex_set})



\subsubsection{Affine Hull}\label{sec:affine_hull}

\subsubsection{Convex Hull}\label{sec:convex_hull}

Vertex representation of a Convex Polytope (\S\ref{sec:convex_polytope})

a Convex Polyhedra (\S\ref{sec:convex_polyhedra}) is the Convex Hull of
finitely many Points not all lying in the same Plane



\subsubsection{Convex Combination}\label{sec:convex_combination}

a Linear Combination (\S\ref{sec:linear_combination}) of Points

\fist Convex Functions (\S\ref{sec:convex_function})



% --------------------------------------------------------------------
\subsection{Rigid Transformation}\label{sec:rigid_transformation}
% --------------------------------------------------------------------

A \emph{Rigid Transformation} or \emph{Euclidean Isometry} is an Isometry
(\S\ref{sec:isometry}) associated with the Euclidean Distance Metric of a
Euclidean Space.

The Euclidean Group (\S\ref{sec:euclidean_group}) $\mathrm{ISO}(n)$ is the
Symmetry Group of $n$-dimensional Euclidean Space with Rigid Transformations as
its Elements.

cf. Affine Transformations (\S\ref{sec:affine_transformation}), Affine Group
(\S\ref{sec:affine_group})



\subsubsection{Translation}\label{sec:translation}

Affine Transformation (\S\ref{sec:affine_transformation})

\fist Periodic Functions (\S\ref{sec:periodic_function}) are Functions with
Graphs that exhibit Translational Symmetry (\S\ref{sec:symmetry_group})



\subsubsection{Rotation}\label{sec:rotation}

Affine Transformation (\S\ref{sec:affine_transformation})

Discrete Rotational Symmetry (\S\ref{sec:discrete_symmetry})



\subsubsection{Reflection}\label{sec:reflection}

Affine Transformation (\S\ref{sec:affine_transformation})

Discrete Symmetry (\S\ref{sec:discrete_symmetry})

Isometry (\S\ref{sec:isometry})

Reflection Groups (\S\ref{sec:reflection_group}) are Discrete Groups Generated
by a Set of Reflections of a Finite-dimensional Euclidean Space
(\S\ref{sec:reflection})

The Coxeter Groups (\S\ref{sec:coxeter_group}) are precisely the Finite
Euclidean Reflection Groups

\begin{itemize}
  \item Elementary Reflector (Householder Transformation
    \S\ref{sec:elementary_reflector}) -- (Linear, Orthogonal) Reflection about
    a Plane containing the Origin
\end{itemize}



\subsubsection{Congruence}\label{sec:congruence}

(wiki):

two Sets of Points are called \emph{Congruent} if and only if one can be
Transformed into the other by an Isometry, i.e. a combination of Rigid Motions
(\S\ref{sec:rigid_transformation}), viz. Translation, Rotation, and Reflection

\fist a Similarity (\S\ref{sec:similarity_transformation}) between objects
additionally allows for Scaling (\S\ref{sec:scaling})



% --------------------------------------------------------------------
\subsection{Similarity Transformation}\label{sec:similarity_transformation}
% --------------------------------------------------------------------

two geometric objects are \emph{Similar} if they have the same ``Shape'' under
Rigid Transformations (\S\ref{sec:rigid_transformation}) and Scaling

in Projective Geometry (\S\ref{sec:projective_geometry}) a Similarity
Transformation fixes a given Elliptic Involution (FIXME: clarify)

in Projective Geometry a Homothety (\S\ref{sec:homothety}) is a Similarity
Transformation that leaves the Line at Infinity Pointwise Invariant

\fist cf. Similarity Measure (\S\ref{sec:similarity_measure})



\subsubsection{Scaling}\label{sec:scaling}

Linear Transformation (\S\ref{sec:linear_transformation}), special case of a
Homothetic Transformation (\S\ref{sec:homothety})



% ====================================================================
\section{Higher-dimensional Geometry}\label{sec:higher_geometry}
% ====================================================================

% --------------------------------------------------------------------
\subsection{Polytope}\label{sec:polytope}
% --------------------------------------------------------------------

\begin{itemize}
  \item Polygon (\S\ref{sec:polygon}) -- 2D
  \item Polyhedra (\S\ref{sec:polyhedra}) -- 3D
  \item Polychoron (Polycell \S\ref{sec:polychoron}) -- 4D
\end{itemize}



\subsubsection{Convex Polytope}\label{sec:convex_polytope}

Vertex representation is a Convex Hull (\S\ref{sec:convex_hull})

may be defined as an intersection of a finite number of Half-spaces

\fist Polyhedral Combinatorics (\S\ref{sec:polyhedral_combinatorics})



\subsubsection{Uniform Polytope}\label{sec:uniform_polytope}

\paragraph{Regular Polytope}\label{sec:regular_polytope}\hfill

Rectification

Truncation (\S\ref{sec:truncation_operator})



\paragraph{Truncation}\label{sec:truncation_operator}\hfill

\begin{enumerate}
  \item Truncation -- 2D (Polygons) and higher
  \item Cantellation (Rectified Rectification) -- 3D (Polyhedra) and higher
\end{enumerate}



\subsubsection{Star Polytope}\label{sec:star_polytope}

\subsubsection{Simplicial Polytope}\label{sec:simplicial_polytope}

all Facets are Simplices

Deltahedron (\S\ref{sec:deltahedron})



\subsubsection{Net}\label{sec:polytope_net}

\fist not to be confused with Topological Nets (\S\ref{sec:net})

\emph{D\"urer's Conjecture}: does every Convex Polyhedron have a Net?

Ghomi (2014): every Convex Polyhedron admits a net after an Affine
Transformation

it is known that every Convex Uniform $4$-polytope can be cut along
Two-dimensional Faces shared by its Three-dimensional Facets and unfolded into
a single non-overlapping Polyhedron ``Net'' in $3$ Dimensions




% --------------------------------------------------------------------
\subsection{Polygon}\label{sec:polygon}
% --------------------------------------------------------------------

2D Polytope

a Plane Figure bounded by a Closed Polygonal Chain (Finite Chain of Straight
Line Segments, i.e. \emph{Sides} or \emph{Edges}) or ``\emph{Circuit}''



\subsubsection{Dissection}\label{sec:dissection}

\paragraph{Equidissection}\label{sec:equidissection}\hfill

\subparagraph{Spectrum}\label{sec:polygon_spectrum}\hfill

the Set of all $n$ for which an $n$-equidissection of a Polygon $P$ exists



% --------------------------------------------------------------------
\subsection{Polyhedra}\label{sec:polyhedra}
% --------------------------------------------------------------------

3D ``Solid''

Nets of Polyhedra (\S\ref{sec:polytope_net})

\emph{D\"urer's Conjecture}: does every Convex Polyhedron have a Net?

Ghomi (2014): every Convex Polyhedron admits a net after an Affine
Transformation



\subsubsection{Convex Polyhedra}\label{sec:convex_polyhedra}

Convex Hull (\S\ref{sec:convex_hull}) of finitely many Points not all on the
same Plane

\fist Polyhedral Combinatorics (\S\ref{sec:polyhedral_combinatorics})



\paragraph{Platonic Solid}\label{sec:platonic_solid}\hfill

\paragraph{Archimedean Solid}\label{sec:archimedean_solid}\hfill

\paragraph{Johnson Solid}\label{sec:johnson_solid}\hfill

Non-uniform Convex Polyhedron with Regular Polygons for faces

\begin{itemize}
  \item Pyramids
  \item Cupolae and Rotundae
  \item Elongated and Gyroelongated Pyramids
  \item Bipyramids, Elongated Bipyramids, Gyroelongated Bipyramids
  \item Elongated Cupolae and Elongated Rotundae
  \item Bicupolae
  \item Bianticupolae
  \item Cupola-rotundae and Birodunta
  \item Triangular Hebesphenorotunda -- only Johnson Solid with Faces of 3, 4,
    5, and 6 Sides
  \item ... MORE
\end{itemize}



\paragraph{Cuboid}\label{sec:cuboid}\hfill

\begin{enumerate}
  \item Hilbert Cube (\S\ref{sec:hilbert_cube})
  \item ... MORE
\end{enumerate}



\paragraph{Pyramid}\label{sec:pyramid}\hfill

\paragraph{Prism}\label{sec:prism}\hfill

\paragraph{Antiprism}\label{sec:antiprism}\hfill

\paragraph{Rotunda}\label{sec:rotunda}\hfill

\paragraph{Cupola}\label{sec:cupola}\hfill



\subsubsection{Deltahedron}\label{sec:deltahedron}

Simplicial Polytope (\S\ref{sec:simplicial_polytope})



\subsubsection{Uniform Polyhedra}\label{sec:uniform_polyhedra}

\fist Uniform Polytope (\S\ref{sec:uniform_polytope})



% --------------------------------------------------------------------
\subsection{Polychoron}\label{sec:polychoron}
% --------------------------------------------------------------------

A \emph{Polycell} or \emph{Polychoron} is a $4$-polytope.



\subsubsection{Uniform Polychoron}\label{sec:uniform_polychoron}

it is known that every Convex Uniform $4$-polytope can be cut along
Two-dimensional Faces shared by its Three-dimensional Facets and unfolded into
a single non-overlapping Polyhedron ``Net'' in $3$ Dimensions



% ====================================================================
\section{Non-euclidean Geometry}\label{sec:noneuclidean_geometry}
% ====================================================================

some applets for exploring Non-euclidean Geometries (listed on
\url{https://www.maa.org/press/periodicals/loci/rethinking-pythagoras-0}):

\url{http://www.cs.unm.edu/~joel/NonEuclid/NonEuclid.html}

\url{http://merganser.math.gvsu.edu/easel/applet.html}

\url{https://www.cinderella.de/tiki-index.php}

\url{http://homepages.gac.edu/~hvidsten/explorer/}



% --------------------------------------------------------------------
\subsection{Hyperbolic Geometry}\label{sec:hyperbolic_geometry}
% --------------------------------------------------------------------

Riemannian Manifold (\S\ref{sec:riemannian_manifold}) of Constant Negative
Sectional Curvature (\S\ref{sec:sectional_curvature})



\subsubsection{Hyperbolic Space}\label{sec:hyperbolic_space}

Homogeneous Space (\S\ref{sec:homogeneous_space}) with Constant Negative
Sectional Curvature (\S\ref{sec:sectional_curvature})

Upper Half-space (\S\ref{sec:half_space})



% --------------------------------------------------------------------
\subsection{Elliptic Geometry}\label{sec:elliptic_geometry}
% --------------------------------------------------------------------

Riemannian Manifold (\S\ref{sec:riemannian_manifold}) of Constant Positive
Sectional Curvature (\S\ref{sec:sectional_curvature})



% ====================================================================
\section{Affine Geometry}\label{sec:affine_geometry}
% ====================================================================

% --------------------------------------------------------------------
\subsection{Affine Space}\label{sec:affine_space}
% --------------------------------------------------------------------

An \emph{Affine Space} generalizes properties of Euclidean Spaces
(\S\ref{sec:euclidean_space}) to be \emph{independent} of the concepts of
Distance (cf. Norm \S\ref{sec:norm}) and Measure of Angles (cf. Rotation
\S\ref{sec:rotation}), keeping only properties relating to
\emph{Parallelism}

any Vector Space (\S\ref{sec:vector_space}) may be considered an Affine Space
over itself

Affine Space is a Homogeneous Space (\S\ref{sec:homogeneous_space}) for its
Symmetry Groups (\S\ref{sec:symmetry_group})

Manifolds (\S\ref{sec:manifold}) built by ``gluing together'' Charts
(\S\ref{sec:chart}, Open Subsets of Real Affine Spaces)

cf. Algebraic Varieties (\S\ref{sec:algebraic_variety}) built by ``gluing
together'' Affine Varities (\S\ref{sec:affine_variety})

Affine Group (\S\ref{sec:affine_group}) $Aff(n,F)$ -- Extension of the General
Linear Group $GL(n,F)$ by Group of Translations in $F^n$ is the Group of all
Affine Transformations (\S\ref{sec:affine_transformation}) on the Affine Space
underlying the Vector Space $F^n$



\subsubsection{Affine Subspace}\label{sec:affine_subspace}

\paragraph{Flat}\label{sec:flat}\hfill

a Subset of an $n$-dimensional Space that is Congruent



\subsubsection{Affine Basis}\label{sec:affine_basis}

\emph{Affinely Independent} -- the Affine analog of Linear Independence

$d+1$ Points in General Position (\S\ref{sec:general_position}) in Affine
$d$-space are an Affine Basis



\paragraph{General Position}\label{sec:general_position}\hfill

or \emph{General Linear Position}

\fist cf. Generic Points (Algebraic Varieties \S\ref{sec:generic_point})

\fist cf. Differential Topoplogy (\S\ref{sec:differential_topology}): Tangency
(\S\ref{sec:tangency}), Transversality (\S\ref{sec:transversality})

(wiki):

notion of \emph{Genericity} for a Set of Points or other Geometric Objects

$d+1$ Points in General Position in Affine $d$-space are an Affine Basis

a Set of Points in a $d$-dimensional Affine Space (\S\ref{sec:affine_space}),
e.g. $d$-dimensional Euclidean Space, is in \emph{General (Linear) Position} if
no $k$ of them lie in a $(k-2)$-dimensional Flat (Affine Subspace
\S\ref{sec:flat}) for $k = 2,3,\ldots,d+1$; if the condition holds for some
$k_0$ then it must also hold for all $k$ with $2 \leq k \leq k_0$, so for a
Set containing at least $d+1$ Points in a $d$-dimensional Affine Space to be in
General Position, it is sufficient to know that no Hyperplane contains more
than $d$ Points-- i.e. the Points do not Satisfy any more Linear Relations than
they must (FIXME: clarify)

a Set of $n$ Vectors in an $n$-dimensional Vector Space are Linearly
Independent if and only if the Points they define in Projective Space
(\S\ref{sec:projective_space}) of Dimension $n-1$ are in General Linear
Position

given any five Points in the Plane in General Linear Position (i.e. no three
are Collinear), there is a unique non-degenerate Conic (\S\ref{sec:conic})
passing through them



\paragraph{Special Position}\label{sec:special_position}\hfill



\subsubsection{Affine Set}\label{sec:affine_set}

(2004 Boyd-Vandenberghe \S2.1.2):

an Affine Set contains the entire Line through any two Points in the Set; cf. a
Convex Set (\S\ref{sec:convex_set}) contains only the Line Segment between any
two Points

Line through Points $x_1$ and $x_2$:
\[
  x = \theta x_1 + (1-\theta) x_2
\]
for $\theta \in \reals$

every Affine Set is also Convex

\fist in a Convex Set, $\theta$ is restricted to $0 \leq \theta \leq 1$

every Affine Set can be expressed as the Solution Set of a System of Linear
Equations (\S\ref{sec:system_of_linear_equations})

\begin{itemize}
  \item Hyperplanes (\S\ref{sec:hyperplane}) are Affine and Convex
  \item the Set $\mathsf{S}^n$ of Symmetric $n \times n$ Matrices is Affine and
    Convex
\end{itemize}

\fist Affine Algebraic Set (\S\ref{sec:affine_algebraic_set}), Affine Variety
(\S\ref{sec:affine_variety})



\subsubsection{Affine Line}\label{sec:affine_line}

a Morphism from a Algebraic Variety to the Affine Line is a \emph{Regular Map}
(\S\ref{sec:regular_map})

a Regular Map between Complex Algebraic Varieties is a Holomorphic Map
(\S\ref{sec:holomorphic_function})



\subsubsection{Affine Plane}\label{sec:affine_plane}

\subsubsection{Affine Hyperplane}\label{sec:affine_hyperplane}

Hyperplane (\S\ref{sec:hyperplane})

cf. Affine Manifold (\S\ref{sec:affine_manifold})



% --------------------------------------------------------------------
\subsection{Affine Transformation}\label{sec:affine_transformation}
% --------------------------------------------------------------------

(or \emph{Affine Map} or \emph{Affinity}) is a Function between Affine
Spaces which preserves Points, Straight Lines, and Planes, and Sets of
Parallel Lines remain Parallel

preserves Ratios of Distances betwen Points lying on a Straight Line

does not necessarily preserve Angles between Lines or Distances
between Points

for Affine Spaces $X$ and $Y$

Affine Functions are both Convex and Concave (\S\ref{sec:convex_function})

\fist Affine Group (\S\ref{sec:affine_group})

\fist All Linear Transformations (\S\ref{sec:linear_transformation}) are
Affine, but not every Affine Transformation is Linear; Affine Transformations
are not required to preserve the Zero Point in a Linear Space (FIXME: clarify)

Affine Transformations of Quaternions (\S\ref{sec:quaternion_function}) have
the form:
\[
  f(q) = aq + b, \;\;\; a,b,q \in \quats
\]

Affine Group (\S\ref{sec:affine_group}) $Aff(n,F)$ -- Extension of the General
Linear Group $GL(n,F)$ by Group of Translations in $F^n$ is the Group of all
Affine Transformations on the Affine Space underlying the Vector Space $F^n$

Affine Transformations:
\begin{itemize}
\item Translation (\S\ref{sec:translation})
\item Rotation (\S\ref{sec:rotation})
\item Reflection (\S\ref{sec:reflection})
\item Scaling
\item Homothety (Homogeneous Dilation or Central Similarity)
\item Similarity Transformation (???)
\item Shear Mapping (???)
\end{itemize}
and combinations of the above in any combination and sequence

cf. Affine Logic (\S\ref{sec:affine_logic})



% --------------------------------------------------------------------
\subsection{Halfspace}\label{sec:halfspace}
% --------------------------------------------------------------------

either of two parts into which a Hyperplane (\S\ref{sec:hyperplane}) divides an
Affine Space

Halfspaces are Convex Sets (\S\ref{sec:convex_set})

note Halfspaces are not Vector Spaces since they are not closed under Scalar
Multiplication

\begin{itemize}
  \item Ray -- 1D
  \item Half-plane -- 2D
  \item Half-space -- 3D
\end{itemize}



% --------------------------------------------------------------------
\subsection{Homothety}\label{sec:homothety}
% --------------------------------------------------------------------

\emph{Scale Transformation} \fist Scaling (\S\ref{sec:scaling})

Linear Transformation (\S\ref{sec:linear_transformation}) of an Affine Space
determined by a Homothety Center Point $S$ and a Nonzero Scalar $\lambda$
called the \emph{Ratio} fixing $S$ and sending any Point $M$ to a Point $N$
such that $\vec{SN}$ is on the same Line as $\vec{SM}$ but \emph{Scaled} by a
factor of $\lambda$:
\[
  M \mapsto S + \lambda\vec{SM}
\]

\fist Projective Geometry (\S\ref{sec:projective_geometry}): a Homothety is a
Similarity Transformation (\S\ref{sec:similarity_transformation}) that leaves
the Line at Infinity Pointwise Invariant

\fist Scale Invariance (\S\ref{sec:scale_invariance})



% ====================================================================
\section{Projective Geometry}\label{sec:projective_geometry}
% ====================================================================

\emph{Homogeneous Coordinates} (or \emph{Projective Coordinates}
\S\ref{sec:homogeneous_coordinate}) -- Points (including Points at Infinity) can
be represented using Finite Coordinates; allows for Affine Transformations
(\S\ref{sec:affine_transformation})

cf. Cartesian Coordinates (\S\ref{sec:cartesian_coordinate}) in
Euclidean Geometry (\S\ref{sec:euclidean_geometry})

``Subsumes'' Euclidean, Elliptic, and Hyperbolic Geometry: Projective
Plane (\S\ref{sec:projective_plane}) includes Euclidean, Elliptic, and
Hyperbolic Planes and its Symmetry Group contains their Symmetry
Groups

``all Projective Geometries are exercises in Linear Algebra'' (E.
Kmett interview -- source?) %FIXME

a Homothety (\S\ref{sec:homothety}) in Projective Geometry is a Similarity
Transformation (\S\ref{sec:similarity_transformation}) that leaves the Line at
Infinity Pointwise Invariant

\fist $\mathbb{F}_1$-geometry (\S\ref{sec:f1_geometry}) -- Projective Geometries
over Finite Fields have meaningful analogue for $1$ Element (Tits57)



% --------------------------------------------------------------------
\subsection{Incidence}\label{sec:incidence}
% --------------------------------------------------------------------

%FIXME: move this section ?

an \emph{Incidence Relation} is a Binary Relation between different types of
objects (FIXME: clarify)

\emph{Incidence Propositions}

\fist cf. Colinearity (TODO)

\fist Incidence Structure (Finite Geometry \S\ref{sec:incidence_structure})



% --------------------------------------------------------------------
\subsection{Projective Space}\label{sec:projective_space}
% --------------------------------------------------------------------

Homogeneous Coordinates (\S\ref{sec:homogeneous_coordinate})

Projective Space is a Homogeneous Space (\S\ref{sec:homogeneous_space}) for its
Symmetry Groups (\S\ref{sec:symmetry_group})

\fist Projective Varieties (\S\ref{sec:projective_variety})

Projective Linear Group (\S\ref{sec:projective_linear_group}) $PGL(n,F)$ --
Quotient of $GL(n,F)$ by its Center, the Subgroup of Nonzero Scalar Matrices
$Z(n,F)$, which is the induced Action (\S\ref{sec:group_action}) on the
associated Projective Space

a Set of $n$ Vectors in an $n$-dimensional Vector Space are Linearly
Independent if and only if the Points they define in Projective Space of
Dimension $n-1$ are in General Linear Position (\S\ref{sec:general_position})

\begin{itemize}
  \item Conics (\S\ref{sec:conic}) can be represented by Points in
    $5$-dimensional Projective Space
  \item Cubic Curves (\S\ref{sec:cubic_plane_curve}) can be represented by
    Points in $9$-dimensional Projective Space
\end{itemize}



\subsubsection{Collineation}\label{sec:collineation}

Bijection (Isomorphism) between Projective Spaces such that the Image of two
Collinear Points are Collinear



\paragraph{Homography}\label{sec:homography}\hfill

or \emph{Projective Transformation}

an Isomorphism of Projective Spaces induced by an Isomorphism of the Vector
Spaces from which the Projective Spaces are derived

\emph{Fundamental Theorem of Projective Geometry} (TODO)



\subsubsection{Projective Line}\label{sec:projective_line}

One-dimensional Projective Space

Line (\S\ref{sec:line}) extended by a Point at Infinity



% --------------------------------------------------------------------
\subsection{Homogeneous Coordinate}\label{sec:homogeneous_coordinate}
% --------------------------------------------------------------------

or \emph{Projective Coordinates}

Projective Geometry (\S\ref{sec:projective_geometry}), Projective
Space (\S\ref{sec:projective_space}) -- Points (including Points at
Infinity) can be represented using Finite Coordinates; allows for
Affine Transformations (\S\ref{sec:affine_transformation})

\fist cf. Cartesian Coordinates (\S\ref{sec:cartesian_coordinate}) in
Euclidean Geometry (\S\ref{sec:euclidean_geometry})

\fist cf. Homogeneous Functions (\S\ref{sec:homogeneous_function})



% --------------------------------------------------------------------
\subsection{Projective Transformation}
\label{sec:projective_transformation}
% --------------------------------------------------------------------

a \emph{Projective Transformation} (or \emph{Homography}) is an
Isomorphism of Projective Spaces induced by an Isomorphism of the
Vector Spaces from which the Projective Spaces derive

Collineation



% --------------------------------------------------------------------
\subsection{Projective Plane}\label{sec:projective_plane}
% --------------------------------------------------------------------

includes Euclidean, Elliptic, and Hyperbolic Planes

Symmetry Group contains Symmetry Groups for Euclidean, Elliptic, and
Hyperbolic Planes



% --------------------------------------------------------------------
\subsection{Duality}\label{sec:projective_duality}
% --------------------------------------------------------------------

different type of Symmetry %FIXME



% --------------------------------------------------------------------
\subsection{Conic Section}\label{sec:conic_section}
% --------------------------------------------------------------------

\fist Conic (\S\ref{sec:conic})

Family of Curves (\S\ref{sec:curve_family})

\fist Linear System of Divisors (\S\ref{sec:linear_system_of_divisors})



% --------------------------------------------------------------------
\subsection{Quadric}\label{sec:quadric}
% --------------------------------------------------------------------

generalization of Conic Sections

a \emph{Quadric Surface} is a Surface that may be defined as the Zero Set
(\S\ref{sec:function_root}) of a Polynomial of Degree $2$, specifically as a
Hypersurface of Dimension $d$ in a $d+1$-dimensional Space defined as the Zero
Set of an Irreducible Polynomial (\S\ref{sec:irreducible_polynomial}) of Degree
$2$ in $d+1$ Variables



\subsubsection{Ellipsoid}\label{sec:ellipsoid}



% --------------------------------------------------------------------
\subsection{Fano Plane}\label{sec:fano_plane}
% --------------------------------------------------------------------

% --------------------------------------------------------------------
\subsection{Grassmanian}\label{sec:grassmanian}
% --------------------------------------------------------------------

(wiki):

The \emph{Grassmanian} $Gr(k,V)$ of $k$-dimensional Subspaces of an
$n$-dimensional Vector Space $V$ is a Space which \emph{parameterizes} all
$k$-dimensional Linear Subspaces (\S\ref{sec:linear_subspace}) of $V$, e.g.
the Grassmanian $Gr(1,V)$ is the Space of Lines through the Origin in $V$ (and
equivalent to the Projective Space of Dimension $n-1$).

When $V$ is a Real or Complex Vector Space, the Grassmanians are Compact Smooth
Manifolds (\S\ref{sec:smooth_manifold}).



\subsubsection{Real Projective Space}\label{sec:real_projective_space}

Compact (\S\ref{sec:compact_space}) Smooth Manifold
(\S\ref{sec:smooth_manifold})



\paragraph{Real Projective Plane}\label{sec:real_projective_plane}\hfill

(or \emph{Extended Euclidean Plane})

can be thought of as the Two-sphere Quotiented out by the Antipodal Map (TODO:
explain)



\subsubsection{Complex Projective Space}
\label{sec:complex_projective_space}



% ====================================================================
\section{Finite Geometry}\label{sec:finite_geometry}
% ====================================================================

\fist Discrete Geometry (\S\ref{sec:discrete_geometry})



% --------------------------------------------------------------------
\subsection{Incidence Structure}\label{sec:incidence_structure}
% --------------------------------------------------------------------

Incidence (\S\ref{sec:incidence})



% ====================================================================
\section{Synthetic Differential Geometry}
\label{sec:synthetic_differential_geometry}
% ====================================================================

Differential Geometry (\S\ref{sec:differential_geometry}) formalized in the
language of Topos Theory (\S\ref{sec:topos_theory})



% ====================================================================
\section{Klein Geometry}\label{sec:klein_geometry}
% ====================================================================

%FIXME this is a larger categorization of geometries, possibly
%reorganize others under this heading

a \emph{Klein Geometry} is a pair $(G,H)$ of a Lie Group
(\S\ref{sec:lie_group}) $G$ and $H$ a Closed Lie Subgroup of $G$ such
that the (Left) Coset Space (\S\ref{sec:coset_space}) $G|H$ is
Connected (\S\ref{sec:connected_space})

$G|H$ is called the \emph{Space} of the Geometry and is a Smooth
Manifold (\S\ref{sec:smooth_manifold}) of Dimension:
\[
  dim(X) = dim(G) - dim(H)
\]

\emph{Erlangen Program} -- Projective Geometry as least restrictive
``unifying frame'' for other Geometries considered; from least to more
restrictive: Projective Geometry, Affine Geometry, Euclidean Geometry

examples: %FIXME

Projective Geometry (\S\ref{sec:projective_geometry})

Conformal Geometry (\S\ref{sec:conformal_geometry}) on a Sphere

Hyperbolic Geometry

Elliptic Geometry

Spherical Geometry

Affine Geometry

Euclidean Geometry



% ====================================================================
\section{Discrete Geometry}\label{sec:discrete_geometry}
% ====================================================================

or \emph{Combinatorial Geometry} \fist Combinatorics (Part
\ref{part:combinatorics})



% --------------------------------------------------------------------
\subsection{Polyhedral Combinatorics}\label{sec:polyhedral_combinatorics}
% --------------------------------------------------------------------

Convex Polyhedra (\S\ref{sec:convex_polyhedra}) and Convex Polytopes
(\S\ref{sec:polytope})



% ====================================================================
\section{Noncommutative Geometry}\label{sec:noncommutative_geometry}
% ====================================================================

Noncommutative Algebras (\S\ref{sec:noncommutative_algebra})

Category of Noncommutative Affine Schemes as Dual of the Category of
Associative Unital Rings

Cyclic Homology (\S\ref{sec:cyclic_homology})



% --------------------------------------------------------------------
\subsection{Fuzzy Sphere}\label{sec:fuzzy_sphere}
% --------------------------------------------------------------------

Hasebe2010 - \emph{Hopf Maps, Lowest Landau Level, and Fuzzy Spheres}
