%%%%%%%%%%%%%%%%%%%%%%%%%%%%%%%%%%%%%%%%%%%%%%%%%%%%%%%%%%%%%%%%%%%%%%
%%%%%%%%%%%%%%%%%%%%%%%%%%%%%%%%%%%%%%%%%%%%%%%%%%%%%%%%%%%%%%%%%%%%%%
\part{Synthetic Geometry}\label{part:synthetic_geometry}
%%%%%%%%%%%%%%%%%%%%%%%%%%%%%%%%%%%%%%%%%%%%%%%%%%%%%%%%%%%%%%%%%%%%%%
%%%%%%%%%%%%%%%%%%%%%%%%%%%%%%%%%%%%%%%%%%%%%%%%%%%%%%%%%%%%%%%%%%%%%%

\emph{Axiomatic Geometry} or \emph{Coordinate-free Geometry}

\fist cf. Analytic Geometry (Coordinate Geometry \S\ref{part:analytic_geometry})

Axiomatic System (\S\ref{sec:axiomatic_system}):
\begin{itemize}
  \item Geometric Primitives (Primitive Notions \S\ref{sec:geometric_primitive})
  \item Geometric Postulates (Axioms \S\ref{sec:geometric_postulate})
\end{itemize}
A Geometric Axiomatizaion imparts a structure to the ``Space'' (Appendix
\S\ref{sec:space}) of Primitive Points (\S\ref{sec:point}) of some underlying
Set.

\fist Spaces defined by Axiomatic Systems may be Modeled (\S\ref{sec:model})
using various Coordinate Systems (Analytic Geometry
\S\ref{sec:coordinate_system}).



% ====================================================================
\section{Coordinate-free}\label{sec:coordinate_free}
% ====================================================================

Coordinate-free treatments of:
\begin{itemize}
  \item Vector Calculus (\S\ref{sec:vector_calculus}) and Tensor Calculus
    (\S\ref{sec:tensor_calculus})
  \item Differential Geometry (\S\ref{sec:synthetic_differential_geometry})
\end{itemize}

Coordinate-free treatments of ``Physical Theories'' is a corollary of the
\emph{Principle of General Covariance} (\S\ref{sec:general_covariance})

1992 - DeRose - \emph{Three-Dimensional Computer Graphics: A Coordinate-Free
  Approach} -
\url{http://graphics.pixar.com/people/derose/publications/GeometryBook/}



% ====================================================================
\section{Geometric Primitive}\label{sec:geometric_primitive}
% ====================================================================

\emph{Geometric Objects} and \emph{Geometric Relations}

cf. Primitive Notion (\S\ref{sec:primitive_notion})

Projective Geometry (\S\ref{sec:projective_geometry}):
\begin{itemize}
  \item Whitehead's Axiom System (3 Axioms):
    \begin{itemize}
      \item Primitive Terms: Point, Line
      \item Primitive Relations: Incidence
    \end{itemize}
\end{itemize}

Euclidean Geometry (\ref{sec:euclidean_geometry}):
\begin{itemize}
  \item Peano's Axiom System: Point, Segment, Motion
  \item Hilbert's Axiom System (20 Axioms):
    \begin{itemize}
      \item Primitive Terms: Point, Line, Plane
      \item Primitive Relations: Congruence, Betweenness,
        Incidence (Containment)
    \end{itemize}
  \item Birkhoff's Axiom System: TODO
  \item Tarski's Axiom System:
    \begin{itemize}
      \item Primitive Terms: Points only
      \item Primitive Relations: Betweenness, Congruence
    \end{itemize}
\end{itemize}



% --------------------------------------------------------------------
\subsection{Geometric Object}\label{sec:geometric_object}
% --------------------------------------------------------------------

\fist Moduli Space (\S\ref{sec:moduli_space}) -- Space with Points representing
Geometric Objects



\subsubsection{Point}\label{sec:point}

an Element of a Set called a \emph{Space}

\begin{itemize}
  \item Topological Point (Point-set Topology \S\ref{sec:point})
  \item ... MORE?
\end{itemize}



\subsubsection{Line}\label{sec:line}

A \emph{Line} (or \emph{(Infinite) Straight Line}) is a Primitive Notion
(\S\ref{sec:primitive_notion}) representing a Geometric Object with no
\emph{Curvature} (\S\ref{sec:curvature})

\begin{itemize}
  \item Line (Algebraic Curve \S\ref{sec:algebraic_line})
  \item ... MORE?
\end{itemize}



\paragraph{Line Segment}\label{sec:line_segment}\hfill

or \emph{Finite Straight Line}



\paragraph{Circle}\label{sec:circle}\hfill

a Straight Line can be seen as a Circle of Infinite Radius

\begin{itemize}
  \item ... TODO
\end{itemize}



\paragraph{Geodesic}\label{sec:geodesic}\hfill

generalization of ``Straight Lines'' to Curved Spaces -- Elliptic Geometry
(\S\ref{sec:elliptic_geometry}), Hyperbolic Geometry
(\S\ref{sec:hyperbolic_geometry}); cf. Riemannian Geometry
\S\ref{sec:riemannian_geometry})



\subsubsection{Plane}\label{sec:plane}



% --------------------------------------------------------------------
\subsection{Geometric Relation}\label{sec:geometric_relation}
% --------------------------------------------------------------------

\subsubsection{Incidence}\label{sec:incidence}

an \emph{Incidence Relation} is a Binary Relation between different types of
objects (FIXME: clarify)

derived concepts:
\begin{itemize}
  \item \emph{Intersection} -- two Lines which are both Coincident to a common
    Point are said to Intersect (\S\ref{sec:intersection})
  \item \emph{Collinearity} -- two Points which are Incident to the same Line
    are said to be Collinear (\S\ref{sec:collinearity})
  \item \emph{Containment} -- when one Object can be thought of as a Set of the
    other type of Object, then the Incidence Relation may be viewed as
    Containment (\S\ref{sec:containment})
\end{itemize}

Incidence Geometry (\S\ref{sec:incidence_geometry})

Projective Geometry (\S\ref{sec:projective_geometry}): Propositions of Incidence
are True without exception (cf. Euclidean Geometry where Lines may be Parallel)

\emph{Incidence Propositions}

\fist Incidence Geometry (\S\ref{sec:incidence_geometry}): study of Incidence
  Structures (\S\ref{sec:incidence_structure})



\paragraph{Intersection}\label{sec:intersection}\hfill

\fist Intersection (Set Theory \S\ref{sec:set_intersection})



\paragraph{Collinearity}\label{sec:collinearity}\hfill

Ternary Relation

(wiki):

Two Points which are Incident to the same Line are said to be \emph{Collinear}.



\paragraph{Containment}\label{sec:containment}\hfill

\fist Subset Relation (\S\ref{sec:subset})



\subsubsection{Intermediacy}\label{sec:intermediacy}

``\emph{Betweenness}''

Ternary Relation on Points $[ABC]$

Ordered Geometry (\S\ref{sec:ordered_geometry})



% ====================================================================
\section{Geometric Postulate}\label{sec:geometric_postulate}
% ====================================================================

The Axioms (\S\ref{sec:axiom}) of a Geometric Axiomatization



% ====================================================================
\section{Geometric Transformation}\label{sec:geometric_transformation}
% ====================================================================

%FIXME: move this section ?

% --------------------------------------------------------------------
\subsection{Infinitesimal Transformation}
\label{sec:infinitesimal_transformation}
% --------------------------------------------------------------------

limiting form of ``small'' Transformation

One-parameter Group (\S\ref{sec:one_parameter_group})

Lie Groups (\S\ref{sec:lie_group})

Lie Algebras (\S\ref{sec:lie_algebra})



% ====================================================================
\section{Locus}\label{sec:locus}
% ====================================================================

%FIXME: move this section ???

Set of all Points (commonly Line, Segment, Curve, or Surface) that Satisfy some
Property




% --------------------------------------------------------------------
\subsection{Smooth Locus}\label{sec:smooth_locus}
% --------------------------------------------------------------------

\emph{example}:

(nlab):

for $D \hookrightarrow \reals$ the Smooth Locus defined by $x^2 = 0$ (in Toposes
that Model Synthetic Differential Geometry this is regarded as an Infinitesimal
Neighborhood about a Point), Differentiation (\S\ref{sec:derivative}) in the
``Context'' $H$ (FIXME: clarify) of Synthetic Differential Geometry $\mathrm{d}$
is the Internal Hom (\S\ref{sec:internal_hom}) out of the Infinitesimal Interval
(\S\ref{sec:infinitesimal_interval}) $D$:
\[
  \mathrm{d} = [D,-] : H \rightarrow H
\]



% ====================================================================
\section{Incidence Geometry}\label{sec:incidence_geometry}
% ====================================================================

Incidence (\S\ref{sec:incidence})



% --------------------------------------------------------------------
\subsection{Incidence Structure}\label{sec:incidence_structure}
% --------------------------------------------------------------------

% --------------------------------------------------------------------
\subsection{Finite Geometry}\label{sec:finite_geometry}
% --------------------------------------------------------------------

study of Finite Incidence Structures (\S\ref{sec:incidence_structure})

cf. Discrete Geometry (\S\ref{sec:discrete_geometry})



\subsubsection{Fano Plane}\label{sec:fano_plane}

Projective Plane (\S\ref{sec:projective_plane})

``the'' Projective Plane of Order $2$ (unique up to Isomorphism)

removing a Line from the Fano Plane, along with the Points on that Line, results
in the Geometry of the Affine Plane of Order 2



% --------------------------------------------------------------------
\subsection{Projective Geometry}\label{sec:projective_geometry}
% --------------------------------------------------------------------

simplest Synthetic expression of any Geometry

Propositions of \emph{Incidence} (\S\ref{sec:incidence}) are True without
exception (cf. Euclidean Geometry where Lines may be Parallel)

\fist cf. Ordered Geometry (\S\ref{sec:ordered_geometry}): no concept of Measure
but with Intermediacy Relation (\S\ref{sec:intermediacy})


\url{https://johncarlosbaez.wordpress.com/2018/12/27/geometric-quantization-part-3/}:
Quantization and Projectivization are Adjoint Functors (FIXME: xref)

2012 - Magaud - \emph{A Case Study in Formalizing Projective Geometry in Coq:
Desargues Theorem}

2017 - Calderon - \emph{Formalizing Constructive Projective Geometry in Agda}
(Agda): \url{https://github.com/GuillermoCalderon/ProjectiveGeometryInAgda}

\textbf{Elliptic Parallel Axiom}: \emph{any two Planes always Intersect in just
  one Line}

\emph{Angles} are not invariant under Projective Transformations
(\S\ref{sec:projective_transformation})

``Subsumes'' Euclidean, Elliptic, and Hyperbolic Geometry: Projective
Plane (\S\ref{sec:projective_plane}) includes Euclidean, Elliptic, and
Hyperbolic Planes and its Symmetry Group contains their Symmetry
Groups

\emph{Projective Space} (\S\ref{sec:projective_space})

Analytic approach: \emph{Homogeneous Coordinates}
(\S\ref{sec:homogeneous_coordinates}) -- Points (including Points at Infinity)
can be represented using Finite Coordinates; allows for Affine Transformations
(\S\ref{sec:affine_transformation}) %FIXME: clarify

``all Projective Geometries are exercises in Linear Algebra'' (E.
Kmett interview -- source?) %FIXME

a Homothety (\S\ref{sec:homothety}) in Projective Geometry is a Similarity
Transformation (\S\ref{sec:similarity_transformation}) that leaves the Line at
Infinity Pointwise Invariant

Projective Model of Elliptic Geometry (\S\ref{sec:elliptic_geometry})

\fist $\mathbb{F}_1$-geometry (\S\ref{sec:f1_geometry}) -- Projective Geometries
over Finite Fields have meaningful analogue for $1$ Element (Tits57)

\fist the Set of One-dimensional (Linear) Subspaces
(\S\ref{sec:linear_subspace}) of a Finite-dimensional Vector Space
(\S\ref{sec:vector_space}) $V$ is the Projective Space of $V$



\subsubsection{Collineation}\label{sec:collineation}

Bijection (Isomorphism) between Projective Spaces such that the Image of two
Collinear Points are Collinear

\fist Homography (Projective Transformation
\S\ref{sec:projective_transformation}): are Collineations %FIXME: clarify



\subsubsection{Correlation}\label{sec:correlation}

reverses Inclusion (Subset) and preserves Incidence



\subsubsection{Projective Range}\label{sec:projective_range}

(wiki): a Set of Points in Projective Geometry considered ``in a unified
fashion''

Dual of a Pencil (\S\ref{sec:pencil}) of Lines on a given Point

may be a Projective Line (\S\ref{sec:projective_line}) or a Conic
(\S\ref{sec:conic_section})

Geometry is Dual of a Pencil of Lines on a given Point, e.g. a Correlation
(\S\ref{sec:correlation}) exchanges the Points of a Projective Range with the
Lines of a Pencil, e.g. a Correlation (\S\ref{sec:correlation}) exchanges the
Points of a Projective Range with the Lines of a Pencil

Conic Range



\paragraph{Projectivity}\label{sec:projectivity}\hfill



\subsubsection{Projective Line}\label{sec:projective_line}

One-dimensional Projective Space

Line (\S\ref{sec:line}) extended by a Point at Infinity



\subsubsection{Projective Plane}\label{sec:projective_plane}

Fano Plane (\S\ref{sec:fano_plane}) -- ``the'' Projective Plane of Order $2$
(unique up to Isomorphism)

includes Euclidean, Elliptic, and Hyperbolic Planes

Symmetry Group contains Symmetry Groups for Euclidean, Elliptic, and
Hyperbolic Planes

(wiki):

Axiomatic Definition: TODO



\paragraph{Plane Duality}\label{sec:plane_duality}\hfill

different type of Symmetry %FIXME



% ====================================================================
\section{Ordered Geometry}\label{sec:ordered_geometry}
% ====================================================================

\emph{Intermediacy} (\S\ref{sec:intermediacy}) without a basic notion of
Measurement

\fist cf. Incidence Geometry (\S\ref{sec:incidence_geometry})

common framework for Euclidean, Hyperbolic, Affine, and Absolute Geometries, but
not Projective Geometry or Elliptic Geometry

Primitive Notions:
\begin{itemize}
  \item Points $A, B, C, \ldots$
  \item Ternary Intermediacy Relation $[ABC]$
\end{itemize}

Axioms:
\begin{enumerate}
  \item all Points are in the same Space (of the chosen Dimension)
  \item there are at least two Points
  \item for all distinct Points $A$ and $B$, there is a third $C$ such that
    $[ABC]$
  \item $[ABC]$ implies $A \neq C$, $[CBA]$ and $\neg[CAB]$
  \item if $C$ and $D$ are distinct Points on Line $AB$ then $A$ is on Line $CD$
  \item if $AB$ is a Line then there is a Point $C$ that is not on $AB$
  \item \emph{Pasch's Axiom}: TODO
  \item \emph{Axiom of Dimensionality}: TODO
  \item \emph{Dedekind's Axiom}: TODO
\end{enumerate}

Affine Geometry adds two Axioms: the \emph{Affine Axiom of Parallelism}, and
\emph{Desargues Axiom}



% --------------------------------------------------------------------
\subsection{Affine Geometry}\label{sec:affine_geometry}
% --------------------------------------------------------------------

Affine Geometry adds two Axioms to those of Ordered Geometry: the \emph{Affine
  Axiom of Parallelism}, and \emph{Desargues Axiom}

Affine Geometry (\S\ref{sec:affine_geometry}) -- Euclidean Geometry without
Metric notions of Distance or Angle (\emph{Congruence} \S\ref{sec:congruence});
does not assume Euclid's third and fourth Axioms (TODO)

see also:
\begin{itemize}
  \item Affine Connection (\S\ref{sec:affine_connection})
  \item ... TODO: more
\end{itemize}



\subsubsection{Parallelism}\label{sec:parallelism}

\emph{Affine Axiom of Parallelism} TODO

\asterism

(wiki):

cases for two Geodesics (\S\ref{sec:geodesic}) belonging to the same Plane:
\begin{itemize}
  \item \emph{Intersecting} -- the Geodesics \emph{Intersect} at a common Point
    in the Plane
  \item \emph{Parallel} -- the Geodesics converge to a common Point in the Plane
    or Limit Point at Infinity
  \item \emph{Ultraparallel} -- the (\emph{Non-intersecting}) Geodesics do not
    have any common limit points
\end{itemize}

\asterism

\fist Elliptic Geometry (\S\ref{sec:elliptic_geometry}): a Line has \emph{no}
  Parallels through a given Point

\fist Planar (Euclidean) Geometry (\S\ref{sec:euclidean_geometry}): a Line has
  exactly \emph{one} Parallel through a given Point

\fist Hyperbolic Geometry (\S\ref{sec:hyperbolic_geometry}): a Line has
\emph{two} Parallels and an Infinite Number of Ultraparallels through a
given Point



\subsubsection{Affine Set}\label{sec:affine_set}

(2004 Boyd-Vandenberghe \S2.1.2):

an Affine Set contains the entire Line through any two Points in the Set; cf. a
Convex Set (\S\ref{sec:convex_set}) contains only the Line Segment between any
two Points

Line through Points $x_1$ and $x_2$:
\[
  x = \theta x_1 + (1-\theta) x_2
\]
for $\theta \in \reals$

every Affine Set is also Convex

\fist in a Convex Set, $\theta$ is restricted to $0 \leq \theta \leq 1$

every Affine Set can be expressed as the Solution Set of a System of Linear
Equations (\S\ref{sec:linear_equation_system})

\begin{itemize}
  \item Hyperplanes (\S\ref{sec:hyperplane}) are Affine and Convex
  \item the Set $\mathsf{S}^n$ of Symmetric $n \times n$ Matrices is Affine and
    Convex
\end{itemize}

\fist Affine Algebraic Set (\S\ref{sec:affine_algebraic_set}), Affine Variety
(\S\ref{sec:affine_variety})



\subsubsection{Affine Line}\label{sec:affine_line}

a Morphism from a Algebraic Variety to the Affine Line is a \emph{Regular Map}
(\S\ref{sec:regular_map})

a Regular Map between Complex Algebraic Varieties is a Holomorphic Map
(\S\ref{sec:holomorphic_function})



\subsubsection{Affine Plane}\label{sec:affine_plane}

\subsubsection{Affine Hyperplane}\label{sec:affine_hyperplane}

Hyperplane (\S\ref{sec:hyperplane})

cf. Affine Manifold (\S\ref{sec:affine_manifold})



\subsubsection{Halfspace}\label{sec:halfspace}

either of two parts into which a Hyperplane (\S\ref{sec:hyperplane}) divides an
Affine Space

Halfspaces are Convex Sets (\S\ref{sec:convex_set})

note Halfspaces are not Vector Spaces since they are not closed under Scalar
Multiplication

\begin{itemize}
  \item Ray -- 1D
  \item Half-plane -- 2D
  \item Half-space -- 3D
\end{itemize}



% --------------------------------------------------------------------
\subsection{Absolute Geometry}\label{sec:absolute_geometry}
% --------------------------------------------------------------------

Euclid's Axioms excluding the Parallel Postulate

\begin{itemize}
  \item Euclidean Geometry (\S\ref{sec:euclidean_geometry})
  \item Hyperbolic Geometry (\S\ref{sec:hyperbolic_geometry}) -- alternative
    Parallel Postulate
\end{itemize}

inconsistent with Elliptic Geometry



% ====================================================================
\section{Metric Geometry}\label{sec:metric_geometry}
% ====================================================================

Metric Spaces (\S\ref{sec:metric_space}); a Topological Space
(\S\ref{sec:topological_space}) is a Metric Space



% --------------------------------------------------------------------
\subsection{Euclidean Geometry}\label{sec:euclidean_geometry}
% --------------------------------------------------------------------

Euclid's Axioms:
\begin{enumerate}
  \item any two Points are connected by a Straight Line
  \item a Line Segment can be extended Continuously in a Straight Line
  \item a Circle may be defined from any Center Point and Distance (Radius)
  \item all Right Angles are equal
  \item \emph{Parallel Postulate}: a Straight Line intersecting two Straight
    Lines such that the Interior Angles on the same side are less than two Right
    Angles, then the two Straight Lines meet on the side on which the Angles are
    less than two Right Angles
\end{enumerate}

\emph{Playfair's Axiom} -- modern version of Euclid's Parallel Postulate

Absolute Geometry (\S\ref{sec:absolute_geometry}) -- Euclidean Geometry
without \emph{Parallelism} (\S\ref{sec:parallelism}); consistent with Hyperbolic
Geometry (\S\ref{sec:hyperbolic_geometry})

(wiki): \emph{Non-euclidean Geometries} (\S\ref{sec:noneuclidean_geometry}) --
Euclidean Geometry lies at the intersection between Affine Geometry
(Parallelism) and \emph{Metric Geometry} (\S\ref{sec:metric_geometry}); a
relaxation of the Metric requirement of Distance and Angle results in Kinematic
Geometries (\S\ref{sec:kinematic_geometry}), while a replacement of the Parallel
Postulate with an alternative yields Hyperbolic
(\S\ref{sec:hyperbolic_geometry}) and Elliptic geometry
(\S\ref{sec:elliptic_geometry})

Affine Geometry (\S\ref{sec:affine_geometry}) -- Euclidean Geometry without
Metric notions of Distance or Angle (\emph{Congruence} \S\ref{sec:congruence})

Planar (Euclidean) Geometry: a Line has exactly \emph{one} Parallel through a
given Point

\fist Elliptic Geometry: a Line has \emph{no} Parallels through a given Point

\fist Hyperbolic Geometry: a Line has \emph{two} Parallels and an Infinite
Number of Ultraparallels through a given Point

Peano's Axioms: TODO

Hilbert's Axioms: (20 Axioms) TODO

Birkhoff's Axioms: (4 Axioms) TODO

Tarski's Axioms (Complete \S\ref{sec:completeness}, Elementary
\S\ref{sec:elementary_theory}): (10 Axioms, 1 Axiom Schema) TODO

\fist leaving out the Parallel Postulate gives Absolute Geometry
(\S\ref{sec:absolute_geometry})

cf. Projective Geometry (\S\ref{sec:projective_geometry})

\emph{Euclidean Space} (\S\ref{sec:euclidean_space})

\emph{Euclidean Group} (\S\ref{sec:euclidean_group}) -- Group of all
Isometries $ISO(n)$ or $E(n)$; makes Euclidean Geometry a case of
Klein Geometry (\S\ref{sec:klein_geometry})

Riemannian Manifold (\S\ref{sec:riemannian_manifold}) of Constant Vanishing
Sectional Curvature (\S\ref{sec:sectional_curvature})

Tarski-Seidenberg Theorem (\S\ref{sec:tarski_seidenberg}): Euclidean Geometry
without the ability to measure Angles is a Model of the Real Field Axioms
and therefore also Decidable

\emph{Cantor-Dedekind Axiom} -- thesis that the Real Numbers
(\S\ref{sec:real_number}) are Order-isomorphic to the Linear Continuum (Order
Theory \S\ref{sec:linear_continuum}) in Geometry; a consequence is that the
Decidability of the Ordered Real Field can be seen as an Algorithm to solve any
problem in Euclidean Geometry



\paragraph{Half-space}\label{sec:half_space}\hfill

Hyperbolic Spaces (\S\ref{sec:hyperbolic_space})



% --------------------------------------------------------------------
\subsection{Non-euclidean Geometry}\label{sec:noneuclidean_geometry}
% --------------------------------------------------------------------

some applets for exploring Non-euclidean Geometries (listed on
\url{https://www.maa.org/press/periodicals/loci/rethinking-pythagoras-0}):

\url{http://www.cs.unm.edu/~joel/NonEuclid/NonEuclid.html}

\url{http://merganser.math.gvsu.edu/easel/applet.html}

\url{https://www.cinderella.de/tiki-index.php}

\url{http://homepages.gac.edu/~hvidsten/explorer/}



\subsubsection{Hyperbolic Geometry}\label{sec:hyperbolic_geometry}

a Line has \emph{two} Parallels and an Infinite Number of \emph{Ultraparallels}
(\S\ref{sec:parallelism}) through a given Point

\fist Planar (Euclidean) Geometry (\S\ref{sec:euclidean_geometry}): a Line has
  exactly \emph{one} Parallel through a given Point

\fist Elliptic Geometry (\S\ref{sec:elliptic_geometry}): a Line has \emph{no}
  Parallels through a given Point

an Absolute Geometry (\S\ref{sec:absolute_geometry})

%FIXME: move ?

Riemannian Manifold (\S\ref{sec:riemannian_manifold}) of Constant Negative
Sectional Curvature (\S\ref{sec:sectional_curvature})

models:
\begin{itemize}
  \item Klein Model
  \item Poincar\'e disk Model
  \item Poincar\'e Half-plane Model
  \item Hyperboloid Model
\end{itemize}



\subsubsection{Elliptic Geometry}\label{sec:elliptic_geometry}

a Line has \emph{no} Parallels through a given Point

cf.:

\fist Planar (Euclidean) Geometry (\S\ref{sec:euclidean_geometry}): a Line has
exactly \emph{one} Parallel through a given Point

\fist Hyperbolic Geometry (\S\ref{sec:hyperbolic_geometry})

Projective Model (TODO)

Riemannian Manifold (\S\ref{sec:riemannian_manifold}) of Constant Positive
Sectional Curvature (\S\ref{sec:sectional_curvature})



\paragraph{Spherical Geometry}\label{sec:spherical_geometry}\hfill

\subparagraph{Spherical Trigonometry}\label{sec:spherical_trigonometry}\hfill



\subsubsection{Kinematic Geometry}\label{sec:kinematic_geometry}\hfill

relaxation of Euclidean Metric requirements (Distance and Angle)

Planar Algebras (\S\ref{sec:planar_algebra})

%FIXME: relation to hyperbolic geometry ???



% ====================================================================
\section{Discrete Geometry}\label{sec:discrete_geometry}
% ====================================================================

or \emph{Combinatorial Geometry} \fist Combinatorics (Part
\ref{part:combinatorics})

cf. Finite Geometry (\S\ref{sec:finite_geometry})



% --------------------------------------------------------------------
\subsection{Polyhedral Combinatorics}\label{sec:polyhedral_combinatorics}
% --------------------------------------------------------------------

Convex Polyhedra (\S\ref{sec:convex_polyhedra}) and Convex Polytopes
(\S\ref{sec:polytope})



% ====================================================================
\section{Convex Geometry}\label{sec:convex_geometry}
% ====================================================================

%FIXME: move this section?

Simplex

Convex Set (\S\ref{sec:convex_set})



% --------------------------------------------------------------------
\subsection{Affine Hull}\label{sec:affine_hull}
% --------------------------------------------------------------------

% --------------------------------------------------------------------
\subsection{Convex Hull}\label{sec:convex_hull}
% --------------------------------------------------------------------

Vertex representation of a Convex Polytope (\S\ref{sec:convex_polytope})

a Convex Polyhedra (\S\ref{sec:convex_polyhedra}) is the Convex Hull of
finitely many Points not all lying in the same Plane



% --------------------------------------------------------------------
\subsection{Convex Combination}\label{sec:convex_combination}
% --------------------------------------------------------------------

an Affine Combination (\S\ref{sec:affine_combination}) where the coefficients
are non-negative

\fist Convex Functions (\S\ref{sec:convex_function})



% ====================================================================
\section{Higher-dimensional Geometry}\label{sec:higher_geometry}
% ====================================================================

% --------------------------------------------------------------------
\subsection{Inversive Geometry}\label{sec:inversive_geometry}
% --------------------------------------------------------------------

% --------------------------------------------------------------------
\subsection{Polytope}\label{sec:polytope}
% --------------------------------------------------------------------

\begin{itemize}
  \item Polygon (\S\ref{sec:polygon}) -- 2D
  \item Polyhedra (\S\ref{sec:polyhedra}) -- 3D
  \item Polychoron (Polycell \S\ref{sec:polychoron}) -- 4D
\end{itemize}



\subsubsection{Convex Polytope}\label{sec:convex_polytope}

Vertex representation is a Convex Hull (\S\ref{sec:convex_hull})

may be defined as an intersection of a finite number of Half-spaces

\fist Polyhedral Combinatorics (\S\ref{sec:polyhedral_combinatorics})



\paragraph{Density}\label{sec:polytope_density}\hfill

(wiki): generalization of ``Winding Number'' (TODO: xref) from two to higher
Dimensions

cf. \empy{Density} (Topology \S\ref{sec:density})



\subsubsection{Uniform Polytope}\label{sec:uniform_polytope}

\paragraph{Regular Polytope}\label{sec:regular_polytope}\hfill

Rectification

Truncation (\S\ref{sec:truncation_operator})



\paragraph{Truncation}\label{sec:truncation_operator}\hfill

\begin{enumerate}
  \item Truncation -- 2D (Polygons) and higher
  \item Cantellation (Rectified Rectification) -- 3D (Polyhedra) and higher
\end{enumerate}



\subsubsection{Star Polytope}\label{sec:star_polytope}

\subsubsection{Simplicial Polytope}\label{sec:simplicial_polytope}

all Facets are Simplices

Deltahedron (\S\ref{sec:deltahedron})



\subsubsection{Net}\label{sec:polytope_net}

\fist not to be confused with Topological Nets (\S\ref{sec:net})

\emph{D\"urer's Conjecture}: does every Convex Polyhedron have a Net?

Ghomi (2014): every Convex Polyhedron admits a net after an Affine
Transformation

it is known that every Convex Uniform $4$-polytope can be cut along
Two-dimensional Faces shared by its Three-dimensional Facets and unfolded into
a single non-overlapping Polyhedron ``Net'' in $3$ Dimensions



\subsubsection{Simplex}\label{sec:simplex}

$n$-simplex $\Delta^n$

Homeomorphic to Closed $n$-ball $D^n$ (\S\ref{sec:closed_ball})

\fist Simplicial Complex (Algebraic Topology \S\ref{sec:simplicial_complex})



% --------------------------------------------------------------------
\subsection{Polygon}\label{sec:polygon}
% --------------------------------------------------------------------

2D Polytope

a Plane Figure bounded by a Closed Polygonal Chain (Finite Chain of Straight
Line Segments, i.e. \emph{Sides} or \emph{Edges}) or ``\emph{Circuit}''



\subsubsection{Dissection}\label{sec:dissection}

\paragraph{Equidissection}\label{sec:equidissection}\hfill

\subparagraph{Spectrum}\label{sec:polygon_spectrum}\hfill

the Set of all $n$ for which an $n$-equidissection of a Polygon $P$ exists



% --------------------------------------------------------------------
\subsection{Polyhedra}\label{sec:polyhedra}
% --------------------------------------------------------------------

3D ``Solid''

Nets of Polyhedra (\S\ref{sec:polytope_net})

\emph{D\"urer's Conjecture}: does every Convex Polyhedron have a Net?

Ghomi (2014): every Convex Polyhedron admits a net after an Affine
Transformation



\subsubsection{Convex Polyhedra}\label{sec:convex_polyhedra}

Convex Hull (\S\ref{sec:convex_hull}) of finitely many Points not all on the
same Plane

\fist Polyhedral Combinatorics (\S\ref{sec:polyhedral_combinatorics})



\paragraph{Platonic Solid}\label{sec:platonic_solid}\hfill

\paragraph{Archimedean Solid}\label{sec:archimedean_solid}\hfill

\paragraph{Johnson Solid}\label{sec:johnson_solid}\hfill

Non-uniform Convex Polyhedron with Regular Polygons for faces

\begin{itemize}
  \item Pyramids
  \item Cupolae and Rotundae
  \item Elongated and Gyroelongated Pyramids
  \item Bipyramids, Elongated Bipyramids, Gyroelongated Bipyramids
  \item Elongated Cupolae and Elongated Rotundae
  \item Bicupolae
  \item Bianticupolae
  \item Cupola-rotundae and Birodunta
  \item Triangular Hebesphenorotunda -- only Johnson Solid with Faces of 3, 4,
    5, and 6 Sides
  \item ... MORE
\end{itemize}



\paragraph{Cuboid}\label{sec:cuboid}\hfill

\begin{enumerate}
  \item Hilbert Cube (\S\ref{sec:hilbert_cube})
  \item ... MORE
\end{enumerate}



\paragraph{Pyramid}\label{sec:pyramid}\hfill

\paragraph{Prism}\label{sec:prism}\hfill

\paragraph{Antiprism}\label{sec:antiprism}\hfill

\paragraph{Rotunda}\label{sec:rotunda}\hfill

\paragraph{Cupola}\label{sec:cupola}\hfill



\subsubsection{Deltahedron}\label{sec:deltahedron}

Simplicial Polytope (\S\ref{sec:simplicial_polytope})



\subsubsection{Uniform Polyhedra}\label{sec:uniform_polyhedra}

\fist Uniform Polytope (\S\ref{sec:uniform_polytope})



% --------------------------------------------------------------------
\subsection{Polychoron}\label{sec:polychoron}
% --------------------------------------------------------------------

A \emph{Polycell} or \emph{Polychoron} is a $4$-polytope.



\subsubsection{Uniform Polychoron}\label{sec:uniform_polychoron}

it is known that every Convex Uniform $4$-polytope can be cut along
Two-dimensional Faces shared by its Three-dimensional Facets and unfolded into
a single non-overlapping Polyhedron ``Net'' in $3$ Dimensions



% ====================================================================
\section{Synthetic Differential Geometry}
\label{sec:synthetic_differential_geometry}
% ====================================================================

Differential Geometry (\S\ref{sec:differential_geometry}) formalized in the
language of Topos Theory (\S\ref{sec:topos_theory})

Differentials (\S\ref{sec:differential}) in Smooth Models of Set Theory
(TODO: clarify)

Formalization of Infinitesimal Objects (\S\ref{sec:infinitesimal_object})

(nlab):

Differentiation (\S\ref{sec:derivative}) (Endo-)functor $\mathrm{d} : \cat{Diff}
\rightarrow \cat{Diff}$ on the Category of Smooth Manifolds and Smooth Maps --
in the ``Context'' $H$ (FIXME: clarify) of Synthetic Differential Geometry
$\mathrm{d}$ is the Internal Hom (\S\ref{sec:internal_hom}) out of the
Infinitesimal Interval (\S\ref{sec:infinitesimal_interval}) $D$:
\[
  \mathrm{d} = [D,-] : H \rightarrow H
\]
where $D \hookrightarrow \reals$ is the Smooth Locus (\S\ref{sec:smooth_locus})
defined by $x^2 = 0$ (in Toposes that Model Synthetic Differential Geometry this
is regarded as an Infinitesimal Neighborhood about a Point)



% --------------------------------------------------------------------
\subsection{Infinitesimal Space}\label{sec:infinitesimal_space}
% --------------------------------------------------------------------

\fist Infinitesimal Object (\S\ref{sec:infinitesimal_object})

(nlab):

Differential $1$-forms (\S\ref{sec:differential_form}) are Functions on the
Space of Infinitesimal Paths (Stel13 - \emph{Cosimplicial $C^\infty$ rings and
  the de Rham complex of Euclidean space})



% --------------------------------------------------------------------
\subsection{Smooth Infinitesimal Analysis}
\label{sec:smooth_infinitesimal_analysis}
% --------------------------------------------------------------------



% ====================================================================
\section{Noncommutative Geometry}\label{sec:noncommutative_geometry}
% ====================================================================

Noncommutative Algebras (\S\ref{sec:noncommutative_algebra})

Category of Noncommutative Affine Schemes (\S\ref{sec:affine_scheme}) as Dual of
the Category of Associative Unital Rings

Cyclic Homology (\S\ref{sec:cyclic_homology})



% --------------------------------------------------------------------
\subsection{Fuzzy Sphere}\label{sec:fuzzy_sphere}
% --------------------------------------------------------------------

Hasebe2010 - \emph{Hopf Maps, Lowest Landau Level, and Fuzzy Spheres}
