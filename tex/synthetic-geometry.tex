%%%%%%%%%%%%%%%%%%%%%%%%%%%%%%%%%%%%%%%%%%%%%%%%%%%%%%%%%%%%%%%%%%%%%%
%%%%%%%%%%%%%%%%%%%%%%%%%%%%%%%%%%%%%%%%%%%%%%%%%%%%%%%%%%%%%%%%%%%%%%
\part{Synthetic Geometry}\label{part:synthetic_geometry}
%%%%%%%%%%%%%%%%%%%%%%%%%%%%%%%%%%%%%%%%%%%%%%%%%%%%%%%%%%%%%%%%%%%%%%
%%%%%%%%%%%%%%%%%%%%%%%%%%%%%%%%%%%%%%%%%%%%%%%%%%%%%%%%%%%%%%%%%%%%%%

``Axiomatic Geometry''



% ====================================================================
\section{Euclidean Geometry}\label{sec:euclidean_geometry}
% ====================================================================

$E^n$ -- $n$-dimensional Euclidean Space (Euclidean Space
\S\ref{sec:euclidean_space}) with Cartesian Coordinates is Modelled by
$\reals^n$ ($n$-dimensional Real Coordinate Space
\S\ref{sec:real_coordinate_space})

\fist cf. Homogenous Coordinates (\S\ref{sec:homogenous_coordinate})
in Projective Geometry (\S\ref{sec:projective_geometry})

Riemannian Manifold (\S\ref{sec:riemannian_manifold}) of Constant Vanishing
Sectional Curvature (\S\ref{sec:sectional_curvature})

\emph{Euclidean Group} (\S\ref{sec:euclidean_group}) -- Group of all
Isometries $ISO(n)$ or $E(n)$; makes Euclidean Geometry a case of
Klein Geometry (\S\ref{sec:klein_geometry})



% --------------------------------------------------------------------
\subsection{Half-space}\label{sec:half_space}
% --------------------------------------------------------------------

% --------------------------------------------------------------------
\subsection{Convex Geometry}\label{sec:convex_geometry}
% --------------------------------------------------------------------

%FIXME:

Simplex



\subsubsection{Affine Hull}\label{sec:affine_hull}

\subsubsection{Convex Hull}\label{sec:convex_hull}



% --------------------------------------------------------------------
\subsection{Translation}\label{sec:translation}
% --------------------------------------------------------------------

% --------------------------------------------------------------------
\subsection{Reflection}\label{sec:reflection}
% --------------------------------------------------------------------

Isometry (\S\ref{sec:isometry})

Reflection Groups (\S\ref{sec:reflection_group}) are Discrete Groups Generated
by a Set of Reflections of a Finite-dimensional Euclidean Space
(\S\ref{sec:reflection})

The Coxeter Groups (\S\ref{sec:coxeter_group}) are precisely the Finite
Euclidean Reflection Groups



% --------------------------------------------------------------------
\subsection{Rotation}\label{sec:rotation}
% --------------------------------------------------------------------



% ====================================================================
\section{Non-euclidean Geometry}\label{sec:noneuclidean_geometry}
% ====================================================================

some applets for exploring Non-euclidean Geometries (listed on
\url{https://www.maa.org/press/periodicals/loci/rethinking-pythagoras-0}):

\url{http://www.cs.unm.edu/~joel/NonEuclid/NonEuclid.html}

\url{http://merganser.math.gvsu.edu/easel/applet.html}

\url{https://www.cinderella.de/tiki-index.php}

\url{http://homepages.gac.edu/~hvidsten/explorer/}



% --------------------------------------------------------------------
\subsection{Hyperbolic Geometry}\label{sec:hyperbolic_geometry}
% --------------------------------------------------------------------

Riemannian Manifold (\S\ref{sec:riemannian_manifold}) of Constant Negative
Sectional Curvature (\S\ref{sec:sectional_curvature})



\subsubsection{Hyperbolic Space}\label{sec:hyperbolic_space}

Homogenous Space (\S\ref{sec:homogenous_space}) with Constant Negative
Sectional Curvature (\S\ref{sec:sectional_curvature})



% --------------------------------------------------------------------
\subsection{Elliptic Geometry}\label{sec:elliptic_geometry}
% --------------------------------------------------------------------

Riemannian Manifold (\S\ref{sec:riemannian_manifold}) of Constant Positive
Sectional Curvature (\S\ref{sec:sectional_curvature})



% ====================================================================
\section{Projective Geometry}\label{sec:projective_geometry}
% ====================================================================

Projective Space (\S\ref{sec:projective_space})

Homogenous Coordinates (\S\ref{sec:homogenous_coordinate}) -- Points
(including Points at Infinity) can be represented using Finite
Coordinates; allows for Affine Transformations
(\S\ref{sec:affine_transformation})

cf. Cartesian Coordinates (\S\ref{sec:cartesian_coordinate}) in
Euclidean Geometry (\S\ref{sec:euclidean_geometry})

``Subsumes'' Euclidean, Elliptic, and Hyperbolic Geometry: Projective
Plane (\S\ref{sec:projective_plane}) includes Euclidean, Elliptic, and
Hyperbolic Planes and its Symmetry Group contains their Symmetry
Groups

``all Projective Geometries are exercises in Linear Algebra'' (E.
Kmett interview -- source?) %FIXME



% --------------------------------------------------------------------
\subsection{Duality}\label{sec:projective_duality}
% --------------------------------------------------------------------

different type of Symmetry %FIXME



% --------------------------------------------------------------------
\subsection{Conic Section}\label{sec:conic_section}
% --------------------------------------------------------------------

% --------------------------------------------------------------------
\subsection{Quadric}\label{sec:quadric}
% --------------------------------------------------------------------



% ====================================================================
\section{Affine Geometry}\label{sec:affine_geometry}
% ====================================================================

% ====================================================================
\section{Differential Geometry}\label{sec:differential_geometry}
% ====================================================================

Differentiable Manifolds (\S\ref{sec:differentiable_manifold})



% --------------------------------------------------------------------
\subsection{Connection}\label{sec:connection}
% --------------------------------------------------------------------

\emph{Flat Connections} %TODO



\subsubsection{Holonomy}\label{sec:holonomy}

a \emph{Holonomy} of a Connection on a Smooth Manifold
(\S\ref{sec:smooth_manifold})

for Flat Connections, the associated Holonomy is a type of Monodromy
(\S\ref{sec:monodromy})



% --------------------------------------------------------------------
\subsection{Riemannian Geometry}\label{sec:riemannian_geometry}
% --------------------------------------------------------------------

Riemannian Manifold (\S\ref{sec:riemannian_manifold})

Pseudo-Riemannian Manifold (\S\ref{sec:pseudo_riemannian})



\subsubsection{Sectional Curvature}\label{sec:sectional_curvature}

%FIXME: xref curvature

Riemannian Manifolds (\S\ref{sec:riemannian _manifold}) with Constant Sectional
Curvature:
\begin{itemize}
  \item Euclidean Geometry (\S\ref{sec:euclidean_geometry}) -- Constant
    Vanishing Sectional Curvature
  \item Hyperbolic Geometry (\S\ref{sec:hyperbolic_geometry}) -- Constant
    Negative Sectional Curvature
  \item Elliptic Geometry (\S\ref{sec:elliptic_geometry}) -- Constant Positive
    Sectional Curvature
\end{itemize}



\subsubsection{Symmetric Space}\label{sec:symmetric_space}

a Pseudo-Riemannian Manifold (\S\ref{sec:pseudo_riemannian}) whose Group of
Symmetries (\S\ref{sec:symmetry_group}) contains an \emph{Inversion Symmetry}
about every Point



% --------------------------------------------------------------------
\subsection{Symplectic Geometry}\label{sec:symplectic_geometry}
% --------------------------------------------------------------------

% --------------------------------------------------------------------
\subsection{Lie Theory}\label{sec:lie_theory}
% --------------------------------------------------------------------

Lie Group (\S\ref{sec:lie_group})

Lie Algebra (\S\ref{sec:lie_algebra})

Lie Group-Lie Algebra Correspondence

uses Lie Groups used for analysing the Continuous Symmetries of
Differential Equations %FIXME

cf. Galois Theory (\S\ref{sec:galois_theory}) uses Permutation Groups
for analysing the Discrete Symmetries of Algebraic Equations %FIXME



\subsubsection{Lie Group}\label{sec:lie_group}\hfill

Continuous Transformation Group
(\S\ref{sec:continuous_transformation_group}) that is a Smooth
Differentiable Manifold (\S\ref{sec:differentiable_manifold})

Any Discrete Group (\S\ref{sec:discrete_group}) can be viewed as a
$0$-dimensional Lie Group %FIXME



\paragraph{Dynkin Diagram}\label{sec:dynkin_diagram}\hfill

cf. Coxeter-Dynkin Diagrams (\S\ref{sec:coxeter_dynkin_diagram})



\paragraph{Special Unitary Group}\label{sec:special_unitary}\hfill

$\mathrm{SU}(n)$ -- the Lie Group of $n \times n$ Unitary Matrices
(\S\ref{sec:unitary_matrix}) with Determinant $1$



\subsubsection{Lie Algebra}\label{sec:lie_algebra}

(or \emph{Infinitesimal Group})

Ininitesimal Transformations
(\S\ref{sec:infinitesimal_transformation})

Vector Space (\S\ref{sec:vector_space}) with a Non-associative
Multiplication called a \emph{Lie Bracket} $[x,y]$

when an Algebraic Product is defined on the Space, the Lie Bracket is
the Commutator $[x,y] = xy - yx$ %FIXME



% ====================================================================
\section{Klein Geometry}\label{sec:klein_geometry}
% ====================================================================

%FIXME this is a larger categorization of geometries, possibly
%reorganize others under this heading

a \emph{Klein Geometry} is a pair $(G,H)$ of a Lie Group
(\S\ref{sec:lie_group}) $G$ and $H$ a Closed Lie Subgroup of $G$ such
that the (Left) Coset Space (\S\ref{sec:coset_space}) $G|H$ is
Connected (\S\ref{sec:connected_space})

$G|H$ is called the \emph{Space} of the Geometry and is a Smooth
Manifold (\S\ref{sec:smooth_manifold}) of Dimension:
\[
  dim(X) = dim(G) - dim(H)
\]

\emph{Erlangen Program} -- Projective Geometry as least restrictive
``unifying frame'' for other Geometries considered; from least to more
restrictive: Projective Geometry, Affine Geometry, Euclidean Geometry

examples: %FIXME

Projective Geometry (\S\ref{sec:projective_geometry})

Conformal Geometry (\S\ref{sec:conformal_geometry}) on a Sphere

Hyperbolic Geometry

Elliptic Geometry

Spherical Geometry

Affine Geometry

Euclidean Geometry
