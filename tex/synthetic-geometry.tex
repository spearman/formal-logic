%%%%%%%%%%%%%%%%%%%%%%%%%%%%%%%%%%%%%%%%%%%%%%%%%%%%%%%%%%%%%%%%%%%%%%
%%%%%%%%%%%%%%%%%%%%%%%%%%%%%%%%%%%%%%%%%%%%%%%%%%%%%%%%%%%%%%%%%%%%%%
\part{Synthetic Geometry}\label{part:synthetic_geometry}
%%%%%%%%%%%%%%%%%%%%%%%%%%%%%%%%%%%%%%%%%%%%%%%%%%%%%%%%%%%%%%%%%%%%%%
%%%%%%%%%%%%%%%%%%%%%%%%%%%%%%%%%%%%%%%%%%%%%%%%%%%%%%%%%%%%%%%%%%%%%%

``Axiomatic Geometry''



% ====================================================================
\section{Projective Geometry}\label{sec:projective_geometry}
% ====================================================================

Projective Space (\S\ref{sec:projective_space})

Homogenous Coordinates (\S\ref{sec:homogenous_coordinate}) -- Points
(including Points at Infinity) can be represented using Finite
Coordinates; allows for Affine Transformations
(\S\ref{sec:affine_transformation})

cf. Cartesian Coordinates (\S\ref{sec:cartesian_coordinate}) in
Euclidean Geometry (\S\ref{sec:euclidean_geomtry})



% --------------------------------------------------------------------
\subsection{Duality}\label{sec:projective_duality}
% --------------------------------------------------------------------

different type of Symmetry %FIXME



% --------------------------------------------------------------------
\subsection{Conic Section}\label{sec:conic_section}
% --------------------------------------------------------------------

% --------------------------------------------------------------------
\subsection{Quadric}\label{sec:quadric}
% --------------------------------------------------------------------



% ====================================================================
\section{Euclidean Geometry}\label{sec:euclidean_geometry}
% ====================================================================

Cartesian Coordinates (\S\ref{sec:cartesian_coordinate})

cf. Homogenous Coordinates (\S\ref{sec:homogenous_coordinate}) in
Projective Geometry (\S\ref{sec:projective_geomtry})



% --------------------------------------------------------------------
\subsection{Half-space}\label{sec:half_space}
% --------------------------------------------------------------------

% --------------------------------------------------------------------
\subsection{Convex Geometry}\label{sec:convex_geometry}
% --------------------------------------------------------------------

\subsubsection{Affine Hull}\label{sec:affine_hull}

\subsubsection{Convex Hull}\label{sec:convex_hull}



% --------------------------------------------------------------------
\subsection{Translation}\label{sec:translation}
% --------------------------------------------------------------------

% --------------------------------------------------------------------
\subsection{Reflection}\label{sec:reflection}
% --------------------------------------------------------------------

Isometry (\S\ref{sec:isometry})



% --------------------------------------------------------------------
\subsection{Rotation}\label{sec:rotation}
% --------------------------------------------------------------------



% ====================================================================
\section{Affine Geometry}\label{sec:affine_geometry}
% ====================================================================

% ====================================================================
\section{Non-euclidean Geometry}\label{sec:noneuclidean_geometry}
% ====================================================================

% --------------------------------------------------------------------
\subsection{Hyperbolic Geometry}\label{sec:hyperbolic_geometry}
% --------------------------------------------------------------------

% --------------------------------------------------------------------
\subsection{Elliptic Geometry}\label{sec:elliptic_geometry}
% --------------------------------------------------------------------



% ====================================================================
\section{Differential Geometry}\label{sec:differential_geometry}
% ====================================================================

% --------------------------------------------------------------------
\subsection{Riemannian Geometry}\label{sec:riemannian_geometry}
% --------------------------------------------------------------------

Riemannian Manifold (\S\ref{sec:riemannian_manifold})



% --------------------------------------------------------------------
\subsection{Symplectic Geometry}\label{sec:symplectic_geometry}
% --------------------------------------------------------------------

% --------------------------------------------------------------------
\subsection{Lie Theory}\label{sec:lie_theory}
% --------------------------------------------------------------------

Lie Group (\S\ref{sec:lie_group})

Lie Algebra (\S\ref{sec:lie_algebra})

Lie Group-Lie Algebra Correspondence

uses Lie Groups used for analysing the Continuous Symmetries of
Differential Equations %FIXME

cf. Galois Theory (\S\ref{sec:galois_theory}) uses Permutation Groups
for analysing the Discrete Symmetries of Algebraic Equations %FIXME



\subsubsection{Lie Group}\label{sec:lie_group}\hfill

Continuous Transformation Group
(\S\ref{sec:continuous_transformation_group}) that is a Smooth
Differentiable Manifold (\S\ref{sec:differentiable_manifold})

Dynkin Diagrams



\subsubsection{Lie Algebra}\label{sec:lie_algebra}

(or \emph{Infinitesimal Group})

Ininitesimal Transformations
(\S\ref{sec:infinitesimal_transformation})

Vector Space (\S\ref{sec:vector_space}) with a Non-associative
Multiplication called a \emph{Lie Bracket} $[x,y]$

when an Algebraic Product is defined on the Space, the Lie Bracket is
the Commutator $[x,y] = xy - yx$ %FIXME
