%%%%%%%%%%%%%%%%%%%%%%%%%%%%%%%%%%%%%%%%%%%%%%%%%%%%%%%%%%%%%%%%%%%%%%
%%%%%%%%%%%%%%%%%%%%%%%%%%%%%%%%%%%%%%%%%%%%%%%%%%%%%%%%%%%%%%%%%%%%%%
\part{Synthetic Geometry}\label{part:synthetic_geometry}
%%%%%%%%%%%%%%%%%%%%%%%%%%%%%%%%%%%%%%%%%%%%%%%%%%%%%%%%%%%%%%%%%%%%%%
%%%%%%%%%%%%%%%%%%%%%%%%%%%%%%%%%%%%%%%%%%%%%%%%%%%%%%%%%%%%%%%%%%%%%%

``Axiomatic Geometry''



% ====================================================================
\section{Euclidean Geometry}\label{sec:euclidean_geometry}
% ====================================================================

$E^n$ -- $n$-dimensional Euclidean Space (Euclidean Space
\S\ref{sec:euclidean_space}) with Cartesian Coordinates is Modelled by
$\reals^n$ ($n$-dimensional Real Coordinate Space
\S\ref{sec:real_coordinate_space})

\fist cf. Homogenous Coordinates (\S\ref{sec:homogenous_coordinate})
in Projective Geometry (\S\ref{sec:projective_geometry})

Riemannian Manifold (\S\ref{sec:riemannian_manifold}) of Constant Vanishing
Sectional Curvature (\S\ref{sec:sectional_curvature})

\emph{Euclidean Group} (\S\ref{sec:euclidean_group}) -- Group of all
Isometries $ISO(n)$ or $E(n)$; makes Euclidean Geometry a case of
Klein Geometry (\S\ref{sec:klein_geometry})



% --------------------------------------------------------------------
\subsection{Half-space}\label{sec:half_space}
% --------------------------------------------------------------------

Hyperbolic Spaces (\S\ref{sec:hyperbolic_space})



% --------------------------------------------------------------------
\subsection{Convex Geometry}\label{sec:convex_geometry}
% --------------------------------------------------------------------

%FIXME:

Simplex



\subsubsection{Affine Hull}\label{sec:affine_hull}

\subsubsection{Convex Hull}\label{sec:convex_hull}



% --------------------------------------------------------------------
\subsection{Translation}\label{sec:translation}
% --------------------------------------------------------------------

Affine Transformation (\S\ref{sec:affine_transformation})

\fist Periodic Functions (\S\ref{sec:periodic_function}) are Functions with
Graphs that exhibit Translational Symmetry (\S\ref{sec:symmetry_group})



% --------------------------------------------------------------------
\subsection{Reflection}\label{sec:reflection}
% --------------------------------------------------------------------

Affine Transformation (\S\ref{sec:affine_transformation})

Isometry (\S\ref{sec:isometry})

Reflection Groups (\S\ref{sec:reflection_group}) are Discrete Groups Generated
by a Set of Reflections of a Finite-dimensional Euclidean Space
(\S\ref{sec:reflection})

The Coxeter Groups (\S\ref{sec:coxeter_group}) are precisely the Finite
Euclidean Reflection Groups



% --------------------------------------------------------------------
\subsection{Rotation}\label{sec:rotation}
% --------------------------------------------------------------------

Affine Transformation (\S\ref{sec:affine_transformation})



% --------------------------------------------------------------------
\subsection{Similarity Transformation}\label{sec:similarity_transformation}
% --------------------------------------------------------------------

in Projective Geometry (\S\ref{sec:projective_geometry}) a Similarity
Transformation fixes a given Elliptic Involution (FIXME: clarify)

in Projective Geometry a Homothety (\S\ref{sec:homothety}) is a Similarity
Transformation that leaves the Line at Infinity Pointwise Invariant

\fist cf. Similarity Measure (\S\ref{sec:similarity_measure})



% ====================================================================
\section{Non-euclidean Geometry}\label{sec:noneuclidean_geometry}
% ====================================================================

some applets for exploring Non-euclidean Geometries (listed on
\url{https://www.maa.org/press/periodicals/loci/rethinking-pythagoras-0}):

\url{http://www.cs.unm.edu/~joel/NonEuclid/NonEuclid.html}

\url{http://merganser.math.gvsu.edu/easel/applet.html}

\url{https://www.cinderella.de/tiki-index.php}

\url{http://homepages.gac.edu/~hvidsten/explorer/}



% --------------------------------------------------------------------
\subsection{Hyperbolic Geometry}\label{sec:hyperbolic_geometry}
% --------------------------------------------------------------------

Riemannian Manifold (\S\ref{sec:riemannian_manifold}) of Constant Negative
Sectional Curvature (\S\ref{sec:sectional_curvature})



\subsubsection{Hyperbolic Space}\label{sec:hyperbolic_space}

Homogenous Space (\S\ref{sec:homogenous_space}) with Constant Negative
Sectional Curvature (\S\ref{sec:sectional_curvature})

Upper Half-space (\S\ref{sec:half_space})



% --------------------------------------------------------------------
\subsection{Elliptic Geometry}\label{sec:elliptic_geometry}
% --------------------------------------------------------------------

Riemannian Manifold (\S\ref{sec:riemannian_manifold}) of Constant Positive
Sectional Curvature (\S\ref{sec:sectional_curvature})



% ====================================================================
\section{Projective Geometry}\label{sec:projective_geometry}
% ====================================================================

\emph{Homogenous Coordinates} (or \emph{Projective Coordinates}
\S\ref{sec:homogenous_coordinate}) -- Points (including Points at Infinity) can
be represented using Finite Coordinates; allows for Affine Transformations
(\S\ref{sec:affine_transformation})

cf. Cartesian Coordinates (\S\ref{sec:cartesian_coordinate}) in
Euclidean Geometry (\S\ref{sec:euclidean_geometry})

``Subsumes'' Euclidean, Elliptic, and Hyperbolic Geometry: Projective
Plane (\S\ref{sec:projective_plane}) includes Euclidean, Elliptic, and
Hyperbolic Planes and its Symmetry Group contains their Symmetry
Groups

``all Projective Geometries are exercises in Linear Algebra'' (E.
Kmett interview -- source?) %FIXME

a Homothety (\S\ref{sec:homothety}) in Projective Geometry is a Similarity
Transformation (\S\ref{sec:similarity_transformation}) that leaves the Line at
Infinity Pointwise Invariant



% --------------------------------------------------------------------
\subsection{Incidence}\label{sec:incidence}
% --------------------------------------------------------------------

%FIXME: move this section ?

an \emph{Incidence Relation} is a Binary Relation between different types of
objects (FIXME: clarify)

\emph{Incidence Propositions}

\fist cf. Colinearity (TODO)

\fist Incidence Structure (Finite Geometry \S\ref{sec:incidence_structure})



% --------------------------------------------------------------------
\subsection{Projective Space}\label{sec:projective_space}
% --------------------------------------------------------------------

Homogenous Coordinates (\S\ref{sec:homogenous_coordinate})

Projective Space is a Homogenous Space (\S\ref{sec:homogenous_space}) for its
Symmetry Groups (\S\ref{sec:symmetry_groups})



% --------------------------------------------------------------------
\subsection{Homogenous Coordinate}\label{sec:homogenous_coordinate}
% --------------------------------------------------------------------

or \emph{Projective Coordinates}

Projective Geometry (\S\ref{sec:projective_geometry}), Projective
Space (\S\ref{sec:projective_space}) -- Points (including Points at
Infinity) can be represented using Finite Coordinates; allows for
Affine Transformations (\S\ref{sec:affine_transformation})

cf. Cartesian Coordinates (\S\ref{sec:cartesian_coordinate}) in
Euclidean Geometry (\S\ref{sec:euclidean_geometry})



% --------------------------------------------------------------------
\subsection{Projective Transformation}
\label{sec:projective_transformation}
% --------------------------------------------------------------------

a \emph{Projective Transformation} (or \emph{Homography}) is an
Isomorphism of Projective Spaces induced by an Isomorphism of the
Vector Spaces from which the Projective Spaces derive

Collineation



% --------------------------------------------------------------------
\subsection{Homogenous Function}\label{sec:homogenous_function}
% --------------------------------------------------------------------

Scale Invariance (\S\ref{sec:scale_invariance})



\subsubsection{Homogenous Polynomial}\label{sec:homogenous_polynomial}

Polynomial (\S\ref{sec:polynomial})



\paragraph{Projective Variety}\label{sec:projective_variety}\hfill

Algebraic Variety (\S\ref{sec:algebraic_variety})



% --------------------------------------------------------------------
\subsection{Projective Plane}\label{sec:projective_plane}
% --------------------------------------------------------------------

includes Euclidean, Elliptic, and Hyperbolic Planes

Symmetry Group contains Symmetry Groups for Euclidean, Elliptic, and
Hyperbolic Planes



% --------------------------------------------------------------------
\subsection{Duality}\label{sec:projective_duality}
% --------------------------------------------------------------------

different type of Symmetry %FIXME



% --------------------------------------------------------------------
\subsection{Conic Section}\label{sec:conic_section}
% --------------------------------------------------------------------

% --------------------------------------------------------------------
\subsection{Quadric}\label{sec:quadric}
% --------------------------------------------------------------------

% --------------------------------------------------------------------
\subsection{Fano Plane}\label{sec:fano_plane}
% --------------------------------------------------------------------

% --------------------------------------------------------------------
\subsection{Grassmanian}\label{sec:grassmanian}
% --------------------------------------------------------------------

\subsubsection{Real Projective Space}\label{sec:real_projective_space}

Compact (\S\ref{sec:compact_space}) Smooth Manifold
(\S\ref{sec:smooth_manifold})



\paragraph{Real Projective Plane}\label{sec:real_projective_plane}\hfill

(or \emph{Extended Euclidean Plane})



\subsubsection{Complex Projective Space}
\label{sec:complex_projective_space}



% ====================================================================
\section{Finite Geometry}\label{sec:finite_geometry}
% ====================================================================

% --------------------------------------------------------------------
\subsection{Incidence Structure}\label{sec:incidence_structure}
% --------------------------------------------------------------------

Incidence (\S\ref{sec:incidence})



% ====================================================================
\section{Affine Geometry}\label{sec:affine_geometry}
% ====================================================================

% ====================================================================
\section{Synthetic Differential Geometry}
\label{sec:synthetic_differential_geometry}
% ====================================================================

Differential Geometry (\S\ref{sec:differential_geometry}) formalized in the
language of Topos Theory (\S\ref{sec:topos_theory})



% ====================================================================
\section{Klein Geometry}\label{sec:klein_geometry}
% ====================================================================

%FIXME this is a larger categorization of geometries, possibly
%reorganize others under this heading

a \emph{Klein Geometry} is a pair $(G,H)$ of a Lie Group
(\S\ref{sec:lie_group}) $G$ and $H$ a Closed Lie Subgroup of $G$ such
that the (Left) Coset Space (\S\ref{sec:coset_space}) $G|H$ is
Connected (\S\ref{sec:connected_space})

$G|H$ is called the \emph{Space} of the Geometry and is a Smooth
Manifold (\S\ref{sec:smooth_manifold}) of Dimension:
\[
  dim(X) = dim(G) - dim(H)
\]

\emph{Erlangen Program} -- Projective Geometry as least restrictive
``unifying frame'' for other Geometries considered; from least to more
restrictive: Projective Geometry, Affine Geometry, Euclidean Geometry

examples: %FIXME

Projective Geometry (\S\ref{sec:projective_geometry})

Conformal Geometry (\S\ref{sec:conformal_geometry}) on a Sphere

Hyperbolic Geometry

Elliptic Geometry

Spherical Geometry

Affine Geometry

Euclidean Geometry
