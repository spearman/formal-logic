%%%%%%%%%%%%%%%%%%%%%%%%%%%%%%%%%%%%%%%%%%%%%%%%%%%%%%%%%%%%%%%%%%%%%%
%%%%%%%%%%%%%%%%%%%%%%%%%%%%%%%%%%%%%%%%%%%%%%%%%%%%%%%%%%%%%%%%%%%%%%
\part{Formal Logic}\label{sec:formal_logic}
%%%%%%%%%%%%%%%%%%%%%%%%%%%%%%%%%%%%%%%%%%%%%%%%%%%%%%%%%%%%%%%%%%%%%%
%%%%%%%%%%%%%%%%%%%%%%%%%%%%%%%%%%%%%%%%%%%%%%%%%%%%%%%%%%%%%%%%%%%%%%

\emph{Formal Logic} (or \emph{Symbolic Logic}) is the study of Logical
Systems (\S\ref{sec:logical_system}), that is, Formal Systems
(\S\ref{sec:formal_system}) with an underlying Language that is
suitable for the particular Logic under consideration, together with a
form of Semantics (Part \ref{sec:formal_semantics}), usually as a
Model-theoretic Interpretation (\S\ref{sec:interpretation}).



% ====================================================================
\section{Logical System}\label{sec:logical_system}
% ====================================================================

A \emph{Logical System} (or \emph{System of Logic} or just
``\emph{Logic}'') is a Formal System (\S\ref{sec:formal_system}) together
with a form of Semantics (Part \ref{sec:formal_semantics}) that gives
Meanings to the Formulas in the underlying Language. Symbols in the
Language are broadly divided into \emph{Variables} or
\emph{Constants}.

Constants may be further divided into \emph{Logical Constants} and
\emph{Non-Logical Constants}.

Symbols that don't have individual Meaning in isolation are called
\emph{Syncategorematic} (see Term Logic \S\ref{sec:term_logic}),
otherwise they are called \emph{Categorematic}.

\emph{Extra-logical Symbols} are those of a Metalanguage
(\S\ref{sec:metalanguage}), such as the Symbol for Logical Consequence
(Entailment \S\ref{sec:logical_consequence}), $\vdash$, or
Metavariables (\emph{Schematic Variables}), $\varphi, \psi, \ldots$.

Other Symbols may be used for \emph{Punctuation}, like Parentheses and
Commas, as long as they don't appear in one of the other kinds of
Symbols.



% --------------------------------------------------------------------
\subsection{Variable}\label{sec:variable}
% --------------------------------------------------------------------

A \emph{Variable} is a placeholder that ranges over the objects in the
Domain of Discourse (\S\ref{sec:domain}).

Variables are called \emph{Bound Variables} or \emph{Free Variables}
depending on whether or not they are Quantified
(\S\ref{sec:quantifier}).

Inductive definition of Free and Bound Variables:
\begin{enumerate}
\item A Variable $x$ is Free in Atomic Formula
  (\S\ref{sec:atomic_formula}) $\varphi$ if $x$ occurs in $\varphi$
  (Atomic Formulas are Quantifier-free)
\item A Variable is Free or Bound in the Compound Formula $\varphi
  \bullet \psi$ if $x$ is Free or Bound in either $\varphi$ or $\psi$,
  where $\bullet$ is a Binary Logical Connective
  (\S\ref{sec:logical_connective})
\item A Variable $x$ is Free in $\forall y \varphi$ iff $x$ is Free in
  $\varphi$ and $x$ is not $y$. Conversely $x$ is Bound in $\forall y
  \varphi$ if $x$ is $y$ or $x$ is Bound in $\varphi$.
\end{enumerate}

A Term with no Free Variables is a \emph{Ground Term} and a Formula
with no Free Variables in First-order Logic is a \emph{Sentence} (also
called a \emph{Closed Formula}). Sentences have well-defined Truth
values. In some treatments, Free Variables may be Implicitly
Universally Quantified, but in general this is not the case.



\subsubsection{Metavariable}\label{sec:metavariable}

A \emph{Metavariable} is a Variable written in a Metalanguage that
stands in for an Element in the Object Language. The Formalization of
Metavariables falls under \emph{Type Theory} (Part
\ref{sec:type_theory}).

Metavariables may be referred to as \emph{Schematic Variables} in the
context of Axiom Schemata (\S\ref{sec:axiom_schema}) and Rule Schemata
(\S\ref{sec:rule_schema}). A Schematic Variable ranges over all
Propositions (\S\ref{sec:proposition}).



% --------------------------------------------------------------------
\subsection{Logical Constant}\label{sec:logical_constant}
% --------------------------------------------------------------------

\emph{Logical Constants} (or \emph{Logical Symbols}) have the same
Semantic Meaning under every Interpretation of the Language.

Two types of Logical Constants are \emph{Logical Connectives} and
\emph{Quantifiers}.



\subsubsection{Logical Connective}\label{sec:logical_connective}

A \emph{Logical Connective} (or \emph{Logical Operator}) is a Truth
Function (\S\ref{sec:truth_function}) that assigns a Truth Value to a
Compound Sentence as a Function of the Truth Value of its
Sub-sentences.

A \emph{Functionally Complete} Set of Logical Connectives allows the
expression of arbitrary Truth Functions. The smallest such Set may be
a single Connective, such as Alternative Denial ($\uparrow$,
\S\ref{sec:alternative_denial}) or Joint Denial ($\downarrow$,
\S\ref{sec:joint_denial})



\paragraph{Disjunction}\label{sec:disjunction}\hfill \\

\emph{Disjunction} (also \emph{Alternation} or ``\emph{Logical Or}'')



\paragraph{Conjunction}\label{sec:conjunction}\hfill \\

\emph{Conjunction} (or ``\emph{Logical And}'')



\paragraph{Material Conditional}\label{sec:material_conditional}\hfill \\

\emph{Material Conditional}, \emph{Material Implication},
\emph{Material Consequence}



\paragraph{Joint Denial}\label{sec:joint_denial}\hfill \\

\emph{Joint Denial} or ``\emph{Nor}''



\paragraph{Alternative Denial}\label{sec:alternative_denial}\hfill \\

\emph{Alternative Denial} or ``\emph{Nand}''



\subsubsection{Quantifier}\label{sec:quantifier}

A \emph{Quantifier} limits (\emph{Binds}) a Variable to a certain
quantity of members of the Domain, the two fundamental Quantifiers
being the \emph{Universal Quantifier} ($\forall$) and
\emph{Existential Quantifier} ($\exists$).

A \emph{Bound Quantifier} is one with a restricted Range, e.g.
$\exists x > 0$ or $\forall x \in \mathbb{R}$.

The \emph{Unique Existential Quantifier}, denoted by $\exists !$, is
expressed in natural language as ``there is one and only one.'' A
First-Order System requires the \emph{Equality Relation}
(\S\ref{sec:firstorder_equality}) in order to be able to express
Uniqueness Quantification.

\emph{Witness}



\paragraph{Quantifier Rank}\label{sec:quantifier_rank}\hfill \\

\emph{Quantifier Rank} is the depth of nesting of Quantifiers in a
Formula.

Inductive definition of Quantifier Rank function $qr$:
\begin{itemize}
\item $qr(\varphi) = 0$ if $\varphi$ is Atomic
\item $qr(\varphi_1 \wedge \varphi_2) = qr(\varphi_1 \vee \varphi_2) = max(qr(\varphi_1),qr(\varphi_2))$
\item $qr(\neg \varphi) = qr(\varphi)$
\item $qr(\exists_x \varphi) = qr(\varphi) + 1$
\end{itemize}



\paragraph{Prenex Normal Form}\label{sec:prenex_normal}\hfill \\

\emph{Normal Form} (\S\ref{sec:normal_form})



\paragraph{Plural Quantification}\label{sec:plural_quantification}\hfill \\

\emph{Plural Quantification}



% --------------------------------------------------------------------
\subsection{Non-logical Constant}\label{sec:nonlogical_constant}
% --------------------------------------------------------------------

Non-Logical Symbols only have meaning under an Interpretation
(\S\ref{sec:interpretation}). These are Symbols such as Relation
Symbols (Predicates \S\ref{sec:predicate}) and Function Symbols.

The set of Non-Logical Symbols used in a particular discourse is
called the \emph{Signature} (\S\ref{sec:signature}) of the discourse.



\subsubsection{Predicate}\label{sec:predicate}



% --------------------------------------------------------------------
\subsection{Formula}\label{sec:formula}
% --------------------------------------------------------------------

Expressions belonging to the underlying Language of a Logical System
are called ``\emph{Well-formed Formulas}'' (\emph{WFFs}) or just
\emph{Formulas}. The Syntax (\S\ref{sec:formal_grammar}) of a Formula
is defined by the Symbols and \emph{Formation Rules} for a particular
kind of Logic.

A Formula capable of being assigned a Truth Value is called a
\emph{Proposition} (\S\ref{sec:proposition}).



\subsubsection{Term}\label{sec:term}

A \emph{Term} represents an object of the Domain (\S\ref{sec:domain}).

Terms can be defined Inductively from Constants, Variables, and
Functions. Given Terms, $T$, Variables, $V$, n-ary Functions, $F = F_0
\cup F_1 \cup F_2 \cup \cdots \cup F_n$, Constants, $C = F_0$:
\[
    V \subseteq T
\]\[
    C \subseteq T
\]\[
    \forall \tau_n=\{t_1,\cdots,t_n\} \in \mathcal{P}(T), \forall f
    \in F_n \exists f(t_1,\cdots,t_n) \in T
\]



\subsubsection{Atomic Formula}\label{sec:atomic_formula}

\emph{Atomic Formula}

Atomic Formulas can be defined Inductively by Formation Rules on Terms
and Relations. Given Terms, $T = \{t_0,\ldots,t_n\}$, and Relations,
$R = \{r_0,\ldots,r_m\}$:
\begin{itemize}
\item $t_i = t_j$ is a Formula
\item $r_k(t_0,\ldots,t_n)$ is a Formula and $r_k$ is an n-ary Relation
\end{itemize}
A \emph{Literal} is an Atomic Formula, $\phi$, or its Negation, $\neg
\phi$. A Finite Disjunction (\S\ref{sec:disjunction}) of Literals
forms a \emph{Clause}.



% --------------------------------------------------------------------
\subsection{Proposition}\label{sec:proposition}
% --------------------------------------------------------------------

\emph{Propositions} are WFF that are assigned a Truth Value
(\S\ref{sec:truth_value}), i.e. they are Truth-bearers
(\S\ref{sec:truth_bearer}). An \emph{Atomic Proposition} contains no
Logical Connectives. A \emph{Compound Proposition} is composed by
Recursive application of Logical Connectives to Propositions by a
corresponding \emph{Concatenation Rule} that assigns a new Truth Value
to the Compound Proposition. A Proposition with no Free Variables is
called a \emph{Sentence}.

A Valuation (\S\ref{sec:valuation}) gives a Proposition a particular
Truth Value, while an Interpretation (\S\ref{sec:interpretation})
associates the Symbols of the Formula with their Realizations
(Meanings).

A Proposition is Satisfiable (\S\ref{sec:satisfaction}) if it is True
under at least one Interpretation. A Proposition is Valid
(\S\ref{sec:validity}) if and only if it is True under very
Interpretation.



% ====================================================================
\section{Propositional Logic}\label{sec:propositional_logic}
% ====================================================================

\emph{Propositional Logic} (also called \emph{Propositional Calculus}
or \emph{Sentential} or \emph{Statement Logic}) studies the Truth
Value of \emph{Propositional Formulas} built up from Atomic Formulas
(\S\ref{sec:atomic_formula}) and Logical Connectives
(\S\ref{sec:logical_connective}).

The Language of Propositional Logic therefore consists of
\emph{Propositional Variables} (or \emph{Sentential Variables}), and a
Functionally Complete set of Logical Connectives. A Propositional
Variable that has been given a Truth Valuation (\S\ref{sec:valuation})
is sometimes called a \emph{Propositional Constant}. Propositional
Variables act as the Atomic Formulas of Propositional Logic. A
Propositional Formula is equivalent to a Boolean Term
(\S\ref{sec:boolean_algebra}).

A Signature (\S\ref{sec:signature}) for Propositional Logic consists
only of Nullary Relation Symbols for each Propositional Variable, and
a Model (\S\ref{sec:structure}) provides a Truth Valuation for each.

In Propositional Logic, the Extra-logical Symbol for Entailment,
$\vdash$, and the Material Implication Operator, $\rightarrow$,
coincide in that:
\[(A \vdash B) \Leftrightarrow (\vdash A \rightarrow B)\]
but the difference is that $\vdash$ describes a Deduction, that is a
relation between Sentences, and $\rightarrow$ is a Logical Connective
within a Formula.

Propositional Calculus is Isomorphic to Simply-typed
$\lambda$-calculus (\S\ref{sec:simply_typed}) and is Strongly
Normalizing (\S\ref{sec:normal_form}) to either a Conjunctive Normal
Form (\S\ref{sec:conjunctive_form}) or Disjunctive Normal Form
(\S\ref{sec:disjunctive_form}), and thus is not a Turing Complete
system.

Formal definition of a System of Propositional Calculus $\mathcal{S}$:
\[
    \mathcal{S} = (\mathbf{A},\mathbf{\Omega},\mathbf{Z},\mathbf{I})
\]
where:
\begin{itemize}
\item $\mathbf{R}$ is a Finite Set of Propositional Variables ($P_1$,
  $P_2$, $P_3$, $\ldots$)
\item $\mathbf{\Omega}$ is a Finite, Functionally Complete Set of
  Logical Connective Symbols ($\neg$, $\wedge$, $\vee$, $\ldots$)
\item $\mathbf{I}$ is a Finite Set of Inference Rules
\item $\mathbf{A}$ is a Finite Set of Axioms
\end{itemize}
The Atomic Formulas of $\mathcal{S}$ are just the Propositional
Variables $P_1, P_2, P_3, \ldots$.

The Formulas of $\mathcal{S}$ are then Inductively defined as the
smallest Class, $\mathbf{X}$, with the following Formation Rules:
\begin{enumerate}
\item Any Literal of $\mathcal{S}$ is in $\mathbf{X}$
\item For Formulas $\phi_1, \phi_2, \cdots, \phi_n$ and $n$-ary
  Logical Operator $f \in \mathbf{\Omega}_n$ where $\mathbf{\Omega}_n$
  is the Partition of $\mathbf{\Omega}$ containing all $n$-ary Logical
  Operators, then $f(\phi_1, \phi_2, \cdots, \phi_n)$ is also a
  Formula
\end{enumerate}
Propositional Logic is closed under Truth-Functional Operators, so the
above is sufficient to define all WFFs: nothing else is a Formula of
$\mathcal{S}$.

Formulas Derived by the Axioms and Inference Rules of a Propositional
Logic are Theorems (\S\ref{sec:theorem}) in that System. Allowing for
Axiom Schemata (an Infinite number of Axioms) extends Propositional
Logic; an example of such a System is Skolem Arithmetic
(\S\ref{sec:skolem_arithmetic}).

Inference Rules of a Propositional Logic define Valid Argument Forms
(\S\ref{sec:logical_form}). The simplest Argument Form that is both
necessary and given a complete set of Axioms is sufficient to define
all other Argument Forms is \emph{Modus Ponens}, shown here
Schematicized:

$\textrm{1. }\varphi \rightarrow \psi$

$\textrm{2. }\varphi$

$\therefore\textrm{ }\psi$ \\
where lines one and two are Premises and
line three is the Conclusion (the symbol $\therefore$ is read as
\emph{therefore}). This is written in Sequent Notation
(\S\ref{sec:sequent_notation}) as:
\[
  (\varphi \rightarrow \psi), \varphi \vdash \psi
\]
The Schematic representation of \emph{Modus Tollens}:

$\textrm{1. }\varphi \rightarrow \psi$

$\textrm{2. }\neg\psi$

$\therefore\textrm{ }\neg\varphi$\\



% --------------------------------------------------------------------
\subsection{Conjunctive Normal Form}\label{sec:conjunctive_form}
% --------------------------------------------------------------------

% --------------------------------------------------------------------
\subsection{Disjunctive Normal Form}\label{sec:disjunctive_form}
% --------------------------------------------------------------------

% --------------------------------------------------------------------
\subsection{Second-order Propositional Logic}
\label{sec:secondorder_propositional_logic}
% --------------------------------------------------------------------

% --------------------------------------------------------------------
\subsection{Multi-valued Logic}\label{sec:multi_valued_logic}
% --------------------------------------------------------------------

\subsubsection{Fuzzy Logic}\label{sec:fuzzy_logic}

\subsubsection{Probabilistic Logic}\label{sec:probabilistic_logic}



% ====================================================================
\section{Zeroth-order Logic}\label{sec:zerothorder_logic}
% ====================================================================

%FIXME

Language:

Propositional Variables (Relations of Arity 0)

Relations of Arity >= 1

Constants (Functions of Arity 0)

Functions of Arity >= 1

Logical Connectives

Grouping Symbols (parentheses, etc.)



% ====================================================================
\section{Predicate Logic}\label{sec:predicate_logic}
% ====================================================================

\emph{Predicate Logic} (or \emph{Predicate Calculus}) describes
Logical Systems with Formulas containing Variables
(\S\ref{sec:variable}) that can be Quantified
(\S\ref{sec:quantifier}). This includes First-order Logic
(\S\ref{sec:firstorder_logic}), Higher-order Logic
(\S\ref{sec:higherorder_logic}), Many-sorted Logic
(\S\ref{sec:many_sorted_logic}), and Infinitary Logic
(\S\ref{sec:infinitary_logic}). This is in contrast to Logical Systems
without Variables or Quantifiers such as Propositional Logic
(\S\ref{sec:propositional_logic}) and Zeroth-order Logic
(\S\ref{sec:zerothorder_logic}).



% --------------------------------------------------------------------
\subsection{First-order Logic}\label{sec:firstorder_logic}
% --------------------------------------------------------------------

% FIXME this section requires rewrite/cleanup

\emph{First-order Logic} (or \emph{Lower Predicate Logic}) allows
Quantification over individual Objects in the Domain of Discourse
(\S\ref{sec:domain}). This is in contrast to Logical Systems that
allow Plural Quantification (Quantification over Sets
\S\ref{sec:plural_quantification}), such as Second-order Logic
(\S\ref{sec:secondorder_logic}) and Higher-order Logic
(\S\ref{sec:higherorder_logic}).

The traditional Signature used in First-order Logic:
\begin{enumerate}
\item For $n \geq 0$, $n$-ary Predicate (also called Relation)
  Symbols: $p^{n}_0, p^{n}_1, p^{n}_2, p^{n}_2, p^{n}_3, \ldots$
\item For $n \geq 0$, $n$-ary Function Symbols: $f^{n}_0, f^{n}_1,
  f^{n}_2, f^{n}_2, f^{n}_3, \ldots$
\end{enumerate}
The contemporary Signature used:
\begin{enumerate}
\item Predicate Symbols denoted by uppercase letters $P$, $Q$, $R$,
  $\ldots$ with arity ($\geq 0$) specified by the \emph{Valence} of the
  parenthetical arguments, eg P(x), Q(x,y).
\item Function Symbols denoted by lowercase letters $f$, $g$, $h$,
  $\ldots$ with arity specified in the usual way.
\end{enumerate}
Here, Functions of Valence 0 are \emph{Constant Symbols} denoted by
letters $a$, $b$, $c$, $\ldots$.

The Formation Rules for WFF in a System of First-order Logic generally
describe a Context-free Grammar (\S\ref{sec:context_free}) with a
infinite Alphabet and multiple Start Symbols.

Terms are limited to those including Variables and a Finite number of
$n$-ary Function applications, but not including Expressions involving
a Predicate Symbol.

Definition of Atomic Formulas (no Logical Connectives or Quantifiers):
\begin{enumerate}
\item If $t_1$ and $t_2$ are Terms, then $t_1 = t_2$ is an Atomic Formula.
\item If $R$ is an $n$-ary Relation (Predicate), and $t_1,\ldots,t_n$
  are terms, then $R(t_1,\ldots,t_n)$ is an Atomic Formula.
\end{enumerate}

Definition of WFF as a finite number of applications of the following rules:
\begin{enumerate}
\item $\neg \phi$ is a WFF when $\phi$ is a WFF
\item $(\phi \bullet \psi)$ is a WFF when $\phi$ and $\psi$ are WFF
  and $\bullet$ is a Binary Connective
\item $\exists x \phi$ is a WFF when $x$ is a Variable and $\phi$ is a WFF
\item $\forall x \phi$ is a WFF when $x$ is a Variable and $\phi$ is a WFF
\end{enumerate}

First-order Logic may be used to devise Deductive Systems with Finite
Domains that are Sound (\S\ref{sec:soundness}) and Complete
(\S\ref{sec:completeness}), but for Domains of Infinite Cardinality a
System of Higher-order Logic is required. First-order Logic is
Semi-decidable (\S\ref{sec:semidecidable}).

The L\"owenheim-Skolem Theorem (\S\ref{sec:lowenheim_skolem}) implies
that First-order Logic is unable to characterize the concept of
Countability or Uncountability \S\ref{sec:cardinality}).

The Compactness theorem (\S\ref{sec:compactness_theorem}) implies that
if a Formula is derived from a System of First-order Logic with an
Infinite Set of Axioms, then it can be derived from a Finite Subset of
those Axioms. This has implications for the determination of Connected
Components of a Directed Graph (\S\ref{sec:directed_graph}).



\subsubsection{Universal Generalization}\label{sec:universal_generalization}

\emph{Universal Generalization} is an Inference Rule:
\[P(x) \vdash \forall x P(x)\]



\subsubsection{Equality Conventions}\label{sec:firstorder_equality}

\textbf{First-order Logic with Equality}

Including a primitive Logical Symbol for Equality, $=$, interpreted
as the real Equality Relation between members of the Domain such that
``two'' given Elements are the same Element. This adds the following
Axioms:

\begin{enumerate}
\item \textbf{Reflexivity}: $\forall x, x=x$
\item \textbf{Substitution for functions}: given a function, $f$,
  $\forall x \forall y, x = y \rightarrow f(\ldots,x,\ldots) =
  f(\ldots,y,\ldots)$
\item \textbf{Substitution for formulas (Leibniz's Law)}: given a
  Formula $\varphi$ with Free occurrences of $x$, and $\varphi '$ with
  Free occurrences of $y$, $\forall x \forall y, x = y \rightarrow
  (\varphi \rightarrow \varphi ')$
\end{enumerate}

Defining a theory with a Binary Relation $A(x,y)$ that satisfies
Reflexivity and Leibniz's law is sufficient to derive any other
equality Theorems.



\textbf{First-order Logic without Equality}

An alternative convention is to consider the Equality Relation to be a
Non-logical Symbol, included as a part of the Signiature of a
particular Theory instead of as a Logical Operator. This allows two
distinct individuals to be considered equal by an arbitraray
Equivalence Relation. If this convention is used, but no distinct
individuals, $a$ and $b$ satisfy $a=b$ then the Interpretation is
termed a \emph{Normal Model} (that is equivalent to a First-order
Logic with Equality).



\subsubsection{Monadic First-order Logic}\label{sec:monadic_firstorder}

\emph{Monadic First-order Logic} (or \emph{Monadic Predicate
  Calculus}) restricts First-order Logic to Unary Relations and no
Function symbols. This weaker form of First-order Logic is fully
Decidable.



\subsubsection{Many-sorted First-order Logic}\label{sec:many_sorted_logic}

\emph{Many-sorted First-order Logic} allows Variables to be Quantified
over different Domains, thus giving Variables different \emph{Sorts}.
With Finitely many Sorts, Many-sorted First-order Logic can be reduced
to Single-sorted First-order Logic. This can be accomplished by adding
Unary Predicates to a First-order Logic that Partition
(\S\ref{sec:partition}) the Domain.



\subsubsection{Infinitary Logic}\label{sec:infinitary_logic}

\emph{Infinitary Logic} allows Formulas of Infinite length, through
either Conjunctons and Disjunctions, Infinite-arity Relations and
Functions, or Quantification over Infinitely many Variables.



\paragraph{$\Omega$-logic}\label{sec:omega_logic}\hfill \\

\emph{$\Omega$-logic}



\subsubsection{Predicate Functor Logic}\label{sec:pfl}

\emph{Predicate Functor Logic} allows the expression of First-order
Logic Algebraically without Quantified Variables by using
\emph{Predicate Functors} that Operate on Terms to yield Terms.



% --------------------------------------------------------------------
\subsection{Second-order Logic}\label{sec:secondorder_logic}
% --------------------------------------------------------------------

\emph{Second-order Logic} allows for Quantifiers to range over
Relations and Functions and thus \emph{Sorts} of Variables that range
over $k$-ary Relations and Functions. It is possible to leave out a
definition of Quantifiers for Functions since $k$-ary Functions can be
represented by $k+1$-ary Relations.\cite{shapiro00} Quantification
over Functions allows the creation of the Analytic Hierarchy
(\S\ref{sec:analytic_hierarchy}).



\subsubsection{Monadic Second-order Logic}\label{sec:monadic_secondorder}

\paragraph{Plural Monadic Second-order Logic}
\label{sec:plural_monadic_secondorder}\hfill \\

An alternative formulation of Second-order Logic is to allow Variables
to take on \emph{Plural} Values. It is equi-interpretable with
\emph{Monadic Second-order Logic}, which restricts Quantification to
Unary Relations (Sets).



\subsubsection{Existential Second-order Logic}
\label{sec:existential_secondorder}

\paragraph{Independence-friendly
  Logic}\label{sec:independence_logic}\hfill \\

\emph{Independence-friendly Logic} has \emph{Branching Quantifiers}.



% --------------------------------------------------------------------
\subsection{Higher-order Logic}\label{sec:higherorder_logic}
% --------------------------------------------------------------------

\emph{Higher-order Logic} is the Union of First-, Second-, Third-,
$\ldots$ order Logic and allows Quantification over arbitrarily deep
nested Sets.



% ====================================================================
\section{Classical Logic}\label{sec:classical_logic}
% ====================================================================

\emph{Classical Logic} is the class of Propositional and
First-order Systems of Logic characterized by the following Inference
Rules:

\begin{description}

\item [Tertium non datur] (\emph{Law of excluded middle})
    \[\vdash(p \vee \neg p)\]

\item [Double Negation]
    \[p \vdash \neg\neg p\]

\item [Law of Non-contradiction]
    \[\vdash \neg(p \wedge \neg p)\]

\item [Ex falso quodlibet] (\emph{Principle of explosion},
  \emph{Principle of Psuedo-Scotus})
    \[\vdash 0 \rightarrow p\]

%FIXME finish properties and rules

\end{description}

The intended Semantic Interpretation (\S\ref{sec:interpretation})
of Classical Logic is subject to the \emph{Principle of Bivalence}
which says that every Proposition has one Truth-value: True or False.
Non-classical Logics such as \emph{Intuitionistic Logic}
(\S\ref{sec:intuitionistic_logic}) does not have this Property.



% ====================================================================
\section{Intensional Logic}\label{sec:intensional_logic}
% ====================================================================

% --------------------------------------------------------------------
\subsection{Modal Logic}\label{sec:modal_logic}
% --------------------------------------------------------------------

\emph{Modal Logic} adds to First-order Logic \emph{Sentential
  Functors} (\emph{Intensions}) that range over Terms. An Intension is
the \emph{Sense} in which a Logical Assertion is made, as opposed to
the \emph{Reference} to which the Assertion applies (\emph{i.e.
  Extensional Quantification}).

\emph{Modal Logic} extends Propositional and Predicate Logic to
include Operators expressing \emph{Modality}. Various meanings for
these Modal Operators include \emph{Alethic Modality}
(\emph{Necessity} and \emph{Possibility}), \emph{Temporal Modality}
(qualification in terms of time, eg \emph{always}, \emph{eventually},
\emph{until}), \emph{Deontic Modality} (\emph{Obligation} and
\emph{Permission}), and \emph{Doxastic Modality} (Modalities with
regards to \emph{Belief}).

An unary \emph{Primitive Modal Operator}, $\square$, defines a Dual
Operator, $\Diamond$, such that the following analogues of de Morgan's
Laws (\S\ref{sec:de_morgan}) hold:
    \[\Diamond P \leftrightarrow \neg \square \neg P\]
    \[\square P \leftrightarrow \neg \Diamond \neg P\]
Modal Logic with more than one Primitive Modal Operator, $\square _i,
i \in \{1, \ldots, n\}$ is \emph{Multimodal Logic}.

By the \emph{Curry-Howard Correspondence}
(\S\ref{sec:curry_howard}), Modal Logic corresponds
to \emph{Monads} (\S\ref{sec:monad}).



\subsubsection{Alethic Logic}\label{sec:alethic_logic}

Most Systems of Alethic Logic are based on an extension of
Propositional Logic called $\mathbf{K}$ which has:

\begin{enumerate}
\item $\square$, unary operator for \emph{Necessity}.
\item $\mathbf{N}$, \emph{Necessitation Rule}: stating if $p$ is a
  Theorem, then $\square p$ is a Theorem.
\item $\mathbf{K}$, \emph{Distribution Axiom}: $\square(p \rightarrow
  q) \rightarrow (\square p \rightarrow \square q)$ (also called the
  \emph{Kripke schema} (\S\ref{sec:frame_semantics}).
\end{enumerate}

Adding further Axioms gives rise to a nested hierarchy of Systems of
\emph{Normal Modal Logic}:

\begin{itemize}
\item $K := \mathbf{K} + \mathbf{N}$
\item $T := K + \mathbf{T}$
\item $S4 := T + \mathbf{4}$
\item $S5 := S4 + \mathbf{5}$
\item $D := K + \mathbf{D}$
\end{itemize} \hfill \\
where

\begin{itemize}
\item $\Diamond$, unary operator for \emph{Possibly}
\item $\mathbf{T}$, \emph{Reflexivity Axiom}: $\square p \rightarrow p$
\item $\mathbf{4}$: $\square p \rightarrow \square \square p$
\item $\mathbf{B}$: $p \rightarrow \square \Diamond p$
\item $\mathbf{D}$: $\square p \rightarrow \Diamond p$
\item $\mathbf{5}$: $\Diamond p \rightarrow \square \Diamond p$
\end{itemize}



\subsubsection{Doxastic Logic}\label{sec:doxastic_logic}

\emph{Doxastic Logic} uses the unary Modal Operator, $\mathcal{B}$, to
denote \emph{Belief}. Example:
\[
    \mathcal{B} x
\]
has the meaning ``It is Believed that x is the case''. A set of
Beliefs is usually denoted
\[
    \mathbb{B}: \{ b_1, b_2, \ldots, b_n \}
\]



\subsubsection{Deontic Logic}\label{sec:deontic_logic}

\emph{Standard Deontic Logic} ($\mathbf{SDL}$) adds the following
Axioms to Propositional Logic (\S\ref{sec:propositional_logic}):
    \[O(A \rightarrow B) \rightarrow (OA \rightarrow OB)\]
    \[PA \rightarrow \neg O \neg A\]
with Primitive Operators $O$ (\emph{Obligatory}) and $P$
(\emph{Permissible}). \emph{Forbidden} is defined as
    \[FA = O \neg A\]
or
    \[FA = \neg P A\]
Deontic Logic may be extended by Alethic Operators with the Axiom:
    \[OA \rightarrow \Diamond A\]
which has the meaning ``ought implies can''.



\subsubsection{Temporal Logic}

\paragraph{Tense Logic} \hfill \\

\emph{Tense Logic} is a 2-modal Logic that adds operators $[F]$ for
\emph{Future} and $[P]$ for \emph{Past} Modalities.



\paragraph{CTL*}
\hfill \\
Superset of \emph{Computation Tree Logic} and \emph{Linear
  Temporal Logic}

\subparagraph{Computation Tree Logic}
\hfill \\
Temporal Logic with \emph{Path} Modalities.

\subparagraph{Linear Temporal Logic}\label{sec:linear_temporal}
\hfill \\
\emph{Linear Temporal Logic} is a Modal Logic with Modalities
referring to Time.



\paragraph{Interval Temporal Logc}

\paragraph{Modal $\mu$-calculus}



\subsubsection{Dynamic Logic}

\emph{Dynamic Logic} adds Terms denoting \emph{Actions}:
\[[a]p\]
where after performing Action $a$ is necessitated that $p$ holds and
\[\langle a \rangle p\]
where after performing Action $a$ it is possible that $p$ holds.



% ====================================================================
\section{Intuitionistic Logic}\label{sec:intuitionistic_logic}
% ====================================================================

\emph{Intuitionistic Logic} (or \emph{Constructive Logic}) replaces
Truth with the concept of \emph{Constructive Provability}. This is to
say that Operations in Intuitionistic Logic preserve
\emph{Justification} rather than Truth-value. Such systems are
restrictions of Classical Logic without the Law of the Excluded Middle
or Double Negation Elimination (\S\ref{sec:classical_logic}).

Whereas First-order Logic (\S\ref{sec:predicate_logic}) is a
foundation for Set Theory (Part \ref{sec:set_theory}), Intuitionistic
Logic is used as a foundation for \emph{Type Theory} (Part
\ref{sec:type_theory}) and \emph{Constructive Set Theory}
(\S\ref{sec:constructive_set_theory}).

A Formula in Intuitionistic Logic does not necessarily have a Prenex
Normal Form (\S\ref{sec:prenex_normal}).

Intuitionistic Logic is \emph{Modelled} by \emph{Heyting Algebra}
(\S\ref{sec:heyting_semantics}) or \emph{Kripke Semantics}
(\S\ref{sec:kripke_semantics}) and lacks the Principle of Bivalence
(\S\ref{sec:classical_logic}); thus there is no sole-sufficient
Operator in Intuitionistic Logic. A Formula is Valid if and only if it
receives the Value of the Top Element for any Valuation on any Heyting
Algebra.

Complete bases are:
\[
    \{ \vee, \leftrightarrow, \bot \}
\]
and
\[
    \{ \vee, \leftrightarrow, \neg \}
\]

Proofs (Part \ref{sec:proof_theory}) in a Theory
(\S\ref{sec:formal_theory}), $\mathcal{T}$, based on Intuitionistic
Logic have the \emph{Existence Property}:
\[
    (\exists x)A(x) \in \mathcal{T} \rightarrow (\exists t)A(t)
\]
where $A(x)$ has $x$ as the only Free Variable and $t$ is a Term.

\emph{Disjunction Property}:
\[
    A \vee B \in \mathcal{T} \rightarrow A \in \mathcal{T} \vee B \in \mathcal{T}
\]

\subsubsection{Minimal Logic}



\subsubsection{Combinatory Logic}\label{sec:combinatory_logic}

\emph{Combinator}



\subsubsection{Intermediate Logic}

\emph{Intermediate Logic} is an extended Intuitionistic Logic
(\emph{Superintuitionistic Logic}) that is Consistent and still weaker
than the strongest Consistent Superintuitionistic Logic: Classical
Logic.



% ====================================================================
\section{Substructural Logic}\label{sec:substructural_logic}
% ====================================================================

% --------------------------------------------------------------------
\subsection{Relevance Logic}\label{sec:relevance_logic}
% --------------------------------------------------------------------

\emph{Relevance Logic} (or \emph{Relevant Logic})



% --------------------------------------------------------------------
\subsection{Linear Logic}\label{sec:linear_logic}
% --------------------------------------------------------------------

\subsubsection{Noncommutative Logic}\label{sec:noncommutative_logic}

\emph{Ordered Logic}



% ====================================================================
\section{Ordinal Logic}\label{sec:ordinal_logic}
% ====================================================================

Alan Turing's PhD Thesis \cite{turing38}



% ====================================================================
\section{Hoare Logic}\label{sec:hoare_logic}
% ====================================================================

\emph{Concurrency}



% ====================================================================
\subsection{Term Logic}\label{sec:term_logic}
% ====================================================================

\emph{Syncategorematic Term}



% ====================================================================
\section{Categorical Logic}\label{sec:categorical_logic}
% ====================================================================



% ====================================================================
\section{Connexive Logic}\label{sec:connexive_logic}
% ====================================================================



% ====================================================================
\section{Non-monotonic Logic}\label{sec:nonmonotonic_logic}
% ====================================================================

Defeasible Inference (\S\ref{sec:defeasible_inference})



% --------------------------------------------------------------------
\subsection{Default Logic}\label{sec:default_logic}
% --------------------------------------------------------------------

Default Inference (\S\ref{sec:default_inference})



% --------------------------------------------------------------------
\subsection{Autoepistemic Logic}\label{sec:autoepistemic_logic}
% --------------------------------------------------------------------



% ====================================================================
\section{Paraconsistent Logic}\label{sec:paraconsistent_logic}
% ====================================================================

% --------------------------------------------------------------------
\subsection{Discursive Logic}\label{sec:discursive_logic}
% --------------------------------------------------------------------

% --------------------------------------------------------------------
\subsection{Non-adjunctive Logic}\label{sec:nonadjunctive_logic}
% --------------------------------------------------------------------

% --------------------------------------------------------------------
\subsection{Adaptive Logic}\label{sec:adaptive_logic}
% --------------------------------------------------------------------

% --------------------------------------------------------------------
\subsection{Formal Inconsistency}\label{sec:formal_inconsistency}
% --------------------------------------------------------------------

% --------------------------------------------------------------------
\subsection{Logic of Paradox}\label{sec:logic_of_paradox}
% --------------------------------------------------------------------

Graham Priest
