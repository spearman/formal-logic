%%%%%%%%%%%%%%%%%%%%%%%%%%%%%%%%%%%%%%%%%%%%%%%%%%%%%%%%%%%%%%%%%%%%%%
%%%%%%%%%%%%%%%%%%%%%%%%%%%%%%%%%%%%%%%%%%%%%%%%%%%%%%%%%%%%%%%%%%%%%%
\part{Abstract Algebra}\label{part:abstract_algebra}
%%%%%%%%%%%%%%%%%%%%%%%%%%%%%%%%%%%%%%%%%%%%%%%%%%%%%%%%%%%%%%%%%%%%%%
%%%%%%%%%%%%%%%%%%%%%%%%%%%%%%%%%%%%%%%%%%%%%%%%%%%%%%%%%%%%%%%%%%%%%%

\iffalse

1 Set X:

  1 Operator
  * : X x X -> X

    Magma         * Closed
    Semigroup     * Closed, Associative
    Monoid        * Closed, Associative, Identity
    Group         * Closed, Associative, Identity, Inverse
    Abelian Group * Closed, Associative, Identity, Inverse, Commutative

  2 Operators
  * : X x X -> X
  + : X x X -> X

    Rng           * + Distributive
                    + Closed, Associative, Identity, Inverse, Commutative
                    * Closed, Associative
    Ring (Unital) * + Distributive
                    + Closed, Associative, Identity, Inverse, Commutative
                    * Closed, Associative, Identity
    Division Ring * + Distributive
                    + Closed, Associative, Identity, Inverse, Commutative
                    * Closed, Associative, Identity, Inverse
    Field         * + Distributive
                    + Closed, Associative, Identity, Inverse, Commutative
                    * Closed, Associative, Identity, Inverse, Commutative

2 Sets X Y:

  2 Operators (Ring, Field)
  * : X x X -> X
  + : X x X -> X
  1 Operator (Group)
  & : Y x Y -> Y
  Scalar Multiplication
  ^ : Y x X -> X

    Module
                    ^ & Additive
                    ^ + Additive
                    ^ * Associative
              Ring: * + Distributive
                      + Closed, Associative, Identity, Inverse, Commutative
                      * Closed, Associative, Identity
            Abelian
             Group:   & Closed, Associative, Identity, Inverse, Commutative

    Vector Space
                    ^ & Additive
                    ^ + Additive
                    ^ * Associative
             Field: * + Distributive
                      + Closed, Associative, Identity, Inverse, Commutative
                      * Closed, Associative, Identity, Inverse, Commutative
            Abelian
             Group:   & Closed, Associative, Identity, Inverse, Commutative

  2 Operators (Ring, Field)
  *  : X x X -> X
  +  : X x X -> X
  1 Operator (Group)
  &  : Y x Y -> Y
  Scalar Multiplication
  ^  : Y x X -> X
  Norm
  || : Y -> X

    Normed Vector Space
                    ^ & Additive
                    ^ + Additive
                    ^ * Associative
                   || ^ Distributive
                   || & Subadditive
             Field: * + Distributive
                      + Closed, Associative, Identity, Inverse, Commutative
                      * Closed, Associative, Identity, Inverse, Commutative
            Abelian
             Group:   & Closed, Associative, Identity, Inverse, Commutative


  2 Operators (Ring, Field)
  *  : X x X -> X
  +  : X x X -> X
  1 Operator (Group)
  &  : Y x Y -> Y
  Scalar Multiplication
  ^  : Y x X -> X
  Norm
  || : Y -> X
  Inner Product
  ,  : Y x Y -> X

    Inner Product Space
                    ^ & Additive
                    ^ + Additive
                    ^ * Associative
                   || ^ Distributive
                   || & Subadditive
                      , Conjugate Symmetric, Sesquilinear, Positive-definite
             Field: * + Distributive
                      + Closed, Associative, Identity, Inverse, Commutative
                      * Closed, Associative, Identity, Inverse, Commutative
            Abelian
             Group:   & Closed, Associative, Identity, Inverse, Commutative

  2 Operators (Ring, Field)
  * : X x X -> X
  + : X x X -> X
  2 Operators
  & : Y x Y -> Y
  x : Y x Y -> Y
  Scalar Multiplication
  ^ : Y x X -> X

    R-algebra
                    ^ & Additive
                    ^ + Additive
                    ^ * Associative
              Ring: * + Distributive
                      + Closed, Associative, Identity, Inverse, Commutative
                      * Closed, Associative, Identity
            Abelian
             Group:   & Closed, Associative, Identity, Inverse, Commutative

            Bilinear
            Product:
                      x Closed
                    x & Right Distributive, Left Distributive
                    x ^ Distributive

    K-algebra
                    ^ & Additive
                    ^ + Additive
                    ^ * Associative
             Field: * + Distributive
                      + Closed, Associative, Identity, Inverse, Commutative
                      * Closed, Associative, Identity, Inverse, Commutative
            Abelian
             Group:   & Closed, Associative, Identity, Inverse, Commutative

            Bilinear
            Product:
                      x Closed
                    x & Right Distributive, Left Distributive
                    x ^ Distributive


\fi


% ====================================================================
\section{Universal Algebra}\label{sec:universal_algebra}
% ====================================================================

\emph{Universal Algebra} is the study of \emph{Algebraic Structures}
(\S\ref{sec:algebraic_structure}), i.e. Structures
(\S\ref{sec:structure}) with Signatures consisting only of Functional
Symbols and no Relation Symbols. Universal Algebra together with
Category Theory (Part \ref{sec:category_theory}) makes up Abstract
Algebra.



% --------------------------------------------------------------------
\subsection{Algebraic Structure}\label{sec:algebraic_structure}
% --------------------------------------------------------------------

Structure (\S\ref{sec:structure})

An Algebra may be limited by Axioms of \emph{Equational Laws}
(\S\ref{sec:equational_law}), e.g. the Associative Axiom.

Allowing for Infinitary Operations leads to the Algebraic Theory of
Complete Lattices (\S\ref{sec:complete_lattice}).



\subsubsection{Algebraic Type}\label{sec:algebraic_type}

The \emph{Algebraic Type} of an Algebraic Structure, $\Omega$, is an
Ordered Sequence of Natural Numbers listing the Arity of the
Operations of the Algebra.



\subsubsection{Equational Law}\label{sec:equational_law}

\subsubsection{Partial Algebra}\label{sec:partial_algebra}

\subsubsection{Subalgebra}\label{sec:subalgebra}

A \emph{Subalgebra} of an Algebraic Structure, $\mathfrak{A}$ is a
Subset of $|\mathfrak{A}|$ that is closed under all the Operations of
$\mathfrak{A}$.



\subsubsection{Homomorphism}\label{sec:homomorphism}

A \emph{Homomorphism} between two Algebraic Structures $\mathfrak{A}$
and $\mathfrak{B}$ is a Function $h: \mathfrak{A} \rightarrow
\mathfrak{B}$ defined for $n$-ary Operations:
\[
  \forall f_\mathfrak{A} \in \mathfrak{A}, f_\mathfrak{B} \in
  \mathfrak{B}, h(f_\mathfrak{A}(x_1, ..., x_n)) =
  f_\mathfrak{B}(h(x_1), ..., h(x_n))
\]

%FIXME merge these definitions

A \emph{Homomorphism} is a Structure-preserving Morphism between
Algebraic Structures (\S\ref{sec:universal_algebra}). That is, for
an Homomorphism $f : A \rightarrow B$ where $A$ and $B$ have Operators
$*$ and $*'$ respectively, for any $a_i \in A$
\[
  f(a_1 * a_2) = f(a_1) *' f(a_2)
\]
\HandRight\;Note that the Operators do not have to be the same.



\paragraph{Algebra Homomorphism Kernel}
\label{sec:algebra_homomorphism_kernel} \hfill \\

For Algebraic Structures $\mathfrak{A}$ and $\mathfrak{B}$, and
Homomorphism $f: \mathfrak{A} \rightarrow \mathfrak{B}$, the
\emph{Kernel} of $f$ is defined as:
\[
    ker(f) = \{ (a,a') \in \mathfrak{A} \times \mathfrak{A} : f(a) = f(a') \}
\]
$ker(f)$ is a Congruence Relation on $\mathfrak{A}$ and $f$ is Injective if and
only if $ker(f) = \{(a,a) : a \in \mathfrak{A}\}$.

The Quotient Algebra $\mathfrak{A}/ker(f)$ is Isomorphic to the Image
of $f$ (which is a Subalgebra of $\mathfrak{B}$, see First Isomorphism
Theorem (\S\ref{sec:isomorphism_theorem}).



\paragraph{Antihomomorphism}\label{sec:antihomomorphism}

\paragraph{Symmetry}\label{sec:structure_symmetry}
\hfill \\

A \emph{Symmetry} is a Structure Endomorphism
(\S\ref{sec:endomorphism})

Mathematical Symmetry (\S\ref{sec:symmetry})

Set: Bijective Map (Permutation Groups \S\ref{sec:permutation_group})

Metric Space: Isometry (\S\ref{sec:isometry})

Symmetry Group (\S\ref{sec:symmetry_group})



\subparagraph{Continuous Symmetry}\label{sec:continuous_symmetry}
\hfill \\

Continuous Transformation Group
(\S\ref{sec:continuous_transformation_group})



\subsubsection{Cancellative Property}\label{sec:cancellative_property}

\subsubsection{Direct Product}\label{sec:direct_product}

The \emph{Direct Product} of a Set of Algebraic Structures is the
Cartesian Product of the Sets with the Operations defined
coordinatewise.



\subsubsection{Direct Sum}\label{sec:direct_sum}

Elements of Direct Product having only finitely many Non-zero Terms
%FIXME

Coproduct (\S\ref{sec:coproduct}) for Abelian Groups
(\S\ref{sec:abelian_group}) and Vector Spaces
(\S\ref{sec:vector_space})



\subsubsection{Archimedean Property}\label{sec:archimedean_property}

Property of having no Infinitely Large or Infinitely Small Elements



% --------------------------------------------------------------------
\subsection{Essentially Algebraic Structure}
\label{sec:essentially_algebraic}
% --------------------------------------------------------------------

Partially Defined Operators satisfying Equational Laws



% --------------------------------------------------------------------
\subsection{Equational Reasoning}\label{sec:equational_reasoning}
% --------------------------------------------------------------------

Equational Logic (\S\ref{sec:equational_logic})



% --------------------------------------------------------------------
\subsection{Variety}\label{sec:variety}
% --------------------------------------------------------------------

A \emph{Variety} is a Class of Algebraic Structures defined only by
Axioms that are Identities in a given Signature. This is equivalent to
saying a Variety is the Class of Algebraic Structures with the same
Signature that are Closed under Homomorphic (\S\ref{sec:homomorphism})
Images, Subalgebras (\S\ref{sec:subalgebra}), and Direct Products
(\S\ref{sec:direct_product}); a result known as Birkhoff's Theorem
(\S\ref{sec:birkhoffs_theorem}). This rules out Logical Connectives,
Existential Quantification, and all Relations for the Signature
besides Equality (thus excluding the Class of Fields
\S\ref{sec:field}) and Identities being implicitly Universally
Quantified over the Domain.

An example of a Variety with Algebraic Signature $\Omega = (2)$ is the
Class of all Semigroups with an equation defining the Associative Law:
\[
    x(yz) = (xy)z
\]

Algebraic Structures in a Variety are Quotient Algebras
(\S\ref{sec:quotient_algebra}) generated by the Identities on the Term
Algebra (\S\ref{sec:term_algebra}) generated from the Signature and
Underlying Set.

Every Variety gives rise to a Monad (\S\ref{sec:monad}) and the Type
of the Algebra can be recovered from the Monad.



\subsubsection{Subvariety}\label{sec:subvariety_theorem}

A \emph{Subvariety} is a Subclass of a Variety with the same
Signature. For example, the Class of Abelian Groups is a Subvariety of
the Class of Groups.



\subsubsection{Pseudovariety}\label{sec:pseudovariety}

A Variety of Finite Algebraic Structures is called a
\emph{Pseudovariety}.



\subsubsection{Quasivariety}\label{sec:quasivariety}



\subsubsection{Birkhoff's Theorem}\label{sec:birkhoffs_theorem}
\cite{birkhoff35}

\emph{Birkhoff's Theorem} (or \emph{HSP Theorem})



% --------------------------------------------------------------------
\subsection{Universal Coalgebra}\label{sec:universal_coalgebra}
% --------------------------------------------------------------------

\cite{rutten00}

Systems

Bisimulation Relation (\S\ref{sec:bisimulation})



% --------------------------------------------------------------------
\subsection{Covariety}\label{sec:covariety}
% --------------------------------------------------------------------

% --------------------------------------------------------------------
\subsection{Free Object}\label{sec:free_object}
% --------------------------------------------------------------------

% --------------------------------------------------------------------
\subsection{Unification}\label{sec:unification}
% --------------------------------------------------------------------

\subsubsection{Syntactic Unification}\label{sec:syntactic_unification}



% ====================================================================
\section{Magma}\label{sec:magma}
% ====================================================================

A \emph{Magma}, $M$, is an Algebraic Structure
(\S\ref{sec:universal_algebra}) with a single Closed Binary Operation,
$M \times M \rightarrow M$.
\\ \\
No special properties:

$\begin{array}{c||c|c|}
  * & \mathbf{a} & \mathbf{b} \\ \hline \hline
  \mathbf{a} & a & b \\ \hline
  \mathbf{b} & a & a \\ \hline
\end{array}$ $\quad$ $\begin{array}{c||c|c|}
  * & \mathbf{a} & \mathbf{b} \\ \hline \hline
  \mathbf{a} & a & a \\ \hline
  \mathbf{b} & b & a \\ \hline
\end{array}$ \\ \hfill \\

$\begin{array}{c||c|c|}
  * & \mathbf{a} & \mathbf{b} \\ \hline \hline
  \mathbf{a} & b & b \\ \hline
  \mathbf{b} & a & b \\ \hline
\end{array}$ $\quad$ $\begin{array}{c||c|c|}
  * & \mathbf{a} & \mathbf{b} \\ \hline \hline
  \mathbf{a} & b & a \\ \hline
  \mathbf{b} & b & b \\ \hline
\end{array}$ \\ \hfill \\

$\begin{array}{c||c|c|}
  * & \mathbf{a} & \mathbf{b} \\ \hline \hline
  \mathbf{a} & b & b \\ \hline
  \mathbf{b} & a & a \\ \hline
\end{array}$ $\quad$ $\begin{array}{c||c|c|}
  * & \mathbf{a} & \mathbf{b} \\ \hline \hline
  \mathbf{a} & b & a \\ \hline
  \mathbf{b} & b & a \\ \hline
\end{array}$
\\ \\
Commutative:

$\begin{array}{c||c|c|}
  * & \mathbf{a} & \mathbf{b} \\ \hline \hline
  \mathbf{a} & b & a \\ \hline
  \mathbf{b} & a & a \\ \hline
\end{array}$ $\quad$ $\begin{array}{c||c|c|}
  * & \mathbf{a} & \mathbf{b} \\ \hline \hline
  \mathbf{a} & b & b \\ \hline
  \mathbf{b} & b & a \\ \hline
\end{array}$
\\ \\
Band (Associative, Idempotent):

$\begin{array}{c||c|c|}
  * & \mathbf{a} & \mathbf{b} \\ \hline \hline
  \mathbf{a} & a & a \\ \hline
  \mathbf{b} & b & b \\ \hline
\end{array}$ $\quad$ $\begin{array}{c||c|c|}
  * & \mathbf{a} & \mathbf{b} \\ \hline \hline
  \mathbf{a} & a & b \\ \hline
  \mathbf{b} & a & b \\ \hline
\end{array}$
\\ \\
Semilattice (Associative, Commutative, Idempotent):

$\begin{array}{c||c|c|}
  * & \mathbf{a} & \mathbf{b} \\ \hline \hline
  \mathbf{a} & a & a \\ \hline
  \mathbf{b} & a & a \\ \hline
\end{array}$ $\quad$ $\begin{array}{c||c|c|}
  * & \mathbf{a} & \mathbf{b} \\ \hline \hline
  \mathbf{a} & b & b \\ \hline
  \mathbf{b} & b & b \\ \hline
\end{array}$ \\ \hfill \\

$\begin{array}{c||c|c|}
  * & \mathbf{a} & \mathbf{b} \\ \hline \hline
  \mathbf{a} & a & b \\ \hline
  \mathbf{b} & b & b \\ \hline
\end{array}$ $\quad$ $\begin{array}{c||c|c|}
  * & \mathbf{a} & \mathbf{b} \\ \hline \hline
  \mathbf{a} & a & a \\ \hline
  \mathbf{b} & a & b \\ \hline
\end{array}$
\\ \\
Abelian Group (Associative, Identity, Invertible, Commutative):

$\begin{array}{c||c|c|}
  * & \mathbf{a} & \mathbf{b} \\ \hline \hline
  \mathbf{a} & a & b \\ \hline
  \mathbf{b} & b & a \\ \hline
\end{array}$ $\quad$ $\begin{array}{c||c|c|}
  * & \mathbf{a} & \mathbf{b} \\ \hline \hline
  \mathbf{a} & b & a \\ \hline
  \mathbf{b} & a & b \\ \hline
\end{array}$



% --------------------------------------------------------------------
\subsection{Commutative Non-associative Magma}\label{sec:commutative_magma}
% --------------------------------------------------------------------

% --------------------------------------------------------------------
\subsection{Free Magma}\label{sec:free_magma}
% --------------------------------------------------------------------

For a Set $X$, the \emph{Free Magma}, $M_X$, is the Set of
Non-associative Words (Strings) on $X$ with parentheses retained.

Free Object (\S\ref{sec:free_object})



% ====================================================================
\section{Semigroup}\label{sec:semigroup}
% ====================================================================

A \emph{Semigroup} is Magma (\S\ref{sec:magma}) with an Associative
Binary Operation. A Semigroup is differentiated from a Monoid
(\S\ref{sec:monoid}) by not requiring an Identity Element, and from
a Group (\S\ref{sec:group}) by not requiring Inverses.



% --------------------------------------------------------------------
\subsection{Subsemigroup}\label{sec:subsemigroup}
% --------------------------------------------------------------------

% --------------------------------------------------------------------
\subsection{Semigroup Action}\label{sec:semigroup_action}
% --------------------------------------------------------------------

A \emph{Semigroup Action} is a rule associating each Element of a
Semigroup, $S$, with a \emph{Transformation} $f : X \rightarrow X$ on
a Set $X$, such that a Product of two Semigroup Elements is associated
with the Composite of the two corresponding Transformations. A
\emph{Left Semigroup Action}, $\bullet$, for a Semigroup $(S,*)$ and
Set $X$ is defined as:
\[
  \forall s,t \in S\;\forall x \in X, s \bullet (t \bullet x) = (s * t)
  \bullet x
\]
and also known as an \emph{$S$-act}. Such a Semigroup Action is
equivalent to a Semigroup Homomorphism on the Set of Functions on $X$.
A \emph{Right Semigroup Action} is defined as:
\[
  \forall s,t \in S\;\forall x \in X, (x \bullet s) \bullet t = x
  \bullet (s * t)
\]
A \emph{Faithful Semigroup Action} (or \emph{Effective Semigroup
  Action}) has the Property:
\[
  \forall s, t \in S, s \bullet x \neq t \bullet x
\]
which is Isomorphic to a Transformation Semigroup
(\S\ref{sec:transformation_semigroup}).



\subsubsection{$S$-homomorphism}\label{sec:s_homomorphism}

For two $S$-acts, $T$ and $T'$, an \emph{$S$-homomorphism} $F : T
\rightarrow T'$ is defined such that:
\[
  \forall s \in S, x \in X, F(sx) = sF(x)
\]
The Set of all $S$-homomorphisms between a pair of $S$-acts is denoted
$Hom_S(T,T')$.

For a fixed Semigroup $S$, one may define Categories
$S\text{-}\mathbf{Act}$ and $\mathbf{Act}\text{-}S$ with Left and
Right $S$-acts as Objects, respectively, with $S$-homomorphisms as
Morphisms.



\subsubsection{Transformation Semigroup}\label{sec:transformation_semigroup}

A \emph{Transformation Semigroup} is a pair $(X,S)$ where $X$ is a Set
and $S$ is a Semigroup of \emph{Transformations} of $X$, that is, a
Set of Functions $f : X \rightarrow X$ that is Closed under
Composition. This is the Semigroup analogue of a Permutation Group
(\S\ref{sec:permutation_group}). Any Semigroup can be realized as a
Transformation Semigroup of some Set. A Transformation Semigroup that
includes the Identity Function is a Transformation Monoid
(\S\ref{sec:transformation_monoid}). A Transformation Semigroup can be
made into a Semigroup Action $\bullet$ of $S$ by Evaluation:
\[
  \forall s \in S, x \in X, s \bullet x = s(x)
\]



% --------------------------------------------------------------------
\subsection{Regular Semigroup}\label{sec:regular_semigroup}
% --------------------------------------------------------------------

A \emph{Regular Semigroup} $(S,*)$ has the Property:
\[
  \forall a \in S, \exists x \in S : axa = a
\]
where $x$ is called a \emph{Pseudoinverse}. Every Group
(\S\ref{sec:group}) is a Regular Semigroup.



% --------------------------------------------------------------------
\subsection{Involution Semigroup}\label{sec:involution_semigroup}
% --------------------------------------------------------------------

Involutive (\S\ref{sec:involution}) Anti-automorphism



% ====================================================================
\section{Monoid}\label{sec:monoid}
% ====================================================================

A \emph{Monoid} is a Semigroup with an Identity Element.

The set of all Endomorphisms of an Object, $X$, in a Category, $C$,
\[
    Hom(X,X)
\]
defines a Monoid and is denoted $End_C(X)$.



% --------------------------------------------------------------------
\subsection{Monoid Congruence}\label{sec:monoid_congruence}
% --------------------------------------------------------------------

A \emph{Monoid Congruence} on a Monoid $(M,*)$ is an Equivalence
Relation, $\sim$, that respects the Monoid Operator:
\[
  a,b,c,d \in M, a \sim c \wedge b \sim d \Rightarrow a*b \sim c*d
\]

See also Syntactic Congruence \S\ref{sec:syntactic_congruence}.



\subsubsection{Quotient Monoid}\label{sec:quotient_monoid}

For a Monoid Congruence, $\sim$, over a Monoid, $(M,*)$, a
\emph{Quotient Monoid}, $M/\sim$, can be defined with Equivalence
Classes as of $\sim$ as Objects and Binary Operation $\bullet$ defined
as:
\[
  a,b \in M, [a]\bullet[b] = [a*b]
\]



% --------------------------------------------------------------------
\subsection{Recognizable Subset}\label{sec:recognizable}
% --------------------------------------------------------------------

A Subset of a Monoid $S \subseteq N$ is \emph{Recognized} by a Monoid
$M$ if there exists a Morphism $\phi : N \rightarrow M$ such that $S =
\phi^{-1}(\phi(S))$, and $S$ is \emph{Recognizable} if it is
Recognized by a Finite Monoid. This requires that there is a Subset $T
\subseteq M$ such that $\phi(S) \subseteq T$ and $\phi(N \setminus S)
\subseteq (M \setminus T)$.

For the Free Monoid (\S\ref{sec:free_monoid}) $A^*$ over an Alphabet
$A$, the Recognizable Subsets of $A^*$ are the Regular Languages
(\S\ref{sec:regular_language}).



% --------------------------------------------------------------------
\subsection{Monoid Action}\label{sec:monoid_action}
% --------------------------------------------------------------------

A \emph{Monoid Action} is a special case of a Semigroup Action
(\S\ref{sec:semigroup_action}) where the Identity Element of the
Monoid is associated with the Identity Transformation of a Set. If the
Monoid is taken to be a Category with one Object, $\mathbf{M}$, the
Monoid Action, $f$, is a Functor from that Category to the Category of
Sets, $\mathbf{Set}$:
\[
  f : \mathbf{M} \rightarrow \mathbf{Set}
\]


\subsubsection{Operator Monoid}\label{sec:operator_monoid}

An \emph{Operator Monoid} is a Monoid $M$ with an Action on a Set.



% --------------------------------------------------------------------
\subsection{Transformation Monoid}\label{sec:transformation_monoid}
% --------------------------------------------------------------------

A \emph{Transformation Monoid} is a Transformation Semigroup
(\S\ref{sec:transformation_semigroup}) that includes the Identity
Function.

A Transformation Monoid with Invertible Elements is a Permutation
Group (\S\ref{sec:permutation_group}).

Any Monoid $M$ is an Effective Transformation Monoid on its underlying
Set.



\subsubsection{Full Transformation Monoid}\label{sec:full_transformation}

The Set of all Transformations of a Set $X$ gives a \emph{Full
  Transformation Monoid}, also called the \emph{Symmetric Semigroup}
of $X$, denoted $T_X$. An arbitrary Transformation Monoid is a
Submonoid of the Full Transformation Monoid. A Full Transformation
Monoid is a Regular Semigroup (\S\ref{sec:regular_semigroup}).



\subsubsection{Transition Monoid}\label{sec:transition_monoid}

Given a Semiautomaton (\S\ref{sec:semiautomaton}), $(\Sigma, X, T)$,
with Input Alphabet $\Sigma$, Set of States $X$, and Transition
Function $T : \Sigma \times X \rightarrow X$, the \emph{Transition
  Monoid} (also \emph{Characteristic Monoid}, \emph{Input Monoid}, or
\emph{Transition System}) is the Transformation Monoid consisting of
the Set of Transformations of $X$, $\{T_a : a \in \Sigma\}$ where
$T_a(x) = T(a,x)$ for $x \in X$, closed under Composition.



% --------------------------------------------------------------------
\subsection{Free Monoid}\label{sec:free_monoid}
% --------------------------------------------------------------------

A \emph{Free Monoid} on a Set $A$ is the Monoid $A^*$ whose Elements
are all possible Finite Sequences of zero or more Elements of $A$ with
String Concatenation (\S\ref{sec:string_concatenation}) as the Monoid
Operation and the Empty String $\varepsilon$ as the Identity Element.

Free Monoids are Unique up to Isomorphism.

Underlying Set $U(M)$

Free Monoid $F(X)$

$i_X : X \rightarrow U F (X)$

$Hom_\mathbf{Mon}(F(X), M) \cong Hom_\mathbf{Set}(X, U(M))$

In Programming Languages:

\begin{itemize}
  \item $\mathtt{[]}$ (Haskell)
\end{itemize}



% --------------------------------------------------------------------
\subsection{Syntactic Monoid}\label{sec:syntactic_monoid}
% --------------------------------------------------------------------

Given a Subset of a Monoid, $S \subset M$, a \emph{Syntactic Monoid},
$M(S)$, is the Quotient Monoid (\S\ref{sec:quotient_monoid}) formed by
the Syntactic Congruence Relation (\S\ref{sec:syntactic_congruence})
$\equiv_S$ where the Objects are Equivalence Classes in $M / \equiv_S$
and the Operation $\bullet$ is defined as:
\[
  [a] \bullet [b] = [ab]
\]

The Syntactic Monoid, $M(L)$, of a Formal Language
(\S\ref{sec:formal_language}), $L$, is the smallest Monoid that
Recognizes (\S\ref{sec:recognizable}) $L$.

The Syntactic Monoid of a Formal Language is Isomorphic to the
Transformation Monoid (\S\ref{sec:transformation_monoid}) of the
Minimal Automaton (\S\ref{sec:minimum_dfa}) accepting the Language.



\subsubsection{Syntactic Quotient}\label{sec:syntactic_quotient}

\emph{Syntactic Quotient} cf. String\S Quotient
\ref{sec:string_quotient}.

The \emph{Right Quotient} of a Subset of a Monoid $S \subset M$, by an
Element $a \in M$, is defined as:
\[
  S / a = \{ s \in M\;|\;sa \in S \}
\]
and the \emph{Left Quotient} as:
\[
  a \backslash S = \{ s \in M\;|\;as \in S \}
\]

A Formal Language $L$ is Regular (\S\ref{sec:regular_language}) if and
only if the Family of Quotients $\{ m \ L \;|\; m \in M \}$ is Finite.



\subsubsection{Syntactic Relation}\label{sec:syntactic_relation}

A Syntactic Quotient (\S\ref{sec:syntactic_quotient}) of a Subset of a
Monoid $S \subset M$ induces an Equivalence Relation called a
\emph{Syntactic Relation} (or \emph{Syntactic Equivalence}). The
\emph{Right Syntactic Equivalence Relation} is defined as:
\[
  \sim_S = \{ (a,b) \in M \times M \;|\; S/a = S/b\}
\]
and the \emph{Left Syntactic Equivalence Relation} as:
\[
  \prescript{}{S}{\sim} = \{ (a,b) \in M \times M \;|\;
  a \backslash S = b \backslash S \}
\]



\subsubsection{Syntactic Congruence}\label{sec:syntactic_congruence}

\emph{Total Syntactic Congruence} (or \emph{Myhill Congruence}) with
respect to a Subset of a Monoid $S \subset M$, denoted $\equiv_S$, is
defined for $u,v \in M$ as:
\[
  u \equiv_S v \Leftrightarrow
  (\forall x,y \in M, xuy \in S \Leftrightarrow xvy \in S)
\]



\paragraph{Disjunctive Set}\label{sec:disjunctive_set}\hfill \\

A \emph{Disjunctive Set} is a Subset of a Monoid $S \subset M$ such
that the Syntactic Congruence defined by $S$ is the Equality Relation.



\subsubsection{Trace Monoid}\label{sec:trace_monoid}



% --------------------------------------------------------------------
\subsection{Graphic Monoid}\label{sec:graphic_monoid}
% --------------------------------------------------------------------

% FIXME move this to category theory?

Graphic Category (\S\ref{sec:graphic_category})

Hyperplane (\S\ref{sec:hyperplane}) arrangements



\subsubsection{Hegelian Taco}\label{sec:hegelian_taco}



% --------------------------------------------------------------------
\subsection{Comonoid}\label{sec:comonoid}
% --------------------------------------------------------------------

Substructural Logic (\S\ref{sec:substructural_logic}), Linear Type
Systems (\S\ref{sec:linear_type})



% ====================================================================
\section{Quasigroup}\label{sec:quasigroup}
% ====================================================================

A \emph{Quasigroup}, $Q$, is a Magma with a Binary Operation that satisfies
the \emph{Latin Square Property}:
\[
  \forall a, b \in Q,\;\exists ! x,y \in Q : a * x = b \wedge y * a = b
\]
which allows for the Unique Equations defining \emph{Left} and
\emph{Right Division}, $x = a \backslash b$ and $y = b / a$
respectively.



% --------------------------------------------------------------------
\subsection{Pique}\label{sec:pique}
% --------------------------------------------------------------------

A \emph{Pique} (\emph{Pointed Idempotent Quasigroup}) is a Quasigroup
with an Idempotent Element. Taking an Abelian Group and its
Subtraction Operation as Quasigorup Multiplication gives a Pique,
$(A,-)$, where the Group's Identity Element is the Pointed Idempotent.



% --------------------------------------------------------------------
\subsection{Loop}\label{sec:quasigroup_loop}
% --------------------------------------------------------------------

A \emph{Loop} is a Quasigroup (\S\ref{sec:quasigroup}) with an
Identity Element. A Loop with the Associative Property is a Group
(\S\ref{sec:group}).



% ====================================================================
\section{Group Theory}\label{sec:group_theory}
% ====================================================================

\HandRight\; See also Groupoids (\S\ref{sec:groupoid})



% --------------------------------------------------------------------
\subsection{Group}\label{sec:group}
% --------------------------------------------------------------------

A \emph{Group} is a Monoid (\S\ref{sec:monoid}) with an Inverse for
every Element. That is, a Group is a Set $G$, and a Binary \emph{Group
  Operation}, $\cdot$, expressed as a Tuple, $(G,\cdot)$, satisfying
four \emph{Group Axioms}:
\begin{enumerate}
    \item Closure: $\forall a,b \in G, a \cdot b \in G$
    \item Associativity: $\forall a,b,c \in G, (a \cdot b) \cdot c = a
      \cdot (b \cdot c)$
    \item Identity Element: $\exists! e \in G : \forall a \in G,
      e \cdot a = a \cdot e = a$
    \item Inverse elements: $\forall a \in G, \exists b \in G :
      a \cdot b = b \cdot a = e$
\end{enumerate}
The Identity Element is the only Element with the unique Property:
\[
    e \cdot e = e
\]

The Signature (\S\ref{sec:signature}) for Groups is $\{\cdot, 1,
^{-1}\}$.

The number of Elements in a Group $\mathrm{G}$ is known as its
\emph{Order} and may be denoted $|\mathrm{G}|$.

\emph{Cayley's Theorem} states that every Group is Isomorphic to a
Group of Permutations (\S\ref{sec:permutation}).

A Group, $G$, within a Category, $\mathbf{C}$, may be viewed as a
Subset of the Hom-set of an Object, $X$:
\[
    G \subseteq Hom_{\mathbf{C}}(X,X)
\]



\subsubsection{Trivial Group}\label{sec:trivial_group}

A \emph{Trivial Group} is a Group $\{e\}$ with only an Identity
Element and no others. The Trivial Group may be denoted $0$ when
Group Operation is thought of as addition or as $1$ when Group
Operation is thought of as multiplication.

Given any Group $G$ with Identity Element $e$, the Trivial Group
$\{e\}$ is a Subgroup (\S\ref{sec:subgroup}) of $G$, called the
\emph{Trivial Subgroup}:
\[
    \{e\} \leq G
\]

All Trivial Groups are Isomorphic to each other, so \emph{the} Trivial
Group may be spoken of. The Trivial Group is the Zero Object in the
Category of Groups.

The Trivial Group is a Cyclic Group (\S\ref{sec:cyclic_group}) of
Order 1 and may be denoted $Z_1$ in this context.



\subsubsection{Group Order}\label{sec:group_order}

\subsubsection{Period}\label{sec:period}

(or \emph{Order}) of an Element

Periodic Group (\S\ref{sec:periodic_group})



\subsubsection{Group Generator}\label{sec:group_generator}

Cyclic Subgroup (\S\ref{sec:cyclic_subgroup})



\subsubsection{Group Word}\label{sec:group_word}

For a Group $G$ and a Subset of $G$, $S$, a \emph{Word} in $S$ is any
Expression of the form:
\[
    s_1^{\varepsilon_1}s_2^{\varepsilon_2} \cdots s_n^{\varepsilon_n}
\]
where $s_1,\ldots,s_n \in S$ and $\varepsilon_i \in \{-1, 1\}$ and $n$
is the \emph{Length} of the Word. The \emph{Empty Word} is used as the
Identity Element in the Free Group of Words (\S\ref{sec:free_group}).



\paragraph{Reduction}\label{sec:word_reduction}
\hfill \\

If an Element appears in a Word next to its Inverse, a
\emph{Reduction} may be applied which removes those Elements from the
Word due to the Group Axioms which imply that the resulting Word is
equivalent. A \emph{Reduced Word} contains no such redundant pairs.

A Word is \emph{Cyclically Reduced} if and only if every Cyclic
Permutation (\S\ref{sec:cyclic_permutation}) of the Word is Reduced
(that is, it is Reduced and the first and last Element are not
Inverses).



\subsubsection{Center}\label{sec:group_center}

The \emph{Center} of a Group $G$, denoted $Z(G)$, is a Normal Subgroup
of $G$ defined as:
\[
    Z(G) = \{ z \in G | \forall g \in G, zg = gz \}
\]
If $G$ is an Abelian Group, $Z(G) = G$.

For a Symmetric Group (\S\ref{sec:symmetric_group}) $S_n$:
\[
    Z(S_n) = \{e\}
\]

For a General Linear Group (\S\ref{sec:general_linear_group})
$GL_n(R)$:
\[
    Z(GL_n(R)) = \{\lambda \mathrm{I}\}
\]

Characteristic Subgroup (\S\ref{sec:characteristic_subgroup})



\subsubsection{Centralizer}\label{sec:group_centralizer}

\subsubsection{Normalizer}\label{sec:group_normalizer}

\subsubsection{Commutator}\label{sec:group_commutator}

Commutator (\S\ref{sec:commutator})

Commutator Subgroup (\S\ref{sec:commutator_subgroup})



\subsubsection{Conjugacy Class}\label{sec:conjugacy_class}

Two Elements $a$ and $b$ of a Group $G$ are \emph{Conjugates} if there
is an Element $g \in G$ such that:
\[
    gag^{-1} = b
\]
which reads ``Conjugation of $a$ by $g$ results in $b$.'' Conjugation
is Invariant if and only if the Elements are Commutative:
\[
    gag^{-1} = a \Leftrightarrow ag = ga
\]

A \emph{Conjugacy Class} for an Element $a$ is defined as:
\[
    Cl(a) = \{ b \in G | \exists g \in G : b = gag^{-1}\}
\]



% --------------------------------------------------------------------
\subsection{Group Homomorphism}\label{sec:group_homomorphism}
% --------------------------------------------------------------------

A \emph{Group Homomorphism} $h$, between two Groups, $G$ and $H$, is a
Morphism $h : G \rightarrow H$ that preserves Group operations
$\cdot_G$ and $\cdot_H$. That is, for $x,y \in G$:
\[
    h(x \cdot_G y) = h(x) \cdot_H h(y)
\]
From this it follows that Group Homomorphisms have the Properties:
\[
    h(e_G) = e_H
\]\[
    h(x^{-1}) = h(x)^{-1}
\]

There is always a \emph{Trivial Homomorphism} between any two Groups
$G$ and $H$:
\[
    \forall x \in G, h (x) = e_H
\]

The Composition of two Homomorphisms $h : G \rightarrow H$ and $g : F
\rightarrow G$, $h \circ g$ is a Homomorphism.

Taken as Monoidal Categories (\S\ref{sec:monoidal_category}), two
Groups $G, H$ may be related by a Functor $f$ which is equivalent to a
Group Homomorphism:
\[
    f : G \rightarrow H
\]
and for $x,y \in G$:
\[
    f(xy) = f(x)f(y)
\]



\subsubsection{Group Homomorphism Image}\label{sec:group_image}

The \emph{Image} of a Group Homomorphism:
\[
    im(h) = \{ x' \in H | h(x) = x', x \in G \}
\]
The Image of a $h$ is a Subgroup (\S\ref{sec:subgroup}) of $H$:
\[
    im(h) \subset H
\]

If the Image of $h$ is equal to $H$, and the Kernel
(\S\ref{sec:group_kernel}) is $\{e_G\}$ (\S\ref{sec:morphism_kernel}),
then $h$ is an Isomorphism (\S\ref{sec:group_isomorphism}).



\subsubsection{Group Homomorphism Kernel}\label{sec:group_kernel}

The \emph{Kernel} of a Group Homomorphism, $f : G \rightarrow G'$,
denoted by $ker(f)$ is defined as:
\[
    ker(f) = \{g \in G | f(g) = e_{G'}\}
\]
where $e_{G'}$ is the Identity Element of $G'$, that is, the Preimage
or Fiber (\S\ref{sec:function}) of the Singleton Set $\{e_{G'}\}$.
The Kernel is a Normal Subgroup (\S\ref{sec:normal_subgroup}) of the
Domain Group of the Group Homomorphism, in this case $G$:
\[
    ker(f) \triangleleft G
\]

If $ker(f) = \{e_G\}$ and the Image (\S\ref{sec:group_image}) is equal
to the Codomain Group, $im(f) = G'$, then $f$ is an Isomorphism
(\S\ref{sec:group_isomorphism}).

An Equivalence Class may be defined for an Element $g \in G$ by taking
the Left Coset (\S\ref{sec:coset}) of the Kernel, $ker(f) = H
\triangleleft G$:
\[
    gH = \{ gh | h \in H \}
\]
Each such Equivalence Class has the same Order as $H$, which results
in the following Corollary for a Group Homomorphism $f : G \rightarrow
G'$ with Kernel $H$:
\[
    |G| = |H||im(f)|
\]



\subsubsection{Group Isomorphism}\label{sec:group_isomorphism}

A \emph{Group Isomorphism} is a Bijective Group Homomorphism.

A Group Homomorphism $h : G \rightarrow H$ is an Isomorphism if
$ker(h) = e_G$ and $im(h) = H$. If $G = H$ then $h$ is an
\emph{Automorphism} (\S\ref{sec:group_automorphism}).

The existence of an Isomorphism between Groups $G$ and $H$ imply the
following Properties:
\begin{itemize}
    \item $|G| = |H|$
    \item $G$ is Abelian $\Leftrightarrow$ $H$ is Abelian
    \item $G$ and $H$ have the same number of Elements of every Order
\end{itemize}



\paragraph{Isomorphism Theorem}\label{sec:isomorphism_theorem}

\emph{First Isomorphism Theorem}

\emph{Second Isomorphism Theorem}

\emph{Third Isomorphism Theorem}



\subsubsection{Group Automorphism}\label{sec:group_automorphism}

A \emph{Automorphism} is an Endomorphism that is also an Isomorphism.



\paragraph{Inner Automorphism}\label{sec:inner_automorphism}
\hfill \\

An \emph{Inner Automorphism} of a Group $G$ is an Automorphism $f : G
\rightarrow G$ defined by:
\[
    \forall g \in G, f(g) = a^{-1}ga
\]
where $a$ is a given fixed Element of $G$.



% --------------------------------------------------------------------
\subsection{Subgroup}\label{sec:subgroup}
% --------------------------------------------------------------------

Given a Group $G$, a \emph{Subgroup} $H$ is a Subset of Group Elements
that are still a Group under the Group Operation of $G$, denoted $H
\leq G$. When a Subgroup $H$ is a Proper Subset of $G$, $H$ is a
\emph{Proper Subgroup} of $G$, denoted $H \neq G$. If a Subgroup has
the same Order as the containing Group then the Groups are equivalent.

All Groups have at least two Subgroups: the Trivial Group
(\S\ref{sec:trivial_group}) and Group itself. A Group with exactly
these two Subgroups and no others is a \emph{Simple Group}
(\S\ref{sec:simple_group}).

If a Subgroup $G$ of $H$ is Non-abelian then $H$ is also Non-abelian.

The Subgroups of $(\mathbb{Z},+)$ are of the form $(b\mathbb{Z},+)$
where $b \in \mathbb{Z}$.



\subsubsection{Coset}\label{sec:coset}

The \emph{Coset} of a Subgroup (\S\ref{sec:subgroup}) $H$ in a Group
$G$ is defined as:
\begin{description}
\item[Left Coset:] $gH = {gh : h \in H}$
\item[Right Coset:] $Hg = {hg : h \in H}$
\end{description}
These Subsets are Disjoint and Partition (\S\ref{sec:partition}) $G$.

The \emph{Index} of a Subgroup $H$ is the number of distinct Left
Cosets of $H$, denoted $[G:H]$. As a Corollary:
\[
    |G| = |H|[G:H]
\]



\subsubsection{Normal Subgroup}\label{sec:normal_subgroup}

A \emph{Normal Subgroup} of a Group $G$ is a \emph{Subgroup} $H$ that
is invariant under Conjugation (\S\ref{sec:conjugacy_class}) by the
Elements of $G$, $gHg^{-1} = H$ and is denoted $H \triangleleft G$.
$H$ is a Normal Subgroup of $G$ if and only if $\forall g \in G, gH =
Hg$.

In an Abelian Group, all Subgroups are Normal Subgroups.



\subsubsection{Characteristic Subgroup}
\label{sec:characteristic_subgroup}

Invariant under all Automorphisms of the Parent Group

Group Center (\S\ref{sec:group_center})



\paragraph{Commutator Subgroup}\label{sec:commutator_subgroup}
\hfill \\

\emph{Commutator Subgroup}

Subgroup Generated (\S\ref{sec:group_generator}) by all the
Commutators (\S\ref{sec:group_commutator}) of a Group



\subsubsection{Cyclic Subgroup}\label{sec:cyclic_subgroup}

Given any Group $G$ with Element $g$, the Subgroup Generated
(\S\ref{sec:group_generator}) by the single Element $g$ is called the
\emph{Cyclic Subgroup} of $g$, denoted $<g>$, containing all Powers of
$g$.

If $m$ is the smallest positive Integer such that $g^m = e$, then $m$
is the \emph{Order} of $g$. By \emph{Lagrange's Theorem}, in a Finite
Group, every Element has a Finite Order and the Order of every Element
divides evenly the Order of the Group.



\subsubsection{Skeleton}\label{sec:group_skeleton}



% --------------------------------------------------------------------
\subsection{Group Product}\label{sec:group_product}
% --------------------------------------------------------------------

The \emph{Group Product} of two Groups $G$ and $H$, denoted $G \times
H$, is a Group with Elements from the Cartesian Product of the
Elements of $G$ and $H$, $\{(g,h) | g \in G, h \in H\}$, and Group
Operation defined Componentwise:
\[
    (g_1, h_1) \cdot_{G \times H} (g_2, h_2)
    = (g_1 \cdot_G g_2, h_1 \cdot_H h_2)
\]

Direct Product (\S\ref{sec:direct_product})



% --------------------------------------------------------------------
\subsection{Free Product}\label{sec:free_product}
% --------------------------------------------------------------------

Coproduct (\S\ref{sec:coproduct}) of Groups $A \oplus B$

cf. Direct Sum of Modules, Disjoint Union of Sets

\HandRight\; Note that for Abelian Groups, the Product and Coproduct
are Isomorphic, given by the Direct Sum (\S\ref{sec:direct_sum})



% --------------------------------------------------------------------
\subsection{Periodic Group}\label{sec:periodic_group}
% --------------------------------------------------------------------

\emph{Periodic Group} (or \emph{Torsion Group}) if all Elements have
Finite Period (\S\ref{sec:group_order})



\subsubsection{Finite Group}\label{sec:finite_group}

A \emph{Finite Group} is a Group with a Finite Order
(\S\ref{sec:group_order})

Any Finite Non-abelian Group is of Even Order.

All Finite Groups are Periodic Groups



\subsubsection{Group Exponent}\label{sec:group_exponent}



% --------------------------------------------------------------------
\subsection{Simple Group}\label{sec:simple_group}
% --------------------------------------------------------------------

A Group $G$ is a \emph{Simple Group} if it has only the Trivial Group
and the entire Group $G$ as Normal Subgroups
(\S\ref{sec:normal_subgroup}).

The only Abelian Simple Groups are the Cyclic Groups
(\S\ref{sec:cyclic_group}) of Prime Order.

For $n \geq 5$, the Alternating Group (\S\ref{sec:alternating_group})
$A_n$ is a Simple Group.



\subsubsection{Finite Simple Group}\label{sec:finite_simple_group}

\subsubsection{Infinite Simple Group}\label{sec:infinite_simple_group}



% --------------------------------------------------------------------
\subsection{Free Group}\label{sec:free_group}
% --------------------------------------------------------------------

The \emph{Free Group}, $F_S$, over a Set, $S$, called the \emph{Free
  Generating Set}, consists of all Reduced Words
(\S\ref{sec:group_word}) in $S$ as Elements and Concatenation of Words
(with Reduction) as the Group Operation. As a Group Presentation
(\S\ref{sec:group_presentation}):
\[
    F_S = \langle S, \rangle
\]
Every Word is Conjugate to a Cyclically Reduced Word
(\S\ref{sec:word_reduction}), and a Cyclically Reduced Conjugate of a
Cyclically Reduced Word is a Cyclic Permutation of the Word.

A Group $G$ is called Free if it is Isomorphic to $F_S$ for some
Subset $S$ of $G$, that is, every Element of $G$ can be written
uniquely as a Product of finitely many Elements of $S$ and their
Inverses.

Every Subgroup of a Free Group is Free. \cite{hatcher02}

$S = {1}, F_S = (\ints,+)$



% --------------------------------------------------------------------
\subsection{Automorphism Group}\label{sec:automorphism_group}
% --------------------------------------------------------------------

The Automorphisms of an Object $x$ form an \emph{Automorphism Group},
denoted $Aut(x)$ under Composition of Automorphisms.

The Automorphism Group is a Subgroup (\S\ref{sec:subgroup}) of the
Symmetric Group.

Automorphism Group is a Submonoid of the Endomorphism Monoid %FIXME

$Aut_\cat{C}(x) = End_\cat{C}(x) \cap Iso_\cat{C}(C) =
Iso_\cat{C}(x,x)$

Every Group is an Automorphism Group up to Equivalence %FIXME



\subsubsection{Symmetric Group}\label{sec:symmetric_group}

The Group whose Elements are all the Permutations (Bijections) of a
Set $S$ is the \emph{Symmetric Group} $Sym(S)$. The Symmetric Group on
${1, 2, ..., n} \in \mathbb{N}$ is denoted $\mathrm{S}_n$.

For finite $n$, $\mathrm{S}_n$ is a Finite Group of Order $n!$.

\begin{itemize}
    \item $S_1 = \{e\}$
    \item $S_2 = \{e,\tau\}$ where $\tau$ is the Transposition
      (\S\ref{sec:transposition}) $(12)$
    \item $S_3 = \{e, \tau, \tau', \tau'', \sigma, \sigma'\}$ where
      $\sigma$ and $\sigma'$ are Permutations of length 3: $(123)$ and
      $(321)$ respectively
\end{itemize}
The Group $S_n$ is Non-abelian for $n \geq 3$.

For $k \leq n$, $S_k \subset S_n$.

Symmetry (\S\ref{sec:structure_symmetry})

Cayley's Theorem (\S\ref{sec:cayleys_theorem}): every Group is
Isomorphic to a Subgroup of the Symmetric Group on $G$



\paragraph{Permutation Group}\label{sec:permutation_group}
\hfill \\

Any Permutation Group on a Set $S$ with Cardinality $n$ is any
Subgroup of the Symmetric Group $\mathrm{S}_n$
(\S\ref{sec:symmetric_group}).



\subparagraph{Alternating Group}\label{sec:alternating_group}
\hfill \\

Group of Even Permutations of a Finite Set

For $n \geq 5$, $A_n$ is a Simple Group (\S\ref{sec:simple_group}).



\paragraph{Cayley's Theorem}\label{sec:cayleys_theorem}
\hfill \\

Every Group $G$ is Isomorphic to a Group of Permutations



\subsubsection{Inner Automorphism Group}\label{sec:inner_automorphism_group}

For the Automorphism Group of a Group $G$, $Aut(G)$, one can always
construct a Homomorphism:
\[
    f : G \rightarrow Aut(G)
\]
defined as:
\[
    \forall g \in G, f (g) (h) = g h g^{-1}
\]
with Kernel equal to the Center (\S\ref{sec:group_center}) of $G$:
\[
    ker(f) = Z(G)
\]
and the Image is a Subgroup of $Aut(G)$ called the \emph{Inner
  Automorphism Group} of $G$, denoted $Inn(G)$, defined as:
\[
    Inn(G) = \{ a \in Aut(G) | \exists g \in G : a(h) = g h g^{-1} \}
\]
where $a$ is an \emph{Inner Automorphism}
(\S\ref{sec:inner_automorphism}) of $G$.

By the First Isomorphism Theorem (\S\ref{sec:isomorphism_theorem}),
$Inn(G)$ is Isomorphic to the Quotient Group
(\S\ref{sec:quotient_group}) $G / Z(G) \cong Inn(G)$.



\subsubsection{Outer Automorphism Group}\label{sec:outer_automorphism_group}

\subsubsection{Complete Group}\label{sec:complete_group}



% --------------------------------------------------------------------
\subsection{Transformation Group}\label{sec:transformation_group}
% --------------------------------------------------------------------

\emph{Permutation Group} (\S\ref{sec:permutation_group}) Set

\emph{Matrix Group} (\S\ref{sec:matrix_group}) Vector Space

Automorphism Group (\S\ref{sec:automorphism_group})



\subsubsection{Group Action}\label{sec:group_action}

\subsubsection{Point Group}\label{sec:point_group}

\subsubsection{Space Group}\label{sec:space_group}

\subsubsection{Braid Group}\label{sec:braid_group}

\subsubsection{Matrix Group}\label{sec:matrix_group}

\paragraph{General Linear Group}\label{sec:general_linear_group}
\hfill \\

A \emph{General Linear Group} of Degree $n$ over a Ring
(\S\ref{sec:ring}) $R$ has as an Underlying Set of $n \times n$
Invertible Matrices and the Group Operation is ordinary Matrix
Multiplication and is denoted $GL_n(R)$ and is a Subgroup of
$Sym(R^n)$.



\paragraph{Special Linear Group}\label{sec:special_linear_group}
\hfill \\

A \emph{Special Linear Group} of Degree $n$ over a Ring
(\S\ref{sec:ring}) $R$, denoted $SL_n(R)$, is the Set of $n \times n$
Matrices with Determinant (\S\ref{sec:determinant}) equal to $1$, with
the Group Operation of ordinary Matrix Multiplication. This is a
Normal Subgroup of the General Linear Group $GL_n(R)$ given by the
Kernel of the Determinant Function $ker(det)$, where:
\[
  det : GL_n(R) \rightarrow R^\times
\]
and $R^\times$ is the Multiplicative Group
(\S\ref{sec:multiplicative_group}) on $R$.

The Elements of $SL_n(R)$ are the Volume and Orientation preserving
Linear Transformations of $R^n$.



\subsubsection{Symmetry Group}\label{sec:symmetry_group}

Isometry Group (\S\ref{sec:isometry_group})



\paragraph{Dihedral Group}\label{sec:dihedral_group}
\hfill \\

\emph{Dihedral Group}


\subparagraph{Klein 4-group}\label{sec:klein_4group}
\hfill \\

The \emph{Klein 4-group} or $K_4$ (also $V$ for
``\emph{Vierergruppe}'') is the Group resulting from the Direct
Product (\S\ref{sec:direct_product}) of two Cyclic Groups
(\S\ref{sec:cyclic_group}) of Order 2: $K_4 = Z_2 \times Z_2$.

As a Permutation Representation on 4 Elements $K_4$ is a Subgroup of
the Alternating Group (\S\ref{sec:alternating_group}) on 4 Elements,
$A_4$:
\[
    K_4 = \{ (), (12)(34), (13)(24), (14)(23) \}
\]

The Klein 4-group is an Abelian Group and is Isomorphic to the
Dihedral Group of Order 4 and is the smallest Non-cyclic Group.

The Automorphism Group (\S\ref{sec:automorphism_group}) $Aut(K_4)$ is
Isomorphic to $S_3$.



\paragraph{Line Group}\label{sec:line_group}
\hfill \\

Unit Cell



\subparagraph{Frieze Group}\label{sec:frieze_group}
\hfill \\

Infinite Cyclic Group (\S\ref{sec:infinite_cyclic})



\subsubsection{Continuous Transformation Group}
\label{sec:continuous_transformation_group}

Continuous Symmetry (\S\ref{sec:continuous_symmetry})



\paragraph{Lie Group}\label{sec:lie_group}
\hfill \\

Smooth Differentiable Manifold (\S\ref{sec:differentiable_manifold})



% --------------------------------------------------------------------
\subsection{Abelian Group}\label{sec:abelian_group}
% --------------------------------------------------------------------

\emph{Abelian Group} also \emph{Commutative Group}

In Abelian Groups, the Coproduct is Isomorphic to the Product:
\[
  A + B \cong A \times B
\]



\subsubsection{Torsion-free Rank}\label{sec:torsionfree_rank}

Torsion (\S\ref{sec:torsion})



\subsubsection{Additive Group}\label{sec:additive_group}

An \emph{Additive Group} refers to a Group where the Group Operation
can be thought of as \emph{Addition} (usually Abelian).



\subsubsection{Multiplicative Group}\label{sec:multiplicative_group}

A \emph{Multiplicative Group} is defined in terms of a Structure with
Invertible Elements such as a \emph{Ring} (\S\ref{sec:ring}) $R$
having Multiplication, $\bullet$, as one of its Operations:
\[
  (R \ {0}, \bullet)
\]

Circle Group (\S\ref{sec:circle_group})



\subsubsection{Divisible Group}\label{sec:divisible_group}

\subsubsection{Cyclic Group}\label{sec:cyclic_group}

A Group $G$ is \emph{Cyclic} when there exists an element $g \in G$
such that:
\[
    G = \langle g \rangle = \{ g^n | n \in \mathbb{Z} \}
\]
Any Group $G$ of Prime Order is Cyclic is an Abelian Simple Group
(\S\ref{sec:simple_group}) and is Generated by any $g \in G : g \neq
e$.

Every Finite Cyclic Group of Order $n$ is Isomorphic
(\S\ref{sec:group_isomorphism}) to the Additive Group
$\mathbb{Z}/n\mathbb{Z}$ of the Integers Modulo $n$.

Every Infinite Cyclic Group is Isomorphic to the Additive Group
$(\mathbb{Z}, +)$ of the Integers.



\paragraph{Finite Cyclic}\label{sec:finite_cyclic}

\paragraph{Infinite Cyclic}\label{sec:infinite_cyclic}
\hfill \\

Frieze Group (\S\ref{sec:frieze_group})



\subsubsection{Circle Group}\label{sec:circle_group}

Multiplicative Group (\S\ref{sec:multiplicative_group})

- Unit Circle on Complex Plane

- Geometrically defined Group due to Pascal's Theorem
  \cite{lemmermeyer-shirali09}



% --------------------------------------------------------------------
\subsection{Non-abelian Group}\label{sec:nonabelian_group}
% --------------------------------------------------------------------

% --------------------------------------------------------------------
\subsection{Quotient Group}\label{sec:quotient_group}
% --------------------------------------------------------------------

For a Normal Subgroup (\S\ref{sec:normal_subgroup}) $H$ of a Group
$G$, $H \triangleleft G$, the \emph{Quotient Group} $G/H$ is the Set
of all Left Cosets (\S\ref{sec:coset}) of $H$ in $G$:
\[
    G/H = \{ aH : a \in G \}
\]



\subsubsection{Group Extension}\label{sec:group_extension}

\subsubsection{Solvable Group}\label{sec:solvable_group}



% --------------------------------------------------------------------
\subsection{Topological Group}\label{sec:topological_group}
% --------------------------------------------------------------------

Group $G$ with Topology (\S\ref{sec:topology}) $\tau$ on $G$ such that
Group Binary and Inverse Operations are Continuous.



\subsubsection{Profinite Group}\label{sec:profinite_group}



% --------------------------------------------------------------------
\subsection{Algebraic Group}\label{sec:algebraic_group}
% --------------------------------------------------------------------

% --------------------------------------------------------------------
\subsection{Abstract Group}\label{sec:abstract_group}
% --------------------------------------------------------------------

\subsubsection{Group Representation}\label{sec:group_representation}

Linear Transformations (\S\ref{sec:linear_map}) of Vector Spaces
(\S\ref{sec:vector_space})



% ====================================================================
\section{Presentation}\label{sec:presentation}
% ====================================================================

\emph{Generators}

\emph{Relations}

\emph{Finitely Presented}

(\S\ref{sec:string_rewriting})

\emph{Semigroup Presentation}



\subsubsection{Finite Presentation}\label{sec:finite_presentation}
\cite{awodey06}

Generators: $g_1, \ldots, g_n$

Relations: $l_1 = r_1, \ldots, l_m = r_m$

Coequalizer (\S\ref{sec:coequalizer}):
\[
  F(m) {\xrightarrow[\quad\quad\quad]{l}
    \atop \xrightarrow[r]{\quad\quad\quad}} F(n)
  \xrightarrow{\quad\quad\quad} Q = F(n) / (l = r)
\]
where $l = [l_1, \ldots, l_m]$ and $r = [r_1, \ldots, r_m]$

Finitely Presented Category (\S\ref{sec:finitely_presented})



\subsubsection{Absolute Presentation}\label{sec:absolute_presentation}

\subsubsection{Group Presentation}\label{sec:group_presentation}

A \emph{Group Presentation} is the definition of a Group $G$ by a Set
$S$ of Generators (\S\ref{sec:free_group}) and a Set $R$ of Relations
between the Words in $S$ that represent the same Element of $G$:
\[
  \langle S | R \rangle
\]
The Group $G$ has this Presentation if it is Isomorphic to the
Quotient (\S\ref{sec:quotient_group}) of a Free Group on $S$. The Set
of Relations $R$ is said to \emph{Define} $G$ if every Relation in $G$
follows from those in $R$.



\subsubsection{Monoid Presentation}\label{sec:monoid_presentation}

A \emph{Monoid Presentation} is a description of a Monoid in terms of
a Set $\Sigma$ of Generators and a Set of Relations on the Free Monoid
$\Sigma*$.



% ====================================================================
\section{Ring Theory}\label{sec:ring_theory}
% ====================================================================

% --------------------------------------------------------------------
\subsection{Ring}\label{sec:ring}
% --------------------------------------------------------------------

A \emph{Ring} is a Set $R$ with two Binary Operators, $+$ and
$\cdot$, where:

\begin{itemize}
\item $R$ is an Abelian Group under $+$
    \begin{enumerate}
        \item $+$ is Associative
        \item $+$ is Commutative
        \item There exists an Additive Identity $0 \in R$
        \item For all $a \in R$, there exists an Additive Inverse $-a
          \in R$
    \end{enumerate}
\item $R$ is a Monoid under $\cdot$
    \begin{enumerate}
        \item $\cdot$ is Associative
        \item There exists a Multiplicative Identity $1 \in R$
    \end{enumerate}
\item $\cdot$ Distributes over $+$
    \begin{enumerate}
        \item $\forall a,b,c \in R,
            a \cdot (b + c) = (a \cdot b) + (a \cdot c)$
            (Left Distributivity)
        \item $\forall a,b,c \in R,
            (b + c) \cdot a = (b \cdot a) + (c \cdot a)$
            (Right Distributivity)
    \end{enumerate}
\end{itemize}
The Signature (\S\ref{sec:signature}) for Rings is $\{+, -, \cdot, 0,
1\}$

A \emph{Finite Ring} is a Ring that has a Finite number of Elements.

A \emph{Unit} is an Element of a Ring $R$ that has an Inverse
Element in the Multiplicative Monoid of $R$.

The term \emph{Unital Ring} (\S\ref{sec:unital_ring}) is used to
indicate a Ring with a Multiplicative Identity, to differentiate from
other \emph{Pseudo-rings} that may lack a Multiplicative Identity.

A \emph{Rng} (\S\ref{sec:rng}) is a Pseudo-ring that satisfies all
Ring axioms except a Multiplicative Identity.



\subsubsection{Unit}\label{sec:ring_unit}

\emph{Invertible Element} in the Multiplicative Monoid of $R$

Set of Units in any Ring is Closed under Multiplication



\subsubsection{Rng}\label{sec:rng}

\subsubsection{Unital Ring}\label{sec:unital_ring}

\subsubsection{Semiring}\label{sec:semiring}

A \emph{Semiring} (or \emph{Rig}) is a Ring without Additive Inverses.

Algebraic Data Types (\S\ref{sec:algebraic_datatype}) form a Semiring.



\subsubsection{Zero Ring}\label{sec:zero_ring}

The \emph{Zero Ring}, denoted $\{0\}$ or $\mathbf{0}$, is the Unique Ring
(up to Isomorphism) consisting of one Element with Operations:
\[
    0 + 0 = 0
\] \[
    0 * 0 = 0
\]
In the Category of all Rings, $\mathbf{Rng}$, the Zero Ring is
Terminal Object (\S\ref{sec:terminal_object}) and the Ring of Integers
$\mathbf{Z}$ is the Initial Object (\S\ref{sec:initial_object}).



% --------------------------------------------------------------------
\subsection{Commutative Ring}\label{sec:commutative_ring}
% --------------------------------------------------------------------

A \emph{Commutative Ring} is a Ring where $\cdot$ is Commutative.

\emph{Commutative Algebra}

\emph{Determinant}

Integers $\ints$ (\S\ref{sec:integer})

Fields (\S\ref{sec:field})



\subsubsection{Integral Domain}\label{sec:integral_domain}

An \emph{Integral Domain}, $R$, is a Non-zero Commutative Ring where
if $ab = 0$ in $R$, then either $a = 0$ or $b = 0$ in $R$.
Equivalently, the product of any two non-zero elements is non-zero.

All Finite Integral Domains are \emph{Finite Fields}
(\S\ref{sec:finite_field}).



\subsubsection{Polynomial Ring}\label{sec:polynomial_ring}

\emph{Polynomial Ring} (or \emph{Free Commutative Algebra})



% --------------------------------------------------------------------
\subsection{Division Ring}\label{sec:division_ring}
% --------------------------------------------------------------------

A \emph{Division Ring} is a Ring where every Nonzero Element has a
Multiplicative Inverse (but $\cdot$ is not required to be Commutative
as in a Field \S\ref{sec:field}).


By \emph{Wedderburn's Little Theorem} all \emph{Finite Division Rings}
are Commutative and therefore \emph{Finite Fields}
(\S\ref{sec:finite_field}).



% --------------------------------------------------------------------
\subsection{Nilpotent}\label{sec:nilpotent}
% --------------------------------------------------------------------

An Element $x$ of a Ring $R$ is \emph{Nilpotent} if there is a
positive Integer $n$ such that $x^n = 0$.



% --------------------------------------------------------------------
\subsection{Free Algebra}\label{sec:free_algebra}
% --------------------------------------------------------------------

A \emph{Free Algebra}, $\mathbf{A}$, is defined by a Set of \emph{Free
  Generators}, $S$, and a Type Signature, $\rho$, which Generate an
Underlying Set, $A$. If $\psi : S \rightarrow A$ is a Function,
$\mathbf{A}$ may be represented by the Free Algebra $(A,\psi)$ if for
every Algebra $\mathbf{B}$ of type $\rho$ with Function $\tau : S
\rightarrow B$, there exists a unique Homomorphism $\sigma : A
\rightarrow B$ such that $\sigma\psi = \tau$.

Effect Handlers (Algebraic Effects \S\ref{sec:effect_handler})



% --------------------------------------------------------------------
\subsection{Involutive Ring}\label{sec:involutive_ring}
% --------------------------------------------------------------------

% --------------------------------------------------------------------
\subsection{Involutive Algebra}\label{sec:involutive_algebra}
% --------------------------------------------------------------------

\emph{Involutive Algebra} (or \emph{*-algebra})



% ====================================================================
\section{Field Theory}\label{sec:field_theory}
% ====================================================================

% --------------------------------------------------------------------
\subsection{Field}\label{sec:field}
% --------------------------------------------------------------------

A \emph{Field} is a Nonzero Commutative Ring
(\S\ref{sec:commutative_ring}) with a Multiplicative Inverse for every
Nonzero Element or equivalently a Ring whose Nonzero Elements form an
Abelian Group (\S\ref{sec:abelian_group}) under Multiplication.

Functions in a Function Space (\S\ref{sec:function_space}) $X
\rightarrow F$ into a Field $F$ have a Vector (\S\ref{sec:vector})
structure with two Pointwise Addition Operators and Scalar
Multiplication.



\subsubsection{Total Field}\label{sec:total_field}

\subsubsection{Closed Field}\label{sec:closed_field}

\subsubsection{Finite Field}\label{sec:finite_field}

A \emph{Finite Field} or \emph{Galois field} is a Field that contains
a finite number of Elements with the \emph{Order} being equal to the
number of Elements.

A Finite Field only exists when the Order is a Prime Power
(\S\ref{sec:prime_number}).



\subsubsection{Ordered Field}\label{sec:ordered_field}

An \emph{Ordered Field} is a Field with a Total Ordering
(\S\ref{sec:total_order}) compatible with the Field Operations.



\subsubsection{Algebraically Closed Field}
\label{sec:algebraically_closed}



% --------------------------------------------------------------------
\subsection{Ring Ideal}\label{sec:ring_ideal}
% --------------------------------------------------------------------

Closure, Absorption

Order Ideal (\S\ref{sec:order_ideal})

For every Ring $R$, $\{0\}$ and $R$ are Ideals where $\{0\}$ is called
the \emph{Zero Ideal} and $R$ is called the \emph{Unit Ideal}. If $R$
is a Division Ring (\S\ref{sec:division_ring}) or Field
(\S\ref{sec:field}) then $\{0\}$ and $R$ are the only Ideals.

Integers: (Principal) Ideals correspond one-for-one with Non-negative Integers



% --------------------------------------------------------------------
\subsection{Principal Ideal}\label{sec:principal_ideal}
% --------------------------------------------------------------------

\emph{Principal Ideal}

Generated by a single Element of $R$ by multiplication with every
Element of $R$



% --------------------------------------------------------------------
\subsection{Quotient Ring}\label{sec:quotient_ring}
% --------------------------------------------------------------------

% --------------------------------------------------------------------
\subsection{Tensor}\label{sec:tensor}
% --------------------------------------------------------------------

\subsubsection{Symmetric Tensor}\label{sec:symmetric_tensor}

Invariant under a Permutation of its Vector Arguments



% --------------------------------------------------------------------
\subsection{Field Extension}\label{sec:field_extension}
% --------------------------------------------------------------------



% ====================================================================
\section{Torsion}\label{sec:torsion}
% ====================================================================

Torsion-free Rank (\S\ref{sec:torsionfree_rank})



% ====================================================================
\section{Relational Algebra}\label{sec:relational_algebra}
% ====================================================================

\emph{Domain Relational Calculus}

% --------------------------------------------------------------------
\subsection{Projection}\label{sec:relational_projection}
% --------------------------------------------------------------------



% ====================================================================
\section{Homological Algebra}\label{sec:homological_algebra}
% ====================================================================

Homology Theory (\S\ref{sec:homology_theory})



% --------------------------------------------------------------------
\subsection{Chain Complex}\label{sec:chain_complex}
% --------------------------------------------------------------------



% ====================================================================
\section{Invariant Theory}\label{sec:invariant_theory}
% ====================================================================
