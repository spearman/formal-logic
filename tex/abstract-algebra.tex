%%%%%%%%%%%%%%%%%%%%%%%%%%%%%%%%%%%%%%%%%%%%%%%%%%%%%%%%%%%%%%%%%%%%%%
%%%%%%%%%%%%%%%%%%%%%%%%%%%%%%%%%%%%%%%%%%%%%%%%%%%%%%%%%%%%%%%%%%%%%%
\part{Abstract Algebra}\label{sec:abstract_algebra}
%%%%%%%%%%%%%%%%%%%%%%%%%%%%%%%%%%%%%%%%%%%%%%%%%%%%%%%%%%%%%%%%%%%%%%
%%%%%%%%%%%%%%%%%%%%%%%%%%%%%%%%%%%%%%%%%%%%%%%%%%%%%%%%%%%%%%%%%%%%%%

\emph{Presentations}: Generators, Relations

\emph{Finitely Presented}



% ====================================================================
\section{Magma}\label{sec:magma}
% ====================================================================

A \emph{Magma}, $M$, is an Algebraic Structure
(\S\ref{sec:universal_algebra}) with a single Closed Binary
Operation, $M \times M \rightarrow M$.



% ====================================================================
\section{Semigroup}\label{sec:semigroup}
% ====================================================================

A \emph{Semigroup} is Magma (\S\ref{sec:magma}) with an Associative
Binary Operation. A Semigroup is differentiated from a Monoid
(\S\ref{sec:monoid}) by not requiring an Identity Element, and from
a Group (\S\ref{sec:group}) by not requiring Inverses.



% --------------------------------------------------------------------
\subsection{Monoid}\label{sec:monoid}
% --------------------------------------------------------------------

A \emph{Monoid} is a Semigroup with an Identity Element. The set of
all Endomorphisms of an Object, $X$, in a Category, $C$,
\[
    Hom(X,X)
\]
defines a Monoid and is denoted $End_C(X)$.

\subsubsection{Free Monoid}\label{sec:free_monoid}

\subsubsection{Trace Monoid}\label{sec:trace_monoid}

\subsubsection{Syntactic Monoid}\label{sec:syntactic_monoid}



% ====================================================================
\section{Group Theory}\label{sec:group_theory}
% ====================================================================

% --------------------------------------------------------------------
\subsection{Group}\label{sec:group}
% --------------------------------------------------------------------

A \emph{Group} is a Monoid (\S\ref{sec:monoid}) with an Inverse for
every element. That is, a Group is a Set $G$, and a Binary \emph{Group
  Operation}, $\cdot$, expressed as a Tuple, $(G,\cdot)$, satisfying
four \emph{Group Axioms}:
\begin{enumerate}
    \item Closure: $\forall a,b \in G, a \cdot b \in G$
    \item Associativity: $\forall a,b,c \in G, (a \cdot b) \cdot c = a
      \cdot (b \cdot c)$
    \item Identity element: $\exists! e \in G : \forall a \in G,
      e \cdot a = a \cdot e = a$
    \item Inverse elements: $\forall a \in G, \exists b \in G :
      a \cdot b = b \cdot a = e$
\end{enumerate}

A Group may be formed from a Set $X$ by adding all Automorphisms
(\S\ref{sec:automorphism}) of $X$: $Aut(X)$. \emph{Cayley's Theorem}
states that every Group is Isomorphic to a Group of Permutations
(\S\ref{sec:permutations}).

A Group, $G$, within a Category, $\mathbf{C}$, may be viewed as a
Subset of the Hom-set of an Object, $X$:
\[
    G \subseteq Hom_{\mathbf{C}}(X,X)
\]



\subsubsection{Trivial Group}\label{sec:trivial_group}

A \emph{Trivial Group} is a Group $\{e\}$ with only an Identity
Element and no others. The Trivial Group may be denoted $0$ when
Group Operation is thought of as addition or as $1$ when Group
Operation is thought of as multiplication.

Given any Group $G$ with Identity Element $e$, the Trivial Group
$\{e\}$ is a Subgroup (\S\ref{sec:subgroup}) of $G$, called the
\emph{Trivial Subgroup}:
\[
    \{e\} \leq G
\]

All Trivial Groups are Isomorphic to one another, so \emph{the}
Trivial Group may be spoken of. The Trivial Group is the Zero Object
in the Category of Groups.

The Trivial Group is a Cyclic Group (\S\ref{sec:cyclic_group}) of
Order 1 and may be denoted $Z_1$ in this context.



% --------------------------------------------------------------------
\subsection{Abelian Group}\label{sec:abelian_group}
% --------------------------------------------------------------------

\emph{Abelian Group} also \emph{Commutative Group}



\subsubsection{Additive Group}\label{sec:additive_group}

An \emph{Additive Group} refers to a Group where the Group Operation
can be thought of as \emph{Addition} (usually Abelian).
\[
    S_1 = \{e\}
\] \[
    S_2 = \{e, \tau\}
\]



\subsubsection{Cyclic Group}\label{sec:cyclic_group}

A Group $G$ is \emph{Cyclic} when there exists an element $g \in G$
such that:
\[
    G = \langle g \rangle = \{ g^n | n \in \mathbb{Z} \}
\]

Every Infinite Cyclic Group is Isomorphic to the Additive Group
$(\mathbb{Z}, +)$ of the Integers.

Every Finite Cyclic Group of Order $n$ is Isomorphic to the Additive
Group $\mathsf{Z}/n\mathsf{Z}$ of the Integers Modulo $n$.



% --------------------------------------------------------------------
\subsection{Groupoid}\label{sec:groupoid}
% --------------------------------------------------------------------

A \emph{Groupoid} is a Group with a Partial Function in place of a
Total Binary Operation. As a Category, a Groupoid is a Category in
which every Morphism is Invertible.

\emph{Fundamental Groupoid}



\subsubsection{$\inf$-groupoid}\label{sec:infinity_groupoid}

An \emph{$inf$-groupoid} is an $inf$-category
(\S\ref{sec:quasicategory}) generalization of a Groupoid.

By the \emph{Homotopy Hypothesis}, $inf$-groupoids are \emph{Spaces}
(\S\ref{sec:topological_space}).



% --------------------------------------------------------------------
\subsection{Subgroup}\label{sec:subgroup}
% --------------------------------------------------------------------

Given a Group $G$, a \emph{Subgroup} $H$ is a Subset of Group Elements
that are still a Group under the Group Operation of $G$, denoted $H
\leq G$. When a Subgroup $H$ is a Proper Subset of $G$, $H$ is a
\emph{Proper Subgroup} of $G$, denoted $H \neq G$.

All Groups have at least two Subgroups: the Trivial Group
(\S\ref{sec:trivial_group}) and Group itself. A Group with exactly
these two Subgroups and no others is a \emph{Simple Group}
(\S\ref{sec:simple_group}).



\subsubsection{Skeleton}

\subsubsection{Coset}\label{sec:group_coset}

The \emph{Coset} of a Subgroup (\S\ref{sec:subgroup}) $H$ in a Group $G$
is defined as
\begin{description}
\item[Left Coset] $gH = {gh : h \in H}$
\item[Right Coset] $Hg = {hg : h \in H}$
\end{description}



\subsubsection{Normal Subgroup}\label{sec:normal_subgroup}

Given a Group $G$ and Subgroup $H$, $H$ is a \emph{Normal Subgroup} if
and only if $\forall g \in G, gH = Hg$.



\subsubsection{Cyclic Subgroup}\label{sec:cyclic_subgroup}

\emph{Order}



\subsubsection{Commutator Subgroup}\label{sec:commutator_subgroup}

\emph{Commutator Subgroup}



% --------------------------------------------------------------------
\subsection{Finite Group}\label{sec:finite_group}
% --------------------------------------------------------------------

% --------------------------------------------------------------------
\subsection{Simple Group}\label{sec:simple_group}
% --------------------------------------------------------------------

A Group $G$ is a \emph{Simple Group} if it has only the Trivial Group
and the entire Group $G$ as Normal Subgroups.



\subsubsection{Finite Simple Group}\label{sec:finite_simple_group}



\subsubsection{Infinite Simple Group}\label{sec:infinite_simple_group}



% --------------------------------------------------------------------
\subsection{Group Homomorphism}\label{sec:group_homomorphism}
% --------------------------------------------------------------------

A \emph{Group Homomorphism} preserves Group operations.

Taken as Monoidal Categories, two Groups $G, H$ may be related by a
Functor $f$ which is equivalent to a Group Homomorphism:
\[
    f : G \rightarrow H
\]
and for $x,y \in G$:
\[
    f(xy) = f(x)f(y)
\]
If the Image of $f$ is equal to $H$, and the Kernel is $\{e\}$ %FIXME



\subsubsection{Group Kernel}\label{sec:group_kernel}

The \emph{Kernel} of a Group Homomorphism, $h : G \rightarrow H$,
denoted by $ker(h)$ is defined as:
\[
    ker(h) = {g \in G : f(g) = e_H}
\]
where $e_H$ is an Element of $H$, and so the Kernel is just the
Preimage of the Singleton Set $\{e_H\}$. Such a Kernel is a Normal
Subgroup.

The Kernel of $G$ is $\{x | f(x)=e_H, x \in G\}$ and always
gives a \emph{Normal Subgroup} (\S\ref{sec:subgroup}) of $G$,
written $H \triangleleft G$.



% --------------------------------------------------------------------
\subsection{Transformation Group}\label{sec:transformation_group}
% --------------------------------------------------------------------

\emph{Permutation Group} (\S\ref{sec:permutation_group}) Set

\emph{Matrix Group} (\S\ref{sec:matrix_group}) Vector Space

\subsubsection{Permutation Group}\label{sec:permutation_group}

Any Permutation Group on a Set $S$ with Cardinality $n$ is a Subgroup
of the Symmetric Group $\mathrm{S}_n$ (\S\ref{sec:symmetric_group}).



\subsubsection{Symmetric Group}\label{sec:symmetric_group}

The Group of all Permutations of a Set $S$ is the \emph{Symmetric
  Group} $Sym(S)$. The Symmetric Group on ${1, 2, ..., n} \in
\mathbb{N}$ is denoted $\mathrm{S}_n$.

\paragraph{Cayley's Theorem}\label{sec:cayleys_theorem}



\subsubsection{Symmetry Group}\label{sec:symmetry_group}

\paragraph{Dihedral Group}\label{sec:dihedral_group}



\subsubsection{Alternating Group}\label{sec:alternating_group}

\subsubsection{Matrix Group}\label{sec:matrix_group}

\subsubsection{Lie Group}\label{sec:lie_group}



% --------------------------------------------------------------------
\subsection{Periodic Group}\label{sec:periodic_group}
% --------------------------------------------------------------------

% --------------------------------------------------------------------
\subsection{Solvable Group}\label{sec:solvable_group}
% --------------------------------------------------------------------

% --------------------------------------------------------------------
\subsection{Conjugacy Class}\label{sec:conjugacy_class}
% --------------------------------------------------------------------

% --------------------------------------------------------------------
\subsection{Quotient Group}\label{sec:quotient_group}
% --------------------------------------------------------------------

% --------------------------------------------------------------------
\subsection{Free Group}\label{sec:free_group}
% --------------------------------------------------------------------

% --------------------------------------------------------------------
\subsection{Abstract Group}\label{sec:abstract_group}
% --------------------------------------------------------------------

\subsubsection{Presentation}\label{sec:presentation}

\emph{Generators}

\emph{Relations}

(\S\ref{sec:semithue_system})

\emph{Semigroup Presentation}

\emph{Monoid Presentation}

\emph{Group Presentation}



\subsubsection{Absolute Presentation}\label{sec:absolute_presentation}



% --------------------------------------------------------------------
\subsection{Topological Group}\label{sec:topological_group}
% --------------------------------------------------------------------

% --------------------------------------------------------------------
\subsection{Algebraic Group}\label{sec:algebraic_group}
% --------------------------------------------------------------------

% --------------------------------------------------------------------
\subsection{Representation Theory}\label{sec:representation_theory}
% --------------------------------------------------------------------

Given a Functor $R$ from a Group $G$ to a general Category
$\mathbf{C}$
\[
    R : G \rightarrow \mathbf{C}
\]
Such a Functor $R$ is termed a \emph{Representation} of $G$ in
$\mathbf{C}$.



% ====================================================================
\section{Ring Theory}\label{sec:ring_theory}
% ====================================================================

% --------------------------------------------------------------------
\subsection{Ring}\label{sec:ring}
% --------------------------------------------------------------------

A \emph{Ring} is a Set $R$ with two Binary Operators, $+$ and
$\cdot$, where:

\begin{itemize}
\item $R$ is an Abelian Group under $+$
    \begin{enumerate}
        \item $+$ is Associative
        \item $+$ is Commutative
        \item There exists an Additive Identity $0 \in R$
        \item For all $a \in R$, there exists an Additive Inverse $-a
          \in R$
    \end{enumerate}
\item $R$ is a \emph{Monoid} under $\cdot$
    \begin{enumerate}
        \item $\cdot$ is Associative
        \item There exists a Multiplicative Identity $1 \in R$
    \end{enumerate}
\item $\cdot$ Distributes over $+$
    \begin{enumerate}
        \item $\forall a,b,c \in R,
            a \cdot (b + c) = (a \cdot b) + (a \cdot c)$
            (Left Distributivity)
        \item $\forall a,b,c \in R,
            (b + c) \cdot a = (b \cdot a) + (c \cdot a)$
            (Right Distributivity)
    \end{enumerate}
\end{itemize}

A \emph{Finite Ring} is a Ring that has a Finite number of Elements.

A \emph{Unit} is an Element of a Ring $R$ that has an Inverse
Element in the Multiplicative Monoid of $R$.

The term \emph{Unital Ring} is used to indicate a Ring with a
Multiplicative Identity, to differentiate from other
\emph{Pseudo-rings} that may lack a Multiplicative Identity.

A \emph{Rng} is a Pseudo-ring that satisfies all Ring axioms except a
Multiplicative Identity.

A \emph{Semiring} is a Ring without the requirement of Additive
Inverses.



\subsubsection{Zero Ring}\label{sec:zero_ring}

The \emph{Zero Ring}, denoted $\{0\}$ or $\mathbf{0}$, is the Unique Ring
(up to Isomorphism) consisting of one Element with Operations:
\[
    0 + 0 = 0
\] \[
    0 * 0 = 0
\]
In the Category of all Rings, $\mathbf{Rng}$, the Zero Ring is
\emph{Terminal Object} and the Ring of Integers $\mathbf{Z}$ is the
\emph{Initial Object} (\S\ref{sec:initial_terminal}).



% --------------------------------------------------------------------
\subsection{Commutative Ring}\label{sec:commutative_ring}
% --------------------------------------------------------------------

A \emph{Commutative Ring} is a Ring where $\cdot$ is Commutative.

\emph{Commutative Algebra}

\emph{Determinant}



\subsubsection{Integral Domain}\label{sec:integral_domain}

An \emph{Integral Domain}, $R$, is a Non-zero Commutative Ring where
if $ab = 0$ in R, then either $a = 0$ or $b = 0$ in $R$. Equivalently,
the product of any two non-zero elements is non-zero.

All Finite Integral Domains are \emph{Finite Fields}
(\S\ref{sec:finite_field}).



% --------------------------------------------------------------------
\subsection{Division Ring}\label{sec:division_ring}
% --------------------------------------------------------------------

A \emph{Division Ring} is a Ring where every Nonzero Element has a
Multiplicative Inverse (but $\cdot$ is not required to be
Commutative).


By \emph{Wedderburn's Little Theorem} all \emph{Finite Division Rings}
are Commutative and therefore \emph{Finite Fields}
(\S\ref{sec:finite_field}).



% ====================================================================
\section{Field Theory}\label{sec:field_theory}
% ====================================================================

% --------------------------------------------------------------------
\subsection{Field}\label{sec:field}
% --------------------------------------------------------------------

A \emph{Field} is a Nonzero Commutative Ring with a Multiplicative
Inverse for every Nonzero Element or equivalently a Ring whose Nonzero
Elements form an Abelian Group (\S\ref{sec:abelian_group}) under
Multiplication.



\subsubsection{Total Field}\label{sec:total_field}

\subsubsection{Closed Field}\label{sec:closed_field}

\subsubsection{Finite Field}\label{sec:finite_field}

A \emph{Finite Field} or \emph{Galois field} is a Field that contains
a finite number of Elements with the \emph{Order} being equal to the
number of Elements.

A Finite Field only exists when the Order is a Prime Power
(\S\ref{sec:prime_number}).



% ====================================================================
\subsection{Module}
% ====================================================================

A \emph{Module} is a Unital Ring, $R$, together with an Abelian Group,
$(M, +)$, and an Operation called \emph{Scalar Multiplication} which
is either:
\[ R \times M \rightarrow M \]
for a \emph{Left $R$-module $M$}, $_R M$, or:
\[ M \times R \rightarrow M \]
for a \emph{Right $R$-module $M$}, $M_R$.

The Scalar Multiplication Operator is required that for all $r,s \in
R$ and $x,y \in M$ in a Left $R$-module $M$:
\begin{enumerate}
    \item $r(x + y) = rx + ry$
    \item $(r + s)x = rx + sx$
    \item $(rs)x = r(sx)$
    \item $1_Rx = x$
\end{enumerate}
or in a Right $R$-module $M$:
\begin{enumerate}
    \item $(x + y)r = xr + yr$
    \item $x(r + s) = xr + xs$
    \item $x(rs) = (sx)r$
    \item $x 1_R = x$
\end{enumerate}
where $1_R$ is the Multiplicative Identity for $R$. If the Ring is not
required to be Unital, then item (4) above can be ommitted, but can be
explicitly required by stating that we are talking about a
\emph{Unital Left/Right $R$-module $M$}.

\emph{Bimodule}

If $R$ is Commutative, then Left $R$-modules are the same as Right
$R$-modules and simply called \emph{$R$-modules}.

A Module Homomorphism is an \emph{Linear Map}.

\emph{Bilinear Map}

\emph{Multilinear Map}



% --------------------------------------------------------------------
\subsection{Ideal}\label{sec:ring_ideal}
% --------------------------------------------------------------------

\emph{Ideal}

\emph{Principal Ideal}



% --------------------------------------------------------------------
\subsection{Quotient Ring}
% --------------------------------------------------------------------



% --------------------------------------------------------------------
\subsection{Vector Space}\label{sec:vector_space}
% --------------------------------------------------------------------

\emph{Span}

\emph{Finite Dimensional Vector Space} - has a Span

\emph{Infinite Dimensional Vector Space} - does not have a Span

\emph{Linear Independence}

\emph{Basis} - Spans and is Linearly Independent

All Bases of a Vector Space $\mathbf{V}$ have the same number of
Elements equal to the \emph{Dimension} of $\mathbf{V}$,
$dim(\mathbf{V})$. The Dimension of a Vector Space is uniquely defined
because for any Vector Space, a Basis exists, and all Bases of a
Vector space have equal Cardinality (\S\ref{sec:cardinality}).

For a Finite Dimensional Vector Space a Subset of a Span defines a
Basis, and a Linearly Independent Subset can be extended to form a
Basis.

The number of Elements in a Spanning Subset of $\mathbf{V}$ is greater
than or equal to the Dimension of $\mathbf{V}$.

The number of Elements in a Linearly Independent Subset of
$\mathbf{V}$ is less than or equal to the Dimension of $\mathbf{V}$.

A Basis defines an Isomorphism of Vector Spaces:
\[
    \mathbf{V} \xrightarrow{f} F^n
\]

\emph{Tensor Product}, \emph{Outer Product}



\subsubsection{Inner Product}\label{sec:inner_product}



\subsubsection{Norm}\label{sec:vector_space_norm}

\emph{Seminorm}

\emph{Quasinorm}



% --------------------------------------------------------------------
\subsection{Bimodule}\label{sec:bimodule}
% --------------------------------------------------------------------



% ====================================================================
\section{Representation Theory}
% ====================================================================

% ====================================================================
\section{Operad Theory}\label{sec:operad_theory}
% ====================================================================

% ====================================================================
\section{Initial Algebra}\label{sec:initial_algebra}
% ====================================================================

% ====================================================================
\section{Heyting Algebra}\label{sec:heyting_algebra}
% ====================================================================

% --------------------------------------------------------------------
\subsection{Boolean Algebra}\label{sec:boolean_algebra}
% --------------------------------------------------------------------

Syntactically, every Boolean Term corresponds to a Propositional
Formula (\S\ref{sec:propositional_logic}).



% ====================================================================
\section{Relation Algebra}
% ====================================================================

% ====================================================================
\section{Relational Algebra}\label{sec:relational_algebra}
% ====================================================================

\emph{Domain Relational Calculus}



% ====================================================================
\section{Quantale}
% ====================================================================

% ====================================================================
\section{Invariant Theory}\label{sec:invariant_theory}
% ====================================================================
