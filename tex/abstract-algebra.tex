%%%%%%%%%%%%%%%%%%%%%%%%%%%%%%%%%%%%%%%%%%%%%%%%%%%%%%%%%%%%%%%%%%%%%%
%%%%%%%%%%%%%%%%%%%%%%%%%%%%%%%%%%%%%%%%%%%%%%%%%%%%%%%%%%%%%%%%%%%%%%
\part{Abstract Algebra}\label{sec:abstract_algebra}
%%%%%%%%%%%%%%%%%%%%%%%%%%%%%%%%%%%%%%%%%%%%%%%%%%%%%%%%%%%%%%%%%%%%%%
%%%%%%%%%%%%%%%%%%%%%%%%%%%%%%%%%%%%%%%%%%%%%%%%%%%%%%%%%%%%%%%%%%%%%%

\emph{Presentations}: Generators, Relations

\emph{Finitely Presented}



% ====================================================================
\section{Magma}\label{subsec:magma}
% ====================================================================

A \emph{Magma}, $M$, is an Algebraic Structure
(\S\ref{subsec:universal_algebra}) with a single Closed Binary
Operation, $M \times M \rightarrow M$.



% ====================================================================
\section{Semigroup}\label{subsec:semigroup}
% ====================================================================

A \emph{Semigroup} is Magma (\S\ref{subsec:magma}) with an Associative
Binary Operation. A Semigroup is differentiated from a Monoid
(\S\ref{subsec:monoid}) by not requiring an Identity Element, and from
a Group (\S\ref{subsec:group}) by not requiring Inverses.



% ====================================================================
\section{Monoid}\label{subsec:monoid}
% ====================================================================

A \emph{Monoid} is a Semigroup with an Identity Element. The set of
all Endomorphisms of an Object, $X$, in a Category, $C$,
\[
    Hom(X,X)
\]
defines a Monoid and is denoted $End_C(X)$.

% --------------------------------------------------------------------
\subsection{Free Monoid}\label{subsec:free_monoid}
% --------------------------------------------------------------------

% --------------------------------------------------------------------
\subsection{Trace Monoid}\label{subsec:trace_monoid}
% --------------------------------------------------------------------

% --------------------------------------------------------------------
\subsection{Syntactic Monoid}\label{subsec:syntactic_monoid}
% --------------------------------------------------------------------



% ====================================================================
\section{Group Theory}\label{sec:group_theory}
% ====================================================================

% --------------------------------------------------------------------
\subsection{Group}\label{subsec:group}
% --------------------------------------------------------------------

A \emph{Group} is a Monoid (\S\ref{subsec:monoid}) with an Inverse for
every Morphism (every Morphism is an Isomorphism). That is, a Group is
a Set $G$, and a Binary \emph{Group Operation}, $\cdot$, expressed as
a Tuple, $(G,\cdot)$, satisfying four \emph{Group Axioms}:
\begin{enumerate}
    \item Closure: $\forall a,b \in G, a \cdot b \in G$
    \item Associativity: $\forall a,b,c \in G, (a \cdot b) \cdot c = a
      \cdot (b \cdot c)$
    \item Identity element: $\exists! e \in G : \forall a \in G,
      e \cdot a = a \cdot e = a$
    \item Inverse elements: $\forall a \in G, \exists b \in G :
      a \cdot b = b \cdot a = e$
\end{enumerate}

A Group may be formed from a Set $X$ by adding all Automorphisms
(\S\ref{subsec:automorphism}) of $X$: $Aut(X)$. \emph{Cayley's
  Theorem} states that every Group is Isomorphic to a Group of
Permutations (\S\ref{subsec:permutations}).

A Group, $G$, within a Category, $\mathbf{C}$, may be viewed as a
Subset of the Hom-set of an Object, $X$:
\[
    G \subseteq Hom_{\mathbf{C}}(X,X)
\]

\emph{Trivial Group} ${1}$

An \emph{Additive Group} refers to one where the Group Operation can
be thought of as \emph{Addition} (usually Abelian).

\[
    S_1 = \{e\}
\] \[
    S_2 = \{e, \tau\}
\]



% --------------------------------------------------------------------
\subsection{Groupoid}\label{subsec:groupoid}
% --------------------------------------------------------------------

A \emph{Groupoid} is a Group with a Partial Function in place of a
Total Binary Operation. As a Category, a Groupoid is a Category in
which every Morphism is Invertible.

\emph{Fundamental Groupoid}



% --------------------------------------------------------------------
\subsection{Group Homomorphism}\label{subsec:group_homomorphism}
% --------------------------------------------------------------------

A \emph{Group Homomorphism} preserves Group operations.

Taken as Monoidal Categories, two Groups $G, H$ may be related by a
Functor $f$ which is equivalent to a Group Homomorphism:
\[
    f : G \rightarrow H
\]
and for $x,y \in G$:
\[
    f(xy) = f(x)f(y)
\]
If the Image of $f$ is equal to $H$, and the Kernel is $\{e\}$ %FIXME



\subsubsection{Group Kernel}\label{subsec:group_kernel}

The \emph{Kernel} of a Group Homomorphism, $h : G \rightarrow H$,
denoted by $ker(h)$ is defined as:
\[
    ker(h) = {g \in G : f(g) = e_H}
\]
where $e_H$ is an Element of $H$, and so the Kernel is just the
Preimage of the Singleton Set $\{e_H\}$. Such a Kernel is a Normal
Subgroup.

The Kernel of $G$ is $\{x | f(x)=e_H, x \in G\}$ and always
gives a \emph{Normal Subgroup} (\S\ref{subsec:subgroup}) of $G$,
written $H \triangleleft G$.



% --------------------------------------------------------------------
\subsection{Abelian Group}\label{subsec:abelian_group}
% --------------------------------------------------------------------

\emph{Abelian Group} also \emph{Commutative Group}



% --------------------------------------------------------------------
\subsection{Group Representations}\label{subsec:group_representation}
% --------------------------------------------------------------------

Given a Functor $R$ from a Group $G$ to a general Category
$\mathbf{C}$
\[
    R : G \rightarrow \mathbf{C}
\]
Such a Functor $R$ is termed a \emph{Representation} of $G$ in
$\mathbf{C}$.



% --------------------------------------------------------------------
\subsection{Subgroup}\label{subsec:subgroup}
% --------------------------------------------------------------------

A \emph{Subgroup} is a Subset of Group Elements that are still a Group
under the Group Operation.

Given a Group $G$ and Subgroup $H$, $H$ is a \emph{Normal Subgroup} if
and only if $\forall g \in G, gH = Hg$.

\emph{Commutator Subgroup}

\subsubsection{Skeleton}

\subsubsection{Coset}\label{subsec:group_coset}

The \emph{Coset} of a Subgroup (\S\ref{subsec:subgroup}) $H$ in a Group $G$
is defined as
\begin{description}
\item[Left Coset] $gH = {gh : h \in H}$
\item[Right Coset] $Hg = {hg : h \in H}$
\end{description}


\subsubsection{Cyclic Subgroup}\label{subsec:cyclic_subgroup}

\emph{Order}



% --------------------------------------------------------------------
\subsection{Permutation Group}\label{subsec:permutation_group}
% --------------------------------------------------------------------

% --------------------------------------------------------------------
\subsection{Symmetric Group}\label{subsec:symmetric_group}
% --------------------------------------------------------------------

The Group of all Permutations of a Set $S$ is the \emph{Symmetric
  Group} $Sym(S)$. The Symmetric Group on ${1, 2, ..., n} \in
\mathbb{N}$ is denoted $\mathrm{S}_n$.

\subsubsection{Cayley's Theorem}\label{subsec:cayleys_theorem}



% --------------------------------------------------------------------
\subsection{Quotient Group}\label{subsec:quotient_group}
% --------------------------------------------------------------------

% --------------------------------------------------------------------
\subsection{Abstract Groups}\label{subsec:abstract_groups}
% --------------------------------------------------------------------



% ====================================================================
\section{Ring Theory}\label{sec:ring_theory}
% ====================================================================

% --------------------------------------------------------------------
\subsection{Ring}\label{subsec:ring}
% --------------------------------------------------------------------

A \emph{Ring} is a Set $R$ with two Binary Operators, $+$ and
$\cdot$, where:

\begin{itemize}
\item $R$ is an Abelian Group under $+$
    \begin{enumerate}
        \item $+$ is Associative
        \item $+$ is Commutative
        \item There exists an Additive Identity $0 \in R$
        \item For all $a \in R$, there exists an Additive Inverse $-a
          \in R$
    \end{enumerate}
\item $R$ is a \emph{Monoid} under $\cdot$
    \begin{enumerate}
        \item $\cdot$ is Associative
        \item There exists a Multiplicative Identity $1 \in R$
    \end{enumerate}
\item $\cdot$ Distributes over $+$
    \begin{enumerate}
        \item $\forall a,b,c \in R,
            a \cdot (b + c) = (a \cdot b) + (a \cdot c)$
            (Left Distributivity)
        \item $\forall a,b,c \in R,
            (b + c) \cdot a = (b \cdot a) + (c \cdot a)$
            (Right Distributivity)
    \end{enumerate}
\end{itemize}

A \emph{Finite Ring} is a Ring that has a Finite number of Elements.

A \emph{Unit} is an Element of a Ring $R$ that has an Inverse
Element in the Multiplicative Monoid of $R$.

The term \emph{Unital Ring} is used to indicate a Ring with a
Multiplicative Identity, to differentiate from other
\emph{Pseudo-rings} that may lack a Multiplicative Identity.

A \emph{Rng} is a Pseudo-ring that satisfies all Ring axioms except a
Multiplicative Identity.

A \emph{Semiring} is a Ring without the requirement of Additive
Inverses.



\subsubsection{Zero Ring}\label{subsec:zero_ring}

The \emph{Zero Ring}, denoted $\{0\}$ or $\mathbf{0}$, is the Unique Ring
(up to Isomorphism) consisting of one Element with Operations:
\[
    0 + 0 = 0
\] \[
    0 * 0 = 0
\]
In the Category of all Rings, $\mathbf{Rng}$, the Zero Ring is
\emph{Terminal Object} and the Ring of Integers $\mathbf{Z}$ is the
\emph{Initial Object} (\S\ref{subsec:initial_terminal}).



% --------------------------------------------------------------------
\subsection{Commutative Ring}\label{subsec:commutative_ring}
% --------------------------------------------------------------------

A \emph{Commutative Ring} is a Ring where $\cdot$ is Commutative.

\emph{Commutative Algebra}

\emph{Determinant}



% --------------------------------------------------------------------
\subsection{Division Ring}\label{subsec:division_ring}
% --------------------------------------------------------------------

A \emph{Division Ring} is a Ring where every Nonzero Element has a
Multiplicative Inverse (but $\cdot$ is not required to be
Commutative).


By \emph{Wedderburn's Little Theorem} all \emph{Finite Division Rings}
are Commutative and therefore \emph{Finite Fields}
(\S\ref{subsec:finite_field}).



% ====================================================================
\section{Field Theory}\label{sec:field_theory}
% ====================================================================

% --------------------------------------------------------------------
\subsection{Field}\label{subsec:field}
% --------------------------------------------------------------------

A \emph{Field} is a Nonzero Commutative Ring with a Multiplicative
Inverse for every Nonzero Element or equivalently a Ring whose Nonzero
Elements form an Abelian Group (\S\ref{subsec:abelian_group}) under
Multiplication.



\subsubsection{Total Field}\label{subsec:total_field}

\subsubsection{Closed Field}\label{subsec:closed_field}

\subsubsection{Finite Field}\label{subsec:finite_field}

A \emph{Finite Field} or \emph{Galois field} is a Field that contains
a finite number of Elements with the \emph{Order} being equal to the
number of Elements.

A Finite Field only exists when the Order is a Prime Power
(\S\ref{subsec:prime_number}).



% ====================================================================
\subsection{Module}
% ====================================================================

A \emph{Module} is a Unital Ring, $R$, together with an Abelian Group,
$(M, +)$, and an Operation called \emph{Scalar Multiplication} which
is either:
\[ R \times M \rightarrow M \]
for a \emph{Left $R$-module $M$}, $_R M$, or:
\[ M \times R \rightarrow M \]
for a \emph{Right $R$-module $M$}, $M_R$.

The Scalar Multiplication Operator is required that for all $r,s \in
R$ and $x,y \in M$ in a Left $R$-module $M$:
\begin{enumerate}
    \item $r(x + y) = rx + ry$
    \item $(r + s)x = rx + sx$
    \item $(rs)x = r(sx)$
    \item $1_Rx = x$
\end{enumerate}
or in a Right $R$-module $M$:
\begin{enumerate}
    \item $(x + y)r = xr + yr$
    \item $x(r + s) = xr + xs$
    \item $x(rs) = (sx)r$
    \item $x 1_R = x$
\end{enumerate}
where $1_R$ is the Multiplicative Identity for $R$. If the Ring is not
required to be Unital, then item (4) above can be ommitted, but can be
explicitly required by stating that we are talking about a
\emph{Unital Left/Right $R$-module $M$}.

\emph{Bimodule}

If $R$ is Commutative, then Left $R$-modules are the same as Right
$R$-modules and simply called \emph{$R$-modules}.



% --------------------------------------------------------------------
\subsection{Ideal}\label{subsec:ring_ideal}
% --------------------------------------------------------------------

\emph{Ideal}

\emph{Principal Ideal}



% --------------------------------------------------------------------
\subsection{Quotient Ring}
% --------------------------------------------------------------------



% --------------------------------------------------------------------
\subsection{Vector Space}\label{subsec:vector_space}
% --------------------------------------------------------------------

\emph{Span}

\emph{Finite Dimensional Vector Space} - has a Span

\emph{Infinite Dimensional Vector Space} - does not have a Span

\emph{Linear Independence}

\emph{Basis} - Spans and is Linearly Independent

For a Finite Dimensional Vector Space a Subset of a Span defines a
Basis, and a Linearly Independent Subset can be extended to form a
Basis.

All Bases of a Vector Space $\mathbf{V}$ have the same number of
Elements equal to the \emph{Dimension} of $\mathbf{V}$,
$dim(\mathbf{V})$.

The number of Elements in a Spanning Subset of $\mathbf{V}$ is greater
than or equal to the Dimension of $\mathbf{V}$.

The number of Elements in a Linearly Independent Subset of
$\mathbf{V}$ is less than or equal to the Dimension of $\mathbf{V}$.

A Basis defines an Isomorphism of Vector Spaces:
\[
    \mathbf{V} \xrightarrow{f} F^n
\]

\emph{Bilinear Map}

\emph{Tensor Product}, \emph{Outer Product}




% --------------------------------------------------------------------
\subsection{Bimodule}\label{subsec:bimodule}
% --------------------------------------------------------------------



% ====================================================================
\section{Representation Theory}
% ====================================================================

% ====================================================================
\section{Operad Theory}\label{subsec:operad_theory}
% ====================================================================

% ====================================================================
\section{Initial Algebra}\label{subsec:initial_algebra}
% ====================================================================

% ====================================================================
\section{Heyting Algebra}\label{subsec:heyting_algebra}
% ====================================================================

% --------------------------------------------------------------------
\subsection{Boolean Algebra}\label{subsec:boolean_algebra}
% --------------------------------------------------------------------

Syntactically, every Boolean Term corresponds to a Propositional
Formula (\S\ref{subsec:propositional_logic}).



% ====================================================================
\section{Relation Algebra}
% ====================================================================

% ====================================================================
\section{Relational Algebra}
% ====================================================================

\emph{Domain Relational Calculus}



% ====================================================================
\section{Quantale}
% ====================================================================

% ====================================================================
\section{Invariant Theory}\label{subsec:invariant_theory}
% ====================================================================

% ====================================================================
\section{Algebraic Topology}\label{subsec:algebraic_topology}
% ====================================================================
