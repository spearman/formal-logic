%%%%%%%%%%%%%%%%%%%%%%%%%%%%%%%%%%%%%%%%%%%%%%%%%%%%%%%%%%%%%%%%%%%%%%%%%%%%%%%%
%%%%%%%%%%%%%%%%%%%%%%%%%%%%%%%%%%%%%%%%%%%%%%%%%%%%%%%%%%%%%%%%%%%%%%%%%%%%%%%%
\part{Abstract Algebra}\label{part:abstract_algebra}
%%%%%%%%%%%%%%%%%%%%%%%%%%%%%%%%%%%%%%%%%%%%%%%%%%%%%%%%%%%%%%%%%%%%%%%%%%%%%%%%
%%%%%%%%%%%%%%%%%%%%%%%%%%%%%%%%%%%%%%%%%%%%%%%%%%%%%%%%%%%%%%%%%%%%%%%%%%%%%%%%

(nlab):

\resizebox{\linewidth}{!}{
\begin{tabular}{|l l l|}
\hline
  \textbf{Algebra} &
  \textbf{Homological Algebra} (\S\ref{sec:homological_algebra}) &
  \textbf{Homotopical Algebra} (\S\ref{sec:homotopical_algebra}) \\
\hline
  Commutative (Abelian) Group (\S\ref{sec:commutative_group}) &
  Chain Complex (\S\ref{sec:chain_complex}) &
  Topological Spectrum (\S\ref{sec:topological_spectrum}) \\

  Ring (\S\ref{sec:ring}) &
  Differential Graded Ring (\S\ref{sec:differential_graded_ring}) &
  Ring Spectrum (\S\ref{sec:ring_spectrum}) \\

  Module (\S\ref{sec:module}) &
  Differential Graded Module (\S\ref{sec:differential_graded_module}) &
  Module Spectrum (\S\ref{sec:module_spectrum}) \\
\hline
\end{tabular}
}

Algebra: Composition, ``Reductionist''

Coalgebra: Decomposition, ``Black-boxes''

\iffalse

1 Set X:

%%%%%%%%%%%%%%%%%%%%%%%%%%%%%%%%%%%%%%%%%%%%%%%%%%%%%%%%%%%%%%%%%%%%%%%%%%%%%%%%
  1 Operator
  * : X x X -> X

    Magma         * Closed
    Semigroup     * Closed, Associative
    Monoid        * Closed, Associative, Identity
    Group         * Closed, Associative, Identity, Inverse
    Abelian Group * Closed, Associative, Identity, Inverse, Commutative

%%%%%%%%%%%%%%%%%%%%%%%%%%%%%%%%%%%%%%%%%%%%%%%%%%%%%%%%%%%%%%%%%%%%%%%%%%%%%%%%
  2 Operators
  * : X x X -> X
  + : X x X -> X

    Semiring (Rig) * + Distributive
                     + Closed, Associative, Identity, Commutative
                     * Closed, Associative, Identity
    Rng            * + Distributive
                     + Closed, Associative, Identity, Inverse, Commutative
                     * Closed, Associative
    Ring (Unital)  * + Distributive
                     + Closed, Associative, Identity, Inverse, Commutative
                     * Closed, Associative, Identity
    Division Ring  * + Distributive
                     + Closed, Associative, Identity, Inverse, Commutative
                     * Closed, Associative, Identity, Inverse
    Field          * + Distributive
                     + Closed, Associative, Identity, Inverse, Commutative
                     * Closed, Associative, Identity, Inverse, Commutative

2 Sets X Y:

%%%%%%%%%%%%%%%%%%%%%%%%%%%%%%%%%%%%%%%%%%%%%%%%%%%%%%%%%%%%%%%%%%%%%%%%%%%%%%%%
  2 Operators (Ring, Field)
  * : X x X -> X
  + : X x X -> X
  1 Operator (Group)
  & : Y x Y -> Y
  Scalar Multiplication
  ^ : Y x X -> X

    Module
                    ^ & Additive
                    ^ + Additive
                    ^ * Associative
              Ring: * + Distributive
                      + Closed, Associative, Identity, Inverse, Commutative
                      * Closed, Associative, Identity
            Abelian
             Group:   & Closed, Associative, Identity, Inverse, Commutative

    Vector Space
                    ^ & Additive
                    ^ + Additive
                    ^ * Associative
             Field: * + Distributive
                      + Closed, Associative, Identity, Inverse, Commutative
                      * Closed, Associative, Identity, Inverse, Commutative
            Abelian
             Group:   & Closed, Associative, Identity, Inverse, Commutative

%%%%%%%%%%%%%%%%%%%%%%%%%%%%%%%%%%%%%%%%%%%%%%%%%%%%%%%%%%%%%%%%%%%%%%%%%%%%%%%%
  2 Operators (Ring, Field)
  *  : X x X -> X
  +  : X x X -> X
  1 Operator (Group)
  &  : Y x Y -> Y
  Scalar Multiplication
  ^  : Y x X -> X
  Norm
  || : Y -> X

    Normed Vector Space
                    ^ & Additive
                    ^ + Additive
                    ^ * Associative
                   || ^ Distributive
                   || & Subadditive
             Field: * + Distributive
                      + Closed, Associative, Identity, Inverse, Commutative
                      * Closed, Associative, Identity, Inverse, Commutative
            Abelian
             Group:   & Closed, Associative, Identity, Inverse, Commutative


%%%%%%%%%%%%%%%%%%%%%%%%%%%%%%%%%%%%%%%%%%%%%%%%%%%%%%%%%%%%%%%%%%%%%%%%%%%%%%%%
  2 Operators (Ring, Field)
  *  : X x X -> X
  +  : X x X -> X
  1 Operator (Group)
  &  : Y x Y -> Y
  Scalar Multiplication
  ^  : Y x X -> X
  Norm
  || : Y -> X
  Inner Product
  ,  : Y x Y -> X

    Inner Product Space
                    ^ & Additive
                    ^ + Additive
                    ^ * Associative
                   || ^ Distributive
                   || & Subadditive
                      , Conjugate Symmetric, Sesquilinear, Positive-definite
             Field: * + Distributive
                      + Closed, Associative, Identity, Inverse, Commutative
                      * Closed, Associative, Identity, Inverse, Commutative
            Abelian
             Group:   & Closed, Associative, Identity, Inverse, Commutative

%%%%%%%%%%%%%%%%%%%%%%%%%%%%%%%%%%%%%%%%%%%%%%%%%%%%%%%%%%%%%%%%%%%%%%%%%%%%%%%%
  2 Operators (Ring, Field)
  * : X x X -> X
  + : X x X -> X
  2 Operators
  & : Y x Y -> Y
  x : Y x Y -> Y
  Scalar Multiplication
  ^ : Y x X -> X

    R-algebra
                    ^ & Additive
                    ^ + Additive
                    ^ * Associative
              Ring: * + Distributive
                      + Closed, Associative, Identity, Inverse, Commutative
                      * Closed, Associative, Identity
            Abelian
             Group:   & Closed, Associative, Identity, Inverse, Commutative

            Bilinear
            Product:
                      x Closed
                    x & Right Distributive, Left Distributive
                    x ^ Distributive

    K-algebra
                    ^ & Additive
                    ^ + Additive
                    ^ * Associative
             Field: * + Distributive
                      + Closed, Associative, Identity, Inverse, Commutative
                      * Closed, Associative, Identity, Inverse, Commutative
            Abelian
             Group:   & Closed, Associative, Identity, Inverse, Commutative

            Bilinear
            Product:
                      x Closed
                    x & Right Distributive, Left Distributive
                    x ^ Distributive


\fi



% ==============================================================================
\section{Universal Algebra}\label{sec:universal_algebra}
% ==============================================================================

\emph{Universal Algebra} is the study of \emph{Algebraic Structures}
(\S\ref{sec:algebraic_structure}), i.e. Structures (\S\ref{sec:structure}) with
Signatures consisting only of Functional Symbols and no Relation Symbols.
Universal Algebra together with Category Theory (Part
\ref{part:category_theory}) makes up Abstract Algebra.

Equational Logic (\S\ref{sec:equational_logic})

Equational Theories (\S\ref{sec:equational_theory})

two main Category Theoretic formulations of Universal Algebra:
\begin{enumerate}
  \item Lawvere Theories (\S\ref{sec:lawvere_theory})
  \item Monads (\S\ref{sec:monad})
\end{enumerate}



% ------------------------------------------------------------------------------
\subsection{Presentation}\label{sec:presentation}
% ------------------------------------------------------------------------------

\emph{Generators} (\S\ref{sec:generator})

\emph{Relations}

\emph{Finitely Presented}

(\S\ref{sec:string_rewriting})

\emph{Semigroup Presentation}

Category Presentation (\S\ref{sec:category_presentation})

$T$-algebra Presentation (\S\ref{sec:t_algebra_presentation})

\emph{Presented by Generators and Relations}: obtained from a Structure Freely
Generated (???) by a Set of Generators and imposed Relations among the
Generators.

Free Algebra (\S\ref{sec:free_algebra})

Free Monoid (\S\ref{sec:free_monoid})

Free Monad (\S\ref{sec:free_monad})

Free Category (\S\ref{sec:free_category})

Free Functor (\S\ref{sec:free_functor})

Free Object (\S\ref{sec:free_object})

\asterism

(Stay-Meredith16):

Lawvere63: an Equational Theory (\S\ref{sec:equational_theory}) is a
Presentation of a Category with Finite Products where all Objects are Powers of
a single Generating Object (i.e. a Lawvere Theory \S\ref{sec:lawvere_theory})

Milner92 \cite{milner92} -- now ``standard'' ``Presentation'' of a Computational
Calculus (\S\ref{sec:computation_model}):
\begin{itemize}
  \item Grammar (\S\ref{sec:formal_grammar}) describing the primary Data Type
    over which \emph{Computations} are carried out
  \item Structural Equivalence (\S\ref{sec:structural_equality}) used to erase
    Syntactic differences that are ``irrelevant'' to Computation
  \item Set of Rewrite Rules (\S\ref{sec:abstract_rewrite}) describing how to
    ``realize'' Computation through Operations on Data Structures
\end{itemize}
generalizing the Generators and Relations Presentations of Universal Algebra
with the Grammar replacing the Generators as the ``free construction'' and the
Structural Equivalence replacing the Relations, and the Rewrite Rules providing
the Computational Semantics (\S\ref{sec:computational_semantics})



\subsubsection{Finite Presentation}\label{sec:finite_presentation}
\cite{awodey06}

Generators: $g_1, \ldots, g_n$

Relations: $l_1 = r_1, \ldots, l_m = r_m$

Coequalizer (\S\ref{sec:coequalizer}):
\[
  F(m) {\xrightarrow[\quad\quad\quad]{l}
    \atop \xrightarrow[r]{\quad\quad\quad}} F(n)
  \xrightarrow{\quad\quad\quad} Q = F(n) / (l = r)
\]
where $l = [l_1, \ldots, l_m]$ and $r = [r_1, \ldots, r_m]$

Finitely Presented Category (\S\ref{sec:finitely_presented})

\url{https://golem.ph.utexas.edu/category/2017/04/gluing_together_finite_shapes.html}:

Thm. \emph{In a Category of Algebras $\cat{C}$ (e.g. Models of a
  Lawvere Theory), an Algebra $A$ is Finitely Presentable if and only
  if the Hom-functor $\cat{C}(A,-)$ Commutes with Filtered Colimits}
%FIXME xref filtered colimits



\subsubsection{Recursive Presentation}\label{sec:recursive_presentation}

\subsubsection{Absolute Presentation}\label{sec:absolute_presentation}

\subsubsection{Group Presentation}\label{sec:group_presentation}

A \emph{Group Presentation} is the definition of a Group $G$ by a Set $S$ of
Generators (\S\ref{sec:free_group}) and a Set $R$ of Relations between the Words
in $S$ that represent the same Element of $G$:
\[
  \langle S | R \rangle
\]
The Group $G$ has this Presentation if it is Isomorphic to the Quotient
(\S\ref{sec:quotient_group}) of a Free Group on $S$. The Set of Relations $R$ is
said to \emph{Define} $G$ if every Relation in $G$ follows from those in $R$.



\subsubsection{Monoid Presentation}\label{sec:monoid_presentation}

A \emph{Monoid Presentation} is a description of a Monoid in terms of a Set
$\Sigma$ of Generators and a Set of Relations on the Free Monoid $\Sigma*$.



% ------------------------------------------------------------------------------
\subsection{Algebraic Structure}\label{sec:algebraic_structure}
% ------------------------------------------------------------------------------

Structure (\S\ref{sec:structure})

Abstract Structure (\S\ref{sec:abstract_structure}): Recognizable Structure vs.
Combinatoric view (requires special arguments)

An Algebra may be limited by Axioms of \emph{Equational Laws}
(\S\ref{sec:equational_law}), e.g. the Associative Axiom.

Allowing for Infinitary Operations leads to the Algebraic Theory of Complete
Lattices (\S\ref{sec:complete_lattice}).

Free Object (\S\ref{sec:free_object})

\fist A Topos (\S\ref{sec:topos}) can be understood as a Cartesian Closed
Category (\S\ref{sec:cartesian_category}) with a ``natural categorical
abstraction'' of the notions of Subalgebra of any Algebraic Structure, as an
Elementary (\S\ref{sec:elementary_logic}) or First-order
(\S\ref{sec:firstorder_logic}) definition of \emph{Subobject} of an
Object------------ Subobject Classifier (\S\ref{sec:subobject_classifier})

\fist Relational Algebra (\S\ref{sec:relational_algebra})

(wiki):

Algebraic Structures on a Single Set:

\begin{itemize}
  \item no Binary Operations -- Sets, Pointed Sets, Unary Systems, Pointed
    Unary Systems
  \item One Binary Operation -- Magmas (Groupoids), Semigroups, Monoids, ...
    MORE
  \item Two Binary Operations
    \begin{itemize}
      \item Ring Structures (\S\ref{sec:ring_theory})
      \item Lattice Structures (\S\ref{sec:lattice_theory})
      \item Arithmetic Structures (\S\ref{sec:arithmetic}) -- Pointed Unary
        Systems with Injective Successor as Unary Operation with Distinguished
        Element $0$
    \end{itemize}
\end{itemize}

... TODO



\subsubsection{Algebraic Operation}\label{sec:algebraic_operation}

Algebraic Effects (\S\ref{sec:algebraic_effect})



\subsubsection{Free Object}\label{sec:universal_free_object}

\fist Free Object (Category Theory \S\ref{sec:free_object})

(wiki):

A \emph{Free Object} over a Set $A$ can be thought of as being a ``generic''
Algebraic Structure over $A$ such that the only Equations holding between
Elements of the Free Object are those that follow from the defining Axioms of
the Algebraic Structure

direct generalization to Categories of the notion of Basis (\S\ref{sec:basis})
in a Vector Space (FIXME: clarify)

construction of a Free Object for Algebras conforming to the Associative Law
starts with the collection of all possible Words formed from an Alphabet; adding
a Set of \emph{Equivalence Relations} (\S\ref{sec:equivalence_relation}) on the
Words as the defining Relations of the Algebraic Structure at hand, then the
Free Object consists of the Set of Equivalence Classes



\subsubsection{Similarity Type}\label{sec:similarity_type}

The \emph{Similarity Type} of an Algebraic Structure, $\Omega$, is an Ordered
Sequence of Natural Numbers listing the Arity of the Operations of the Algebra.



\subsubsection{Equational Law}\label{sec:equational_law}

\subsubsection{Partial Algebra}\label{sec:partial_algebra}

Semigroupoid (\S\ref{sec:semigroupoid})



\subsubsection{Subalgebra}\label{sec:subalgebra}

A \emph{Subalgebra} of an Algebraic Structure, $\mathfrak{A}$ is a Subset of
$|\mathfrak{A}|$ that is closed under all the Operations of $\mathfrak{A}$.



\subsubsection{Homomorphism}\label{sec:homomorphism}

A \emph{Homomorphism} between two Algebraic Structures $\mathfrak{A}$ and
$\mathfrak{B}$ is a Function $h: \mathfrak{A} \rightarrow \mathfrak{B}$ defined
for $n$-ary Operations:
\[
  \forall f_\mathfrak{A} \in \mathfrak{A}, f_\mathfrak{B} \in
  \mathfrak{B}, h(f_\mathfrak{A}(x_1, ..., x_n)) =
  f_\mathfrak{B}(h(x_1), ..., h(x_n))
\]

%FIXME merge these definitions

A \emph{Homomorphism} is a Structure-preserving Morphism between Algebraic
Structures (\S\ref{sec:universal_algebra}). That is, for an Homomorphism $f : A
\rightarrow B$ where $A$ and $B$ have Operators $*$ and $*'$ respectively, for
any $a_i \in A$
\[
  f(a_1 * a_2) = f(a_1) *' f(a_2)
\]
\fist Note that the Operators do not have to be the same.



\paragraph{Algebra Homomorphism Kernel}\hfill
\label{sec:algebra_homomorphism_kernel} \hfill \\

For Algebraic Structures $\mathfrak{A}$ and $\mathfrak{B}$, and Homomorphism
$f: \mathfrak{A} \rightarrow \mathfrak{B}$, the \emph{Kernel} of $f$ is defined
as:
\[
    ker(f) = \{ (a,a') \in \mathfrak{A} \times \mathfrak{A} : f(a) = f(a') \}
\]
$ker(f)$ is a Congruence Relation on $\mathfrak{A}$ and $f$ is Injective if and
only if $ker(f) = \{(a,a) : a \in \mathfrak{A}\}$.

The Quotient Algebra $\mathfrak{A}/ker(f)$ is Isomorphic to the Image of $f$
(which is a Subalgebra of $\mathfrak{B}$, see First Isomorphism Theorem
(\S\ref{sec:isomorphism_theorem}).



\paragraph{Antihomomorphism}\label{sec:antihomomorphism}\hfill

\paragraph{Symmetry}\label{sec:structure_symmetry}\hfill

A \emph{Symmetry} is a Structure Endomorphism (\S\ref{sec:endomorphism})

Mathematical Symmetry (\S\ref{sec:symmetry})

Set: Bijective Map (Permutation Groups \S\ref{sec:permutation_group})

Metric Space: Isometry (\S\ref{sec:isometry})

Symmetry Group (\S\ref{sec:symmetry_group})



\subparagraph{Discrete Symmetry}\label{sec:discrete_symmetry}\hfill

Discrete Symmetry can always be reinterpreted as a Subset of some
Higher-dimensional Continuous Symmetry (\S\ref{sec:continuous_symmetry}), e.g.
Reflection of a 2-dimensional object in 3-dimensional Space can be achieved by
Continuously Rotating 180 Degrees accross a Non-parallel Plane

\begin{itemize}
  \item Reflection (\S\ref{sec:reflection})
  \item Discrete Rotational Symmetry (\S\ref{sec:rotation})
\end{itemize}



\subparagraph{Continuous Symmetry}\label{sec:continuous_symmetry}\hfill

Continuous Transformation Group
(\S\ref{sec:continuous_transformation_group})

Discrete Symmetry (\S\ref{sec:discrete_symmetry}) can always be reinterpreted as
a Subset of some Higher-dimensional Continuous Symmetry, e.g. Reflection of a
2-dimensional object in 3-dimensional Space can be achieved by Continuously
Rotating 180 Degrees accross a Non-parallel Plane



\subparagraph{Symmetry Breaking}\label{sec:symmetry_breaking}\hfill

\textbf{Spontaneous Symmetry Breaking}



\subparagraph{Goldstone's Theorem}\label{sec:goldstones_theorem}\hfill

generic Continuous Symmetry which is Spontaneously Broken, i.e. its ``Currents''
(FIXME: ???) are Conserved but the Ground State (FIXME: xref) is \emph{not}
Invariant under the Action of the corresponding Charges; necessarily new
Massless (or low-mass if Symmetry is not exact) Scalar Particles appear in the
Spectrum of possibl Excitations

there is one Scalar Particle for each Generator of the Symmetry that is Broken,
i.e. that does not preserve the Ground State

(wiki): Chaos (\S\ref{sec:chaos_theory}) as a Spontaneous Breakdown of
Topological Supersymmetry (FIXME: explain) which is an intrinsic property of
\emph{Evolution Operators} of all Stochastic and Deterministic (Partial)
Differential Equations (\S\ref{sec:sde}); the long-range Dynamical Behavior
associated with \emph{Chaotic Dynamics} is a consequence of Goldstone's Theorem
in the application of Spontaneous Topological Supersymmetry Breaking



\subsubsection{Cancellative Property}\label{sec:cancellative_property}

\subsubsection{Direct Product}\label{sec:direct_product}

The \emph{Direct Product} of a Set of Algebraic Structures is the Cartesian
Product of the Sets with the Operations defined coordinatewise.



\paragraph{Ultrapower}\label{sec:ultrapower}\hfill



\subsubsection{Direct Sum}\label{sec:direct_sum}

Elements of Direct Product having only finitely many Non-zero Terms
%FIXME

Coproduct (\S\ref{sec:coproduct}) for Abelian Groups
(\S\ref{sec:commutative_group}) and Vector Spaces (\S\ref{sec:vector_space})

Cayley-Dixon Construction

Direct Sum of the Complex Numbers with itself gives the Quaternions
(\S\ref{sec:quaternion})



\subsubsection{Archimedean Property}\label{sec:archimedean_property}

Property of having no Infinitely Large or Infinitely Small Elements

cf. Cofinality (\S\ref{sec:cofinality})

\begin{itemize}
  \item Archimedean Field (\S\ref{sec:archimedean_field}) -- e.g. the Real
    Numbers
\end{itemize}



\subsubsection{Filtration}\label{sec:filtration}

cf. Filter (\S\ref{sec:filter})

examples:
\begin{itemize}
  \item Continuous Functions (\S\ref{sec:continuous_function})
    $f : \reals^n \rightarrow \reals$:
    \[
      \cdots \subset C^n(\reals^k) \subset \cdots \subset C^1(R^k) \subset
        C^0(\reals^k)
    \]
\end{itemize}



% ------------------------------------------------------------------------------
\subsection{Essentially Algebraic Structure}
\label{sec:essentially_algebraic}
% ------------------------------------------------------------------------------

Partially Defined Operators satisfying Equational Laws

\url{https://golem.ph.utexas.edu/category/2017/04/gluing_together_finite_shapes.html}:

an Essentially Algebraic Theory $\class{T}$ is a \emph{Small Finitely Complete
  Category} (\S\ref{sec:finitely_complete})



% ------------------------------------------------------------------------------
\subsection{Equational Reasoning}\label{sec:equational_reasoning}
% ------------------------------------------------------------------------------

Equational Logic (\S\ref{sec:equational_logic})



% ------------------------------------------------------------------------------
\subsection{Algebraic Theory}\label{sec:algebraic_theory}
% ------------------------------------------------------------------------------

An \emph{Algebraic Theory} is a Theory (\S\ref{sec:formal_theory}) using Axioms
consisting of Equations between Terms with Free Variables (no Inequalities or
Quantifiers).

(wiki): an Algebraic Theory consists of a collection of $n$-ary Function Terms
with additional Axioms; e.g. a Group Theory is an Algebraic Theory consisting of
three Function Terms (Binary $*$, Nullary $1$, and Unary $^{-1}$) together with
Axioms of Associativity, Neutrality, and Inversion, respectively; cf. a
\emph{Geometric Theory} which involves \emph{Partial Functions} (or Binary
Relations or Existential Quantifiers), e.g. Euclidean Geometry where existence
of Points or Lines are \emph{postulated} (FIXME: clarify)

Stronger condition than Elementary Theory (\S\ref{sec:elementary_theory})

Sentential (Propositional) Logic \S\ref{sec:propositional_logic})

Algebraic Structure (\S\ref{sec:algebraic_structure})

Equational Logic (\S\ref{sec:equational_logic}),
Equational Theory (\S\ref{sec:equational_theory})

Presentation: Function Symbols, Axioms of Universally Quantified Equations

1963 - Lawvere - \emph{Functorial Semantics of Algebraic Theories} ------------
Categorical Semantics (\S\ref{sec:categorical_semantics}): Lawvere Theories
(\S\ref{sec:lawvere_theory})

1965 - MacLane - \emph{Categorical Algebra} -- PROPs (PROducts and Permutations
Categories \S\ref{sec:prop_category})

Monad (\S\ref{sec:monad})

Models

Free Model

Category of Models (\S\ref{sec:category_of_models})

Axioms can be presented as $\top \vdash t = s$


\url{https://golem.ph.utexas.edu/category/2017/03/algebra_valued_functors_in_gen.html}



\subsubsection{Lawvere Theory}\label{sec:lawvere_theory}

\fist Equational Logic (\S\ref{sec:equational_logic}): Quantifier-free Terms of
First-order Logic with Equality as the only Predicate Symbol

A \emph{Lawvere Theory} (or \emph{Finite-product Theory}) is (equivalently
encoded by its Syntactic Category \S\ref{sec:syntactic_category}) as a Small
Category (\S\ref{sec:small_category}) $\cat{L}$ with (Strictly Associative)
Finite Products and a Strict Identity-on-objects Functor $I : \aleph_0^{op}
\rightarrow \cat{L}$ Preserving Finite Products, where $\aleph_0$ is a Skeleton
of $\cat{FinSet}$, i.e. it is a Category with Finite Cartesian Products and a
distinguished Object $X$ such that every Object is a Power $X^n$, that is a
Category that can serve as a Theory of a Mathematical Structure $X$ equipped
with $n$-ary Operations obeying Equational Laws.

\fist (MacLane 1965) introduced \emph{PROPs} (\emph{PROducts and Permutations
  Categories} \S\ref{sec:prop_category}) to extend Functorial Semantics to
Structures (e.g. Vector Spaces) with Operations of the form $f : X^{\otimes m}
\rightarrow X^{\otimes n}$, i.e. every Object is a Tensor Power $X^{\otimes n}$

Category-theoretic formulation of Universal Algebra
(\S\ref{sec:universal_algebra}) with primitive notion of Operation
(\S\ref{sec:algebraic_operation}): unlike Universal Algebra, the notion of
Lawver Theory is Presentation (\S\ref{sec:presentation}) Independent, i.e. the
Category of Models determines the Theory uniquely up to Coherent
(\S\ref{sec:coherence_condition}) Isomorphism. \cite{hyland-power06}

A \emph{Model} $M$ of a Lawvere Theory is Product-preserving Functor out of the
Category (into a Category $\cat{C}$ with Finite Products):
\[
  M : \cat{L} \rightarrow \cat{C}
\]

\emph{Morphism of Models} $h : M \rightarrow N$ is a Natural Transformation
(\S\ref{sec:natural_transformation})

Equivalently encoded by a Syntactic Category (\S\ref{sec:syntactic_category}):
Category with Finite Products such that all Objects are Finite Products of a
given Object

Every Object is Isomorphic to a Finite Cartesian Power $X^n = X \times \ldots
\times X$ of a \emph{Generic Object} $X \in \cat{T}_0$

Hom-set $\cat{L}(n,1) := \cat{L}(x^n,x)$ is the Set of $n$-ary Operations
Definable in the Theory.

Dual to the Category of Finitely Generated Free Algebras of the Theory

$\cat{Law}$ Category of Lawvere Theories

$\cat{S}$ Syntactic Category on Objects $n \in \nats$ with Morphisms Generated
by Projections $\pi_i : n \rightarrow 1$

$\cat{S} \simeq \cat{FinSet}^{op}$ Initial Object in $\cat{Law}$

Algebra over $\cat{S}$ is $\cat{Set}$

Generators: Operations

Relations: Axioms

Finitary Monads (\S\ref{sec:finitary_monad}) on $\cat{Set}$

every Lawvere Theory corresponds to a Finitary Monad on $\cat{Set}$ and
vice-versa (Hyland-Power07)

Operad Algebra (\S\ref{sec:operad_algebra})

Generic Algebra

Category of Countable Lawvere Theories (???): has Sums, Tensors, and
Distributive Tensors, Modelling Combinations of Effects; Closed under
taking Images \cite{hyland-power06}

Meredith-Stay17 - \emph{Name-free Combinators for Concurrency} ------------
Combinator Calculus with no Bound Names into which Asynchronous
$\pi$-calculus has a Faithful Embedding with Multisorted Lawvere
Theories Enriched over Graphs as the Operational Semantics

\asterism

2016 - Stay, Meredith - \emph{Logic as a Distributive Law}

Lawvere63: an Equational Theory is a \emph{Presentation}
(\S\ref{sec:presentation}) of a Category with Finite Products
where every Object is a Power of a single Generating Object, that is, a
\emph{Lawvere Theory}

for example, a Presentation of the Syntactic Category (Lawvere Theory) $Th(Mon)$
of Monoids on a Sort $M$ with Nullary Term Constructor $e$ and Binary Term
Constructor $*$ has Objects $1, M, M^2, M^3, \ldots$ and has Morphisms generated
by ``Projections'' (FIXME: clarify) $e$ and $*$ modulo the Equations defining a
Monoid (Associativity and Left and Right Identity)

to \emph{Interpret} (\S\ref{sec:interpretation}) the Equational Theory as
denoting Sets and Functions, use a Product-preserving Functor from the Lawvere
Theory to $\cat{Set}$, mapping $M$ to a Set and $e$ and $*$ to Functions
Satisfying the Equations

if the Category $\cat{Vect}$ of Vector Spaces and Linear Maps is used to encode
the Semantics instead of $\cat{Set}$, and the Product in $Th(Mon)$ is mapped to
the Tensor Product in $\cat{Vect}$, then a Model of the Theory would be an
Associative Algebra (\S\ref{sec:associative_algebra}) instead

if the Category $\cat{End}(\cat{C})$ of Endofunctors and Natural Transformations
of $\cat{C}$ is used, then a Model of the Theory would be a Monad
(\S\ref{sec:monad})

the Category of Product-preserving Functors from $Th(Mon)$ to $\cat{Set}$ and
Natural Transformations between them is equivalent to the Category $\cat{Mon}$
of Monoids and Monoid Homomorphisms

every Lawvere Theory corresponds to a Finitary Monad on $\cat{Set}$ and
vice-versa (Hyland-Power07)

cf. ``modules'' or ``interfaces'' used to model Data Structures in computing:
\begin{itemize}
\item $\lambda$-calculus (\S\ref{sec:untyped_lambda}) is a paradigmatic model of
  Functional Programming with a single Data Type, the Lambda Term, described by
  a one-line Grammar, where Computation consists of repeatedly applying a single
  Rewrite Rule called $\beta$-reduction-- the Rewrite Rule matches the current
  Term to a pattern and rearranges parts of the Term and when the Term no longer
  matches the pattern the Rewriting stops and the resulting Term is taken as the
  ``answer''
\item $\pi$-calculus (\S\ref{sec:pi_calculus}) as a paradigmatic model of
  Concurrent Programming with a structured Data Type called a ``Process''
  described by a simple Grammar and Computations carried out by repeated
  application of Rewrite Rules, primarily the $\mathsf{comm}$-rule
\end{itemize}

cf. Milner92 \cite{milner92} for the now ``standard'' presentation of a
Computational Calculus (\S\ref{sec:computation_model}):
\begin{itemize}
  \item Grammar (\S\ref{sec:formal_grammar}) describing the primary Data Type
    over which \emph{Computations} are carried out
  \item Structural Equivalence (\S\ref{sec:structural_equality}) used to erase
    Syntactic differences that are ``irrelevant'' to Computation
  \item Set of Rewrite Rules (\S\ref{sec:abstract_rewrite}) describing how to
    ``realize'' Computation through Operations on Data Structures
\end{itemize}

capturing ``higher-order computational phenomena'', expressed as ``generalized
Rewrites'' motivates a movement to Higher Categorical Semantics

Lawvere Theories can be generalized from $1$-categories to $2$-categories where
the corresponding notion of Equational Theory involves one more ``level of
structure'':
\begin{itemize}
  \item Sorts
  \item Term Constructors
  \item Rewrites (instead of Equations beween Term Constructors)
  \item Equations between Rewrites
\end{itemize}

such a Theory may be Interpreted in $\cat{Cat}$, the $2$-category of Categories,
Functors, and Natural Transformations



\paragraph{Lawvere Algebra}\label{sec:lawvere_algebra}\hfill

Algebra over a Lawvere Theory

Model for Syntactic Category (\S\ref{sec:syntactic_category})

Category of Lawvere Algebras is the Full Subcategory of the Functor
Category on the Product-preserving Functors %FIXME

Canonically the Category of Lawvere Algebras is a Closed Symmetric
Monoidal Category (\S\ref{sec:closed_symmetric_monoidal})



\paragraph{Enriched Lawvere Theory}\label{sec:enriched_lawvere}\hfill
\cite{hyland-power06}

or \emph{Lawvere $V$-theory}



\paragraph{Nominal Lawvere Theory}\label{sec:nominal_lawvere_theory}\hfill

Clouston2014 - \emph{Nominal Lawvere Theories: A Category theoretic account of
  equational theories with names}

Name Binding (\S\ref{sec:name_binding})



\subsubsection{Commutative Algebraic Theory}
\label{sec:commutative_algebraic_theory}

an Algebraic Theory is \emph{Commutative} if its Operations are Algebra
Homomorphisms under any Interpretation

generalization of Commutative Monoids

generalized to Monoidal Monads (\S\ref{sec:monoidal_monad})

\fist cf. Commutative Algebra (\S\ref{sec:commutative_algebra})



% ------------------------------------------------------------------------------
\subsection{Variety}\label{sec:variety}
% ------------------------------------------------------------------------------

A \emph{Variety} is a Class of Algebraic Structures defined only by
Axioms that are Identities in a given Signature. This is equivalent to
saying a Variety is the Class of Algebraic Structures with the same
Signature that are Closed under Homomorphic (\S\ref{sec:homomorphism})
Images, Subalgebras (\S\ref{sec:subalgebra}), and Direct Products
(\S\ref{sec:direct_product}); a result known as Birkhoff's Theorem
(\S\ref{sec:birkhoffs_theorem}). This rules out Logical Connectives,
Existential Quantification, and all Relations for the Signature
besides Equality (thus excluding the Class of Fields
\S\ref{sec:field}) and Identities being implicitly Universally
Quantified over the Domain.

An example of a Variety with Algebraic Signature $\Omega = (2)$ is the
Class of all Semigroups with an equation defining the Associative Law:
\[
    x(yz) = (xy)z
\]

Algebraic Structures in a Variety are Quotient Algebras
(\S\ref{sec:quotient_algebra}) generated by the Identities on the Term
Algebra (\S\ref{sec:term_algebra}) generated from the Signature and
Underlying Set.

Every Variety gives rise to a Monad (\S\ref{sec:monad}) and the Type
of the Algebra can be recovered from the Monad.

\begin{itemize}
  \item Interior Algebras (\S\ref{sec:interior_algebra}) form a Varity of Modal
    Algebras (\S\ref{sec:modal_algebra})
  \item the Variety of all Modal Algebras is the equivalent Algebraic Semantics
    (\S\ref{sec:algebraic_semantics}) of the Modal $K$ Logic
    (\S\ref{sec:modal_logic}) in the sense of Astract Algebraic Logic
    (\S\ref{sec:abstract_algebraic_logic})
\end{itemize}



\subsubsection{Subvariety}\label{sec:subvariety_theorem}

A \emph{Subvariety} is a Subclass of a Variety with the same
Signature. For example, the Class of Abelian Groups is a Subvariety of
the Class of Groups.



\subsubsection{Pseudovariety}\label{sec:pseudovariety}

A Variety of Finite Algebraic Structures is called a
\emph{Pseudovariety}.



\subsubsection{Quasivariety}\label{sec:quasivariety}



\subsubsection{Birkhoff's Theorem}\label{sec:birkhoffs_theorem}
\cite{birkhoff35}

\emph{Birkhoff's Theorem} (or \emph{HSP Theorem})



% ------------------------------------------------------------------------------
\subsection{Universal Coalgebra}\label{sec:universal_coalgebra}
% ------------------------------------------------------------------------------

\cite{rutten00}

Systems

Bisimulation Relation (\S\ref{sec:bisimulation})



% ------------------------------------------------------------------------------
\subsection{Covariety}\label{sec:covariety}
% ------------------------------------------------------------------------------

% ------------------------------------------------------------------------------
\section{Wheel Theory}\label{sec:wheel_theory}
% ------------------------------------------------------------------------------

%FIXME



% ==============================================================================
\section{Magma}\label{sec:magma}
% ==============================================================================

A \emph{Magma}, $M$, is an Algebraic Structure
(\S\ref{sec:universal_algebra}) with a single Closed Binary Operation,
$M \times M \rightarrow M$.

\fist Sometimes called ``Groupoids'', but not to be confused with Groupoids
(\S\ref{sec:groupoid}) in Category Theory
\\ \\
No special properties:

$\begin{array}{c||c|c|}
  * & \mathbf{a} & \mathbf{b} \\ \hline \hline
  \mathbf{a} & a & b \\ \hline
  \mathbf{b} & a & a \\ \hline
\end{array}$ $\quad$ $\begin{array}{c||c|c|}
  * & \mathbf{a} & \mathbf{b} \\ \hline \hline
  \mathbf{a} & a & a \\ \hline
  \mathbf{b} & b & a \\ \hline
\end{array}$ \\ \hfill \\

$\begin{array}{c||c|c|}
  * & \mathbf{a} & \mathbf{b} \\ \hline \hline
  \mathbf{a} & b & b \\ \hline
  \mathbf{b} & a & b \\ \hline
\end{array}$ $\quad$ $\begin{array}{c||c|c|}
  * & \mathbf{a} & \mathbf{b} \\ \hline \hline
  \mathbf{a} & b & a \\ \hline
  \mathbf{b} & b & b \\ \hline
\end{array}$ \\ \hfill \\

$\begin{array}{c||c|c|}
  * & \mathbf{a} & \mathbf{b} \\ \hline \hline
  \mathbf{a} & b & b \\ \hline
  \mathbf{b} & a & a \\ \hline
\end{array}$ $\quad$ $\begin{array}{c||c|c|}
  * & \mathbf{a} & \mathbf{b} \\ \hline \hline
  \mathbf{a} & b & a \\ \hline
  \mathbf{b} & b & a \\ \hline
\end{array}$
\\ \\
Commutative:

$\begin{array}{c||c|c|}
  * & \mathbf{a} & \mathbf{b} \\ \hline \hline
  \mathbf{a} & b & a \\ \hline
  \mathbf{b} & a & a \\ \hline
\end{array}$ $\quad$ $\begin{array}{c||c|c|}
  * & \mathbf{a} & \mathbf{b} \\ \hline \hline
  \mathbf{a} & b & b \\ \hline
  \mathbf{b} & b & a \\ \hline
\end{array}$
\\ \\
Band (Associative, Idempotent):

$\begin{array}{c||c|c|}
  * & \mathbf{a} & \mathbf{b} \\ \hline \hline
  \mathbf{a} & a & a \\ \hline
  \mathbf{b} & b & b \\ \hline
\end{array}$ $\quad$ $\begin{array}{c||c|c|}
  * & \mathbf{a} & \mathbf{b} \\ \hline \hline
  \mathbf{a} & a & b \\ \hline
  \mathbf{b} & a & b \\ \hline
\end{array}$
\\ \\
Semilattice (Associative, Commutative, Idempotent):

$\begin{array}{c||c|c|}
  * & \mathbf{a} & \mathbf{b} \\ \hline \hline
  \mathbf{a} & a & a \\ \hline
  \mathbf{b} & a & a \\ \hline
\end{array}$ $\quad$ $\begin{array}{c||c|c|}
  * & \mathbf{a} & \mathbf{b} \\ \hline \hline
  \mathbf{a} & b & b \\ \hline
  \mathbf{b} & b & b \\ \hline
\end{array}$ \\ \hfill \\

$\begin{array}{c||c|c|}
  * & \mathbf{a} & \mathbf{b} \\ \hline \hline
  \mathbf{a} & a & b \\ \hline
  \mathbf{b} & b & b \\ \hline
\end{array}$ $\quad$ $\begin{array}{c||c|c|}
  * & \mathbf{a} & \mathbf{b} \\ \hline \hline
  \mathbf{a} & a & a \\ \hline
  \mathbf{b} & a & b \\ \hline
\end{array}$
\\ \\
Abelian Group (Associative, Identity, Invertible, Commutative):

$\begin{array}{c||c|c|}
  * & \mathbf{a} & \mathbf{b} \\ \hline \hline
  \mathbf{a} & a & b \\ \hline
  \mathbf{b} & b & a \\ \hline
\end{array}$ $\quad$ $\begin{array}{c||c|c|}
  * & \mathbf{a} & \mathbf{b} \\ \hline \hline
  \mathbf{a} & b & a \\ \hline
  \mathbf{b} & a & b \\ \hline
\end{array}$



% ------------------------------------------------------------------------------
\subsection{Commutative Non-associative Magma}
\label{sec:commutative_magma}
% ------------------------------------------------------------------------------

% ------------------------------------------------------------------------------
\subsection{Free Magma}\label{sec:free_magma}
% ------------------------------------------------------------------------------

For a Set $X$, the \emph{Free Magma}, $M_X$, is the Set of
Non-associative Words (Strings) on $X$ with parentheses retained.

Free Object (\S\ref{sec:free_object})



% ==============================================================================
\section{Semigroup}\label{sec:semigroup}
% ==============================================================================

A \emph{Semigroup} is Magma (\S\ref{sec:magma}) with an Associative Binary
Operation. A Semigroup is differentiated from a Monoid (\S\ref{sec:monoid}) by
not requiring an Identity Element, and from a Group (\S\ref{sec:group}) by not
requiring Inverses.



% ------------------------------------------------------------------------------
\subsection{Subsemigroup}\label{sec:subsemigroup}
% ------------------------------------------------------------------------------

% ------------------------------------------------------------------------------
\subsection{Semigroup Action}\label{sec:semigroup_action}
% ------------------------------------------------------------------------------

A \emph{Semigroup Action} is a rule associating each Element of a
Semigroup, $S$, with a \emph{Transformation} $f : X \rightarrow X$ on
a Set $X$, such that a Product of two Semigroup Elements is associated
with the Composite of the two corresponding Transformations. A
\emph{Left Semigroup Action}, $\bullet$, for a Semigroup $(S,*)$ and
Set $X$ is defined as:
\[
  \forall s,t \in S\;\forall x \in X, s \bullet (t \bullet x) = (s * t)
  \bullet x
\]
and also known as an \emph{$S$-act}. Such a Semigroup Action is
equivalent to a Semigroup Homomorphism on the Set of Functions on $X$.
A \emph{Right Semigroup Action} is defined as:
\[
  \forall s,t \in S\;\forall x \in X, (x \bullet s) \bullet t = x
  \bullet (s * t)
\]
A \emph{Faithful Semigroup Action} (or \emph{Effective Semigroup
  Action}) has the Property:
\[
  \forall s, t \in S, s \bullet x \neq t \bullet x
\]
which is Isomorphic to a Transformation Semigroup
(\S\ref{sec:transformation_semigroup}).



\subsubsection{$S$-homomorphism}\label{sec:s_homomorphism}

For two $S$-acts, $T$ and $T'$, an \emph{$S$-homomorphism} $F : T
\rightarrow T'$ is defined such that:
\[
  \forall s \in S, x \in X, F(sx) = sF(x)
\]
The Set of all $S$-homomorphisms between a pair of $S$-acts is denoted
$Hom_S(T,T')$.

For a fixed Semigroup $S$, one may define Categories
$S\text{-}\mathbf{Act}$ and $\mathbf{Act}\text{-}S$ with Left and
Right $S$-acts as Objects, respectively, with $S$-homomorphisms as
Morphisms.



% ------------------------------------------------------------------------------
\subsection{Commutative Semigroup}\label{sec:commutative_semigroup}
% ------------------------------------------------------------------------------

examples:
\begin{itemize}
  \item the Real Closed Unit Interval $[-1,1]$ is a Semigroup under
    Multiplication
\end{itemize}



% ------------------------------------------------------------------------------
\subsection{Regular Semigroup}\label{sec:regular_semigroup}
% ------------------------------------------------------------------------------

A \emph{Regular Semigroup} $(S,*)$ has the Property:
\[
  \forall a \in S, \exists x \in S : axa = a
\]
where $x$ is called a \emph{Pseudoinverse}. Every Group
(\S\ref{sec:group}) is a Regular Semigroup.



\subsubsection{Full Linear Semigroup}\label{sec:full_linear_semigroup}

Linear Operators (\S\ref{sec:linear_operator})

adding the requirement that Operators be Invertible (i.e. having Non-zero
Determinant) gives the General Linear Group (\S\ref{sec:general_linear_group})
$GL(V)$



% ------------------------------------------------------------------------------
\subsection{Transformation Semigroup}\label{sec:transformation_semigroup}
% ------------------------------------------------------------------------------

A \emph{Transformation Semigroup} is a pair $(X,S)$ where $X$ is a Set
and $S$ is a Semigroup of \emph{Transformations} of $X$, that is, a
Set of Functions $f : X \rightarrow X$ that is Closed under
Composition. This is the Semigroup analogue of a Permutation Group
(\S\ref{sec:permutation_group}). Any Semigroup can be realized as a
Transformation Semigroup of some Set. A Transformation Semigroup that
includes the Identity Function is a Transformation Monoid
(\S\ref{sec:transformation_monoid}).

A Transformation Semigroup can be made into a Semigroup Action
(\S\ref{sec:semigroup_action}) $\bullet$ of $S$ by Evaluation:
\[
  \forall s \in S, x \in X, s \bullet x = s(x)
\]



% ------------------------------------------------------------------------------
\subsection{Involution Semigroup}\label{sec:involution_semigroup}
% ------------------------------------------------------------------------------

Involutive (\S\ref{sec:involution}) Anti-automorphism



% ==============================================================================
\section{Monoid}\label{sec:monoid}
% ==============================================================================

A \emph{Monoid} is a Semigroup with an Identity Element.

The set of all Endomorphisms of an Object, $X$, in a Category, $C$,
\[
    Hom(X,X)
\]
defines a Monoid and is denoted $End_C(X)$.

Monoid Object (\S\ref{sec:monoid_object}) in $\cat{Set}$ with the
Cartesian Product $\times$

a Category (\S\ref{sec:category}) can be seen as a Monoid with \emph{Type
  Parameters}, i.e. a Monoid is a Category with only one Object

\fist cf. \emph{Monoidal Categories} (\S\ref{sec:monoidal_category}),
Monoidal Adjunctions (\S\ref{sec:monoidal_adjunction})



% ------------------------------------------------------------------------------
\subsection{Submonoid}\label{sec:submonoid}
% ------------------------------------------------------------------------------

% ------------------------------------------------------------------------------
\subsection{Monoid Congruence}\label{sec:monoid_congruence}
% ------------------------------------------------------------------------------

A \emph{Monoid Congruence} on a Monoid $(M,*)$ is an Equivalence
Relation, $\sim$, that respects the Monoid Operator:
\[
  a,b,c,d \in M, a \sim c \wedge b \sim d \Rightarrow a*b \sim c*d
\]

See also Syntactic Congruence \S\ref{sec:syntactic_congruence}.



\subsubsection{Quotient Monoid}\label{sec:quotient_monoid}

For a Monoid Congruence, $\sim$, over a Monoid, $(M,*)$, a
\emph{Quotient Monoid}, $M/\sim$, can be defined with Equivalence
Classes as of $\sim$ as Objects and Binary Operation $\bullet$ defined
as:
\[
  a,b \in M, [a]\bullet[b] = [a*b]
\]



% ------------------------------------------------------------------------------
\subsection{Recognizable Subset}\label{sec:recognizable}
% ------------------------------------------------------------------------------

A Subset of a Monoid $S \subseteq N$ is \emph{Recognized} by a Monoid $M$ if
there exists a Morphism $\phi : N \rightarrow M$ such that $S =
\phi^{-1}(\phi(S))$, and $S$ is \emph{Recognizable} if it is Recognized by a
Finite Monoid. This requires that there is a Subset $T \subseteq M$ such that
$\phi(S) \subseteq T$ and $\phi(N \setminus S) \subseteq (M \setminus T)$.

For the Free Monoid (\S\ref{sec:free_monoid}) $A^*$ over an Alphabet
$A$, the Recognizable Subsets of $A^*$ are the Regular Languages
(\S\ref{sec:regular_language}).



% ------------------------------------------------------------------------------
\subsection{Monoid Action}\label{sec:monoid_action}
% ------------------------------------------------------------------------------

A \emph{Monoid Action} is a special case of a Semigroup Action
(\S\ref{sec:semigroup_action}) where the Identity Element of the
Monoid is associated with the Identity Transformation of a Set. If the
Monoid is taken to be a Category with one Object, $\cat{M}$, the
Monoid Action, $f$, is a Functor from that Category to the Category of
Sets, $\cat{Set}$:
\[
  f : \cat{M} \rightarrow \cat{Set}
\]

Binary Function (Sets) $A \times B \rightarrow B$ is equivalent to an
Action on the Free Monoid $A^*$ on $B$. This is analogous to the way a
Category of $F$-algebras (\S\ref{sec:f_algebra}) on an Endofunctor $F
: \cat{C} \rightarrow \cat{C}$ is equivalent to the Category of
$T$-algebras (\S\ref{sec:t_algebra}) on the Free Monad
(\S\ref{sec:free_monad}) of $F$.



\subsubsection{Operator Monoid}\label{sec:operator_monoid}

An \emph{Operator Monoid} is a Monoid $M$ with an Action on a Set.



% ------------------------------------------------------------------------------
\subsection{Commutative Monoid}\label{sec:commutative_monoid}
% ------------------------------------------------------------------------------

(or \emph{Abelian Monoid})



% ------------------------------------------------------------------------------
\subsection{Transformation Monoid}\label{sec:transformation_monoid}
% ------------------------------------------------------------------------------

A \emph{Transformation Monoid} is a Transformation Semigroup
(\S\ref{sec:transformation_semigroup}) that includes the Identity
Function.

A Transformation Monoid with Invertible Elements is a Permutation
Group (\S\ref{sec:permutation_group}).

Any Monoid $M$ is an Effective Transformation Monoid on its underlying
Set.



\subsubsection{Full Transformation Monoid}\label{sec:full_transformation}

The Set of all Transformations of a Set $X$ gives a \emph{Full
  Transformation Monoid}, also called the \emph{Symmetric Semigroup}
of $X$, denoted $T_X$. An arbitrary Transformation Monoid is a
Submonoid (\S\ref{sec:submonoid}) of the Full Transformation Monoid. A
Full Transformation Monoid is a Regular Semigroup
(\S\ref{sec:regular_semigroup}).



\subsubsection{Transition Monoid}\label{sec:transition_monoid}

Given a Semiautomaton (\S\ref{sec:semiautomaton}), $(\Sigma, X, T)$,
with Input Alphabet $\Sigma$, Set of States $X$, and Transition
Function $T : \Sigma \times X \rightarrow X$, the \emph{Transition
  Monoid} (also \emph{Characteristic Monoid}, \emph{Input Monoid}, or
\emph{Transition System}) is the Transformation Monoid consisting of
the Set of Transformations of $X$, $\{T_a : a \in \Sigma\}$ where
$T_a(x) = T(a,x)$ for $x \in X$, closed under Composition.

The Syntactic Monoid (\S\ref{sec:syntactic_monoid}) of a Formal
Language is Isomorphic to the Transition Monoid of the minimal
Automaton Accepting the Language.



% ------------------------------------------------------------------------------
\subsection{Free Monoid}\label{sec:free_monoid}
% ------------------------------------------------------------------------------

A \emph{Free Monoid} on a Set $A$ is the Monoid $A^*$ whose Elements
are all possible Finite Sequences of zero or more Elements of $A$ with
String Concatenation (\S\ref{sec:string_concatenation}) as the Monoid
Operation and the Empty String $\varepsilon$ as the Identity Element.

Free Monoids are Unique up to Isomorphism.

Underlying Set $U(M)$

Free Monoid $F(X)$

$i_X : X \rightarrow U F (X)$

$Hom_\mathbf{Mon}(F(X), M) \cong Hom_\cat{Set}(X, U(M))$

In Programming Languages:

\begin{itemize}
  \item $\mathtt{[]}$ (Haskell)
\end{itemize}



% ------------------------------------------------------------------------------
\subsection{Syntactic Monoid}\label{sec:syntactic_monoid}
% ------------------------------------------------------------------------------

Given a Subset of a Monoid, $S \subset M$, a \emph{Syntactic Monoid},
$M(S)$, is the Quotient Monoid (\S\ref{sec:quotient_monoid}) formed by
the Syntactic Congruence Relation (\S\ref{sec:syntactic_congruence})
$\equiv_S$ where the Objects are Equivalence Classes in $M / \equiv_S$
and the Operation $\bullet$ is defined as:
\[
  [a] \bullet [b] = [ab]
\]

The Syntactic Monoid, $M(L)$, of a Formal Language
(\S\ref{sec:formal_language}), $L$, is the smallest Monoid that
Recognizes (\S\ref{sec:recognizable}) $L$.

The Syntactic Monoid of a Formal Language is Isomorphic to the
Transformation Monoid (\S\ref{sec:transformation_monoid}) of the
Minimal Automaton (\S\ref{sec:minimum_dfa}) accepting the Language.

\fist See also Trace Monoids (\S\ref{sec:trace_monoid}) in Trace
Theory (\S\ref{sec:trace_theory})

The Syntactic Monoid of a Formal Language is Isomorphic to the
Transition Monoid (\S\ref{sec:transition_monoid}) of the minimal
Automaton Accepting the Language.



\subsubsection{Syntactic Quotient}\label{sec:syntactic_quotient}

\emph{Syntactic Quotient}

\fist Cf. String Quotient \ref{sec:string_quotient}

The \emph{Right Quotient} of a Subset of a Monoid $S \subset M$, by an
Element $a \in M$, is defined as:
\[
  S / a = \{ s \in M\;|\;sa \in S \}
\]
and the \emph{Left Quotient} as:
\[
  a \backslash S = \{ s \in M\;|\;as \in S \}
\]

A Formal Language $L$ is Regular (\S\ref{sec:regular_language}) if and
only if the Family of Quotients $\{ m \ L \;|\; m \in M \}$ is Finite.



\subsubsection{Syntactic Relation}\label{sec:syntactic_relation}

A Syntactic Quotient (\S\ref{sec:syntactic_quotient}) of a Subset of a
Monoid $S \subset M$ induces an Equivalence Relation called a
\emph{Syntactic Relation} (or \emph{Syntactic Equivalence}). The
\emph{Right Syntactic Equivalence Relation} is defined as:
\[
  \sim_S = \{ (a,b) \in M \times M \;|\; S/a = S/b\}
\]
and the \emph{Left Syntactic Equivalence Relation} as:
\[
  \prescript{}{S}{\sim} = \{ (a,b) \in M \times M \;|\;
  a \backslash S = b \backslash S \}
\]



\subsubsection{Syntactic Congruence}\label{sec:syntactic_congruence}

\emph{Total Syntactic Congruence} (or \emph{Myhill Congruence}) with
respect to a Subset of a Monoid $S \subset M$, denoted $\equiv_S$, is
defined for $u,v \in M$ as:
\[
  u \equiv_S v \Leftrightarrow
  (\forall x,y \in M, xuy \in S \Leftrightarrow xvy \in S)
\]



\paragraph{Disjunctive Set}\label{sec:disjunctive_set}\hfill \\\hfill

A \emph{Disjunctive Set} is a Subset of a Monoid $S \subset M$ such
that the Syntactic Congruence defined by $S$ is the Equality Relation.



% ------------------------------------------------------------------------------
\subsection{Graphic Monoid}\label{sec:graphic_monoid}
% ------------------------------------------------------------------------------

a Monoid in which the \emph{Graphic Identity} $xyx = xy$ holds for all $x$ and
$y$

every Element $x$ of a Graphic Monoid is \emph{Idempotent}: $x^2 = x$

a Commutative Monoid obeying $x^2 = x$ automatically obeys the Graphic Identity
since $xyx = x^2y = xy$

for a Noncommutative Monoid, the Graphic Identity is \emph{stronger} than $x^2
= x$

% FIXME move this to category theory?

Graphic Category (\S\ref{sec:graphic_category})

Hyperplane (\S\ref{sec:hyperplane}) arrangements

\url{https://johncarlosbaez.wordpress.com/2018/02/19/complex-adaptive-systems-part-7/}

Network Operads (\S\ref{sec:network_operad}); Commitment Network Models:

in any Graphic Monoid, the relation $x \leq y$ can be defined by:
\[
  x \leq y \Leftrightarrow x a = y
\]
for some $a$, making the Graphic Monoid into a Poset

in the context of Commitment Networks this means that starting from $x$, one
can reach $y$ by making some further commitment $a$



\subsubsection{Hegelian Taco}\label{sec:hegelian_taco}



% ------------------------------------------------------------------------------
\subsection{Comonoid}\label{sec:comonoid}
% ------------------------------------------------------------------------------

Substructural Logic (\S\ref{sec:substructural_logic}), Linear Type
Systems (\S\ref{sec:linear_type})

Exponential Modality $!$ has Structure of a Comonoid: Weakening $!A
\rightarrow ()$, Contraction $!A \rightarrow !A \times !A$



% ==============================================================================
\section{Quasigroup}\label{sec:quasigroup}
% ==============================================================================

A \emph{Quasigroup}, $Q$, is a Magma with a Binary Operation that satisfies
the \emph{Latin Square Property}:
\[
  \forall a, b \in Q,\;\exists ! x,y \in Q : a * x = b \wedge y * a = b
\]
which allows for the Unique Equations defining \emph{Left} and
\emph{Right Division}, $x = a \backslash b$ and $y = b / a$
respectively.



% ------------------------------------------------------------------------------
\subsection{Pique}\label{sec:pique}
% ------------------------------------------------------------------------------

A \emph{Pique} (\emph{Pointed Idempotent Quasigroup}) is a Quasigroup
with an Idempotent Element. Taking an Abelian Group and its
Subtraction Operation as Quasigorup Multiplication gives a Pique,
$(A,-)$, where the Group's Identity Element is the Pointed Idempotent.



% ------------------------------------------------------------------------------
\subsection{Loop}\label{sec:quasigroup_loop}
% ------------------------------------------------------------------------------

A \emph{Loop} is a Quasigroup (\S\ref{sec:quasigroup}) with an
Identity Element. A Loop with the Associative Property is a Group
(\S\ref{sec:group}).



% ==============================================================================
\section{Group Theory}\label{sec:group_theory}
% ==============================================================================

Macauley - Clemson University Math 4120: \emph{Visual Group Theory} - Youtube
playlist:
\url{https://www.youtube.com/watch?v=UwTQdOop-nU&list=PLwV-9DG53NDxU337smpTwm6sef4x-SCLv}

\fist See also Groupoids (\S\ref{sec:groupoid})

\fist Lie Groups (\S\ref{sec:lie_group}): Manifolds with Group Structure

(wiki)

Topological \& Lie Groups:
\begin{itemize}
  \item Circle
  \item Solenoid
  \item General Linear Group
  \item Special Linear Group
  \item Orthogonal Group
  \item Euclidean Group
  \item Special Orthogonal Group $SO(3)$
  \item Special Euclidean Group $SE(3)$ -- group of \emph{Spatial Rigid
    Transformations}
  \item Unitary Group
  \item Special Unitary Group
  \item Symplectic Group
  \item $G_2$, $F_4$, $E_6$, $E_7$, $E_8$
  \item Lorentz
  \item Poincaire
  \item Conformal
  \item Diffeomorphism
  \item Loop
  \item Infinite Dimensinal Lie Group
\end{itemize}

%FIXME xrefs, split topological and lie groups



% ------------------------------------------------------------------------------
\subsection{Group}\label{sec:group}
% ------------------------------------------------------------------------------

A \emph{Group} is a Monoid (\S\ref{sec:monoid}) with an Inverse for every
Element. That is, a Group is a Set $G$, and a Binary \emph{Group Operation},
$\cdot$, expressed as a Tuple, $(G,\cdot)$, satisfying four \emph{Group Axioms}:
\begin{enumerate}
  \item Closure: $\forall a,b \in G, a \cdot b \in G$
  \item Associativity: $\forall a,b,c \in G, (a \cdot b) \cdot c = a
    \cdot (b \cdot c)$
  \item Identity Element: $\exists! e \in G : \forall a \in G,
    e \cdot a = a \cdot e = a$
  \item Inverse elements: $\forall a \in G, \exists b \in G :
    a \cdot b = b \cdot a = e$
\end{enumerate}
The Identity Element is the only Element with the unique Property:
\[
  e \cdot e = e
\]

The Signature (\S\ref{sec:signature}) for Groups is $\{\cdot, 1, ^{-1}\}$.

The number of Elements in a Group $\mathrm{G}$ is known as its \emph{Order} and
may be denoted $|\mathrm{G}|$.

an Integer $k$ that solves the Equation $b^k = a$ for an Element $a$ of a Group
$G$ is a \emph{Discrete Logarithm} (\S\ref{sec:discrete_logarithm}):
$k = \log_b a$

\fist Group Presentation (\S\ref{sec:group_presentation})

\fist an \emph{Algebraic Group} (or \emph{Group Variety}
\S\ref{sec:algebraic_group}) is a Group that is an Algebraic Variety such that
the Multiplication and Inversion Operations are given by Regular Maps
(\S\ref{sec:regular_map}) on the Variety

Every Group can be realized as the Fundamental Group
(\S\ref{sec:fundamental_group}) of some Space (\S\ref{sec:space})

\emph{Cayley's Theorem} (\S\ref{sec:cayleys_theorem}): every Group is Isomorphic
to the Permutation Group (\S\ref{sec:permutation_group}) of some Set

\fist Homogeneous Space (\S\ref{sec:homogeneous_space})

A Group, $G$, within a Category, $\mathbf{C}$, may be viewed as a Subset of the
Hom-set of an Object, $X$:
\[
    G \subseteq Hom_{\mathbf{C}}(X,X)
\]

\fist cf. \emph{Groupoid} (\S\ref{sec:groupoid}) -- generalizes a Group
replacing the Binary Operation with a Partial Function; where a Group can be
thought of as a Group of Endomorphic Symmetry Transformations relating an Object
to itself, a Groupoid can be thought of as a collection of Symmetry
Transformations acting between possibly more than one Object; a Group then can
be seen as a Groupoid with a single Object



\subsubsection{Group Generator}\label{sec:group_generator}

a \emph{Generating Set} of a Group is a Subset such that every Element of the
Group can be expressed as a combination (under the Group Operation) of Finitely
many Elements of the Subset and their Inverses

Cyclic Subgroup (\S\ref{sec:cyclic_subgroup})

\begin{itemize}
  \item Roots of Unity (\S\ref{sec:unity_root}) -- the $n$th Roots of Unity form
    an irreducible Representation (\S\ref{sec:group_representation}) of any
    Cyclic Group of Order $n$
  \item Primitive Element (\S\ref{sec:primitive_element}) -- a Generator of the
    Multiplicative Group of a Finite Field (\S\ref{sec:finite_field})
  \item ...
\end{itemize}



\subsubsection{Group Word}\label{sec:group_word}

For a Group $G$ and a Subset of $G$, $S$, a \emph{Word} in $S$ is any Expression
of the form:
\[
    s_1^{\varepsilon_1}s_2^{\varepsilon_2} \cdots s_n^{\varepsilon_n}
\]
where $s_1,\ldots,s_n \in S$ and $\varepsilon_i \in \{-1, 1\}$ and $n$ is the
\emph{Length} of the Word. The \emph{Empty Word} is used as the Identity Element
in the Free Group of Words (\S\ref{sec:free_group}).



\paragraph{Reduction}\label{sec:word_reduction}\hfill

If an Element appears in a Word next to its Inverse, a \emph{Reduction} may be
applied which removes those Elements from the Word due to the Group Axioms which
imply that the resulting Word is equivalent. A \emph{Reduced Word} contains no
such redundant pairs.

A Word is \emph{Cyclically Reduced} if and only if every Cyclic Permutation
(\S\ref{sec:cyclic_permutation}) of the Word is Reduced (that is, it is Reduced
and the first and last Element are not Inverses).



\subsubsection{Group Order}\label{sec:group_order}

\subsubsection{Period}\label{sec:period}

(or \emph{Order}) of an Element

Periodic Group (\S\ref{sec:periodic_group})



\subsubsection{Center}\label{sec:group_center}

The \emph{Center} of a Group $G$, denoted $Z(G)$, is a Normal Subgroup of $G$
defined as:
\[
    Z(G) = \{ z \in G | \forall g \in G, zg = gz \}
\]
If $G$ is an Abelian Group, $Z(G) = G$.

For a Symmetric Group (\S\ref{sec:symmetric_group}) $S_n$:
\[
    Z(S_n) = \{e\}
\]

For a General Linear Group (\S\ref{sec:general_linear_group}) $GL_n(R)$:
\[
    Z(GL_n(R)) = \{\lambda \mathrm{I}\}
\]

Characteristic Subgroup (\S\ref{sec:characteristic_subgroup})



\subsubsection{Centralizer}\label{sec:group_centralizer}

\subsubsection{Normalizer}\label{sec:group_normalizer}

\subsubsection{Commutator}\label{sec:group_commutator}

Commutator (\S\ref{sec:commutator})

Commutator Subgroup (\S\ref{sec:commutator_subgroup})



\subsubsection{Conjugacy Class}\label{sec:conjugacy_class}

Two Elements $a$ and $b$ of a Group $G$ are \emph{Conjugates} if there is an
Element $g \in G$ such that:
\[
    gag^{-1} = b
\]
which reads ``Conjugation of $a$ by $g$ results in $b$.'' Conjugation is
Invariant if and only if the Elements are Commutative:
\[
    gag^{-1} = a \Leftrightarrow ag = ga
\]

A \emph{Conjugacy Class} for an Element $a$ is defined as:
\[
    Cl(a) = \{ b \in G | \exists g \in G : b = gag^{-1}\}
\]



\subsubsection{Cayley Graph}\label{sec:cayley_graph}

the Cayley Graph for a Cyclic Group (\S\ref{sec:cyclic_group}) is a Directed
Cycle Graph (\S\ref{sec:directed_cycle})



% ------------------------------------------------------------------------------
\subsection{Group Homomorphism}\label{sec:group_homomorphism}
% ------------------------------------------------------------------------------

A \emph{Group Homomorphism} $h$, between two Groups, $G$ and $H$, is a
Morphism $h : G \rightarrow H$ that preserves Group operations
$\cdot_G$ and $\cdot_H$. That is, for $x,y \in G$:
\[
    h(x \cdot_G y) = h(x) \cdot_H h(y)
\]
From this it follows that Group Homomorphisms have the Properties:
\[
    h(e_G) = e_H
\]\[
    h(x^{-1}) = h(x)^{-1}
\]

There is always a \emph{Trivial Homomorphism} between any two Groups
$G$ and $H$:
\[
    \forall x \in G, h (x) = e_H
\]

The Composition of two Homomorphisms $h : G \rightarrow H$ and $g : F
\rightarrow G$, $h \circ g$ is a Homomorphism.

Taken as Monoidal Categories (\S\ref{sec:monoidal_category}), two
Groups $G, H$ may be related by a Functor $f$ which is equivalent to a
Group Homomorphism:
\[
    f : G \rightarrow H
\]
and for $x,y \in G$:
\[
    f(xy) = f(x)f(y)
\]



\subsubsection{Group Homomorphism Image}\label{sec:group_image}

The \emph{Image} of a Group Homomorphism:
\[
    im(h) = \{ x' \in H | h(x) = x', x \in G \}
\]
The Image of a $h$ is a Subgroup (\S\ref{sec:subgroup}) of $H$:
\[
    im(h) \subset H
\]

If the Image of $h$ is equal to $H$, and the Kernel
(\S\ref{sec:group_kernel}) is $\{e_G\}$ (\S\ref{sec:morphism_kernel}),
then $h$ is an Isomorphism (\S\ref{sec:group_isomorphism}).



\subsubsection{Group Homomorphism Kernel}\label{sec:group_kernel}

The \emph{Kernel} of a Group Homomorphism, $f : G \rightarrow G'$,
denoted by $ker(f)$ is defined as:
\[
    ker(f) = \{g \in G | f(g) = e_{G'}\}
\]
where $e_{G'}$ is the Identity Element of $G'$, that is, the Preimage
or Fiber (\S\ref{sec:fiber}) of the Singleton Set $\{e_{G'}\}$. The
Kernel is a Normal Subgroup (\S\ref{sec:normal_subgroup}) of the
Domain Group of the Group Homomorphism, in this case $G$:
\[
    ker(f) \triangleleft G
\]

If $ker(f) = \{e_G\}$ and the Image (\S\ref{sec:group_image}) is equal
to the Codomain Group, $im(f) = G'$, then $f$ is an Isomorphism
(\S\ref{sec:group_isomorphism}).

An Equivalence Class may be defined for an Element $g \in G$ by taking
the Left Coset (\S\ref{sec:coset}) of the Kernel, $ker(f) = H
\triangleleft G$:
\[
    gH = \{ gh | h \in H \}
\]
Each such Equivalence Class has the same Order as $H$, which results
in the following Corollary for a Group Homomorphism $f : G \rightarrow
G'$ with Kernel $H$:
\[
    |G| = |H||im(f)|
\]



\subsubsection{Group Isomorphism}\label{sec:group_isomorphism}

A \emph{Group Isomorphism} is a Bijective Group Homomorphism.

A Group Homomorphism $h : G \rightarrow H$ is an Isomorphism if
$ker(h) = e_G$ and $im(h) = H$. If $G = H$ then $h$ is an
\emph{Automorphism} (\S\ref{sec:group_automorphism}).

The existence of an Isomorphism between Groups $G$ and $H$ imply the
following Properties:
\begin{itemize}
    \item $|G| = |H|$
    \item $G$ is Abelian $\Leftrightarrow$ $H$ is Abelian
    \item $G$ and $H$ have the same number of Elements of every Order
\end{itemize}



\paragraph{Isomorphism Theorem}\label{sec:isomorphism_theorem}\hfill

\emph{First Isomorphism Theorem}

\emph{Second Isomorphism Theorem}

\emph{Third Isomorphism Theorem}



\subsubsection{Group Automorphism}\label{sec:group_automorphism}

A \emph{Automorphism} is an Endomorphism that is also an Isomorphism.



\paragraph{Inner Automorphism}\label{sec:inner_automorphism}\hfill

An \emph{Inner Automorphism} of a Group $G$ is an Automorphism $f : G
\rightarrow G$ defined by:
\[
    \forall g \in G, f(g) = a^{-1}ga
\]
where $a$ is a given fixed Element of $G$.



% ------------------------------------------------------------------------------
\subsection{Subgroup}\label{sec:subgroup}
% ------------------------------------------------------------------------------

Given a Group $G$, a \emph{Subgroup} $H$ is a Subset of Group Elements
that are still a Group under the Group Operation of $G$, denoted $H
\leq G$. When a Subgroup $H$ is a Proper Subset of $G$, $H$ is a
\emph{Proper Subgroup} of $G$, denoted $H \neq G$. If a Subgroup has
the same Order as the containing Group then the Groups are equivalent.

All Groups have at least two Subgroups: the Trivial Group
(\S\ref{sec:trivial_group}) and Group itself. A Group with exactly
these two Subgroups and no others is a \emph{Simple Group}
(\S\ref{sec:simple_group}).

If a Subgroup $G$ of $H$ is Non-abelian then $H$ is also Non-abelian.

The Subgroups of $(\mathbb{Z},+)$ are of the form $(b\mathbb{Z},+)$
where $b \in \mathbb{Z}$.



\subsubsection{Coset}\label{sec:coset}

The \emph{Coset} of a Subgroup (\S\ref{sec:subgroup}) $H$ in a Group
$G$ is defined as:
\begin{description}
\item[Left Coset:] $gH = {gh : h \in H}$
\item[Right Coset:] $Hg = {hg : h \in H}$
\end{description}
These Subsets are Disjoint and Partition (\S\ref{sec:partition}) $G$.

The \emph{Index} of a Subgroup $H$ is the number of distinct Left
Cosets of $H$, denoted $[G:H]$. As a Corollary:
\[
    |G| = |H|[G:H]
\]



\subsubsection{Normal Subgroup}\label{sec:normal_subgroup}

A \emph{Normal Subgroup} of a Group $G$ is a \emph{Subgroup} $H$ that is
invariant under Conjugation (\S\ref{sec:conjugacy_class}) by the Elements of
$G$, $gHg^{-1} = H$ and is denoted $H \triangleleft G$. $H$ is a Normal Subgroup
of $G$ if and only if $\forall g \in G, gH = Hg$.

In an Abelian Group, all Subgroups are Normal Subgroups.



\subsubsection{Characteristic Subgroup}\label{sec:characteristic_subgroup}

Invariant under all Automorphisms of the Parent Group

Group Center (\S\ref{sec:group_center})



\paragraph{Commutator Subgroup}\label{sec:commutator_subgroup}\hfill

\emph{Commutator Subgroup}

Subgroup Generated (\S\ref{sec:group_generator}) by all the
Commutators (\S\ref{sec:group_commutator}) of a Group



\subsubsection{Cyclic Subgroup}\label{sec:cyclic_subgroup}

Given any Group $G$ with Element $g$, the Subgroup Generated
(\S\ref{sec:group_generator}) by the single Element $g$ is called the
\emph{Cyclic Subgroup} of $g$, denoted $<g>$, containing all Powers of
$g$.

If $m$ is the smallest positive Integer such that $g^m = e$, then $m$
is the \emph{Order} of $g$. By \emph{Lagrange's Theorem}, in a Finite
Group, every Element has a Finite Order and the Order of every Element
divides evenly the Order of the Group.



\subsubsection{Skeleton}\label{sec:group_skeleton}

\subsubsection{Zappa-Szep Product}\label{sec:zappa_szep}

Effect-coeffect (\S\ref{sec:computational_effect},
\S\ref{sec:coeffect}) matched pair



% ------------------------------------------------------------------------------
\subsection{Group Product}\label{sec:group_product}
% ------------------------------------------------------------------------------

The \emph{Group Product} of two Groups $G$ and $H$, denoted $G \times
H$, is a Group with Elements from the Cartesian Product of the
Elements of $G$ and $H$, $\{(g,h) | g \in G, h \in H\}$, and Group
Operation defined Componentwise:
\[
    (g_1, h_1) \cdot_{G \times H} (g_2, h_2)
    = (g_1 \cdot_G g_2, h_1 \cdot_H h_2)
\]

Direct Product (\S\ref{sec:direct_product})



% ------------------------------------------------------------------------------
\subsection{Free Product}\label{sec:free_product}
% ------------------------------------------------------------------------------

Coproduct (\S\ref{sec:coproduct}) of Groups $A \oplus B$

\fist Cf. Direct Sum of Modules, Disjoint Union of Sets

\fist Note that for Abelian Groups, the Product and Coproduct
are Isomorphic, given by the Direct Sum (\S\ref{sec:direct_sum})



% ------------------------------------------------------------------------------
\subsection{Trivial Group}\label{sec:trivial_group}
% ------------------------------------------------------------------------------

A \emph{Trivial Group} is a Group $\{e\}$ with only an Identity Element and no
others. The Trivial Group may be denoted $0$ when Group Operation is thought of
as addition or as $1$ when Group Operation is thought of as multiplication.

Given any Group $G$ with Identity Element $e$, the Trivial Group $\{e\}$ is a
Subgroup (\S\ref{sec:subgroup}) of $G$, called the \emph{Trivial Subgroup}:
\[
    \{e\} \leq G
\]

All Trivial Groups are Isomorphic to each other, so \emph{the} Trivial Group may
be spoken of. The Trivial Group is the Zero Object in the Category of Groups.

The Trivial Group is a Cyclic Group (\S\ref{sec:cyclic_group}) of Order 1 and
may be denoted $Z_1$ in this context.



% ------------------------------------------------------------------------------
\subsection{Commutative Group}\label{sec:commutative_group}
% ------------------------------------------------------------------------------

or \emph{Abelian Group}

In Abelian Groups, the Coproduct is Isomorphic to the Product:
\[
  A + B \cong A \times B
\]

\fist Chain Complex (Homological Algebra \S\ref{sec:chain_complex}), Topological
Spectrum (Homotopical Algebra \S\ref{sec:topological_spectrum})

\fist a Graded Ring (\S\ref{sec:graded_ring}) is a Ring that is the Direct Sum
(the \emph{Gradation} or \emph{Grading}) of Abelian Groups $R_i$ such that
$R_iR_j \subseteq R_{i+j}$

\fist Modules (\S\ref{sec:module}) -- an Abelian Group and a Scalar Ring with a
Scalar Multiplication Operation

\fist Vector Spaces (\S\ref{sec:vector_space}) -- an Abelian Group and a Scalar
Field with a Scalar Multiplication Operation

\fist a Free Abelian Group (\S\ref{sec:free_commutative_group}) is exactly a
Free Module over the Ring $\ints$ of Integers

\begin{itemize}
  \item Elliptic Curves (\S\ref{sec:elliptic_curve})
  \item ...
\end{itemize}



\subsubsection{Torsion-free Rank}\label{sec:torsionfree_rank}

Torsion (\S\ref{sec:torsion})



\subsubsection{Additive Group}\label{sec:additive_group}

An \emph{Additive Group} refers to a Group where the Group Operation
can be thought of as \emph{Addition} (usually Abelian).



\paragraph{Group Lattice}\label{sec:group_lattice}\hfill

Subgroup of the Additive Group $\reals^n$, which is Isomorphic to the Additive
Group $\ints^n$ and which Spans the Real Vector Space $\reals^n$



\subparagraph{Integer Lattice}\label{sec:integer_lattice}\hfill



\subsubsection{Multiplicative Group}\label{sec:multiplicative_group}

A \emph{Multiplicative Group} is defined in terms of a Structure with
Invertible Elements such as a \emph{Ring} (\S\ref{sec:ring}) $R$
having Multiplication, $\bullet$, as one of its Operations:
\[
  (R \ {0}, \bullet)
\]

Circle Group (\S\ref{sec:circle_group})



\subsubsection{Divisible Group}\label{sec:divisible_group}

\subsubsection{Cyclic Group}\label{sec:cyclic_group}

A Group $G$ is \emph{Cyclic} when there exists an element $g \in G$ such that:
\[
    G = \langle g \rangle = \{ g^n | n \in \mathbb{Z} \}
\]
Any Group $G$ of Prime Order is Cyclic is an Abelian Simple Group
(\S\ref{sec:simple_group}) and is Generated by any $g \in G : g \neq
e$.

Every Finite Cyclic Group of Order $n$ is Isomorphic
(\S\ref{sec:group_isomorphism}) to the Additive Group $\mathbb{Z}/n\mathbb{Z}$
of the Integers Modulo $n$. The Infinite Class of Cyclic Groups of Prime Order
is one of the possible Classes for the Classification of Finite Simple Groups
(\S\ref{sec:finite_simple_group}).

Every Infinite Cyclic Group is Isomorphic to the Additive Group $(\mathbb{Z},
+)$ of the Integers.

the Cayley Graph (\S\ref{sec:cayley_graph}) for a Cyclic Group is a
Directed Cycle Graph (\S\ref{sec:directed_cycle})



\paragraph{Finite Cyclic}\label{sec:finite_cyclic}\hfill

\paragraph{Infinite Cyclic}\label{sec:infinite_cyclic}\hfill

Frieze Group (\S\ref{sec:frieze_group})



\subsubsection{Circle Group}\label{sec:circle_group}

Multiplicative Group (\S\ref{sec:multiplicative_group})

- Unit Circle on Complex Plane

- Geometrically defined Group due to Pascal's Theorem
  \cite{lemmermeyer-shirali09}



% ------------------------------------------------------------------------------
\subsection{Non-abelian Group}\label{sec:noncommutative_group}
% ------------------------------------------------------------------------------

Non-commutative Group



% ------------------------------------------------------------------------------
\subsection{Simple Group}\label{sec:simple_group}
% ------------------------------------------------------------------------------

A Group $G$ is a \emph{Simple Group} if it has only the Trivial Group and the
entire Group $G$ as Normal Subgroups (\S\ref{sec:normal_subgroup}).

The only Abelian Simple Groups are the Cyclic Groups
(\S\ref{sec:cyclic_group}) of Prime Order.

For $n \geq 5$, the Alternating Group (\S\ref{sec:alternating_group})
$A_n$ is a Simple Group.



\subsubsection{Finite Simple Group}\label{sec:finite_simple_group}

\emph{Classification Theorem} -- every Finite Simple Group is Isomorphic to one
of:
\begin{itemize}
  \item the Infinite Class of \emph{Cyclic Groups} (\S\ref{sec:cyclic_group}) of
    Prime Order
  \item the Infinite Class of \emph{Alternating Groups}
    (\S\ref{sec:alternating_group}) of Degree at least $5$
  \item the Infinite Class of \emph{Groups of Lie Type}
    (\S\ref{sec:lie_type_group})
  \item 26 \emph{Sporadic Groups} (\S\ref{sec:sporadic_group})
  \item the Tits Group (\S\ref{sec:tits_group})
\end{itemize}



\paragraph{Sporadic Group}\label{sec:sporadic_group}\hfill

\emph{Happy Family} -- Subgroups or Subquotients of the Monster Group

\emph{Pariahs} -- others



\subparagraph{Monster Group}\label{sec:monster_group}\hfill

largest Sporadic Group; contains all but six of the other Sporadic Groups as
Subgroups or Subquotients

Griess Algebra (\S\ref{sec:griess_algebra}) -- Commutative Non-associative
Algebra on a Real Vector Space of Dimension 196884 that has the Monaster Group
$M$ as its Automorphism Group



\paragraph{Tits Group}\label{sec:tits_group}\hfill



\subsubsection{Infinite Simple Group}\label{sec:infinite_simple_group}



% ------------------------------------------------------------------------------
\subsection{Quotient Group}\label{sec:quotient_group}
% ------------------------------------------------------------------------------

For a Normal Subgroup (\S\ref{sec:normal_subgroup}) $H$ of a Group
$G$, $H \triangleleft G$, the \emph{Quotient Group} $G/H$ is the Set
of all Left Cosets (\S\ref{sec:coset}) of $H$ in $G$:
\[
    G/H = \{ aH : a \in G \}
\]



\subsubsection{Group Extension}\label{sec:group_extension}

\subsubsection{Solvable Group}\label{sec:solvable_group}

a Group that can be constructed from Commutative (Abelian) Groups using
Extensions

or equivalently a Solvable Group is a Group whose Derived Series terminates in
the Trivial Subgroup



% ------------------------------------------------------------------------------
\subsection{Periodic Group}\label{sec:periodic_group}
% ------------------------------------------------------------------------------

\emph{Periodic Group} (or \emph{Torsion Group}) if all Elements have
Finite Period (\S\ref{sec:group_order})



\subsubsection{Finite Group}\label{sec:finite_group}

A \emph{Finite Group} is a Group with a Finite Order (\S\ref{sec:group_order})

Any Finite Non-abelian Group is of Even Order.

All Finite Groups are Periodic Groups



\subsubsection{Group Exponent}\label{sec:group_exponent}



% ------------------------------------------------------------------------------
\subsection{Free Group}\label{sec:free_group}
% ------------------------------------------------------------------------------

The \emph{Free Group}, $F_S$, over a Set, $S$, called the \emph{Free
  Generating Set}, consists of all Reduced Words
(\S\ref{sec:group_word}) in $S$ as Elements and Concatenation of Words
(with Reduction) as the Group Operation. As a Group Presentation
(\S\ref{sec:group_presentation}):
\[
    F_S = \langle S, \rangle
\]
Every Word is Conjugate to a Cyclically Reduced Word
(\S\ref{sec:word_reduction}), and a Cyclically Reduced Conjugate of a
Cyclically Reduced Word is a Cyclic Permutation of the Word.

A Group $G$ is called Free if it is Isomorphic to $F_S$ for some
Subset $S$ of $G$, that is, every Element of $G$ can be written
uniquely as a Product of finitely many Elements of $S$ and their
Inverses.

Every Subgroup of a Free Group is Free. \cite{hatcher02}

$S = {1}, F_S = (\ints,+)$



% ------------------------------------------------------------------------------
\subsection{Automorphism Group}\label{sec:automorphism_group}
% ------------------------------------------------------------------------------

The Automorphisms of an Object $x$ form an \emph{Automorphism Group}, denoted
$Aut(x)$, under Composition of Automorphisms.

the Automorphism Group is a Subgroup (\S\ref{sec:subgroup}) of the Symmetric
Group (\S\ref{sec:symmetric_group}) %FIXME: correct ?

Automorphism Group is a Submonoid (\S\ref{sec:submonoid})
of the Endomorphism Monoid %FIXME

$Aut_\cat{C}(x) = End_\cat{C}(x) \cap Iso_\cat{C}(C) = Iso_\cat{C}(x,x)$

Every Group is an Automorphism Group up to Equivalence %FIXME

Isometry Groups (\S\ref{sec:isometry_group})

Diffeomorphism Groups %FIXME

Homeomorphism Groups %FIXME

\begin{itemize}
\item the Monster Group (\S\ref{sec:monster_group}) $M$ is the Automorphism
  Group of the Griess Algebra (\S\ref{sec:griess_algebra})-- a Commutative
  Non-associative Algebra on a Real Vector Space of Dimension 196884
\item ... TODO
\end{itemize}



\subsubsection{Symmetric Group}\label{sec:symmetric_group}

The Group whose Elements are all the Permutations (\S\ref{sec:permutation}) of a
Set $S$ is the \emph{Symmetric Group} $Sym(S)$. The Symmetric Group on
${1, 2, ..., n} \in \mathbb{N}$ is denoted $\mathrm{S}_n$.

For finite $n$, $\mathrm{S}_n$ is a Finite Group of Order $n!$.

\begin{itemize}
    \item $S_1 = \{e\}$
    \item $S_2 = \{e,\tau\}$ where $\tau$ is the Transposition
      (\S\ref{sec:transposition}) $(12)$
    \item $S_3 = \{e, \tau, \tau', \tau'', \sigma, \sigma'\}$ where $\sigma$ and
      $\sigma'$ are Permutations of length 3: $(123)$ and $(321)$ respectively
\end{itemize}
The Group $S_n$ is Non-abelian for $n \geq 3$.

the Automorphism Group (\S\ref{sec:automorphism_group}) is a Subgroup of the
Symmetric Group %FIXME: correct ?

For $k \leq n$, $S_k \subset S_n$.

Symmetry (\S\ref{sec:structure_symmetry})

Cayley's Theorem (\S\ref{sec:cayleys_theorem}): every Group is Isomorphic to a
Subgroup of the Symmetric Group on $G$

\fist the General Linear Group (\S\ref{sec:general_linear_group}) $GL_n(\reals)$
is a Subgroup of the Symmetric Group $Sym(\reals)$ (FIXME: clarify)

\fist $\mathbb{F}_q$-geometry (\S\ref{sec:f1_geometry}) -- the General Linear
Group of Invertible Matrices with Coefficients in Finite Field $\mathbf{F}_q$
converges towards the Symmetric Group $S_n$ (\S\ref{sec:symmetric_group}) on $n$
Elements as $q \rightarrow 1$

\begin{itemize}
  \item Group of $n \times n$ Permutation Matrices
    (\S\ref{sec:permutation_matrix})
\end{itemize}



\paragraph{Permutation Group}\label{sec:permutation_group}\hfill

Permutations (\S\ref{sec:permutation})

Any Permutation Group on a Set $S$ with Cardinality $n$ is any Subgroup of the
Symmetric Group $\mathrm{S}_n$



\subparagraph{Cayley's Theorem}\label{sec:cayleys_theorem}\hfill

Every Group is Isomorphic to the Permutation Group of some Set



\subparagraph{Alternating Group}\label{sec:alternating_group}\hfill

Group of Even Permutations of a Finite Set

the Alternating Groups of Degree at least $5$ forms one of the Classes of Simple
Finite Groups (\S\ref{sec:finite_simple_group})



\paragraph{Projective Linear Group}\label{sec:projective_linear_group}\hfill

\subparagraph{Special Projective Linear Group}
\label{sec:special_projective_linear_group}\hfill




\subsubsection{Group of Lie Type}\label{sec:lie_type_group}

Finite Groups of Lie Type are one of the Classes of Finite Simple Groups
(\S\ref{sec:finite_simple_group})



\subsubsection{Inner Automorphism Group}\label{sec:inner_automorphism_group}

For the Automorphism Group of a Group $G$, $Aut(G)$, one can always
construct a Homomorphism:
\[
    f : G \rightarrow Aut(G)
\]
defined as:
\[
    \forall g \in G, f (g) (h) = g h g^{-1}
\]
with Kernel equal to the Center (\S\ref{sec:group_center}) of $G$:
\[
    ker(f) = Z(G)
\]
and the Image is a Subgroup of $Aut(G)$ called the \emph{Inner
  Automorphism Group} of $G$, denoted $Inn(G)$, defined as:
\[
    Inn(G) = \{ a \in Aut(G) | \exists g \in G : a(h) = g h g^{-1} \}
\]
where $a$ is an \emph{Inner Automorphism}
(\S\ref{sec:inner_automorphism}) of $G$.

By the First Isomorphism Theorem (\S\ref{sec:isomorphism_theorem}),
$Inn(G)$ is Isomorphic to the Quotient Group
(\S\ref{sec:quotient_group}) $G / Z(G) \cong Inn(G)$.



\subsubsection{Outer Automorphism Group}\label{sec:outer_automorphism_group}

\subsubsection{Complete Group}\label{sec:complete_group}



% ------------------------------------------------------------------------------
\subsection{Transformation Group}\label{sec:transformation_group}
% ------------------------------------------------------------------------------

\emph{Permutation Group} (\S\ref{sec:permutation_group}) Set

\emph{Matrix Group} (\S\ref{sec:matrix_group}) Vector Space

Automorphism Group (\S\ref{sec:automorphism_group})



\subsubsection{Group Action}\label{sec:group_action}

Invariant Theory (\S\ref{sec:invariant_theory}): Actions of Groups on
Algebraic Varieties (\S\ref{sec:algebraic_variety})



\paragraph{Transitive Group Action}\label{sec:transitive_action}\hfill

$G$ Acts \emph{Transitively} on $X$ if $X$ is Non-empty and if for each pair
$x,y \in X, \exists g \in G : g \cdot x = y$

When $X$ is a Topological Space, $X$ is called the \emph{Homogeneous Space}
(\S\ref{sec:homogeneous_space}) of $G$, and the Elements of $G$ are called the
\emph{Symmetries} (\S\ref{sec:symmetry}) of $X$.



\paragraph{Flow}\label{sec:flow}\hfill

A \emph{Flow} on a Set $X$ is a Group Action of the Additive Group of Real
Numbers on $X$:
\[
  \varphi : X \times \reals \rightarrow X
\]
such that for all $x \in X$ and all $s,t \in \reals$:
\[
  \varphi(x,0) = x
\]
and:
\[
  \varphi(\varphi(x,t),s) = \varphi(x,s+t)
\]
Alternatively, for all $t \in \reals$, $\varphi^t : X \rightarrow X$ is a
Bijection with Inverse $\varphi^{-t} : X \rightarrow X$ when:
\[
  \varphi^0 = 1_X
\]
and:
\[
  \varphi^s \circ \varphi^t = \varphi^{s + t}
\]
and the Real Parameter $t$ can be taken as a generalized Functional Power
(\S\ref{sec:functional_power}) as in an Iterated Function

Sometimes the implicit notation $x(t)$ is written for $\varphi^t(x_0)$.

if $X$ is a Topological Space, $\varphi$ is usually required to be Continuous
(\S\ref{sec:continuous_function}); in this case the Flow forms a One-parameter
Subgroup (\S\ref{sec:one_parameter_group}) of Homeomorphisms
(\S\ref{sec:homeomorphism})

if $X$ is a Differentiable Manifold, $\varphi$ is usually required to be
Differentiable (\S\ref{sec:differentiable_function}); in this case the Flow
forms a One-parameter Subgroup of Diffeomorphisms (\S\ref{sec:diffeomorphism})

In Dynamical Systems (\S\ref{sec:dynamical_system}), given $x \in X$, the Set
$\varphi(x,t)$ is called the \emph{Trajectory} (\S\ref{sec:trajectory}) of $x$
under $\varphi$. If the Flow is generated by a Vector Field
(\S\ref{sec:vector_field}) then its Trajectories are the Images of its Integral
Curves (\S\ref{sec:integral_curve}).

\fist Ordinary Differential Equations (ODEs \S\ref{sec:ode})

\fist Gradient Flow (\S\ref{sec:gradient_flow})

\fist Vector Flow (\S\ref{sec:vector_flow})



\subparagraph{Schr\"oder's Equation}\label{sec:schroders_equation}\hfill

Functional Equation (\S\ref{sec:functional_equation})

$\Psi(h(x)) = s\Psi(x)$ where $h(x)$ is a given Function and $\Psi(x)$ is
an Unknown Function

Flows specified through Solutions of Schr\"oder's Equation

%FIXME: move this section ?



\paragraph{Orbit}\label{sec:orbit}\hfill

for a Group $G$ Acting on a Set $X$, the \emph{Orbit} $G \dot x$ of an Element
$x$ in $X$ is the Set of Elements in $X$ to which $x$ can be \emph{moved} by
the Elements of $G$, denoted:
\[
  G \dot x = \{ g \dot x \| g \in G \}
\]

cf. \emph{Coinvariants} (\S\ref{sec:coinvariant}): Equivalence Classes
(\S\ref{sec:equivalence_class}) in Classification Problems
(\S\ref{sec:classification_problem})

\fist Orbit Space (\S\ref{sec:orbit_space})

\fist Trajectory (\S\ref{sec:trajectory})

the Fundamental Domain (\S\ref{sec:fundamental_domain}) of a Group Action of
Group $G$ on a Topological Space (\S\ref{sec:topological_space}) $X$ by
Homeomorphisms (\S\ref{sec:homeomorphism}), is a Set $D$ of representatives for
the Orbits



\paragraph{Continuous Group Action}\label{sec:continuous_group_action}

A \emph{Continuous Group Action} on a Topological Space
(\S\ref{sec:topological_space}) $X$ is a Group Action of a Topological Group
(\S\ref{sec:topological_group}) $G$ that is a \emph{Continuous Map}
(\S\ref{sec:continuous_map})



\subsubsection{Point Group}\label{sec:point_group}

\paragraph{Crystallographic Point Group}\label{sec:crystallographic_point_group}
\hfill

Crystallographic Point Groups



\subsubsection{Space Group}\label{sec:space_group}

\paragraph{Crystallographic Group}\label{sec:crystallographic_group}\hfill

Cocompact (\S\ref{sec:cocompact_space}), Discrete Subgroup
(\S\ref{sec:discrete_group}) of the Isometries (\S\ref{sec:isometry}) of some
Euclidean Space (\S\ref{sec:euclidean_space})



\subsubsection{Braid Group}\label{sec:braid_group}

as an Infinite Abelian Group, used by Group-based and Non-commutative
Cryptography protocols



\paragraph{Loop Braid Group}\label{sec:loop_braid_group}\hfill



\subsubsection{Matrix Group}\label{sec:matrix_group}

any Finite Group is Linear



\paragraph{General Linear Group}\label{sec:general_linear_group}\hfill

A \emph{General Linear Group} of Degree $n$ over a Ring (\S\ref{sec:ring}) $R$
has as an Underlying Set of $n \times n$ Invertible Matrices and the Group
Operation is ordinary Matrix Multiplication and is denoted $GL_n(R)$ and is a
Subgroup of $Sym(R^n)$ (FIXME: clarify)

Abstractly, $GL(V)$ is the Automorphism Group of Linear Automorphisms
(Isomorphic Endomorphisms) on $V$

without the requirement of Invertibility (Nonzero Determinant), the resulting
Algebraic Structure is the Full Linear Semigroup
(\S\ref{sec:full_linear_semigroup})

Elementary Matrices (\S\ref{sec:elementary_matrix}) generate the General Linear
Group of Invertible Matrices

the Special Linear Group $SL(V)$ is the Normal Subgroup of $GL(V)$ given by the
Kernel of the Determinant $\mathrm{det} : GL(n,F) \rightarrow F^\times$,
consisting of Matrices with Determinant $1$, i.e. \emph{Orientation and Volume
  Preserving Transformation}

\fist Linear Algebraic Groups (\S\ref{sec:linear_algebraic_group})

\fist $\mathbb{F}_q$-geometry (\S\ref{sec:f1_geometry}) -- the General Linear
Group of Invertible Matrices with Coefficients in Finite Field $\mathbf{F}_q$
converges towards the Symmetric Group $S_n$ (\S\ref{sec:symmetric_group}) on $n$
Elements as $q \rightarrow 1$

Classical Subgroups -- Subgroups of $GL(V)$ preserving a sort of Bilinear Form
(\S\ref{sec:bilinear_form}) on the Vector Space $V$:
\begin{itemize}
  \item the Orthogonal Group (\S\ref{sec:orthogonal_group}) $O(V)$ -- preserves
    a Non-degenerate Quadratic Form (\S\ref{sec:quadratic_form}) on $V$
  \item the Symplectic Group (\S\ref{sec:symplectic_group}) $Sp(V)$
    -- preserves a Symplectic Form (Non-degenerate Alternating Form
    \S\ref{sec:symplectic_form}) on $V$
  \item the Unitary Group (\S\ref{sec:unitary_group}) $U(V)$
    -- preserves a Non-degenerate Hermitian Form (\S\ref{sec:hermitian_form})
    on $V$
\end{itemize}

other Subgroups:
\begin{itemize}
  \item the Subgroup of Invertible Upper Triangular Matrices
    (\S\ref{sec:upper_triangular}) is a Borel Subgroup
    (\S\ref{sec:borel_subgroup})
  \item the Subgroup of Invertible Diagonal Matrices
    (\S\ref{sec:diagonal_matrix}) is Isomorphic to the Multiplicative Group
    $(F^\times)^n$, corresponding to \emph{Dilations} and \emph{Contractions}
  \item the Subgroup of Nonzero Scalar Matrices (\S\ref{sec:scalar_matrix})
    $Z(V)$ is Isomorphic to the Multiplicative Group $F^\times$, forming the
    Center (\S\ref{sec:group_center}) of $GL(n,F)$ and is a Normal Abelian
    Subgroup
  \item the Subgroup of Permutation Matrices (\S\ref{sec:permutation_matrix})
\end{itemize}

Extensions:
\begin{itemize}
  \item Affine Group (\S\ref{sec:affine_group}) $Aff(n,F)$ -- Extension by
    Group of Translations in $F^n$ is the Group of all Affine Transformations
    (\S\ref{sec:affine_transformation}) on the Affine Space underlying the
    Vector Space $F^n$
  \item Projective Linear Group (\S\ref{sec:projective_linear_group})
    $PGL(n,F)$ -- Quotient of $GL(n,F)$ by its Center, the Subgroup of Nonzero
    Scalar Matrices $Z(n,F)$, which is the induced Action
    (\S\ref{sec:group_action}) on the associated Projective Space
    (\S\ref{sec:projective_space})
\end{itemize}



\subparagraph{Special Linear Group}\label{sec:special_linear_group}\hfill

A \emph{Special Linear Group} of Degree $n$ over a Ring (\S\ref{sec:ring}) $R$,
denoted $SL_n(R)$, is the Set of $n \times n$ Matrices with Determinant
(\S\ref{sec:determinant}) equal to $1$, with the Group Operation of ordinary
Matrix Multiplication. This is a Normal Subgroup of the General Linear Group
$GL_n(R)$ given by the Kernel of the Determinant Function $ker(det)$, where:
\[
  det : GL_n(R) \rightarrow R^\times
\]
and $R^\times$ is the Multiplicative Group
(\S\ref{sec:multiplicative_group}) on $R$.

The Elements of $SL_n(R)$ are the Volume and Orientation preserving
Linear Transformations of $R^n$.

Elements forms a Subvariety (\S\ref{sec:subvariety_theorem}) of $GL(V)$, i.e.
they Satisfy a Polynomial Equation since the Determinant is Polynomial in the
entries

$SZ(n,F)$ -- the Center (\S\ref{sec:group_center}) of $SL(n,F)$ is the Set of
all Scalar Matrices (\S\ref{sec:scalar_matrix}) with Unit Determinant and is
Isomorphic to the Group of $n$th Roots of Unity (\S\ref{sec:unity_root}) of the
Field $F$



\subparagraph{Unitary Group}\label{sec:unitary_group}\hfill

$U(n)$ Group of $n \times n$ Unitary Matrices

Subgroup of the General Linear Group

preserves a Non-degenerate Hermitian Form (\S\ref{sec:hermitian_form}) on $V$

3-fold Intersection of the Orthogonal Group (\S\ref{sec:orthogonal_group}), the
Symplectic Group (???) and the Complex Group (???): %FIXME xrefs
\[
  U(n) = O(2n) \cap Sp(2n,\reals) \cap GL(n,\comps)
\]



\textbf{Special Unitary Group}

$SU(2)$ is Isomorphic to the Unit Quaternions (\S\ref{sec:versor}) and
Diffeomorphic with $S^3$ (\S\ref{sec:unit_glome})

Solovay-Kitaev Theorem (TODO)



\subparagraph{Symplectic Group}\label{sec:symplectic_group}\hfill

$Sp(V)$ -- Symplectic Group: Subgroup of the General Linear Group $GL(V)$ which
Preserves a Symplectic Form (\S\ref{sec:symplectic_form}) on $V$ (i.e. a
Non-degenerate Alternating Form)



\subparagraph{Jet Group}\label{sec:jet_group}\hfill

Algebraic Group (\S\ref{sec:algebraic_group})



\subsubsection{Symmetry Group}\label{sec:symmetry_group}

A \emph{Symmetry Group} $G$ is one which Acts Transitively
(\S\ref{sec:transitive_action}) on a Topological Space $X$, called the
\emph{Homogeneous Space} (\S\ref{sec:homogeneous_space}) of $G$, and the Elements
of $G$ are called the \emph{Symmetries} of $X$.

Special cases when $G$ is the Automorphism Group
(\S\ref{sec:automorphism_group}) of $X$:
\begin{itemize}
  \item Homeomorphism Group (\S\ref{sec:homeomorphism}) -- Topology
  \item Isometry Group (\S\ref{sec:isometry_group}) -- ``Rigid'' Geometry
  \item Diffeomorphism Group (\S\ref{sec:diffeomorphism}) -- Differential
    Geometry
\end{itemize}


Inversion Symmetry %FIXME

\fist Symmetric Space (\S\ref{sec:symmetric_space})

Analytic Geometry (Part \ref{part:analytic_geometry}): Euclidean Space
(\S\ref{sec:euclidean_space}), Affine Space (\S\ref{sec:affine_space}), and
Projective Space (\S\ref{sec:projective_space}) are Homogeneous Spaces
(\S\ref{sec:homogeneous_space}) for their respective Symmetry Groups

Symmetry Groups of Regular Polyhedra are an example of Coxeter Groups
(\S\ref{sec:coxeter_group}), i.e. Finite Euclidean Reflection Groups
(\S\ref{sec:reflection_group})

Integral Geometry (\S\ref{sec:integral_geometry}): Theory of Measures
(\S\ref{sec:measure}) on ``Geometrical Space'' Invariant under Symmetry Group of
that Space

1D Symmetry Groups: %FIXME TODO
\begin{enumerate}
  \item Trivial
  \item Reflection
\end{enumerate}



\paragraph{Discrete Symmetry Group}\label{sec:discrete_symmetry_group}\hfill

a Symmetry Group that is a Discrete Isometry Group
(\S\ref{sec:discrete_isometry_group})



\paragraph{Dihedral Group}\label{sec:dihedral_group}\hfill

\emph{Dihedral Group}


\subparagraph{Klein 4-group}\label{sec:klein_4group}\hfill

The \emph{Klein 4-group} or $K_4$ (also $V$ for
``\emph{Vierergruppe}'') is the Group resulting from the Direct
Product (\S\ref{sec:direct_product}) of two Cyclic Groups
(\S\ref{sec:cyclic_group}) of Order 2: $K_4 = Z_2 \times Z_2$.

As a Permutation Representation on 4 Elements $K_4$ is a Subgroup of
the Alternating Group (\S\ref{sec:alternating_group}) on 4 Elements,
$A_4$:
\[
    K_4 = \{ (), (12)(34), (13)(24), (14)(23) \}
\]

The Klein 4-group is an Abelian Group and is Isomorphic to the
Dihedral Group of Order 4 and is the smallest Non-cyclic Group.

The Automorphism Group (\S\ref{sec:automorphism_group}) $Aut(K_4)$ is
Isomorphic to $S_3$.



\paragraph{Line Group}\label{sec:line_group}\hfill

Unit Cell



\subparagraph{Frieze Group}\label{sec:frieze_group}\hfill

Infinite Cyclic Group (\S\ref{sec:infinite_cyclic})

Discrete Subgroup (\S\ref{sec:discrete_group}) of the Isometry Group
(\S\ref{sec:isometry_group}) of the Euclidean Plane
(\S\ref{sec:euclidean_plane})



\paragraph{Wallpaper Group}\label{sec:wallpaper_group}\hfill

Cocompact (\S\ref{sec:cocompact_space}), Discrete Subgroup
(\S\ref{sec:discrete_group}) of the Isometry Group (\S\ref{sec:isometry_group})
of the Euclidean Plane (\S\ref{sec:euclidean_plane})



\paragraph{Orthogonal Group}\label{sec:orthogonal_group}\hfill

$O(n)$

Orthogonal Matrices (\S\ref{sec:orthogonal_matrix})

the Unitary Group (\S\ref{sec:unitary_group}) 3-fold Intersection of the
Orthogonal Group (\S\ref{sec:orthogonal_group}), the Symplectic Group (???) and
the Complex Group (???) %FIXME xrefs

the Orthogonal Group is Compact

the Isometry Group of the single Positive Definite Real Quadratic form of a
given Dimension is a Compact Orthogonal Group $O(n)$

$O(1)$ is a two Point Discrete Space (\S\ref{sec:discrete_space}), Isomorphic
to the $0$-sphere $S^0$

the Homotopy Groups (\S\ref{sec:homotopy_group}) $\pi_k(O)$ of the Real
Orthogonal Group are related to the Homotopy Groups of Spheres
... TODO



\paragraph{Special Orthogonal Group}\label{sec:special_orthogonal_group}\hfill

$SO(n)$

Subgroup of $O(n)$ of Orthogonal Matrices with Determinant $1$ (\emph{Rotation
  Matrices} \S\ref{sec:rotation_matrix})



\paragraph{Indefinite Orthogonal Group}
\label{sec:indefinite_orthogonal_group}\hfill

\emph{Indefinite Orthogonal Group} $O(p,q)$ is Non-compact

Lie Group of Dimension $n(n-1)/2$ with all Linear Transfomrations leaving
invariant a Nondegenerate Symmetric Billinear Form of signature $(p,q)$ where
$n = p+q$



\paragraph{Reflection Group}\label{sec:reflection_group}\hfill

Discrete Group Generated by a Set of Reflections (\S\ref{sec:reflection}) of a
Finite-dimensional Euclidean Space (\S\ref{sec:euclidean_space})
%FIXME: xref generated



\subparagraph{Coxeter Group}\label{sec:coxeter_group}\hfill

The Coxeter Groups are precisely the Finite Euclidean Reflection Groups

examples:
\begin{itemize}
  \item Symmetry Groups (\S\ref{sec:symmetry_group}) of Regular Polyhedra
\end{itemize}




\subparagraph{Coxeter-Dynkin Diagram}\label{sec:coxeter_dynkin_diagram}\hfill

cf. Dynkin Diagram (\S\ref{sec:dynkin_diagram})



\paragraph{Euclidean Group}\label{sec:euclidean_group}\hfill

$E(n)$ or $ISO(n)$

Symmetry Group of $n$-dimensional Euclidean Space (\S\ref{sec:euclidean_space})
with Euclidean Isometries associated with the Euclidean Distance (Rigid
Transformations \S\ref{sec:rigid_transformation}) as its Elements

makes Euclidean Geometry (\S\ref{sec:euclidean_geometry}) a case of
Klein Geometry (\S\ref{sec:klein_geometry})

%FIXME isometry ?

\emph{Euclidean Motions}

Translation, Rotation, Reflection

$(2 \leq n)$ -- Glide Reflections

$(3 \leq n)$ -- Screw Operations



\subparagraph{Special Euclidean Group}\label{sec:special_euclidean}\hfill

Screw Displacements

$SE(3)$

Homographies of Quaternions (FIXME)



\paragraph{Affine Group}\label{sec:affine_group}\hfill

Affine Transformations (\S\ref{sec:affine_transformation})

Extension of General Linear Group $GL(n,F)$ by Group of Translations in $F^n$

$Aff(V) = V \rtimes GL(V)$



\subparagraph{Special Affine Group}\label{sec:special_affine_group}\hfill



\subsubsection{Continuous Transformation Group}
\label{sec:continuous_transformation_group}

Continuous Symmetry (\S\ref{sec:continuous_symmetry})

Lie Groups (\S\ref{sec:lie_group}) -- a Group that is also a
Differentiable Manifold



% ------------------------------------------------------------------------------
\subsection{Topological Group}\label{sec:topological_group}
% ------------------------------------------------------------------------------

A \emph{Topological Group} is a Group $G$ with Topology (\S\ref{sec:topology})
$\tau$ on $G$ such that the Group \emph{Binary Operation} and \emph{Inverse
  Function} are \emph{Continuous Maps} (\S\ref{sec:continuous_map})
with respect to $\tau$

Any Group can be considered as a Topological Group by giving it the Discrete
Topology (\S\ref{sec:discrete_topology}), therefore any Theorems about
Topological Groups apply to all Groups.

examples:
\begin{itemize}
  \item Real Numbers under Addition
  \item Abelian Varieties (\S\ref{sec:abelian_variety}) as Topological Groups
    are Tori %FIXME
\end{itemize}

Topological Groups are Homogeneous Spaces (\S\ref{sec:homogeneous_space})

A \emph{Continuous Group Action} (\S\ref{sec:continuous_group_action}) on a
Topological Space (\S\ref{sec:topological_space}) $X$ is a Group Action of a
Topological Group $G$ that is a \emph{Continuous Map}
(\S\ref{sec:continuous_map}).



\subsubsection{Discrete Group}\label{sec:discrete_group}

A \emph{Discrete Group} is a Topological Group with a Discrete Topology
(\S\ref{sec:discrete_topology})

because Topological Groups are Homogeneous (\S\ref{sec:homogeneous_space}), a
Group is Discrete if and only if the Singleton containing the Identity is an
Open Set
%FIXME: identity = origin ???

Any Group can be considered as a Topological Group by giving it the Discrete
Topology (\S\ref{sec:discrete_topology}).

Since every Map from a Discrete Space is Continuous
(\S\ref{sec:continuous_map}), Topological Homomorphisms between Discrete Groups
are exactly the Group Homomorphisms between the underlying Groups and therefore
there is an Isomorphism between the Category of Groups and the Category of
Discrete Groups.

Any Discrete Group can be viewed as a $0$-dimensional Lie Group
(\S\ref{sec:lie_group}) %FIXME

cf. Discrete Isometry Group (\S\ref{sec:discrete_isometry_group}), Discrete
Symmetry Group (\S\ref{sec:discrete_symmetry_group})

examples:
\begin{itemize}
  \item Frieze Groups (\S\ref{sec:frieze_group}), Wallpaper Groups
    (\S\ref{sec:wallpaper_group}) are Discrete Subgroups of the Isometry Group
    (\S\ref{sec:isometry_group}) of the Euclidean Plane
    (\S\ref{sec:euclidean_plane})
  \item Crystallographic Groups (\S\ref{sec:crystallographic_group}) are
    Cocompact (\S\ref{sec:cocompact_space}), Discrete Subgroups of the
    Isometries (\S\ref{sec:isometry}) of some Euclidean Space
  \item TODO
  ...
\end{itemize}



\subsubsection{One-parameter Group}\label{sec:one_parameter_group}

a Continuous (\S\ref{sec:continuous_function}) Group Homomorphism:
\[
  \varphi : \reals \rightarrow G
\]
from the Real Line as an Additive Group to some other Topological Group $G$

Infinitesimal Transformation (\S\ref{sec:infinitesimal_transformation})

Lie Algebra (\S\ref{sec:lie_algebra})

Flows (\S\ref{sec:flow}) on Topological Spaces (\S\ref{sec:topological_space})
and Differentiable Manifolds (\S\ref{sec:differentiable_manifold}) form
One-parameter Subgroups of Homeomorphisms (\S\ref{sec:homeomorphism}) and
Diffeomorphisms (\S\ref{sec:diffeomorphism}), respectively



\subsubsection{Profinite Group}\label{sec:profinite_group}



% ------------------------------------------------------------------------------
\subsection{Abstract Group}\label{sec:abstract_group}
% ------------------------------------------------------------------------------

\subsubsection{Group Representation}\label{sec:group_representation}

Linear Transformations (\S\ref{sec:linear_transformation}) of Vector Spaces
(\S\ref{sec:vector_space})



\subsubsection{Character Theory}\label{sec:character_theory}

the \emph{Character} of a Group Representation is a Function on the Group that
associates to each Group Element the Trace of the Corresponding Matrix



% ------------------------------------------------------------------------------
\subsection{Combinatorial Group Theory}\label{sec:combinatorial_group_theory}
% ------------------------------------------------------------------------------

Fundamental (Undecidable) Decision Problems in Group Theory:
\begin{itemize}
  \item Word Problem (\S\ref{sec:word_problem})
  \item Conjugacy Problem
  \item Group Isomorphism Problem
\end{itemize}



\subsubsection{Word Problem}\label{sec:word_problem}

because of the Word Problem for Groups, it is not possible to classify
Manifolds up to Diffeo- or Homeo-morphism in Dimensions $\geq 4$--but they can
be classified up to Cobordism

%FIXME: explain



% ==============================================================================
\section{Ring Theory}\label{sec:ring_theory}
% ==============================================================================

% ------------------------------------------------------------------------------
\subsection{Ring}\label{sec:ring}
% ------------------------------------------------------------------------------

A \emph{Ring} is a Set $R$ with two Binary Operators, $+$ and $\cdot$, where:
\begin{itemize}
\item $R$ is an Abelian Group under $+$
    \begin{enumerate}
        \item $+$ is Associative
        \item $+$ is Commutative
        \item There exists an Additive Identity $0 \in R$
        \item For all $a \in R$, there exists an Additive Inverse $-a
          \in R$
    \end{enumerate}
\item $R$ is a Monoid under $\cdot$
    \begin{enumerate}
        \item $\cdot$ is Associative
        \item There exists a Multiplicative Identity $1 \in R$
    \end{enumerate}
\item $\cdot$ Distributes over $+$
    \begin{enumerate}
        \item $\forall a,b,c \in R,
            a \cdot (b + c) = (a \cdot b) + (a \cdot c)$
            (Left Distributivity)
        \item $\forall a,b,c \in R,
            (b + c) \cdot a = (b \cdot a) + (c \cdot a)$
            (Right Distributivity)
    \end{enumerate}
\end{itemize}
The Signature (\S\ref{sec:signature}) for Rings is $\{+, -, \cdot, 0, 1\}$.

A Ring may be seen as an Abelian Group with extra Structure, namely Ring
Multiplication.

a Ring Homomorphism is sometimes called an \emph{Algebra}

cf. Differential Graded Ring (Homological Algebra
\S\ref{sec:differential_graded_ring}), Ring Spectrum (Homotopical Algebra
\S\ref{sec:ring_spectrum})

A \emph{Finite Ring} is a Ring that has a Finite number of Elements.

A \emph{Unit} is an Element of a Ring $R$ that has an Inverse
Element in the Multiplicative Monoid of $R$.

The term \emph{Unital Ring} (\S\ref{sec:unital_ring}) is used to
indicate a Ring with a Multiplicative Identity, to differentiate from
other \emph{Pseudo-rings} that may lack a Multiplicative Identity.

A \emph{Rng} (\S\ref{sec:rng}) is a Pseudo-ring that satisfies all
Ring axioms except a Multiplicative Identity.

a Non-trivial Commutative Ring such that every Non-zero element has a
Multiplicative Inverse is a \emph{Field} (\S\ref{sec:field})

\begin{itemize}
  \item $\ints$ Integers (\S\ref{sec:integer})
  \item Holonomic Functions (\S\ref{sec:holonomic_function})
  \item ...
\end{itemize}

\fist Modules (\S\ref{sec:module}) can be viewed as \emph{Representations}
(\S\ref{sec:representation}) of Rings; every Ring $R$ has a ``natural''
$R$-module structure on itself where the Module Action is defined as the
Multiplication in the Ring; it is therefore useful to study a Ring by the
Category of Modules over a Ring; two Rings are Morita Equivalent
(\S\ref{sec:morita_equivalence}) if their Module Categories are Equivalent

every Ring can be thought of as a Monoid in the Category $\cat{Ab}$ of Abelian
Groups (thought of as a Monoidal Category under the Tensor Product of
$\ints$-modules), and the Monoid Action ofa Ring $R$ on an Abelian Group is an
$R$-module

\begin{itemize}
  \item Differential Ring (\S\ref{sec:differential_ring})
  \item to any Topological Space $X$ can be associated its \emph{Integral
    Cohomology Ring} (\S\ref{sec:cohomology_ring})
  \item Coordinate Ring (\S\ref{sec:coordinate_ring}) of an Affine Variety
  \item ... MORE?
\end{itemize}

(from #math IRC user unyu): by ``Initiality'' there is excactly one Ring
Homomorphism $f : \ints \rightarrow R$ for any Ring $R$



\subsubsection{Unit}\label{sec:ring_unit}

\emph{Invertible Element} in the Multiplicative Monoid of $R$

Set of Units in any Ring is Closed under Multiplication



\subsubsection{Nilpotent}\label{sec:nilpotent}

An Element $x$ of a Ring $R$ is \emph{Nilpotent} if there is a positive Integer
$n$ such that $x^n = 0$.

\begin{itemize}
  \item the Dual Numbers (\S\ref{sec:dual_number}) extend the Real Numbers with
    a Nilpotent Element $\varepsilon$ such that $\varepsilon^2 = 0$
\end{itemize}



\subsubsection{Rng}\label{sec:rng}

\subsubsection{Unital Ring}\label{sec:unital_ring}

\subsubsection{Semiring}\label{sec:semiring}

A \emph{Semiring} (or \emph{Rig}) is a Ring without Additive Inverses,
i.e. a Ri\emph{n}g without \emph{Negation}

Algebraic Types (\S\ref{sec:algebraic_type}) form a Semiring.

McBride16 uses a Rig for Resource Annotations of Dependent Linear
Types (\S\ref{sec:dependent_linear})

\begin{itemize}
  \item with the Operations of Disjoint Sum and Product, the
    Category of Combinatorial Species (\S\ref{sec:combinatorial_species})
    becomes a kind of Semi-ring, $\cat{FinSet}\llbracket{X}\rrbracket$, the
    Ring of Hurwitz Series (TODO: xref) with Coefficients in $\cat{FinSet}$, and
    the Ring of Hurwitz Series over $\ints$
  \item ...
\end{itemize}



\subsubsection{Zero Ring}\label{sec:zero_ring}

The \emph{Zero Ring}, denoted $\{0\}$ or $\mathbf{0}$, is the Unique Ring
(up to Isomorphism) consisting of one Element with Operations:
\[
    0 + 0 = 0
\] \[
    0 * 0 = 0
\]
In the Category of all Rings, $\mathbf{Rng}$, the Zero Ring is
Terminal Object (\S\ref{sec:terminal_object}) and the Ring of Integers
$\mathbf{Z}$ is the Initial Object (\S\ref{sec:initial_object}).



\subsubsection{Characteristic}\label{sec:ring_characteristic}

the \emph{Characteristic} of a Ring $R$ is the smallest number of times the
Ring's Multiplicative Identity ($1$) must be Summed to get the Ring's Additive
Identity ($0$), otherwise if the Additive Identity is \emph{never} reached by
Summing the Multiplicative Identity, the Ring is said to have
\emph{Characteristic Zero}

\fist $\mathbb{F}_1$ -- ``Field of Characteristic $1$''; $\mathbb{F}_1$-geometry
(\S\ref{sec:f1_geometry})

\url{https://johncarlosbaez.wordpress.com/2018/03/03/nonstandard-integers-as-complex-numbers/}:

any two Algebraically Closed Fields (i.e. Fields containing the Roots of all
Polynomials) of Characteristic Zero that have the same Uncountable Cardinality
must be Isomorphic

it follows that any Field of Characteristic Zero whose Cardinality is less than
that of the Continuum is \emph{Isomorphic} to some Sub-field of the Complex
Numbers



\subsubsection{Morita Equivalence}\label{sec:morita_equivalence}

two Rings are Morita Equivalent (\S\ref{sec:morita_equivalence}) if their
Module Categories are Equivalent



% ------------------------------------------------------------------------------
\subsection{Ring Ideal}\label{sec:ring_ideal}
% ------------------------------------------------------------------------------

Closure, Absorption

plays the role of an ``idealized generalization'' of an Element in a Ring

for Commutative Rings, Ideals generalize the notion of \emph{Divisibility} and
Decomposition of an Integer into Prime Numbers in Algebra

there is a Bijection between the Ideals and the Congruence Relations
(\S\ref{sec:congruence_relation}) on a Ring

\fist cf. Order Ideal (\S\ref{sec:order_ideal})

For every Ring $R$, $\{0\}$ and $R$ are Ideals where $\{0\}$ is called
the \emph{Zero Ideal} and $R$ is called the \emph{Unit Ideal}.

If $R$ is a Division Ring (\S\ref{sec:division_ring}) or Field
(\S\ref{sec:field}) then $\{0\}$ and $R$ are its only Ideals.

\emph{Hilbert's Nullstellensatz}: fundamental correspondence between Ideals of
Polynomial Rings (\S\ref{sec:polynomial_ring}) and Algebraic Sets
(\S\ref{sec:algebraic_set})

a Ring in which there is no Strictly Increasing Infinite Chain
(\S\ref{sec:chain}) of Left Ideals is called a Noetherian Ring
(\S\ref{sec:noetherian_ring})

Right Ideal

Left Ideal

Two-sided Ideal

\fist a Weyl Algebra (\S\ref{sec:weyl_algebra}) is Isomorphic to the Quotient
of the Free Algebra (\S\ref{sec:free_algebra}) on two Generators, $X$ and $Y$,
by the Ideal generated by the Element $YX - XY - 1$



\subsubsection{Proper Ideal}\label{sec:proper_ideal}

an Ideal that is a Proper Subset of $R$

Ideals as \emph{Kernels} (TODO)



\paragraph{Maximal Ideal}\label{sec:maximal_ring_ideal}\hfill

Ideal that is Maximal with respect to Set Inclusion among Proper Ideals

a Local Ring (\S\ref{sec:local_ring}) $R$ has a Unique Maximal Left or Right
Ideal

cf. Maximal Order Ideal (\S\ref{sec:maximal_ideal})



\subsubsection{Principal Ideal}\label{sec:principal_ideal}

\emph{Principal Ideal}

Generated by a single Element of $R$ by multiplication with every
Element of $R$

Ring $\ints$ of Integers: Principal Ideals correspond one-for-one with
Non-negative Integers



\subsubsection{Primary Ideal}\label{sec:primary_ideal}

Prime Ideals (\S\ref{sec:prime_ideal}) are both Primary and Semiprime
(\S\ref{sec:semiprime_ideal})



\subsubsection{Semiprime Ideal}\label{sec:semiprime_ideal}

Prime Ideals (\S\ref{sec:prime_ideal}) are both Primary
(\S\ref{sec:semiprime_ideal}) and Semiprime



\subsubsection{Prime Ideal}\label{sec:prime_ideal}

both Primary (\S\ref{sec:primary_ideal}) and Semiprime
(\S\ref{sec:semiprime_ideal})

the Prime Ideals of the Ring $\ints$ of Integers are the Subsets that contain
all the Multiples of a given Prime Number together with the Zero Ideal $\{0\}$

\fist using Isomorphism of the Categories of Boolean Algebras and Boolean Rings
the notion of Prime Ideal of a Poset (\S\ref{sec:order_prime_ideal}) coincides
with the notion of Prime Ideal for Rings

the Spectrum (\S\ref{sec:ring_spectrum}), $\Spec(R)$ of a Commutative Ring $R$
is the Set of all Prime Ideals of $R$



\paragraph{Primitive Ideal}\label{sec:primitive_ideal}\hfill



% ------------------------------------------------------------------------------
\subsection{Coring}\label{sec:coring}
% ------------------------------------------------------------------------------

a Biring (\S\ref{sec:biring}) is both a Ring and a Coring



% ------------------------------------------------------------------------------
\subsection{Finite Ring}\label{sec:finite_ring}
% ------------------------------------------------------------------------------

any Finite Simple Ring (\S\ref{sec:simple_ring}) is Isomorphic to the Ring
$M_n(GF(q))$ of $n \times n$ Matrices (\S\ref{sec:matrix_ring}) over a Finite
Field of Order $q$

every Finite Field (\S\ref{sec:finite_field}) is an example of a Finite Ring



% ------------------------------------------------------------------------------
\subsection{Simple Ring}\label{sec:simple_ring}
% ------------------------------------------------------------------------------

A \emph{Simple Ring} is a Nonzero Ring that has no Two-sided Ideal
(\S\ref{sec:ring_ideal}) besides the Zero Ideal and itself.

can always be considered as a \emph{Simple Algebra} (\S\ref{sec:simple_algebra})

any Finite (\S\ref{sec:finite_ring}) Simple Ring is Isomorphic to the Ring
$M_n(GF(q))$ of $n \times n$ Matrices (\S\ref{sec:matrix_ring}) over a Finite
Field of Order $q$

examples:
\begin{itemize}
  \item Weyl Algebra (\S\ref{sec:weyl_algebra}) -- Ring of Differential
    Operators with Polynomial Coefficients in one Variable:\[
      f_m(X)\partial^m_X + f_m{-1}(X)\partial^{m-1}_X + \cdots +
      f_1(X)\partial_X + f_0(X)
    \]
  \item Matrix Ring (\S\ref{sec:matrix_ring}) $M_n(D)$ over a Division Ring
    (\S\ref{sec:division_ring}) $D$ is an Artinian (Semi-)simple Ring
    (\S\ref{sec:artinian_ring})
\end{itemize}



\subsubsection{Semisimple Ring}\label{sec:semisimple_ring}

\paragraph{Artinian Ring}\label{sec:artinian_ring}\hfill

special class of Semisimple Ring

every Left Artinian Ring is Left Noetherian (\S\ref{sec:noetherian_ring}) and a
Left Artinian Ring is Left Noetherian if and only if it is Right Artinian, and
the equivalently for exchange of ``Left'' and ``Right''



\subparagraph{Matrix Ring}\label{sec:matrix_ring}\hfill

Matrix Ring $M_n(D)$ over a Division Ring (\S\ref{sec:division_ring}) $D$ is an
Artinian (Semi-)simple Ring (\S\ref{sec:artinian_ring})

any Finite (\S\ref{sec:finite_ring}) Simple Ring (\S\ref{sec:simple_ring}) is
Isomorphic to the Ring $M_n(GF(q))$ of $n \times n$ Matrices over a Finite Field
(\S\ref{sec:finite_field}) of Order $q$



% ------------------------------------------------------------------------------
\subsection{Noetherian Ring}\label{sec:noetherian_ring}
% ------------------------------------------------------------------------------

a Ring in which there is no Strictly Increasing Infinite Chain of Left Ideals

every Left Artinian Ring (\S\ref{sec:artinian_ring}) is Left Noetherian and a
Left Artinian Ring is Left Noetherian if and only if it is Right Artinian, and
the equivalently for exchange of ``Left'' and ``Right''

\fist a Weyl Algebra (\S\ref{sec:weyl_algebra}) is a (Left and Right)
Noetherian Ring



% ------------------------------------------------------------------------------
\subsection{Commutative Ring}\label{sec:commutative_ring}
% ------------------------------------------------------------------------------

A \emph{Commutative Ring} is a Ring where $\cdot$ is Commutative.

\fist \emph{Commutative Algebra} (\S\ref{sec:commutative_algebra}): study of
Commutative Rings, their Ideals (\S\ref{sec:ring_ideal}), and Modules
(\S\ref{sec:module}) over Commutative Rings

\emph{Determinant}

Integers $\ints$ (\S\ref{sec:integer})

Fields (\S\ref{sec:field})

Algebraic Geometry (Part \ref{part:algebraic_geometry}): study of
Geometry using Commutative Rings

\fist a \emph{Ringed Space} (\S\ref{sec:ringed_space}) is a Topological Space
with a collection of Commutative Rings with elements as Functions on each Open
Set of the Space; equivalently a Family of Commutative Rings parameterized by
Open Subsets of a Topological Space

\fist a \emph{Scheme} (\S\ref{sec:scheme}) is a Ringed Space that is locally a
Spectrum (\S\ref{sec:spectrum}) of a Commutative Ring



\subsubsection{Integral Element}\label{sec:integral_element}

\subsubsection{Ring Spectrum}\label{sec:ring_spectrum}

The \emph{Spectrum}, $\Spec(R)$, of a Commutative Ring $R$ is the Set of all
Prime Ideals of $R$. A Locally Ringed Space (\S\ref{sec:locally_ringed_space})
of this form is called an \emph{Affine Scheme} (\S\ref{sec:affine_scheme}).

a Module Spectrum (\S\ref{sec:module_spectrum}) is a Module over a Ring Spectrum

cf. Differential Graded Ring (Homological Algebra
\S\ref{sec:differential_graded_ring})

\fist Note that ``Spectrum'' is a highly overloaded term and may otherwise
refer to:
\begin{itemize}
  \item Spectrum (Analysis \S\ref{sec:spectrum}) -- generalization of Matrix
    Spectrum (Set of Eigenvalues (\S\ref{sec:matrix_spectrum}) of a Matrix to
    general Operators

  \item Graph Spectrum (Spectral Graph Theory \S\ref{sec:graph_spectrum}) ------------
    Spectrum of the Adjacency Matrix (\S\ref{sec:adjacency_matrix}) of a Graph

  \item Spectral Norm (\S\ref{sec:spectral_norm}) -- Schatten $\infty$-norm of
    a Matrix (\S\ref{sec:spectral_norm})

  \item C$^*$-algebra Spectrum (Dual \S\ref{sec:cstar_dual}) -- Set of Unitary
    Equivalence Classes of Irreducible $*$-representations of a C$^*$-algebra
    (similar notion to Ring Spectrum)

  \item Topological Spectrum (Stable Homotopy Theory
    \S\ref{sec:topological_spectrum}) -- represents a Generalized Cohomology
    Theory (\S\ref{sec:generalized_cohomology_theory})

  \item Polygon Spectrum (\S\ref{sec:polygon_spectrum}) -- the Set of all $n$
    for which an $n$-equidissection (\S\ref{sec:equidissection}) of a Polygon
    $P$ exists

  \item Sentence Spectrum (\S\ref{sec:sentence_spectrum}) -- the Set of Natural
    Numbers occurring as the size of a Finite Model (\S\ref{sec:finite_model})
    in which a given Sentence is True
  \item Theory Spectrum (\S\ref{sec:theory_spectrum}) -- the number of
    Isomorphism Classes of Models of various Cardinalities
\end{itemize}

\fist Derived Algebraic Geometry (\S\ref{sec:derived_algebraic_geometry}) ------------
generalization of Algebraic Geometry replacing Commutative Rings with Ring
Spectra in Algebraic Topology where higher Homotopy accounts for
Non-discreteness (Tor) of the Structure Sheaf (FIXME: Tor ???)
%TODO xref

(wiki):

\fist \emph{Schemes} (\S\ref{sec:scheme}) are Locally Ringed Spaces
(\S\ref{sec:locally_ringed_space}) obtained by ``gluing together'' Spectra of
Commutative Rings

\fist \emph{Affine Scheme} (\S\ref{sec:affine_scheme}) -- a Spectrum augmented
with the Zariski Topology (\S\ref{sec:zariski_topology}) and a Structure Sheaf
(\S\ref{sec:structure_sheaf}), turning it into a Locally Ringed Space

\fist cf. C$^*$-algebra Spectrum (Dual \S\ref{sec:cstar_dual})

$Spec$ is a Functor (FIXME: details)



\subsubsection{Place}\label{sec:place}

\emph{Place} of a Commutative Unital Ring

an Equivalence Class of Absolute Values (Non-trivial Multiplicative Seminorms
\S\ref{sec:seminorm})

\fist Global Analytic Geometry (\S\ref{sec:global_analytic_geometry}): treats
all Places ``on equal footing'', contrary to Scheme Theory
(\S\ref{sec:scheme_theory})



\subsubsection{Real Closed Ring}\label{sec:real_closed_ring}

(wiki):

Commutative Ring $A$ that is a Subring of a Product of Real Closed Fields and
Closed under Continuous Semi-algebraic Functions
(\S\ref{sec:semialgebraic_function}) defined over the Integers

the Real Closed Fields (\S\ref{sec:real_closed}) are exactly the Real Closed
Rings that are Fields

the Real Closure of the Polynomial Ring (\S\ref{sec:polynomial_ring})
$\reals[T_1,\ldots,T_n]$ is the Ring of Continuous Semi-algebraic Functions
$\reals^n \rightarrow \reals$

an arbitrary Ring $R$ is \emph{Semi-real} (\S\ref{sec:semireal_ring}), i.e.
$-1$ is not a Sum of Squares in $R$, if and only if the Real Closure of $R$ is
not the Null Ring (Zero Ring \S\ref{sec:zero_ring})

the Real Closure of an Ordered Field (\S\ref{sec:ordered_field}) is not in
general the Real Closure of the underlying Field: the Real Closure of a Field
$F$ is a certain ``Subdirect Product'' of the Real Closures of the Ordered
Fields $(F,P)$ where $P$ ``runs through'' the Orderings of $F$ (FIXME: clarify)

the Category $\cat{RCR}$ of Real Closed Rings has Real Closed Rings as Objects
and Ring Homomorphisms as Maps; Closed under arbitrary Products, Direct and
Inverse Limits (in the Category of Commutative Unital Rings), Tensor Products,
has arbitrary Limits and Co-limits and is a Variety (Universal Algebra
\S\ref{sec:variety})

the Class of Real Closed Rings is First-order Axiomatizable (TODO) and
Undecidable

the Class of all Real Closed Valuation Rings (TODO) is Decidable

the Class of all Real Closed Fields is Decidable (\fist Tarski-Seidenberg
Theorem \S\ref{sec:tarski_seidenberg})

if $F$ is a Field with no assumed Ordering, then $F$ has a Real Closure which
may not be Field but is just a Real Closed Ring



\paragraph{Semi-real Ring}\label{sec:semireal_ring}\hfill

$-1$ is not a Sum of Squares in $R$

an arbitrary Ring $R$ is Semi-real if and only if the Real Closure
(\S\ref{sec:real_closed_ring}) of $R$ is not the Null Ring (Zero Ring
\S\ref{sec:zero_ring})



\subsubsection{Polynomial Ring}\label{sec:polynomial_ring}

\emph{Polynomial Ring} (or \emph{Free Commutative Algebra})

Free Algebra (\S\ref{sec:free_algebra}) is the Non-commutative
analogue of a Polynomial Ring

the Set of Complex Numbers (\S\ref{sec:complex_number}) is the Quotient Ring of
the Polynomial Ring in the Indeterminate $i$ by the Ideal generated by the
Polynomial $i^2 + 1$

$R[x]$ -- Ring of Polynomials in the Indeterminate $x$ over the Ring $R$

$x^0, x^1, x^2, x^3, \ldots$ -- Monomials

$p(x) = \sum_{i=0}^\infty a_i x^i$, $a_i \in R$ where only a Finite Number of
$a_i$ are Non-zero -- Polynomials (\S\ref{sec:polynomial})

Degree $n$, Leading Coefficient $a_n$

Monic Polynomial: Leading Coefficient $a_n = 1$

$R[x_1, x_2]$ is Isomorphic to $R[x_1][x_2]$

Polynomial Ring in $n$ Variables $R[x_1, \ldots, x_n]$

an Algebraically Closed Field (\S\ref{sec:algebraically_closed}) $F$ contains a
Root (\S\ref{sec:function_root}) for every Non-constant Polynomial in the Ring
of Polynomials $F[x]$ in the Variable $x$ with Coefficients in $F$

an Affine Algebraic Set (Affine Variety \S\ref{sec:affine_variety}) is the
Intersection of the Zero Sets of a number of Polynomials in a Polynomial Ring
(\S\ref{sec:polynomial_ring}) $k[x_1,\ldots,x_n]$ over a Field

\emph{Hilbert's Nullstellensatz}: fundamental correspondence between Ideals
(\S\ref{sec:ring_ideal}) of Polynomial Rings and Algebraic Sets
(\S\ref{sec:algebraic_set})

\fist Gr\"obner Basis (Computational Algebraic Geometry
\S\ref{sec:grobner_basis})



\subsubsection{Biring}\label{sec:biring}

\fist Differential Algebra (\S\ref{sec:differential_algebra})

(nlab):

A \emph{Biring} is a Commutative Ring $R$ equipped with the following Ring
Homomorphisms:
\begin{itemize}
  \item \emph{Coaddition} -- $R \rightarrow R \otimes R$
  \item \emph{Cozero} -- $R \rightarrow R \otimes R$
  \item \emph{Co-additive Inverse} -- $R \rightarrow R$
  \item \emph{Comultiplication} -- $R \rightarrow R \otimes R$
  \item \emph{Multiplicative Counit} -- $R \rightarrow \ints$
\end{itemize}

a Biring is both a Ring and Co-ring (\S\ref{sec:coring}), i.e. it is a
Commutative Ring Object in the Opposite Category of the Category of Commutative
Rings $\cat{CRing}^{op}$ (\emph{a.k.a.} the Category of Affine Schemes
\S\ref{sec:affine_scheme})



% ------------------------------------------------------------------------------
\subsection{Non-commutative Ring}\label{sec:noncommutative_ring}
% ------------------------------------------------------------------------------

\subsubsection{Free Algebra}\label{sec:free_algebra}

or \emph{Free Non-commutative Ring}

Non-commutative analogue of \emph{Polynomial Rings} (\emph{Free Commutative
  Rings} \S\ref{sec:polynomial_ring})

A \emph{Free Algebra}, $\mathbf{A}$, is defined by a Set of \emph{Free
  Generators}, $S$, and a Type Signature, $\rho$, which Generate an
Underlying Set, $A$. If $\psi : S \rightarrow A$ is a Function,
$\mathbf{A}$ may be represented by the Free Algebra $(A,\psi)$ if for
every Algebra $\mathbf{B}$ of type $\rho$ with Function $\tau : S
\rightarrow B$, there exists a unique Homomorphism $\sigma : A
\rightarrow B$ such that $\sigma\psi = \tau$.

Effect Handlers (Algebraic Effects \S\ref{sec:effect_handler})

Generators: Operations

Relations: Axioms

Presentation (\S\ref{sec:presentation})

\fist a Weyl Algebra (\S\ref{sec:weyl_algebra}) is Isomorphic to the Quotient
of the Free Algebra on two Generators, $X$ and $Y$, by the Ideal
(\S\ref{sec:ring_ideal}) generated by the Element $YX - XY - 1$



% ------------------------------------------------------------------------------
\subsection{Domain Ring}\label{sec:domain_ring}
% ------------------------------------------------------------------------------

Zero-product Property: a Nonzero Ring in which $ab = 0$ implies $a = 0$ or
$b = 0$

equivalently, a Domain is a Ring in which $0$ is the only Left Zero Divisor (or
equivalently the only Right Zero Divisor)

a Weyl Algebra (\S\ref{sec:weyl_algebra}) is an example of a Non-commutative
Domain Ring



\subsubsection{Integral Domain}\label{sec:integral_domain}

a Commutative Domain

An \emph{Integral Domain}, $R$, is a Non-zero Commutative Ring where if $ab = 0$
in $R$, then either $a = 0$ or $b = 0$ in $R$, i.e. every Non-zero Element $a$
has the \emph{Cancellation Property} that $ab = ac$ implies $b = c$.

Equivalently, the product of any two non-zero elements is non-zero.

generalization of the Ring of Integers

All Finite Integral Domains are \emph{Finite Fields} (\S\ref{sec:finite_field}).

Principal Ideal Domain

\begin{itemize}
  \item Gaussian Integers (\S\ref{sec:gaussian_integer})
  \item ...
\end{itemize}



\paragraph{Fraction Field}\label{sec:fraction_field}\hfill

The \emph{Fraction Field} of an Integral Domain that is the smallest Field in
which it can be \emph{Embedded}, where the Elements of the Fraction Field are
Equivalence Classes written as $\frac{a}{b}$ with $a, b \in R$ and $b \neq 0$.

$\Frac(R)$



\paragraph{Dedekind Domain}\label{sec:dedekind_domain}\hfill

(or \emph{Dedekind Ring}) is an Integral Domain in which every Non-zero Proper
Ideal Factors into a Product of Prime Ideals; such a Factorization can be shown
to be necessarily Unique up to the order of the Factors

since a Field (\S\ref{sec:field}) is a Commutative Ring in which there are no
Non-trivial Proper Ideals, any Field is a Dedekind Domain

\fist Arithmetic Surface (\S\ref{sec:arithmetic_surface}) over a Dedekind Domain



% ------------------------------------------------------------------------------
\subsection{Division Ring}\label{sec:division_ring}
% ------------------------------------------------------------------------------

A \emph{Division Ring} (or \emph{Skew-field}) is a Ring where every Nonzero
Element has a Multiplicative Inverse (but $\cdot$ is not required to be
Commutative as in a Field \S\ref{sec:field}).

the only Ideals (\S\ref{sec:ring_ideal}) of a Division Ring $R$ are $R$ and
$\{0\}$

By \emph{Wedderburn's Little Theorem} all \emph{Finite Division Rings}
are Commutative and therefore \emph{Finite Fields}
(\S\ref{sec:finite_field}).

two-way correspondence between Abstract Projective Geometries
(\S\ref{sec:projective_geometry}) and Skew-fields (Birkhoff, Von Neumann36)



% ------------------------------------------------------------------------------
\subsection{Quotient Ring}\label{sec:quotient_ring}
% ------------------------------------------------------------------------------

% ------------------------------------------------------------------------------
\subsection{Local Ring}\label{sec:local_ring}
% ------------------------------------------------------------------------------

(wiki):

$R$ is a Local Ring if it has one of the following equivalent Properties:
\begin{itemize}
  \item $R$ has a Unique Maximum Left Ideal (\S\ref{sec:maximal_ideal})
  \item $R$ has a Unique Maximum Right Ideal
  \item $1 \neq 0$ and the Sum of any two Non-units in $R$ is a Non-unit
  \item $1 \neq 0$ and if any $x$ is an Element of $R$ then $x$ or $x - 1$ is a
    Unit
  \item if a Finite Sum is a Unit then it has a Term that is a Unit, in
    particular the Empty Sum cannot be a Unit so it implies $1 \neq 0$
\end{itemize}

a Ringed Space where the ``Stalks'' (FIXME: ???) of the Structure Sheaf
(\S\ref{sec:structure_sheaf}) are Local Rings is a Locally Ringed Space
(\S\ref{sec:locally_ringed_space})

\begin{itemize}
  \item the Algebra of Dual Numbers (\S\ref{sec:dual_number}) is a Ring that is
    a Local Ring since the Principal Ideal generated by $\varepsilon$ is its
    only Maximal Ideal
\end{itemize}


% ------------------------------------------------------------------------------
\subsection{Graded Ring}\label{sec:graded_ring}
% ------------------------------------------------------------------------------

Ring that is the Direct Sum (the \emph{Gradation} or \emph{Grading}) of Abelian
Groups $R_i$ such that $R_iR_j \subseteq R_{i+j}$

Graded Algebra (\S\ref{sec:graded_algebra}),
Differential Graded Algebra (\S\ref{sec:differential_graded_algebra}),
Graded Derivation (\S\ref{sec:graded_derivation})

Graded Vector Space (\S\ref{sec:graded_vectorspace})



% ------------------------------------------------------------------------------
\subsection{Ring Completion}\label{sec:ring_completion}
% ------------------------------------------------------------------------------

% ------------------------------------------------------------------------------
\subsection{Involutive Ring}\label{sec:involutive_ring}
% ------------------------------------------------------------------------------

% ------------------------------------------------------------------------------
\subsection{Involutive Algebra}\label{sec:involutive_algebra}
% ------------------------------------------------------------------------------

\emph{Involutive Algebra} (or \emph{*-algebra})



% ------------------------------------------------------------------------------
\subsection{Topological Ring}\label{sec:topological_ring}
% ------------------------------------------------------------------------------



% ==============================================================================
\section{Field Theory}\label{sec:field_theory}
% ==============================================================================

% ------------------------------------------------------------------------------
\subsection{Field}\label{sec:field}
% ------------------------------------------------------------------------------

A \emph{Field} is a Nonzero Commutative Ring (\S\ref{sec:commutative_ring}) with
a Multiplicative Inverse for every Nonzero Element or equivalently a Ring whose
Nonzero Elements form an Abelian Group (\S\ref{sec:commutative_group}) under
Multiplication.

since a Field is a Commutative Ring in which there are no Non-trivial Proper
Ideals, any Field is a \emph{Dedekind Domain} (\S\ref{sec:dedekind_domain})

cf. Division Algebras (\S\ref{sec:division_algebra}) which are not required to
be Commutative

\emph{Field Axioms}:
\begin{itemize}
  \item \emph{Associativity}
  \item \emph{Commutativity}
  \item \emph{Additive Identity}
  \item \emph{Multiplicative Identity}
  \item \emph{Additive Inverse}
  \item \emph{Multiplicative Inverse}
  \item \emph{Distributivity}
\end{itemize}

Functions in a Function Space (\S\ref{sec:function_space}) $X
\rightarrow F$ into a Field $F$ have a Vector (\S\ref{sec:vector})
structure with two Pointwise Addition Operators and Scalar
Multiplication

\url{https://johncarlosbaez.wordpress.com/2018/03/03/nonstandard-integers-as-complex-numbers/}:

any two Algebraically Closed Fields (i.e. Fields containing all Roots of
Polynomials) of Characteristic Zero (\S\ref{sec:ring_characteristic}) that have
the same Uncountable Cardinality must be Isomorphic

it follows that any Field of Characteristic Zero whose Cardinality is less than
that of the Continuum is \emph{Isomorphic} to some Sub-field of the Complex
Numbers

the only Ideals (\S\ref{sec:ring_ideal}) of a Field $F$ are $F$ and $\{0\}$

\fist $\mathbb{F}_1$ -- ``Field of Characteristic $1$''; $\mathbb{F}_1$-geometry
(\S\ref{sec:f1_geometry})



\subsubsection{Total Field}\label{sec:total_field}

\subsubsection{Closed Field}\label{sec:closed_field}

\paragraph{Real Closed Field}\label{sec:real_closed}\hfill

exactly the Real Closed Rings (\S\ref{sec:real_closed_ring}) that are Fields

Elementary Theory (\S\ref{sec:elementary_theory})

Not Algebraically Closed (\S\ref{sec:algebraically_closed})

Field Extension $F(\sqrt{-1})$ is Algebraically Closed

(wiki):

Tarski-Seidenberg Theorem (\S\ref{sec:tarski_seidenberg}): the First-order
Theory of the Real Field is Decidable using Quantifier Elinimation; Euclidean
Geometry without the ability to measure Angles is also a Model of the Real
Field Axioms and therefore also Decidable

the Class of Real Closed Rings (\S\ref{sec:real_closed_ring}) is First-order
Axiomatizable (TODO) and Undecidable

the Class of all Real Closed Valuation Rings (TODO) is Decidable

a Real Closed Field with its unique Ordering is a \emph{Maximally Ordered
  Field} (\S\ref{sec:maximally_ordered_field})

if $F$ is a Field with no assumed Ordering, then $F$ has a Real Closure which
may not be Field but is just a Real Closed Ring

Cardinality (\S\ref{sec:cardinal_number}) of $F$

Cofinality (\S\ref{sec:cofinality}) of $F$

Density (\S\ref{sec:density}) of $F$

if the General Continuum Hypothesis (\S\ref{sec:generalized_continuum}) holds
then all Real Closed Fields (\S\ref{sec:real_closed}) with Cardinality of the
Continuum and having the $\eta_1$ (\S\ref{sec:eta_set}) Property are Order
Isomorphic (Isomorphic as Posets \S\ref{sec:order_isomorphism});
this unique Field $F$ can be defined by an Ultrapower (\S\ref{sec:ultrapower})
as $\reals^\nats / \mathsf{M}$ where $\mathsf{M}$ is a Maximal Ideal
(\S\ref{sec:maximal_ideal}) \emph{not} resulting in a Field that is
Order-isomorphic to $\reals$-- this is the most commonly used Hyperreal Number
(\S\ref{sec:hyperreal}) Field used in Non-standard Analysis
(\S\ref{sec:nonstandard_analysis})-- the uniqueness of this Field is equivalent
to the Continuum Hypothesis-- not a \emph{Complete Field} (???) %FIXME

$F$ can also be constructed (more ``Constructively'') as the Subfield of Series
with a Countable number of Nonzero Terms on the Field $\reals((G))$ of the
Formal Power Series on a Totally Ordered Abelian Divisible Group $G$ that is an
$\eta_1$ Group of Cardinality $\aleph_1$ (Alling62)-- but $F$ is not a
\emph{Complete Field} and taking its Completion yields a Field $K$ of
Cardinality Larger thatn $F$ (FIXME: clarify)

without the Continuum Hypothesis, if the Cardinality of the Continuum is
$\aleph_\beta$, then there is a Unique $\eta_\beta$ Field of Size $\eta_beta$



\subparagraph{Artin-Schreier Theorem}\label{sec:artin_schreier}\hfill

an Ordered Field (\S\ref{sec:ordered_field}) $F$ has an Algebraic Extension
(\S\ref{sec:algebraic_extension}) called the \emph{Real Closure} $K$ of $F$
such that $K$ is a Real Closed Field with Ordering as an Extension of the given
Ordering on $F$ and is Unique up to Unique Isomorphism of Fields identical to
$F$

the Real Closure of the Ordered Field of Rational Numbers is the Field
$\algs$ of Real Algebraic Numbers (\S\ref{sec:algebraic_number})



\paragraph{Algebraically Closed Field}\label{sec:algebraically_closed}
\hfill

(wiki):

an \emph{Algebraically Closed Field} $F$ contains a Root
(\S\ref{sec:function_root}) for every Non-constant Polynomial in the Ring of
Polynomials (\S\ref{sec:polynomial_ring}) $F[x]$ in the Variable $x$ with
Coefficients in $F$

The Field Extension (\S\ref{sec:field_extension}) $F(\sqrt{-1})$ of a
Real Closed Field (\S\ref{sec:real_closed}) is Algebraically Closed.

the Fundamental Theorem of Algebra (\S\ref{sec:fundamental_algebra_theorem})
states that the Field of Complex Numbers is Algebraically Closed
(\S\ref{sec:algebraically_closed})

\url{https://johncarlosbaez.wordpress.com/2018/03/03/nonstandard-integers-as-complex-numbers/}:

any two Algebraically Closed Fields (i.e. Fields containing the Roots of all
Polynomials) of Characteristic Zero (\S\ref{sec:ring_characteristic}) that have
the same Uncountable Cardinality must be Isomorphic

any two Models of the Axioms for an Algebraically Closed Field with the same
Cardinality must be Isomorphic

any Algebraically Closed Field with Cardinality equal to that of the Continuum
is Isomorphic to the Complex Numbers



\subsubsection{Finite Field}\label{sec:finite_field}

A \emph{Finite Field} (or \emph{Galois field)} is a Field that contains a
finite number of Elements with the \emph{Order} being equal to the number of
Elements.

A Finite Field of Order $q$ exists if and only if $q$ is a Prime Power
(\S\ref{sec:prime_power}) $p^k$ for some Prime $p$ and Positive Integer $k$.
All Finite Fields of a given Order are \emph{Isomorphic}.

Every Finite Field is an example of a Finite Ring (\S\ref{sec:finite_ring}).

any Finite Simple Ring (\S\ref{sec:simple_ring}) is Isomorphic to the Ring
$M_n(GF(q))$ of $n \times n$ Matrices (\S\ref{sec:matrix_ring}) over a Finite
Field of Order $q$

All Finite Integral Domains (\S\ref{sec:integral_domain}) are Finite Fields

no Finite Field is Algebraically Closed

Weil Conjectures (\S\ref{sec:weil_conjectures}): Propositions on the Generating
Functions (\S\ref{sec:generating_function}) derived from counting the number of
Points on Algebraic Varieties (\S\ref{sec:algebraic_variety}) over Finite Fields

\begin{itemize}
  \item $GF(2)$ -- Binary Field (\S\ref{sec:binary_field})
  \item $GF(3)$
  \item $GF(4)$
  \item ...
\end{itemize}

\fist $\mathbb{F}_1$ -- ``Field of Characteristic $1$''; $\mathbb{F}_1$-geometry
(\S\ref{sec:f1_geometry})



\paragraph{Primitive Element}\label{sec:primitive_element}\hfill

a (Group) Generator (\S\ref{sec:group_generator}) of the Multiplicative Group of
the Field

the number of Primitive Elements of a Finite Field $GF(q)$ is $\varphi(q - 1)$
where $\varphi$ is Euler's Totient Function (\S\ref{sec:eulers_totient})-- only
for $q=2,3$ is the Primitive Element unique

implies that $x^q = x$ for every $x$ in $GF(q)$; when $q$ is Prime this is
Fermat's Little Theorem

if $a$ is a Primitive Element in $GF(q)$, then for any Non-zero Element $x$ in
$F$, there is a unique Integer $n$ with $0 \leq n \leq q - 2$ such that
$x = a^n$ where $n$ is called the \emph{Discrete Logarithm}
(\S\ref{sec:discrete_logarithm}) of $x$ to the Base $a$



\paragraph{Binary Field}\label{sec:binary_field}\hfill

$GF(2)$

$0$, $1$

$+$ -- Logical XOR

$\times$ -- Logical AND

Characteristic $2$

Vector Spaces over $GF(2)$ -- Bit Strings (``Machine Words''); \fist Nimbers
(\S\ref{sec:nimber})



\subsubsection{Ordered Field}\label{sec:ordered_field}

An \emph{Ordered Field} is a Field with a Total Ordering
(\S\ref{sec:total_order}) compatible with the Field Operations.

Axiomatic definition of the Reals: the Real Numbers are the Unique up to
Isomorphism Dedekind-complete (Least-upper-bound Property
\S\ref{sec:least_upperbound}) Ordered Field $(\reals, +, *, <)$

the Real Closure (\S\ref{sec:real_closed_ring}) of an Ordered Field
(\S\ref{sec:ordered_field}) is not in general the Real Closure of the
underlying Field: the Real Closure of a Field $F$ is a certain ``Subdirect
Product'' of the Real Closures of the Ordered Fields $(F,P)$ where $P$ ``runs
through'' the Orderings of $F$ (FIXME: clarify)

\emph{Artin-Schreier Theorem} (\S\ref{sec:artin_schreier}): an Ordered Field
$F$ has an Algebraic Extension (\S\ref{sec:algebraic_extension}) called the
\emph{Real Closure} $K$ of $F$ such that $K$ is a Real Closed Field with
Ordering as an Extension of the given Ordering on $F$ and is Unique up to
Unique Isomorphism of Fields identical to $F$

the Real Closure of the Ordered Field of Rational Numbers is the Field $\algs$
of Real Algebraic Numbers (\S\ref{sec:algebraic_number})



\paragraph{Archimedean Field}\label{sec:archimedean_field}\hfill

an Ordered Fieldin which every Element is Bounded Above by a Natural Number

Archimedean Property (\S\ref{sec:archimedean_property})

the Real Numbers



\paragraph{Maximally Ordered Field}\label{sec:maximally_ordered_field}\hfill

a Real Closed Field (\S\ref{sec:real_closed}) with its unique Ordering is a
Maximally Ordered Field



\paragraph{Non-archimedean Field}\label{sec:nonarchimedean_field}\hfill

e.g. $p$-adic Numbers (\S\ref{sec:padic_number})

\fist Non-archimedean Analytic Geometry
(\S\ref{sec:nonarchimedean_analytic_geometry})



% ------------------------------------------------------------------------------
\subsection{Field Extension}\label{sec:field_extension}
% ------------------------------------------------------------------------------

$L/K$ -- ``$L$ over $K$''; a Pair of a Subfield $K$ and an Extension Field $L$

\emph{note}: the notation $L/K$ is purely formal and \emph{does not} imply
Quotient or any other kind of Division

given a Field Extension $L/K$, $L$ is a $K$-vector Space
(\S\ref{sec:vector_space}) of Dimension $[L:K]$ called the \emph{Degree} of
the Extension

the Degree of an Extension is $1$ if and only if the two Fields are equal
(Trivial Extension)

Extensions of Degree $2$ are called \emph{Quadratic Extensions} and Extensions
of Degree $3$ are called \emph{Cubic Extensions}

if $D$ is Connected, the Meromorphic Functions
(\S\ref{sec:meromorphic_function}) on $D$ form a Field Extension of the Complex
Numbers



\subsubsection{Subfield}\label{sec:subfield}

\subsubsection{Extension Field}\label{sec:extension_field}

$\reals$ is an Extension Field of $\rats$

$\comps$ is an Extension Field of $\reals$



\subsubsection{Tower}\label{sec:tower}

example $\rats \subseteq \reals \subseteq \comps$



\subsubsection{Algebraic Extension}\label{sec:algebraic_extension}

\emph{Artin-Schreier Theorem} (\S\ref{sec:artin_schreier}): an Ordered Field
(\S\ref{sec:ordered_field}) $F$ has an Algebraic Extension called the
\emph{Real Closure} $K$ of $F$ such that $K$ is a Real Closed Field with
Ordering as an Extension of the given Ordering on $F$ and is Unique up to
Unique Isomorphism of Fields identical to $F$

the Real Closure of the Ordered Field of Rational Numbers is the Field $\algs$
of Real Algebraic Numbers (\S\ref{sec:algebraic_number})



\subsubsection{Transcendental Extension}
\label{sec:transcendental_extension}



% ------------------------------------------------------------------------------
\subsection{Topological Field}\label{sec:topological_field}
% ------------------------------------------------------------------------------

$\reals$ and $\comps$ are the only Connected Locally Compact
Topological Fields



% ==============================================================================
\section{Galois Theory}\label{sec:galois_theory}
% ==============================================================================

Finite Field (\S\ref{sec:finite_field}) or \emph{Galois Field}

Differentiable Galois Theory (\S\ref{sec:differential_galois})

uses Permutation Groups for analysing the Discrete Symmetries of
Algebraic Equations %FIXME

cf. Lie Theory (\S\ref{sec:lie_theory}) uses Lie Groups
(\S\ref{sec:lie_group}) for analysing the Continuous Symmetries of
Differential Equations

gives a characterization of the Ratios of Lengths that can be constructed with
Compass \& Straightedge-- e.g. which Regular Polygons are Constructible
Polygons?, why is it not possible to Trisect every Angle using a Compass and
Straightedge



% ------------------------------------------------------------------------------
\subsubsection{Galois Group}\label{sec:galois_group}
% ------------------------------------------------------------------------------

specific Group associated with a Field Extension %FIXME

Permutation Groups (\S\ref{sec:permutation_group})



% ==============================================================================
\section{Commutative Algebra}\label{sec:commutative_algebra}
% ==============================================================================

study of Commutative Rings (\S\ref{sec:commutative_ring}), their Ideals
(\S\ref{sec:ring_ideal}), and Modules (\S\ref{sec:module}) over Commutative
Rings

Polynomial Ring (Free Commutative Ring \S\ref{sec:polynomial_ring})

three closely related fields:
\begin{itemize}
  \item Algebraic Number Theory (\S\ref{sec:algebraic_number_theory})
  \item Algebraic Geometry (Part \ref{part:algebraic_geometry})
  \item Commutative Algebra
\end{itemize}

\fist $R$-algebra (Algebra over a Commutative Unital Ring \S\ref{sec:r_algebra})
-- a Module with a Bilinear Product (\S\ref{sec:bilinear_product})

\fist cf. Commutative Algebraic Theory
(\S\ref{sec:commutative_algebraic_theory})

\fist Symmetric Polynomials (\S\ref{sec:symmetric_polynomial})

\fist Differential Algebra (\S\ref{sec:differential_algebra})

\begin{itemize}
  \item the Dual Numbers (\S\ref{sec:dual_numbers}) form a Two-dimensional
    Commutative, Unital, Associative (\S\ref{sec:associative_algebra}) Algebra
    over the Real Numbers
\end{itemize}



% ------------------------------------------------------------------------------
\subsection{Local Algebra}\label{sec:local_algebra}
% ------------------------------------------------------------------------------

Local Rings (\S\ref{sec:local_ring}) and their Modules (\S\ref{sec:module})



% ==============================================================================
\section{Noncommutative Algebra}\label{sec:noncommutative_algebra}
% ==============================================================================

Non-commutative Rings (\S\ref{sec:noncommutative_ring}),
Free Algebra (Free Non-commutative Ring \S\ref{sec:free_algebra})

Noncommutative Geometry (\S\ref{sec:noncommutative_geometry})



% ==============================================================================
\section{Homological Algebra}\label{sec:homological_algebra}
% ==============================================================================

(wiki):

Homology (\S\ref{sec:homology_theory}) in the ``general abstract'' setting

Abelian Category (\S\ref{sec:abelian_category})

(ncat):

Chain Complex (\S\ref{sec:chain_complex})

Differential Graded Ring (\S\ref{sec:differential_graded_ring})

Differential Graded Module (\S\ref{sec:differential_graded_module})



% ------------------------------------------------------------------------------
\subsection{Chain Complex}\label{sec:chain_complex}
% ------------------------------------------------------------------------------

(wiki):

A \emph{Chain Complex} $(A_\bullet, d_\bullet)$ is a Sequence of Commutative
(Abelian) Groups (\S\ref{sec:commutative_group}) or Modules
(\S\ref{sec:module}):
\[
  \ldots, A_0, A_1, A_2, A_3, \ldots
\]
connected by Homomorphisms called \emph{Boundary Operators} (or
\emph{Differentials} \S\ref{sec:boundary_operator}) $d_n : A_n \rightarrow
A_{n-1}$ such that the Composition of any two consecutive Differentials is
the Zero Map:
\[
  d_n \circ d_{n+1} = 0
\]
or written without indices:
\[
  d^2 = 0
\]

Algebraic Structure consisting of a Sequence of Abelian Groups or Modules and a
Sequence of Homomorphisms called \emph{Boundary Operators}
(\S\ref{sec:boundary_operator}) between consecutive Groups (Modules) such that
the Image of each Homomorphism is included in the Kernel of the next

\fist cf. Commutative (Abelian) Group (\S\ref{sec:commutative_group}),
Topological Spectrum (Homotopical Algebra \S\ref{sec:topological_spectrum})

\fist Homology (\S\ref{sec:homology}): describes how Images are included in
Kernels

\begin{itemize}
  \item in $\cat{C} = \cat{Vect}_k$, a Chain Complex is called a
    \emph{Differential Graded Vector Space} (\S\ref{sec:differential_graded})
\end{itemize}

\fist for Additive Category (\S\ref{sec:additive_category}) $\cat{A}$, the
Homotopy Category of Chain Complexes (\S\ref{sec:chain_homotopy_category})
$\cat{K}(\cat{A})$ has as Objects Chain Complexes as in $\cat{Kom}(\cat{A})$
(FIXME: clarify) and as Morphisms maps of Complexes modulo Homotopy



\subsubsection{Boundary Operator}\label{sec:boundary_operator}

or \emph{Differential}

(nlab):

A \emph{Boundary Operator} (or \emph{Differential}) is a Morphism of a
Differential Object (\S\ref{sec:differential_object}) defining a Chain Complex.
Concretely, a Boundary Operator on a Chain Complex is called a ``Differential''
if it is part of the structure of a Differential Graded Algebra
(\S\ref{sec:differential_graded_algebra}) on the Complex.
(FIXME: clarify)



\subsubsection{Chain Map}\label{sec:chain_map}

A \emph{Chain Map} $f$ on Chain Complexes $(A_\bullet, d_{A,\bullet})$ and
$(B_\bullet, d_{B,\bullet})$ is a Sequence $f_\bullet$ of Homomorphisms $f_n :
A_n \rightarrow B_n$ for each $n$ that Commutes with the Boundary Operators of
the two Chain Complexes, so:
\[
  d_{B,n} \circ f_n = f_{n-1} \circ d_{A,n}
\]

induces a map on Homology (TODO)

Smooth Maps between Manifolds induce Chain Maps and Smooth Homotopies between
Maps induce \emph{Chain Homotopies} (\S\ref{sec:chain_homotopy})



\subsubsection{Bounded Chain Complex}\label{sec:bounded_chain_complex}

almost all $A_n$ are $0$, i.e. a Finite Complex extended to the left and right
by $0$

an example of a Bounded Chain Complex is the Chain Complex defining the
Simplicial Homology (\S\ref{sec:simplicial_homology}) of a Finite Simplicial
Complex



\subsubsection{Singular Chain Complex}\label{sec:singular_chain_complex}

(wiki):

the \emph{Singular Chain Complex} of a Topological Space $X$ is constructed
using Continuous Maps from a Simplex (\S\ref{sec:simplex}) to $X$, and the
Homomorphisms of the Chain Complex capture how the maps ``restrict to the
boundary of the Simplex''; the Homology of this Chain Complex is called the
Singular Homology (\S\ref{sec:singular_homology}) of $X$ (FIXME: clarify)

the Singular Homology is constructed by taking Maps of the standard $n$-simplex
to a Topological Space and composing them into Formal
Sums (\S\ref{sec:formal_sum}) called Singular Chains and the Boundary Operation
(\S\ref{sec:boundary_operator}) maps each $n$-dimensional Simplex to its
$(n-1)$-dimensional Boundary inducing the \emph{Singular Chain Complex}



% ------------------------------------------------------------------------------
\subsection{Cochain Complex}\label{sec:cochain_complex}
% ------------------------------------------------------------------------------

\fist Cohomology (\S\ref{sec:cohomology})

\emph{Co-chain Complex} $(A^\bullet, d^\bullet)$



\subsubsection{de Rham Complex}\label{sec:derham_complex}

(wiki):

The \emph{de Rham Complex} is the Cochain Complex of Differential Forms
(\S\ref{sec:differential_form}) on some Smooth Manifold
(\S\ref{sec:smooth_manifold}) $M$ with the Exterior Derivative
(\S\ref{sec:exterior_derivative}) as the Differential
(\S\ref{sec:differential}):
\[
  0 \rightarrow \Omega^0(M) \xrightarrow{d} \Omega^1(M) \xrightarrow{d}
  \Omega^2(M) \xrightarrow{d} \Omega^3(M) \rightarrow \cdots
\]
where $\Omega^0(M)$ is the Space of Smooth Functions on $M$, $\Omega^1(M)$ is
the Space of $1$-forms, etc.; Differential Forms which are the Image of other
Forms under the Exterior Derivative plus the Constant $0$ Function in
$\Omega^0(M)$ are called \emph{Exact} (\S\ref{sec:exact_differential_form}) and
Differential Forms whose Exterior Derivative is $0$ are called \emph{Closed}
(\S\ref{sec:closed_differential_form}); the relationship $d^2 = 0$ says that
Exact Forms are Closed (FIXME: clarify), but the converse is not generally true
(Closed Differential Forms need not be Exact, e.g. for the $1$-form of Angle
Measurement on the Unit Circle $\diffy{\theta}$ there is no actual Function
$\theta$ defined on the whole Circle of which $\diffy{\theta}$ is the
Derivative)

Coboundary Map (i.e. the ``Differential'') in the de Rham Complex is the de Rham
Differential (\S\ref{sec:derham_differential}), ``Exterior Derivative'' acting
on Differential Forms

\emph{de Rham Cohomology}

two Closed Differential Forms $\alpha, \beta \in \Omega^k(M)$ are
\emph{Cohomologous} if they differ by an Exact Differential Form, i.e. $\alpha -
\beta$ is Exact; this induces an Equivalence Relation on the Space of Closed
Forms in $\Omega^k(M)$ and the $k$-th \emph{de Rham Cohomology Group}
$H_{\mathrm{dR}}^k(M)$ is the Set of Equivalence Classes (i.e. the Set of Closed
Forms in $\Omega^k(M)$ modulo the Exact Forms)

for any Manifold $M$ with $n$ Connected Components (Maximal Connected Subset
Ordered by Inclusion \S\ref{sec:connected_space}), the $0$-th de Rham Cohomology
Group (\S\ref{sec:derham_complex}) $H_{\mathrm{dR}}^0(M)$ is Isomorphic to
$\reals^n$:
\[
  H_{\mathrm{dR}}^0(M) \cong \reals^n
\]
following from the fact that any Smooth Function on $M$ with Zero Derivative
(Locally Constant) is Constant on each of the Connected Components of $M$

the Homology Group in Dimension Zero is Isomorphic to the Vector Space of
Locally Constant Functions from $M$ to $\reals$; for a Compact Manifold this is
the Real Vector Space with Dimension equal to the number f Connected Components
of $M$

Smooth Maps between Manifolds induce \emph{Chain Maps} (\S\ref{sec:chain_map})
and Smooth Homotopies between Maps induce \emph{Chain Homotopies}
(\S\ref{sec:chain_homotopy})


(nlab):

the archetypical example of \emph{Differential} (\S\ref{sec:differential}) is
the Differential in the de Rham Complex $\Omega^\bullet(X)$ of a Smooth manifold
$X$, give by Differentiation (\S\ref{sec:derivative}) of Smooth Functions and
Differential Forms



\subsubsection{de Rham Differential}\label{sec:derham_differential}

Coboundary Map (i.e. the ``Differential'') in the de Rham Complex is the de Rham
Differential, ``Exterior Derivative'' acting on Differential Forms



% ------------------------------------------------------------------------------
\subsection{Differential Algebra}\label{sec:differential_algebra}
% ------------------------------------------------------------------------------

generalization of Commutative Algebra (\S\ref{sec:commutative_algebra})

cf. Differential Calculus (\S\ref{sec:differential_calculus}), Differential
Geometry (\S\ref{sec:differential_geometry}), Differential Topology
(\S\ref{sec:differential_topology})

Birings (\S\ref{sec:biring})

2011 - Hubbard, Lundell - \emph{A First Look at Differential Algebra}

Ovchinnikov - CUNY Math 86600 lecture notes -
\url{http://qcpages.qc.cuny.edu/~aovchinnikov/MATH86600/notes.pdf}

the Exterior Derivative (\S\ref{sec:exterior_derivative}) gives the Exterior
Algebra (\S\ref{sec:exterior_algebra}) of Differential Forms
(\S\ref{sec:differential_form}) on a Manifold the structure of a Differential
Algebra



\subsubsection{Derivation}\label{sec:derivation}

(wiki):

generalizes features of the Derivative Operator (\S\ref{sec:derivative})

For an Algebra $A$ over a Ring or a Field $K$, a \emph{$K$-derivation} is a
$K$-linear Map (\S\ref{sec:linear_transformation}) $D : A \rightarrow A$
satisfying Leibniz's Law (Product Rule \S\ref{sec:product_rule}):
\[
  D(ab) = D(a)b + aD(b)
\]
Generally, if $M$ is an $A$-bimodule, then a $K$-linear Map $D : A \rightarrow
M$ that Satisfies the Leibniz Law s also called a ``\emph{Derivation}'', and the
collection of all $K$-derivations of $A$ to itself is denoted
$\mathrm{Der}_K(A)$, and the collection of $K$-derivations of $A$ into an
$A$-module $M$ is denoted $\mathrm{Der}_K(A, M)$.

Product Law (\S\ref{sec:product_rule}) generalized to Differential Forms
(\S\ref{sec:differential_form}): the Product Law states that $\mathrm{d}$ is a
\emph{Derivation} of Degree $+1$ on the Graded Commutative Algebra
(\S\ref{sec:differential_graded_algebra}) of Differential Forms:
\[
\diffy{f \wedge g} = (\diffy{f}) \wedge g +
  (-1)^{\mathrm{deg}\ f}f \wedge (\diffy{g})
\]
(FIXME: clarify derivation degree)

examples:
\begin{itemize}
  \item the Partial Derivative (\S\ref{sec:partial_derivative}) with respect to
    a Variable is a $\reals$-derivation on the Algebra of Real-valued
    Differentiable Functions on $\reals^n$
  \item Lie Derivative
  \item ... MORE
\end{itemize}



\paragraph{Graded Derivation}\label{sec:graded_derivation}\hfill

\paragraph{Augmented Derivation}\label{sec:augmented_derivation}\hfill

An \emph{Augmented Derivation} is a Module Homomorphism (Linear Map
\S\ref{sec:linear_transformation}), $D : A \rightarrow B$, ``\emph{Augmented}''
by an Algebra Homomorphism (\S\ref{sec:algebra_homomorphism})
$\epsilon : A \rightarrow B$ such that:
\[
  D(\vec{a} \times \vec{b}) =
  D(\vec{a})\epsilon(\vec{b}) + \epsilon(\vec{a})D(\vec{b})
\]

\fist \emph{Differentiation} (\S\ref{sec:derivative}) of Smooth Functions can be
defined as an \emph{(Endo-)functor} on the Category of Smooth Manifolds and
Smooth Maps:
\[
  \mathrm{d} : \cat{Diff} \rightarrow \cat{Diff}
\]
sending:
\begin{itemize}
  \item Smooth Manifolds $X$ to Tangent Bundles (\S\ref{sec:tangent_bundle}) $T
    X$, with Points of $T X$ being Ordered Pairs $(x, v)$ where $v$ is a Tangent
    Vector at $x$, i.e. an (Augmented) Derivation $v : C^\infty(X) \rightarrow
    \reals$ on the Algebra of Smooth Functions, Augmented by Evaluation
    $\mathrm{ev}_x : C^\infty(X) \rightarrow \reals$ at $x$ (FIXME: clarify)
  \item Smooth Functions $f : X \rightarrow Y$ to Derivatives (FIXME: nlab calls
    this here a ``Differential'' of $f$ but that seems to be inconsistent with
    the terminology used elsewhere) $\diffy{f} : TX \rightarrow TY$; if $\gamma
    : [-1,1] \rightarrow X$ is a Path in $X$ representing a Vector
    $v \in T_x{X}$ then $(\diffy{f})(v) \in T_{f(x)}Y$ is the Vector represented
    by the Path $[-1, 1] \xrightarrow{\gamma} X \xrightarrow{f} Y$
    (FIXME: clarify)
\end{itemize}



\paragraph{Antiderivation}\label{sec:antiderivation}\hfill

cf. Antiderivative (Indefinite Integral \S\ref{sec:antiderivative})

examples:
\begin{itemize}
  \item Exterior Derivative (\S\ref{sec:exterior_derivative})
  \item Interior Product (Interior Derivative \S\ref{sec:interior_product})
\end{itemize}



\subsubsection{Differential Ring}\label{sec:differential_ring}

A \emph{Differential Ring} is a Unital Ring with a Finite Set of
\emph{Derivations} $\Delta = \{ \partial_1, \ldots, \partial_2 \}$ that are
Homomorphisms of Additive Groups: (FIXME: why additive groups ???)
\[
  \partial : R \rightarrow R
\]
such that each Derivation $\partial$ Satisfies the Leibniz Product Rule:
\[
  \partial(r_1r_2) = (\partial{r_1})r_2 + r_1(\partial{r_2})
\]
for every $r_1, r_2 \in R$


any Ring is a Differential Ring with respect to the Trivial Derivation mapping
any Ring Element to Zero



\paragraph{Differential Graded Ring}\label{sec:differential_graded_ring}\hfill

cf. Ring (\S\ref{sec:ring}), Ring Spectrum (Homotopical Algebra
\S\ref{sec:ring_spectrum})



\subsubsection{Differential Field}\label{sec:differential_field}

\subsubsection{Differential Graded Module}\label{sec:differential_graded_module}

cf. Module (\S\ref{sec:module}), Module Spectrum (Homotopical Algebra
\S\ref{sec:module_spectrum})



\subsubsection{Differential Graded Algebra}
\label{sec:differential_graded_algebra}

Graded Algebra (\S\ref{sec:graded_algebra})

a Boundary Operator (\S\ref{sec:boundary_operator}) on a Chain Complex
(\S\ref{sec:chain_complex}) is called a ``Differential''
(\S\ref{sec:differential}) if it is part of the structure of a Differential
Graded Algebra on the Complex

Under the Wedge (Exterior) Product (\S\ref{sec:exterior_product}) the de Rham
Complex (\S\ref{sec:derham_complex}) becomes a Differential Graded Algebra
(\S\ref{sec:differential_graded_algebra})

Product Law (\S\ref{sec:product_rule}) generalized to Differential Forms
(\S\ref{sec:differential_form}): the Product Law states that $\mathrm{d}$ is a
\emph{Derivation} (\S\ref{sec:derivation}) of Degree $+1$ on the Graded
Commutative Algebra of Differential Forms:
\[
\diffy{f \wedge g} = (\diffy{f}) \wedge g +
  (-1)^{\mathrm{deg}\ f}f \wedge (\diffy{g})
\]
(FIXME: clarify derivation degree)



\subsubsection{Differential Graded Scheme}\label{sec:differential_graded_scheme}

\subsubsection{Differential Graded Category}
\label{sec:differential_graded_category}

\subsubsection{Differential Galois Theory}\label{sec:differential_galois}

Galois Theory (\S\ref{sec:galois_theory})

Extensions of Differential Fields

Galois Groups (\S\ref{sec:galois_group}) of Differential Equations



% ==============================================================================
\section{Homotopical Algebra}\label{sec:homotopical_algebra}
% ==============================================================================

or \emph{Higher Algebra}

cf. Higher Category Theory (\S\ref{sec:higher_category})

Topological Spectrum (\S\ref{sec:topological_spectrum})

Ring Spectrum (\S\ref{sec:ring_spectrum})

Module Spectrum (\S\ref{sec:module_spectrum})



% ------------------------------------------------------------------------------
\subsection{Topological Spectrum}\label{sec:topological_spectrum}
% ------------------------------------------------------------------------------

%FIXME: xrefs

Stable Homotopy Theory (\S\ref{sec:stable_homotopy}), Stable Homotopy Category
(\S\ref{sec:stable_homotopy_category})

cf. Commutative (Abelian) Group (\S\ref{sec:commutative_group}), Chain Complex
(Homological Algebra \S\ref{sec:chain_complex})

Symmetric Spectrum, Simplicial Spectrum



% ------------------------------------------------------------------------------
\subsection{Module Spectrum}\label{sec:module_spectrum}
% ------------------------------------------------------------------------------

a Module over a Ring Spectrum (\S\ref{sec:ring_spectrum})

cf. Differential Graded Module (Homological Algebra
\S\ref{sec:differential_graded_module})



% ------------------------------------------------------------------------------
\subsection{Derivator}\label{sec:derivator}
% ------------------------------------------------------------------------------

Derived Categories (\S\ref{sec:derived_category})

Language for Homotopical Algebra



% ==============================================================================
\section{Invariant Theory}\label{sec:invariant_theory}
% ==============================================================================

Actions of Groups (\S\ref{sec:group_action}) on Algebraic Varieties
(\S\ref{sec:algebraic_variety})

Invariant (\S\ref{sec:invariant})
