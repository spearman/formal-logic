%%%%%%%%%%%%%%%%%%%%%%%%%%%%%%%%%%%%%%%%%%%%%%%%%%%%%%%%%%%%%%%%%%%%%%
%%%%%%%%%%%%%%%%%%%%%%%%%%%%%%%%%%%%%%%%%%%%%%%%%%%%%%%%%%%%%%%%%%%%%%
\part{Abstract Algebra}\label{sec:abstract_algebra}
%%%%%%%%%%%%%%%%%%%%%%%%%%%%%%%%%%%%%%%%%%%%%%%%%%%%%%%%%%%%%%%%%%%%%%
%%%%%%%%%%%%%%%%%%%%%%%%%%%%%%%%%%%%%%%%%%%%%%%%%%%%%%%%%%%%%%%%%%%%%%

% ====================================================================
\section{Magma}\label{sec:magma}
% ====================================================================

A \emph{Magma}, $M$, is an Algebraic Structure
(\S\ref{sec:universal_algebra}) with a single Closed Binary
Operation, $M \times M \rightarrow M$.



% ====================================================================
\section{Semigroup}\label{sec:semigroup}
% ====================================================================

A \emph{Semigroup} is Magma (\S\ref{sec:magma}) with an Associative
Binary Operation. A Semigroup is differentiated from a Monoid
(\S\ref{sec:monoid}) by not requiring an Identity Element, and from
a Group (\S\ref{sec:group}) by not requiring Inverses.



% --------------------------------------------------------------------
\subsection{Monoid}\label{sec:monoid}
% --------------------------------------------------------------------

A \emph{Monoid} is a Semigroup with an Identity Element. The set of
all Endomorphisms of an Object, $X$, in a Category, $C$,
\[
    Hom(X,X)
\]
defines a Monoid and is denoted $End_C(X)$.



\subsubsection{Free Monoid}\label{sec:free_monoid}

A \emph{Free Monoid} on a Set $A$ is the Monoid $A^*$ whose Elements
are all possible Finite Sequences of zero or more Elements of $A$ with
\emph{String Concatenation} as the Monoid Operation and the Empty
String $\varepsilon$ as the Identity Element.



\subsubsection{Trace Monoid}\label{sec:trace_monoid}

\subsubsection{Syntactic Monoid}\label{sec:syntactic_monoid}

\emph{Syntactic Quotient}

\emph{Syntactic Relation} (\emph{Syntactic Equivalence})



% --------------------------------------------------------------------
\subsection{Transformation Semigroup}\label{sec:transformation_semigroup}
% --------------------------------------------------------------------



% ====================================================================
\section{Group Theory}\label{sec:group_theory}
% ====================================================================

% --------------------------------------------------------------------
\subsection{Group}\label{sec:group}
% --------------------------------------------------------------------

A \emph{Group} is a Monoid (\S\ref{sec:monoid}) with an Inverse for
every Element. That is, a Group is a Set $G$, and a Binary \emph{Group
  Operation}, $\cdot$, expressed as a Tuple, $(G,\cdot)$, satisfying
four \emph{Group Axioms}:
\begin{enumerate}
    \item Closure: $\forall a,b \in G, a \cdot b \in G$
    \item Associativity: $\forall a,b,c \in G, (a \cdot b) \cdot c = a
      \cdot (b \cdot c)$
    \item Identity Element: $\exists! e \in G : \forall a \in G,
      e \cdot a = a \cdot e = a$
    \item Inverse elements: $\forall a \in G, \exists b \in G :
      a \cdot b = b \cdot a = e$
\end{enumerate}
The Identity Element is the only Element with the unique Property:
\[
    e \cdot e = e
\]

The Signature (\S\ref{sec:signature}) for Groups is $\{\cdot, 1,
^{-1}\}$.

The number of Elements in a Group $\mathrm{G}$ is known as its
\emph{Order} and may be denoted $|\mathrm{G}|$.

\emph{Cayley's Theorem} states that every Group is Isomorphic to a
Group of Permutations (\S\ref{sec:permutation}).

A Group, $G$, within a Category, $\mathbf{C}$, may be viewed as a
Subset of the Hom-set of an Object, $X$:
\[
    G \subseteq Hom_{\mathbf{C}}(X,X)
\]



\subsubsection{Trivial Group}\label{sec:trivial_group}

A \emph{Trivial Group} is a Group $\{e\}$ with only an Identity
Element and no others. The Trivial Group may be denoted $0$ when
Group Operation is thought of as addition or as $1$ when Group
Operation is thought of as multiplication.

Given any Group $G$ with Identity Element $e$, the Trivial Group
$\{e\}$ is a Subgroup (\S\ref{sec:subgroup}) of $G$, called the
\emph{Trivial Subgroup}:
\[
    \{e\} \leq G
\]

All Trivial Groups are Isomorphic to one another, so \emph{the}
Trivial Group may be spoken of. The Trivial Group is the Zero Object
in the Category of Groups.

The Trivial Group is a Cyclic Group (\S\ref{sec:cyclic_group}) of
Order 1 and may be denoted $Z_1$ in this context.



\subsubsection{Group Word}\label{sec:group_word}

For a Group $G$ and a Subset of $G$, $S$, a \emph{Word} in $S$ is any
Expression of the form:
\[
    s_1^{\varepsilon_1}s_2^{\varepsilon_2} \cdots s_n^{\varepsilon_n}
\]
where $s_1,\ldots,s_n \in S$ and $\varepsilon_i \in \{-1, 1\}$ and $n$
is the \emph{Length} of the Word. The \emph{Empty Word} is used as the
Identity Element in the Free Group of Words (\S\ref{sec:free_group}).

\paragraph{Reduction}\label{sec:word_reduction}
\hfill \\

If an Element appears in a Word next to its Inverse, a
\emph{Reduction} may be applied which removes those Elements from the
Word due to the Group Axioms which imply that the resulting Word is
equivalent. A \emph{Reduced Word} contains no such redundant pairs.

A Word is \emph{Cyclically Reduced} if and only if every Cyclic
Permutation (\S\ref{sec:cyclic_permutation}) of the Word is Reduced
(that is, it is Reduced and the first and last Element are not
Inverses).



\subsubsection{Commutator}\label{sec:commutator}



% --------------------------------------------------------------------
\subsection{Abelian Group}\label{sec:abelian_group}
% --------------------------------------------------------------------

\emph{Abelian Group} also \emph{Commutative Group}



\subsubsection{Additive Group}\label{sec:additive_group}

An \emph{Additive Group} refers to a Group where the Group Operation
can be thought of as \emph{Addition} (usually Abelian).



\subsubsection{Multiplicative Group}\label{sec:multiplicative_group}

A \emph{Multiplicative Group} is defined in terms of a Structure with
Invertible Elements such as a \emph{Ring} (\S\ref{sec:ring}) $R$
having Multiplication, $\bullet$, as one of its Operations:
\[
  (R \ {0}, \bullet)
\]



\subsubsection{Cyclic Group}\label{sec:cyclic_group}

A Group $G$ is \emph{Cyclic} when there exists an element $g \in G$
such that:
\[
    G = \langle g \rangle = \{ g^n | n \in \mathbb{Z} \}
\]
Any Group $G$ of Prime Order is Cyclic is an Abelian Simple Group
(\S\ref{sec:simple_group}) and is Generated by any $g \in G : g \neq
e$.

Every Infinite Cyclic Group is Isomorphic to the Additive Group
$(\mathbb{Z}, +)$ of the Integers.

Every Finite Cyclic Group of Order $n$ is Isomorphic
(\S\ref{sec:group_isomorphism}) to the Additive Group
$\mathbb{Z}/n\mathbb{Z}$ of the Integers Modulo $n$.



% --------------------------------------------------------------------
\subsection{Groupoid}\label{sec:groupoid}
% --------------------------------------------------------------------

A \emph{Groupoid} is a Group with a Partial Function in place of a
Total Binary Operation. As a Category, a Groupoid is a Category in
which every Morphism is Invertible.

\emph{Fundamental Groupoid}



\subsubsection{$\infty$-groupoid}\label{sec:infinity_groupoid}

An \emph{$\infty$-groupoid} is an $\infty$-category
(\S\ref{sec:quasicategory}) generalization of a Groupoid.

By the \emph{Homotopy Hypothesis}, $\infty$-groupoids are \emph{Spaces}
(\S\ref{sec:topological_space}).



% --------------------------------------------------------------------
\subsection{Subgroup}\label{sec:subgroup}
% --------------------------------------------------------------------

Given a Group $G$, a \emph{Subgroup} $H$ is a Subset of Group Elements
that are still a Group under the Group Operation of $G$, denoted $H
\leq G$. When a Subgroup $H$ is a Proper Subset of $G$, $H$ is a
\emph{Proper Subgroup} of $G$, denoted $H \neq G$. If a Subgroup has
the same Order as the containing Group then the Groups are equivalent.

All Groups have at least two Subgroups: the Trivial Group
(\S\ref{sec:trivial_group}) and Group itself. A Group with exactly
these two Subgroups and no others is a \emph{Simple Group}
(\S\ref{sec:simple_group}).

If a Subgroup $G$ of $H$ is Non-abelian then $H$ is also Non-abelian.

The Subgroups of $(\mathbb{Z},+)$ are of the form $(b\mathbb{Z},+)$
where $b \in \mathbb{Z}$.



\subsubsection{Coset}\label{sec:group_coset}

The \emph{Coset} of a Subgroup (\S\ref{sec:subgroup}) $H$ in a Group $G$
is defined as
\begin{description}
\item[Left Coset:] $gH = {gh : h \in H}$
\item[Right Coset:] $Hg = {hg : h \in H}$
\end{description}
These Subsets are Disjoint and Partition (\S\ref{sec:partition}) $G$.

The \emph{Index} of a Subgroup $H$ is the number of distinct Left
Cosets of $H$, denoted $[G:H]$. As a Corollary:
\[
    |G| = |H|[G:H]
\]



\subsubsection{Normal Subgroup}\label{sec:normal_subgroup}

A \emph{Normal Subgroup} of a Group $G$ is a \emph{Subgroup} $H$ that
is invariant under Conjugation (\S\ref{sec:conjugacy_class}) by the
Elements of $G$, $gHg^{-1} = H$ and is denoted $H \triangleleft G$.
$H$ is a Normal Subgroup of $G$ if and only if $\forall g \in G, gH =
Hg$.

In an Abelian Group, all Subgroups are Normal Subgroups.



\subsubsection{Center}\label{sec:group_center}

The \emph{Center} of a Group $G$, denoted $Z(G)$, is a Normal Subgroup
of $G$ defined as:
\[
    Z(G) = \{ z \in G | \forall g \in G, zg = gz \}
\]
If $G$ is an Abelian Group, $Z(G) = G$.

For a Symmetric Group (\S\ref{sec:symmetric_group}) $S_n$:
\[
    Z(S_n) = \{e\}
\]

For a General Linear Group (\S\ref{sec:general_linear_group})
$GL_n(R)$:
\[
    Z(GL_n(R)) = \{\lambda \mathrm{I}\}
\]



\subsubsection{Skeleton}\label{sec:group_skeleton}



% --------------------------------------------------------------------
\subsection{Group Homomorphism}\label{sec:group_homomorphism}
% --------------------------------------------------------------------

A \emph{Group Homomorphism} $h$, between two Groups, $G$ and $H$, is a
Morphism $h : G \rightarrow H$ that preserves Group operations
$\cdot_G$ and $\cdot_H$. That is, for $x,y \in G$:
\[
    h(x \cdot_G y) = h(x) \cdot_H h(y)
\]
From this it follows that Group Homomorphisms have the Properties:
\[
    h(e_G) = e_H
\]\[
    h(x^{-1}) = h(x)^{-1}
\]

There is always a \emph{Trivial Homomorphism} between any two Groups
$G$ and $H$:
\[
    \forall x \in G, h (x) = e_H
\]

The Composition of two Homomorphisms $h : G \rightarrow H$ and $g : F
\rightarrow G$, $h \circ g$ is a Homomorphism.

Taken as Monoidal Categories (\S\ref{sec:monoidal_category}), two
Groups $G, H$ may be related by a Functor $f$ which is equivalent to a
Group Homomorphism:
\[
    f : G \rightarrow H
\]
and for $x,y \in G$:
\[
    f(xy) = f(x)f(y)
\]



\subsubsection{Group Homomorphism Image}\label{sec:group_image}

The \emph{Image} of a Group Homomorphism:
\[
    im(h) = \{ x' \in H | h(x) = x', x \in G \}
\]
The Image of a $h$ is a Subgroup (\S\ref{sec:subgroup}) of $H$:
\[
    im(h) \subset H
\]

If the Image of $h$ is equal to $H$, and the Kernel
(\S\ref{sec:group_kernel}) is $\{e_G\}$ (\S\ref{sec:morphism_kernel}),
then $h$ is an Isomorphism (\S\ref{sec:group_isomorphism}).



\subsubsection{Group Homomorphism Kernel}\label{sec:group_kernel}

The \emph{Kernel} of a Group Homomorphism, $f : G \rightarrow G'$,
denoted by $ker(f)$ is defined as:
\[
    ker(f) = \{g \in G | f(g) = e_{G'}\}
\]
where $e_{G'}$ is the Identity Element of $G'$, that is, the Preimage
or Fiber (\S\ref{sec:set_function}) of the Singleton Set $\{e_{G'}\}$.
The Kernel is a Normal Subgroup (\S\ref{sec:normal_subgroup}) of the
Domain Group of the Group Homomorphism, in this case $G$:
\[
    ker(f) \triangleleft G
\]

If $ker(f) = \{e_G\}$ and the Image (\S\ref{sec:group_image}) is equal
to the Codomain Group, $im(f) = G'$, then $f$ is an Isomorphism
(\S\ref{sec:group_isomorphism}).

An Equivalence Class may be defined for an Element $g \in G$ by taking
the Left Coset (\S\ref{sec:group_coset}) of the Kernel, $ker(f) = H
\triangleleft G$:
\[
    gH = \{ gh | h \in H \}
\]
Each such Equivalence Class has the same Order as $H$, which results
in the following Corollary for a Group Homomorphism $f : G \rightarrow
G'$ with Kernel $H$:
\[
    |G| = |H||im(f)|
\]



\subsubsection{Group Isomorphism}\label{sec:group_isomorphism}

A \emph{Group Isomorphism} is a Bijective Group Homomorphism.

A Group Homomorphism $h : G \rightarrow H$ is an Isomorphism if
$ker(h) = e_G$ and $im(h) = H$. If $G = H$ then $h$ is an
\emph{Automorphism} (\S\ref{sec:group_automorphism}).

The existence of an Isomorphism between Groups $G$ and $H$ imply the
following Properties:
\begin{itemize}
    \item $|G| = |H|$
    \item $G$ is Abelian $\Leftrightarrow$ $H$ is Abelian
    \item $G$ and $H$ have the same number of Elements of every Order
\end{itemize}



\paragraph{Isomorphism Theorem}\label{sec:isomorphism_theorem}

\emph{First Isomorphism Theorem}

\emph{Second Isomorphism Theorem}

\emph{Third Isomorphism Theorem}



\subsubsection{Group Automorphism}\label{sec:group_automorphism}

A \emph{Automorphism} is an Endomorphism that is also an Isomorphism.



\paragraph{Inner Automorphism}\label{sec:inner_automorphism}
\hfill \\

An \emph{Inner Automorphism} of a Group $G$ is an Automorphism $f : G
\rightarrow G$ defined by:
\[
    \forall g \in G, f(g) = a^{-1}ga
\]
where $a$ is a given fixed Element of $G$.



% --------------------------------------------------------------------
\subsection{Group Generator}\label{sec:group_generator}
% --------------------------------------------------------------------

\subsubsection{Cyclic Subgroup}\label{sec:cyclic_subgroup}

Given any Group $G$ with Element $g$, the Subgroup Generated by the
single Element $g$ is called the \emph{Cyclic Subgroup} of $g$,
denoted $<g>$, containing all Powers of $g$.

If $m$ is the smallest positive Integer such that $g^m = e$, then $m$
is the \emph{Order} of $g$. By \emph{Lagrange's Theorem}, in a Finite
Group, every Element has a Finite Order and the Order of every Element
divides evenly the Order of the Group.



\subsubsection{Commutator Subgroup}\label{sec:commutator_subgroup}

\emph{Commutator Subgroup}



% --------------------------------------------------------------------
\subsection{Group Centralizer}\label{sec:group_centralizer}
% --------------------------------------------------------------------

% --------------------------------------------------------------------
\subsection{Group Commutator}\label{sec:group_commutator}
% --------------------------------------------------------------------


% --------------------------------------------------------------------
\subsection{Finite Group}\label{sec:finite_group}
% --------------------------------------------------------------------

A \emph{Finite Group} is a Group with a finite number of Elements.

Any Finite Non-abelian Group is of Even Order.



% --------------------------------------------------------------------
\subsection{Simple Group}\label{sec:simple_group}
% --------------------------------------------------------------------

A Group $G$ is a \emph{Simple Group} if it has only the Trivial Group
and the entire Group $G$ as Normal Subgroups.

The only Abelian Simple Groups are the Cyclic Groups
(\S\ref{sec:cyclic_group}) of Prime Order.

For $n \geq 5$, the Alternating Group (\S\ref{sec:alternating_group})
$A_n$ is a Simple Group.



\subsubsection{Finite Simple Group}\label{sec:finite_simple_group}



\subsubsection{Infinite Simple Group}\label{sec:infinite_simple_group}



% --------------------------------------------------------------------
\subsection{Direct Product}\label{sec:direct_product}
% --------------------------------------------------------------------

The \emph{Direct Product} of two Groups $G$ and $H$, denoted $G \times
H$, is a Group with Elements from the Cartesian Product of the
Elements of $G$ and $H$, $\{(g,h) | g \in G, h \in H\}$, and Group
Operation defined Componentwise:
\[
    (g_1, h_1) \cdot_{G \times H} (g_2, h_2)
    = (g_1 \cdot_G g_2, h_1 \cdot_H h_2)
\]



% --------------------------------------------------------------------
\subsection{Transformation Group}\label{sec:transformation_group}
% --------------------------------------------------------------------

\emph{Permutation Group} (\S\ref{sec:permutation_group}) Set

\emph{Matrix Group} (\S\ref{sec:matrix_group}) Vector Space



\subsubsection{Symmetric Group}\label{sec:symmetric_group}

The Group whose Elements are all the Permutations (Bijections) of a
Set $S$ is the \emph{Symmetric Group} $Sym(S)$. The Symmetric Group on
${1, 2, ..., n} \in \mathbb{N}$ is denoted $\mathrm{S}_n$.

For finite $n$, $\mathrm{S}_n$ is a Finite Group of Order $n!$.

\begin{itemize}
    \item $S_1 = \{e\}$
    \item $S_2 = \{e,\tau\}$ where $\tau$ is the Transposition
      (\S\ref{sec:transposition}) $(12)$
    \item $S_3 = \{e, \tau, \tau', \tau'', \sigma, \sigma'\}$ where
      $\sigma$ and $\sigma'$ are Permutations of length 3: $(123)$ and
      $(321)$ respectively
\end{itemize}
The Group $S_n$ is Non-abelian for $n \geq 3$.

For $k \leq n$, $S_k \subset S_n$.



\paragraph{Cayley's Theorem}\label{sec:cayleys_theorem}



\subsubsection{Alternating Group}\label{sec:alternating_group}

For $n \geq 5$, $A_n$ is a Simple Group (\S\ref{sec:simple_group}).



\subsubsection{Permutation Group}\label{sec:permutation_group}

Any Permutation Group on a Set $S$ with Cardinality $n$ is any
Subgroup of the Symmetric Group $\mathrm{S}_n$
(\S\ref{sec:symmetric_group}).



\subsubsection{Automorphism Group}\label{sec:automorphism_group}

The Automorphisms of a Set $S$ form an \emph{Automorphism
  Group}, denoted $Aut(S)$ under Composition of Automorphisms. The
Automorphism Group is a Subgroup (\S\ref{sec:subgroup}) of the
Symmetric Group.



\paragraph{Inner Automorphism Group}\label{sec:inner_automorphism_group}
\hfill \\

For the Automorphism Group of a Group $G$, $Aut(G)$, one can always
construct a Homomorphism:
\[
    f : G \rightarrow Aut(G)
\]
defined as:
\[
    \forall g \in G, f (g) (h) = g h g^{-1}
\]
with Kernel equal to the Center (\S\ref{sec:group_center}) of $G$:
\[
    ker(f) = Z(G)
\]
and the Image is a Subgroup of $Aut(G)$ called the \emph{Inner
  Automorphism Group} of $G$, denoted $Inn(G)$, defined as:
\[
    Inn(G) = \{ a \in Aut(G) | \exists g \in G : a(h) = g h g^{-1} \}
\]
where $a$ is an \emph{Inner Automorphism}
(\S\ref{sec:inner_automorphism}) of $G$.

By the First Isomorphism Theorem (\S\ref{sec:isomorphism_theorem}),
$Inn(G)$ is Isomorphic to the Quotient Group
(\S\ref{sec:quotient_group}) $G / Z(G) \cong Inn(G)$.



\paragraph{Outer Automorphism Group}\label{sec:outer_automorphism_group}
\hfill \\



\subsubsection{Symmetry Group}\label{sec:symmetry_group}

\paragraph{Dihedral Group}\label{sec:dihedral_group}
\hfill \\
\emph{Dihedral Group}


\paragraph{Klein 4-group}\label{sec:klein_4group}
\hfill \\

The \emph{Klein 4-group} or $K_4$ (also $V$ for
``\emph{Vierergruppe}'') is the Group resulting from the Direct
Product (\S\ref{sec:direct_product}) of two Cyclic Groups
(\S\ref{sec:cyclic_group}) of Order 2: $K_4 = Z_2 \times Z_2$.

As a Permutation Representation on 4 Elements $K_4$ is a Subgroup of
the Alternating Group (\S\ref{sec:alternating_group}) on 4 Elements,
$A_4$:
\[
    K_4 = \{ (), (12)(34), (13)(24), (14)(23) \}
\]

The Klein 4-group is an Abelian Group and is Isomorphic to the
Dihedral Group of Order 4 and is the smallest Non-cyclic Group.

The Automorphism Group (\S\ref{sec:automorphism_group}) $Aut(K_4)$ is
Isomorphic to $S_3$.



\subsubsection{Matrix Group}\label{sec:matrix_group}

\paragraph{General Linear Group}\label{sec:general_linear_group}

\hfill \\ A \emph{General Linear Group} of Degree $n$ over a Ring
(\S\ref{sec:ring}) $R$ has as an Underlying Set of $n \times n$
Invertible Matrices and the Group Operation is ordinary Matrix
Multiplication and is denoted $GL_n(R)$ and is a Subgroup of
$Sym(R^n)$.



\paragraph{Special Linear Group}\label{sec:special_linear_group}

\hfill \\ A \emph{Special Linear Group} of Degree $n$ over a Ring
(\S\ref{sec:ring}) $R$, denoted $SL_n(R)$, is the Set of $n \times n$
Matrices with Determinant (\S\ref{sec:determinant}) equal to $1$, with
the Group Operation of ordinary Matrix Multiplication. This is a
Normal Subgroup of the General Linear Group $GL_n(R)$ given by the
Kernel of the Determinant Function $ker(det)$, where:
\[
  det : GL_n(R) \rightarrow R^\times
\]
and $R^\times$ is the Multiplicative Group
(\S\ref{sec:multiplicative_group}) on $R$.

The Elements of $SL_n(R)$ are the Volume and Orientation preserving
Linear Transformations of $R^n$.



\subsubsection{Lie Group}\label{sec:lie_group}



% --------------------------------------------------------------------
\subsection{Periodic Group}\label{sec:periodic_group}
% --------------------------------------------------------------------

% --------------------------------------------------------------------
\subsection{Solvable Group}\label{sec:solvable_group}
% --------------------------------------------------------------------

% --------------------------------------------------------------------
\subsection{Conjugacy Class}\label{sec:conjugacy_class}
% --------------------------------------------------------------------

Two Elements $a$ and $b$ of a Group $G$ are \emph{Conjugates} if there
is an Element $g \in G$ such that:
\[
    gag^{-1} = b
\]
which reads ``Conjugation of $a$ by $g$ results in $b$.'' Conjugation
is Invariant if and only if the Elements are Commutative:
\[
    gag^{-1} = a \Leftrightarrow ag = ga
\]

A \emph{Conjugacy Class} for an Element $a$ is defined as:
\[
    Cl(a) = \{ b \in G | \exists g \in G : b = gag^{-1}\}
\]



% --------------------------------------------------------------------
\subsection{Quotient Group}\label{sec:quotient_group}
% --------------------------------------------------------------------

For a Normal Subgroup (\S\ref{sec:normal_subgroup}) $H$ of a Group
$G$, $H \triangleleft G$, the \emph{Quotient Group} $G/H$ is the Set of
all Left Cosets (\S\ref{sec:group_coset}) of $H$ in $G$:
\[
    G/H = \{ aH : a \in G \}
\]



% --------------------------------------------------------------------
\subsection{Free Group}\label{sec:free_group}
% --------------------------------------------------------------------

The \emph{Free Group}, $F_S$, over a Set, $S$, called the \emph{Free
  Generating Set}, consists of all Reduced Words
(\S\ref{sec:group_word}) in $S$ as Elements and Concatenation of Words
(with Reduction) as the Group Operation. As a Group Presentation
(\S\ref{sec:group_presentation}):
\[
    F_S = \langle S, \rangle
\]
Every Word is Conjugate to a Cyclically Reduced Word
(\S\ref{sec:word_reduction}), and a Cyclically Reduced Conjugate of a
Cyclically Reduced Word is a Cyclic Permutation of the Word.

A Group $G$ is called Free if it is Isomorphic to $F_S$ for some
Subset $S$ of $G$, that is, every Element of $G$ can be written
uniquely as a Product of finitely many Elements of $S$ and their
Inverses.



% --------------------------------------------------------------------
\subsection{Topological Group}\label{sec:topological_group}
% --------------------------------------------------------------------

% --------------------------------------------------------------------
\subsection{Algebraic Group}\label{sec:algebraic_group}
% --------------------------------------------------------------------

% --------------------------------------------------------------------
\subsection{Abstract Group}\label{sec:abstract_group}
% --------------------------------------------------------------------



% ====================================================================
\section{Presentation}\label{sec:presentation}
% ====================================================================

\emph{Generators}

\emph{Relations}

\emph{Finitely Presented}

(\S\ref{sec:string_rewriting})

\emph{Semigroup Presentation}



\subsubsection{Absolute Presentation}\label{sec:absolute_presentation}



\subsubsection{Group Presentation}\label{sec:group_presentation}

A \emph{Group Presentation} is the definition of a Group $G$ by a Set
$S$ of Generators (\S\ref{sec:free_group}) and a Set $R$ of Relations
between the Words in $S$ that represent the same Element of $G$:
\[
  \langle S | R \rangle
\]
The Group $G$ has this Presentation if it is Isomorphic to the
Quotient (\S\ref{sec:quotient_group}) of a Free Group on $S$. The Set
of Relations $R$ is said to \emph{Define} $G$ if every Relation in $G$
follows from those in $R$.



\subsubsection{Monoid Presentation}\label{sec:monoid_presentation}

A \emph{Monoid Presentation} is a description of a Monoid in terms of
a Set $\Sigma$ of Generators and a Set of Relations on the Free Monoid
$\Sigma*$.





% --------------------------------------------------------------------
\subsection{Representation Theory}\label{sec:representation_theory}
% --------------------------------------------------------------------

Given a Functor $R$ from a Group $G$ to a general Category
$\mathbf{C}$
\[
    R : G \rightarrow \mathbf{C}
\]
Such a Functor $R$ is termed a \emph{Representation} of $G$ in
$\mathbf{C}$.



% ====================================================================
\section{Ring Theory}\label{sec:ring_theory}
% ====================================================================

% --------------------------------------------------------------------
\subsection{Ring}\label{sec:ring}
% --------------------------------------------------------------------

A \emph{Ring} is a Set $R$ with two Binary Operators, $+$ and
$\cdot$, where:

\begin{itemize}
\item $R$ is an Abelian Group under $+$
    \begin{enumerate}
        \item $+$ is Associative
        \item $+$ is Commutative
        \item There exists an Additive Identity $0 \in R$
        \item For all $a \in R$, there exists an Additive Inverse $-a
          \in R$
    \end{enumerate}
\item $R$ is a \emph{Monoid} under $\cdot$
    \begin{enumerate}
        \item $\cdot$ is Associative
        \item There exists a Multiplicative Identity $1 \in R$
    \end{enumerate}
\item $\cdot$ Distributes over $+$
    \begin{enumerate}
        \item $\forall a,b,c \in R,
            a \cdot (b + c) = (a \cdot b) + (a \cdot c)$
            (Left Distributivity)
        \item $\forall a,b,c \in R,
            (b + c) \cdot a = (b \cdot a) + (c \cdot a)$
            (Right Distributivity)
    \end{enumerate}
\end{itemize}
The Signature (\S\ref{sec:signature}) for Rings is $\{+, -, \cdot, 0,
1\}$

A \emph{Finite Ring} is a Ring that has a Finite number of Elements.

A \emph{Unit} is an Element of a Ring $R$ that has an Inverse
Element in the Multiplicative Monoid of $R$.

The term \emph{Unital Ring} is used to indicate a Ring with a
Multiplicative Identity, to differentiate from other
\emph{Pseudo-rings} that may lack a Multiplicative Identity.

A \emph{Rng} is a Pseudo-ring that satisfies all Ring axioms except a
Multiplicative Identity.

A \emph{Semiring} is a Ring without the requirement of Additive
Inverses.



\subsubsection{Zero Ring}\label{sec:zero_ring}

The \emph{Zero Ring}, denoted $\{0\}$ or $\mathbf{0}$, is the Unique Ring
(up to Isomorphism) consisting of one Element with Operations:
\[
    0 + 0 = 0
\] \[
    0 * 0 = 0
\]
In the Category of all Rings, $\mathbf{Rng}$, the Zero Ring is
\emph{Terminal Object} and the Ring of Integers $\mathbf{Z}$ is the
\emph{Initial Object} (\S\ref{sec:initial_terminal}).



% --------------------------------------------------------------------
\subsection{Commutative Ring}\label{sec:commutative_ring}
% --------------------------------------------------------------------

A \emph{Commutative Ring} is a Ring where $\cdot$ is Commutative.

\emph{Commutative Algebra}

\emph{Determinant}



\subsubsection{Integral Domain}\label{sec:integral_domain}

An \emph{Integral Domain}, $R$, is a Non-zero Commutative Ring where
if $ab = 0$ in R, then either $a = 0$ or $b = 0$ in $R$. Equivalently,
the product of any two non-zero elements is non-zero.

All Finite Integral Domains are \emph{Finite Fields}
(\S\ref{sec:finite_field}).



% --------------------------------------------------------------------
\subsection{Division Ring}\label{sec:division_ring}
% --------------------------------------------------------------------

A \emph{Division Ring} is a Ring where every Nonzero Element has a
Multiplicative Inverse (but $\cdot$ is not required to be
Commutative).


By \emph{Wedderburn's Little Theorem} all \emph{Finite Division Rings}
are Commutative and therefore \emph{Finite Fields}
(\S\ref{sec:finite_field}).



% --------------------------------------------------------------------
\subsection{Nilpotent}\label{sec:nilpotent}
% --------------------------------------------------------------------

An Element $x$ of a Ring $R$ is \emph{Nilpotent} if there is a
positive Integer $n$ such that $x^n = 0$.



% ====================================================================
\section{Field Theory}\label{sec:field_theory}
% ====================================================================

% --------------------------------------------------------------------
\subsection{Field}\label{sec:field}
% --------------------------------------------------------------------

A \emph{Field} is a Nonzero Commutative Ring with a Multiplicative
Inverse for every Nonzero Element or equivalently a Ring whose Nonzero
Elements form an Abelian Group (\S\ref{sec:abelian_group}) under
Multiplication.



\subsubsection{Total Field}\label{sec:total_field}

\subsubsection{Closed Field}\label{sec:closed_field}

\subsubsection{Finite Field}\label{sec:finite_field}

A \emph{Finite Field} or \emph{Galois field} is a Field that contains
a finite number of Elements with the \emph{Order} being equal to the
number of Elements.

A Finite Field only exists when the Order is a Prime Power
(\S\ref{sec:prime_number}).



% --------------------------------------------------------------------
\subsection{Module}\label{sec:module}
% --------------------------------------------------------------------

A \emph{Module} is a Unital Ring, $R$, together with an Abelian Group,
$(M, +)$, and an Operation called \emph{Scalar Multiplication} which
is either:
\[ R \times M \rightarrow M \]
for a \emph{Left $R$-module $M$}, $_R M$, or:
\[ M \times R \rightarrow M \]
for a \emph{Right $R$-module $M$}, $M_R$.

The Scalar Multiplication Operator is required that for all $r,s \in
R$ and $x,y \in M$ in a Left $R$-module $M$:
\begin{enumerate}
    \item $r(x + y) = rx + ry$
    \item $(r + s)x = rx + sx$
    \item $(rs)x = r(sx)$
    \item $1_Rx = x$
\end{enumerate}
or in a Right $R$-module $M$:
\begin{enumerate}
    \item $(x + y)r = xr + yr$
    \item $x(r + s) = xr + xs$
    \item $x(rs) = (sx)r$
    \item $x 1_R = x$
\end{enumerate}
where $1_R$ is the Multiplicative Identity for $R$. If the Ring is not
required to be Unital, then item (4) above can be ommitted, but can be
explicitly required by stating that we are talking about a
\emph{Unital Left/Right $R$-module $M$}.

\emph{Bimodule}

If $R$ is Commutative, then Left $R$-modules are the same as Right
$R$-modules and simply called \emph{$R$-modules}.

A Module Homomorphism is an \emph{Linear Map}.

\emph{Bilinear Map}

\emph{Multilinear Map}



\subsubsection{Free Module}\label{sec:free_module}

\paragraph{Group Ring}\label{sec:group_ring}



% --------------------------------------------------------------------
\subsection{Ideal}\label{sec:ring_ideal}
% --------------------------------------------------------------------

\emph{Ideal}

\emph{Principal Ideal}



% --------------------------------------------------------------------
\subsection{Quotient Ring}
% --------------------------------------------------------------------



% --------------------------------------------------------------------
\subsection{Vector Space}\label{sec:vector_space}
% --------------------------------------------------------------------

\emph{Span}

\emph{Finite Dimensional Vector Space} - has a Span

\emph{Infinite Dimensional Vector Space} - does not have a Span

\emph{Linear Independence}

\emph{Basis} - Spans and is Linearly Independent

All Bases of a Vector Space $\mathbf{V}$ have the same number of
Elements equal to the \emph{Dimension} of $\mathbf{V}$,
$dim(\mathbf{V})$. The Dimension of a Vector Space is uniquely defined
because for any Vector Space, a Basis exists, and all Bases of a
Vector space have equal Cardinality (\S\ref{sec:cardinality}).

For a Finite Dimensional Vector Space a Subset of a Span defines a
Basis, and a Linearly Independent Subset can be extended to form a
Basis.

The number of Elements in a Spanning Subset of $\mathbf{V}$ is greater
than or equal to the Dimension of $\mathbf{V}$.

The number of Elements in a Linearly Independent Subset of
$\mathbf{V}$ is less than or equal to the Dimension of $\mathbf{V}$.

A Basis defines an Isomorphism of Vector Spaces:
\[
    \mathbf{V} \xrightarrow{f} F^n
\]

\emph{Tensor Product}, \emph{Outer Product}



\subsubsection{Inner Product}\label{sec:inner_product}



\subsubsection{Norm}\label{sec:vector_space_norm}

\emph{Seminorm}

\emph{Quasinorm}



% --------------------------------------------------------------------
\subsection{Bimodule}\label{sec:bimodule}
% --------------------------------------------------------------------



% ====================================================================
\section{Representation Theory}
% ====================================================================

% ====================================================================
\section{Operad Theory}\label{sec:operad_theory}
% ====================================================================

% ====================================================================
\section{Initial Algebra}\label{sec:initial_algebra}
% ====================================================================

% ====================================================================
\section{Heyting Algebra}\label{sec:heyting_algebra}
% ====================================================================

% --------------------------------------------------------------------
\subsection{Boolean Algebra}\label{sec:boolean_algebra}
% --------------------------------------------------------------------

Syntactically, every Boolean Term corresponds to a Propositional
Formula (\S\ref{sec:propositional_logic}).



% ====================================================================
\section{Relation Algebra}
% ====================================================================

% ====================================================================
\section{Relational Algebra}\label{sec:relational_algebra}
% ====================================================================

\emph{Domain Relational Calculus}



% ====================================================================
\section{Quantale}
% ====================================================================

% ====================================================================
\section{Invariant Theory}\label{sec:invariant_theory}
% ====================================================================
