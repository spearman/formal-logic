%%%%%%%%%%%%%%%%%%%%%%%%%%%%%%%%%%%%%%%%%%%%%%%%%%%%%%%%%%%%%%%%%%%%%%
%%%%%%%%%%%%%%%%%%%%%%%%%%%%%%%%%%%%%%%%%%%%%%%%%%%%%%%%%%%%%%%%%%%%%%
\part{Formal Logic}\label{sec:formal_logic}
%%%%%%%%%%%%%%%%%%%%%%%%%%%%%%%%%%%%%%%%%%%%%%%%%%%%%%%%%%%%%%%%%%%%%%
%%%%%%%%%%%%%%%%%%%%%%%%%%%%%%%%%%%%%%%%%%%%%%%%%%%%%%%%%%%%%%%%%%%%%%

\emph{Formal Logic} (or \emph{Symbolic Logic})

Formal System (\S\ref{sec:formal_system}) + Formal Semantics
(\S\ref{sec:formal_semantics})

Symbols in a Formal System may be divided into \emph{Logical Symbols}
and \emph{Non-Logical Symbols}.

Logical Symbols always have the same meaning, e.g. Symbols such as
$\forall$, $\vee$, $\rightarrow$, $\neg$, etc., and \emph{Variables}
(\S\ref{sec:variable}) $x_0$, $x_1$, etc.

Non-Logical Symbols only have meaning under an \emph{Interpretation}
(\S\ref{sec:interpretation}), e.g. Symbols such as \emph{Predicates}
(Relations \S\ref{sec:set_relation}), \emph{Functions}
(\S\ref{sec:set_function}), and individual \emph{Constants} (Nullary
Functions). The set of Non-Logical Symbols used in a particular
discourse is called the \emph{Signature} of the discourse. The
Signature may be defined as a triple:
\[
    \sigma = (S_{func},S_{rel},ar)
\]
where
\[
    ar: S_{func} \cup S_{rel} \rightarrow \mathbb{N}_0
\]
\emph{Extra-logical Symbols} are those of a Metalanguage
(\S\ref{sec:metalanguage}), such as the symbol for \emph{Logical
  Consequence} (\emph{Entailment}), $\vdash$, (read \emph{yields} or
\emph{proves}) or Metavariables, $\varphi, \psi, \ldots$.



% ====================================================================
\section{Deductive Logic}\label{sec:deductive_logic}
% ====================================================================



% ====================================================================
\section{Inductive Logic}\label{sec:inductive_logic}
% ====================================================================

\emph{Axiom of Induction}

\emph{Prefix Induction}

\emph{Complete Induction}

\emph{Transfinite Induction}

\emph{Noetherian Induction}



% ====================================================================
\section{Abductive Logic}\label{sec:abductive_logic}
% ====================================================================



% ====================================================================
\section{Terminology}\label{sec:logic_terminology}
% ====================================================================

\begin{description}
\item[Universe] class containing elements of \emph{objects} considered
  in a particular Logical discourse, also called \emph{Domain of
    Discourse}
\item[Constant] a named object from the Domain
\item[Variable] a placeholder that ranges over the objects in the
  Domain
\item[Function] $n$-ary functions maps $n$-tuples of objects in the
  Domain to objects
\item[Term] an object of the Domain (Variables, Constants and Compound
  Statements)
\item[Formula] a mathematical Fact
\end{description}
\hfill \\ Inductive definition of \emph{Terms} from \emph{Constants},
\emph{Variables}, and \emph{Functions}:

Given Terms, $T$, Variables, $V$, n-ary Functions, $F = F_0 \cup F_1
\cup F_2 \cup \cdots \cup F_n$, Constants, $C = F_0$:
\[
    V \subseteq T
\]\[
    C \subseteq T
\]\[
    \forall \tau_n=\{t_1,\cdots,t_n\} \in \mathcal{P}(T), \forall f \in F_n
    \exists f(t_1,\cdots,t_n) \in T
\]
Inductive definition of \emph{Formulas} from Terms and
\emph{Relations}:

Given Terms, $T = \{t_0,\ldots,t_n\}$, and Relations, $R = \{r_0,\ldots,r_m\}$:
\begin{itemize}
\item $t_i = t_j$ is a Formula
\item $r_k(t_0,\ldots,t_n)$ is a Formula and $r_k$ is an n-ary Relation
\end{itemize}
A Formula is \emph{Valid} if and only if it is True under every
\emph{Interpretation} (\S\ref{sec:interpretation}).

\emph{Mapping} % Functions, Morphisms

A \emph{Formal Argument} is an Ordered Set of \emph{Formulas}.

\emph{Witness}



% --------------------------------------------------------------------
\subsection{Metavariable}\label{sec:metavariable}
% --------------------------------------------------------------------

A \emph{Metavariable} is a Variable written in a Metalanguage that
stands in for an Element in the Object Language. The Formalization of
Metavariables falls under \emph{Type Theory} (Part
\ref{sec:type_theory}).

Metavariables may be referred to as \emph{Schematic Variables} in the
context of \emph{Axiom Schemata} and \emph{Rule Schemata}
(\S\ref{sec:deductive_apparatus}).



% --------------------------------------------------------------------
\subsection{Quantification}\label{sec:firstorder_quantification}
% --------------------------------------------------------------------

A \emph{Quantifier} limits (\emph{Binds}) a Variable to a certain
quantity of members of the Domain, the two fundamental Quantifiers
being \emph{Universal} ($\forall$) and \emph{Existential} ($\exists$).

A \emph{Bound Quantifier} is one with a restricted Range, e.g.
$\exists x > 0$ or $\forall x \in \mathbb{R}$.



\subsubsection{Bound \& Free Variables}\label{sec:bound_free}

Variables are \emph{Free Variables} if they are not Quantified in any
Formula, and \emph{Bound Variables} when they are Quantified.
Inductive definition of \emph{Free} and \emph{Bound Variables}:
\begin{enumerate}
\item A Variable $x$ is Free in Atomic Formula $\varphi$ if $x$ occurs
  in $\varphi$ (there are no Bound Variables in Atomic Formulas)
\item A Variable is Free or Bound in $\varphi \bullet \psi$ if $x$ is
  Free or Bound in either $\varphi$ or $\psi$, where $\bullet$ is a
  Binary Connective
\item A Variable $x$ is Free in $\forall y \varphi$ iff $x$ is Free in
  $\varphi$ and $x$ is not $y$. Conversely $x$ is Bound in $\forall y
  \varphi$ if $x$ is $y$ or $x$ is Bound in $\varphi$.
\end{enumerate}

A Term with no Free Variables is a \emph{Ground Term} and a Formula
with no Free Variables in First-order Logic is a \emph{First-order
  Sentence} (also called a \emph{Closed Formula}). First-order
Sentences have well-defined Truth values. Free Variables are
implicitly Universally Quantified.

\subsubsection{Uniqueness Quantification}\hfill

\emph{Unique Existential Quantification}, denoted by $\exists !$,
is expressed in natural language as ``there is one and only one.'' A
First-Order System requires the \emph{Equality Relation}
(\S\ref{sec:firstorder_equality}) in order to be able to express
Uniqueness Quantification.



\subsubsection{Quantifier Rank}

Inductive definition of \emph{Quantifier Rank} function $qr$:
\begin{itemize}
\item $qr(\varphi) = 0$ if $\varphi$ is Atomic
\item $qr(\varphi_1 \wedge \varphi_2) = qr(\varphi_1 \vee \varphi_2) = max(qr(\varphi_1),qr(\varphi_2))$
\item $qr(\neg \varphi) = qr(\varphi)$
\item $qr(\exists_x \varphi) = qr(\varphi) + 1$
\end{itemize}



\subsubsection{Quantifier Nesting}

% FIXME



\subsubsection{Conjunctive Normal Form}\label{sec:conjunctive_normal}



\subsubsection{Prenex Normal Form}\label{sec:prenex_normal}

\emph{Normal Form} (\S\ref{sec:normal_form})



% --------------------------------------------------------------------
\subsection{Zeroth-order - Propositional}\label{sec:propositional_logic}
% --------------------------------------------------------------------

\emph{Propositional Logic} (also called \emph{Sentential} or
\emph{Statement Logic}) is represented by a Formal Language with WFF
consisting only of \emph{Operators} (\emph{Logical Connectives}) and
\emph{Primitive Symbols} representing \emph{Propositions}.

Propositions are WFF that are assigned a truth value. An \emph{Atomic
  Proposition} contains no Operators. A \emph{Composite Proposition}
is composed by recursive application of Operators to Propositions by a
corresponding \emph{Concatenation Rule} that assigns a new truth value
to the Composite string.

A Propositional Formula is a \emph{Boolean Term}
(\S\ref{sec:boolean_algebra}).

Primitive Symbols are usually divided into three different categories:
\begin{description}
\item[Propositional Constants] \hfill \\
Represent particular Propositions: $A$, $B$, $C$, $\ldots$
\item[Propositional Variables] \hfill \\
Range over set of all Atomic Propositions: $p$, $q$, $r$, $\ldots$
\item[Schematic Variables] \hfill \\
Metavariables; range over set of all propositions: $\varphi$, $\psi$,
$\chi$, $\ldots$
\end{description}

The Domain of a Propositional Calculus is \emph{Truth} and
\emph{Falsity}, so Variables are not necessarily \emph{Bound} or
\emph{Free} as in \emph{Predicate Logic}. In fact, a Propositional
Variable is equivalent to a \emph{Nullary Predicate} in \emph{First
  Order Logic}.

Formal definition of a \emph{Propositional Calculus}:
\[
    \mathcal{S} = (\mathbf{A},\mathbf{\Omega},\mathbf{Z},\mathbf{I})
\]
where
\begin{itemize}
\item $\mathbf{A}$ is a finite set of Proposition symbols ($p$, $q$,
  $r$, $\ldots$)
\item $\mathbf{\Omega}$ is a finite set of Operator symbols ($\neg$,
  $\wedge$, $\vee$, $\ldots$)
\item $\mathbf{Z}$ is a finite set of Inference Rules
\item $\mathbf{I}$ is a finite set of Axioms
\end{itemize}
$\mathcal{S}$ is then inductively defined as follows, where
$\mathbf{\Omega_j}$ is the partition of $\mathbf{\Omega}$ containing
Operators of arity $\mathbf{j}$:
\begin{enumerate}
\item Any element of $\mathbf{A}$ is a Formula of $\mathcal{S}$
\item For Formulas $p_1, p_2, \cdots, p_j$ and $f \in
  \mathbf{\Omega_j}$ then $f(p_1, p_2, \cdots, p_j)$ is a formula
\end{enumerate}
Propositional Logic is closed under Truth-Functional Connectives, so
the above is sufficient to define all WFF: nothing else is a formula
of $\mathcal{S}$.

Formulas derived by the Axioms and Inference Rules of a Propositional
Logic are termed \emph{Theorems}. Allowing for Axiom Schema (an
infinite number of axioms) extends Propositional Logic; an example of
such a system is \emph{Skolem Arithmetic}\cite{skolem23}.

The set of WFF of a System, $\mathcal{S}$, may be defined inductively:
\begin{itemize}
\item Propositional Variables are WFF
\item If $\varphi$ is a WFF, then $\neg\varphi$ is a WFF
\item If $\varphi$ and $\psi$ are WFF and $\bullet$ is a binary Operator,
  then $\varphi \bullet \psi$ is a WFF.
\end{itemize}

\subsubsection{Argument Forms}

Inference Rules of a Propositional Logic define Valid \emph{Argument
  Forms} (\S\ref{sec:inference_rule}). The simplest Argument Form
that is both necessary (and given a complete set of Axioms is
sufficient to define all other Argument Forms) is \emph{Modus Ponens},
shown here Schematicized:

$\textrm{1. }\varphi \rightarrow \psi$

$\textrm{2. }\varphi$

$\therefore\textrm{ }\psi$
\\
where lines one and two are Premises and line three is the Conclusion
(the symbol $\therefore$ is read as \emph{therefore}). This is written in
\emph{Sequent Notation}(\S\ref{sec:sequent_notation}) as
\[(\varphi \rightarrow \psi), \varphi \vdash \psi\]
The Schematic representation of \emph{Modus Tollens}:

$\textrm{1. }\varphi \rightarrow \psi$

$\textrm{2. }\neg\psi$

$\therefore\textrm{ }\neg\varphi$

\subsubsection{Operators}

The minimal set of primitive Operators is the \emph{negation} symbol
($\neg$) plus a \emph{sole sufficient} Operator of either $\land$,
$\lor$ or $\rightarrow$. Choosing one of these Operators, the other
two, and any other Operator, can be defined in terms of it and
negation. It is also possible to construct functionally complete sets
of one element: $\mathbf{\Omega} = \{\uparrow\}$ or $\mathbf{\Omega} =
\{\downarrow\}$ ($\uparrow$ and $\downarrow$ being \emph{NAND} and
\emph{NOR}, respectively).
\\
Example simple Axiom system:

$\mathbf{\Omega} = \mathbf{\Omega_1} \cup \mathbf{\Omega_2}$

$\mathbf{\Omega_1} = \{\neg\}$

$\mathbf{\Omega_2} = \{\rightarrow\}$

$\mathbf{I} = \{ (p \rightarrow (q \rightarrow p)),$

$\qquad((p \rightarrow (q \rightarrow r)) \rightarrow
(( p \rightarrow q) \rightarrow (p \rightarrow r))),$

$\qquad(( \neg p \rightarrow \neg q ) \rightarrow (q \rightarrow p ))
\}$\\ with Modus Ponens as the sole inference rule (see Hilbert
Systems, \S\ref{sec:hilbert_systems}).

In Propositional Logic, the Extra-logical Symbol for Entailment,
$\vdash$, and the Logical \emph{Implication Symbol}, $\rightarrow$,
coincide in that
\[(A \vdash B) \leftrightarrow (\vdash A \rightarrow B)\]
but the difference is that $\vdash$ describes a Deduction, that is a
relation between Sentences, and $\rightarrow$ is a Logical Connective
within a Formula.

Another possible system is a \emph{Natural Deduction
  System}\cite{jaskowski34} (\S \ref{sec:natural_deduction})
which has no Axioms ($\mathbf{I}=\varnothing$) and ten Inference
Rules.



% --------------------------------------------------------------------
\subsection{First-order - Predicate}
\label{sec:predicate_logic}\label{sec:first_order}
% --------------------------------------------------------------------

Systems of \emph{First-order Predicate Logic} add \emph{Extensional
  Quantifiers} that may be applied to Variables, which may be Objects
of the Universe of discussion, or Relations or Functions. A
\emph{First-Order Theory} may be formed by a system of First-order
Logic together with a Domain of Discourse over which Variables may
range, plus finitely many Functions and \emph{Predicates} defined on
that Domain, and a recursive set of Axioms.

A Predicate in First-order Logic takes one or more Objects from the
Domain and returns either True or False, that is a Relation on the
Domain. A Predicate taking no Objects (a Nullary Predicate) is
equivalent to a Proposition in Zeroth-order Logic.

\emph{Higher-order Logic}(\S\ref{sec:higher_order}) allows Predicates
to be applied to other Predicates or Functions, or Quantifiers may be
applied to Predicates or Functions. In First-order Logic, Predicates
are associated with Sets, in Higher-order Logic, with Sets of
Sets.
\\
The traditional Signature used in First-order Logic:
\begin{enumerate}
\item For $n \geq 0$, $n$-ary Predicate (also called Relation)
  Symbols: $p^{n}_0, p^{n}_1, p^{n}_2, p^{n}_2, p^{n}_3, \ldots$
\item For $n \geq 0$, $n$-ary Function Symbols: $f^{n}_0, f^{n}_1,
  f^{n}_2, f^{n}_2, f^{n}_3, \ldots$
\end{enumerate}
The contemporary Signature used:
\begin{enumerate}
\item Predicate Symbols denoted by uppercase letters $P$, $Q$, $R$,
  $\ldots$ with arity ($\geq 0$) specified by the \emph{Valence} of the
  parenthetical arguments, eg P(x), Q(x,y).
\item Function Symbols denoted by lowercase letters $f$, $g$, $h$,
  $\ldots$ with arity specified in the usual way.
\end{enumerate}
Here, Functions of Valence 0 are \emph{Constant Symbols} denoted by
letters $a$, $b$, $c$, $\ldots$.

\subsubsection{Properties}\label{sec:firstorder_properties}

First-order Logic may be used to devise Deductive Systems with finite
Domains that are \emph{Sound} (\S\ref{sec:soundness}) and
\emph{Complete}, but for infinite Domains a system of Higher-order
Logic is required. First-order Logic is
Semi-decidable(\S\ref{sec:semidecidable}).

%FIXME ref Lowenheim-skolem thoerem
The \emph{L\"owenheim-Skolem theorem} implies that First-order Logic
is unable to characterize the concept of Countability (or
Uncountability).

%FIXME def/ref compactness theorem
The \emph{Compactness theorem} implies that if a Formula is derived
from a System of First-order Logic with an infinite set of Axioms,
then it can be derived from a finite number of those Axioms. This has
implications for the determination of \emph{Connected Components} of a
\emph{Directed Graph} (\S\ref{sec:directed_graph}).

\subsubsection{Formation Rules}\label{sec:formation_rules}

The \emph{Formation Rules} for WFF in a System of First-order Logic
generally describe a Context-free Grammar with a infinite Alphabet and
many Start Symbols.

Terms are limited to those derived from Variables and a finite number
of $n$-ary Function applications, but not Expressions involving a
Predicate Symbol. See Section \ref{sec:logic_terminology} for a recursive
definition of Terms.

Definition of \emph{Atomic Formulas} (no Logical Connectives or Quantifiers):
\begin{enumerate}
\item If $t_1$ and $t_2$ are Terms, then $t_1 = t_2$ is an Atomic Formula.
\item If $R$ is an $n$-ary Relation (Predicate), and $t_1,\ldots,t_n$
  are terms, then $R(t_1,\ldots,t_n)$ is an Atomic Formula.
\end{enumerate}
Atomic Formulas or their negations are also called \emph{Literals}. A
\emph{Clause} is a finite Disjunction of Literals.

Definition of WFF as a finite number of applications of the following rules:
\begin{enumerate}
\item $\neg \phi$ is a WFF when $\phi$ is a WFF
\item $(\phi \bullet \psi)$ is a WFF when $\phi$ and $\psi$ are WFF
  and $\bullet$ is a Binary Connective
\item $\exists x \phi$ is a WFF when $x$ is a Variable and $\phi$ is a WFF
\item $\forall x \phi$ is a WFF when $x$ is a Variable and $\phi$ is a WFF
\end{enumerate}




\subsubsection{Inference Rules}

\paragraph{Universal Generalization}\label{sec:universal_generalization} \hfill
\\
\[P(x) \vdash \forall x P(x)\]

\subsubsection{Equality Conventions}\label{sec:firstorder_equality}

\paragraph{First-order Logic with Equality}\hfill
\\ Including a primitive Logical Symbol for equality, $=$, interpreted
as the real equality relation between members of the Domain such that
``two'' given members are the same member. This adds the following
Axioms:

\begin{enumerate}
\item \textbf{Reflexivity}: $\forall x, x=x$
\item \textbf{Substitution for functions}: given a function, $f$,
  $\forall x \forall y, x = y \rightarrow f(\ldots,x,\ldots) =
  f(\ldots,y,\ldots)$
\item \textbf{Substitution for formulas (Leibniz's Law)}: given a
  formula $\varphi$ with Free occurrences of $x$, and $\varphi '$ with
  Free occurrences of $y$, $\forall x \forall y, x = y \rightarrow
  (\varphi \rightarrow \varphi ')$
\end{enumerate}

Defining a theory with a Binary Relation $A(x,y)$ that satisfies
Reflexivity and Leibniz's law is sufficient to derive any other
equality Theorems.

\paragraph{First-order Logic without Equality} \hfill
\\ An alternative convention is to consider the Equality Relation to
be a Non-logical Symbol, included as a part of the Signiature of a
particular Theory instead of as a Rule of Logic. This allows two
distinct individuals to be considered equal by an arbitraray
Equivalence Relation. If this convention is used, but no distinct
individuals, $a$ and $b$ satisfy $a=b$ then the interpretation is
termed a \emph{Normal Model} (that is equivalent to a First-order
Logic with Equality).


\subsubsection{Monadic First-order Logic}

\emph{Monadic First-order Logic}, also called \emph{Monadic Predicate
  Calculus} restricts First-order Logic to unary Relations and no
Function symbols. This weaker form of First-order Logic is fully
Decidable.

\subsubsection{Many-sorted First-order Logic}\label{sec:manysorted_logic}

\emph{Many-sorted First-order Logic} allows Variables to be Quantified
over different Domains, thus giving Variables different
\emph{Sorts}. With finitely many Sorts, Many-sorted First-order Logic
can be reduced to Single-sorted First-order Logic. This can be
accomplished by adding unary Predicates to a First-order Logic that
partition the Domain.

\subsubsection{Infinitary First-order Logic}

\emph{Infinitary Logic} allows Formulas of infinite length, through
either Conjunctons and Disjunctions, infinite-arity Relations and
Functions, or Quantification over infinitely many Variables.

\paragraph{$\Omega$-logic}\label{sec:omega_logic}

% --------------------------------------------------------------------
\subsection{Higher-order - Plural}\label{sec:higher_order}
% --------------------------------------------------------------------

\subsubsection{Second-order}\label{sec:second_order}

\emph{Second-order Logic} allows for Quantifiers to range over
Relations and Functions and thus \emph{Sorts} of Variables that range
over $k$-ary Relations and Functions. It is possible to leave out a
definition of Quantifiers for Functions since $k$-ary Functions can be
represented by $k+1$-ary Relations.\cite{shapiro00} Quantification
over Functions allows the creation of the \emph{Analytic Hierarchy}
(\S\ref{sec:analytic_hierarchy}).

\subsubsection{Plural, Monadic Second-order Logic}
\label{sec:monadic_second_order}

An alternative formulation of Second-order Logic is to allow Variables
to take on \emph{Plural} Values. It is equi-interpretable with
\emph{Monadic Second-order Logic}, which restricts Quantification to
Unary Relations (sets).

\subsubsection{Independence-friendly Logic}\label{sec:independence_logic}

\emph{Independence-friendly Logic} has \emph{Branching Quantifiers}.

% --------------------------------------------------------------------
\subsection{Classical Logic}\label{sec:classical_logic}
% --------------------------------------------------------------------

\emph{Classical Logic} is the class of Propositional and
First-order Systems of Logic characterized by the following Inference
Rules:

\begin{description}

\item [Tertium non datur] (\emph{Law of excluded middle})
    \[\vdash(p \vee \neg p)\]

\item [Double Negation]
    \[p \vdash \neg\neg p\]

\item [Law of Non-contradiction]
    \[\vdash \neg(p \wedge \neg p)\]

\item [Ex falso quodlibet] (\emph{Principle of explosion},
  \emph{Principle of Psuedo-Scotus})
    \[\vdash 0 \rightarrow p\]

%FIXME finish properties and rules

\end{description}

The intended Semantic Interpretation (\S\ref{sec:interpretation})
of Classical Logic is subject to the \emph{Principle of Bivalence}
which says that every Proposition has one Truth-value: True or False.
Non-classical Logics such as \emph{Intuitionistic Logic}
(\S\ref{sec:intuitionistic_logic}) does not have this Property.

% --------------------------------------------------------------------
\subsection{Modal (Intensional) Logic} \label{sec:modal_logic}
% --------------------------------------------------------------------

\emph{Intensional Logic} adds to First-order Logic \emph{Sentential
  Functors} (\emph{Intensions}) that range over Terms. An Intension is
the \emph{Sense} in which a Logical Assertion is made, as opposed to
the \emph{Reference} to which the Assertion applies (\emph{i.e.
  Extensional Quantification}).

\emph{Modal Logic} extends Propositional and Predicate Logic to
include Operators expressing \emph{Modality}. Various meanings for
these Modal Operators include \emph{Alethic Modality}
(\emph{Necessity} and \emph{Possibility}), \emph{Temporal Modality}
(qualification in terms of time, eg \emph{always}, \emph{eventually},
\emph{until}), \emph{Deontic Modality} (\emph{Obligation} and
\emph{Permission}), and \emph{Doxastic Modality} (Modalities with
regards to \emph{Belief}).

An unary \emph{Primitive Modal Operator}, $\square$, defines a Dual
Operator, $\Diamond$, such that the following analogues of de Morgan's
Laws (\S\ref{sec:de_morgan}) hold:
    \[\Diamond P \leftrightarrow \neg \square \neg P\]
    \[\square P \leftrightarrow \neg \Diamond \neg P\]
Modal Logic with more than one Primitive Modal Operator, $\square _i,
i \in \{1, \ldots, n\}$ is \emph{Multimodal Logic}.

By the \emph{Curry-Howard Correspondence}
(\S\ref{sec:curry_howard}), Modal Logic corresponds
to \emph{Monads} (\S\ref{sec:monad}).

\subsubsection{Alethic Logic}\label{sec:alethic_logic}

Most Systems of Alethic Logic are based on an extension of
Propositional Logic called $\mathbf{K}$ which has:

\begin{enumerate}
\item $\square$, unary operator for \emph{Necessity}.
\item $\mathbf{N}$, \emph{Necessitation Rule}: stating if $p$ is a
  Theorem, then $\square p$ is a Theorem.
\item $\mathbf{K}$, \emph{Distribution Axiom}: $\square(p \rightarrow
  q) \rightarrow (\square p \rightarrow \square q)$ (also called the
  \emph{Kripke schema} (\S\ref{sec:frame_semantics}).
\end{enumerate}

Adding further Axioms gives rise to a nested hierarchy of Systems of
\emph{Normal Modal Logic}:

\begin{itemize}
\item $K := \mathbf{K} + \mathbf{N}$
\item $T := K + \mathbf{T}$
\item $S4 := T + \mathbf{4}$
\item $S5 := S4 + \mathbf{5}$
\item $D := K + \mathbf{D}$
\end{itemize} \hfill \\
where

\begin{itemize}
\item $\Diamond$, unary operator for \emph{Possibly}
\item $\mathbf{T}$, \emph{Reflexivity Axiom}: $\square p \rightarrow p$
\item $\mathbf{4}$: $\square p \rightarrow \square \square p$
\item $\mathbf{B}$: $p \rightarrow \square \Diamond p$
\item $\mathbf{D}$: $\square p \rightarrow \Diamond p$
\item $\mathbf{5}$: $\Diamond p \rightarrow \square \Diamond p$
\end{itemize}

\subsubsection{Doxastic Logic}

\emph{Doxastic Logic} uses the unary Modal Operator, $\mathcal{B}$, to
denote \emph{Belief}. Example:
\[
    \mathcal{B} x
\]
has the meaning ``It is Believed that x is the case''. A set of
Beliefs is usually denoted
\[
    \mathbb{B}: \{ b_1, b_2, \ldots, b_n \}
\]

\subsubsection{Deontic Logic}

\emph{Standard Deontic Logic} ($\mathbf{SDL}$) adds the following
Axioms to Propositional Logic (\S\ref{sec:propositional_logic}):
    \[O(A \rightarrow B) \rightarrow (OA \rightarrow OB)\]
    \[PA \rightarrow \neg O \neg A\]
with Primitive Operators $O$ (\emph{Obligatory}) and $P$
(\emph{Permissible}). \emph{Forbidden} is defined as
    \[FA = O \neg A\]
or
    \[FA = \neg P A\]
Deontic Logic may be extended by Alethic Operators with the Axiom:
    \[OA \rightarrow \Diamond A\]
which has the meaning ``ought implies can''.

\subsubsection{Temporal Logic}

\paragraph{Tense Logic} \hfill \\

\emph{Tense Logic} is a 2-modal Logic that adds operators $[F]$ for
\emph{Future} and $[P]$ for \emph{Past} Modalities.

\paragraph{CTL*}
\hfill \\
Superset of \emph{Computation Tree Logic} and \emph{Linear
  Temporal Logic}

\subparagraph{Computation Tree Logic}
\hfill \\
Temporal Logic with \emph{Path} Modalities.

\subparagraph{Linear Temporal Logic}\label{sec:linear_temporal}
\hfill \\
\emph{Linear Temporal Logic} is a Modal Logic with Modalities
referring to Time.

\paragraph{Interval Temporal Logc}

\paragraph{Modal $\mu$-calculus}

\subsubsection{Dynamic Logic}

\emph{Dynamic Logic} adds Terms denoting \emph{Actions}:
\[[a]p\]
where after performing Action $a$ is necessitated that $p$ holds and
\[\langle a \rangle p\]
where after performing Action $a$ it is possible that $p$ holds.

% --------------------------------------------------------------------
\subsection{Intuitionistic Logic}\label{sec:intuitionistic_logic}
% --------------------------------------------------------------------

\emph{Intuitionistic Logic} (or \emph{Constructive Logic}) replaces
Truth with the concept of \emph{Constructive Provability}. This is to
say that Operations in Intuitionistic Logic preserve
\emph{Justification} rather than Truth-value. Such systems are
restrictions of Classical Logic without the Law of the Excluded Middle
or Double Negation Elimination (\S\ref{sec:classical_logic}).

Whereas First-order Logic (\S\ref{sec:predicate_logic}) is a
foundation for Set Theory (Part \ref{sec:set_theory}), Intuitionistic
Logic is used as a foundation for \emph{Type Theory} (Part
\ref{sec:type_theory}) and \emph{Constructive Set Theory}
(\S\ref{sec:constructive_set_theory}).

A Formula in Intuitionistic Logic does not necessarily have a Prenex
Normal Form (\S\ref{sec:prenex_normal}).

Intuitionistic Logic is \emph{Modelled} by \emph{Heyting Algebra}
(\S\ref{sec:heyting_semantics}) or \emph{Kripke Semantics}
(\S\ref{sec:kripke_semantics}) and lacks the Principle of Bivalence
(\S\ref{sec:classical_logic}); thus there is no sole-sufficient
Operator in Intuitionistic Logic. A Formula is Valid if and only if it
receives the Value of the Top Element for any Valuation on any Heyting
Algebra.

Complete bases are:
\[
    \{ \vee, \leftrightarrow, \bot \}
\]
and
\[
    \{ \vee, \leftrightarrow, \neg \}
\]

Proofs (Part \ref{sec:proof_theory}) in a Theory
(\S\ref{sec:formal_theory}), $\mathcal{T}$, based on Intuitionistic
Logic have the \emph{Existence Property}:
\[
    (\exists x)A(x) \in \mathcal{T} \rightarrow (\exists t)A(t)
\]
where $A(x)$ has $x$ as the only Free Variable and $t$ is a Term.

\emph{Disjunction Property}:
\[
    A \vee B \in \mathcal{T} \rightarrow A \in \mathcal{T} \vee B \in \mathcal{T}
\]

\subsubsection{Minimal Logic}



\subsubsection{Combinatory Logic}\label{sec:combinatory_logic}

\emph{Combinator}



\subsubsection{Intermediate Logic}

\emph{Intermediate Logic} is an extended Intuitionistic Logic
(\emph{Superintuitionistic Logic}) that is Consistent and still weaker
than the strongest Consistent Superintuitionistic Logic: Classical
Logic.

% --------------------------------------------------------------------
\subsection{Substructural Logic}\label{sec:substructural_logic}
% --------------------------------------------------------------------

\subsubsection{Relevance Logic}\label{sec:relevance_logic}

\subsubsection{Linear Logic}\label{sec:linear_logic}

\paragraph{Noncommutative Logic}\label{sec:noncommutative_logic}

\emph{Ordered Logic}



% --------------------------------------------------------------------
\subsection{Multi-valued Logic} \label{sec:multi_valued_logic}
% --------------------------------------------------------------------

\subsubsection{Fuzzy Logic}

\subsubsection{Probabilistic Logic}



% --------------------------------------------------------------------
\subsection{Ordinal Logic}
% --------------------------------------------------------------------

Alan Turing's PhD Thesis \cite{turing38}

% --------------------------------------------------------------------
\subsection{Hoare Logic}
% --------------------------------------------------------------------

\emph{Concurrency}

% --------------------------------------------------------------------
\subsection{Categorical Logic}
% --------------------------------------------------------------------

% --------------------------------------------------------------------
\subsection{Paraconsistent Logic}
% --------------------------------------------------------------------

% --------------------------------------------------------------------
\subsection{Term Logic}\label{sec:term_logic}
% --------------------------------------------------------------------



% ====================================================================
\section{Non-monotonic Logic}\label{sec:nonmonotonic_logic}
% ====================================================================

% --------------------------------------------------------------------
\subsection{Default Logic}\label{sec:default_logic}
% --------------------------------------------------------------------

\emph{Default Reasoning} (\S\ref{sec:logical_reasoning})



% --------------------------------------------------------------------
\subsection{Autoepistemic Logic}\label{sec:autoepistemic_logic}
% --------------------------------------------------------------------



% ====================================================================
\section{Heyting Arithmetic}\label{sec:heyting_arithmetic}
% ====================================================================
