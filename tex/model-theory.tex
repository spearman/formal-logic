%%%%%%%%%%%%%%%%%%%%%%%%%%%%%%%%%%%%%%%%%%%%%%%%%%%%%%%%%%%%%%%%%%%%%%
%%%%%%%%%%%%%%%%%%%%%%%%%%%%%%%%%%%%%%%%%%%%%%%%%%%%%%%%%%%%%%%%%%%%%%
\part{Model Theory}\label{part:model_theory}\cite{hodges97}
%%%%%%%%%%%%%%%%%%%%%%%%%%%%%%%%%%%%%%%%%%%%%%%%%%%%%%%%%%%%%%%%%%%%%%
%%%%%%%%%%%%%%%%%%%%%%%%%%%%%%%%%%%%%%%%%%%%%%%%%%%%%%%%%%%%%%%%%%%%%%

\emph{Model Theory} deals with construction and classification of
Structures (\S\ref{sec:structure}) as Models (\S\ref{sec:model}) of
Formal Theories (\S\ref{sec:formal_theory}).

The Language of Model Theory has two Sorts (\S\ref{sec:sort}) of
Mathematical Objects: Sets and Relations (see Part
\ref{part:set_theory} Set Theory). As such Model Theory forms the
Domain of Discourse for Predicate Logic (\S\ref{sec:predicate_logic}).

One definition of Model Theory is the combination of Formal Logic
(Part \ref{part:formal_logic}) with Universal Algebra
(\S\ref{sec:universal_algebra}). An alternative view of Model Theory
equates it with Algebraic Geometry (Part
\ref{part:algebraic_geometry}). The broadest definition of Model
Theory includes four divisions:
\begin{itemize}
  \item Classical Model Theory
  \item Model Theory of Groups and Fields
  \item Computable Model Theory (\S\ref{sec:computable_model})
  \item Geometric Model Theory (\S\ref{sec:geometric_model})
\end{itemize}

Model Theory forms the foundation of \emph{Formal (Truth-conditional)
  Semantics} (Part \ref{part:formal_semantics})-- a reduction of the
Meaning (\S\ref{sec:meaning}) of Assertions in Natural Languages to
their Truth-conditions (\S\ref{sec:truth_condition}). Model Theory
uses Tarski's Semantic Theory of Truth (\S\ref{sec:semantic_truth}) as
the definition of Truth (\S\ref{sec:logical_truth}).



% ====================================================================
\section{Structure}\label{sec:structure}
% ====================================================================

A \emph{Structure} $\mathcal{A}$ is composed of a Set, $A$, called the
\emph{Domain} (\S\ref{sec:domain}) along with a Signature
(\S\ref{sec:signature}), $\sigma$, and an \emph{Interpretation
  Function} (\S\ref{sec:interpretation}), $I$, that maps Symbols in
the Signature (\S\ref{sec:signature}) to their \emph{Meanings}
(\S\ref{sec:meaning}).

Formal definition of a Structure:
\[
  \mathcal{A} = (A, \sigma, I)
\]
with Domain $A$, Signature $\sigma$, and Interpretation Function $I$.

Relative to a $\sigma$-structure $\mathcal{A}$, a
$\sigma(S)$-structure $\mathcal{B}$ is called an \emph{Expansion} of
$\mathcal{A}$ and $\mathcal{A}$ is the \emph{Reduct} of $\mathcal{B}$.

For two $\sigma$-structures $\mathcal{A}$ and $\mathcal{B}$ such that
$dom(\mathcal{A}) \subseteq dom(\mathcal{B})$, $\mathcal{B}$ is then
an \emph{Extension} of $\mathcal{A}$ and $\mathcal{A}$ is a
\emph{Substructure} (\S\ref{sec:substructure}) of $\mathcal{B}$.



% --------------------------------------------------------------------
\subsection{Domain}\label{sec:domain}
% --------------------------------------------------------------------

The \emph{Domain} of a Structure (also called the \emph{Carrier Set},
\emph{Underlying Set}, \emph{Domain of Discourse}, or \emph{Universe};
cf. Set Universe \S\ref{sec:set_universe}) is a Set of Elements over
which the Relations of the Signature are defined, and over which
Variables (\S\ref{sec:variable}) and Quantifiers
(\S\ref{sec:quantifier}) are allowed to range.

The Domain $A$ of a Structure $\mathcal{A}$ may also be written as
$dom(\mathcal{A})$ or $|\mathcal{A}|$ (not to be confused with the
Cardinality of the Domain, $|A|$).



\subsubsection{Closed Subsets}\label{sec:closed_subset}

A Subset of a Domain is a \emph{Closed Subset} if it is closed under
the Operators of the Structure's Signature. For any Subset, $B$, of a
Domain, $|\mathcal{A}|$, there is a \emph{smallest Closed Subset} of
$|\mathcal{A}|$ that contains $B$ called the \emph{Hull} of $B$
denoted by $\langle B \rangle$ or $\langle B \rangle_{\mathcal{A}}$,
which is said to be \emph{Generated} (\S\ref{sec:generator}) by $B$.
$\langle \rangle$ is the \emph{Finitary Closure Operator}
(\S\ref{sec:finitary_closure}).



\subsubsection{Sort}\label{sec:sort}

\emph{One-sorted}

A \emph{Many-sorted} Structure allows an arbitrary number of Domains.

Many-sorted Signature (\S\ref{sec:manysorted_signature})

Many-sorted Logic (\S\ref{sec:manysorted_logic})



% --------------------------------------------------------------------
\subsection{Interpretation Function}
\label{sec:interpretation_function}
% --------------------------------------------------------------------

The \emph{Interpretation Function}, $I$, in a Mathematical Structure,
$\mathcal{A} = (A, \sigma, I)$, maps Predicate and Function Symbols,
$R$ and $f$, of the Signature to Predicates and Functional Predicates
on the Domain, respectively:

$I (\prescript{n}{}{f}) = f^{\mathcal{A}} : A^n \rightarrow A$

$I (\prescript{m}{}{R}) = R^{\mathcal{A}} : A^m \rightarrow \{true,
false\}$ \\
Thus the Interpretation Function gives the Extension of
the Non-logical Symbols (\S\ref{sec:nonlogical_constant}) of a Formal
Language.

Operation = Realized Function

An Interpretation that assigns the Value $true$ to a Sentence or
Theory is called a \emph{Model} (\S\ref{sec:model}) of that Sentence
or Theory, and the Sentence is said to be \emph{Satisfied}
(\S\ref{sec:satisfaction}) by that assignment. A Sentence is Valid
(\S\ref{sec:validity}) if it is Satisfied by every Interpretation (a
Tautology \S\ref{sec:tautology}) and a Sentence is Satisfiable (or
Consistent \S\ref{sec:consistency}) if it is Satisfied by at least one
Interpretation.

If a Sentence $\varphi$ is Satisfied by every Interpretation that
Satisfies another Sentence $\psi$, then $\varphi$ is a Semantic
Consequence (\S\ref{sec:semantic_consequence}) of $\psi$:
\[
  \psi \vDash \varphi
\]

An Interpretation Function of Propositional Logic maps Propositional
Variables to Truth Values and is called a \emph{Truth Assignment}.

An Interpretation Function of First-order Logic maps Function and
Predicate Symbols to Functional Predicates and Predicates on the
Domain and by Induction assigns Terms to Elements in the Domain.

\emph{Stratification}



% --------------------------------------------------------------------
\subsection{Structure Homomorphism}\label{sec:structure_homomorphism}
% --------------------------------------------------------------------

For two Structures $\mathcal{A}$ and $\mathcal{B}$, both with
Signature $\sigma = (F,R,ar)$, a \emph{$\sigma$-homomorphism} $h$ is a
Function:
\[
  h : |\mathcal{A}| \rightarrow |\mathcal{B}|
\]
with Properties:
\begin{itemize}
  \item $\forall \prescript{n}{}{f} \in F,
    a_1, a_2,\ldots, a_n \in |\mathcal{A}|$:
    \[
      h (f^\mathcal{A}(a_1, a_2,\ldots, a_n))
      = f^\mathcal{B} (h(a_1), h(a_2),\ldots, h(a_n))
    \]
  \item $\forall \prescript{n}{}{r} \in R,
    a_1, a_2,\ldots, a_n \in |\mathcal{A}|$:
    \[
      (a_1,a_2,\ldots,a_n) \in r^\mathcal{A} \Rightarrow
      (h(a_1), h(a_2),\ldots,h(a_n)) \in r^\mathcal{B}
    \]
\end{itemize}
A Homomorphism $f : |\mathcal{A}| \rightarrow |\mathcal{A}|$ is an
Endomorphism. An Endomorphism that is also an Isomorphism
(\S\ref{sec:structure_isomorphism}) is called an \emph{Automorphism}.
The \emph{Identity Map} of $|\mathcal{A}|$ is an Automorphism denoted
$1_\mathcal{A}$.

A Structure $\mathcal{B}$ is a \emph{Homomorphic Image} of
$\mathcal{A}$ if there exists a Surjective Homomorphism $g :
\mathcal{A} \rightarrow \mathcal{B}$.

For every Signature $\sigma$ there is a Concrete Category
(\S\ref{sec:concrete_category}) $\sigma$-$\mathbf{Hom}$ with
$\sigma$-structures as Objects and $\sigma$-homomorphisms as
Morphisms.



\subsubsection{Strong Homomorphism}\label{sec:strong_homomorphism}

A $\sigma$-homomorphism $h : \mathcal{A} \rightarrow \mathcal{B}$ is
\emph{Strong} if $\forall \prescript{n}{}{s} \in \sigma, a_1,
a_2,\ldots, a_n \in |\mathcal{A}|$:
\[
  (a_1, a_2,\ldots, a_n) \in s^\mathcal{A} \Leftrightarrow
  (h(a_1), h(a_2),\ldots,h(a_n)) \in s^\mathcal{B}
\]



\subsubsection{Embedding}\label{sec:embedding}

A Strong $\sigma$-homomorphism that is also Injective is called an
\emph{Embedding}, denoted $h : \mathcal{A} \hookrightarrow
\mathcal{B}$, for $\sigma$-structures $\mathcal{A}$ and $\mathcal{B}$,
where:
\begin{itemize}
\item for every $\prescript{n}{}{f} \in \sigma$ and $a_1, \ldots, a_n
  \in |\mathcal{A}|^n$:
  \[
    h(\prescript{n}{}{f^\mathcal{A}}(a_1,\ldots,a_n)) =
    \prescript{n}{}{f^\mathcal{B}}(h(a_1),\ldots,h(a_n))
  \]
\item for every $\prescript{n}{}{R} \in \sigma$ and $a_1, \ldots, a_n
  \in |\mathcal{A}|^n$:
  \[
    \mathcal{A} \models \prescript{n}{}{R}(a_1, \ldots, a_n)
    \Leftrightarrow \mathcal{B} \models \prescript{n}{}{R}(h(a_1),
    \ldots, h(a_n))
  \]
\end{itemize}
Such an Embedding is an \emph{Elementary Embedding} if $h(A)$ is an
Elementary Substructure (\S\ref{sec:substructure}) of $B$.

The Category $\sigma$-$\mathbf{Emb}$ of $\sigma$-structures and
$\sigma$-embeddings is a Concrete Subcategory of
$\sigma$-$\mathbf{Hom}$.



\paragraph{Elementary Embedding}\label{sec:elementary_embedding}\hfill

Preserves and Reflects all First-order Logic over Signature $\sigma$

Elementary Substructure



\paragraph{Structure Isomorphism}\label{sec:structure_isomorphism}\hfill

An Embedding that is Surjective is an \emph{Isomorphism}. The
existence of an Isomorphism between two Structures $\mathcal{A}$ and
$\mathcal{B}$ is denoted $\mathcal{A} \cong \mathcal{B}$ and $\cong$
is an Equivalence Relation on the Class of $\sigma$-structures.



\subsubsection{Retraction}\label{sec:retraction}

For two Structures in a Substructure Relation
(\S\ref{sec:substructure}) $\mathcal{A} \subseteq \mathcal{B}$, a
\emph{Retraction} from $\mathcal{B}$ to $\mathcal{A}$ is a
Homomorphism $h : \mathcal{B} \rightarrow \mathcal{A}$ such that
$\forall a \in |\mathcal{A}|, h(a) = a$. Thus $h$ is Idempotent ($h^2
= h$).

For an Endomorphism $g : \mathcal{B} \rightarrow \mathcal{B}$, if $g^2
= g$, then $g$ is a Retraction to a Substructure $\mathcal{A}$ of
$\mathcal{B}$.



% --------------------------------------------------------------------
\subsection{Parameter}\label{sec:parameter}
% --------------------------------------------------------------------

For a $\sigma$-structure $\mathcal{A}$ with Elements $\bar{a}$ for
which Constant Symbols do not already exist, the addition of Constant
Symbols $\bar{c}$ gives a new Signature $\sigma(\bar{c})$ and
$\bar{a}$ and $\bar{c}$ are called \emph{Parameters}.



% --------------------------------------------------------------------
\subsection{Substructure}\label{sec:substructure}
% --------------------------------------------------------------------

A Structure $\mathcal{A}$ is an \emph{Induced Substructure} of
Structure $\mathcal{B}$ when:
\begin{itemize}
  \item $\sigma(\mathcal{A}) = \sigma(\mathcal{B})$
  \item $A \subseteq B$
  \item $I_{\mathcal{A}}=I_{\mathcal{B}}$
\end{itemize}
denoted by the notation $\mathcal{A} \subseteq \mathcal{B}$ where
$\mathcal{B}$ is called the \emph{Extension} or \emph{Superstructure}
of $\mathcal{A}$.

Given $\sigma$-structures $\mathcal{A}$ and $\mathcal{B}$, if
$|\mathcal{A}| \subseteq |\mathcal{B}|$ and the Inclusion Map
(\S\ref{sec:inclusion_map}) $i : |\mathcal{A}| \rightarrow
|\mathcal{B}|$ is an Embedding (\S\ref{sec:embedding}), then
$\mathcal{A}$ is a \emph{Substructure} of $\mathcal{B}$ and
$\mathcal{B}$ is an \emph{Extension} of $\mathcal{A}$, denoted
$\mathcal{A} \subseteq \mathcal{B}$.

By the definition of Structure Homomorphisms
(\S\ref{sec:structure_homomorphism}), the following Properties apply
to the Function Symbols $f$ and Predicate Symbols $r$ in the Signature
of the two Structures in the Relation $\mathcal{A} \subseteq
\mathcal{B}$:
\begin{itemize}
  \item Constant Symbols $\prescript{0}{}{f}^\mathcal{A}$ are
    equivalent in both Structures:
    \[
      \prescript{0}{}{f}^\mathcal{A} = \prescript{0}{}{f}^\mathcal{B}
    \]
  \item for $n >0$, $\prescript{n}{}{f}^\mathcal{A}$ is the
    Restriction of $\prescript{n}{}{f}^\mathcal{B}$ to the Domain of
    $\mathcal{A}$:
    \[
      \prescript{n}{}{f}^\mathcal{A}
      = \prescript{n}{}{f}^\mathcal{B} |_{|\mathcal{A}|}
    \]
  \item $\prescript{n}{}{r}^\mathcal{A}$ is the Intersection of
    $\prescript{n}{}{r}^\mathcal{B}$ with $|\mathcal{A}|^n$:
    \[
      \prescript{n}{}{r}^\mathcal{A}
      = \prescript{n}{}{r}^\mathcal{B} \cap |\mathcal{A}|^n
    \]
\end{itemize}



When a Structure is applied as a \emph{Model} (\S\ref{sec:model}) of a
particular Theory (\S\ref{sec:formal_theory}), if no extensions of
that Structure result in Theories that are Consistent, that Theory is
termed \emph{Complete}. A Theory $T$ is called \emph{Model Complete}
(\S\ref{sec:model_completion}) if every Substructure of a Model of $T$
is itself a Model of $T$.

Induced Substructures (and \emph{Closed Subsets} described in the next
section) on a Structure form a \emph{Lattice}.



\subsubsection{Elementary Substructure}
\label{sec:elementary_substructure}

For $\sigma$-structures in the Substructure Relation $\mathcal{A}
\subseteq \mathcal{B}$, $\mathcal{A}$ is an \emph{Elementary
  Substructure} if every First-order $\sigma$-formula
$\varphi(\bar{a})$ with Parameters $\bar{a}$ from $\mathcal{A}$ is
True in $\mathcal{A}$ if and only if it is True in $\mathcal{B}$. That
is, Substructure $\mathcal{A}$ is an Elementary Substructure of
$\mathcal{B}$ if $\mathcal{A}$ and $\mathcal{B}$ both \emph{Satisfy}
(\S\ref{sec:satisfaction}) the same Sentences. Here $\mathcal{B}$
would be an \emph{Elementary Extension} of $\mathcal{A}$.

Elementary Embedding (\S\ref{sec:elementary_embedding})



% --------------------------------------------------------------------
\subsection{Generator}\label{sec:generator}
% --------------------------------------------------------------------

For a Subset $Y \subset |\mathcal{B}|$, there is a unique smallest
Substructure $\mathcal{A}$ \emph{Generated} by $Y$ such that $Y
\subseteq |\mathcal{A}|$ called the \emph{Hull} of $Y$, denoted by:
\[
  \mathcal{A} = \langle Y \rangle_\mathcal{B}
\]
The Set $Y$ is said to be a \emph{Set of Generators} for $\mathcal{A}$
and if $Y$ is finite then $\mathcal{A}$ is said to be \emph{Finitely
  Generated}. For a Finite Signature $\sigma$ with no Function
Symbols, every Finitely Generated $\sigma$-structure is Finite.

The following equality holds for the Cardinality of Generated
Structures where the Signature is $\sigma$
(\S\ref{sec:signature_cardinality}):
\[
  |\langle Y \rangle_\mathcal{B}| = |Y| + |\sigma|
\]

Generators and Relations: Presentation (\S\ref{sec:presentation})



% --------------------------------------------------------------------
\subsection{Structure Interpretation}
\label{sec:structure_interpretation}
% --------------------------------------------------------------------

Given two Structures, $M$ and $N$, an \emph{Interpretation} of $M$ in
$N$ is a pair $(n,f)$ where
\begin{itemize}
  \item $n \in \mathbb{N}$
  \item $f:f_{dom} \subset N^n \rightarrow M$ such that the
    $f^k$-preimage of every set $X \subseteq M^k$ definable in $M$ by
    a First-order Formula is definable in $N$ by a First-order Formula
\end{itemize}

Two Structures are \emph{Bi-interpretable} if they can be Interpreted
in each other. This can be used to define an Equivalence Relation
between Structures.

An Interpretation without Parameters (\S\ref{sec:parameter}) may be
called a \emph{$0$-interpretation}.



% --------------------------------------------------------------------
\subsection{Amalgam}\label{sec:amalgam}
% --------------------------------------------------------------------

\subsubsection{Amalgamation Property}\label{sec:amalgamation_property}



% ====================================================================
\section{Satisfaction}\label{sec:satisfaction}
% ====================================================================

Each $\sigma$-Structure $\mathcal{A}$ has a \emph{Satisfaction
  Relation}:
\[
  \mathcal{A} \models \varphi
\]
defined for all Formulas $\varphi$ in the Language of
$\sigma(\bar{c})$ with Parameters (\S\ref{sec:parameter}) $\bar{c}$
being Constants naming Elements of $A = dom(\mathcal{A})$. The
Structure $\mathcal{A}$ is a \emph{Model} (\S\ref{sec:model}) for each
of the Formulas in the Satisfaction Relation.

The Satisfaction Relation is defined Inductively using T-schema
(\S\ref{sec:t_schema}).

Intitution Theory (\S\ref{sec:institution_theory})

If $\varphi(\bar{x})$ is a Formula with Free Variables $\bar{x}$, and
$A$ is a Structure with Elements $\bar{a}$ that gives a Closed Formula
under Substitution (\S\ref{sec:substitution}), then:
\[
  \mathcal{A} \models \varphi(\bar{x})[\bar{a}]
\]

In First-order Logic, Universally Quantified Sentences such as
$\forall x \varphi (x)$ are Satisfied when every Substitution Instance
by an Element in $dom(\mathcal{A})$ is Satisfied, and Existentially
Quantified Sentences such as $\exists x \psi (x)$ are Satisfied when
there is at least one Substitution Instance of an Element from
$dom(\mathcal{A}$ that Satisfies $\psi$.



% --------------------------------------------------------------------
\subsection{Satisfiability}\label{sec:satisfiability}
% --------------------------------------------------------------------

A Formula is \emph{Satisfiable} if there exists a Model
(\S\ref{sec:model}) that makes it True.

A Theory (\S\ref{sec:formal_theory}) is Satisfiable if there exists a
Model that Satisfies all of the Axioms of the Theory.

Satisfiability of Propositional Formulas is a Decidable problem of NP
Complexity % FIXME ref NP complexity

Satisfiability of arbitrary First-order Formulas is Undecidable. By
the Completeness Theorem (\S\ref{sec:completeness}), a First-order
Theory is Satisfiable if and only if it is Consistent.



\subsubsection{Finite Satisfiability}\label{sec:finite_satisfiability}



% --------------------------------------------------------------------
\subsection{Validity}\label{sec:validity}
% --------------------------------------------------------------------

\emph{Valid}, \emph{Logically True}, \emph{Tautological}

A Formula (\S\ref{sec:formula}) is Valid if and only if it is True
under every possible Interpretation (\S\ref{sec:interpretation}).

An Inference is \emph{Semantically Valid}
(\S\ref{sec:semantic_validity}) if all Interpretations that Validate
the Premises also Validate the Conclusion.

A Logical Argument (\S\ref{sec:logical_argument}) is Valid if and only
if its Corresponding Conditional
(\S\ref{sec:corresponding_conditional}) is a Logical Truth ($\top$),
or equivalently, if and only if the Conclusion is Logically Entailed
(\S\ref{sec:logical_consequence}) by the Premises, i.e. there would be
a Contradiction if the Premises were True and the Conclusion was
False.

Validity of arbitrary First-order Formulas is Semidecidable.



% --------------------------------------------------------------------
\subsection{Categoricity}\label{sec:categoricity}
% --------------------------------------------------------------------

A Theory (\S\ref{sec:formal_theory}) is termed \emph{Categorical} if
all its Models are Isomorphic. With this definition and the
L\"owenheim-Skolem Theorem (\S\ref{sec:lowenheim_skolem}) it follows
that any First-order Theory with a Model of Infinite Cardinality can't
be Categorical.

For a Cardinal (\S\ref{sec:cardinal_number}) $\kappa$, a Theory $T$ is
\emph{$\kappa$-Categorical} if any two Models of $T$ of Cardinality
$\kappa$ are Isomorphic to each another. By \emph{Morley's
  Categoricity Theorem}\cite{morley65} if a First-order Theory in a
Countable Language is Categorical in an Uncountable Cardinal $\kappa$,
then it is Categorical in all Uncountable Cardinalities. There are
three possible cases for $\kappa$-Categoricity:
\begin{description}
\item[Totally Categorical]: $\kappa$-Categorical for all Infinite
  Cardinals
\item[Uncountably Categorical]: $\kappa$-Categorical if and only if
  $\kappa$ is an Uncountable Cardinal
\item[Countably Categorical]: $\kappa$-Categorical if and only if
  $\kappa$ is a Countable Cardinal
\end{description}
The special case of $\kappa = \aleph_0$ is called
\emph{$\omega$-Categorical}.



% --------------------------------------------------------------------
\subsection{Definability}\label{sec:definability}
% --------------------------------------------------------------------

An $n$-ary Relation $R$ on the Domain of a Structure $\mathcal{A} =
(A, \sigma, I)$ is \emph{Definable} (or \emph{Explicity Definable} or
\emph{$\emptyset$-definable}) if there is a Formula $\varphi
(x_1,\ldots,x_n)$ such that:
\[
  R = \{(a_1,\ldots,a_n) \in A^n \;|\; \mathcal{A} \models \varphi
    [a_1, \ldots, a_n]\}
\]
or equivalently:
\[
  (a_1, \ldots, a_n) \in R \Leftrightarrow \mathcal{A} \models
    \varphi[a_1, \ldots, a_n]
\]
An Element $a$ of $A$ is Definable in $\mathcal{A}$ if and only if
there is a Formula $\varphi(a)$ such that:
\[
  \mathcal{A} \models \forall x (x = a \leftrightarrow \varphi(x))
\]
A Relation $R$ is \emph{$A$-definable} (or \emph{Definable with
  Parameters}) if there is a Formula $\varphi$ with Parameters
(\S\ref{sec:parameter}) from $\mathcal{A}$ such that $R$ is definable
using $\varphi$. Every Element is Definable using itself as a
Parameters.



\subsubsection{Implicit Definability}\label{sec:implicit_definability}

For a Relation $R$ and a $\sigma$-structure $\mathcal{A}$, $R$ is
\emph{Implicitly Definable} if there is a Formula $\varphi$ in the
Extended Language $\sigma(R)$ and $R$ is the only Relation on
$\mathcal{A}$ such that $\mathcal{A} \models \varphi$. By Beth's
Theorem (\S\ref{sec:beth_definability}), a Relation is Implicitly
Definable if and only if it is Explicitly Definable.



\subsubsection{Beth Definability}\label{sec:beth_definability}



% --------------------------------------------------------------------
\subsection{Discernibility}\label{sec:discernibility}
\cite{ladyman-linnebo-pettigrew11}
% --------------------------------------------------------------------

\emph{Trivially Discernible}



\subsubsection{Absolute Discernibility}\label{sec:absolute_discernibility}

Two objects are \emph{Absolutely Discernible} (or \emph{Monadically
  Discernible}) when one object has any kind of Property that the
other lacks.



\subsubsection{Relative Discernibility}\label{sec:relative_discernibility}

Two objects are \emph{Relatively Discernible} when the first stands in
an Asymmetric Binary Relation to the second, i.e. for objects $a$ and
$b$ and Relation $R$, it is the case that $aRb$ but not the case that
$bRa$.



\subsubsection{Weak Discernibility}\label{sec:weak_discernibility}

Two objects are \emph{Weakly Discernible} when the first stands in an
Irreflexive Binary Relation to the second, i.e. for objects $a$ and
$b$ and Relation $R$, it is the case that $aRb$ but not the case that
$aRa$.

\paragraph{Very Weak Discernibility}\label{sec:very_weak_discernibility}\hfill

\paragraph{Hilbert-Bernays Discernibility}\hfill
\label{sec:hilbert_bernays_discernibility}

\emph{First-order Discernibility}


\subsubsection{Intrinsic Discernibility}\label{sec:intrinsic_discernibility}

% FIXME ref intrinsic property

Two objects are \emph{Intrinsically Discernible} when one object has
an \emph{Intrinsic Property} that the other lacks.



\subsubsection{Strong Indiscernibility}\label{sec:strong_indiscernibility}

Two objects are \emph{Strongly Indiscernible} (or \emph{Utterly
  Indiscernible}) when they are not Discernible in any way, especially
not Weakly Discernible (\S\ref{sec:weak_discernibility}).



% --------------------------------------------------------------------
\subsection{Constraint Satisfaction Problem}
\label{sec:constraint_satisfaction}
% --------------------------------------------------------------------

\fist Decision Problem (\S\ref{sec:decision_problem})

formally a triple $\langle{X,D,C}\rangle$ where $X$ is a Set of Variables, $D$
is the Set of Domains for each Variable, and $C$ is Set of Constraints

a \emph{Constraint} is a Pair $\langle{t,R}\rangle$ where $t \subset X$ is a
Subset of $k$ Variables and $R$ is a $k$-ary Relation on the corresponding
Subset of Domains in $D$

an \emph{Evaluation} of the Variables is a Function from a Subset of Variables
to a particular Set of Values in the corresponding Subset of Domains; an
Evaluation $v$ \emph{Satisfies} the Constraint $\langle{t,R}\rangle$ if the
Values assigned to the Variables $t$ Satisfies the Relation $R$

an Evaluation is \emph{Consistent} if it does not violate \emph{any} of the
Constraints

an Evaluation is \emph{Complete} if it includes \emph{all} of the Variables

an Evaluation is a \emph{Solution} if it is Consistent \emph{and} Complete--
such an Evaluation is said to \emph{Solve} the Constraint Satisfaction Problem



\subsubsection{Boolean Satisfiability Problem}\label{sec:sat}

\emph{SAT}

Conjunctive Normal Form (\S\ref{sec:conjunctive_form})

Chaff

GRASP

generalization: Satisfiability Modulo Theories (SMT \S\ref{sec:smt})




\paragraph{Davis-Putnam-Logemann-Loveland (DPLL) Algorithm}
\label{sec:dpll}\hfill

(Conjunctive Normal Form) CNF-SAT



\paragraph{Conflict-driven Clause Learning}\label{sec:cdcl}\hfill

\emph{CDCL}



\paragraph{Two-watched Literal Scheme}\label{sec:twowatched_literal}\hfill



\subsubsection{Satisfiability Modulo Theories (SMT)}\label{sec:smt}

Quantifier-free Presburger Arithmetic
(\S\ref{sec:presburger_arithmetic})

formalized approach to Constraint Programming

cf. Answer Set Programming (ASP)

(wiki):

generalization of Boolean SAT

an SMT Instance is a Formula in First-order Logic where some Function
and Predicate Symbols have \emph{additional Interpretations} and
\emph{SMT} is the problem (cf. Decision Problem) of determining
whether such a Formula is \emph{Satisfiable}

Predicates are classified according to each respective Theory, e.g.:
\begin{itemize}
  \item Linear Equalities over Real Numbers -- Theory of Linear Real
    Arithmetic
  \item Uninterpreted Terms and Function Symbols -- Theory of
    Uninterpreted Functions with Equality (\S\ref{sec:empty_theory})
\end{itemize}

Quantifier Free fragments

Decidable Theories: %FIXME
\begin{itemize}
  \item Real Closed Fields (\S\ref{sec:real_closed}), Real Numbers --
    Decidable using Quantifier Elimination
  \item Presburger Arithmetic (\S\ref{sec:presburger_arithmetic})
\end{itemize}

DPLL (\S\ref{sec:dpll})

SMTLIB2 -- common interface format $\mono{.smt2}$


\textbf{Z3}

Haskell: Liquid Haskell -- Refinement Type based verifier that can use
any SMTLIB2 compliant solver

ATS: contrib/ATS-extsolve-z3

ATS: contrib/ATS-extsolve-smt2

Rust: $\mono{rustproof}$ ``verification condition generation'' crate

\begin{itemize}
  \item Empty Theory
  \item Linear Arithmetic
  \item Nonlinear Arithmetic
  \item Bitvectors
  \item Arrays
  \item Datatypes
  \item Quantifiers
  \item Strings
\end{itemize}


\textbf{Alt-Ergo}

Frama-C ACSL verifier



\subsubsection{Answer Set Programming (ASP)}\label{sec:asp}

no Quantifiers; cf. Satisfiability Modulo Theories (SMT)

most suited for Boolean problems that reduce to the Free Theory of
Uninterpreted Functions (\S\ref{sec:empty_theory})



\subsubsection{Constraint Logic Programming}
\label{sec:constraint_logic_programming}

Logic Programming extended to include Constraint Satisfaction



% ====================================================================
\section{Model}\label{sec:model}
% ====================================================================

A Structure (\S\ref{sec:structure}) $\mathcal{A}$ that Satisfies a
Sentence $\varphi$, $\mathcal{A} \models \varphi$, is called a
\emph{Model} of $\varphi$ (equivalently ``$\varphi$ is True in
$\mathcal{A}$'').

If $\mathcal{A}$ Satisfies all the Sentences of a Theory
$\mathcal{T}$, $\mathcal{A} \models \mathcal{T}$, is a Model of
$\mathcal{T}$. For $\mathcal{A}$ to be a Model of a $\mathcal{T}$, it
is required that:
\begin{itemize}
  \item The Language of $\mathcal{A}$ is the same as the Language of
    $\mathcal{T}$
  \item Every Sentence in $\mathcal{T}$ is Satisfied by $A$
\end{itemize}
By the Completeness Theorem (\S\ref{sec:completeness}) a Consistent
Theory is Satisfiable, that is, a Theory has a Model if and only if it
is Consistent.

The Compactness Theorem (\S\ref{sec:compactness}) implies that a
Theory has a Model if and only if every Finite Subset of the Sentences
of that Theory also have Models.

A Model is an Analytic example of a Synthetic Theory
(\S\ref{sec:synthetic_theory}). \cite{shulman15}


<http://jdh.hamkins.org/wp-content/uploads/2016/06/Pluralism-inspired-mathematics.pdf>:

Thm. \emph{Every Model of ZFC has an Element that is, externally a
  Model of ZFC. Specifically, if $\langle{M,\in^M}\rangle \vDash ZFC$,
then there is $\langle{m,E}\rangle$ in $M$ which when extracted as an
actual Structure, Satisfies ZFC}



% --------------------------------------------------------------------
\subsection{Standard Model}\label{sec:standard_model}
% --------------------------------------------------------------------

\emph{Standard Model}

Intended Interpretation (\S\ref{sec:intended_interpretation})



% --------------------------------------------------------------------
\subsection{Non-standard Model}\label{sec:nonstandard_model}
% --------------------------------------------------------------------

% --------------------------------------------------------------------
\subsection{Canonical Model}\label{sec:canonical_model}
% --------------------------------------------------------------------

% --------------------------------------------------------------------
\subsection{Realizability Model}\label{sec:realizability_model}
% --------------------------------------------------------------------

Brouwer-Heyting-Kolmogorov Interpretation
(\S\ref{sec:brouwer_heyting_kolmogorov})

Realizability (\S\ref{sec:realizability})

Meaning Explanation of Type Theory (\S\ref{sec:meaning_explanation})

Proof-theoretic Semantics (\S\ref{sec:proof_semantics})



% --------------------------------------------------------------------
\subsection{Elementary Equivalence}\label{sec:elementary_equivalence}
% --------------------------------------------------------------------

Two $\sigma$-structures $M$ and $N$ are \emph{Elementarily Equivalent}
if they both Satisfy the same First-order $\sigma$-sentences.



% --------------------------------------------------------------------
\subsection{Elementary Class}\label{sec:elementary_class}
% --------------------------------------------------------------------

A Class of $\sigma$-structures, $\mathcal{K}$, is an \emph{Elementary
  Class} if there is a First-order Theory $\mathcal{T}$ in the
Language $\sigma$ such that $\mathcal{K}$ contains all Models of
$\mathcal{T}$. Expressed with the Satisfaction Relation
(\S\ref{sec:satisfaction}):
\[
  \mathcal{M} \in \mathcal{K}_\mathcal{T}
  \Leftrightarrow \mathcal{M} \models \mathcal{T}
\]
where $\mathcal{K}_\mathcal{T}$ is an Elementary Class, $\mathcal{M}$
is a Model, and $\mathcal{T}$ is a Theory.

If $\mathcal{T}$ has only a single Sentence, then $\mathcal{K}$ is a
\emph{Basic Elementary Class}. The Reduct (\S\ref{sec:structure}) of
an Elementary Class is a \emph{Pseudoelementary Class}.

Elementary Classes are termed \emph{Axiomatizable in First-Order
  Logic} (or simply \emph{Axiomatizable} when implicitly First-Order).



\subsubsection{Strength}\label{sec:strength}

The notion of \emph{Strength} of Logical Systems
(\S\ref{sec:logical_system}) is defined in terms of Elementary
Classes. A Logical System $\mathcal{S}$ is equal in Strength to
another Logical System $\mathcal{S}'$ when every Elementary Class in
$\mathcal{S}'$ is an Elementary Class in $\mathcal{S}$.



% --------------------------------------------------------------------
\subsection{Model Completion}\label{sec:model_completion}
% --------------------------------------------------------------------

A First-order Theory $\mathcal{T}$ is called \emph{Model Complete} if
every Embedding (\S\ref{sec:embedding}) of Models of $\mathcal{T}$ is
an Elementary Embedding.

A Theory $\mathcal{T}^*$ is a \emph{Companion} of another Theory
$\mathcal{T}$ if every Model of $\mathcal{T}$ can be Embedded in a
Model of $\mathcal{T}^*$ and likewise every Model of $\mathcal{T}^*$
can be Embedded in a Model of $\mathcal{T}$. A \emph{Model Companion}
is a Companion of a Theory that is Model Complete.

A \emph{Model Completion} is a Model Companion $\mathcal{T}^*$ of a
Model $\mathcal{T}$ that has the Amalgamation Property
(\S\ref{sec:amalgamation_property}). This means that every Model of
$\mathcal{T}$ can be uniquiely Embedded in a Model of $\mathcal{T}^*$.



% --------------------------------------------------------------------
\subsection{Diagram}\label{sec:diagram}
% --------------------------------------------------------------------

The (\emph{Robinson}) \emph{Diagram} of a $\sigma$-structure,
$\mathcal{A}$, denoted $diag(\mathcal{A})$, is the Set of all Closed
Literals (\S\ref{sec:literal}) in the Language $\sigma(\bar{c})$ that
are True in $(\mathcal{A}, \bar{a})$ where $\bar{a}$ is a Set of
Generators (\S\ref{sec:generator}) of $\mathcal{A}$ named by Constants
$\bar{c}$. Either the Diagram or Positive Diagram
(\S\ref{sec:positive_diagram}) are not uniquely determined, depending
on the choice of Generators, but they do determine the Model
$\mathcal{A}$ up to Isomorphism.



\subsubsection{Positive Diagram}\label{sec:positive_diagram}

The \emph{Positive Diagram} of a $\sigma$-structure, $\mathcal{A}$,
denoted $diag^+(\mathcal{A})$, is the Set of all Atomic Sentences of
the Language $\sigma(\bar{c})$ that are True in $(\mathcal{A},
\bar{a})$ where $\bar{a}$ is a Set of Generators
(\S\ref{sec:generator}) of $\mathcal{A}$ named by Constants $\bar{c}$.



% --------------------------------------------------------------------
\subsection{Compactness}\label{sec:compactness}
% --------------------------------------------------------------------

The \emph{Compactness Theorem} states that a First-order Theory
(\S\ref{sec:formal_theory}) $\mathcal{T}$ has a Model if and only if
every Finite Subset of $\mathcal{T}$ has a Model.

Compactness (Topology \S\ref{sec:compact_space})



% --------------------------------------------------------------------
\subsection{L\"owenheim Number}\label{sec:lowenheim_number}
% --------------------------------------------------------------------

smallest Cardinal $\kappa$ such that for a Logical System
$\mathcal{L}$ such that if an arbitrary Sentence of $\mathcal{L}$ has
a Model, the Sentence has a Model no larger than $\kappa$.

Non-constructive Proof



\subsubsection{L\"owenheim-Skolem Theorem}\label{sec:lowenheim_skolem}



% --------------------------------------------------------------------
\subsection{Model Existence Theorem}\label{sec:model_existence}
% --------------------------------------------------------------------

% --------------------------------------------------------------------
\subsection{Free Model}\label{sec:free_model}
% --------------------------------------------------------------------

Theory (\S\ref{sec:formal_theory}) $\thy{T}$

Forgetful Functor $U : \cat{Mod}_\thy{T} \rightarrow \cat{Set}$

Free-model Functor $F$

Equational Theory (\S\ref{sec:equational_theory})

Algebraic Theory (\S\ref{sec:algebraic_theory})

Effect Theory (\S\ref{sec:effect_theory})



% ====================================================================
\section{Finite Model Theory}\label{sec:finite_model}
% ====================================================================

\emph{Finite Model Theory} (FMT) is a restriction of Model Theory to
Interpretations of Finite Structures (\S\ref{sec:structure}).

A Finite Structure can always be described by a single First-order
Sentence. An example structure of $n$ Elements:
\[
  \exists x_1 \cdots \exists x_n ( \varphi_1 \wedge \cdots \wedge
  \varphi_m )
\]
This may be extended to a Finite number of Structures:
\[
  \exists x_1 \cdots \exists x_n ( \varphi_1 \wedge \cdots \wedge
  \varphi_m )
  \vee
  \cdots
  \vee
  \exists x_1 \cdots \exists x_p ( \psi_1 \wedge \cdots \wedge
  \psi_q )
\]
Note the difference here with Infinite First-order Model Theory in
which a Model cannot be uniquely determined by a set of First-order
Sentences because of the Compactness Theorem (for every Infinite Model
a Non-isomorphic Model exists).

The ability of a Property (\S\ref{sec:property}) $P$ to be expressed
in First-order Logic may be determined by whether two Structures $A
\in P$ and $B \notin P$ satisfy all the same First-order Sentences:
\[
  A \models \alpha \Leftrightarrow B \models \alpha
\]



% --------------------------------------------------------------------
\subsection{Finite Model Property}\label{sec:finite_model_property}
% --------------------------------------------------------------------

A System of Logic $S$ has the \emph{Finite Model Property}, $fmp(S)$,
if there is a Class of Models $\mathcal{E}$ such that any non-Theorem
of $S$ is Falsified by some Finite Model in $\mathcal{E}$. If
$fmp(S)$, $A$ is a $S$-theorem if and only if $A$ is a Theorem of the
Theory of Finite Models of $S$.

If $S$ is Finitely Axiomatizable (\S\ref{sec:axiom}) and $fmp(S)$ then
it is Decidable (\S\ref{sec:computable_function}).



% ====================================================================
\section{Abstract Model Theory}\label{sec:abstract_model}
% ====================================================================

% --------------------------------------------------------------------
\subsection{Abstract Logic}\label{sec:abstract_logic}
% --------------------------------------------------------------------

An \emph{Abstract Logic} is a Formal System
(\S\ref{sec:formal_system}) that consists of a Class of Sentences with
a Satisfaction Relation (\S\ref{sec:satisfaction}).



% --------------------------------------------------------------------
\subsection{Lindstr\"om's Theorem}\label{sec:lindstroms_theorem}
% --------------------------------------------------------------------

\emph{Lindstr\"om's Theorem} states that First-order Logic is the
Strongest (\S\ref{sec:elementary_class}) Logic which has both
Countable Compactness (\S\ref{sec:compactness}) and the Downward
L\"owenheim-Skolem Property (\S\ref{sec:lowenheim_skolem}).



% --------------------------------------------------------------------
\subsection{Institutional Model Theory}\label{sec:institutional_model}
% --------------------------------------------------------------------

\emph{Institutional Model Theory} generalizes First-order Model Theory
to arbitrary Logical Systems formalized as \emph{Institutions}
(\S\ref{sec:institution_theory}).

Models and Theories may be arbitrary objects along with an assumed
Satisfaction Relation between Models and Sentences within a Context of
a Signature. Signature Morphisms are assumed to preserver
Satisfaction.



% ====================================================================
\section{Computable Model Theory}\label{sec:computable_model}
\cite{harizanov98}
% ====================================================================




% ====================================================================
\section{Geometric Model Theory}\label{sec:geometric_model}
% ====================================================================

% --------------------------------------------------------------------
\subsection{Classification Theory}\label{sec:classification_theory}
% --------------------------------------------------------------------

\emph{Classification Theory} is the division of Theories based on
their \emph{Stability} (\S\ref{sec:model_stability}) which is the
ability of the Models of the Theory to be \emph{Classified}.



% --------------------------------------------------------------------
\subsection{Type}\label{sec:model_type}
% --------------------------------------------------------------------

An \emph{$n$-type} of a Model, $\mathcal{M}$, over a (possibly empty)
Subset of Constants, $A \in M$, is a set of Formulas,
$p(x_1,\ldots,x_n) = p(\mathbf{x})$, with at most $n$ Free Variables
in the Language $L(A)$, formed by adding the members of $A$ to the
Language of $\mathcal{M}$:
\[
  L(A) = L \cup \{ c_a : a \in A \}
\]
such that for every Finite Subset $p_0(\mathbf{x}) \subseteq
p(\mathbf{x})$ there exist Elements $b_1,\ldots,b_n \in M$ with
$\mathcal{M} \models p_0(b_1,\ldots,b_n)$.

A \emph{Complete Type} is \emph{Maximal} (\S\ref{sec:formal_theory}))
under Inclusion such that $\forall \varphi(\mathbf{x}) \in
L(A,\mathbf{x})$ either $\varphi(\mathbf{x}) \in p(\mathbf{x})$ or
$\neg \varphi(\mathbf{x}) \in p(\mathbf{x})$. A non-Complete type is
called a \emph{Partial Type}.

An $n$-type is \emph{Realized} in $\mathcal{M}$ if there is an Element
$\mathbf{b} \in M^n$ such that $\mathcal{M} \models p(\mathbf{b})$.
This is guaranteed by the Compactness Theorem
(\S\ref{sec:compactness}) in either $\mathcal{M}$ or an Elementary
Extension (\S\ref{sec:substructure}) of $\mathcal{M}$. This is denoted
by $tp_{n}^{\mathcal{M}}(\mathbf{b}/A)$ which is read as ``the
Complete Type of $\mathbf{b}$ over $A$''.

A Type $p(\mathbf{x})$ is \emph{Isolated} by a Formula
$\varphi(\mathcal{x})$ if $\forall \psi(\mathbf{x}) \in
p(\mathbf{x})$, $\varphi (\mathbf{x}) \rightarrow
\psi(\mathbf{x})$. Isolated Types are Realized in every Elementary
Substructure or Extension.



\subsubsection{Stone Space}\label{sec:stone_space}

Power (\S\ref{sec:power})

Ultrafilter (\S\ref{sec:ultrafilter})



\paragraph{Stone Duality}\label{sec:stone_duality}\hfill

Every Boolean Algebra is Isomorphic to one consisting of Subsets of a
Set $X$ with Set-theoretical Boolean Operations.

Equivalence between the Category of all Boolean Algebras $\mathbf{BA}$
and the Opposite of the Category of Stone Spaces.



\subsubsection{Saturation}\label{sec:model_saturation}

A Model $\mathcal{M}$ is \emph{$\kappa$-saturated} (where $\kappa$ is
a Cardinal number) if for all $A \subseteq M$ of Cardinality $<
\kappa$, $\mathcal{M}$ Realizes all Complete Types over $A$. A Model
is \emph{Saturated} if it is $|M|$-saturated where $|M|$ is the
Cardinality of $M = dom(\mathcal{M})$.



% --------------------------------------------------------------------
\subsection{Stability}\label{sec:model_stability}
% --------------------------------------------------------------------

A Theory $\mathcal{T}$ is \emph{$\kappa$-stable} for an Infinite
Cardinal $\kappa$ if for every Set $A$ such that $|A| = \kappa$, the
Set of Complete Types over $A$ has Cardinality $\kappa$. Theories are
Classified with the following terms:
\begin{description}
\item [Stable]: $\kappa$-stable for some Infinite Cardinal $\kappa$
\item [Unstable]: not $\kappa$-stable for all Infinite Cardinals
  $\kappa$
\item [Superstable]: $\kappa$-stable for all sufficiently large
  Cardinals $\kappa$
\item [Totally Transcendental]: \emph{Morley Rank}\cite{morley65} less
  than $\infty$
\end{description}



% ====================================================================
\section{Model Checking}\label{sec:model_checking}
% ====================================================================

State Transition System (\S\ref{sec:state_transition}) including
additional Labelling Function for States giving a Kripke Structure
(\S\ref{sec:kripke_structure})

B\"uchi Automata (\S\ref{sec:buchi_automaton}) are used as a
Automata-theoretic version of Formulas in Linear Temporal Logic

Clarke-Grumberg-Peled99 \emph{Model Checking}



% --------------------------------------------------------------------
\subsection{Kripke Structure}\label{sec:kripke_structure}
% --------------------------------------------------------------------

State Transition System (\S\ref{sec:state_transition}) with Labelling
Function for States: represents \emph{Behavior} of the System

%FIXME section for behavior? xref to behavior in other uses?

may be used to Interpret Temporal Logics (\S\ref{sec:temporal_logic};
corresponding by Curry-Howard to Reactive Programming \S\ref{sec:frp}
in Type Theory)


$AP$ of \emph{Atomic Propositions} (Boolean
Expressions over Variables, Constants, and Predicates
\S\ref{sec:propositional_logic})

A \emph{Kripke Structure} $M$ is a $4$-tuple:
\[
  M = (S, I, R, L)
\]
where:
\begin{itemize}
  \item $S$ -- Finite Set of States
  \item $I \subseteq S$ -- Subset of Initial States
  \item $R \subseteq S \times S$ -- Left-total Transition Relation
  \item $L : S \rightarrow 2^{AP}$ -- Labelling
    (\emph{Interpretation}) Function mapping each State $s$ to the set
    of all Atomic Propositions that are Valid in $s$
    %FIXME is this ``valid'' in the logical sense ?
\end{itemize}
$R$ being Left-total implies that it is always possible to take an
Infinite Path through the Kripke Structure.

\emph{Deadlock} is Modelled by a State with a single outgoing
Transition back to itself.

A Path $\rho$ in the Structure $M$ is a Sequence of States $\rho =
s_1s_2s_3\ldots$ such that for each $0 < i$, $R(s_i, s_{i+1})$ holds.

A \emph{Word} $w$ on the Path $\rho$ is a Sequence of Sets of Atomic
Propositions:
\[
  w = L(s_1)L(s_2)L(s_3)\ldots
\]
which is an $\omega$-word (i.e. an Infinite String) over the Alphabet
$2^{AP}$ (\fist $\omega$-languages \S\ref{sec:omega_language})



% --------------------------------------------------------------------
\subsection{Abstract Interpretation}\label{sec:abstract_interpretation}
% --------------------------------------------------------------------

An \emph{Abstract Interpretation} gives several Semantic
Interpretations related by levels of \emph{Abstraction}. The most
precise Semantics are called \emph{Concrete Semantics}.

(wiki):

Computer Science: Theory of Sound approximation of the Semantics of
Computer Programs based on Monotonic Functions over Ordered Sets (esp.
Lattices); can be viewed as a partial execution of a Program which
gains information about its Semantics (e.g. Control Flow, Data Flow)
without performing all the calculations -- Frama-C \emph{Value
  Analysis}

\emph{Concrete Set}

\emph{Abstract Set}

\emph{Abstraction Function}

\emph{Concretization Function}

\emph{Valid Abstraction}

2016 - Cousot, Cousot - A Galois Connection Calculus for Abstract
Interpretation %FIXME

Concrete Domain

Abstract Domain

Galois Connection



\paragraph{Abstract Domain}\label{sec:abstract_domain}\hfill

\emph{Relational Domain}

\emph{Non-relational Domain}



% ====================================================================
\section{Algebraic Logic}\label{sec:algebraic_logic}
% ====================================================================

\emph{Algebraic Logic} is the reasoning arising from the manipulation
of Equations with Free Variables. Algebraic Logic deals with Algebraic
Semantics (\S\ref{sec:algebraic_semantics}) of Classes of Algebras
(\S\ref{sec:algebraic_structure}) which are the specification of
Semantics based on Abstract Algebra (Part
\ref{part:abstract_algebra}). This allows the matching of Logical
Systems with Structures that Model (\S\ref{sec:model}) them.

\emph{Lindenbaum-Tarski}

\fist See also: Universal Algebra
(\S\ref{sec:universal_algebra}), Algebraic Theory
(\S\ref{sec:algebraic_theory})



% --------------------------------------------------------------------
\subsection{Term Algebra}\label{sec:term_algebra}
% --------------------------------------------------------------------

A \emph{Term Algebra} (also termed \emph{Absolutely Free Algebra} or
\emph{Anarchic Algebra}) is an Algebraic Structure Freely Generated
(\S\ref{sec:free_object}) over a given Signature
(\S\ref{sec:signature}).

Given a Signature $\sigma$ and a Set of Variables $X$, the Term
Algebra of $\sigma$ with Basis (\S\ref{sec:free_module}) $X$ is the
Structure (\S\ref{sec:structure}) $A$ with Domain consisting of all
the Terms (\S\ref{sec:term}) of $\sigma$ with Variables taken from
$X$, such that:
\[
  f^A(\bar{t}) = f(\bar{t})
\] and: \[
  R^A = \varnothing
\]
for Function Symbols $f$ and Relation Symbols $R$ in $\sigma$, and
$\bar{t}$ $n$-tuples of Elements in $|A|$ (i.e. Terms). The Signature
of $A$, $\sigma'$, is the Subset of $\sigma$ without any Relation
Symbols.

In Category Theory a Term Algebra is an Initial Algebra
(\S\ref{sec:initial_algebra}) for the Category of all Algebras with a
given Signature.



\subsubsection{Herbrand Universe}\label{sec:herbrand_universe}

A \emph{Herbrand Universe} is the Freely Generated Structure over the
Function Symbols of a Signature, resulting in all Ground Terms
(\S\ref{sec:term}) over that Signature.

A \emph{Herbrand Base} is the set of all Ground Atoms
(\S\ref{sec:atomic_formula}) that can be formed from the Predicate
Symbols of the Signature.

\emph{Herbrand Structure}



% --------------------------------------------------------------------
\subsection{Quotient Algebra}\label{sec:quotient_algebra}
% --------------------------------------------------------------------

\emph{Quotient Algebra} (or \emph{Factor Algebra})

For an Algebra $\mathbf{A}$ with Underlying Set $A$, the
\emph{Quotient Set}, $A / E$ is the Partitioning of $A$ into
Equivalence Classes by a \emph{Congruence Relation}
(\S\ref{sec:congruence_relation}) $E$. Since the Operators are
Compatible with the Equivalence Classes of the Quotient Set, these
Classes are \emph{Quotient Algebras}.



% --------------------------------------------------------------------
\subsection{Representation}\label{sec:model_representation}
% --------------------------------------------------------------------

Representation Theory (\S\ref{sec:representation_theory})

Group Representation (\S\ref{sec:group_representation})

%FIXME



% --------------------------------------------------------------------
\subsection{Duality}\label{sec:duality}
% --------------------------------------------------------------------

% --------------------------------------------------------------------
\subsection{Coherence}\label{sec:coherence}
% --------------------------------------------------------------------

\fist Coherence Theory of Semantic Truth
(\S\ref{sec:coherence_theory})

\emph{Coherence Generalises Duality: A Logical Explanation of
  Multiparty Session Types} --
Carbone-Lindley-Montesi-Schurmann-Wadler16 (Multiparty Session Types)
%TODO relation to semantics ???



% --------------------------------------------------------------------
\subsection{Ultraproduct}\label{sec:ultraproducts}
% --------------------------------------------------------------------

% --------------------------------------------------------------------
\subsection{Abstract Algebraic Logic}
\label{sec:abstract_algebraic_logic}
% --------------------------------------------------------------------

\subsubsection{Leibniz Operator}\label{sec:leibniz_operator}



\subsubsection{Abstract Algebraic Hierarchy}\label{sec:leibniz_hierarchy}

\emph{Abstract Algebraic Hierarchy} (also called the \emph{Leibniz
  Hierarchy})

Equivalential Logics, Algebraizable Logics, Weakly Algebraizable
Logics $\subset$ Protoalgebraic Logics



% --------------------------------------------------------------------
\subsection{Categorical Abstract Algebraic Logic}
\label{sec:categorical_abstract}
% --------------------------------------------------------------------

$\pi$-institution
