%%%%%%%%%%%%%%%%%%%%%%%%%%%%%%%%%%%%%%%%%%%%%%%%%%%%%%%%%%%%%%%%%%%%%%
%%%%%%%%%%%%%%%%%%%%%%%%%%%%%%%%%%%%%%%%%%%%%%%%%%%%%%%%%%%%%%%%%%%%%%
\part{Model Theory}\label{sec:model_theory}\cite{hodges97}
%%%%%%%%%%%%%%%%%%%%%%%%%%%%%%%%%%%%%%%%%%%%%%%%%%%%%%%%%%%%%%%%%%%%%%
%%%%%%%%%%%%%%%%%%%%%%%%%%%%%%%%%%%%%%%%%%%%%%%%%%%%%%%%%%%%%%%%%%%%%%

\emph{Model Theory} deals with construction and classification of
Structures (\S\ref{sec:structure}) as Models
(\S\ref{sec:interpretation}) of Formal Theories
(\S\ref{sec:formal_theory}).

The Language of Model Theory has two Kinds of Mathematical Objects:
Sets and Relations (Part \ref{sec:set_theory}). As such Model Theory
forms the Domain of Discourse for Predicate Logic
(\S\ref{sec:predicate_logic}).

One definition of Model Theory is the combination of Formal Logic
(Part \ref{sec:formal_logic}) with Universal Algebra
(\S\ref{sec:universal_algebra}). An alternative view of Model Theory
equates it with Algebraic Geometry (Part
\ref{sec:algebraic_geometry}). The broadest definition of Model Theory
includes four divisions: Classical Model Theory, Model Theory of
Groups and Fields, Geometric Model Theory, and Computable Model Theory
(\S\ref{sec:computable_model_theory}).



% ====================================================================
\section{Structure}\label{sec:structure}
% ====================================================================

A \emph{Structure} $\mathcal{A}$ is composed of a Set $A$ called the
\emph{Domain} (\S\ref{sec:domain}) along with a collection of Sets of
$n$-ary Function and Relation Symbols called the \emph{Signature}
(\S\ref{sec:signature}) of $\mathcal{A}$.

Formal definition of a Structure:
\[
    \mathcal{A} = (A, \sigma)
\]
with Domain $A$, Signature $\sigma$.

Combining a Structure with an \emph{Interpretation Function} $I$ gives
a \emph{Model} (\S\ref{sec:model}).



% --------------------------------------------------------------------
\subsection{Domain}\label{sec:domain}
% --------------------------------------------------------------------

The \emph{Domain} of a Structure (also called the \emph{Carrier Set},
\emph{Underlying Set}, \emph{Domain of Discourse}, or \emph{Universe})
is a Set of Elements over which the Relations of the Signature of the
Structure are defined. The Domain $A$ of a Structure $\mathcal{A}$ may
also be written as $dom(\mathcal{A})$ or $|\mathcal{A}|$ (not to be
confused with the Cardinality of the Domain, $|A|$).



% --------------------------------------------------------------------
\subsection{Signature}\label{sec:signature}
% --------------------------------------------------------------------

The \emph{Signature} (or \emph{Vocabulary}) of a Structure is the Set
of Operators (Function and Relation Symbols) that characterize it
together with an \emph{Arity Function} ``$ar$'' that maps each
Operator Symbol $s$ to a corresponding Natural Number $n$ which is the
Arity of the Interpretation of $s$: $n = ar(s)$.

A Signature is given as a triple:
\[
    (F,R,ar)
\]
where $F$ is a Set of Function Symbols, and $R$ is a Set of Relation
Symbols (or Predicates) and $ar$ is the Arity Function:
\[
    ar: F \cup R \rightarrow \mathbb{N}_0
\]
If both $F$ and $R$ are Finite Sets, $\sigma$ is a \emph{Finite
  Signature}. A \emph{Nullary Function Symbol} is a \emph{Constant
  Symbol}. A Function or Relation Symbol $s$ of Arity $n$ is sometimes
denoted $\prescript{n}{}{s}$. Allowing for \emph{Infinitary Operators}
leads to a Theory of \emph{Complete Lattices}.

A Structure with Signature $\sigma$ may be called a
\emph{$\sigma$-structure}.

A Signature with no Relation Symbols is an \emph{Algebraic
  Signature}, and the associated Structure an \emph{Algebra}
(\S\ref{sec:universal_algebra}).

A Signature with no Function Symbols is a \emph{Relational Signature},
and the associated Structure is a \emph{Relational Structure}. A
Relational Structure may be used as the basis for a \emph{Relational
  Model} (\S\ref{sec:relational_model}) in \emph{Database Theory}.

The Signature is a synonym for the \emph{Type} of the Structure
(Schematically represented by $\Omega$), and can be written as an
ordered sequence of Natural Numbers representing the arity of the
Operators. The arity, $n$, of a particular Operator symbol, $s$, may
be written $n=ar(s)$.

A \emph{Reduct} of a Structure is created by omitting certain
Operations and Relations from the Signature the converse is
\emph{Expansion} (\S\ref{sec:reduct_expansion}).



\subsubsection{Cardinality}\label{sec:signature_cardinality}

The \emph{Cardinality} of a Signature $\sigma = (F,R,ar)$ is denoted
$|\sigma|$ and is the least Infinite Cardinal
(\S\ref{sec:cardinal_number}) $n$ such that $n \geq (|F| + |R|)$, that
is, greater than or equal to the number of Symbols in the Signature
$\sigma$. This is equal to the number of First-order Formulas
(\S\ref{sec:predicate_logic}) of $\sigma$.



\subsubsection{Reduction \& Expansion}\label{sec:reduct_expansion}

For two Signatures $\sigma^-$ and $\sigma^+$ where $\sigma^- \subset
\sigma^+$, any $\sigma^+$-structure $\mathcal{A}$ is also a
$\sigma^-$-structure called the \emph{$\sigma^-$-reduct} of
$\mathcal{A}$ denoted $\mathcal{A}|\sigma^-$, having the same Domain
as $\mathcal{A}$ (although the Set of Constant Elements corresponding
to Constant Functions in the Signature of $\sigma^-$ may be smaller).

A Homomorphism of $\sigma^+$-structures $h : \mathcal{A} \rightarrow
\mathcal{B}$ is also a Homomorphism of $\sigma^-$-structures $h :
\mathcal{A} | \sigma^- \rightarrow \mathcal{B} | \sigma^-$.

If a Structure $\mathcal{B}$ is a $\sigma^-$-reduct of a
$\sigma^+$-structure $\mathcal{A}$ then $\mathcal{A}$ is called the
\emph{Expansion} of $\mathcal{B}$ to $\sigma^+$. Such an Expansion is
not necessarily unique. The notation for an Expansion by certain
additional Function or Relation Symbols, $f, g, h$, is denoted
$\mathcal{A} = (\mathcal{B}, f, g, h)$.



\subsubsection{Many-sorted Signature}\label{sec:many_sorted_signature}



% --------------------------------------------------------------------
\subsection{Structure Homomorphism}\label{sec:structure_homomorphism}
% --------------------------------------------------------------------

For two Structures $\mathcal{A}$ and $\mathcal{B}$, both with
Signature $\sigma = (F,R,ar)$, a \emph{$\sigma$-homomorphism} $h$ is a
Function:
\[
    h : |\mathcal{A}| \rightarrow |\mathcal{B}|
\]
with Properties:
\begin{itemize}
    \item $\forall \prescript{n}{}{f} \in F,
    a_1, a_2,\ldots, a_n \in |\mathcal{A}|$:
\[
    h (f^\mathcal{A}(a_1, a_2,\ldots, a_n))
    = f^\mathcal{B} (h(a_1), h(a_2),\ldots, h(a_n))
\]
    \item $\forall \prescript{n}{}{r} \in R,
    a_1, a_2,\ldots, a_n \in |\mathcal{A}|$:
\[
    (a_1,a_2,\ldots,a_n) \in r^\mathcal{A} \Rightarrow
    (h(a_1), h(a_2),\ldots,h(a_n)) \in r^\mathcal{B}
\]
\end{itemize}
A Homomorphism $f : |\mathcal{A}| \rightarrow |\mathcal{A}|$ is an
Endomorphism. An Endomorphism that is also an \emph{Isomorphism}
(\S\ref{sec:structure_isomorphism}) is called an \emph{Automorphism}.
The \emph{Identity Map} of $|\mathcal{A}|$ is an Automorphism denoted
$1_\mathcal{A}$.

A Structure $\mathcal{B}$ is a \emph{Homomorphic Image} of
$\mathcal{A}$ if there exists a Surjective Homomorphism $g :
\mathcal{A} \rightarrow \mathcal{B}$.

For every Signature $\sigma$ there is a \emph{Concrete Category}
(\S\ref{sec:concrete_category}) $\sigma$-$\mathbf{Hom}$ with
$\sigma$-structures as Objects and $\sigma$-homomorphisms as
Morphisms.



\subsubsection{Strong Homomorphism}\label{sec:strong_homomorphism}

A $\sigma$-homomorphism $h : \mathcal{A} \rightarrow \mathcal{B}$ is
\emph{Strong} if $\forall \prescript{n}{}{s} \in \sigma, a_1,
a_2,\ldots, a_n \in |\mathcal{A}|$:
\[
    (a_1, a_2,\ldots, a_n) \in s^\mathcal{A} \Leftrightarrow
    (h(a_1), h(a_2),\ldots,h(a_n)) \in s^\mathcal{B}
\]



\subsubsection{Embedding}\label{sec:embedding}

A Strong $\sigma$-homomorphism that is also Injective is called an
\emph{Embedding}.

The Category $\sigma$-$\mathbf{Emb}$ of $\sigma$-structures and
$\sigma$-embeddings is a Concrete Subcategory of
$\sigma$-$\mathbf{Hom}$.



\paragraph{Structure Isomorphism}\label{sec:structure_isomorphism}

An Embedding that is Surjective is an \emph{Isomorphism}. The
existence of an Isomorphism between two Structures $\mathcal{A}$ and
$\mathcal{B}$ is denoted $\mathcal{A} \cong \mathcal{B}$ and $\cong$
is an Equivalence Relation on the Class of $\sigma$-structures.



\subsubsection{Retraction}\label{sec:retraction}

For two Structures in a Substructure Relation
(\S\ref{sec:substructure}) $\mathcal{A} \subseteq \mathcal{B}$, a
\emph{Retraction} from $\mathcal{B}$ to $\mathcal{A}$ is a
Homomorphism $h : \mathcal{B} \rightarrow \mathcal{A}$ such that
$\forall a \in |\mathcal{A}|, h(a) = a$. Thus $h$ is Idempotent ($h^2
= h$).

For an Endomorphism $g : \mathcal{B} \rightarrow \mathcal{B}$, if $g^2
= g$, then $g$ is a Retraction to a Substructure $\mathcal{A}$ of
$\mathcal{B}$.



% --------------------------------------------------------------------
\subsection{Substructure}\label{sec:substructure}
% --------------------------------------------------------------------

Given $\sigma$-structures $\mathcal{A}$ and $\mathcal{B}$, if
$|\mathcal{A}| \subseteq |\mathcal{B}|$ and the Inclusion Map
(\S\ref{sec:inclusion_map}) $i : |\mathcal{A}| \rightarrow
|\mathcal{B}|$ is an Embedding (\S\ref{sec:embedding}), then
$\mathcal{A}$ is a \emph{Substructure} of $\mathcal{B}$ and
$\mathcal{B}$ is an \emph{Extension} of $\mathcal{A}$, denoted
$\mathcal{A} \subseteq \mathcal{B}$.

By the definition of Structure Homomorphisms, the following Properties
apply to the Function Symbols $f$ and Relation Symbols $r$ in the
Signature of the two Structures in the Relation $\mathcal{A} \subseteq
\mathcal{B}$:
\begin{itemize}
    \item Constant Symbols $\prescript{0}{}{f}^\mathcal{A}$ are
      equivalent in both Structures:
\[
    \prescript{0}{}{f}^\mathcal{A} = \prescript{0}{}{f}^\mathcal{B}
\]
    \item for $n >0$, $\prescript{n}{}{f}^\mathcal{A}$ is the
      Restriction of $\prescript{n}{}{f}^\mathcal{B}$ to the Domain of
      $\mathcal{A}$:
\[
    \prescript{n}{}{f}^\mathcal{A}
    = \prescript{n}{}{f}^\mathcal{B} |_{|\mathcal{A}|}
\]
    \item $\prescript{n}{}{r}^\mathcal{A}$ is the Intersection of
      $\prescript{n}{}{r}^\mathcal{B}$ with $|\mathcal{A}|^n$:
\[
    \prescript{n}{}{r}^\mathcal{A}
    = \prescript{n}{}{r}^\mathcal{B} \cap |\mathcal{A}|^n
\]
\end{itemize}



\subsubsection{Generator} \label{sec:generator}

For a Subset $Y \subset |\mathcal{B}|$, there is a unique smallest
Substructure $\mathcal{A}$ \emph{Generated} by $Y$ such that $Y
\subseteq |\mathcal{A}|$ called the \emph{Hull} of $Y$, denoted by:
\[
    \mathcal{A} = \langle Y \rangle_\mathcal{B}
\]
The Set $Y$ is said to be a \emph{Set of Generators} for $\mathcal{A}$
and if $Y$ is finite then $\mathcal{A}$ is said to be \emph{Finitely
  Generated}. For a Finite Signature $\sigma$ with no Function
Symbols, every Finitely Generated $\sigma$-structure is Finite.

The following equality holds for the Cardinality of Generated
Structures where the Signature is $\sigma$
(\S\ref{sec:signature_cardinality}):
\[
  |\langle Y \rangle_\mathcal{B}| = |Y| + |\sigma|
\]



% ====================================================================
\section{Valuation}\label{sec:valuation}
% ====================================================================



% ====================================================================
\section{Interpretation}\label{sec:interpretation}
% ====================================================================

A Structure that can be given as an \emph{Interpretation} of a Formal
Theory (\S\ref{sec:formal_theory}) is called a \emph{Model} of that
Theory. That is, an Interpretation is a Model if it assigns Truth
values to the Sentences of a Theory. Model Theory uses Tarski's
Semantic Theory of Truth (\S\ref{sec:semantic_truth}) as the
definition of Truth. Model Theory also forms the foundation of
\emph{Formal (Truth-conditional) Semantics}-- a reduction of the
Meaning of Assertions in Natural Languages to their Truth-conditions.
A Model is an Analytic example of a Synthetic Theory. \cite{shulman15}

In Model Theory an Interpretation fixes the \emph{Domain of Discourse}
for expressions of Symbolic Logic (Part \ref{sec:formal_logic}); that
is it assigns the Sets and Relations that Variables can range over.

Roughly, an Interpretation of a Formal Language is an assignment of
\emph{Meanings} to Symbols and \emph{Truth-Conditions}
(\S\ref{sec:semantic_truth}) to Sentences. An Interpretation of
First-order Logic Maps Terms to Individuals in the Universe and
Propositions to Truth Values.

The Interpretation Function in a Mathematical Structure maps Function
and Relation Symbols of the Signature to actual Functions and
Relations on the Domain:
\[
    f^{\mathcal{A}} = I (f)
\]
\[
    R^{\mathcal{A}} = I (R) \subseteq A^{ar(R)}
\]
A Constant (Nullary) Symbol is identified with an Element of the
Domain:
\[
    I(c) \in A
\]

Thus the Interpretation Function is the \emph{Extension}
(\S\ref{sec:set_property}) of the Symbols and Strings of Symbols of
the Object Language.

\emph{Intended Interpretation}, \emph{Standard Model}

\emph{Stratification}



% ====================================================================
\section{Satisfaction}\label{sec:satisfaction}
% ====================================================================

When defining a Theory as a set of Sentences in a Formal Language, a
\emph{Model} (\S\ref{sec:model}) is a Structure
(\S\ref{sec:structure}) that \emph{Satisfies} the Sentences of that
Theory. For a Formula $\phi$ and a Structure $\mathcal{M}$, a
\emph{Satisfaction Relation} is denoted:
\[
    \mathcal{M} \vDash \phi
\]
For $\mathcal{M}$ to be a Model of a Theory, $T$, it is required that:
\begin{itemize}
\item The Language of $\mathcal{M}$ is the same as the Language of $T$
\item Every Sentence in $T$ is Satisfied by $\mathcal{M}$
\end{itemize}
By the Completeness Theorem (\S\ref{sec:completeness}) a Consistent
Theory is Satisfiable, that is, a Theory has a Model if and only if it
is Consistent. The Compactness Theorem
(\S\ref{sec:compactness_theorem}) implies that a Theory has a Model if
and only if every Finite Subset of the Sentences in that Theory also
have Models.



% --------------------------------------------------------------------
\subsection{Model}\label{sec:model}
% --------------------------------------------------------------------



% --------------------------------------------------------------------
\subsection{Compactness Theorem}\label{sec:compactness_theorem}
% --------------------------------------------------------------------

The \emph{Compactness Theorem} states that a Set of First-order
Sentences (\S\ref{sec:firstorder_logic}) has a Model if and only if
every Finite Subset of it has a Model.

Compactness (Topology \S\ref{sec:compactness})



% --------------------------------------------------------------------
\subsection{Elementary Equivalence}\label{sec:elementary_equivalence}
% --------------------------------------------------------------------

Two $\sigma$-structures $M$ and $N$ are \emph{Elementarily Equivalent}
if they both Satisfy the same First-order $\sigma$-sentences.



% --------------------------------------------------------------------
\subsection{L\"owenheim-Skolem Theorem}\label{sec:lowenheim_skolem}
% --------------------------------------------------------------------



% ====================================================================
\section{Conservativity}\label{sec:conservativity}
% ====================================================================



% ====================================================================
\section{Quantifier Elimination}
% ====================================================================

Within a Theory $T$, if every First-order Formula $\varphi(x_1,
\ldots, x_n)$ with Quantifiers is equivalent to a First-order Formula
$\psi(x_1, \ldots, x_n)$ without Quantifiers, $T$ is said to have the
property of \emph{Quantifier Elimination}. A Theory without Quantifier
Elimination may be made to have it by adding Symbols to its Signature.



% ====================================================================
\section{Model Completion}\label{sec:model_completion}
% ====================================================================

A First-order Theory $T$ is called \emph{Model Complete} if every
Embedding (\S\ref{sec:sigma_embedding}) of Models of $T$ is an
Elementary Embedding.

A Theory $T^*$ is a \emph{Companion} of another Theory $T$ if every
Model of $T$ can be Embedded in a Model of $T^*$ and likewise every
Model of $T^*$ can be Embedded in a Model of $T$. A \emph{Model
  Companion} is a \emph{Companion} of a Theory that is \emph{Model
  Complete}.

%FIXME ref Amalgamation Property
A \emph{Model Completion} is a Model Companion $T^*$ of a Model $T$
that has the \emph{Amalgamation Property}. This means that every Model
of $T$ can be uniquiely Embedded in a Model of $T^*$.



% ====================================================================
\section{Categoricity}
% ====================================================================

%FIXME ref Cardinal, Lowenheim-Skolem
A Theory is termed \emph{Categorical} if all its Models are
Isomorphic. With this definition and the L\"owenheim-Skolem Theorem it
follows that any First-order Theory with a Model of infinite
Cardinality can't be Categorical.

For a Cardinal $\kappa$, a Theory $T$ is \emph{$\kappa$-Categorical}
if any two Models of $T$ of Cardinality $\kappa$ are Isomorphic to one
another. By \emph{Morley's Categoricity Theorem}\cite{morley65} if a
First-order Theory in a Countable Language is Categorical in an
Uncountable Cardinal $\kappa$, then it is Categorical in all
Uncountable Cardinalities. There are three possible cases for
$\kappa$-Categoricity:
\begin{description}
\item[Totally Categorical] $\kappa$-Categorical for all Infinite
  Cardinals
\item[Uncountably Categorical] $\kappa$-Categorical if and only if
  $\kappa$ is an Uncountable Cardinal
\item[Countably Categorical] $\kappa$-Categorical if and only if
  $\kappa$ is a Countable Cardinal
\end{description}
The special case of $\kappa = \aleph_0$ is called
\emph{$\omega$-Categorical}.



% ====================================================================
\section{Interpretability}
% ====================================================================

Given two Structures, $M$ and $N$, an \emph{Interpretation} of $M$ in
$N$ is a pair $(n,f)$ where
\begin{itemize}
    \item $n \in \mathbb{N}$
    \item $f:f_{dom} \subset N^n \rightarrow M$ such that the
      $f^k$-preimage of every set $X \subseteq M^k$ definable in $M$
      by a First-order Formula is definable in $N$ by a First-order
      Formula
\end{itemize}

Two Structures are \emph{Bi-interpretable} if they can be interpreted
in each other. This can be used to define an Equivalence Relation
between Structures.



% ====================================================================
\section{Abstract Model Theory}
% ====================================================================

% --------------------------------------------------------------------
\subsection{Abstract Logic}
% --------------------------------------------------------------------

An \emph{Abstract Logic} is a Formal System that consists of a Class
of Sentences with a Satisfaction Relation
(\S\ref{sec:satisfaction}).

%FIXME compactness, lowenheim-skolem
\emph{Lindstr\"om's Theorem} states that First-order Logic is the
Strongest (\S\ref{sec:elementary_class}) Logic which has both
Countable Compactness and the Downward L\"owenheim-Skolem Property.



% --------------------------------------------------------------------
\subsection{Institutional Model Theory}
% --------------------------------------------------------------------

\emph{Institutional Model Theory} generalizes First-order Model Theory
to arbitrary Logical Systems formalized as \emph{Institutions}
(\S\ref{sec:institution_theory}).



% ====================================================================
\section{Finite Model Theory}
% ====================================================================

\emph{Finite Model Theory} (FMT) is a restriction of Model Theory to
Interpretations of Finite Structures.

A Finite Structure can always be described by a single First-order
Sentence. An example structure of $n$ Elements:
\[
    \exists x_1 \cdots \exists x_n ( \varphi_1 \wedge \cdots \wedge
    \varphi_m )
\]
This may be extended to a Finite number of Structures:
\[
    \exists x_1 \cdots \exists x_n ( \varphi_1 \wedge \cdots \wedge
    \varphi_m )
    \vee
    \cdots
    \vee
    \exists x_1 \cdots \exists x_p ( \psi_1 \wedge \cdots \wedge
    \psi_q )
\]
Note the difference here with Infinite First-order Model Theory in
which a Model cannot be uniquely determined by a set of First-order
Sentences because of the Compactness Theorem (For every Infinite Model
a Non-isomorphic Model exists).

The ability of a \emph{Property} (\S\ref{sec:set_property}) $P$ to
be expressed in First-order Logic may be determined by whether two
Structures $A \in P$ and $B \notin P$ satisfy all the same First-order
Sentences:
\[
    A \vDash \alpha \leftrightarrow B \vDash \alpha
\]



% --------------------------------------------------------------------
\subsection{Finite Model Property}
% --------------------------------------------------------------------

A System of Logic $S$ has the \emph{Finite Model Property} if there is
a Class of Models $\mathrm{M}$ such that any non-Theorem of $S$ is
Falsified by some Finite Model in $\mathrm{M}$. If $fmp(S)$, $A$ is a
$S$-theorem if and only if $A$ is a Theorem of the Theory of Finite
Models of $S$.

If $S$ is Finitely Axiomatizable (\S\ref{sec:axiom}) and $fmp(S)$ then
it is Decidable (\S\ref{sec:computable_function}).



% ====================================================================
\section{Computable Model Theory}\label{sec:computable_model_theory}
% ====================================================================

\cite{harizanov98}



% ====================================================================
\section{Geometric Model Theory}
% ====================================================================

% --------------------------------------------------------------------
\subsection{Classification Theory}
% --------------------------------------------------------------------

\emph{Classification Theory} is the division of Theories based on
their \emph{Stability} which is the ability of the Models of the
Theory to be \emph{Classified}.



% --------------------------------------------------------------------
\subsection{Types}
% --------------------------------------------------------------------

An \emph{$n$-type} of a Model, $\mathcal{M}$, over a (possibly empty)
Subset of Constants, $A \in M$, is a set of Formulas,
$p(x_1,\ldots,x_n) = p(\mathbf{x})$, with at most $n$ Free Variables
in the Language $L(A)$, formed by adding the members of $A$ to the
Language of $\mathcal{M}$:
\[
    L(A) = L \cup \{ c_a : a \in A \}
\]
such that for every Finite Subset $p_0(\mathbf{x}) \subseteq
p(\mathbf{x})$ there exist Elements $b_1,\ldots,b_n \in M$ with
$\mathcal{M} \vDash p_0(b_1,\ldots,b_n)$.

A \emph{Complete Type} is \emph{Maximal}
(\S\ref{sec:formal_theory})) under Inclusion such that $\forall
\phi(\mathbf{x}) \in L(A,\mathbf{x})$ either $\phi(\mathbf{x}) \in
p(\mathbf{x})$ or $\neg \phi(\mathbf{x}) \in p(\mathbf{x})$. A
non-Complete type is called a \emph{Partial Type}.

An $n$-type is \emph{Realized} in $\mathcal{M}$ if there is an Element
$\mathbf{b} \in M^n$ such that $\mathcal{M} \vDash
p(\mathbf{b})$. This is guaranteed by the Compactness Theorem
(\S\ref{sec:compactness_theorem}) in either $\mathcal{M}$ or an
Elementary Extension (\S\ref{sec:model_substructure}) of
$\mathcal{M}$. This is denoted by $tp_{n}^{\mathcal{M}}(\mathbf{b}/A)$
which is read as ``the Complete Type of $\mathbf{b}$ over $A$''.

A Type $p(\mathbf{x})$ is \emph{Isolated} by a Formula
$\varphi(\mathcal{x})$ if $\forall \psi(\mathbf{x}) \in
p(\mathbf{x})$, $\varphi (\mathbf{x}) \rightarrow
\psi(\mathbf{x})$. Isolated Types are Realized in every Elementary
Substructure or Extension.



\subsubsection{Saturation}\label{sec:model_saturation}

A Model $\mathcal{M}$ is \emph{$\kappa$-saturated} (where $\kappa$ is
a Cardinal number) if for all $A \subseteq M$ of Cardinality $<
\kappa$, $M$ Realizes all Complete Types over $A$. A Model is
\emph{Saturated} if it is $|M|$-saturated where $|M|$ is the
Cardinality of $M$.



% --------------------------------------------------------------------
\subsection{Stability}\label{sec:model_stability}
% --------------------------------------------------------------------

A Theory $T$ is \emph{$\kappa$-stable} for an Infinite Cardinal $\kappa$
if for every set $A$ such that $|A| = \kappa$, the Set of Complete
Types over $A$ has Cardinality $\kappa$. Theories are Classified with
the following terms:
\begin{description}
\item [Stable] $\kappa$-stable for some Infinite Cardinal $\kappa$
\item [Unstable] not $\kappa$-stable for all Infinite Cardinals $\kappa$
\item [Superstable] $\kappa$-stable for all sufficiently large
  Cardinals $\kappa$
\item [Totally Transcendental] \emph{Morley Rank}\cite{morley65} less
  than $\infty$
\end{description}



% ====================================================================
\section{Inner Model Theory}\label{sec:inner_model_theory}
% ====================================================================
