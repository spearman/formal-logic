%%%%%%%%%%%%%%%%%%%%%%%%%%%%%%%%%%%%%%%%%%%%%%%%%%%%%%%%%%%%%%%%%%%%%%
%%%%%%%%%%%%%%%%%%%%%%%%%%%%%%%%%%%%%%%%%%%%%%%%%%%%%%%%%%%%%%%%%%%%%%
\part{Systems Theory}\label{sec:systems_theory}
%%%%%%%%%%%%%%%%%%%%%%%%%%%%%%%%%%%%%%%%%%%%%%%%%%%%%%%%%%%%%%%%%%%%%%
%%%%%%%%%%%%%%%%%%%%%%%%%%%%%%%%%%%%%%%%%%%%%%%%%%%%%%%%%%%%%%%%%%%%%%

%FIXME new tex file?

Operad Theory (\S\ref{sec:operad_theory}): Systems of Systems

Operad (\S\ref{sec:operad}): compositional style

Algebra (??? \S\ref{sec:universal_algebra}): System type



% ====================================================================
\section{State Variable}\label{sec:state_variable}
% ====================================================================

% ====================================================================
\section{Steady State}\label{sec:steady_state}
% ====================================================================

``Fixed-point'', ``Equilibria'' %FIXME



% ====================================================================
\section{Closed System}\label{sec:closed_system}
% ====================================================================

%FIXME

Actor Model (\S\ref{sec:actor_model}) -- \emph{Computational
  Representation Theorem} -- Consequence: a Finite Actor can
Nondeterministically Respond with an Uncountable number of different
Outputs %FIXME



% ====================================================================
\section{Open System}\label{sec:open_system}
% ====================================================================

(2016 - Fong - The Algebra of Open and Interconnected Systems):

\fist Hypergraph (\S\ref{sec:hypergraph}), Hypergraph Category
(\S\ref{sec:hypergraph_category})

Interconnection (``Integration'') of Systems modelled by Cospans
(\S\ref{sec:cospan})

Principle of Compositionality

``Network-style Diagrammatic Languages''

Network Diagram (\S\ref{sec:network_diagram}), ``Network-style
Diagrammatic Languages'' -- Electrical Circuits

other examples: Signal Flow Graphs (\S\ref{sec:signal_flow}),
Markov Processes (\S\ref{sec:markov_process}), Automata
(\S\ref{sec:automaton}), Petri Nets (\S\ref{sec:petri_net}), Chemical
Reaction Networks
%FIXME

\emph{Terminal} -- ``point of Interconnection''; ``boundary''
``marked'' using Cospan (\S\ref{sec:decorated_cospan}), Connected to
others using Pushouts (\S\ref{sec:pushout}); having a ``marked
boundary'' in this way makes a Closed System into an Open System;
process of ``freely''  ``marking boundaries'' constructs a Hypergraph
Category (\S\ref{sec:hypergraph_category}) from a ``library'' of
Network pieces

\emph{Components} with multiple Input/Output Terminals (possibly
labelled with some Type) connected to form a larger \emph{Network}

Components form \emph{Hyperedges} between labelled Vertices

\begin{itemize}
  \item each Terminal of an Open System may make ``Measurements''
    appropriate to the ``Type'' of the Terminal
  \item given a collection of Terminals, the \emph{Universum} is the
    Set of all possible Measurement outcomes
  \item each Open System has a collection of Terminals (and a Universum)
  \item the Semantics of an Open System is the Subset of Measurement
    outcomes on the Terminals that are ``permitted'' by the System,
    known as the \emph{Behavior} of the System
\end{itemize}

%FIXME universum = phase space?

``Laws'' (e.g. Ohm's Law) are mechanisms for Partitioning Behaviors
into \emph{Possible} and \emph{Impossible} Behaviors

given a Universum $\class{U}$, a Behavior of a System is an Element
of the Power Set $\pow(\class{U})$ (representing all possible
Measurements of the System), and a Law is an Element of
$\pow(\pow(\class{U}))$ representing all possible Behaviors of a
\emph{Class} of Systems

\emph{Interconnection} of Terminals asserts the Identification of
Variables at the Identified Terminals

Algebra of Semantic Objects and Homomorphism from Syntax to Semantics
(Principle of Compositionality \S\ref{sec:compositionality})



% ====================================================================
\section{Dynamical System}\label{sec:dynamical_system}
% ====================================================================

A \emph{Dynamical System} is defined as a tuple $(T,M,\Phi)$ where $T$
is a Monoid (\S\ref{sec:monoid}), M is a Set and $\Phi$ is a Function
(\S\ref{sec:set_function}).

Discrete, Continuous, Hybrid

Control Theory (\S\ref{sec:control_theory})

Initial Conditions, Driving Functions

%FIXME relate to transition systems, automata ?

\url{https://www.youtube.com/watch?v=cu718EbCOPs} Spivak 16:

can't have Identities and Feedback without Partiality %FIXME

Traced Ideals %FIXME



% ====================================================================
\section{Communicating System}\label{sec:communicating_system}
% ====================================================================

Denielou-Yoshida13 -- arises from a Choreographed collection of
Communicating Finite State Machines (CFSMs
\S\ref{sec:communicating_fsm}) in the context of Multiparty Session
Types (\S\ref{sec:multiparty_session})

Linear Multirole Logic (LMRL \S\ref{sec:lmrl})

\fist Process Calculus (\S\ref{sec:process_calculus})

Lange-Tuosto-Yoshida15:

\emph{Generalized Multiparty Compatibility} (GMC) -- Decidable
Condition characterizing a Set of Communicating Systems for which
questions of Safety (Deadlock-freedom, Orphan Messages, and
Unspecified Reception Configuration) can be Decided

\emph{Communicating System}:
\[
  \class{S} = (M_p)_{p \in \class{P}}
\]
where $M_p = (Q_p, q_{0p}, \Sigma, \delta_p)$ are CFSMs for each
Participant $p \in \class{P}$

\emph{Configuration} of $\class{S}$ %FIXME



% --------------------------------------------------------------------
\subsection{Choreography}\label{sec:choreography}
% --------------------------------------------------------------------

\emph{Choreographies}: ``Models of Interactions among Software
Components from a \emph{Global} point of view'' --
Lange-Tuosto-Yoshida15

\fist Global Types (\S\ref{sec:global_type}) give \emph{Choreographic
  Specifications} of Interactions (\S\ref{sec:interaction_geometry})



\subsubsection{Graphical Choreography}\label{sec:graphical_choreography}

or \emph{Global Graphs}

\begingroup

\newcommand{\party}{\mono}

Lange-Tuosto-Yoshida15:

Communicating Finite State Machines (\S\ref{sec:communicating_fsm})

CFSMs as Behavioral Specifications of Distributed Components from
which a Choreography can be built

Sound and Complete characterization of ``safe'' CFSMs from which
Global Graphs can be constructed

CFSM model based on Asynchronous FIFO Message-passing Communication

Algorithm produces a Choreography expressed as a Global Graph (closely
related to BPMN 2.0 Choreography) given a Set of CFSMs (a Set of
Behavioral Specifications of components Interacting through
Asynchronous FIFO Message Passing)

example implementation for Golang in Ng-Yoshida16

\fist \emph{Note on notation}: Lange-Tuosto-Yoshida15 uses $A$ for
Alphabet and $Act$ for Actions, here we use $\Sigma$ for Alphabet and
$A$ for Actions

$\class{P}$ -- Finite Set of \emph{Participants}

$\party{p, q, r, s}, \ldots$ -- ranges over Participants

$\Sigma$ -- Finite Alphabet

$C \defeq \{ \party{pq}
  \ |\ \party{p,q} \in \class{P}, \party{p \neq q} \}$
-- Set of \emph{Channels}

$A \defeq C \times \{!,?\} \times \Sigma$ -- Set of \emph{Actions}

$\ell$ -- ranges over Actions

$\Sigma^*$, $A^*$ -- Set of Finite Words on $\Sigma$, $A$

$\varphi$ -- ranges over $\Sigma^*$, $A^*$

CFSM as a $4$-tuple:
\[
  M = (Q, q_0, \Sigma, \delta)
\]

$Q$ -- Finite Set of \emph{States}

$q_0 \in Q$ -- \emph{Initial State}

$\delta \subseteq Q \times A \times Q$ -- Set of \emph{Transitions}

Transitions are \emph{Labelled} by Actions

TODO: could a Categorization be given in terms of Decorated Cospan
Categories (\S\ref{sec:decorated_cospan_category}) ???

$\party{pq}!a$ -- Label representing Sending a Message $a$ from
$\party{p}$ to $\party{q}$

$\party{pq}?a$ -- Label representing Receiving a Message $a$ by
$\party{q}$ from $\party{p}$

$\class{L}(M) \subseteq A^*$ -- Language on $A$ Accepted by the
Automaton corresponding to $M$ where each State of $M$ is an Accepting
State %TODO explain accepting state

a \emph{Final State} $q \in Q$ has no Outgoing Transitions

a \emph{Sending State} $q \in Q$ has all Outgoing Transitions that are
Labelled by Send Actions

a \emph{Receiving State} $q \in Q$ has all Outgoing Transitions that are
Labelled by Receive Actions

a \emph{Mixed State} $q \in Q$ has both Send and Receive Actions as
Outgoing Transitions

CFSM $M$ is \emph{Deterministic} if for all States $q \in Q$ and all
Actions $\ell \in A$:
\begin{itemize}
  \item if $(q,\ell,q'),(q,\ell,q'') \in \delta$ then $q = q''$
\end{itemize}

CFSM $M$ is \emph{Minimal} if there are no other CFSMs $M'$ with fewer
States and Transitions such that $\class{L}(M) = \class{L}(M')$

\emph{Communicating System}:
\[
  \class{S} = (M_p)_{p \in \class{P}}
\]
where $M_p = (Q_\party{p}, q_{0\party{p}}, \Sigma, \delta_\party{p})$
are CFSMs for each Participant $\party{p} \in \class{P}$

$s = (\vec{q};\vec{w})$ -- \emph{Configuration} of $\class{S}$
where:
\begin{itemize}
  \item $\vec{q} = (q_\party{p})_{\party{p} \in \class{P}})$ with
    $q_\party{p} \in Q_\party{p}$ is the \emph{Control State} and
    $q_\party{p} \in Q_\party{p}$ is the \emph{Local State} of Machine
    $M_\party{p}$
  \item $\vec{w} = (w_\party{pq})_{\party{pq} \in C}$ with
    $w_\party{pq} \in \Sigma^*$
\end{itemize}

\emph{Initial Configuration} of $\class{S}$:
\[
  s_0 = (\vec{q}_0;\vec{\varepsilon})
\]
with $\vec{q}_0 = (q_{0\party{p}})_{\party{p} \in \class{P}}$

A Configuration $s' = (\vec{q}',\vec{w}')$ is \emph{Reachable} from
Configuration $s = (\vec{q},\vec{w})$ by Transition $\ell$, written $s
\xrightarrow{\ell} s'$, if there is an Symbol (Message) $a \in \Sigma$
such that either:
\begin{itemize}
  \item $\ell = \party{sr}!a$ and $(q_\party{s},\ell,q_\party{s}') \in
    \delta_\party{s}$
  \item $q_\party{p}' = q_\party{p}$ for all $\party{p \neq s}$
  \item $w_\party{sr}' = w_\party{sr}.a$ and $w_\party{pq}' =
    w_\party{pq}$ for all $\party{pq \neq sr}$
\end{itemize}
or:
\begin{itemize}
  \item $\ell = \party{sr}?a$ and $(q_\party{r},\ell,q_\party{r}') \in
    \delta_\party{r}$
  \item $q_\party{p}' = q_\party{p}$ for all $\party{p \neq r}$
  \item $w_\party{sr} = a.w_\party{sr}'$ and $w_\party{pq}' =
    w_\party{pq}$ for all $\party{pq \neq sr}$
\end{itemize}
%TODO some explanation

a Sequence of Transitions is \emph{$k$-bounded} if no Channel of any
intermediate Configuration in the Sequence contains more than $k$
Messages

\emph{Reachable State} (???)

$Rc(\class{S}) = \{ s \ |\ s_0 \rightarrow^* s \}$ -- Set of
\emph{Reachable Configurations} of Communicating System $\class{S}$
where $\rightarrow^*$ is the Reflexive and Transitive Closure of the
Reachability Relation $\rightarrow$

$Rc_k(\class{S})$ -- the \emph{$k$-reachability Set} of Communicating
System $\class{S}$ is the largest Subset of $Rc(\class{S})$ within
which each Configuration $s$ can be reached by a $k$-bounded Sequence
from $s_0$

\emph{Deadlock} -- a Configuration $s$ is a \emph{Deadlock
  Configuration} when:
\begin{itemize}
  \item $\vec{w} = \vec{e}$
  \item there exists an $\party{r} \in \class{P}$ such that
    $(q_\party{r},\party{sr}?a,q_\party{r}') \in \delta_\party{r}$
  \item for all $\party{p} \in \class{P}$ and $q_\party{p}$ is a
    Receiving or Final State
\end{itemize}
i.e. all Buffers are empty, there is at least one Machine waiting for
a Message, and all the other Machines are either in a Final or
Receiving State

\emph{Orphan Messages} -- a Configuration $s$ is an \emph{Orphan
  Message Configuration} if all $q_\party{p} \in \vec{q}$ are Final
but $\vec{w} \neq \vec{\varepsilon}$, i.e. there is at least one
non-empty Buffer and all Machines are in a Final State

\emph{Unspecified Reception} -- a Configuration $s$ is an
\emph{Unspecified Reception Configuration} if there exists an
$\party{r} \in \class{P}$ such that $q_\party{r}$ is a Receiving State
and $(q_\party{r},\party{sr}?a,q_\party{r}') \in \delta_\party{r}$
Implies $0 < |w_\party{sr}|$ and $w_\party{sr} \notin aA^*$, i.e.
$q_\party{r}$ is prevented from Receiving any Messages from any of its
Buffers %TODO better explanation ?

Def. \emph{Safe Communicating System}: a Communicating System
$\class{S}$ is \emph{Safe} if each $s \in Rc(\class{S})$ is not a
Deadlock, Orphan Message, or Unspecified Reception Configuration

the following definitions are provided as a means of specifying a
Subset of Safe Communicating Systems from which Global Graphs can be
constructed, namely by identifying Sets of \emph{Concurrent Actions},
i.e. Actions that can be Interleaved

Communicating Systems amenable for being transformed into Global
Graphs will be identified through their \emph{Synchronous Transition
  System} (see below)


Equivalence Classes on CFSM Transitions

$act(q,q') \defeq \{ \ell \ |\ (q,\ell,q') \in \delta \}$ for $q,q'
\in Q$ -- Set of Labels for all Transitions between two given States

$\lozenge,\blacklozenge \subseteq \delta \times \delta$ -- smallest
Equivalence Relations on Transitions that respectively contain the
Relations $\underline{\lozenge}$:

\begin{itemize}
\item $(q_1,\ell,q_2)\underline{\lozenge}(q_1',\ell,q_2')$ if and only
  if $\ell \notin act(q_1,q_1') = act(q_2,q_2') \neq \varnothing$
\end{itemize}

and $\underline{\blacklozenge}$:

\begin{itemize}
\item $(q_1,\ell,q_2)\underline{\blacklozenge}(q_1',\ell,q_2')$ if and
  only if $(q_1,\ell,q_2)\underline{\lozenge}(q_1',\ell,q_2')$ and for
  all $(q,\ell,q') \in [(q_1,\ell,q_2)]^\lozenge$, $act(q_1,q) =
  act(q_2,q') \wedge act(q_1',q) = act(q_2',q')$ where
  $[(q,\ell,q')]^\lozenge$ is the Equivalence Class of $(q,\ell,q')$
  with respect to $\lozenge$
\end{itemize}
intuitively two Transitions are $\blacklozenge$-related if they refer
to the same Action up to Interleaving
%TODO clearer explanation

\emph{note on the above $\underline{\blacklozenge}$ definition}: what
is being stated can more easily be worked out on paper; essentially
two ``parallel'' Labelled Transitions $A$, $B$ sharing the same Label
$\ell$ are $\underline{\blacklozenge}$-related when for every
Transition $C$ in the $\lozenge$-equivalence Class of $A$ (note all
$C$ are also $\ell$-labelled by the definition of $\lozenge$), the Set
of Transition Labels from the Source and Target State of $A$ into the
Source and Target State of each $C$, resp., are \emph{equal} Sets, and
the same is true for the Set of Transitions Labels from the Source and
Target State of $B$ into the Source and Target State of each $C$,
resp., are also \emph{equal} Sets

TODO: if one pair of such Transitions are $\underline{\blacklozenge}$,
does that imply that all pairs in the $\lozenge$-equivalence Class are
\emph{also} $\underline{\blacklozenge}$ ???

%FIXME what does this relation allow for ???

\emph{Synchronous Transition Systems} -- used to identify
Communicating Systems which can be transformed into Global Graphs

a Synchronous Transition System has:
\begin{itemize}
\item a Node consists of a Vector of Local CFSM States
\item a Transitions (Edges) is Labelled by Elements in the \emph{Set
  of Events} (taken up to $\blacklozenge$-equivalence):
\[
  \class{E} \defeq \bigcup_{\party{s,r\in\class{P}}} Q_\party{s} \times
    Q_\party{r} \times \{(\party{s,r})\} \times \Sigma \}
\]
where Events are Tuples
$(q_\party{s},q_\party{r},\party{s},\party{r},a) \in \class{E}$, also
written:
\[
  (q_\party{s},q_\party{r},\party{s}\rightarrow\party{r}:a)
\]
indicating that Participants $\party{s}$ and $\party{r}$ can exchange
a Message $a$ when they are in States $q_\party{s}$ and $q_\party{r}$,
respectively
\end{itemize}

the indexing Events by Local (CFSM) States (i.e. $q_\party{s},
q_\party{r}$) distinguishes Communication of the same Message at
different points in the Global Graph

Def. \emph{$\class{E}$-equivalence} (\emph{Event Equivalence}):
\[
  \bowtie \defeq \bowtie_\party{s} \cap \bowtie_\party{r}
    \subseteq \class{E} \times \class{E}
\]
where:
\begin{align*}
  (q_1,q_2, \party{s}\rightarrow\party{r} : a) \bowtie_\party{s}&
    (q_1',q_2', \party{s}\rightarrow\party{r} : a) \Leftrightarrow
    \forall(q_1,\party{sr}!a,q_3),(q_1',\party{sr}!a,q_3')
      \in \delta_\party{s}
    . (q_1,\party{sr}!a,q_3)\blacklozenge(q_1',\party{sr}!a,q_3')
  \\
  (q_1,q_2, \party{s}\rightarrow\party{r} : a) \bowtie_\party{r}&
    (q_1',q_2', \party{s}\rightarrow\party{r} : a) \Leftrightarrow
    \forall(q_2,\party{sr}?a,q_4),(q_2',\party{sr}?a,q_4')
      \in \delta_\party{s}
    . (q_2,\party{sr}?a,q_4)\blacklozenge(q_2',\party{sr}?a,q_4')
\end{align*}
is an Equivalence Relation over Events that identifies Events with
underlying Local Transitions that are $\blacklozenge$-equivalent

$[e]$ -- Equivalence Class of Event $e$

Synchronous Transition System is the Labelled Transition System
generated by the possible interleavings of Synchronous Transitions
(Send/Receive pairs) of the Local CFSMs

Squares in the Synchronous Transition System with identical
Sender/Receiver/Message Events (but not the same Local States) on
opposite parallel sides (both ``horizontal'' and ``vertical'')
identify pairs of \emph{Concurrent Interactions}

$\vec{n}, \vec{n}', \ldots$ -- Vectors of Local States

$\vec{n}[\party{p}]$ -- the State of $\party{p} \in \class{P}$ in
$\vec{n}$

for Communicating System $\class{S} =
(M_\party{p})_{\party{p}\in\class{P}}$, $STS(\class{S})$ is the
\emph{Synchronous Transition System} of $\class{S}$:
\[
  STS(\class{S}) = (N,\vec{n}_0,E/\bowtie,\rightrightarrows)
\]
where:

$N \defeq \{ \vec{q} \ |\ (\vec{q};\vec{\varepsilon}) \in Rc_1(\class{S}) \}$

$\vec{n}_0 = \vec{q}_0$ -- Initial State

$\hat{\delta} \defeq \{ (\vec{n},e,\vec{n}')
\ |\ (\vec{n};\vec{\varepsilon})
    \xrightarrow{\party{sr}!a} \xrightarrow{\party{sr}?a}
     (\vec{n}';\vec{\varepsilon}) \wedge
    e = (\vec{n}[\party{s}],\vec{n}[\party{r}],
      \party{s}\rightarrow\party{r}:a)
\}$

$E \defeq \{ e \ |\ \exists\vec{n},\vec{n}' \in N :
  (\vec{n},e,\vec{n}') \in \hat{\delta} \} \subseteq \class{E}$

$\vec{n} \stackrel{[e]}{\rightrightarrows} \vec{n}' \Leftrightarrow
  (\vec{n},e,\vec{n}') \in \hat{\delta}$

in $\hat{\delta}$, Events are $\bowtie$-equivalent if they have the
same Interaction $\party{s}\rightarrow\party{r} : a$
%TODO explain

note: original paper uses a representative Set $\hat{E}$ of Events for
each $\bowtie$-equivalence Class but here we will just continue
writing $[e]$

$\pi$ -- ranges over Sequences of Events and $\rightrightarrows$ is
extended to Sequences of Events (as with the Reachibility Relation
$\rightarrow$ for Configurations above)

$e\downharpoonright_\party{p}$ -- Projection of Event $e$ onto
Participant $\party{p}$:
\[
  (q_\party{s},q_\party{r},\party{s}\rightarrow\party{r}:a)
    \downharpoonright_\party{p}
  \defeq \begin{cases}
    \party{pr}!a &\ \text{if}\ \party{s = p} \\
    \party{sp}?a &\ \text{if}\ \party{r = p} \\
    \varepsilon  &\ \text{otherwise} \\
  \end{cases}
\]

$STS(\class{S})\downharpoonright_\party{p}$ -- Projection of
Synchronous Transition System onto Participant $\party{p}$ is the
Automaton $(Q, \vec{q}_0, \Sigma, \delta)$ where $Q = N$, $\vec{q}_0 =
\vec{n}_0$ and $\delta \subseteq Q \times A \cup \{\varepsilon\}
\times Q$ is such that $(\vec{n}_1, e\downharpoonright_\party{p},
\vec{n}_2) \in \delta \Leftrightarrow \vec{n}_1
\stackrel{e}{\rightrightarrows} \vec{n}_2$


\textbf{Generalized Multiparty Compatibility} (GMC)

Sound and Complete condition for constructing Global Graphs

\emph{Representability} -- for each CFSM, each Trace and each Choice
are ``\emph{represented}'' in $STS(\class{S})$; guarantees that
$STS(\class{S})$ contains enough information to decide \emph{Safety
  Properties} of any Asynchronous Execution of $\class{S}$

\emph{Branching Property} -- for each Choice in $STS(\class{S})$ a
unique CFSM makes the decision and each other Participant is either
notified of the Choice or else is not involved in the Choice; ensures
that if a ``\emph{branching}'' in $STS(\class{S})$ represents a
\emph{Choice}, then the Choice is ``Well-formed''


$\class{L}$ -- Language

$hd(\class{L})$ -- \emph{First Actions} of $\class{L}$:
\begin{align*}
  hd(\class{L}) \defeq& \{ \ell \ |\ \exists\varphi \in A^*
    : \ell\cdot\varphi \in \class{L} \} & \\
  hd(\{\varepsilon\}) \defeq& \{ \varepsilon \} &
\end{align*}

$STS(\class{S})<\vec{n}>$ -- Synchronous Transition System
$STS(\class{S})$ where Initial State $\vec{n}_0$ is replaced by
$\vec{n}$

$LTS(\class{S},\vec{n},\party{p})$ -- the Language
$\class{L}(STS(\class{S})<\vec{n}>\downharpoonright_p)$ obtained by
setting the Initial State (Node) of $STS(\class{S})$ to $\vec{n}$ and
then Projecting the resulting Synchronous Transition System onto
Participant $\party{p}$


Def. (Representability) Communicating System $\class{S}$ is
\emph{Representable} if for all $\party{p} \in \class{P}$:
\begin{enumerate}
  \item $\class{L}(M_\party{p}) = LTS(\class{S},\vec{n}_0,\party{p})$ and
  \item $\forall q \in Q_\party{p} \exists \vec{n} \in N :
    \vec{n}[\party{p}] = q \wedge
      \bigcup_{(q,\ell,q')\in\delta_\party{p}} \{ \ell \}
        \subseteq hd(LTS(\class{S},\vec{n},\party{p}))$
\end{enumerate}
(1) guarantees that each Trace of each Machine is represented in
$STS(\class{S})$ and (2) is necessary to ensure that every Choice in
each Machine is represented in $STS(\class{S})$

checking the Representability Condition for a Communicating System
$\class{S}$ has Exponential worst-case Time Complexity


\endgroup



% ====================================================================
\section{Phase Space}\label{sec:phase_space}
% ====================================================================

% --------------------------------------------------------------------
\subsection{Attractor \& Repeller}\label{sec:attractor_repeller}
% --------------------------------------------------------------------



% ====================================================================
\section{Self-organized Criticality}\label{sec:self_organized_criticality}
% ====================================================================

Property of Dynamical Systems with a Critical Point as an Attractor

%FIXME



% ====================================================================
\section{Reaction-diffusion System}\label{sec:reaction_diffusion}
% ====================================================================

% ====================================================================
\section{Hamiltonian System}\label{sec:hamiltonian_system}
% ====================================================================

% ====================================================================
\section{Ergodic Theory}\label{sec:ergodic_theory}
% ====================================================================

% --------------------------------------------------------------------
\subsection{Normal Number}\label{sec:normal_number}
% --------------------------------------------------------------------



% ====================================================================
\section{Chaos Theory}\label{sec:chaos_theory}
% ====================================================================

% ====================================================================
\section{Control Theory}\label{sec:control_theory}
% ====================================================================

Properties of possible Trajectories of Dynamical Systems
(\S\ref{sec:dynamical_system}) with certain Initial Conditions and
Driving Functions:

\emph{Observability}

\emph{Controllability} -- Property of Vector-valued Coupled
Differential Equations; \fist Cf. Information Theoretic \emph{Channel
  Capacity Theorem} (\S\ref{sec:channel_capacity})
%FIXME xref



% --------------------------------------------------------------------
\subsection{Signal Flow}\label{sec:signal_flow}
% --------------------------------------------------------------------

(\emph{SFG} or \emph{Mason Graph})

Nodes represent ``System Variables'' and Edges represent Functional
Connections between pairs of Nodes

%FIXME: move to information theory or control theory?

Semantics given by Corelation Categories
(\S\ref{sec:corelation_category})

Signal Flow Graph (\S\ref{sec:flow_graph})



% --------------------------------------------------------------------
\subsection{Controller}\label{sec:controller}
% --------------------------------------------------------------------

%FIXME

monitors and physically alters the operating conditions of a given
Dynamical System (\S\ref{sec:dynamical_system})
