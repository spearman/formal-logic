%%%%%%%%%%%%%%%%%%%%%%%%%%%%%%%%%%%%%%%%%%%%%%%%%%%%%%%%%%%%%%%%%%%%%%
%%%%%%%%%%%%%%%%%%%%%%%%%%%%%%%%%%%%%%%%%%%%%%%%%%%%%%%%%%%%%%%%%%%%%%
\part{Systems Theory}\label{sec:systems_theory}
%%%%%%%%%%%%%%%%%%%%%%%%%%%%%%%%%%%%%%%%%%%%%%%%%%%%%%%%%%%%%%%%%%%%%%
%%%%%%%%%%%%%%%%%%%%%%%%%%%%%%%%%%%%%%%%%%%%%%%%%%%%%%%%%%%%%%%%%%%%%%

%FIXME new tex file?

Operad Theory (\S\ref{sec:operad_theory}): Systems of Systems

Operad (\S\ref{sec:operad}): compositional style

Algebra (??? \S\ref{sec:universal_algebra}): System type



% ====================================================================
\section{State Variable}\label{sec:state_variable}
% ====================================================================

% ====================================================================
\section{Steady State}\label{sec:steady_state}
% ====================================================================

``Fixed-point'', ``Equilibria'' %FIXME



% ====================================================================
\section{Closed System}\label{sec:closed_system}
% ====================================================================

%FIXME

Actor Model (\S\ref{sec:actor_model}) -- \emph{Computational
  Representation Theorem} -- Consequence: a Finite Actor can
Nondeterministically Respond with an Uncountable number of different
Outputs %FIXME



% ====================================================================
\section{Open System}\label{sec:open_system}
% ====================================================================

(2016 - Fong - The Algebra of Open and Interconnected Systems):

\fist Hypergraph (\S\ref{sec:hypergraph}), Hypergraph Category
(\S\ref{sec:hypergraph_category})

Interconnection (``Integration'') of Systems modelled by Cospans
(\S\ref{sec:cospan})

Principle of Compositionality

``Network-style Diagrammatic Languages''

Network Diagram (\S\ref{sec:network_diagram}), ``Network-style
Diagrammatic Languages'' -- Electrical Circuits

other examples: Signal Flow Graphs (\S\ref{sec:signal_flow}),
Markov Processes (\S\ref{sec:markov_process}), Automata
(\S\ref{sec:automaton}), Petri Nets (\S\ref{sec:petri_net}), Chemical
Reaction Networks
%FIXME

\emph{Terminal} -- ``point of Interconnection''; ``boundary''
``marked'' using Cospan (\S\ref{sec:decorated_cospan}), Connected to
others using Pushouts (\S\ref{sec:pushout}); having a ``marked
boundary'' in this way makes a Closed System into an Open System;
process of ``freely''  ``marking boundaries'' constructs a Hypergraph
Category (\S\ref{sec:hypergraph_category}) from a ``library'' of
Network pieces

\emph{Components} with multiple Input/Output Terminals (possibly
labelled with some Type) connected to form a larger \emph{Network}

Components form \emph{Hyperedges} between labelled Vertices

\begin{itemize}
  \item each Terminal of an Open System may make ``Measurements''
    appropriate to the ``Type'' of the Terminal
  \item given a collection of Terminals, the \emph{Universum} is the
    Set of all possible Measurement outcomes
  \item each Open System has a collection of Terminals (and a Universum)
  \item the Semantics of an Open System is the Subset of Measurement
    outcomes on the Terminals that are ``permitted'' by the System,
    known as the \emph{Behavior} of the System
\end{itemize}

%FIXME universum = phase space?

``Laws'' (e.g. Ohm's Law) are mechanisms for Partitioning Behaviors
into \emph{Possible} and \emph{Impossible} Behaviors

given a Universum $\class{U}$, a Behavior of a System is an Element
of the Power Set $\pow(\class{U})$ (representing all possible
Measurements of the System), and a Law is an Element of
$\pow(\pow(\class{U}))$ representing all possible Behaviors of a
\emph{Class} of Systems

\emph{Interconnection} of Terminals asserts the Identification of
Variables at the Identified Terminals

Algebra of Semantic Objects and Homomorphism from Syntax to Semantics
(Principle of Compositionality \S\ref{sec:compositionality})



% ====================================================================
\section{Dynamical System}\label{sec:dynamical_system}
% ====================================================================

A \emph{Dynamical System} is defined as a tuple $(T,M,\Phi)$ where $T$
is a Monoid (\S\ref{sec:monoid}), M is a Set and $\Phi$ is a Function
(\S\ref{sec:set_function}).

Discrete, Continuous, Hybrid

Control Theory (\S\ref{sec:control_theory})

Initial Conditions, Driving Functions

%FIXME relate to transition systems, automata ?

\url{https://www.youtube.com/watch?v=cu718EbCOPs} Spivak 16:

can't have Identities and Feedback without Partiality %FIXME

Traced Ideals %FIXME



% ====================================================================
\section{Communicating System}\label{sec:communicating_system}
% ====================================================================

Denielou-Yoshida13 -- arises from a Choreographed collection of
Communicating Finite State Machines (\S\ref{sec:communicating_fsm}) in
the context of Multiparty Session Types
(\S\ref{sec:multiparty_session})

Linear Multirole Logic (LMRL \S\ref{sec:lmrl})

\fist Process Calculus (\S\ref{sec:process_calculus})



% --------------------------------------------------------------------
\subsection{Choreography}\label{sec:choreography}
% --------------------------------------------------------------------

Global Types (\S\ref{sec:global_type})



\subsubsection{Graphical Choreography}\label{sec:graphical_choreography}

or \emph{Global Graphs}

Lange-Tuosto-Yoshida15:

CFSMs as Behavioral Specifications of Distributed Components from
which a Choreography can be built

CFSM model based on Asynchronous FIFO Message-passing Communication

Algorithm produces a Choreography expressed as a Global Graph
(closely related to BPMN 2.0 Choreography) given a Set of CFSMs



% ====================================================================
\section{Phase Space}\label{sec:phase_space}
% ====================================================================

% --------------------------------------------------------------------
\subsection{Attractor \& Repeller}\label{sec:attractor_repeller}
% --------------------------------------------------------------------



% ====================================================================
\section{Self-organized Criticality}\label{sec:self_organized_criticality}
% ====================================================================

Property of Dynamical Systems with a Critical Point as an Attractor

%FIXME



% ====================================================================
\section{Reaction-diffusion System}\label{sec:reaction_diffusion}
% ====================================================================

% ====================================================================
\section{Hamiltonian System}\label{sec:hamiltonian_system}
% ====================================================================

% ====================================================================
\section{Ergodic Theory}\label{sec:ergodic_theory}
% ====================================================================

% --------------------------------------------------------------------
\subsection{Normal Number}\label{sec:normal_number}
% --------------------------------------------------------------------



% ====================================================================
\section{Chaos Theory}\label{sec:chaos_theory}
% ====================================================================

% ====================================================================
\section{Control Theory}\label{sec:control_theory}
% ====================================================================

Properties of possible Trajectories of Dynamical Systems
(\S\ref{sec:dynamical_system}) with certain Initial Conditions and
Driving Functions:

\emph{Observability}

\emph{Controllability} -- Property of Vector-valued Coupled
Differential Equations; \fist Cf. Information Theoretic \emph{Channel
  Capacity Theorem} (\S\ref{sec:channel_capacity})
%FIXME xref



% --------------------------------------------------------------------
\subsection{Signal Flow}\label{sec:signal_flow}
% --------------------------------------------------------------------

(\emph{SFG} or \emph{Mason Graph})

Nodes represent ``System Variables'' and Edges represent Functional
Connections between pairs of Nodes

%FIXME: move to information theory or control theory?

Semantics given by Corelation Categories
(\S\ref{sec:corelation_category})

Signal Flow Graph (\S\ref{sec:flow_graph})



% --------------------------------------------------------------------
\subsection{Controller}\label{sec:controller}
% --------------------------------------------------------------------

%FIXME

monitors and physically alters the operating conditions of a given
Dynamical System (\S\ref{sec:dynamical_system})
