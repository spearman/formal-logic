%%%%%%%%%%%%%%%%%%%%%%%%%%%%%%%%%%%%%%%%%%%%%%%%%%%%%%%%%%%%%%%%%%%%%%
%%%%%%%%%%%%%%%%%%%%%%%%%%%%%%%%%%%%%%%%%%%%%%%%%%%%%%%%%%%%%%%%%%%%%%
\part{Systems Theory}\label{sec:systems_theory}
%%%%%%%%%%%%%%%%%%%%%%%%%%%%%%%%%%%%%%%%%%%%%%%%%%%%%%%%%%%%%%%%%%%%%%
%%%%%%%%%%%%%%%%%%%%%%%%%%%%%%%%%%%%%%%%%%%%%%%%%%%%%%%%%%%%%%%%%%%%%%

%FIXME new tex file?

Operad Theory (\S\ref{sec:operad_theory}): Systems of Systems

Operad (\S\ref{sec:operad}): compositional style

Algebra (??? \S\ref{sec:universal_algebra}): System type

Stochastic Processes (Probability Theory \S\ref{sec:stochastic_process})

Iterated Function Systems (Fractal Geometry \S\ref{sec:ifs}), Multifractal
Systems (\S\ref{sec:multifractal_system})



% ====================================================================
\section{State Variable}\label{sec:state_variable}
% ====================================================================

% ====================================================================
\section{Phase Space}\label{sec:phase_space}
% ====================================================================

cf. Product Space (\S\ref{sec:product_space})

Witkin-Baraff97 - Ch.2: Particle System Dynamics -- example of phase space of a
particle system in 3D

(FIXME: the above reference seems to use ``phase space'' only to refer to a
state space that includes velocity components in addition to position
components)

2014 - MAE5790 - Non-linear Dynamics and Chaos - Strogatz - "Theory of
Dynamical Systems" -
\url{https://www.youtube.com/playlist?list=PLj_l4pOO0YKhJHLQVbjpPlbNyYtT8aAQz}

``Phase Portrait'': picture of all qualitatively different Trajectories
(\S\ref{sec:trajectory})

Lec.16: a Lorenz Systems (\S\ref{sec:lorenz_system}) may be
``\emph{Dissipative}'' in the sense that Volumes in Phase Space
``\emph{contract}'' under the ``flow''

$\dot{x} = f(x), x \in \reals$

example First-order Non-linear System:
\[
  \dot{x} = sin(x)
\]

Flow

$\dot{x} = 0$ at \emph{Fixed Points} $x^*$

cf. Moduli Space (\S\ref{sec:moduli_space})


\asterism


Quantum Mechanics: Schr\"odinger Formulation

Phase Space Formulation -- Position and Momentum Phase Space; Quasiprobability
Distributions (\S\ref{sec:quasiprobability_distribution})



% --------------------------------------------------------------------
\subsection{Generalized Coordinate}\label{sec:generalized_coordinate}
% --------------------------------------------------------------------

Generalized Coordinates in Lagrangian Mechanics (\S\ref{sec:lagrangian_system})
are related to Canonical Coordinates (\S\ref{sec:canonical_coordinate}) of
Hamiltonian Mechanics (\S\ref{sec:hamiltonian_system}) by the Hamilton-Jacobi
Equations (\S\ref{sec:hamilton_jacobi})



% --------------------------------------------------------------------
\subsection{Canonical Coordinate}\label{sec:canonical_coordinate}
% --------------------------------------------------------------------

\emph{Canonical Transformation}: a change of Canonical Coordinates that
preserves Hamilton's Equations (\S\ref{sec:hamiltonian_system}), e.g. Geometric
Integrators (\S\ref{sec:geometric_integrator})

generalized to definition of Coordinates on the Phase Space as a Cotangent
Bundle (\S\ref{sec:cotangent_bundle}) of a Manifold

Generalized Coordinates (\S\ref{sec:generalized_coordinate}) in Lagrangian
Mechanics (\S\ref{sec:lagrangian_system}) are related to Canonical Coordinates
of Hamiltonian Mechanics (\S\ref{sec:hamiltonian_system}) by the
Hamilton-Jacobi Equations (\S\ref{sec:hamilton_jacobi})



% --------------------------------------------------------------------
\subsection{Bond Graph}\label{sec:bond_graph}
% --------------------------------------------------------------------

(Bond1961)

allows conversion of a System into a State-space Representation



% --------------------------------------------------------------------
\subsection{Steady State}\label{sec:steady_state}
% --------------------------------------------------------------------

``Fixed-point'', ``Equilibria'' %FIXME

Stable Fixed Points $\bullet$

Unstable Fixed Points $\circ$

Half-stable Fixed Points

Exponential Relaxation, Algebraic Relaxation



% --------------------------------------------------------------------
\subsection{Bifurcation}\label{sec:bifurcation}
% --------------------------------------------------------------------

changing a Parameter alters the structure of the Vector Field such that Fixed
Points may be created or destroyed or their Stability may change

\fist Period-doubling Bifurcation (Discrete Dynamical Systems
\S\ref{sec:period_doubling})

\emph{Bifurcation Point}: value of the Parameter at which the Bifurcation
occurs

\emph{Saddle-node Bifurcation}: Normal Form $\dot{x} = r + x^2$ where $r$ is
the Control Parameter;
Fixed-points are created in \emph{pairs} of Stable and Unstable Fixed Points


Critical Value (Bifurcation Value)

\emph{Transcritical Bifurcation}: Normal Form $\dot{x} = rx - x^2$; Fixed Point
that cannot be destroyed but can change Stability

\emph{Pitchfork Bifurcation}: Normal Form $\dot{x} = rx - x^3$; a Stable Fixed
Point splits into a pair of Stable Fixed Points and a central Unstable Fixed
Point

\emph{Coalescence of Limit Cycles}

\emph{Saddle Node Infinite Period Bifurcation} (SNIPER) -- ; e.g. EKG (heart)
dynamics

Bifurcations of:
\begin{itemize}
  \item Fixed Points -- Saddle-node, Transcritical, Pitchfork; Hopf Bifurcations
  \item Closed Orbits -- Coalescence of Cycles,
    Saddle-node Infinite Period Bifurcation (``Sniper'' or ``Snic''),
    Homoclinic Bifurcation
\end{itemize}

(MAE5790 Lec. 14)

Bifurcations of Cycles:
\begin{itemize}
  \item Supercritical Hopf
  \item Saddle Node
  \item ``Sniper''
  \item Homoclinic
\end{itemize}



% --------------------------------------------------------------------
\subsection{Attractor \& Repeller}\label{sec:attractor_repeller}
% --------------------------------------------------------------------

\fist Self-Organized Criticality (SOC \S\ref{sec:soc}): Property of Dynamical
Systems (\S\ref{sec:dynamical_system}) that have a Critical Point (TODO) as an
Attractor

MAE5790 Lec.18:

an \emph{Attractor} is a Set $A$ such that:
\begin{enumerate}
  \item $A$ is \emph{Invariant}: Trajectories beginning in $A$ remain in $A$
  \item $A$ ``attracts'' an Open Set (``Basin'') of Initial Conditions
  \item $A$ is \emph{Minimal}: no Proper Subset of $A$ satisfies both (1.) and
    (2.)
\end{enumerate}

a Set satisfying (1.) and (2.) but not (3.) is an \emph{Attracting Set}

an additional requirement is that Trajectories ``near'' $A$, they stay ``near''
$A$ for all time $t > 0$; this would exclude e.g. a Half-stable point on the
$1$-manifold of a Circle



\subsubsection{Strange Attractor}\label{sec:strange_attractor}

MAE5790 Lec.18:

\emph{Strange Attractor}

\emph{Dynamic} aspect: a \emph{Chaotic Attractor} is an Attractor that exhibits
sensitive dependence to Initial Conditions

\emph{Geometric} aspect: a \emph{Fractal Attractor} is an Attractor whose
``local structure'' is Fractal (\S\ref{sec:fractal})

Lec.23:

R\"ossler Attractor (\S\ref{sec:rossler_system}) -- simplest Strange Attractor

\begin{align*}
  \dot{x} &= -y-z \\
  \dot{y} &= x + ay \\
  \dot{z} &= b + z(x - c) \\
\end{align*}



% --------------------------------------------------------------------
\subsection{Time Evolution}\label{sec:time_evolution}
% --------------------------------------------------------------------

\fist Initial Value Problem (IVP \S\ref{sec:ivp})

Hamilton's Equations (\S\ref{sec:hamiltonian_system})



\subsubsection{Propagator}\label{sec:propagator}

a State Space with a distinguished Propagator (Evolution Function
\S\ref{sec:evolution_function}) is a \emph{Dynamical System}
(\S\ref{sec:dynamical_system})



% --------------------------------------------------------------------
\subsection{Phase Plane}\label{sec:phase_plane}
% --------------------------------------------------------------------

%FIXME: clarify Phase Plane vs 2D Dynamical System

Autonomous Systems (\S\ref{sec:autonomous_system}) can be subjected to Phase
Plane Analysis

\url{http://math.rice.edu/~dfield/dfpp.html} -- Phase Plane applet

\emph{Two-dimensional Systemss}

Phase Plane -- Phase Space (\S\ref{sec:phase_space}) of two State Variables
$x,y \in \reals$

Position Vector $(x,y) \in \reals^2$

Velocity Vector $(\dot{x},\dot{y}) \in \reals^2$

Vector Field (\S\ref{sec:vector_field}) on Phase Plane:
\begin{align*}
  \dot{x} & = f(x,y) \\
  \dot{y} & = g(x,y)
\end{align*}
Trajectories (\S\ref{sec:trajectory}) of Points in the Phase Plane are
Solutions (FIXME: clarify)

Point in Phase Plane: $\vec{p} = (x,y)$

Vector Field: $\dot{\vec{p}} = \vec{f}(\vec{p})$

\emph{Fixed Points} $\vec{p}^*$ have $\dot{x} = 0$ and $\dot{y} = 0$:
\[
  \vec{f}(\vec{p}^*) = \vec{0}
\]

if the Vector Field $\vec{f}$ is Continuously Differentiable ($\vec{f}$ and its
Derivative are Continuous), then Solutions (Trajectories) $\vec{p}(t)$ exist
and are \emph{Unique} for any Initial Condition

\emph{Phase Portrait} -- representation of the ``qualitatively different''
Trajectories for a Vector Field

Uniqueness of Solutions implies that two Trajectories \emph{cannot cross}, but
they can approach the same Fixed Point

a \emph{Closed} Trajectory represents a Periodic Solution (a repeating
``motion'' or ``cycle'') of the Differential Equation of the System and acts
like a ``barrier'' since Trajectories cannot enter or exit the enclosed region

there is no \emph{Chaotic} motion possible in Two-dimensions
(\S\ref{sec:chaos_theory})

given $\dot{\vec{x}} = \vec{f}(\vec{p})$, deduce Phase Portrait: Fixed Points,
Closed Orbits

for Linear Two-dimensional Systems (\S\ref{sec:linear_dynamical_system}) of the
form $\dot{\vec{p}} = A\vec{p}$, the Origin is always a Fixed Point with Phase
Portrait around the Origin determined by the Eigenvectors
(\S\ref{sec:eigenvector}) and Eigenvalues (\S\ref{sec:eigenvalue}) of $A$


MAE 5790 Lec. 8 - \url{https://www.youtube.com/watch?v=O2fcpxLT5wk}:

``Index Theory'' \fist cf. Winding Number (\S\ref{sec:winding_number})

``Index'' of a Simple Closed Curve (\S\ref{sec:simple_closed_curve}) in a
Continuous Vector Field (\S\ref{sec:vector_field})

Index is not defined for Curves that intersect a Fixed Point (Vector Angle is
undefined)

by the Continuity of the Vector Field, an Index is an Integer

if the Curve does not enclose a Fixed Point then the Index of the Curve is Zero

any Closed Trajectory has a Index of Positive $1$

Index is Additive when subdividing the region enclosed by the Curve

a Continuous Deformation of a Curve without crossing any Fixed Points leaves
the Index unchanged

reversing Time direction does not change Index

the Index of a Point is equal to the Index of any Curve that encloses the Point
and no other Fixed Points

\begin{itemize}
  \item Non-fixed Point -- Index $0$
  \item Nodes -- Index $1$
  \item Saddle Points -- Index $-1$
  \item Spirals -- Index $1$
  \item Centers -- Index $1$
\end{itemize}

Thm. \emph{Any Closed Trajectory on $\reals^2$ must enclose Fixed Points with
  Indices obeying $\sum_{k=1}^n I_k = 1$}

\emph{Hairy Ball Theorem}: for Continuously Differentiable Vector Field
on a Sphere, the Sum of the Indices of the Fixed Points on the Sphere will
always be $2$

\emph{Poincare-Hopf Index Theorem}: in general, the Sum of the Indices of the
Fixed Points for a Continuously Differentiable Vector Field on any Surface is
$2 - 2g$ where $g$ is the Genus (\S\ref{sec:genus}) of the Surface

generalization of ``Index Theory'' to higher Dimensions: ``Degree Theory''
(\S\ref{sec:degree_theory})


\asterism

\emph{Limit Cycles} -- \emph{Non-linear Systems} only

Stable, Unstable, Half-stable


MAE 5790 Lec. 9 - \url{https://www.youtube.com/watch?v=nWO74rlr58Y}:

Closed Orbits -- Index Theory, Dulac's Critereon, Poincare-Bendixson Theorem

MAE 5790 Lec. 12 - \url{https://www.youtube.com/watch?v=oqKAVqe71vw}

Bifurcations (\S\ref{sec:bifurcation}) of:
\begin{itemize}
  \item Fixed Points -- Saddle-node, Transcritical, Pitchfork; Hopf Bifurcations
  \item Closed Orbits -- Coalescence of Cycles,
    Saddle-node Infinite Period Bifurcation (``Sniper'' or ``Snic''),
    Homoclinic Bifurcation
\end{itemize}

MAE5790 Lec. 14 - \url{https://www.youtube.com/watch?v=CvDJnqScUVI}

Phase Plane of a Torus parameterized by angles $\theta_1$, $\theta_2$

Orbit is Periodic when $\dot{\theta_1}$ and $\dot{\theta_2}$ are related by a
Rational Number $p/q$; whenever $p$ and $q$ are relatively prime, the Orbit is a
Knot

when the Ratio is Irrational, the Trajectories are ``Quasiperiodic'', i.e. they
are Dense but never Closed



% ====================================================================
\section{Closed System}\label{sec:closed_system}
% ====================================================================

%FIXME

Actor Model (\S\ref{sec:actor_model}) -- consequence of \emph{Computational
  Representation Theorem}: a Finite Actor can Nondeterministically Respond with
an Uncountable number of different Outputs %FIXME



% ====================================================================
\section{Open System}\label{sec:open_system}
% ====================================================================

(2016 - Fong - The Algebra of Open and Interconnected Systems):

2017 - Hedges - \emph{A First Look at Open Games} -
\url{https://julesh.com/2017/09/29/a-first-look-at-open-games/} \fist
Compositional Game Theory (\S\ref{sec:compositional_game_theory}), Open Games
(\S\ref{sec:open_game})

2007 - Willems - \emph{The Behavioral Approach to Open and Interconnected
  Systems}

\fist Behavioral Control Theory (\S\ref{sec:behavioral_control})

\fist Hypergraph (\S\ref{sec:hypergraph}), Hypergraph Category
(\S\ref{sec:hypergraph_category})

Interconnection (``Integration'') of Systems modelled by Cospans
(\S\ref{sec:cospan})

\fist Network Theory (\S\ref{sec:network_theory})

\url{https://golem.ph.utexas.edu/category/2018/04/props_in_network_theory.html}:
Systems with Inputs and Outputs can be seen as Morphisms in a Category where
Composition uses the Outputs of one System as the Inputs of another; a
'\emph{Black Box}' Functor is a Functor \emph{out} of such a Category

\url{https://johncarlosbaez.wordpress.com/2017/06/20/the-theory-of-devices/}

Principle of Compositionality

``Network-style Diagrammatic Languages''

Network Diagram (\S\ref{sec:network_diagram}), ``Network-style
Diagrammatic Languages'' -- Electrical Circuits

other examples: Signal Flow Graphs (\S\ref{sec:signal_flow}),
Markov Processes (\S\ref{sec:markov_process}), Automata
(\S\ref{sec:automaton}), Petri Nets (\S\ref{sec:petri_net}), Chemical
Reaction Networks
%FIXME

\emph{Terminal} -- ``point of Interconnection''; ``boundary''
``marked'' using Cospan (\S\ref{sec:decorated_cospan}), Connected to
others using Pushouts (\S\ref{sec:pushout}); having a ``marked
boundary'' in this way makes a Closed System into an Open System;
process of ``freely''  ``marking boundaries'' constructs a Hypergraph
Category (\S\ref{sec:hypergraph_category}) from a ``library'' of
Network pieces

\emph{Components} with multiple Input/Output Terminals (possibly
labelled with some Type) connected to form a larger \emph{Network}

Components form \emph{Hyperedges} between labelled Vertices

\begin{itemize}
  \item each Terminal of an Open System may make ``Measurements''
    appropriate to the ``Type'' of the Terminal
  \item given a collection of Terminals, the \emph{Universum} is the
    Set of all possible Measurement outcomes
  \item each Open System has a collection of Terminals (and a Universum)
  \item the Semantics of an Open System is the Subset of Measurement
    outcomes on the Terminals that are ``permitted'' by the System,
    known as the \emph{Behavior} of the System
\end{itemize}

%FIXME universum = phase space?

``Laws'' (e.g. Ohm's Law) are mechanisms for Partitioning Behaviors
into \emph{Possible} and \emph{Impossible} Behaviors

given a Universum $\class{U}$, a Behavior of a System is an Element
of the Power Set $\pow(\class{U})$ (representing all possible
Measurements of the System), and a Law is an Element of
$\pow(\pow(\class{U}))$ representing all possible Behaviors of a
\emph{Class} of Systems

\emph{Interconnection} of Terminals asserts the Identification of
Variables at the Identified Terminals

Algebra of Semantic Objects and Homomorphism from Syntax to Semantics
(Principle of Compositionality \S\ref{sec:compositionality})



% --------------------------------------------------------------------
\subsection{Dissipative System}\label{sec:dissipative_system}
% --------------------------------------------------------------------

Non-Hermitian Systems, Topological Phases

2018 - Gong, Ashida, Kawabata, Takasan, Higashikawa, Ueda -
\emph{Topological Phases of Non-Hermitian Systems} -
\url{https://physics.aps.org/articles/v11/96}
-- \emph{Non-Hermitian Quantum Mechanics}; usually Operators are assumed to be
Hermitian (only reaturns Real values, i.e. Observables); a Non-hermitian
Operator can allow Complex values useful in describing Open (Dissipative)
Systems

\emph{Parity-time Symmetry}; corresponds to the presence of balanced Gain/Loss
in Optical Systems



% ====================================================================
\section{Coupling}\label{sec:coupling}
% ====================================================================

\emph{Coupling Through Emergent Conservation Laws} (blog series):
\url{https://johncarlosbaez.wordpress.com/2018/06/27/coupling-through-emergent-conservation-laws-part-1/}

2018 - Baez,Pollard,Lorand,Sarazola -
\emph{Biochemical Coupling Through Emergent Conservation Laws}



% ====================================================================
\section{Deterministic System}\label{sec:deterministic_system}
% ====================================================================

\fist Deterministic Dynamical Systems
(\S\ref{sec:deterministic_dynamical_system})

\fist Pseudorandom Processes (\S\ref{sec:pseudorandom_process}) -- a
Deterministic System exhibiting \emph{Statistical Randomness}
(\S\ref{sec:statistical_randomness})



% ====================================================================
\section{Differential System}\label{sec:differential_system}
% ====================================================================

%FIXME: move section ???



% --------------------------------------------------------------------
\subsection{Integrable System}\label{sec:integrable_system}
% --------------------------------------------------------------------

% --------------------------------------------------------------------
\subsection{Non-integrable System}\label{sec:nonintegrable_system}
% --------------------------------------------------------------------

when a System is ``non-conservative'', Integrals over Paths depend on the Path
and not just the endpoints

cf. Holonomy (\S\ref{sec:holonomy})



% ====================================================================
\section{Dynamical System}\label{sec:dynamical_system}
% ====================================================================

A \emph{Dynamical System} is a State (Phase) Space (\S\ref{sec:phase_space})
with a distinguished \emph{Propagator} (\S\ref{sec:propagator}) called the
\emph{Evolution Function} (\S\ref{sec:evolution_function}).

\fist cf. Initial Value Problem (IVP \S\ref{sec:ivp})

\fist cf. Recurrence Relation (\S\ref{sec:recurrence_relation})

A Dynamical System can be defined as a tuple $(T,M,\Phi)$ where \emph{Time
  Domain} $T$ is a Monoid (\S\ref{sec:monoid}), \emph{State Space} $M$ is a
Non-empty Set and $\Phi$ is an \emph{Evolution Function}:
\[
  \Phi : U \subseteq T \times M \rightarrow M
\]

In a Geometrical setting $M$ is a Manifold (locally a Banach Space or Euclidean
Space) and the Evolution Rule $\Phi(t)$ gives a \emph{Diffeomorphism} of the
Manifold to itself for every $t \in T$. %FIXME clarify

Discrete, Continuous, Hybrid \fist Time-scale Calculus
(\S\ref{sec:timescale_calculus}): unification of Integral and Differential
Calculus with the Calculus of Finite Differences

Control Theory (\S\ref{sec:control_theory})

Initial Conditions, Driving Functions

%FIXME relate to transition systems, automata ?

Arithmetic Dynamics (TODO)

1943 - Olson - \emph{Dynamical Analogies} -- application of Circuit Diagrams to
Networks (\S\ref{sec:network}) of Mechanical, Hydraulic, Thermodynamic, and
Chemical Components; cf. \emph{Bond Graphs} \S\ref{sec:bond_graph} (Bond1961)
allowing conversion of a System into a State-space Representation
(\S\ref{sec:phase_space})

2006 - Villate - \emph{Introduction to Dynamical Systems: A Hands-on Approach
  with Maxima}

2018 - Schultz, Spivak, Vasilakopoulou - \emph{Dynamical Systems and Sheaves}

(wiki):

the Integral Curves (\S\ref{sec:integral_curve}) for the Differential Equation
governing the System are referred to as \emph{Trajectories} or \emph{Orbits}

when the Differential Equations determining the Evolution Function are Ordinary
Differential Equations (\S\ref{sec:ode}), then the Phase Space $M$ is a Finite
Dimensional Manifold

concepts Dynamical Systems can be extended to Infinite-dimensional Manifolds
that (locally Banach Spaces \S\ref{sec:banach_space}) where the Differential
Equations are Partial Differential Equations (\S\ref{sec:pde})

the Evolution Function $\Phi(x,t)$ (or $\Phi^t$) is often the Solution of a
Differential Equation (\S\ref{sec:differential_equation}) of Motion $\dot{x} =
v(x)$ giving the Time Derivative (\S\ref{sec:time_derivative}) of a Trajectory
$x(t)$ on the Phase Space starting at some Point $x_0$ where $v(x)$ is a Vector
Field (\S\ref{sec:vector_field}) and a Smooth Function providing the Velocity
Vector of the Dynamical System at every Point of the Phase Space $M$, with
Velocity Vectors lying in the Tangent Space (\S\ref{sec:tangent_space}) $T_xM$
of the Point $x$

neither Higher-order Derivatives in the Differential Equation, nor Time
Dependence in $v(x)$ are required since they can be eliminated by considering
Systems of Higher Dimensions

given a Smooth $\Phi^t$, an Autonomous Vector Field can be derived from it
(FIXME: clarify)

\asterism

2014 - MAE5790 - Non-linear Dynamics and Chaos - Strogatz - "Theory of
Dynamical Systems" -
\url{https://www.youtube.com/playlist?list=PLj_l4pOO0YKhJHLQVbjpPlbNyYtT8aAQz}

\emph{Three-body problem} -- no ``general'' Closed-form Solution for every
``Condition'': ``Numerical Methods'' are needed
(FIXME: clarify and xref)

Deterministic Dynamical Systems (\S\ref{sec:deterministic_dynamical_system})

Non-linear Dynamical Systems (\S\ref{sec:nonlinear_dynamical_system})

Chaos (\S\ref{sec:chaos_theory})

Differential Equations $\dot{\vec{x}} = \vec{f}(\vec{x})$ where
$\vec{x} \in \reals^n$ and $\vec{x} = (x_1,\ldots,x_n)$

Component-wise as System of $n$ Coupled Ordinary Differential Equations
(\S\ref{sec:ode_system}):
\begin{align*}
  \dot{x}_1 & = f_1(x_1,\ldots,x_n) \\
            & \vdots \\
  \dot{x}_n & = f_n(x_1,\ldots,x_n)
\end{align*}

$\reals^n$ -- \emph{Phase Space} (\S\ref{sec:phase_space})

$f_1, \ldots, f_n$ -- given Functions

a Dynamical System is \emph{Linear} (\S\ref{sec:linear_dynamical_system}) if
all RHS $x_i$ are of Degree $1$ and are not Multiplied by eachother

a Dynamical System is \emph{Autonomous}
(\S\ref{sec:autonomous_dynamical_system}) if the RHS does not depend on Time,
but a Dynamical System can usually be rewritten as Autonomous by introducing
the Equation $\dot{t} = 1$

Linear Systems in increasing number of Equations:
\begin{itemize}
  \item $n = 1$ -- Resistor-Capacitor (RC) Circuit
  \item $n = 2$ -- Simple Harmonic Oscillator
  \item $n >> 1$
  \item $n = \infty$ (Continuum, Linear PDEs) -- Wave Equation, Maxwell's
    Equations, Schr\"odinger Equation
\end{itemize}

Non-Linear Systems in increasing number of Equations:
\begin{itemize}
  \item $n = 1$ -- Logistic Growth, Drag
  \item $n = 2$ -- Pendulum
  \item $n = 3$ -- Lorenz (Chaos)
  \item $n >> 1$
  \item $n = \infty$ (Continuum, Non-linear PDEs) -- General Relativity,
    Turbulence, Fibrillation
\end{itemize}

Solutions to $\dot{x} = f(x)$ exist and are unique if $f(x)$ and $f'(x)$ are
Continuous ($f$ is ``Continuously Differentiable''
\S\ref{sec:continuously_differentiable})

\emph{Impossibility of Oscillations for a 1D System} -- 1D (First-order)
Systems either approach an Equilibrium (Fixed) Point or else proceed to
Positive of Negative Infinity since all Trajectories $x(t)$ increase or
decrease Monotonically or stay fixed

First-order Systems represent a ``Force'' being balanced by Damping (a
Non-linear spring pulling on a mass with no Intertia); no chance for
\emph{overshoot}

\asterism

\fist Self-Organized Criticality (SOC \S\ref{sec:soc}): Property of Dynamical
Systems (e.g. Abelian Sandpile \S\ref{sec:abelian_sandpile}) that have a
Critical Point (TODO) as an Attractor (\S\ref{sec:attractor_repeller});
``macroscopic behavior'' displays Spatial and/or Temporal Scale-Invariance
(\S\ref{sec:scale_invariance})

\asterism

\url{https://www.youtube.com/watch?v=cu718EbCOPs} Spivak 16:

can't have Identities and Feedback without Partiality %FIXME

Traced Ideals %FIXME

2017 - Baez, Pollard - \emph{A Compositional Framework for Reaction Networks}
-- $\cat{RxNet}$ Cospan Category (\S\ref{sec:cospan_category}) of Open Reaction
Networks (Petri Nets \S\ref{sec:petri_net}), $\cat{Dynam}$ Cospan Category of
Open Dynamical Systems; the mapping associating an Open Reaction Network to its
corresponding Dynamical System is \emph{Compositional} and the mapping taking
an Open Dynamical System to the Relation that ``holds between its
constituents'' in Steady State is also Compositional (FIXME: clarify)

2018 - \url{https://golem.ph.utexas.edu/category/2018/04/dynamical_systems_and_their_st.html} -
\emph{Dynamical Systems and their Steady States} -- Open Reaction Networks
(Petri Nets)

Cospan Category $\cat{Dynam}$ of Open Dynamical Systems

a Dynamical System is usually defined as a Manifold $M$ with Points
representing ``States'', together with a Smooth Vector Field on $M$ describing
the Time Evolution of States; e.g. in a Chemical System, the Manifold is
$\reals^S$ where $S$ is a Finite Set of \emph{Species} and a Vector $\vec{c}
\in \reals^S$ assigns the Concentration of each Species, giving the
Differential Equation of the Dynamical System:
\[
  \frac{d\vec{c}(t)}{dt} = \vec{v}(\vec{c}(t))
\]
where $\vec{c}(t) : \reals \rightarrow \reals^S$ gives Concentrations as a
Function of Time

an \emph{Open Dynamical System} (e.g. a Chemical System where Molecules of some
Species can be ``injected'' or ``removed'') is a Cospan of Finite Sets:
\[
  X \xrightarrow{i} S \xleftarrow{o} Y
\]
where $i$ is Input and $o$ is Output, together with a Vector Field $\vec{v}$ on
$\reals^S$; the Vector Fields give \emph{Decorations} of the Cospan through a
Lax Monoidal Functor:
\[
  F : (\cat{FinSet}, +) \rightarrow (\cat{Set}, \times)
\]
allowing the formation of a Category with Objects of Finite Sets and Morphisms
as Isomorphism Classes of Decorated Cospans, giving rise to the Category
$\cat{Dynam}$ whose \emph{Morphisms} are Open Dynamical Systems

for an Open Dynamical System $(X \xrightarrow{i} S \xleftarrow{o} Y, \vec{v})$
together with a Constant ``Inflow'' $I \in \reals^X$ and a Constant ``Outflow''
$O \in \reals^Y$ , a \emph{Steady State} with Inflows $I$ and Outflows $O$ is a
Constant Vector of Concentrations $\vec{c} \in \reals^S$ such that:
\[
  \vec{v}(\vec{c}) + \vec{i}_*(I) - \vec{o}_*(O) = \vec{0}
\]
where $\vec{i}_*(I)$ is the Vector in $\reals^S$ given by the Inflow
Concentration of all Species marked by the ``Input'' leg of the Cospan; i.e. in
a Steady State the ``Inflows'' and ``Outflows'' exactly compensate for the
``Reaction Velocities''

Black-boxing Functor $\blacksquare : \cat{Dynam} \rightarrow \cat{Rel}$ --
given an Open Dynamical System, records all possible combinations of Input
Concentrations, Output Concentrations, Inflows, and Outflows that hold in
Steady State; called ``Black-boxing'' because it ``discards'' information that
can't be seen at the Inputs and Outputs

the Functoriality of this operation implies that to study the Steady States of
a Complex Open Dynamical System, one can study the Steady States of smaller
individual Systems (Morphisms in $\cat{Dynam}$) and composing them (as
Morphisms in $\cat{Rel}$)

Syntax for \emph{Boxes} (Spivak):
Category $\cat{W}_{\cat{C}}$ of $\cat{C}$-boxes and Wiring Diagrams for a
Category $\cat{C}$ with Finite Products

for a Dynamical System $f = (\reals^S, f^{dyn}, f^{rdt})$ and an Element $(I,O)
\in \reals^X \times \reals^Y$, an \emph{$(I,O)$-steady State} is a State
$\vec{c} \in \reals^S$ such that:
\begin{align*}
  f^{dyn}(I,\vec{c}) & = O \\
  f^{rdt}(\vec{c})   & = O
\end{align*}
where $f^{dyn} : \reals^X \times \reals^S \rightarrow \reals^S$ is the Vector
Field Paramaterized by the Inputs $\reals^X$, giving the Differential Equation
of the System, and $f^{rdt} : \reals^S \rightarrow \reals^Y$ is the ``Readout
Function'' at the Outputs $\reals^Y$

Functor $ODS : \cat{W}_{\cat{Euc}} \rightarrow \cat{S}$ where $\cat{Euc}$ is
the Category of Euclidean Spaces $\reals^n$ and Smooth Maps -- takes a Box $X =
(X^{in},X^{out})$ to the Set of all
$(\reals^{X^{in}}, \reals^{X^{out}})$-dynamical Systems

$\boxtimes$ -- Parallel Composition of two Dynamical Systems

Functor $Mat : \cat{W}_{\cat{C}} \rightarrow \cat{Set}$ of Set-matrices

Steady-state (Monoidal) Natural Transformation $Stst : ODS \rightarrow Mat$ --
assigns each Box $X$ the Function $Stst_X : ODS(X) \rightarrow Mat(X)$ taking a
Dynamical System to its Matrix of Steady States

the fact that $Stst$ is a Natural Transformation together with the fact that
Functors $ODS$ and $Mat$ respect Parallel Composition also indicates that the
Steady States of a Composition of Systems comes from the Composition of the
Steady States of the parts

\url{https://golem.ph.utexas.edu/category/2018/04/dynamical_systems_and_their_st.html}:

(paraphrasing the conclusion)

note that Spivak's approach is broader in scope than Baez \& Pollard, so the
results of Spivak won't be implied by those of Baez \& Pollard

however, setting two Box Dynamical Systems in Serial Composition will
\emph{not} yield the Box System representing the Composition of the Cospan
Systems, so it doesn't seem that Spivak's Compositional results will imply
those of Baez \& Pollard



% --------------------------------------------------------------------
\subsection{Initial Condition}\label{sec:initial_condition}
% --------------------------------------------------------------------

Initial Time

Initial Value

Initial Value Problem (\S\ref{sec:ivp})



% --------------------------------------------------------------------
\subsection{Evolution Function}\label{sec:evolution_function}
% --------------------------------------------------------------------

distinguished Propagator (\S\ref{sec:propagator}) of a Dynamical System

\emph{System Function}, \emph{Transfer Function}

\fist Input Function, Output Function (FIXME)

\begin{itemize}
  \item Time Invariant (TIV) Systems (\S\ref{sec:tiv_system})
\end{itemize}



% --------------------------------------------------------------------
\subsection{Trajectory}\label{sec:trajectory}
% --------------------------------------------------------------------

or \emph{Orbit} or \emph{Path} or \emph{History}

\fist cf. Orbits of Groups acting on Sets (\S\ref{sec:orbit})

Phase Space (\S\ref{sec:phase_space}) ``\emph{Phase Portrait}'': picture of all
the ``qualitatively'' different Trajectories

Given $x \in X$, the Set $\varphi(x,t)$ is called the \emph{Trajectory} of $x$
under the Flow (\S\ref{sec:flow}) $\varphi$. If the Flow is generated by a
Vector Field (\S\ref{sec:vector_field}) then its Trajectories are the Images of
its Integral Curves (\S\ref{sec:integral_curve}).

Conservative Systems (\S\ref{sec:conservative_dynamical_system}):
$\dot{\vec{x}} = \vec{f}(\vec{x})$ is \emph{Conservative} if it has a
``Conserved Quantity'' (``Constant of Motion'') $E(\vec{x})$ that is Constant
on \emph{Trajectories}, but otherwise required to be \emph{not} Constant on any
Open Set;
by not allowing $E$ to be Constant on Open Sets, there are no Attracting Fixed
Points in Conservative Systems



\subsubsection{Action}\label{sec:trajectory_action}

Functional (\S\ref{sec:functional}) taking the Trajectory of a System to a Real
Number

the Function $S$ in the Hamilton-Jacobi Equation (\S\ref{sec:hamilton_jacobi})
is equal to the ``\emph{Classical Action})''



% --------------------------------------------------------------------
\subsection{Poincar\'e Recurrence Theorem}\label{sec:poincare_recurrence}
% --------------------------------------------------------------------

a Dynamical System defined by an Ordinary Differential Equation
(\S\ref{sec:ode}) determines a \emph{Flow Map} (\S\ref{sec:flow}) $f^t$ mapping
the Phase Space to itself

the System is said to be \emph{Volume-preserving} (\emph{Measure-preserving}
\S\ref{sec:measure_preserving_system}) if the Volume of a Set in Phase Space is
Invariant under the Flow;
e.g. all Hamiltonian Systems (\S\ref{sec:hamiltonian_system}) are
Measure-preserving because of Liouville's Theorem

therefore if a Flow preserves Volume and has only \emph{Bounded Orbits
  (Trajectories)}, then for each Open Set there exists Orbits that Intersect
the Set infinitely often



% --------------------------------------------------------------------
\subsection{Discrete-time Dynamical System}\label{sec:discrete_dynamical_system}
% --------------------------------------------------------------------

\emph{Discrete-time Dynamical System} (or \emph{Sampled System}), \emph{Map}

\fist Linear Shift-invariant Systems (\S\ref{sec:lsi_system})

\fist cf. Recurrence Relation (\S\ref{sec:recurrence_relation})

Difference Equations (\S\ref{sec:difference_equation}):
\begin{itemize}
  \item Logistic Map (\S\ref{sec:logistic_map}) -- Non-linear (has an $x^2$
    term)
  \item Tent Map (TODO)
\end{itemize}

MAE5790 Lec.19 - \emph{One dimensional maps}

Iterated Maps

Logistic Map

(R.May 1976)

Feigenbaum Constant

Villate06, Ch.2

State changes only at Discrete Sequence of Time Instants $\{t_0, t_1, \ldots,
\}$; the interval of Time between a consecutive pair of Instants $t_n$ and
$t_{n+1}$ need not be constant for different $n$

First-order System: single State Variable $y$ giving a Sequence of State values
$\{y_0, y_1, y_2, \ldots\}$ at the Instants $\{t_0, t_1, t_2\}$

the \emph{Evolution Equation} is a First-order Difference Equation
(\S\ref{sec:difference_equation}):
\[
  y_{n+1} = F(y_n)
\]
computes State $y_{n+1}$ at an Instant $t_{n+1}$ given the State value $y_n$ at
the previous Instant $t_n$, where $F(y)$ is an \emph{Unknown Function}

given an Initial State $y_0$, successive applications of $F$ will generate the
Sequence of States $y_n$ determining the evolution of the System:
\[
  \{ y_0, F(y_0), F^2(y_0), \ldots, y_n = F^n(y_0) \}
\]

a \emph{Fixed Point} of the System is a $y_0$ such that $F(y_0) = y_0$

\emph{Cobweb Plot} (or \emph{Staircase Diagram}): qualitative behavior of
One-dimensional Iterated Functions (\S\ref{sec:iterated_function}), e.g.
Logistic Map; values of $F(y)$ that are Strictly Less-than (\emph{below}) the
Line defined by $G(y) = y$ move the System in the \emph{Negative} (Left)
direction, and values of $F(y)$ that are Strictly Greater-than (\emph{above})
the Line $G(y) = y$ move the System in the \emph{Positive} (Right) direction;
values of $y$ where $F(y) = G(y)$ (the Intersection Points of the Iterated
Function and the Line $G(y) = y$) are \emph{Fixed Points} of the System

\emph{Repulsive Node} -- derivative $F'(y_0) > 1$; the curve $F$ crosses under
the Line $y=x$ from the left to over the Line on the right

\emph{Repulsive Focus} -- $F'(y_0) < -1$; sequence will move
away but alternate sides

\emph{Attractive Node} -- $0 \leq F'(y_0) < 1$

\emph{Attractive Focus} -- $-1 < F'(y_0) < 0$

for $F'$ equal to $1$ or $-1$, a Fixed Point could either be Attractive or
Repulsive or Attractive from one side and Repulsive from the other

for the Sequence $\{ y_0, y_1, y_2, \ldots \}$ of the Solution of a Dynamical
System $y_{n+1} = F(y_n)$, any Element of the Sequence can be obtained form
$y_0$ by applying the Composed Function $F^n$:
\[
  y_n = F^n(y_0) = F(F(\cdots(F(y_0))))
\]
the Points of a Solution Sequence containing a \emph{Cycle}
(\S\ref{sec:periodic_sequence}) are called \emph{Periodic Points}, and the
number of Periodic Points is called the \emph{Period} of the Cycle and for
a Cycle of Period $n$:
\[
  F^n(y) = y
\]
i.e. the Periodic Points are Fixed Points of the Composed Function $F^n$ where
$n$ is the Period of the Cycle (but none of them are Fixed Points of the
original Function $F$);
the Derivative of $F^n$ as $y_0$ can be computed using the Chain Rule:
\[
  (F^n(y_0))' = F'(y_0)F'(y_1)\cdots{F'(y_{n-1})}
\]
(FIXME: does this only work for $n$ of an $n$-period cycle ???)

for a Cycle of Period $m$, all Points $F^j(y_0)$ with $j < m$ are not Fixed
Points of $F^j$

for the Absolute Value of the Product of the Derivative at the $m$ Points of the
Cycle:
\[
  \prod_{j=0}^{m-1} F'(y_j)
\]
greater than $1$ then the Cycle is \emph{Repulsive}, less than $1$ the Cycle is
\emph{Attractive} and if identical to $1$ the Cycle could be either Attractive
or Repulsive in different regions
(FIXME: is this product equal to the derivative $F^m(y_0)'$ ???)

Discrete First-order Dynamical Systems can be used to Numerically
(\S\ref{sec:numerical_method}) Solve Single-variable Equations-- the problem to
be Solved consisting of finding the Roots (\S\ref{sec:root_finding}) of a Real
Function $f$, i.e. the values of $x$ such that:
\[
  f(x) = 0
\]
For Equations that cannot be solved Analytically, the Numerical Method involves
defining a Dynamical System with Convergent Sequences
(\S\ref{sec:convergent_sequence}) which approach Solutions to the Equation:
\begin{itemize}
  \item Iterative Methods (\S\ref{sec:iterative_method})
  \item Newton's Method (\S\ref{sec:newtons_method})
\end{itemize}


MAE5790, Lec.24:

H\'enon Map -- a 2-dimensional Iterated Mapping



\subsubsection{Period-doubling Bifurcation}\label{sec:period_doubling}

MAE5790 Lec.20

universal aspects of Period-doubling

``Universal Scaling''

Statistical Physics

\emph{Feigenbaum's Constants}

the ratio of successive Period-doublings approaches $\delta = 4.6692...$, when
approaching the onset of Chaos, independent of the choice of underlying Map

$\alpha = -2.5029...$ -- involves the ratio of successive heights of Bifurcation
tines relative to (above or below) ``Super-stable'' points

MAE5790 Lec.21

Superstability -- Quadratic Convergence; $f'(x^*) = 0$; cf. Newton's Method

Renormalization (\S\ref{sec:renormalization}

\emph{Universality Classes} -- global aspects of $f$ are lost, only the order of
the max is important

\emph{Feigenbaum-Cvitanovic Functional Equation}
(\S\ref{sec:functional_equation}): $g(x) = \alpha g^2(\frac{x}{\alpha})$

$g$ is a Fixed Point (specifically, a Saddle Point) in Function Space of the
Higher-order Renormalization Operator $T f(x) = \alpha f^2(\frac{x}{\alpha})$



\subsubsection{Graph Dynamical System}\label{sec:graph_dynamical_system}

\paragraph{Sequential Dynamical System}
\label{sec:sequential_dynamical_system}\hfill

\fist Boolean Networks (\S\ref{sec:boolean_network})



% --------------------------------------------------------------------
\subsection{Continuous-time Dynamical System}
\label{sec:continuous_dynamical_system}
% --------------------------------------------------------------------

\fist Linear Translation-invariant Systems (\S\ref{sec:lti_system})

Villate06, Ch. 3 - \emph{Continuous dynamical systems}:

Time-dependent Function -- $x(t)$

Time Derivatives (\S\ref{sec:time_derivative}) of $x(t)$:
\begin{align*}
  \dot{x} &= \deriv{x}{t} & \ddot{x} &= \deriv{^2x}{t^2} \\
\end{align*}
Derivatives of Functions $y(x)$ with respect to $x$:
\begin{align*}
  y' &= \deriv{y}{x} & y'' &= \deriv{^2y}{x^2} \\
\end{align*}



\subsubsection{Poincar\'e Map}\label{sec:poincare_map}

\emph{Poincar\'e Section}



% --------------------------------------------------------------------
\subsection{Time-Variant System}\label{sec:time_variant_system}
% --------------------------------------------------------------------

% --------------------------------------------------------------------
\subsection{Time-Invariant (TIV) System}\label{sec:tiv_system}
% --------------------------------------------------------------------

a System with a Time-dependent System Function (Evolution Function
\S\ref{sec:evolution_function}) that is not a \emph{direct} Function of Time,
instead it is a Function of the Time-dependent \emph{Input Function}, i.e. the
Transfer Function of the System is only a Function of Time as expressed by the
Input and Output

described by an Autonomous System of ODEs (\S\ref{sec:autonomous_ode_system});
cf. Autonomous Systems (\S\ref{sec:autonomous_dynamical_system})

the ``System Block'' \emph{Commutes} with an arbitrary Delay (FIXME: clarify)



\subsubsection{Linear Shift-Invariant (LSI) System}\label{sec:lsi_system}

Discrete (\S\ref{sec:discrete_dynamical_system})

\fist Linear Translation-invariant (Continuous \S\ref{sec:lti_system})



\subsubsection{Linear Translation-Invariant (LTI) System}
\label{sec:lti_system}

or \emph{Linear Time-Invariant Systems}

Continuous (\S\ref{sec:continuous_dynamical_system})

\fist Linear Shift-invariant (Discrete \S\ref{sec:lsi_system})

for a Linear Time-invariant System, when taking the Laplace Transform
(\S\ref{sec:laplace_transform}) and if the Poles (\S\ref{sec:complex_pole}) in
the Complex Plane are:
\begin{itemize}
  \item in the Right Half Plane (positive Real component) then the System is
    \emph{Unstable}
  \item in the Left Half Plane (negative Real component) then it will be
    \emph{Stable}
  \item on the Imaginary Axis, it will have \emph{Marginal Stability}
\end{itemize}



\subsubsection{Non-linear Time-Invariant (NLTI) System}\label{sec:nlti_system}



% --------------------------------------------------------------------
\subsection{Autonomous Dynamical System}\label{sec:autonomous_dynamical_system}
% --------------------------------------------------------------------

Time does not appear on RHS

normally a Dynamical System can be reformulated as Autonomous by introducing
the Equation $\dot{t} = 1$

Autonomous Systems can be subjected to Phase Plane Analysis
(\S\ref{sec:phase_plane})

Autonomous Differential Equation (\S\ref{sec:autonomous_differential_equation})

cf. Autonomous System of ODEs (\S\ref{sec:autonomous_ode_system}),
Time-invariant Systems (\S\ref{sec:tiv_system})

(wiki): ``any Autonomous System can be transformed into a Dynamical System and,
using very weak assumptions, a Dynamical System can be transformed into an
Autonomous System''

%FIXME: same as time-invariant systems ???



% --------------------------------------------------------------------
\subsection{Deterministic Dynamical System}
\label{sec:deterministic_dynamical_system}
% --------------------------------------------------------------------

Lorenz (\S\ref{sec:lorenz_system})

\fist Deterministic Systems (\S\ref{sec:deterministic_system})



% --------------------------------------------------------------------
\subsection{Non-deterministic Dynamical System}
\label{sec:nondeterministic_dynamical_system}
% --------------------------------------------------------------------

\fist Stochastic Processes (\S\ref{sec:stochastic_process})



% --------------------------------------------------------------------
\subsection{Linear Dynamical System}\label{sec:linear_dynamical_system}
% --------------------------------------------------------------------

In a \emph{Linear Dynamical System} the Variation of a State Vector $\vec{x}$
is equal to a Constant Matrix $A$ multiplied by $\vec{x}$

when $\vec{x}$ Varies Continuously with Time it is a \emph{Flow}
(\S\ref{sec:flow}):
\[
  \frac{d}{dt}\vec{x}(t) = A\vec{x}(t)
\]
otherwise if $\vec{x}$ Varies in Discrete steps it is a \emph{Map}
(\S\ref{sec:discrete_dynamical_system}):
\[
  \vec{x}_{m+1} = A\vec{x}_m
\]
(FIXME: clarify)

If $\vec{x}(t)$ and $\vec{y}(t)$ are two valid Solutions, then so is any Linear
Combination of $\vec{x}(t)$ and $\vec{y}(t)$.

\emph{Superposition Principle}: if $u(t)$ and $w(t)$ satisfy the Differential
Equation for the Vector Field (but not necessarily the Initial Condition) then
so will $u(t) + w(t)$

(wiki): Linear Dynamical Systems have ``exact'' (FIXME: Closed-form ???)
Solutions

\asterism

2014 - MAE5790 - Non-linear Dynamics and Chaos - Strogatz - "Theory of
Dynamical Systems" -
\url{https://www.youtube.com/playlist?list=PLj_l4pOO0YKhJHLQVbjpPlbNyYtT8aAQz}

Linear Dynamical System has all Components $x_i$ Terms only appear to the first
Power and are not Multiplied together

e.g. Simple Harmonic Oscillator $m\ddot{x} + kx = 0$
rewritten in the form of a $\dot{\vec{x}} = \vec{f}(\vec{x})$:
\begin{align*}
  \dot{x_1} & = x_2 \\
  \dot{x_2} & = \frac{-kx_1}{m}
\end{align*}
-- Linear 2nd-order System

\emph{Two-dimensional Linear System}:

$\dot{\vec{p}} = A\vec{p}$

$\vec{p}^* = \vec{0}$ is always a Fixed Point for any given $A$ with the Phase
Portrait (\S\ref{sec:phase_plane}) around that Point determined by the
Eigenvalues (\S\ref{sec:eigenvalue}) and Eigenvectors (\S\ref{sec:eigenvector})
of $A$

Classification of Fixed Points:
\begin{itemize}
  \item \textbf{Saddle Points} -- $|A| < 0$ implies that $\lambda_1 > 0$,
    $\lambda_2 < 0$ are distinct Eigenvalues, which implies there are two
    Linearly Independent Eigenvectors, one a Stable Eigendirection and the
    other an Unstable Eigendirection
  \item \textbf{Attracting/Repelling Points}: \\
    Attracting (Stable) Points -- $|A| > 0$, $\mathrm{tr}(A) < 0$ \\
    Repelling (Unstable) Points -- $|A| > 0$, $\mathrm{tr}(A) > 0$
    \begin{itemize}
      \item \textbf{Nodes} -- $\mathrm{tr}(A)^2 - 4|A| > 0$; Real $\lambda$
        with the same Sign
        \begin{itemize}
          \item $\lambda_1 < \lambda_2 < 0$ -- distinct Negative EigenValues
            gives two Linearly Independent, Stable Eigendirections:
            Trajectories approach the Fixed Point from the ``slower''
            ($\lambda_2$) Eigendirection; in the \emph{reverse-time} direction
            Trajectories move away from the Origin parallel to the ``fast''
            Eigendirection
        \end{itemize}
      \item \textbf{Spirals} -- $\mathrm{tr}(A)^2 - 4|A| < 0$; Complex
        $\lambda$ which are distinct Complex Conjugates $\mu \pm{i\omega}$:
        $\mu$ controls the ``decay rate'' and $\omega$ controls the ``rotation
        rate''; geometric counterpart of a \emph{Damped Harmonic Oscillator}
      \item \textbf{Centers} -- $|A| > 0$, $\mathrm{tr}(A) = 0$; Eigenvalues
        are Imaginary $\lambda \pm i\omega$; every Trajectory is \emph{Closed}
        (Elliptical); ``perfectly periodic motion'', e.g. Simple Harmonic
        Oscillator w/o Damping ($\ddot{p} + p = 0$)
      \item \textbf{Non-isolated} -- $|A| = 0$; Line (or Plane in the case of
        constant $\dot{x} = \vec{0}$) of Fixed Points
      \item \textbf{Repeated Roots} -- $\mathrm{tr}(A)^2 - 4|A| = 0$: every
        Direction is an Eigendirection (``Star Nodes'') or a single
        Eigendirection (``Critically Damped'')
    \end{itemize}
\end{itemize}

Linear (2D) Dynamical Systems cannot have Limit Cycles



% --------------------------------------------------------------------
\subsection{Non-linear Dynamical System}\label{sec:nonlinear_dynamical_system}
% --------------------------------------------------------------------

Non-linear Oscillators, Phase-Locked Loops (PLLs \S\ref{sec:pll}), Lasers
(Physics)

\fist Chaos Theory (\S\ref{sec:chaos_theory}) -- Non-linear Deterministic
Systems must 3rd-order to exhibit Chaos (Lorenz \S\ref{sec:lorenz_system})

2014 - MAE5790 - Non-linear Dynamics and Chaos - Strogatz - "Theory of
Dynamical Systems" -
\url{https://www.youtube.com/playlist?list=PLj_l4pOO0YKhJHLQVbjpPlbNyYtT8aAQz}

e.g. Pendulum:
\begin{align*}
  \dot{x}_1 & = x_2 \\
  \dot{x}_2 & = -sin(x_1)
\end{align*}
-- Non-linear 2nd-order System ($-sin(x_1)$ is a Non-linear Term); cf. Elliptic
Functions
(\S\ref{sec:elliptic_function})

\emph{Two-dimesional Non-linear Systems}

\fist Phase Plane (\S\ref{sec:phase_plane})

only Non-linear (2D) Systems can have Limit Cycles

$\dot{\vec{p}} = \vec{f}(\vec{p})$

Fixed Points $(x^*,y^*)$:
\begin{align*}
  \dot{x} & = f(x,y) \\
  \dot{y} & = g(x,y)
\end{align*}

$u, v$

Jacobian Matrix (\S\ref{sec:jacobian}):
\[
  A = \begin{bmatrix}
    \frac{\partial{f}}{\partial{x}} & \frac{\partial{f}}{\partial{y}} \\
    \frac{\partial{g}}{\partial{x}} & \frac{\partial{g}}{\partial{y}}
  \end{bmatrix}
\]
gives the Linearization (\S\ref{sec:linearization}) around the Fixed
Point $\vec{x}^* = (x^*, y^*)$:
\[
  \dot{\vec{u}} = A \vec{u} + \text{higher order terms}
\]
where the Higher-order Terms can be ignored as $\vec{u} \rightarrow \vec{x}^*$

The Linearization $A$ gives qualitatively correct Dynamics if the Fixed Point
is a \emph{Saddle}, a \emph{Node}, or a \emph{Spiral}--
Degenerate Nodes, Stars, Centers, or Non-isolated Fixed Points \emph{may be
altered by Higher-order Terms} since they are ``borderline'' cases in the Phase
Transition Diagrams

MAE5790 Lec.11 - Weakly Non-linear Systems, Averaging Theory



% --------------------------------------------------------------------
\subsection{Conservative Dynamical System}
\label{sec:conservative_dynamical_system}
% --------------------------------------------------------------------

(MAE 5790-7)

$\dot{\vec{x}} = \vec{f}(\vec{x})$ is \emph{Conservative} if it has a
``Conserved Quantity'' (``Constant of Motion'') $E(\vec{x})$ that is Constant
on \emph{Trajectories}, but otherwise is required to be \emph{not} Constant on
any Open Set;
by not allowing $E$ to be Constant on Open Sets, there are no Attracting Fixed
Points in Conservative Systems

\emph{Example - Mechanical System with 1 Degree of Freedom}:
\[
  m\ddot{x} = F(x) = \frac{-dV}{x}
\]
with Force depending only on the Position of $x$, expressed as the negative
Derivative of a Potential (Potential Energy $V$), and is independent of Time
$t$ (no Driving) and Velocity $\dot{x}$ (no Damping)

under these conditions there is a \emph{Conserved Quantity} $E$:
\[
  E = \frac{1}{2}m\dot{x}^2 + V(x)
\]
that is the combined Kinetic Energy ($\frac{1}{2}m\dot{x}^2$) and Potential
Energy $V(x)$

$E$ is Constant on \emph{Trajectories}

Proof:
\[
  \frac{d}{dt}\Big(\frac{m\dot{x}^2}{2} + V(x(t))\Big) = 0
\]
i.e. the quantity inside the parentheses is \emph{constant}


\emph{Example - Particle in a Double-well Potential}

Potential Energy Function:
\[
  V(x) = -\frac{1}{2}x^2 + \frac{1}{4}x^4
\]

Second-order Equation of the System:
\begin{align*}
  \ddot{x} & = \frac{dV}{dx} = x - x^3
\end{align*}

rewritten as a Pair of First-order Equations:
\begin{align*}
  \dot{x} & = y \\
  \dot{y} & = x - x^3
\end{align*}

Jacobian Matrix:
\[
  \begin{bmatrix}
    0        & 1 \\
    1 - 3x^2 & 0 \\
  \end{bmatrix}
\]

at Fixed Points (i.e. where $\dot{y} = 0$):
\begin{itemize}
  \item $(0,0)$ -- Jacobian has Trace of $0$ and Determinant of $-1$ (Saddle
    Point)
  \item $(1,0)$ -- Jacobian has Trace of $0$ and Determinant of $2$ (Linear
    Center)
  \item $(-1,0)$ -- Jacobian is same as for $(1,0)$ (Linear Center)
\end{itemize}

there are two ``Homoclinic Orbits'' from the central Saddle Point

$(1,0)$ and $(-1,0)$ are ``true'' Centers due to Conservation of Energy

$E(x,y) = \frac{1}{2}y^2 - \frac{1}{2}x^2 + \frac{1}{4}x^4$ -- Constant on
Trajectories; usually Closed Curves

%FIXME: are the centers above actually ``non-linear centers'' ???

Thm. \emph{Non-linear Centers in 2D Conservative Systems}:
For Conservative $\dot{\vec{x}} = \vec{f}(\vec{x})$ with $\vec{x} \in
\reals^2$, Continuously Differentiable $\vec{f}$, and $E(\vec{x})$ a Conserved
Quantity, with Isolated Fixed Point $\vec{x}^*$, if $\vec{x}^*$ is a Local
Minimum or Local Maximum of $E(\vec{x})$ is a Center, i.e. all Trajectories
sufficiently close to $\vec{x}^*$ are Closed


\emph{Example - Dimensionless Pendulum}

Phase Space is a Cylinder (not a Plane)

Non-linear Center

``Heteroclinic Orbits'' connect two Saddle Points



% --------------------------------------------------------------------
\subsection{Multibody Dynamical System}\label{sec:multibody_system}
% --------------------------------------------------------------------

(wiki):

Dynamic Simulation, Molecular Dynamics

Multibody Simulation (MBS)

Degree of Freedom -- number of independent Kinematical possibilities for
movement (e.g. Translation, Rotation)

\emph{Kinematic Behavior}

\emph{Kinematic Constraints} (\S\ref{sec:kinematic_constraint})

\emph{Kinematic Chain} -- a number of Rigid Bodies (``Links'') connected by
Kinematic Pairs (``Joints'')

\emph{Kinematic Pairs} -- ``Joints'' connecting two Rigid Bodies (``Links'') in
a Kinematic Chain

\emph{Rigid Bodies} -- ``Links'' connected by Kinematic Pairs (``Joints'') in a
Kinematic Chain

\emph{Type Synthesis}: enumeration of Topologies of Kinematic Chains

\emph{Constraint Conditions} -- implies the \emph{restriction} in the
Kinematical Degrees of Freedom of one or more Bodies

Constraints are usually an Algebraic (Polynomial) Equation
(\S\ref{sec:polynomial_equation}) defining the relative Translation or Rotation
between two Bodies, but may also constrain the relative Velocity between two
Bodies or a Body and a ``ground'' Plane; also Non-classical Constraints may
introduce a new unknown Coordinate, e.g. as in Sliding Joints where a Body is
allowed to move along the Surface of another Body

in the case of \emph{Contact}, the Constraint Condition is based on
\emph{Inequalities} so that the Constraint does not \emph{permanently} restrict
the Degrees of Freedom of the Bodies

Constraint Conditions $C_i = 0$

Lagrange Multipliers (\S\ref{sec:lagrange_multiplier}) $\lambda_i$ related to
Constraint Condition $C_i$ represents a Force or a ``Moment''; Lagrange
Multipliers do no ``Work'' compared to External Forces, i.e. they do not change
the Potential Energy of a Body

Constraint Algorithms (Computational Chemistry): methods for satisfying
Newtonian Motion of a Rigid Body consisting of Mass Points

the motion of $N$ Particles can be described by a Set (FIXME: system ???) of
Second-order ODEs by Newton's Second Law, written in Matrix Form:
\[
  \mat{M} \cdot \frac{d^2\vec{q}}{dt^2}
    = \vec{f} = -\frac{\partial{V}}{\partial{q}}
\]
where $\mat{M}$ is the \emph{Mass Matrix}, $\vec{q}$ is the Vector of
Generalized Coordinates describing Particle Positions, $\vec{f}(\vec{q})$
represents the Generalized Forces, and Scalar-valued Function $V(\vec{q})$
represents the Potential Energy

\emph{in the absence of Constraints} the Mass Matrix $\mat{M}$ is a $3N \times
3N$ Diagonal Square Matrix with the Particle Masses along the diagonal

in the presence of $M$ Constraints, the Coordinates $\vec{q}$ must also satisfy
$M$ Time-independent Algebraic (Polynomial) Equations:
\[
  \vec{g} = (g_1(\vec{q}) = 0, \ldots, g_M(\vec{q}) = 0)
\]
giving a System of DAEs (\S\ref{sec:dae_system}) to Solve

\fist a System of DAEs differs from an Implicit ODE System
(\S\ref{sec:ode_system}) that may be rendered Explicit in that the Jacobian
Matrix (\S\ref{sec:jacobian}) is Singular (Non-invertible
\S\ref{sec:singular_matrix}) for a System of DAEs

in practical terms, the Solution to a DAE System depends on the Derivatives of
the \emph{Input Signal} (\S\ref{sec:signal_flow}), and not just the ``Signal
itself'' as in the case of ODEs (FIXME: clarify); e.g. see Systems with
Hysteresis such as the Schmitt Trigger

the ``simplest'' approach to Solving is to define new Generalized Coordinates
that are Unconstrained, eliminating the Algebraic (Polynomial) Equations and
reducing the problem to an ODE; e.g. the motion of a Rigid Body is described by
six Independent, Unconstrained Coordinates, instead of describing the Positions
of the Particles that make it up and the Constraints maintaining their relative
Distances-- this approach may become complex as the Mass Matrix may become
Non-diagonal and depend on the Generalized Coordinates

a second approach introduces Explicit Forces working to maintain Constraints--
this approach may be computationally inefficient

a third approach is to use methods such as Lagrange Multipliers
(\S\ref{sec:lagrange_multiplier}) or Projection
(\S\ref{sec:orthogonal_projection}) to determine Coordinate ``adjustments''
necessary to Satisfy the Constraints



\subsubsection{Kinematic Constraint}\label{sec:kinematic_constraint}

%TODO: move this section ?

UPenn MEAM 535
\url{https://alliance.seas.upenn.edu/~meam535/cgi-bin/pmwiki/uploads/Main/Constraints10.pdf}



\paragraph{Unilateral Constraint}\label{sec:unilateral_constraint}\hfill

\paragraph{Bilateral Constraint}\label{sec:bilateral_constraint}\hfill

\paragraph{Complementarity Constraint}
\label{sec:complementarity_constraint}\hfill

\fist Linear Complementarity Problem (LCP \S\ref{sec:linear_complementarity})



\paragraph{Holonomic Constraint}\label{sec:holonomic_constraint}\hfill

(wiki):

a \emph{Holonomic Constraint} is expressable as a Function: (FIXME: equation
???)
\[
  f(x_1, x_2, x_3, \ldots, x_N, t) = 0
\]
depending only on Coordinates $x_j$ and Time $t$

example: rolling a Cylinder without slipping-- returning to the starting
position will always be in the same orientation

\fist cf. Holonomic Functions (\S\ref{sec:holonomic_function}), Holonomic
Modules (\S\ref{sec:holonomic_module})

Constraint on \emph{Position}

\asterism

2013 - Tanedo - \emph{Notes on non-holonomic constraints} -
\url{https://www.physics.uci.edu/~tanedo/files/teaching/P3318S13/Sec_05_nonholonomic.pdf}

only Holonomic Constraints reduce the Degrees of Freedom: a Holonomic
Constraint can be Integrated (to give $f(q, t) = 0$) and then ``applied'' to
reduce the Degrees of Freedom in a System

a Holonomic Constraint is one where there is \emph{no} Geometric Holonomy
(Path-dependence)

cf. Non-integrability (Non-integrable Systems \S\ref{sec:nonintegrable_system})

Involutions are Holonomic Constraints, i.e. an Involution is a Tangent Bundle
Distribution (\S\ref{sec:tangent_bundle_distribution}) which is Closed under
the Lie Bracket (FIXME: clarify)

all Holonomic Constraints can be written as ``seemingly'' Non-holonomic
Constraints that are Integrable

\asterism

if a (Mechanical) System is both Holonomic and Monogenic (FIXME: xref), then it
is possible to derive Lagrange's Equations from d'Alembert's Principle or
Hamilton's Principle

Conservative Systems

cf. \emph{Holonomy} (\S\ref{sec:holonomy})

\url{https://physics.stackexchange.com/questions/409951/what-are-holonomic-and-non-holonomic-constraints/410034}:

\emph{Holonomy Group} of a Connection (\S\ref{sec:connection}) is the Set of
Transformations an ``object'' can ``experience'' when it is Parallel
Transported in a Loop-- if the associated Holonomy Groups of a Constraint are
``Non-trivial'', then the Constraint cannot be Holonomic
(\S\ref{sec:holonomic_constraint}) because the Orientation of the object will
depend on the Loop traversed, not just the current State (FIXME: clarify)



\paragraph{Non-holonomic Constraint}\label{sec:nonholonomic_constraint}\hfill

Constraint on \emph{Velocity}

2013 - Tanedo - \emph{Notes on non-holonomic constraints} -
\url{https://www.physics.uci.edu/~tanedo/files/teaching/P3318S13/Sec_05_nonholonomic.pdf}

a Tangent Bundle Distribution (\S\ref{sec:tangent_bundle_distribution}) is a
manifestation of a Non-holonomic Constraint: it is a restriction of the Tangent
Space (Velocity Space), i.e. a Velocity Constraint

example: Rolling a Unit Sphere without slipping along a closed Path can
'reorient' the Sphere depending on the Path

2004 - Flannery - \emph{The Enigma of Nonholonomic Constraints}

Lagrange Multipliers can be used for Non-holonomic Constraints that are
\emph{Linear} in $\dot{q}$



\paragraph{Semi-holonomic Constraint}\label{sec:semiholonomic_constraint}\hfill

\url{https://physics.stackexchange.com/a/410110/36436}:

depends Affinely (\S\ref{sec:affine_transformation}) on generalized Velocities



\paragraph{Rheonomic Constraint}\label{sec:rheonomic_constraint}\hfill

Time is explicit in the Constraint Equation



\paragraph{Scleronomic Constraint}\label{sec:scleronomic_constraint}\hfill

Time is not explicitly present in the Constraint Equation



\paragraph{Equality Constraint}\label{sec:equality_constraint}\hfill

(sometimes \emph{Bilateral Constraint})



\paragraph{Inequality Constraint}\label{sec:inequality_constraint}\hfill

(sometimes \emph{Unilateral Constraint})



\paragraph{Conservative Constraint}\label{sec:conservative_constraint}\hfill

\paragraph{Dissipative Constraint}\label{sec:dissipative_constraint}\hfill



% --------------------------------------------------------------------
\subsection{Measure-preserving Dynamical System}
\label{sec:measure_preserving_system}
% --------------------------------------------------------------------

\fist cf. Measure-preserving Maps (\S\ref{sec:measure_preserving_map})

Probability Space (\S\ref{sec:probability_space})

\fist Ergodic Theory (\S\ref{sec:ergodic_theory})

from Poincar\'e's Theorem (\S\ref{sec:poincare_recurrence}): a System is said to
be \emph{Measure-preserving} or \emph{Volume-preserving} if the Volume of a Set
in Phase Space is Invariant under the Flow Map (\S\ref{sec:flow}) $f^t$ mapping
Phase Space to itself

``Analytical Mechanics'' -- System as a whole is represented by \emph{Scalar}
properties of motion, usually total Kinetic Energy and Potential Energy (cf.
``Vectorial (Netwonian) Mechanics''); Equations of Motion are derived from the
Scalar quantity by some underlying principle of the Scalar's Variation:
\begin{itemize}
  \item Lagrangian Systems (\S\ref{sec:lagrangian_system}) -- Function of the
    Coordinates and Velocities
  \item Hamiltonian Systems (\S\ref{sec:hamiltonian_system}) -- Function of the
    Coordinates and Momenta
\end{itemize}



\subsubsection{Lagrangian System}\label{sec:lagrangian_system}

Classical Mechanics: Equations of Motion are First- and Second-order
Differential Equations (\S\ref{sec:differential_equation}) on a Manifold $M$ or
various Fiber Bundles $Q$ over $\reals$; a Solution to the Equations of Motion
is called a \emph{Motion}

in Classical Field Theory all Field Systems are the Lagrangian Ones

related to Hamiltonian Mechanics (\S\ref{sec:hamiltonian_system}) by the
Tautological $1$-form (\S\ref{sec:tautological_1form})

Hamiltonian (and the Routhian) can be obtained from Legendre Transform
(\S\ref{sec:legendre_transform}) of the Lagrangian

Generalized Coordinates (\S\ref{sec:generalized_coordinate}) in Lagrangian
Mechanics are related to Canonical Coordinates
(\S\ref{sec:canonical_coordinate}) of Hamiltonian Mechanics
(\S\ref{sec:hamiltonian_system}) by the Hamilton-Jacobi Equations
(\S\ref{sec:hamilton_jacobi})

\[
  \mathcal{L} = T - V
\]
with $T$ the total Kinetic Energy and $V$ the total Potential Energy of the
System

\emph{Lagrangian Relations}, \emph{Minimum Princples}

``Black-box Functor'': Category of Circuits $\rightarrow$ Category of
Symplectic Vector Spaces (Baez2015 - \emph{Passive Linear Networks})

\url{https://golem.ph.utexas.edu/category/2018/04/props_in_network_theory.html}

Black-box Functor respects the Symplectic Structure, i.e. it is a
\emph{Lagrangian Relation}

whenever a System obeys a \emph{Minimum Principle}, it establishes a Lagrangian
Relation between Input and Output Variables

\begin{itemize}
  \item \emph{Principle of Least Action} (or \emph{Principle of Stationary
    Action})
  \item \emph{Principle of Minimum Power} -- a Circuit made of Linear Resistors
    acts to \emph{Minimize} the total Power Dissapation in the form of Heat
\end{itemize}

generally, for any System goverened by a Minimum Principle, Black-boxing should
give a Functor to some Category where Morphisms are Lagrangian Relations

$\blacksquare : \cat{Circ}_k \rightarrow \cat{LagRel}_k$

Black-boxing Non-linear Circuits, i.e. including Voltage and Current Sources,
gives \emph{Lagrangian Affine Relations} between Symplectic Vector Spaces
(\S\ref{sec:symplectic_vectorspace})

\asterism

\emph{Noether's Theorem} asserts that to every Continuous Symmetry of ``the''
Lagrangian Physical System (Prequantum Field Theory) there is a naturally
associated \emph{Conservation Law} stating the conservation of a ``Charge''
(Conserved Current) when the Equations of Motion hold (FIXME: clarify)

e.g. the \emph{Time-translation} Invariance of a Physical System equivalently
means that the quantity of \emph{Energy} is conserved, and the
\emph{Space-translation} Invariant of a Physical System means that
\emph{Momentum} is preserved



\subsubsection{Hamiltonian System}\label{sec:hamiltonian_system}

(note: it may be customary to use $q$ for positions and $p$ for momenta, but
here $p$ is position and $q$ is momentum; see Canonical Coordinates
\S\ref{sec:canonical_coordinate})

(wiki):

A \emph{Hamiltonian System} is a Dynamical System completely described by the
Scalar Function:
\[
  H(\vec{p},\vec{q},t)
\]
called \emph{the Hamiltonian}, where the State $\vec{s}$ of the System is
described by Generalized Coordinates for Position $\vec{p}$ and Momentum
$\vec{q}$ where both are Vectors of the same Dimension $N$:
\[
  \vec{s} = (\vec{p},\vec{q})
\]
with the Evolution Equation given by \emph{Hamilton's Equations}:
\begin{align*}
  \frac{d\vec{p}}{dt} & = \frac{\partial{H}}{\partial{\vec{q}}} \\
  \frac{d\vec{q}}{dt} & = -\frac{\partial{H}}{\partial{\vec{p}}}
\end{align*}
and the Trajectory (\S\ref{sec:trajectory}) $\vec{s}(t)$ is the Solution of
the Initial Value Problem (\S\ref{sec:ivp}) defined by Hamiltonian's Equations
and the Initial Condition $\vec{s}(0) = \vec{s}_0 \in \reals^{2N}$

when the Hamiltonian is \emph{not} explicitly Time Dependent, i.e.
$H(\vec{p},\vec{q},t) = H(\vec{p},\vec{q})$, then the Hamiltonian is a
constant equal to the Total Energy of the System (e.g. a Pendulum, a Harmonic
Oscillator, Dynamical Billiards)

\fist Hamilton-Jacobi Equation (\S\ref{sec:hamilton_jacobi})

\emph{Symplectic Structure} (Symplectic Geometry
\S\ref{sec:symplectic_geometry}); related to Lagrangian Mechanics
(\S\ref{sec:lagrangian_system}) by the Tautological $1$-form
(\S\ref{sec:tautological_1form}); arises in formalisms of Classical Mechanics
considering either the Even-dimensional Phase Space (\S\ref{sec:phase_space})
of a Mechanical System or the Odd-dimensional Constant-energy Hypersurface (see
\emph{Contact Geometry} \S\ref{sec:contact_geometry})

Hamiltonian (and the Routhian) can be obtained from Legendre Transform
(\S\ref{sec:legendre_transform}) of the Lagrangian

\fist Symplectic Integrator (\S\ref{sec:symplectic_integrator}): Subclass of
Geometric Integration Schemes for Hamiltonian Systems (\emph{Canonical
  Transformations}-- changes of Canonical Coordinates
\S\ref{sec:canonical_coordinate} that preserve Hamilton's Equations)

Generalized Coordinates (\S\ref{sec:generalized_coordinate}) in Lagrangian
Mechanics (\S\ref{sec:lagrangian_system}) are related to Canonical Coordinates
(\S\ref{sec:canonical_coordinate}) of Hamiltonian Mechanics by the
Hamilton-Jacobi Equations (\S\ref{sec:hamilton_jacobi})

a Hamiltonian is \emph{Separable} if it can be written in the form:
\[
  H(p,q) = T(q) + V(p)
\]
where $T$ is the Kinetic Energy and $V$ is the Potential Energy

Liouville's Theorem: corollary of the Symplectic Structure property of
preservation of Infinitesimal Phase-space Volume

Scott Aaronson (\url{http://www.scottaaronson.com/blog/?p=3294}):

``\emph{the [Physical] Universe is ultimately made of Quantum Fields
  evolving by some Hamiltonian}''



% --------------------------------------------------------------------
\subsection{Dissipative System}\label{sec:dissipative_system}
% --------------------------------------------------------------------

Thermodynamics



% --------------------------------------------------------------------
\subsection{Complex Dynamics}\label{sec:complex_dynamics}
% --------------------------------------------------------------------

Analytic Functions (\S\ref{sec:analytic_function})



\subsubsection{Julia Set}\label{sec:julia_set}

\subsubsection{Fatou Set}\label{sec:fatou_set}



% ====================================================================
\section{Communicating System}\label{sec:communicating_system}
% ====================================================================

Denielou-Yoshida13 -- arises from a Choreographed collection of
Communicating Finite State Machines (CFSMs
\S\ref{sec:communicating_fsm}) in the context of Multiparty Session
Types (\S\ref{sec:multiparty_session})

Linear Multirole Logic (LMRL \S\ref{sec:lmrl})

\fist Process Calculus (\S\ref{sec:process_calculus})

Lange-Tuosto-Yoshida15:

\emph{Generalized Multiparty Compatibility} (GMC) -- Decidable
Condition characterizing a Set of Communicating Systems for which
questions of Safety (Deadlock-freedom, Orphan Messages, and
Unspecified Reception Configuration) can be Decided

\emph{Communicating System}:
\[
  \class{S} = (M_p)_{p \in \class{P}}
\]
where $M_p = (Q_p, q_{0p}, \Sigma, \delta_p)$ are CFSMs for each
Participant $p \in \class{P}$

\emph{Configuration} of $\class{S}$ %FIXME



% --------------------------------------------------------------------
\subsection{Choreography}\label{sec:choreography}
% --------------------------------------------------------------------

\emph{Choreographies}: ``Models of Interactions among Software
Components from a \emph{Global} point of view'' --
Lange-Tuosto-Yoshida15

\fist Global Types (\S\ref{sec:global_type}) give \emph{Choreographic
  Specifications} of Interactions (\S\ref{sec:interaction_geometry})



\subsubsection{Graphical Choreography}\label{sec:graphical_choreography}

or \emph{Global Graphs}

\fist Labelled Graph (\S\ref{sec:labelled_graph})

\begingroup

\newcommand{\party}{\mono}

\emph{From Communicating Machines to Graphical Choreographies} \\
Lange-Tuosto-Yoshida 2015

includes a Haskell prototype

Ng-Yoshida16 \emph{Static Deadlock Detection for Concurrent Go by
  Global Session Graph Synthesis} builds on this work: tool extracts
Communication Operations as Session Types, converts into CFSMs, and
then applies the Choreography Synthesis to generate a Global Graph
representing the overal Communication pattern

\fist Communicating Finite State Machines (CFSMs
\S\ref{sec:communicating_fsm})

\begin{itemize}
\item CFSMs as \emph{Behavioral Specifications} of \emph{Distributed
  Components} from which a \emph{Choreography} can be built
\item Sound and Complete characterization of ``safe'' CFSMs from which
  Global Graphs can be constructed
\item CFSM model based on Asynchronous FIFO Message-passing
  Communication
\end{itemize}

presented Algorithm produces a Choreography expressed as a Global Graph
(closely related to BPMN 2.0 Choreography) given a Set of CFSMs (a Set
of Behavioral Specifications of components Interacting through
Asynchronous FIFO Message Passing)

\fist an example implementation for Golang in Ng-Yoshida16

TODO: could a Categorization be given in terms of Decorated Cospan
Categories (\S\ref{sec:decorated_cospan_category}) ?

\asterism

\emph{Note on notation}: Lange-Tuosto-Yoshida15 uses $A$ for
Alphabet and $Act$ for Actions, here we use $\Sigma$ for Alphabet and
$A$ for Actions

detailed Proofs can be found in the full version of
Lange-Tuosto-Yoshida15 (36pp.):
\url{http://www.doc.ic.ac.uk/~jlange/papers/lty15.pdf}

Haskell code:

\url{https://bitbucket.org/julien-lange/gmc-synthesis}


\textbf{CFSMs}

$\Sigma = \{ a,b,c,d,\ldots \}$ -- Finite Alphabet (Messages) \\
$\class{P} = \{ \party{p, q, r, s},\ldots\}$
-- Finite Set of \emph{Participants} \\
$C \defeq \{ \party{pq}
  \ |\ \party{p,q} \in \class{P}, \party{p \neq q} \}$
-- Set of \emph{Channels} \\
$A \defeq C \times \{!,?\} \times \Sigma$ -- Set of \emph{Actions} \\
$\ell$ -- ranges over Actions \\
$\Sigma^*$, $A^*$ -- Set of Finite Words on $\Sigma$, $A$ \\
$\varphi$ -- ranges over $\Sigma^*$, $A^*$ \\

\emph{CFSM} $M$ as a $4$-tuple:
\[
  M = (Q, q_0, \Sigma, \delta)
\]
$Q$ -- Finite Set of \emph{States} \\
$q_0 \in Q$ -- \emph{Initial State} \\
$\delta \subseteq Q \times A \times Q$ -- Set of \emph{Transitions}
that are \emph{Labelled} by Actions \\

$\party{pq}!a$ -- Action Label representing Sending a Message $a$ from
$\party{p}$ to $\party{q}$ \\
$\party{pq}?a$ -- Action Label representing Receiving a Message $a$ by
$\party{q}$ from $\party{p}$

$\class{L}(M) \subseteq A^*$ -- Language on $A$ Accepted by the
Automaton corresponding to CFSM $M$ where each State of $M$ is an
Accepting State %TODO explain accepting state

For $q \in Q$:
\begin{itemize}
  \item $q$ is a \emph{Final State} if $q$ has no Outgoing Transitions
  \item $q$ is a \emph{Sending State} if $q$ has all Outgoing
    Transitions that are Labelled by Send Actions
  \item $q$ is a \emph{Receiving State} if $q$ has all Outgoing
    Transitions that are Labelled by Receive Actions
  \item $q$ is a \emph{Mixed State} if $q$ has both Send and
    Receive Actions as Outgoing Transitions
\end{itemize}

\emph{Deterministic CFSMs}

CFSM $M$ is \emph{Deterministic} if for all States $q \in Q$ and all
Actions $\ell \in A$:
\begin{itemize}
  \item if $(q,\ell,q'),(q,\ell,q'') \in \delta$ then $q = q''$
\end{itemize}

\emph{Minimal CFSMs}

CFSM $M$ is \emph{Minimal} if there are no other CFSMs $M'$ with fewer
States and Transitions such that $\class{L}(M) = \class{L}(M')$


\textbf{Communicating Systems}

\emph{Communicating System}:
\[
  \class{S} = (M_p)_{p \in \class{P}}
\]
where $M_p = (Q_\party{p}, q_{0\party{p}}, \Sigma, \delta_\party{p})$
are CFSMs for each Participant $\party{p} \in \class{P}$

\emph{Configurations}

$s = (\vec{q};\vec{w})$ -- a \emph{Configuration} of $\class{S}$
where:
\begin{itemize}
  \item $\vec{q} = (q_\party{p})_{\party{p} \in \class{P}})$ with
    $q_\party{p} \in Q_\party{p}$ is the \emph{Control State} and
    $q_\party{p} \in Q_\party{p}$ is the \emph{Local State} of CFSM
    $M_\party{p}$
  \item $\vec{w} = (w_\party{pq})_{\party{pq} \in C}$ with
    $w_\party{pq} \in \Sigma^*$
\end{itemize}
$s_0 = (\vec{q}_0;\vec{\varepsilon})$ -- \emph{Initial Configuration}
of $\class{S}$ with $\vec{q}_0 = (q_{0\party{p}})_{\party{p} \in
  \class{P}}$

\emph{Configuration Reachability}

$s \xrightarrow{\ell} s'$ -- Configuration $s' = (\vec{q}',\vec{w}')$
is \emph{Reachable} from Configuration $s = (\vec{q},\vec{w})$ by
Transition $\ell$ if there is a Symbol (Message) $a \in \Sigma$ such
that either:
\begin{itemize}
  \item $\ell = \party{sr}!a$ and $(q_\party{s},\ell,q_\party{s}') \in
    \delta_\party{s}$
  \item $q_\party{p}' = q_\party{p}$ for all $\party{p \neq s}$
  \item $w_\party{sr}' = w_\party{sr}.a$ and $w_\party{pq}' =
    w_\party{pq}$ for all $\party{pq \neq sr}$
\end{itemize}
or:
\begin{itemize}
  \item $\ell = \party{sr}?a$ and $(q_\party{r},\ell,q_\party{r}') \in
    \delta_\party{r}$
  \item $q_\party{p}' = q_\party{p}$ for all $\party{p \neq r}$
  \item $w_\party{sr} = a.w_\party{sr}'$ and $w_\party{pq}' =
    w_\party{pq}$ for all $\party{pq \neq sr}$
\end{itemize}
%TODO some explanation

$Rc(\class{S}) = \{ s \ |\ s_0 \rightarrow^* s \}$ -- Set of
\emph{Reachable Configurations} of Communicating System $\class{S}$
where $\rightarrow^*$ is the Reflexive and Transitive Closure of the
Reachability Relation $\rightarrow$

\emph{$k$-bounded Transition Sequences}

A Sequence of Transitions is \emph{$k$-bounded} if no Channel of any
intermediate Configuration in the Sequence contains more than $k$
Messages.

$Rc_k(\class{S})$ -- the \emph{$k$-reachability Set} of Communicating
System $\class{S}$ is the largest Subset of $Rc(\class{S})$ within
which each Configuration $s$ can be reached by a $k$-bounded Sequence
from $s_0$


\textbf{Safe Communicating Systems}

\emph{Deadlock Configurations}

Configuration $s = (\vec{q},\vec{w})$ is a \emph{Deadlock
  Configuration} when:
\begin{itemize}
  \item $\vec{w} = \vec{e}$
  \item there exists an $\party{r} \in \class{P}$ such that
    $(q_\party{r},\party{sr}?a,q_\party{r}') \in \delta_\party{r}$
  \item for all $\party{p} \in \class{P}$ and $q_\party{p}$ is a
    Receiving or Final State
\end{itemize}
i.e. all Buffers are empty, there is at least one Machine waiting for
a Message, and all the other Machines are either in a Final or
Receiving State

\emph{Orphan Message Configurations}

Configuration $s = (\vec{q},\vec{w})$ is an \emph{Orphan Message
  Configuration} if all $q_\party{p} \in \vec{q}$ are Final but
$\vec{w} \neq \vec{\varepsilon}$, i.e. there is at least one non-empty
Buffer and all Machines are in a Final State

\emph{Unspecified Reception Conrigurations}

Configuration $s = (\vec{q},\vec{w})$ is an \emph{Unspecified
  Reception Configuration} if there exists an $\party{r} \in
\class{P}$ such that $q_\party{r}$ is a Receiving State and
$(q_\party{r},\party{sr}?a,q_\party{r}') \in \delta_\party{r}$ Implies
$0 < |w_\party{sr}|$ and $w_\party{sr} \notin aA^*$, i.e.
$q_\party{r}$ is prevented from Receiving any Messages from any of its
Buffers %TODO better explanation ?


Def. (\emph{Safe Communicating System}) A Communicating System
$\class{S}$ is \emph{Safe} if each $s \in Rc(\class{S})$ is not a
Deadlock, Orphan Message, or Unspecified Reception Configuration.


\textbf{Concurrent Actions}

The following definitions are provided as a means of specifying a
Subset of Safe Communicating Systems from which \emph{Global Graphs}
can be constructed, namely by identifying Sets of \emph{Concurrent
  Actions}, i.e. Actions that can be \emph{Interleaved}.

Communicating Systems that are amenable for being transformed into
Global Graphs will be identified through their \emph{Synchronous
  Transition System} (see below).


\emph{Equivalence Classes on CFSM Transitions}

$act(q,q') \defeq \{ \ell \ |\ (q,\ell,q') \in \delta \}$ -- Set of
Labels for all Transitions between two given States $q,q' \in Q$

$\lozenge,\blacklozenge \subseteq \delta \times \delta$ -- smallest
Equivalence Relations on Transitions that respectively contain the
Relations $\underline{\lozenge}$:
\begin{flalign*}
\quad (q_1,\ell,q_2)\underline{\lozenge} (q_1',\ell,q_2')
  \Longleftrightarrow
  \ell \notin act(q_1,q_1') = act(q_2,q_2') \neq \varnothing &
\end{flalign*}
and $\underline{\blacklozenge}$:
\begin{flalign*}
\quad (q_1,\ell,q_2)\underline{\blacklozenge} (q_1',\ell,q_2')
  \Longleftrightarrow &
  (q_1,\ell,q_2)\underline{\lozenge}(q_1',\ell,q_2') \wedge
   \forall(q,\ell,q') \in [(q_1,\ell,q_2)]^\lozenge . & \\
   & act(q_1,q) = act(q_2,q') \wedge act(q_1',q) = act(q_2',q') &
\end{flalign*}
where $[(q,\ell,q')]^\lozenge$ is the Equivalence Class of
$(q,\ell,q')$ with respect to $\lozenge$

Intuitively, two Transitions are $\blacklozenge$-related if they refer
to the same Action up to Interleaving.
%TODO more explanation

\emph{note on the above $\underline{\blacklozenge}$ definition}: what
is being stated can more easily be worked out on paper; essentially
two ``parallel'' Labelled Transitions $A$, $B$ sharing the same Label
$\ell$ are $\underline{\blacklozenge}$-related when for every
Transition $C$ in the $\lozenge$-equivalence Class of $A$ (note all
$C$ are also $\ell$-labelled by the definition of $\lozenge$), the Set
of Transition Labels from the Source and Target State of $A$ into the
Source and Target State of each $C$, resp., are \emph{equal} Sets, and
the same is true for the Set of Transitions Labels from the Source and
Target State of $B$ into the Source and Target State of each $C$,
resp., are also \emph{equal} Sets

TODO: if one pair of such Transitions are $\underline{\blacklozenge}$,
does that imply that all pairs in the $\lozenge$-equivalence Class are
\emph{also} $\underline{\blacklozenge}$ ?

%FIXME what does this relation allow for ???


\textbf{Synchronous Transition Systems}

\emph{Synchronous Transition Systems} (see below) are used to identify
Communicating Systems which can be transformed into Global Graphs. A
Synchronous Transition System has:
\begin{itemize}
\item \emph{Nodes} each consisting of a Vector of Local CFSM States
\item \emph{Transitions} (Edges) Labelled by Elements in the \emph{Set
  of Events} (taken up to $\blacklozenge$-equivalence):
\[
  \class{E} \defeq \bigcup_{\party{s,r\in\class{P}}} Q_\party{s} \times
    Q_\party{r} \times \{(\party{s,r})\} \times \Sigma \}
\]
where Events are Tuples
$(q_\party{s},q_\party{r},\party{s},\party{r},a) \in \class{E}$, also
written:
\[
  (q_\party{s},q_\party{r},\party{s}\rightarrow\party{r}:a)
\]
indicating that Participants $\party{s}$ and $\party{r}$ can exchange
a Message $a$ when they are in States $q_\party{s}$ and $q_\party{r}$,
respectively
\end{itemize}

The \emph{indexing} of Events by Local (CFSM) States (i.e.
$q_\party{s}, q_\party{r}$) distinguishes Communication of the same
Message at different points in the Global Graph.


\textbf{Event Equivalence}

Def. \emph{$\class{E}$-equivalence} (\emph{Event Equivalence}):
\[
  \bowtie \defeq \bowtie_\party{s} \cap \bowtie_\party{r}
    \subseteq \class{E} \times \class{E}
\]
where:
\begin{flalign*}
& \quad (q_1,q_2, \party{s}\rightarrow\party{r} : a) \bowtie_\party{s}
    (q_1',q_2', \party{s}\rightarrow\party{r} : a)
  \Longleftrightarrow \\
& \quad\quad\quad \forall(q_1,\party{sr}!a,q_3),(q_1',\party{sr}!a,q_3')
      \in \delta_\party{s}
    . (q_1,\party{sr}!a,q_3)\blacklozenge(q_1',\party{sr}!a,q_3') \\
& \quad (q_1,q_2, \party{s}\rightarrow\party{r} : a) \bowtie_\party{r}
  (q_1',q_2', \party{s}\rightarrow\party{r} : a)
  \Longleftrightarrow \\
& \quad\quad\quad \forall(q_2,\party{sr}?a,q_4),(q_2',\party{sr}?a,q_4')
      \in \delta_\party{s}
    . (q_2,\party{sr}?a,q_4)\blacklozenge(q_2',\party{sr}?a,q_4')
\end{flalign*}
is an Equivalence Relation over Events that identifies Events with
underlying Local Transitions that are $\blacklozenge$-equivalent

$[e]$ -- Equivalence Class of Event $e$


\emph{Concurrent Interactions}

Squares in the Synchronous Transition System with Equivalent
Sender/Receiver/Message Events (but not the same Local States) on
opposite parallel sides (both ``horizontal'' and ``vertical'')
identify pairs of \emph{Concurrent Interactions}.


\textbf{Synchronous Transitions}

A \emph{Synchronous Transition System} is the Labelled Transition
System generated by the possible interleavings of \emph{Synchronous
  Transitions} (Send/Receive pairs) of the Local CFSMs.

$\vec{n}, \vec{n}', \ldots$ -- Vectors of Local States \\
$\vec{n}[\party{p}]$ -- the State of $\party{p} \in \class{P}$ in
$\vec{n}$ \\
$\hat{\delta} \defeq \{ (\vec{n},e,\vec{n}')
\ |\ (\vec{n};\vec{\varepsilon})
    \xrightarrow{\party{sr}!a} \xrightarrow{\party{sr}?a}
     (\vec{n}';\vec{\varepsilon}) \wedge
    e = (\vec{n}[\party{s}],\vec{n}[\party{r}],
      \party{s}\rightarrow\party{r}:a)
\}$ -- Set of Synchronous Transitions \\
%TODO is this the correct description?

$STS(\class{S})$ is the \emph{Synchronous Transition System} of
Communicating System $\class{S} =
(M_\party{p})_{\party{p}\in\class{P}}$ is a $4$-tuple:
\[
  STS(\class{S}) = (N,\vec{n}_0,E/\bowtie,\rightrightarrows)
\]
$N \defeq \{ \vec{q} \ |\ (\vec{q};\vec{\varepsilon}) \in
Rc_1(\class{S}) \}$ %FIXME explain this definition

$\vec{n}_0 = \vec{q}_0$ -- Initial State

$E \defeq \{ e \ |\ \exists\vec{n},\vec{n}' \in N :
  (\vec{n},e,\vec{n}') \in \hat{\delta} \} \subseteq \class{E}$
-- Set of Synchronous Events %TODO is this the correct description?

$\vec{n} \stackrel{[e]}{\rightrightarrows} \vec{n}'
  \Longleftrightarrow (\vec{n},e,\vec{n}') \in \hat{\delta}$ --
Synchronous Transition Relation %TODO is this the correct description?

note: original paper uses a representative Set $\hat{E}$ of Events for
each $\bowtie$-equivalence Class but here we will just continue
writing $[e]$

in $\hat{\delta}$, Events are $\bowtie$-equivalent if they have the
same Interaction $\party{s}\rightarrow\party{r} : a$
%TODO explain

$\pi$ -- ranges over Sequences of Events and $\rightrightarrows$ is
extended to Sequences of Events (as with the Reachibility Relation
$\rightarrow$ for Configurations above)


\emph{Projections}

$e\downharpoonright_\party{p}$ -- Projection of Event $e$ onto
Participant $\party{p}$:
\[
  (q_\party{s},q_\party{r},\party{s}\rightarrow\party{r}:a)
    \downharpoonright_\party{p}
  \defeq \begin{cases}
    \party{pr}!a &\ \text{if}\ \party{s = p} \\
    \party{sp}?a &\ \text{if}\ \party{r = p} \\
    \varepsilon  &\ \text{otherwise} \\
  \end{cases}
\]

$STS(\class{S})\downharpoonright_\party{p}$ -- Projection of
Synchronous Transition System onto Participant $\party{p}$ is the
Automaton $(Q, \vec{q}_0, \Sigma, \delta)$ where $Q = N$, $\vec{q}_0 =
\vec{n}_0$ and $\delta \subseteq Q \times A \cup \{\varepsilon\}
\times Q$ is such that $(\vec{n}_1, e\downharpoonright_\party{p},
\vec{n}_2) \in \delta \Leftrightarrow \vec{n}_1
\stackrel{e}{\rightrightarrows} \vec{n}_2$


\textbf{Generalized Multiparty Compatibility} (GMC)

a Sound and Complete condition for constructing Global Graphs

Def. (\emph{Generalized Multiparty Compatibility}) A Communicating
System $\class{S}$ is \emph{Generalized Multiparty Compatible (GMC)}
if it is \emph{Representable} and has the \emph{Branching Property}

\emph{Representability} -- for each CFSM, each Trace and each Choice
are ``\emph{represented}'' in $STS(\class{S})$; guarantees that
$STS(\class{S})$ contains enough information to decide \emph{Safety
  Properties} of any Asynchronous Execution of $\class{S}$

\emph{Branching Property} -- for each Choice in $STS(\class{S})$ a
unique CFSM makes the decision and each other Participant is either
notified of the Choice or else is not involved in the Choice; ensures
that if a ``\emph{branching}'' in $STS(\class{S})$ represents a
\emph{Choice}, then the Choice is ``Well-formed''


$\class{L}$ -- Language

$hd(\class{L})$ -- \emph{First Actions} of $\class{L}$:
\begin{align*}
  hd(\class{L}) \defeq& \{ \ell \ |\ \exists\varphi \in A^*
    : \ell\cdot\varphi \in \class{L} \} & \\
  hd(\{\varepsilon\}) \defeq& \{ \varepsilon \} &
\end{align*}

$STS(\class{S})<\vec{n}>$ -- Synchronous Transition System
$STS(\class{S})$ where Initial State $\vec{n}_0$ is replaced by
$\vec{n}$

$LTS(\class{S},\vec{n},\party{p})$ -- the Language
$\class{L}(STS(\class{S})<\vec{n}>\downharpoonright_p)$ obtained by
setting the Initial State (Node) of $STS(\class{S})$ to $\vec{n}$ and
then Projecting the resulting Synchronous Transition System onto
Participant $\party{p}$


\emph{Representability}

Def. (\emph{Representability}) Communicating System $\class{S}$ is
\emph{Representable} if for all $\party{p} \in \class{P}$:
\begin{enumerate}
  \item $\class{L}(M_\party{p}) = LTS(\class{S},\vec{n}_0,\party{p})$ and
  \item $\forall q \in Q_\party{p} \exists \vec{n} \in N :
    \vec{n}[\party{p}] = q \wedge
      \bigcup_{(q,\ell,q')\in\delta_\party{p}} \{ \ell \}
        \subseteq hd(LTS(\class{S},\vec{n},\party{p}))$
\end{enumerate}
(1) guarantees that each Trace of each CFSM is represented in
$STS(\class{S})$ and (2) is necessary to ensure that every Choice in
each CFSM is represented in $STS(\class{S})$

checking the Representability Condition for a Communicating System
$\class{S}$ has Exponential worst-case Time Complexity


\emph{Branching Property}

(auxiliary definitions for specifying Branching Property follow)

for $\vec{n},\vec{n}' \in N$, $\vec{n} \prec \vec{n}'$ if and only if:
\begin{enumerate}
  \item $\vec{n}\rightrightarrows^*\vec{n}'$ and
  \item for all Paths
    $\vec{n}_0\rightrightarrows\cdots\rightrightarrows\vec{n}_k =
    \vec{n}$ in $STS(\class{S})$ such that
    $\vec{n}_0,\ldots,\vec{n}_k$ are pairwise distinct,
    $\vec{n}'\neq\vec{n}_h$ for all $0 \leq h \leq k$
\end{enumerate}
i.e. $\vec{n} \prec \vec{n}'$ holds if $\vec{n}'$ is Reachable from
$\vec{n}$ and no ``simple'' Path from $\vec{n}_0$ to $\vec{n}$ goes
through $\vec{n}'$; note $\prec$ is not a Preorder in general

$ln(\vec{n},[e_1],[e_2])$ -- the \emph{Last Nodes} Reachable from
$\vec{n} \in N$ with $[e_1] \neq [e_2] \in E/\bowtie$ and $i,j \in
\{1,2\}$:
\begin{align*}
  ln(\vec{n},[e_1],[e_2]) \defeq \{ (\vec{n_1},\vec{n_2}) \ |\ &
    \exists\vec{n}' \in N : \vec{n}\rightrightarrows^*\vec{n}'
        \stackrel{[e_i]}\rightrightarrows\vec{n}_i \\
    \wedge & \forall\vec{n}'' \in N . \vec{n}'\rightrightarrows\vec{n}''
        \Longrightarrow
          \neg(\vec{n}'\prec\vec{n}''\stackrel{[e_i]}\rightrightarrows
  \}
\end{align*}
i.e. if $(\vec{n}_1,\vec{n}_2) \in ln(\vec{n},[e_1],[e_2])$ then
$\vec{n}_i$ is an $\stackrel{[e_i]}\rightrightarrows$-successor of a
Node $\vec{n}'$ on a Path from $\vec{n}$ whose Successors are either
not able to fire both $[e_1]$ and $[e_2]$ or else not $\prec$-related
to $\vec{n}'$

TODO why are ``last nodes'' here defined for a pair of events, instead
of a single ``last node'' for a single event-- branching ???

for Event $e =
(q_\party{s},q_\party{r},\party{s}\rightarrow\party{r}:a) \in
\class{E}$, let $\iota(e) = \party{s}\rightarrow\party{r}:a$ and
define a Dependency Relation $\lhd \subseteq \class{E}\times\class{E}$
on Events:
\[
  e \lhd e' \Leftrightarrow
    \iota(e) = \party{s}\rightarrow\party{r}:a \wedge
    (\iota(e') = \party{s}\rightarrow\party{r}:a' \vee
      \iota(e') = \party{r}\rightarrow\party{t}:a')
\]
i.e. $e$ and $e'$ are $\lhd$-related if there is a Dependency Relation
(\S\ref{sec:dependency_relation}) between the two Interactions from
the point of view of the \emph{Receiver} %TODO explain further

the Relation $e \blacktriangleleft_\pi e'$ in a Sequence of Events
$\pi$ is defined when there is a $\lhd$-relation between $e$ and $e'$
``in'' $\pi$:
\[
  e \blacktriangleleft_\pi e' \Leftrightarrow \begin{cases}
    (e \lhd e'' \wedge e'' \blacktriangleleft_{\pi'} e') \vee
      e \blacktriangleleft_{\pi'} e' & \text{if }\pi = e''\cdot\pi' \\
    e \lhd e' & \text{otherwise}
  \end{cases}
\]
%TODO explain

$dep(\iota(e),\pi,\iota(e'))$ if and only if:
\[
  (\pi = \pi_1\cdot{e}\cdot\pi_2\cdot{e'}\cdot\pi' \wedge
    (-,-,\iota(e)) \notin \pi_1 \wedge (-,-,\iota(e'))\notin \pi_2)
  \Longrightarrow e \blacktriangleleft_{\pi_2}
\]
checks whether there is a Dependency between two Interactions on a
Path $\pi$ (if these Interactions appear in $\pi$)
%TODO explain

Def. (\emph{Branching Property}) A Communicating System $\class{S}$
has the \emph{Branching Property} if for all $\vec{n} \in N$ and for
all $[e_1] \neq [e_2] \in E/\bowtie$ such that $\vec{n}
\stackrel{[e_1]}\rightrightarrows \vec{n}_1$ and $\vec{n}
\stackrel{[e_2]}\rightrightarrows \vec{n}_2$, then either:
\begin{enumerate}
  \item $\exists\vec{n}' \in N :
    \vec{n}_1\stackrel{[e_2]}\rightrightarrows\vec{n}' \wedge
    \vec{n}_2\stackrel{[e_1]}\rightrightarrows\vec{n}'$ \\ or
  \item for each $(\vec{n}_1',\vec{n}_2') \in ln(\vec{n},[e_1],[e_2])$,
    the following conditions hold:
    \begin{enumerate}
      \item \emph{Choice-awareness} -- $\forall\party{p}\in\class{P}$,
        either:
        \begin{itemize}
          \item $L_\party{p}^1 \cap L_\party{p}^2 \subseteq
            \{\varepsilon\}$ and $\varepsilon \in L_\party{p}^1
              \Longleftrightarrow \varepsilon \in L_\party{p}^2$ \\
            or
          \item $\exists\vec{n}'\in{N},\pi_1,\pi_2 :
            \vec{n}_1'\stackrel{\pi_1}\rightrightarrows\vec{n}' \wedge
            \vec{n}_2'\stackrel{\pi_2}\rightrightarrows\vec{n}' \wedge
            (e_1\cdot\pi_1)\downharpoonright_\party{p} =
            (e_2\cdot\pi_2)\downharpoonright_\party{p} = \varepsilon$
        \end{itemize}
      \item \emph{Unique Selector} -- $\exists!\party{s}\in\class{P} :
        L_\party{s}^1 \cap L_\party{s}^2 = \varnothing \wedge
        \exists\party{sr}!a \in L_\party{s}^1 \cup L_\party{s}^2$
      \item \emph{No Race} -- $\forall\party{r}\in\class{P} .
        L_\party{r}^1 \cap L_\party{r}^2 = \varnothing \\
        \Longrightarrow
          \forall\party{s}_1\party{r}?a_1\in{L_\party{r}^1},
          \forall\party{s}_2\party{r}?a_2\in{L_\party{r}^2} .
          \vec{n}'_i\stackrel{\pi_i}\rightrightarrows \\
        \Longrightarrow
          dep(\party{s}_i\rightarrow\party{r}:a_i,e_i\cdot\pi_i,
            \party{s}_j\rightarrow\party{r}:a_j)$ \\
    \end{enumerate}
    where:
    \[
      L_\party{p}^i \defeq
        hd(\{ e_i\downharpoonright_\party{p}\cdot\varphi
        \ |\ \varphi \in LTS(\class{S},\vec{n}_i',\party{p}) \})
    \]
    for $\party{p} \in \class{P}$ and $i,j \in \{1,2\}$
\end{enumerate}

checking the Branching Property is of Factorial Computation Complexity
in the size of $STS(\class{S})$


\textbf{Soundness Theorem}

Thm. (Soundness) \emph{If $\class{S}$ is GMC then it is
  \emph{Safe} (no Orphan Message, Deadlock, or Unspecified Reception
  Configurations).}

Soundness Theorem states that no (Asynchronous) execution of
$\class{S}$ will result in an Orphan Message, Deadlock, or Unspecified
Reception Configuration

with Representability (every Transition and Branching in each CFSM is
``represented'' in $STS(\class{S})$), Proof shows that for each
Branching Node $\vec{n}$, the Function $ln(\vec{n},[e_1],[e_2])$
allows enough Branches to be verified against the Branching Property
and then shows that any Sent Message is eventually Received and that a
CFSM in a Receiving State \emph{eventually} Receives a Message it
expected (by the Branching Property)


\textbf{Amending non-GMC Communicating Systems}

(Prop.) \emph{If Communicating System $\class{S}$ satisfies all but
condition (1) of the Representability Property, then the System
consisting of the (Minimized) Projections of $STS(\class{S})$ is GMC}

if the Projections of $STS(\class{S})$ do not have viable alternatives
then Language Equivalence check can be used to highlight which
Transitions (or Paths) of each CFSM are \emph{not} represented in
$STS(\class{S})$, and Local States and Transitions violating it can be
identified according to condition (2) of the Representability Property
%TODO explain

when the Branching Property is violated, one can give a Vector of
Local States and the two Branching Events for which a violation
occurs, as well as a witnessing Execution that leads to the offending
Configuration

suggestions for violating conditions:
\begin{itemize}
\item (2a) \emph{Choice-awareness} -- list the CFSMs for which the
  condition is not satisfied and:
  \begin{itemize}
    \item if the CFSM has a first same Receiving Action in both
      Branches, then it may be corrected by renaming some Messages
    \item if the CFSM Terminates in one Branch but not the other, then
      suggest to add a new Label and a Transition to the Final State
      in the Terminated Branch as well as a Dual Transition in a
      Sending Machine
  \end{itemize}
\item (2b) \emph{Unique Selector} -- highlight the Set of CFSMs
  Sending Messages at the Branching Node and suggest the
  identification of a ``genuine selecting Machine'' and add
  Communication from this CFSM to the others
\item (2c) \emph{No Race} -- highlight for each violating CFSM the
  Messages on which a Race Condition may occur and suggest adding an
  acknowledgement Message between the two corresponding Actions
\end{itemize}


\textbf{Global Graph Construction Algorithm}

\fist Petri Nets (\S\ref{sec:petri_net})

Communicating System $\longrightarrow$ Synchronous Transition System
  $\longrightarrow$ GMC check \\
\-\ \-\ \-\ $\longrightarrow$ Petri Net
  $\longrightarrow$ One-source Net $\longrightarrow$ Joined Net
  $\longrightarrow$ Pre-global Graph \\
\-\ \-\ \-\
  $\longrightarrow$ Global Graph

\begin{align*}
  & \class{S} = (M_\party{p})_{\party{p}\in\class{P}}
    \longrightarrow STS(\class{S})
    \longrightarrow assert(GMC(STS(\class{S}))) \\
  & \longrightarrow \mathbf{N}_1 \longrightarrow \mathbf{N}_2
    \longrightarrow \mathbf{N}_3 \longrightarrow G_{pre}
    \longrightarrow G
\end{align*}

\begin{enumerate}
  \item Petri Net $\mathbf{N}_1$ -- Safe and Extended Free-choice
    Petri Net with a Reachability Graph that is Bisimilar to the
    original Transition System; construction from
    [Cortadella-Kishinevsky-Lavagno-Yakovlev98]: transforms Events of
    $STS(\class{S})$ into Transitions of $\mathbf{N}_1$ with Places
    built out of \emph{Regions} (Sets of States having uniform
    Behavior with regard to Events)
  \item One-source Net $\mathbf{N}_2$ -- transform Petri Net
    $\mathbf{N}_1$ so that its ``Initial Markings'' consist of exactly
    ``one place'' %FIXME explain
  \item Joined Net $\mathbf{N}_3$ -- Join Transitions wherever
    possible to make explicit Join and Fork points of the work-flow
  \item Pre-global Graph -- create a Vertex in the Global Grpah for
    each Place, Transition, and Element of the Flow Relation in the
    Joined Net $\mathbf{N}_3$, then Vertices are connected via
    ``Gates''
  \item Global Graph -- ``clean up'' Pre-global Graph of unnecessary
    Vertices
\end{enumerate}

(Prop) \emph{(2)-(4) is Computable in Polynomial Time on the size of
  $\mathbf{N}_1$.}

$\approx$ -- Weak Bisimilarity Relation on Reachability Graphs

Lemma (\emph{Weak Bisimilarity}) The Reachability Graphs $T_1$, $T_2$,
$T_3$ of $\mathbf{N}_1$, $\mathbf{N}_2$, $\mathbf{N}_3$, respectively,
are Weakly Bisimilar: $T_1 \approx T_2 \approx T_3$


\textbf{Global Graph Definition}

Def. (\emph{Global Graph}) A \emph{Global Graph} $G$ over Participants
$\class{P}$ and Messages (Alphabet) $\Sigma$ is a \emph{Labelled
  Graph} (\S\ref{sec:labelled_graph}):
\[
  G = \langle{V,D,\Lambda}\rangle
\]
with:

$V$ -- Vertices \\
$D\subseteq{V \times V}$ -- Edges \\
$\Lambda : V \rightarrow
  \{ \ocircle, \ocirc, \diamondplus, \boxvert \} \cup
  \{ \party{s}\rightarrow\party{r}:a
    \ |\ \party{s,r}\in\class{P} \wedge a \in \Sigma \}$
-- Labelling Function mapping Vertices to \emph{Interactions}
$\party{s}\rightarrow\party{r}:a$ and \emph{Gates} $\{ \ocircle,
\ocirc, \diamondplus, \boxvert \}$

$\party{s}\rightarrow\party{r}:a$ -- an Interaction where $\party{s}$
Sends a Message $a$ to $\party{r}$

$\ocircle$ -- Initial (Source) Vertex Gate: Source of a Global Graph

$\ocirc$ -- Terminal Vertex Gate: Termination of a Branch or Thread

$\boxvert$ -- Fork or Join Threads Gate

$\diamondplus$ -- Branch or Merge points, or entry points of Loops
Gate

$\Lambda^{-1}(\ocircle)$ is a Singleton

if $\Lambda(v) = \ocirc$ then $v$ has no outgoing Edges

if $\Lambda(v) \in \{\diamondplus, \boxvert\}$ then $v$ has at least
one incoming and one outgoing Edge

if $\Lambda(v) = \party{s}\rightarrow\party{r}:a$ then $v$ has
\emph{unique} incoming and \emph{unique} outgoing Edges


\textbf{Transformation from $\mathbf{N}_3$ to $G_{pre}$ to $G$}

$G_{pre}$ is obtained from $\mathbf{N}_3$ by:
\begin{enumerate}
  \item create a Vertex in the Global Graph for each Place,
    Transition, and Element of the Flow Relation in $\mathbf{N}_3$
  \item connect above Vertices via Gates:
    \begin{itemize}
      \item connect an Initial (Source) Vertex $\ocircle$ to the
        Vertex without a Predecessor
      \item connect Terminal (Sink) Vertex $\ocirc$ to any Vertex
        without Successors
      \item connect $\boxvert$-gates to Transitions if they have more
        than one Predecessor or Successor
      \item connect $\diamondplus$-gates Places if they have more than
        one Predecessor or Successor
    \end{itemize}
  \item connect each ``component'' of the Graph by merging ``Ports''
    corresponding to Elements in the Flow Relation %TODO clarify
\end{enumerate}

the Global Graph $G$ is obtained from $G_{pre}$ by removing all
``unnecessary'' Nodes (i.e. former Places and Transitions) and
Re-labelling Events into \emph{Interactions}-- Event $e$ is replaced
by $\iota(e)$ (i.e. the Local States are ``forgotten'')


\textbf{Global Graph Projection}

Global Graph $G$ can be Projected to Local CFSMs by either:
\begin{enumerate}[label=(\alph*)]
  \item transform $G$ into a Petri Net and Project the Reachability
    Graph of the Petri Net (cf. Projection of $LTS(\class{S})$)
  \\ or:
  \item \begin{itemize}
    \item transform $G$ into an Automaton with States as Nodes of $G$
      and each Transition Labelled by
      $(\party{s}\rightarrow\party{r}:a)\downharpoonright_\party{p}$
      if the Source State corresponds to a Vertex with Label
      $\party{s}\rightarrow\party{r}:a$ and by $\varepsilon$ otherwise
    \item take the Parallel Composition of the Automata resulting from
      the Projectin of each Successor of a $\boxvert$-gate
      %TODO explain
    \item minimize the resulting Automata with respect to Language
      Equivalence
  \end{itemize}
\end{enumerate}

$G\downharpoonright_\party{p}$ -- Projection of Global Graph $G$ onto
Participant $\party{p}\in\class{P}$ (Formal definition given in the
extended version of Lange-Tuosto-Yoshida15)


\textbf{Completeness Theorem}

a Synchronous Transition System $STS(\class{S}) =
(N,\vec{n}_0,E/\bowtie,\rightrightarrows)$ is \emph{Self-loop Free} if
$\forall\vec{n},\vec{n}'\in{N} . \vec{n}\rightrightarrows\vec{n}'
\Longrightarrow \vec{n}\neq\vec{n}'$

Thm. (Completeness) \emph{For GMC Communicating System $\class{S} =
  (M_\party{p})_\party{p\in\class{P}}$ and Global Graph $G$ built from
  $\class{S}$ with $STS(\class{S}) =
  (N,\vec{n}_0,E/\bowtie,\rightrightarrows)$, if $STS(\class{S})$ is
  Self-loop Free then $\class{S}$ is Isomorphic to the Projected
  Communicating System
  $(G\downharpoonright_\party{p})_{p\in\party{P}}$}

the Completeness Theorem relies on the fact that each CFSM is
\emph{preserved} during the construction:
\begin{enumerate}
  \item Projection of $STS(\class{S})$ onto each $\party{p}$ is
    Language Equivalent with $M_\party{p}$
  \item the Petri Net $\mathbf{N}_1$ obtained from $STS(\class{S})$ is
    Bisimilar to $STS(\class{S})$
  \item each transformation preserves (Weak) Bisimilarity with the
    Derived Petri Net (cf. Weak Bisimilarity Lemma)
  \item transformation to a Global Graph is \emph{Sound} since the
    Petri Net is ``Extended Free Choice'' %TODO clarify
\end{enumerate}


\textbf{Haskell Prototype Tool}

takes as Input a textual representation of a Communicating System
$\class{S}$, builds $STS(\class{S})$, checks Representability
Condition and Branching Property--uses HKC (Bonchi-Pous13) to check
for Language Equivalence--and constructs Petri Net from
$STS(\class{S})$ using Petrify tool:
\url{http://www.lsi.upc.edu/~jordicf/petrify/}


\endgroup


\asterism


Ng-Yoshida16 \emph{Static Deadlock Detection for Concurrent Go by
  Global Session Graph Synthesis}:

approach uses Sets of CFSMs to represent (Communicating) Systems of
Concurrent Processes

uses Haskell GMC tool from Lange-Tuosto-Yoshida15

introduces Dingo static analysis tool: analyzes Go source code to
derive a Set of Local Session Types by abstracting Concurrent
Interaction Patterns amongst Goroutines (Green Threads)

Synchronous Communication Model (Unbuffered ``Rendezvous'' Channels,
i.e. Channels with $0$-sized Buffers)

a \emph{Program} is a Closed Communicating System called a
\emph{Session}

each Local Session Type represents a single \emph{Participant} of the
Session, i.e. corresponds to a single instance of a Goroutine
($\mono{main}$ is an implicit Goroutine)

extracts Communication Operations in source code as a Local
Session Type-- a Control-flow Graph with Session Primitives where
each Node is one of:
\begin{itemize}
  \item $Channel\ ch\ T$ -- Create a new Channel named $ch$ of Type
    $T$
  \item $Send\ ch$ -- Send to a Channel $ch$:
    $ch!\langle{T}\rangle; T'$ where $T'$ is the Continuation
    (Child Node)
  \item $Recv\ ch$ -- Receive from a Channel $ch$:
    $ch?\langle{T}\rangle; T'$ where $T'$ is the Continuation
    (Child Node)
  \item $Close\ ch$ -- Close a Channel $ch$
  \item $Label$ -- a named Jump Label %TODO explain
\end{itemize}

Static Single Assignment (SSA) Intermediate Representation --
constructed by $\mono{go/ssa}$ package in the Go Standard Library:
simplifies Syntax of Go program into a limited Set of Instructions and
flattens the Control Flow of the Program as Jumps between Blocks of
Instructions for analysis

\emph{Communication Instructions} and \emph{Control Flow Instructions}
are sufficient to infer Local Types from Go Programs; \emph{Memory
  Access Instructions} do not affect Control Flow, nor perform any
Communication with other Goroutines, but they will be required to
track where Channels have been stored (e.g. in other Data Structures)

\begin{enumerate}
  \item analysis starts from Program Main Entry Point ($\mono{main}$)
    and interprets SSA instructions following the Program Control
    Flow:
    \begin{itemize}
      \item instructions related to Communications are converted to
        Nodes of the Local Session Type Graph
      \item instructions related to Control Flow are converted to
        Edges of the Local Session Type Graph
    \end{itemize}
  \item a Set of Local Session Type Graphs is generated to represent
    the Set of Goroutines in the Program
\end{enumerate}

SSA Communication Instructions:
\begin{itemize}
  \item $\mathsf{MakeChan\{Size\}}$ -- $Channel\ ch\ T$ \\
    -- creates a Channel with $\mathsf{Size}$-sized Buffer, in
    Ng-Yoshida16 this is always $\mathsf{Size = 0}$
  \item $\mathsf{Send\{Chan=ch, X\}}$ -- $Send\ ch$ \\
    -- Send a Value $\mathsf{X}$ to a Channel $\mathsf{Chan})$
  \item $\mathsf{UnOp\{Op=ARROW, X=ch\}}$ -- $Recv\ ch$ \\
    -- Receive from a Channel $X$ %TODO returns received?
  \item $\mathsf{Select\{States, Blocking\}}$ -- the Local Type of
    each Case in a $\mono{select}$ is appended as a Child to the
    Parent Local Graph Node of $\mathsf{Select}$ \\
    -- Non-deterministic Choice on Channel Communication: each Case is
    guarded by a Communication Operation and a Case is \emph{chosen}
    if it does not Block; $\mathsf{Blocking}$ is a Boolean indicating
    whether a $\mono{default}$ Case exists (chosen if all the
    Communication Cases are Blocking), which is given an $EmptyLabel$
    to denote no Communication Operations
  \item $\mathsf{Builtin\{Name=close, Arg=[ch]\}}$ -- $Close\ ch$ \\
    -- use of built-in Function $\mathsf{close}$ that Closes the given
    Channel
\end{itemize}
(??? Cases are tried in order FIXME)

Function Bodies in SSA IR are Segmented into \emph{Blocks} with
\emph{Jumps} between Blocks following the Control Flow of the Program

SSA Control Flow Instructions:
\begin{itemize}
  \item $\mathsf{Call\{Func, Method, Args\}}$ -- Edge is added in
    Local Type Graph from Caller to the Subgraph representing the
    Local Types of the Callee Body \\
    -- dual of $\mathsf{Return}$: Control Flow leaves the Caller and
    enters called Function
  \item $\mathsf{Go\{Func, Method, Args\}}$ -- add a new Local Type
    Graph to Session \\
    -- Spawns a Goroutine with Initial Node pointing to the Subgraph
    representing the Local Types of the Callee Body
  \item $\mathsf{Jump}$ -- Edge in Local Type Graph from the Local
    Types of the current Block to the Local Types of the next Block \\
    -- unconditional Jump from the current Block to a Successor Block:
    Sequential Transition from the end of the current Block to the
    next Block in the same Scope
  \item $\mathsf{Return}$ -- Edge in Local Type Graph joining the last
    Node of a Function to the Continuation of the Caller \\
    -- dual of $\mathsf{Call}$: Control Flow continues at the Caller
    of the called Function
  \item $\mathsf{If\{Cond\}}$ -- two Edges pointing to the Local Types
    for the ``Then'' and ``Else'' Blocks as Child Subgraphs\\
    -- Conditional Jump based on the Boolean $\mathsf{Cond}$
  \item $\mathsf{Defer\{Func, Method, Args\}}$, $\mathsf{RunDefers}$
    -- Edge in Local Type Graph at end of Caller pointing to the Local
    Type Subgraph of the Deferred Function \\
    -- pair of Instructions that \emph{Defer} a Call to a Function to
    the end of the current Function; this pushes the Deferred Function
    onto a special Stack of Deferred Function Calls which are
    guaranteed to be run with $\mathsf{RunDefers}$ once in each
    Control Flow Path per Function
\end{itemize}


\textbf{Global Graph Synthesis}

uses Haskell GMC tool from Lange-Tuosto-Yoshida15: takes Set of
Communicating Finite State Machines (CFSMs) as input and produces a
Global Graph of Transitions


\emph{Goroutine CFSMs}

Models the Communication Behavior of Goroutines: represents Control
Flows of the Inferred Local Types

\begin{enumerate}
  \item covert Local Session Graphs obtained from Go source code
    (``Inferred Local Types'') into CFSMs -- each Local Session Graph
    is translated to a single Goroutine CFSM

    Nodes in the Local Session Graph are \emph{Events} represented as
    State Transitions in the CFSM Model with Labels determined by the
    Type of the Local Type Node:
    \begin{itemize}
      \item Node $Send\ ch$ becomes Transition with Label $ch!T$ where
        $T$ is the Type of Channel $ch$
      \item Node $Recv\ ch$ becomes Transition with Label $ch?T$ where
        $T$ is the Type of Channel $ch$
      \item Node $Close\ ch$ becomes Transition with Label
        $ch!\mathsf{STOP}$
    \end{itemize}
\end{enumerate}


\emph{Channel CFSMs}

Channels between CFSMs are \emph{fixed} between two Machines: a CFSM
can use multiple Channels, but the Endpoints are always the same

Go Channels are \emph{Shared Names} which can be used by multiple
Goroutines as a \emph{Shared Location} for two or more Goroutines to
\emph{Synchronize} on Send and Receive Operations

this is \emph{preserved} when translated from Local Types to CFSMs:
CFSMs Communicate with Channels which are \emph{Variables} and
\emph{not} ``Endpoint Machines''

Channels connect to CFSMs as ``\emph{proxies}'': CFSMs do not
Communicate directly but Go Channels resemble \emph{Switches} or
\emph{Multiplexors} between CFSMs

Go Channels are Modelled as CFSMs themselves--\emph{Channel CFSMs}--to
represent all possible transitions that a Go Channel is allowed (such
as connecting multiple Goroutines dynamically)

\begin{enumerate}
  \setcounter{enumi}{1}
  \item construct a Channel CFSM for each identified Channel where the
    first outgoing Transition from the Initial State is a Receive
    Action and the second is a Send Action matched to the Initial
    Action, returning to the Initial State
\end{enumerate}

\emph{Global Graph Synthesis}: given both Goroutine CFSMs and Channel
CFSMs, Synthesize Global Graph from the CFSMs by procedure involving
generating \emph{all possible combinations} of Synchronous Labelled
Transitions of the composed CFSMs

\emph{For-loops}: translated as two Branches of Transitions from a
State: one for \emph{Exit Loop} and another for \emph{Continue Loop}
that loops back to the starting State
(FIXME is this in the CFSM or Global Graph ???)

Limitations:

allowing for ``flexible range of Program Control Flow patterns'', e.g.
For-loops (above), results in limited support for \emph{Dynamic
  Concurrency}, e.g. creating Channels in a Loop or \emph{conditional}
creation of Goroutines for ``Error Handling'', corresponding to
\emph{Runtime Spawning} of new Channel CFSMs and new Goroutine CFSMs,
respectively

to overcome limitations, in the Inference technique outlined above:
\begin{itemize}
\item if a condition inthe Control Flow of a Program decides if a
  Goroutine will be Spawned, then a \emph{Subset} of all generated
  CFSMs -- ``which (are ???) translated from Goroutines that are
  Spawned under the same condition'' (???) -- are \emph{selected} for
  Global Graph Synthesis to check that the System is Safe under the
  condition
\end{itemize}
(FIXME clarify)


\textbf{Generalized Multiparty Compatibility (GMC)}

Global Graph Nodes are \emph{Interactions} (matched Send/Receive
Actions) between two Participants (CFSMs)

Generalized Multiparty Compatibility (GMC -- Lange-Tuosto-Yoshida15):
\begin{enumerate}[label=(\roman*)]
  \item \emph{Representability} -- each Trace and Choices in the CFSMs
    are represented in the Global Graph: ensures no information is
    lost in the construction of the Global Type
  \item \emph{Branching Property} -- for all Choices in the Global
    Graph, a \emph{unique} CFSM takes (FIXME makes ???) the decision
    and the decision is \emph{propagated} to ``other'' CFSMs (FIXME to
    *all* other CFSMs ???) (the \emph{Unique Machine Condition}):
    ensures that all Branches are ``Well-formed''
\end{enumerate}

Representability only applies to Goroutine CFSMs: Channel CFSMs do not
give any influence on the Communication Behavior of Goroutine CFSMs


\textbf{Mixed Choices}

(TODO examples, clarify)

original Theory in Lange-Tuosto-Yoshida15 does not allow a CFSM to
have a \emph{Mixed Choice}, i.e. a Choice which has both Sending and
Receiving Actions from the same State

Go $\mono{select}$ allows writing Mixed Choices

in Lange-Tuosto-Yoshida15 the Branching Property requires unique
Senders (i.e. a unique Participant initiates Sending Actions)-- with
the Synchronous Model in Ng-Yoshida16 (as opposed to an Asynchronous
Model in Lange-Tuosto-Yoshida15), this condition can be relaxed by
replacing the unique Sender by the \emph{Unique Machine Condition}
(see above)


\textbf{Case Studies}

\begin{itemize}
  \item \emph{Pipeline} -- only safe if the Main Loop is
    \emph{Bounded} (cannot be Statically Verified unless Loops are
    ``Unrolled'')
  \item \emph{Fan-in}
\end{itemize}



% ====================================================================
\section{Distributed System}\label{sec:distributed_system}
% ====================================================================

\emph{Location Transparency}

\fist Distributed Computation (\S\ref{sec:distributed_computation})



\subsection{Centralized System}\label{sec:centralized_system}

\subsection{Decentralized System}\label{sec:decentralized_system}



% ====================================================================
\section{Complex System}\label{sec:complex_system}
% ====================================================================

Network Theory (\S\ref{sec:network_theory})

Adaptive Systems (TODO)

2018 - Bar-Yam, Lynch, Bar-Yam - \emph{The Inherent Instability of Disordered
  Systems}



% --------------------------------------------------------------------
\subsection{Self-Organized Criticality (SOC)}\label{sec:soc}
% --------------------------------------------------------------------

1987 - Bak, Tang, Wiesenfeld - \emph{Self-organized Criticality: an explanation
  of $1/f$ noise}

(wiki):

Property of Dynamical Systems with a Critical Point (TODO) as an Attractor
(\S\ref{sec:attractor_repeller}); ``macroscopic behavior'' displays Spatial
and/or Temporal Scale-Invariance (\S\ref{sec:scale_invariance})

\fist Tweedie Hypothesis (Tweedie Distributions
\S\ref{sec:tweedie_distribution}): alternative paradigm to explain Power Law
manifestations attributed to SOC



\subsubsection{Abelian Sandpile Model}\label{sec:abelian_sandpile}

or \emph{Bak-Tang-Wiesenfeld Model}

Cellular Automaton (\S\ref{sec:cellular_automaton})

``Abelian'' because the final configuration is independent of the update order

Sandpile Group

System is attracted to its ``Critical State'' and once reached there is no
correlation between the System's response to a perturbation and the details of
the perturbation, i.e. adding another grain may cause nothing to happen or may
cause the entire pile to collapse

Least Action Principle



% --------------------------------------------------------------------
\subsection{Hierarchy Theory}\label{sec:hierarchy_theory}
% --------------------------------------------------------------------

Multi-scale Analysis (\S\ref{sec:multiscale_analysis})



% ====================================================================
\section{Reaction-diffusion System}\label{sec:reaction_diffusion}
% ====================================================================

% ====================================================================
\section{Stability Theory}\label{sec:stability_theory}
% ====================================================================

% --------------------------------------------------------------------
\subsection{Stability Radius}\label{sec:stability_radius}
% --------------------------------------------------------------------

\fist Info-gap Decision Theory (\S\ref{sec:info_gap}): application of
Sensitivity Analysis (\S\ref{sec:sensitivity_analysis}) of the Stability Radius
type to Perturbations in the value of a given Estimate of a Parameter of
interest (FIXME: clarify)



% --------------------------------------------------------------------
\subsection{Stability Critereon}\label{sec:stability_critereon}
% --------------------------------------------------------------------

\subsubsection{Root Locus Analysis}\label{sec:root_locus_analysis}

graphical method for examining how \emph{Roots} of a System change with
variation of a System Parameter, e.g. Gain within a Feedback System (TODO:
xrefs)

Laplace Transform (\S\ref{sec:laplace_transform})



% --------------------------------------------------------------------
\subsection{Lyapunov Function}\label{sec:lyapunov_function}
% --------------------------------------------------------------------

MAE5790 Lec.17

an ``energy-like'' function in a system with friction that decreases along
all trajectories



% ====================================================================
\section{Control Theory}\label{sec:control_theory}
% ====================================================================

Properties of possible Trajectories of Dynamical Systems
(\S\ref{sec:dynamical_system}) with certain Initial Conditions and
Driving Functions:

\emph{Observability}

\emph{Controllability} -- Property of Vector-valued Coupled
Differential Equations; \fist Cf. Information Theoretic \emph{Channel
  Capacity Theorem} (\S\ref{sec:channel_capacity})
%FIXME xref

Backpropagation ?


for a Linear Time-invariant System (\S\ref{sec:lti_system}), when taking the
Laplace Transform (\S\ref{sec:laplace_transform}) and if the Poles
(\S\ref{sec:complex_pole}) in the Complex Plane are:
\begin{itemize}
  \item in the Right Half Plane (positive Real component) then the System is
    \emph{Unstable}
  \item in the Left Half Plane (negative Real component) then it will be
    \emph{Stable}
  \item on the Imaginary Axis, it will have \emph{Marginal Stability}
\end{itemize}



% --------------------------------------------------------------------
\subsection{Signal Flow}\label{sec:signal_flow}
% --------------------------------------------------------------------

\emph{Signal Flow Graph} (\emph{SFG} or \emph{Mason Graph})

\fist Flow Graph (\S\ref{sec:flow_graph})

Nodes represent ``System Variables'' and Edges represent Functional
Connections between pairs of Nodes

%FIXME: move to information theory or control theory?

Semantics given by Corelation Categories
(\S\ref{sec:corelation_category})

(Props \S\ref{sec:prop_category} in Networks \S\ref{sec:network}) -- relation to
Circuit Diagrams given by Morphisms of Props (Baez,Coya,Rebro2018)

\emph{Input Signal}

\fist the Solution of a System of DAEs (\S\ref{sec:dae_system}) depends on
the Derivatives of the Input Signal and not just the ``Signal itself'' as in
the case of ODE Systems (\S\ref{sec:ode_system})
(FIXME: clarify)



% --------------------------------------------------------------------
\subsection{Controller}\label{sec:controller}
% --------------------------------------------------------------------

%FIXME

monitors and physically alters the operating conditions of a given
Dynamical System (\S\ref{sec:dynamical_system})



% --------------------------------------------------------------------
\subsection{Process Control}\label{sec:process_control}
% --------------------------------------------------------------------

\subsubsection{Model Predictive Control}\label{sec:model_predictive_control}



% --------------------------------------------------------------------
\subsection{Control System}\label{sec:control_system}
% --------------------------------------------------------------------

\subsubsection{Phase-Locked Loop (PLL)}\label{sec:pll}

Non-linear Dynamical System (\S\ref{sec:nonlinear_dynamical_system})



% --------------------------------------------------------------------
\subsection{Control Strategy}\label{sec:control_strategy}
% --------------------------------------------------------------------

TODO



\subsubsection{Optimal Control Theory}\label{sec:optimal_control}

\fist Calculus of Variations (\S\ref{sec:calculus_of_variations})



\paragraph{Hamilton-Jacobi-Bellman Equation}
\label{sec:hamilton_jacobi_bellman}\hfill

PDE (\S\ref{sec:pde})



\subparagraph{Hamilton-Jacobi Equation}\label{sec:hamilton_jacobi}\hfill

(wiki):

alternative formulation of Classical Mechanics equivalent to Lagrangian
(\S\ref{sec:lagrangian_system}) and Hamiltonian Mechanics
(\S\ref{sec:hamiltonian_system})

only formulation of Mechanics where the Motion of a Particle can be represented
as a \emph{Wave}; ``closest approach'' of Classical mechanics to Quantum
Mechanics; cf. Schr\"odinger Equation

\fist Calculus of Variations (\S\ref{sec:calculus_of_variations})

\emph{Hamilton's Principal Function}:
\[
  S(q,t) = \int^{(q,t)} \mathcal{L} dt
\]
where $\mathcal{L}$ is the Lagrangian of the System

the $S$ is equal to the \emph{Classical Action} (\S\ref{sec:trajectory_action})

Generalized Coordinates (\S\ref{sec:generalized_coordinate}) in Lagrangian
Mechanics (\S\ref{sec:lagrangian_system}) are related to Canonical Coordinates
(\S\ref{sec:canonical_coordinate}) of Hamiltonian Mechanics
(\S\ref{sec:hamiltonian_system}) by the Hamilton-Jacobi Equations



\paragraph{Linear Quadratic Regulator (LQR)}\label{sec:lqr}\hfill

dual of Linear Quadratic Estimation (LQE or \emph{Kalman Filtering}
\S\ref{sec:lqe})



% --------------------------------------------------------------------
\subsection{Behavioral Control Theory}\label{sec:behavioral_control}
% --------------------------------------------------------------------

2007 - Willems - \emph{The Behavioral Approach to Open and Interconnected
  Systems}

\fist Open Systems (\S\ref{sec:open_system})

\fist cf. Emergent Behavior in Open Games (\S\ref{sec:open_games})
\url{https://julesh.com/2017/04/22/on-compositionality/}



% ====================================================================
\section{Ergodic Theory}\label{sec:ergodic_theory}
% ====================================================================

\fist Measure-preserving Dynamical Systems
(\S\ref{sec:measure_preserving_system})

Ornstein Isomorphism Theorem, Bernoulli Flow

Fermi-Pasta-Ulam-Tsingou Problem



% --------------------------------------------------------------------
\subsection{Normal Number}\label{sec:normal_number}
% --------------------------------------------------------------------

% --------------------------------------------------------------------
\subsection{Mixing}\label{sec:mixing}
% --------------------------------------------------------------------

\emph{Strong Mixing}

\emph{Weak Mixing}



\subsubsection{Topological Mixing}\label{sec:topological_mixing}



% ====================================================================
\section{Chaos Theory}\label{sec:chaos_theory}
% ====================================================================

Deterministic Dynamical Systems (\S\ref{sec:deterministic_dynamical_system})

Non-linear Dynamical Systems (\S\ref{sec:nonlinear_dynamical_system})

Iterated Function (\S\ref{sec:iterated_function})

Fractal Geometry (Part \ref{sec:fractal_geometry})

relation between Chaotic Systems and Universal Computation (FIXME)

2014 - MAE5790 - Non-linear Dynamics and Chaos - Strogatz - "Theory of
Dynamical Systems" -
\url{https://www.youtube.com/playlist?list=PLj_l4pOO0YKhJHLQVbjpPlbNyYtT8aAQz}

Lec.16 - Lorenz Equations (\S\ref{sec:lorenz_system})

Logistic Equation (Population Biology)

Turbulence, Navier-Stokes

Lec.18 - \emph{Chaos}: Aperiodic, ``long-term'' (not transient) behavior in a
Deterministic (not Probabilistic) System that exhibits sensitivity to Initial
Conditions (Exponential Divergence of Trajectories; positive Lyapunov Exponent)

\emph{Attractor} (\S\ref{sec:attractor_repeller}) is a Set $A$ such that:
\begin{enumerate}
  \item $A$ is \emph{Invariant}: Trajectories beginning in $A$ remain in $A$
  \item $A$ ``attracts'' an Open Set of Initial Conditions
  \item $A$ is \emph{Minimal}: no Proper Subset of $A$ Satisfies both (1.) and
    (2.)
\end{enumerate}

\emph{Renormalization Group} (Conformal Geometry
\S\ref{sec:renormalization_group}); Phase Transitions in Statistical Physics
(Feigenbaum)

Complex Systems (\S\ref{sec:complex_system})

Network Theory (\S\ref{sec:network_theory})

Non-linear Deterministic Systems must 3rd-order to exhibit Chaos (Lorenz)

(wiki): Chaos as a Spontaneous Breakdown of Topological Supersymmetry (FIXME:
explain) which is an intrinsic property of \emph{Evolution Operators} of all
Stochastic and Deterministic (Partial) Differential Equations (\S\ref{sec:sde});
the long-range Dynamical Behavior associated with \emph{Chaotic Dynamics} is a
consequence of Goldstone's Theorem (\S\ref{sec:goldstones_theorem}) in the
application of Spontaneous Topological Supersymmetry Breaking

\asterism

%TODO: move the following ???

\emph{Many-body Quantum Chaos}

\url{https://video.ias.edu/PiTP/2018/0723-DouglasStanford}

\url{https://video.ias.edu/PiTP/2018/0724-DouglasStanford}



% --------------------------------------------------------------------
\subsection{R\"ossler System}\label{sec:rossler_system}
% --------------------------------------------------------------------

R\"ossler Attractor --
simplest Strange Attractor (\S\ref{sec:strange_attractor})



% --------------------------------------------------------------------
\subsection{Lorenz System}\label{sec:lorenz_system}
% --------------------------------------------------------------------

MAE5790 Lec. 16

Lorenz Equations

Lorenz 1963 - \emph{Deterministic Nonperiodic Flow} -- earliest example of a
System with a \emph{Chaotic Attractor}; cf. ``transient Chaos'' such as 3-body
Hamiltonian Systems

Conduction

Convection Cells

Rayleigh Number $r$

\begin{itemize}
  \item Equations are Symmetric under $(x,y) \mapsto (-x,-y)$
  \item System is ``\emph{Dissipative}'' in the sense that Volumes in Phase
    Space ``contract'' exponentially fast under the ``flow'', and all
    Trajectories eventually tend to a Limiting Set of zero Volume (Point, Cycle,
    or ``Strange Attractor'')
\end{itemize}

\emph{Fixed Points}

$(x,y,z) = (0,0,0)$ for all $r$; saddle point for $r > 1$, (globally) stable
node for $r < 1$ (all trajectories asymptotically approach the origin; proof
involves Lyapunov Functions \S\ref{sec:lyapunov_function})

$x = y = \pm\sqrt{b(r-1)}$ and $z = r-1$ where $r > 1$ -- stable "convection";
pitchfork bifurcation at $r=1$ where the fixed points merge at the origin;
linearly stable in range $1 < r < r_{Hopf}$, up to a Subcritical Bifurcation
(there is a ``jump'' to a different attractor)

Lorenz Attractor (Warwick Tucker)

Interactive Differential Equations:
\url{https://media.pearsoncmg.com/aw/ide/index.html}



% --------------------------------------------------------------------
\subsection{Synchronized Chaos}\label{sec:synchronized_chaos}
% --------------------------------------------------------------------

cf. Synchronization, Kuramoto Model

MAE5790, Lec.25



% ====================================================================
\section{Quantum System}\label{sec:quantum_system}
% ====================================================================

%FIXME: move section ???

a Quantum System that can exist in any Quantum Superposition
(\S\ref{sec:quantum_superposition}) of two independent (physically
distinguishable) Quantum States (\S\ref{sec:quantum_state}) is called a
\emph{Two-level} (\emph{Two-state}) \emph{System} and any Two-state System can
also be seen as a \emph{Qubit} (\S\ref{sec:qubit})

Quantum Information Theory (\S\ref{sec:quantum_information_theory})

2018 - Gong, Ashida, Kawabata, Takasan, Higashikawa, Ueda -
\emph{Topological Phases of Non-Hermitian Systems} -
\url{https://physics.aps.org/articles/v11/96} -- \emph{Non-Hermitian Quantum
  Mechanics}; usually Operators are assumed to be Hermitian (only reaturns Real
values, i.e. Observables); a Non-hermitian Operator can allow Complex values
useful in describing Open (Dissipative) Systems
(\S\ref{sec:dissipative_system}); \emph{Parity-time Symmetry}

(Video) - Witten - \emph{Introduction to Information Theory} -
\url{https://video.ias.edu/PiTP/2018/0716-EdwardWitten}

\asterism

\emph{Firewalls, AdS/CFT, and the Complexity of States and Unitaries: A Computer
  Science Perspective}
(video lectures)
-
\url{https://video.ias.edu/PiTP/2018/0719-ScottAaronson}

a Quantum Circuit $C$ consists of a Sequence of Unitary Transformations
(\S\ref{sec:unitary_transformation}), each of which acts on only a small number
of Qubits, so is a Tensor Product of a Unitary on those Qubits with the Identity
on the remaining Qubits that are not being acted on; the (Pure) State $\Psi$ of
the System at any point in Time is an Element of $(\comps^2)^{\otimes n}$, i.e.
a Unit Vector in a $2^n$-dimensional Vector Space and the size of the Circuit is
the number of Gates in it (a measure of ``resources'' used in the Circuit)



% --------------------------------------------------------------------
\subsection{Quantum State}\label{sec:quantum_state}
% --------------------------------------------------------------------

Pure State -- State Vector (Ket \S\ref{sec:state_vector})

Mixed State -- Density Matrix (\S\ref{sec:density_matrix})

Qubit (Two-level System \S\ref{sec:qubit}) -- Points on the Surface of the Bloch
Sphere (\S\ref{sec:bloch_sphere}) correspond to the Pure States of the System
and Interior Points correspond to Mixed States

(wiki): given an Orthonormal Basis, any Pure State $|\psi\rangle$ of a Two-level
Quantum System can be written as a \emph{Superposition}
(\S\ref{sec:superposition}) of the Basis Vectors $|0\rangle$ and $|1\rangle$,
where the Coefficient or ``amount'' of each Basis Vector is a Complex Number;
since only the \emph{relative Phase} between Coefficients of the two Basis
Vectors has any physical meaning, the Coefficient of $|0\rangle$ can be taken
to be a Non-negative Real (FIXME: clarify)

\asterism

(Video) - Witten - \emph{Introduction to Information Theory} -
\url{https://video.ias.edu/PiTP/2018/0716-EdwardWitten}

if the Universe is described by a Pure Quantum Mechanical State depending on all
possible Degrees of Freedom, a \emph{Subsystem} may still not be describable by
a Pure State

Schmidt Decomposition (TODO)

Projective Measurements (TODO)

Positive Operator-Valued Measurement (POVM)

\emph{Firewalls, AdS/CFT, and the Complexity of States and Unitaries: A Computer
  Science Perspective}
(video lectures)
-
\url{https://video.ias.edu/PiTP/2018/0719-ScottAaronson}

Circuit Complexity (\S\ref{sec:circuit_complexity}) of an $n$-qubit Quantum
State

the Circuit Complexity of a State is not an Observable



\subsubsection{State Vector}\label{sec:state_vector}

or \emph{Kets}

describe only \emph{Pure States}



\subsubsection{Density Matrix}\label{sec:density_matrix}

describes the Statistical State of a Quantum System, including Mixed States

the Quantum analog of a classical Probability Distribution
(\S\ref{sec:probability_distribution}) is a \emph{Density Matrix} --
representation of Density Operator (\S\ref{sec:density_operator}) with a choice
of Basis

Self-adjoint (Hermitian), Positive Semi-definite, Trace One, and may be
Infinite-dimensional

every Density Matrix arises from a ``Bipartite State'' and every Bipartite State
produces a Density Matrix

every Matrix with these properties can be ``Purified'', meaning that it is the
Density Matrix of \emph{some} Pure State on some ``Bipartite'' System $AB$

there is no ``classical analog'' for Purification, i.e. there is no way to make
Probability Distribution ``pure'' (one outcome with Probability $1$) by adding
more Variables

von Neumann Entropy (\S\ref{sec:vonneumann_entropy})

a System $A$ and a Purifying System $B$ always have the same Entropy

\emph{Reduced Density Matrix}, cf. Reduced Probability Distribution (TODO: xref)

Free Energy can only go down under a Quantum Channel that preserves Thermal
Equilibrium (Second Law of Thermodynamics)
--FIXME



\subsubsection{Qubit}\label{sec:qubit}

a Quantum System that can exist in any Quantum Superposition
(\S\ref{sec:quantum_superposition}) of two independent (physically
distinguishable) Quantum States is called a \emph{Two-level} (\emph{Two-state})
\emph{System} and any Two-state System can also be seen as a \emph{Qubit}

a Two-state System, resp. Qubit, has a Two-dimensional Hilbert Space
(\S\ref{sec:hilbert_space})

the Pure State Space of a Qubit has a geometric representation as the Bloch
Sphere (\S\ref{sec:bloch_sphere}), i.e. the Complex Projective Line
$\mathbb{CP}^1$, a unit $2$-sphere with Antipodal Points corresponding to a
point of mutually Orthogonal State Vectors and North and South Poles typically
chosen to correspond to the standard Basis Vectors $|0\rangle$ and $|1\rangle$

\fist Bell State (\S\ref{sec:bell_state}) -- specific States of two Qubits that
represent the simplest and maximal examples of Quantum Entanglement



% --------------------------------------------------------------------
\subsection{Quantum Superposition}\label{sec:quantum_superposition}
% --------------------------------------------------------------------

(wiki): given an Orthonormal Basis, any Pure State $|\psi\rangle$ of a Two-level
Quantum System (Qubit \S\ref{sec:qubit}) can be written as a
\emph{Superposition} of the Basis Vectors $|0\rangle$ and $|1\rangle$, where the
Coefficient or ``amount'' of each Basis Vector is a Complex Number; since only
the \emph{relative Phase} between Coefficients of the two Basis Vectors has any
physical meaning, the Coefficient of $|0\rangle$ can be taken to be a
Non-negative Real (FIXME: clarify) \fist see Bloch Sphere
(\S\ref{sec:bloch_sphere})



% --------------------------------------------------------------------
\subsection{Quantum Entanglement}\label{sec:quantum_entanglement}
% --------------------------------------------------------------------

\subsubsection{Bell State}\label{sec:bell_state}

specific States of two Qubits (\S\ref{sec:qubit}) that represent the simplest
and maximal examples of Quantum Entanglement

\emph{Firewalls, AdS/CFT, and the Complexity of States and Unitaries: A Computer
  Science Perspective}
(video lectures)
-
\url{https://video.ias.edu/PiTP/2018/0719-ScottAaronson}

Monogamy of Entanglement

Black Hole Complementarity

\emph{Firewall Paradox}

2013 - Harlow, Hayden - \emph{Quantum Computation vs. Firewalls}

\emph{Harlow-Hayden Decoding Problem} -- assume Quantum Polynomial Time
Church-Turing Thesis, from which it follows that all knowledge of the relevant
Laws of Physics could be encapsulated by specifying what Quantum Circuit takes
the Initial State of the in-falling Matter into a Black Hole and produces the
Quantum State of the Hawking Radiation and the Black Hole at a later Time; given
a Circuit $C|0 \cdots 0\rangle = |\psi\rangle_{RBH}$ that maps the all Zero
State to some Pure State $\psi$ that can be entangled between the three regions
$R$ (the Hawking Radiation, i.e. degrees of freedom which have already ``come
out'' of the Black Hole), $B$ (some Hawking Quantum which is just now emerging
from the Black Hole), $H$ (inaccessible degrees of freedom in the Black Hole);
\emph{task}: apply some Unitary Transformation $U_R$ to $R$ only that should
have the effect of putting $B$ and some designated Qubit of $R$ (say, the last
Qubit), into a Bell State, ``decoding'' or ``making manifest'' the Entanglement
between $R$ and $B$-- this is \emph{possible}, i.e. there is \emph{Maximal
  Entanglement} between $R$ and $B$; if the Observer were to \emph{fall into}
the Black Hole later, would give them an Observable violation of the
``monogamy'' of Entanglement, when combined with the Entanglement that is
supposed to be present between just inside and outside the Event Horizon because
of Quantum Field Theory, implying a ``Firewall'' or breakdown of Quantum
Mechanics



% --------------------------------------------------------------------
\subsection{Bloch Sphere}\label{sec:bloch_sphere}
% --------------------------------------------------------------------

(or \emph{Poincar\'e Sphere} in Optics, representing different types of
Polarizations)

geometrical representation of the Pure State of a Two-level Quantum System
(Qubit \S\ref{sec:qubit})

the Complex Projective Line (\S\ref{sec:complex_projective_space})
$\mathbb{CP}^1$, a unit $2$-sphere (\S\ref{sec:unit_sphere}) with Antipodal
Points corresponding to a point of mutually Orthogonal State Vectors and North
and South Poles typically chosen to correspond to the standard Basis Vectors
$|0\rangle$ and $|1\rangle$

Points on the Surface of the Sphere correspond to the Pure States of the System
and Interior Points correspond to Mixed States

may be generalized to $n$-level Quantum Systems

natural Metric is the Fubini-Study Metric

mapping from Unit $3$-sphere (\S\ref{sec:unit_glome}) in the Two-dimensional
State Space $\comps^2$ (FIXME: clarify) to the Bloch Sphere is the \emph{Hopf
  Fibration} (\S\ref{sec:hopf_fibration})
