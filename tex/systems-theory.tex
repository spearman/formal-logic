%%%%%%%%%%%%%%%%%%%%%%%%%%%%%%%%%%%%%%%%%%%%%%%%%%%%%%%%%%%%%%%%%%%%%%
%%%%%%%%%%%%%%%%%%%%%%%%%%%%%%%%%%%%%%%%%%%%%%%%%%%%%%%%%%%%%%%%%%%%%%
\part{Systems Theory}\label{sec:systems_theory}
%%%%%%%%%%%%%%%%%%%%%%%%%%%%%%%%%%%%%%%%%%%%%%%%%%%%%%%%%%%%%%%%%%%%%%
%%%%%%%%%%%%%%%%%%%%%%%%%%%%%%%%%%%%%%%%%%%%%%%%%%%%%%%%%%%%%%%%%%%%%%

%FIXME new tex file?

Operad Theory (\S\ref{sec:operad_theory}): Systems of Systems

Operad (\S\ref{sec:operad}): compositional style

Algebra (??? \S\ref{sec:universal_algebra}): System type



% ====================================================================
\section{State Variable}\label{sec:state_variable}
% ====================================================================

% ====================================================================
\section{Steady State}\label{sec:steady_state}
% ====================================================================

``Fixed-point'', ``Equilibria'' %FIXME



% ====================================================================
\section{Closed System}\label{sec:closed_system}
% ====================================================================

%FIXME

Actor Model (\S\ref{sec:actor_model}) -- \emph{Computational
  Representation Theorem} -- Consequence: a Finite Actor can
Nondeterministically Respond with an Uncountable number of different
Outputs %FIXME



% ====================================================================
\section{Open System}\label{sec:open_system}
% ====================================================================

(2016 - Fong - The Algebra of Open and Interconnected Systems):

\fist Hypergraph (\S\ref{sec:hypergraph}), Hypergraph Category
(\S\ref{sec:hypergraph_category})

Interconnection (``Integration'') of Systems modelled by Cospans
(\S\ref{sec:cospan})

Principle of Compositionality

``Network-style Diagrammatic Languages''

Network Diagram (\S\ref{sec:network_diagram}), ``Network-style
Diagrammatic Languages'' -- Electrical Circuits

other examples: Signal Flow Graphs (\S\ref{sec:signal_flow}),
Markov Processes (\S\ref{sec:markov_process}), Automata
(\S\ref{sec:automaton}), Petri Nets (\S\ref{sec:petri_net}), Chemical
Reaction Networks
%FIXME

\emph{Terminal} -- ``point of Interconnection''; ``boundary''
``marked'' using Cospan (\S\ref{sec:decorated_cospan}), Connected to
others using Pushouts (\S\ref{sec:pushout}); having a ``marked
boundary'' in this way makes a Closed System into an Open System;
process of ``freely''  ``marking boundaries'' constructs a Hypergraph
Category (\S\ref{sec:hypergraph_category}) from a ``library'' of
Network pieces

\emph{Components} with multiple Input/Output Terminals (possibly
labelled with some Type) connected to form a larger \emph{Network}

Components form \emph{Hyperedges} between labelled Vertices

\begin{itemize}
  \item each Terminal of an Open System may make ``Measurements''
    appropriate to the ``Type'' of the Terminal
  \item given a collection of Terminals, the \emph{Universum} is the
    Set of all possible Measurement outcomes
  \item each Open System has a collection of Terminals (and a Universum)
  \item the Semantics of an Open System is the Subset of Measurement
    outcomes on the Terminals that are ``permitted'' by the System,
    known as the \emph{Behavior} of the System
\end{itemize}

%FIXME universum = phase space?

``Laws'' (e.g. Ohm's Law) are mechanisms for Partitioning Behaviors
into \emph{Possible} and \emph{Impossible} Behaviors

given a Universum $\class{U}$, a Behavior of a System is an Element
of the Power Set $\pow(\class{U})$ (representing all possible
Measurements of the System), and a Law is an Element of
$\pow(\pow(\class{U}))$ representing all possible Behaviors of a
\emph{Class} of Systems

\emph{Interconnection} of Terminals asserts the Identification of
Variables at the Identified Terminals

Algebra of Semantic Objects and Homomorphism from Syntax to Semantics
(Principle of Compositionality \S\ref{sec:compositionality})



% ====================================================================
\section{Dynamical System}\label{sec:dynamical_system}
% ====================================================================

A \emph{Dynamical System} is defined as a tuple $(T,M,\Phi)$ where $T$
is a Monoid (\S\ref{sec:monoid}), M is a Set and $\Phi$ is a Function
(\S\ref{sec:set_function}).

Discrete, Continuous, Hybrid

Control Theory (\S\ref{sec:control_theory})

Initial Conditions, Driving Functions

%FIXME relate to transition systems, automata ?

\url{https://www.youtube.com/watch?v=cu718EbCOPs} Spivak 16:

can't have Identities and Feedback without Partiality %FIXME

Traced Ideals %FIXME



% ====================================================================
\section{Communicating System}\label{sec:communicating_system}
% ====================================================================

Denielou-Yoshida13 -- arises from a Choreographed collection of
Communicating Finite State Machines (CFSMs
\S\ref{sec:communicating_fsm}) in the context of Multiparty Session
Types (\S\ref{sec:multiparty_session})

Linear Multirole Logic (LMRL \S\ref{sec:lmrl})

\fist Process Calculus (\S\ref{sec:process_calculus})

Lange-Tuosto-Yoshida15:

\emph{Generalized Multiparty Compatibility} (GMC) -- Decidable
Condition characterizing a Set of Communicating Systems for which
questions of Safety (Deadlock-freedom, Orphan Messages, and
Unspecified Reception Configuration) can be Decided

\emph{Communicating System}:
\[
  \class{S} = (M_p)_{p \in \class{P}}
\]
where $M_p = (Q_p, q_{0p}, \Sigma, \delta_p)$ are CFSMs for each
Participant $p \in \class{P}$

\emph{Configuration} of $\class{S}$ %FIXME



% --------------------------------------------------------------------
\subsection{Choreography}\label{sec:choreography}
% --------------------------------------------------------------------

\emph{Choreographies}: ``Models of Interactions among Software
Components from a \emph{Global} point of view'' --
Lange-Tuosto-Yoshida15

\fist Global Types (\S\ref{sec:global_type}) give \emph{Choreographic
  Specifications} of Interactions (\S\ref{sec:interaction_geometry})



\subsubsection{Graphical Choreography}\label{sec:graphical_choreography}

or \emph{Global Graphs}

Lange-Tuosto-Yoshida15:

Communicating Finite State Machines (\S\ref{sec:communicating_fsm})

CFSMs as Behavioral Specifications of Distributed Components from
which a Choreography can be built

Sound and Complete characterization of ``safe'' CFSMs from which
Global Graphs can be constructed

CFSM model based on Asynchronous FIFO Message-passing Communication

Algorithm produces a Choreography expressed as a Global Graph (closely
related to BPMN 2.0 Choreography) given a Set of CFSMs (a Set of
Behavioral Specifications of components Interacting through
Asynchronous FIFO Message Passing)

example implementation for Golang in Ng-Yoshida16

\fist \emph{Note on notation}: Lange-Tuosto-Yoshida15 uses $A$ for
Alphabet and $Act$ for Actions, here we use $\Sigma$ for Alphabet and
$A$ for Actions

$\class{P}$ -- Finite Set of \emph{Participants}

$p, q, r, s$ -- ranges over Participants

$\Sigma$ -- Finite Alphabet

$C = \{ pq \ |\ p,q \in \class{P}, p \neq q \}$ -- Set of
\emph{Channels}

$A = C \times \{!,?\} \times \Sigma$ -- Set of \emph{Actions}

$\ell$ -- ranges over Actions

$\Sigma^*$, $A^*$ -- Set of Finite Words on $\Sigma$, $A$

$\varphi$ -- ranges over $\Sigma^*$, $A^*$

CFSM as a $4$-tuple:
\[
  M = (Q, q_0, \Sigma, \delta)
\]

$Q$ -- Finite Set of \emph{States}

$q_0 \in Q$ -- \emph{Initial State}

$\delta \subseteq Q \times A \times Q$ -- Set of \emph{Transitions}

Transitions are \emph{Labelled} by Actions

$sr!a$ -- Label representing Sending a Message $a$ from $s$ to $r$

$sr?a$ -- Label representing Receiving a Message $a$ by $r$ from $s$

$\class{L}(M) \subseteq A^*$ -- Language on $A$ Accepted by the
Automaton corresponding to $M$ where each State of $M$ is an Accepting
State %TODO explain accepting state

a \emph{Final State} $q \in Q$ has no Outgoing Transitions

a \emph{Sending State} $q \in Q$ has all Outgoing Transitions that are
Labelled by Send Actions

a \emph{Receiving State} $q \in Q$ has all Outgoing Transitions that are
Labelled by Receive Actions

a \emph{Mixed State} $q \in Q$ has both Send and Receive Actions as
Outgoing Transitions

CFSM $M$ is \emph{Deterministic} if for all States $q \in Q$ and all
Actions $\ell \in A$:
\begin{itemize}
  \item if $(q,\ell,q'),(q,\ell,q'') \in \delta$ then $q = q''$
\end{itemize}

CFSM $M$ is \emph{Minimal} if there are no other CFSMs $M'$ with fewer
States and Transitions such that $\class{L}(M) = \class{L}(M')$

\emph{Communicating System}:
\[
  \class{S} = (M_p)_{p \in \class{P}}
\]
where $M_p = (Q_p, q_{0p}, \Sigma, \delta_p)$ are CFSMs for each
Participant $p \in \class{P}$

\emph{Configuration} of $\class{S}$ %FIXME



% ====================================================================
\section{Phase Space}\label{sec:phase_space}
% ====================================================================

% --------------------------------------------------------------------
\subsection{Attractor \& Repeller}\label{sec:attractor_repeller}
% --------------------------------------------------------------------



% ====================================================================
\section{Self-organized Criticality}\label{sec:self_organized_criticality}
% ====================================================================

Property of Dynamical Systems with a Critical Point as an Attractor

%FIXME



% ====================================================================
\section{Reaction-diffusion System}\label{sec:reaction_diffusion}
% ====================================================================

% ====================================================================
\section{Hamiltonian System}\label{sec:hamiltonian_system}
% ====================================================================

% ====================================================================
\section{Ergodic Theory}\label{sec:ergodic_theory}
% ====================================================================

% --------------------------------------------------------------------
\subsection{Normal Number}\label{sec:normal_number}
% --------------------------------------------------------------------



% ====================================================================
\section{Chaos Theory}\label{sec:chaos_theory}
% ====================================================================

% ====================================================================
\section{Control Theory}\label{sec:control_theory}
% ====================================================================

Properties of possible Trajectories of Dynamical Systems
(\S\ref{sec:dynamical_system}) with certain Initial Conditions and
Driving Functions:

\emph{Observability}

\emph{Controllability} -- Property of Vector-valued Coupled
Differential Equations; \fist Cf. Information Theoretic \emph{Channel
  Capacity Theorem} (\S\ref{sec:channel_capacity})
%FIXME xref



% --------------------------------------------------------------------
\subsection{Signal Flow}\label{sec:signal_flow}
% --------------------------------------------------------------------

(\emph{SFG} or \emph{Mason Graph})

Nodes represent ``System Variables'' and Edges represent Functional
Connections between pairs of Nodes

%FIXME: move to information theory or control theory?

Semantics given by Corelation Categories
(\S\ref{sec:corelation_category})

Signal Flow Graph (\S\ref{sec:flow_graph})



% --------------------------------------------------------------------
\subsection{Controller}\label{sec:controller}
% --------------------------------------------------------------------

%FIXME

monitors and physically alters the operating conditions of a given
Dynamical System (\S\ref{sec:dynamical_system})
