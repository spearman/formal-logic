%%%%%%%%%%%%%%%%%%%%%%%%%%%%%%%%%%%%%%%%%%%%%%%%%%%%%%%%%%%%%%%%%%%%%%
%%%%%%%%%%%%%%%%%%%%%%%%%%%%%%%%%%%%%%%%%%%%%%%%%%%%%%%%%%%%%%%%%%%%%%
\part{Database Theory}\label{part:database_theory}
%%%%%%%%%%%%%%%%%%%%%%%%%%%%%%%%%%%%%%%%%%%%%%%%%%%%%%%%%%%%%%%%%%%%%%
%%%%%%%%%%%%%%%%%%%%%%%%%%%%%%%%%%%%%%%%%%%%%%%%%%%%%%%%%%%%%%%%%%%%%%

% ====================================================================
\section{Data Structure}\label{sec:data_structure}
% ====================================================================

%FIXME possibly move this section?

potentially Infinite Data Structures: Abstracted as $F$-coalgebras
(\S\ref{sec:f_coalgebra})

(also Data Structure as Folds of Church Encodings)
%??? FIXME

(Milewski - Understanding F-Algebras)

As Recursive Functions are defined as Fixed Points of regular
Functions, (Nested) Data Structures can be defined as Fixed Points of
regular Type Constructors.

Functors as Type Constructors give rise to Nested Data Structures that
allow Recursive Evaluation (generalized Folding).



% --------------------------------------------------------------------
\subsection{Ordered Tree}\label{sec:ordered_tree}
% --------------------------------------------------------------------

\subsubsection{Trie}\label{sec:trie}

\subsubsection{Prefix Tree}\label{sec:prefix_tree}

\subsubsection{Suffix Tree}\label{sec:suffix_tree}



% --------------------------------------------------------------------
\subsection{Fingertree}\label{sec:fingertree}
% --------------------------------------------------------------------



% ====================================================================
\section{Relational Model}\label{sec:relational_model}
% ====================================================================

\emph{Relational Calculus}

% ====================================================================
\section{Database Schema}\label{sec:database_schema}
% ====================================================================
