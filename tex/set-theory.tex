%%%%%%%%%%%%%%%%%%%%%%%%%%%%%%%%%%%%%%%%%%%%%%%%%%%%%%%%%%%%%%%%%%%%%%%%%%%%%%%%
%%%%%%%%%%%%%%%%%%%%%%%%%%%%%%%%%%%%%%%%%%%%%%%%%%%%%%%%%%%%%%%%%%%%%%%%%%%%%%%%
\part{Set Theory}\label{part:set_theory}
%%%%%%%%%%%%%%%%%%%%%%%%%%%%%%%%%%%%%%%%%%%%%%%%%%%%%%%%%%%%%%%%%%%%%%%%%%%%%%%%
%%%%%%%%%%%%%%%%%%%%%%%%%%%%%%%%%%%%%%%%%%%%%%%%%%%%%%%%%%%%%%%%%%%%%%%%%%%%%%%%

\emph{Set Theory} is formulated within First-order Logic
(\S\ref{sec:firstorder_logic}) and as such the objects of Set Theory are
\emph{Sets} (\S\ref{sec:set}) and \emph{Relations} (\S\ref{sec:set_relation}).
See Axiomatic Set Theory (\S\ref{sec:axiomatic_set_theory}) for specific Systems
of Set Theory.

\fist cf. Combinatory Logic (\S\ref{sec:combinatory_logic}): Quantifier-free
Higher-order Logic with expressive power of Set Theory

``collections''

Membership (\S\ref{sec:membership}) as the Fundamental Primitive of Set
Theory

Composition (\S\ref{sec:composition}) as the Fundamental Primitive of
Category Theory (Part \ref{part:category_theory})

Set Theory as Linear Algebra (Part \ref{part:linear_algebra}) over the
``Field with one Element'' \fist $F_1$-geometry (\S\ref{sec:f1_geometry})

the Category of Sets is equivalent to the Category of Discrete Topological
Spaces
(\url{https://math.stackexchange.com/questions/3133963/is-the-theory-of-the-category-of-topological-spaces-computable})

(wiki):
Boolean Algebras (\S\ref{sec:boolean_algebra}) are to Set Theory and ordinary
Propositional Logic (\S\ref{sec:propositional_logic}) what Interior Algebras
(\S\ref{sec:interior_algebra}) are to Topology (Part \S\ref{part:topology}) and
Modal $\mathsf{S4}$ Logic (\S\ref{sec:modal_logic})

\fist \emph{Counting} (\S\ref{sec:counting}) -- a means of establishing a
Bijection (One-to-one correspondence \S\ref{sec:bijective_function}) between the
Subset of Natural Numbers (\S\ref{sec:natural_number}) $\{1,2,\ldots,n\}$ and a
Set of Cardinality (\S\ref{sec:cardinality}) $n$; a Set for which no such
Bijection exists for any Finite $n$ is called an \emph{Infinite Set}, otherwise
it is a Finite Set;
cf. Countable (\S\ref{sec:countably_infinite}), Uncountable
(\S\ref{sec:uncountably_infinite})

\fist Combinatorial Set Theory (Infinitary Combinatorics
\S\ref{sec:infinitary_combinatorics})

an Elementary Topos (\S\ref{sec:elementary_topos}) can be thought of as a
Categorical Theory (\S\ref{sec:categoricity}) of Sets

\fist Synthetic Differential Geometry
(\S\ref{sec:synthetic_differential_geometry}) -- formalization of Differential
Geometry (\S\ref{sec:differential_geometry}) in the language of Topos Theory
(\S\ref{sec:topos_theory}); Differentials (\S\ref{sec:differential}) in Smooth
Models of Set Theory



% ==============================================================================
\section{Preset}\label{sec:preset}
% ==============================================================================

a Set without an Equality Relation on Elements



% ==============================================================================
\section{Set}\label{sec:set}
% ==============================================================================

A \emph{Set} is a Well-defined (\S\ref{sec:well_defined}) collection
of distinct objects defined by the Property of \emph{Set Membership}
which can be expressed as a Binary Relation
(\S\ref{sec:binary_relation}) ``$\in$''. An object that satisfies this
Relation for a particular Set is a \emph{Member} (or \emph{Element})
of that Set, e.g. $x \in A$ is True when $x$ is a Member of the Set
$A$.

A Set is allowed to have other Sets as Members and conversely a Set is
allowed to be a Member of other Sets. An object in Set Theory that is
not allowed to have Members is called an \emph{Individual} or
\emph{Urelement} (\S\ref{sec:urelement}) and an object that may have
Members but may not be a Member of another object is called a
\emph{Class} (\S\ref{sec:class}).

A Set $A$ may be given by its Extension (\S\ref{sec:extension}), that
is, listing each member in the Set:
\[
  A = \{x,y,z\}
\]
The Intentionsional (\S\ref{sec:intension}) Definition of a Set $B$
may be given by a Property (\S\ref{sec:property}) or other rule that
specifies its Members:
\[
  B = \{ x : x \in \nats_0 \wedge x < 4 \}
\]
defines the Set $B$ with Extension $\{ 0, 1, 2, 3 \}$.

By the Axiom of Extensionality (\S\ref{sec:extensionality_axiom}) Sets
are uniquely defined by their constituent Elements and therefore Sets
with multiple Members that are Identical are Equal and the order in
which Elements are given is unimportant:
\[
  \{ 2, 3 \} = \{ 3, 2 \} = \{ 2, 3, 3, 2 \}
\]

De Morgan's Laws (\S\ref{sec:de_morgan}) for Sets

a Set can be seen as a $0$-groupoid (\S\ref{sec:groupoid})



% ------------------------------------------------------------------------------
\subsection{Membership}\label{sec:membership}
% ------------------------------------------------------------------------------

Membership Relation $\in$



% ------------------------------------------------------------------------------
\subsection{Empty Set}\label{sec:empty_set}
% ------------------------------------------------------------------------------

% ------------------------------------------------------------------------------
\subsection{Singleton}\label{sec:singleton}
% ------------------------------------------------------------------------------

% ------------------------------------------------------------------------------
\subsection{Subset}\label{sec:subset}
% ------------------------------------------------------------------------------

For two Sets $A$ and $B$, if all the Members of $A$ are also Members
of $B$, then $A$ is a \emph{Subset} of $B$, denoted $A \subseteq B$:
\[
  (\forall x) x \in A \to x \in B \to A \subseteq B
\]
By this definition a Set is a Subset of itself, $A = B \to A
\subseteq B$, but a \emph{Proper Subset} may be defined as a Subset
that is not equal to the containing Set, $A \subset B$:
\[
  A \subseteq B \wedge B \nsubseteq A \to A \subset B
\]

\fist cf. Containment (Geometry \S\ref{sec:containment})



\subsubsection{Indicator Function}\label{sec:indicator_function}

or \emph{Characteristic Function}

\fist cf. Characteristic Function of a Random Variable
(\S\ref{sec:characteristic_function})

\fist cf. Convex Indicator Function (Convex Analysis
\S\ref{sec:convex_indicator_function}) -- Convex Indicator Function
$\chi_A : X \to \reals \cup \{ +\infty \}$ mapping Members of the
Subset $A \subset X$ to $0$ and to $+\infty$ otherwise

\begin{itemize}
  \item a Baire Set (Topology \S\ref{sec:baire_set}) is a Set whose
    Characteristic Function is a Baire Function (\S\ref{sec:baire_function})
  \item ...
\end{itemize}



% -----------------------------------------------------------------------------
\subsection{Set Union}\label{sec:set_union}
% -----------------------------------------------------------------------------

\emph{Union}

$A \cup B$



\subsubsection{Disjoint Union}\label{sec:disjoint_union}

Given a Family (\S\ref{sec:family}) of Sets ${A_i : i \in I}$,
the \emph{Disjoint Union} is defined as:
\[
  \bigsqcup_{i \in I} A_i = \bigcup_{i \in I} \{(x,i) | x \in A_i \}
\]

Coproduct (\S\ref{sec:coproduct})



\subsubsection{Infinitary Union}\label{sec:infinitary_union}

\[
  x \in \bigcup S \leftrightarrow \exists y \in S : x \in y
\]



% -----------------------------------------------------------------------------
\subsection{Set Intersection}\label{sec:set_intersection}
% -----------------------------------------------------------------------------

\emph{Intersection}

\fist cf. Intersection (Geometry \S\ref{sec:intersection})

$A \cap B$



% -----------------------------------------------------------------------------
\subsection{Relative Complement}\label{sec:relative_complement}
% -----------------------------------------------------------------------------

The \emph{Relative Complement} (or \emph{Difference}) of two Sets $A$
and $B$, denoted $A \setminus B$, is the Set of Elements in $A$ but
not $B$.



\subsubsection{Absolute Complement}\label{sec:absolute_complement}

The \emph{Absolute Complement} (or \emph{Complement}) of a Set $A$ is
the Relative Complement of $A$ with the given Universe
(\S\ref{sec:set_universe}) $\mathcal{U}$:
\[
  \mathcal{U} \setminus A
\]



% -----------------------------------------------------------------------------
\subsection{Symmetric Difference}\label{sec:symmetric_difference}
% -----------------------------------------------------------------------------

The \emph{Symmetric Difference} of two Sets $A$ and $B$, denoted $A
\oplus B$, is the Set of Elements in $A$ or $B$ but not both $A$ and
$B$:
\[
  A \oplus B =
  \{ x : (x \in A \vee x \in B) \wedge x \notin A \cap B \}
\]
or:
\[
  A \oplus B = A \setminus B \cup B \setminus A
\]



% -----------------------------------------------------------------------------
\subsection{Cartesian Product}\label{sec:cartesian_product}
% -----------------------------------------------------------------------------

\fist Product (Category Theory \S\ref{sec:product})

Left-adjoint to Exponentiation (\S\ref{sec:exponentiation})

(ncat):

A \emph{Cartesian Space} (\S\ref{sec:cartesian_space}) $\reals^n$ is a Finite
Cartesian Product of the Real Line $\reals$ with itself, where $n$ is some
Natural Number (possibly Zero).



% ------------------------------------------------------------------------------
\subsection{Partition}\label{sec:set_partition}
% ------------------------------------------------------------------------------

A \emph{Partition} of a Set $X$ is a Set of Non-empty Subsets of $X$ such that
every Element of $X$ is in exactly one Subset, thus $X$ is the Disjoint Union
(\S\ref{sec:disjoint_union}) of the Subsets. A Partition $P$ of $X$ has the
Properties:
\begin{enumerate}
  \item $\emptyset \notin P$
  \item $\bigcup_{A \in P}A = X$
  \item $A,B \in P \wedge A \neq B \to A \cap B = \emptyset$
\end{enumerate}

cf. Partition Theory (Combinatorics \S\ref{sec:partition_theory})

Bell Numbers (\S\ref{sec:bell_number}) $B_n$: count of number of Partitions for
a Set of Cardinality $n$, i.e. the number of possible \emph{Equivalence
Relations} on a Set of $n$ Elements



% ------------------------------------------------------------------------------
\subsection{Powerset}\label{sec:powerset}
% ------------------------------------------------------------------------------

$\pow{X}$, $2^X$

(wiki):

the Power Set of a Set always has a strictly higher Cardinality than the Set
itself

the Power Set of the Natural Numbers is Bijective with the Set of the Real
Numbers

a Power Set with the Operations of Union, Intersection, and Complement forms a
Boolean Algebra (\S\ref{sec:boolean_algebra})

any Finite Boolean Algebra is Isomorphic to the Boolean Algebra of the Power Set
of a Finite Set

(see Stone's Representation Theorem with regards to Infinite Boolean Algebras)



\subsubsection{Cantor's Theorem}\label{sec:cantors_theorem}

the Power Set of a Countably Infinite Set is Uncountably Infinite



% ------------------------------------------------------------------------------
\subsection{Cardinality}\label{sec:cardinality}
% ------------------------------------------------------------------------------

The \emph{Cardinality} (or \emph{Size}) of a Set $A$, denoted $|A|$ or
$card(A)$, is a measure of the number of Elements in the Set. Two Sets with the
same Cardinality are said to be \emph{Equinumerous} (\emph{Equipotent}
\S\ref{sec:equipotent}), i.e. there exists a Bijection
(\S\ref{sec:bijective_function}) between them.

The unique Set with Cardinality 0 is called the \emph{Empty Set} and is denoted
$\{\}$ or $\varnothing$. A Set with Cardinality 1 is called a \emph{Singleton
  Set}.

A Cardinal Number (\S\ref{sec:cardinal_number}) is an Isomorphism Class of Sets.

(wiki):

\emph{Counting} (\S\ref{sec:counting}) -- a means of establishing a Bijection
(One-to-one correspondence \S\ref{sec:bijective_function}) between the Subset of
Natural Numbers (\S\ref{sec:natural_number}) $\{1,2,\ldots,n\}$ and a Set of
Cardinality $n$; a Set for which no such Bijection exists for any Finite $n$ is
called an \emph{Infinite Set}, otherwise it is a Finite Set;
cf. Countable (\S\ref{sec:countably_infinite}), Uncountable
(\S\ref{sec:uncountably_infinite})

\textbf{Thm.} no Bijection can exist between Sets of different
Cardinality (\S\ref{sec:cardinality}), and two Bijections can be Composed to
give another Bijection; therefore Counting (\S\ref{sec:counting}) the same Set
in different ``ways'' can never result in different Cardinality

\fist \emph{Enumerative Combinatorics} (\S\ref{sec:enumerative_combinatorics})
-- (wiki): finding \emph{Counting Functions} (\S\ref{sec:counting_function}) to
Compute (\S\ref{sec:computability}) the Cardinality of Elements of Finite Sets
\emph{without} ``actually'' Counting them

(Joyal81) Cardinality induces a Homomorphism
$Card : \cat{FinSet}\llbracket{X}\rrbracket \to \ints\llbracket{x}\rrbracket$
between the Category of Combinatorial Species
(\S\ref{sec:combinatorial_species}) (seen as a Ring of Hurwitz Series over
the Category $\cat{FinSet}$ of Finite Sets and Total Functions) and the Ring
of Hurwitz Series over $\ints$

\fist Infinitary Combinatorics (\S\ref{sec:infinitary_combinatorics})

nLab:

a Cardinal is a Set

The Cardinality of a Set $S$ is the smallest possible Ordinal Rank
(\S\ref{sec:ordinal_rank}) of any Well-order on $S$. (Requires Axiom
of Choice \S\ref{sec:choice_axiom} which Implies the Well-order
Theorem \S\ref{sec:wellorder_theorem}, i.e. all Sets are
Well-orderable)

Bijection (\S\ref{sec:bijective_function}) and Injection
(\S\ref{sec:injective_function})

Cantor's Theorem (\S\ref{sec:cantors_theorem})

The Axiom of Choice (\S\ref{sec:choice_axiom}) Implies the Law of
Trichotomy (\S\ref{sec:trichotomy_law}) for Cardinality which assigns
a given Set $X$ to one of the Classes:

\begin{description}
\item [Finite] $|X| < |\nats|$ (\S\ref{sec:finite_cardinality})
\item [Countably Infinite] $|X| = |\nats| = \aleph_0$
  (\S\ref{sec:countably_infinite})
\item [Uncountably Infinite] $|\nats| < |X|$
  (\S\ref{sec:uncountably_infinite})
\end{description}

Infinitary Combinatorics (\S\ref{sec:infinitary_combinatorics})

\asterism


Generalizations (Topology):

\begin{itemize}
  \item Euler Characteristic (\S\ref{sec:euler_characteristic}):
    admits Negative Integer values
  \item Homotopy Cardinality (\S\ref{sec:homotopy_cardinality}):
    admits Positive Real values
\end{itemize}

the Space of Finite Sets has Homotopy Cardinality $e$



\subsubsection{Equipotent}\label{sec:equipotent}

two Sets or Classes are \emph{Equinumerous} or \emph{Equipotent} if there exists
a Bijection (\S\ref{sec:bijective_function}) between them; Equipotent Sets are
said to have the same Cardinality



\subsubsection{Finite}\label{sec:finite_cardinality}

\subsubsection{Negative}\label{sec:negative}

\fist Antisets (\S\ref{sec:antiset})



\subsubsection{Transfinite}\label{sec:transfinite}

Large Cardinal (\S\ref{sec:large_cardinal})

Infinite but not Absolutely Infinite (\S\ref{sec:absolute_infinity})



\paragraph{Countably Infinite}\label{sec:countably_infinite}\hfill

\fist cf. Counting (\S\ref{sec:counting})



\paragraph{Uncountably Infinite}\label{sec:uncountably_infinite}\hfill

\fist cf. Counting (\S\ref{sec:counting})



\paragraph{Dedekind-infinite}\label{sec:dedekind_infinite}\hfill



\subsubsection{Subcountable}\label{sec:subcountable}

A Set $X$ is \emph{Subcountable} if there is a Partial Surjection from
$\nats$ onto the $X$, that is, if $X$ is Finite
(\S\ref{sec:finite_cardinality}) or Countably Infinite
(\S\ref{sec:countably_infinite})



% ------------------------------------------------------------------------------
\subsection{Well-founded Set}\label{sec:wellfounded_set}
% ------------------------------------------------------------------------------

Membership Relation $\in$ is Well-founded (\S\ref{sec:well_founded})

Axiom of Foundation (\S\ref{sec:foundation_axiom})

Non-well-founded Set Theory (\S\ref{sec:non_wellfounded})



% ------------------------------------------------------------------------------
\subsection{Index Set}\label{sec:index_set}
% ------------------------------------------------------------------------------

An \emph{Index Set} is one that \emph{Indexes} (or \emph{Labels})
Members of another Set. Indexing is a Surjective Function
(\S\ref{sec:surjective_function}) from an Index Set onto a target Set.

\fist cf. Filtration (\S\ref{sec:filtration})



\subsubsection{Indexed Family}\label{sec:indexed_family}

An \emph{Indexed Family} of Sets is a Function from an Index Set to the Class
(\S\ref{sec:class}) of all Sets. For Index Set $J$ and Indexed Set $A$, the
Indexed Family may be denoted
\[
  (A_j)_{j \in J}
\]

\fist cf. Family of Sets (\S\ref{sec:family}) -- collection that may be a
Multiset (\S\ref{sec:multiset}) and/or a Small or Proper Class
(\S\ref{sec:class})



\paragraph{Cover}\label{sec:cover}\hfill

A \emph{Cover} of a Set $X$ is an Indexed Family of Sets $C = \{ U_i :
i \in I \}$ such that their Union contains $X$:
\[
  X \subseteq \bigcup_{i \in I} U_i
\]



% ------------------------------------------------------------------------------
\subsection{Transitive Set}\label{sec:transitive_set}
% ------------------------------------------------------------------------------

A Set, $A$, is \emph{Transitive} if and only if:
\[
  \bigcup A \subseteq A
\]
That is for each non-empty Set $B \in A$:
\[
  B \in A \to B \subset A
\]
Transitivity for Classes (\S\ref{sec:class}) is defined in the same
way.

For two Transitive Sets, $A$ and $B$, the Set $A \cup B \cup \{A,B\}$
is Transitive.

A Set, $B$, containing no Urelements is Transitive if and only if $A
\subset \pow(X)$



\subsubsection{Admissible Set}\label{sec:admissible_set}

An \emph{Admissible Set}, $A$, is a Transitive Set such that $\langle A, \in
\rangle$ is a Model (\S\ref{sec:model}) of Kripke-Platek Set Theory
(\S\ref{sec:kripke_platek}).

The smallest example of an Admissible Set is the Set of \emph{Hereditarily
  Finite Sets} (Sets that have Finite Pictures, i.e. Finite Accessible Pointed
Graphs \S\ref{sec:accessible_pointed} with the Set assigned to the Point), an
example of which is a Quine Atom (\S\ref{sec:quine_atom}) $Q = \{Q\}$.
\cite{aczel88} %FIXME ref hereditarily finite



% ------------------------------------------------------------------------------
\subsection{Pointed Set}\label{sec:pointed_set}
% ------------------------------------------------------------------------------

A \emph{Pointed Set} (or \emph{Based Set} or \emph{Rooted Set}) is an Ordered
Pair $(X, x_0)$ where $X$ is a Set and $x_0 \in X$ is the \emph{Basepoint} of
$X$. This defines an Algebraic Structure (\S\ref{sec:algebraic_structure}) on
$X$ with a single Nullary Function that returns the Basepoint.



% ------------------------------------------------------------------------------
\subsection{Nominal Set}\label{sec:nominal_set}
% ------------------------------------------------------------------------------

\url{https://www.youtube.com/watch?v=b38uoZccGuU} Silva

Infinite Set with Finite Representation

Nominal Automata (\S\ref{sec:nominal_automaton})



% ------------------------------------------------------------------------------
\subsection{Multiset}\label{sec:multiset}
% ------------------------------------------------------------------------------

A \emph{Multiset} (or \emph{Bag}) is a 2-tuple $(A,m)$ of an
\emph{Underlying Set}, $A$, together with a \emph{Multiplicity
  Function}, $m : A \to \nats_{\geq 1}$, mapping Elements
of $A$ to Positive Natural Numbers representing the
\emph{Multiplicity} Elements, that is the number of times an Element
occurs in the Multiset. A Multiset may be denoted with square
brackets:
\[
  [a,a,b]
\]
If the Underlying Set is restricted to a Subset of a given
\emph{Universe} (\S\ref{sec:set_universe}), $U$, the Multiplicity
Function may be extended to $m_U : U \to \nats$ where $a \in
U, a \notin A \leftrightarrow m(a)=0$.

\emph{Indicator Function} (\S\ref{sec:indicator_function})



% ------------------------------------------------------------------------------
\subsection{Antiset}\label{sec:antiset}
% ------------------------------------------------------------------------------

2013 - Baldo - \emph{Introduction to Antisets, Antigroups, and Antirings}

formulated within Axiomatic ZFC: treat Antisets a Sets of \emph{Abstract
  Spaces}, i.e. Sets that have \emph{Potential} to contain Elements but not
containing any

\emph{Negative Spaces} -- $\delta^-$ such that for an Element $a$ of any Set:
\[
  \{\delta^-\} \cup \{a\} = \varnothing
\]
Cardinality counts ``holes''

For a Set $A$, the \emph{Antiset} $\tilde{A}$ is such that
$A \cup \tilde{A} = \varnothing$

for $|A|=n$, then $|\tilde{A}| = -n$

\emph{Positive Spaces} -- gives a Set the potential to ``lose'' Space;
Cardinality counts empty ``platforms''

\fist Negative Cardinality (\S\ref{sec:negative})



% ==============================================================================
\section{Urelement}\label{sec:urelement}
% ==============================================================================

An \emph{Urelement} (or \emph{Atom} or \emph{Individual}) is an Object
that may be an Element of a Set, but is not itself a Set.

Urelements are the dual to Proper Classes (\S\ref{sec:proper_class})
as an Urelement cannot have Members while a Proper Class cannot be a
Member.

A Two-sorted First-order Theory with Sets and Urelements has $a \in b$
defined only when $b$ is a Set.

A One-sorted First-order Theory may be defined with an Unary Relation
distinguishing Sets and Urelements, which can be achieved if the Unary
Relation can at least distinguish Urelements from the Empty Set, since
all other Sets have Members. The Axiom of Extensionality must also be
formulated to apply only to Sets and not Urelements.

Quine Atom (\S\ref{sec:quine_atom})



% ==============================================================================
\section{Class}\label{sec:class}
% ==============================================================================

A \emph{Class} is any Subset of the \emph{Universe} (\S\ref{sec:set_universe})
of discussion.

\begin{description}
  \item [Proper Class] a Class that is not a Set
  \item [Small Class] a Class that is a Set
\end{description}

\fist von Neumann-Bernays-G\"odel Set Theory (\S\ref{sec:nbg_set_theory})



% ------------------------------------------------------------------------------
\subsection{Proper Class}\label{sec:proper_class}
% ------------------------------------------------------------------------------

A \emph{Proper Class} is an Object that cannot be a Member of another Object.
This is the Dual concept of an Urelement which cannot have another Object as a
Member.



% ------------------------------------------------------------------------------
\subsection{Subclass}\label{sec:subclass}
% ------------------------------------------------------------------------------

% ------------------------------------------------------------------------------
\subsection{Class Map}\label{sec:class_map}
% ------------------------------------------------------------------------------



% ==============================================================================
\section{Family}\label{sec:family}
% ==============================================================================

A \emph{Family} is a collection of Sets that is allowed to be a Multiset
(\S\ref{sec:multiset}) and/or a Small or Proper Class (\S\ref{sec:class}).

\fist cf. Indexed Family (\S\ref{sec:indexed_family})



% ------------------------------------------------------------------------------
\subsection{Transversal}\label{sec:transversal}
% ------------------------------------------------------------------------------

given a Family of Sets, $C$, a \emph{Transversal} (or \emph{Cross-section}) is a
Set containing exactly one Element from each Member of the Collection

cf. Transversality (\S\ref{sec:transversality})



% ==============================================================================
\section{Universe}\label{sec:set_universe}
% ==============================================================================

% FIXME differentiate between Universes
A \emph{Universe} is a Set, $U$, with the following Closure Properties
\cite{maclane69}:
\begin{enumerate}
\item $x \in A \in U \to x \in U$
\item $x \in U \wedge y \in U \to \{x,y\}, \langle x,y
  \rangle, x \times y \in U$
\item $x \in U \to \pow(x) \in U \wedge \bigcup x \in U$
\item $\omega = \{0,1,2,\ldots\} \in U$
\item Given a Surjective Function, $f : a \to b, a \in
  U, b \subset U \to b \in U$
\end{enumerate}

nCatlab: \url{https://ncatlab.org/nlab/show/universe}

\fist Universe (Category Theory \S\ref{sec:category_universe})

%FIXME ``mathematical universe'' ??? appendix ???



% ------------------------------------------------------------------------------
\subsection{Small Set}\label{sec:small_set}
% ------------------------------------------------------------------------------

A \emph{Small Set} may be said to be a member of a Universe that is
not itself a Universe.



% ------------------------------------------------------------------------------
\subsection{Cumulative Hierarchy}\label{sec:cumulative_hierarchy}
% ------------------------------------------------------------------------------

a Family of Sets $W_\alpha$ Indexed by \emph{Ordinals} $\alpha$ such
that:
\begin{itemize}
  \item $W_\alpha \subseteq W_{\alpha+1}$
  \item if $\alpha$ is a Limit Ordinal then $W_\alpha = \cup_{\beta
    \leq \alpha} W_\beta$
\end{itemize}

Satisfies a form of the Reflection Principle
(\S\ref{sec:reflection_principle}): any Formula in the Language of Set
Theory that holds of the Union $W$ of the Hierarchy also holds in some
stages $W_\alpha$

the ``standard'' Cumulative Hierarchy is the Von Neumann Universe
(\S\ref{sec:vonneumann_universe}) with $V_{\alpha+1} = \pow(V_\alpha)$

other Cumulative Hierarchies:
\begin{itemize}
  \item $L_\alpha$ -- Sets of the Constructible Universe
    (\S\ref{sec:constructible_universe})
  \item Boolean Valued Models constructed by Forcing
    (\S\ref{sec:forcing}) -- built using a Cumulative Hierarchy
  \item Well-founded Sets in a Model of Set Theory (possibly not
    Satisfying the Axiom of Foundation) forms a Cumulative Hierarchy
    whose Union Satsifies the Axiom of Foundation
\end{itemize}



\subsubsection{Constructible Hierarchy}\label{sec:constructible_hierarchy}

\paragraph{Constructible Universe}\label{sec:constructible_universe}\hfill

Union of the Constructible Hierarchy $L_\alpha$



% ------------------------------------------------------------------------------
\subsection{Reflection Principle}\label{sec:reflection_principle}
% ------------------------------------------------------------------------------

nCatlab \url{https://ncatlab.org/nlab/show/reflection+principle}:

the \emph{Reflection Principle} states that it is possible to find
Sets that ``resemble'' the Class (\S\ref{sec:class}) of all Sets that
are ``constructed in some ways''; any Statement that is True of the
Universe $U$ \emph{already} holds at some ``Initial Segment''
$U_\alpha$-- i.e. no First-order Property can distinguish the
Set-theoretic Universe from the \emph{Extensions} of all its Member
Sets, suggesting the \emph{Expansiveness} of $U$: $U$ gets
``arbitrarily large''

%FIXME: is this related to skolem's ``paradox'' ???

the Type of Small Types (\S\ref{sec:type_universe}) in Type Theory is
a Reflection of the Type System in itself

G\"odel's encoding of an \emph{Unprovability Predicate} into Peano
Arithmetic can be viewed as an \emph{Anti-reflection Principle} for
Peano Arithmetic; suggests that the Validity of Reflection Principle
for some Theory $T$ informally expresses or approximates the
\emph{Internal Consistency} or \emph{Completeness} of $T$



% ------------------------------------------------------------------------------
\subsection{Superstructure}\label{sec:superstructure}
% ------------------------------------------------------------------------------

An Universe may be generated over a Set resulting in a
\emph{Superstructure}. The Superstructure over a Set $X$:
\[
  \mathbf{S}X := \bigcup^{\infty}_{i=0}\mathbf{S}_i X
\]
can be defined by Structural Recursion
(\S\ref{sec:recursive_definition}) as follows:
\begin{itemize}

\item $\mathbf{S}_0 X = X$
\item $\mathbf{S}_1 X = X \cup \pow(X)$
\item $\mathbf{S}_{n+1} X =
  \mathbf{S}_n X \cup \pow(\mathbf{S}_n X)$

\end{itemize}
Some Properties of $\mathbf{S}\{\}$ (the Superstructure over the Empty
Set):
\begin{itemize}

\item $\nats \subset \mathbf{S}\{\}$
\item $\nats \notin \mathbf{S}\{\}$ (Elements of $\mathbf{S}\{\}$
  are Finite Sets)
\item $\mathbf{S}\{\}$ contains all of the Hereditarily Finite Sets
%FIXME ref Hereditarily finite sets

\end{itemize}

The Superstructure over $\nats$, $\mathbf{S}\nats$, is
considered the \emph{Universe of Ordinary Mathematics}.



% ------------------------------------------------------------------------------
\subsection{Von Neumann Universe}\label{sec:vonneumann_universe}
% ------------------------------------------------------------------------------

The Von Neumann Universe is Extended by various Anti-foundation Axioms
(\S\ref{sec:anti_foundation})

the ``standard'' Cumulative Hierarchy
(\S\ref{sec:cumulative_hierarchy}) defined as $V_{\alpha+1} =
\pow(V_\alpha)$



% ------------------------------------------------------------------------------
\subsection{Grothendieck Universe}\label{sec:grothendieck_universe}
% ------------------------------------------------------------------------------

%FIXME similar to definitions above



% ------------------------------------------------------------------------------
\subsection{Constructible Universe}\label{sec:constrictible_universe}
% ------------------------------------------------------------------------------

Cumulative Hierarchy (\S\ref{sec:cumulative_hierarchy})



% ------------------------------------------------------------------------------
\subsection{Forcing}\label{sec:forcing}
% ------------------------------------------------------------------------------

Boolean Valued Models constructed by Forcing are built using a
Cumulative Hierarchy (\S\ref{sec:cumulative_hierarchy})



% ==============================================================================
\section{Relation}\label{sec:set_relation}
% ==============================================================================

Domain (\S\ref{sec:domain})

Predicate (\S\ref{sec:predicate})

The Sets $X_1, X_2, \ldots$ are called the \emph{Domain}
(\S\ref{sec:domain}) of $R$. If any Domain of $R$ is empty, then $R$
is the unique \emph{Empty Relation} $R = \varnothing$.

\fist cf. \emph{Relation} (Databases \S\ref{sec:database_relation})

The \emph{Graph} of a Relation $R$, $G(R)$, defines the Relation in
Extension (\S\ref{sec:extension}) as a Subset of the Cartesian Product
(\S\ref{sec:cartesian_product}) of the Domain:
\[
  G(R) \subseteq X_1 \times X_2 \times \ldots
\]

The \emph{Arity} of a Relation is the Dimension of the Domain in the
Graph of a Relation. %FIXME

The Expression $R x_1 x_2 \ldots$ where $x_i \in X_i$ is True when
$(x_1, x_2, \ldots) \in G(R)$ and False otherwise. As such, any
Relation may defined by a Boolean-valued \emph{Characteristic
  Function} (or \emph{Indicator Function}
\S\ref{sec:indicator_function}, cf. Predicate \S\ref{sec:predicate}):
\[
  f_R : X_1 \times X_2 \times \ldots \to \{\top,\bot\}
\]

\fist a \emph{Function} (\S\ref{sec:set_function}) is a Left-total
(\S\ref{sec:left_total}) Right-unique (Functional
\S\ref{sec:functional_relation}) Relation

\fist \emph{Implicit Function Theorem} (\S\ref{sec:implicit_function}): allows
Relations to be converted to Functions of several Real Variables by representing
the Relation as the \emph{Graph} of a Function



% ------------------------------------------------------------------------------
\subsection{Finitary Relation}\label{sec:finitary_relation}
% ------------------------------------------------------------------------------

A Relation with a Finite Arity $n$ is called a \emph{Finitary
  Relation} or \emph{$n$-ary Relation}.

There are only two $0$-ary Relations on the Empty Tuple $()$: one that
is always True and one that is always False.

A $2$-ary Relation is usually called a Binary Relation
(\S\ref{sec:binary_relation}), and a $3$-ary Relation may be called a
Ternary Relation.



\subsubsection{Correspondence}\label{sec:correspondence}

An $n$-ary Relation may be completely specified by an $n + 1$-tuple
called a \emph{Correspondence} (or \emph{Embedded} or \emph{Included
  Relation}):
\[
  (X_1, \ldots, X_n, G(R))
\]



% ------------------------------------------------------------------------------
\subsection{Binary Relation}\label{sec:binary_relation}
% ------------------------------------------------------------------------------

A \emph{Binary} (or \emph{Dyadic}) Relation is a $2$-ary Relation.

When the Graph of a Binary Relation is plotted on a Coordinate Plane
(\S\ref{sec:cartesian_coordinates}), the First Elements of the Ordered
Pairs of $G(R)$, mapped to the Horizontal Axis, are \emph{Abscissae},
and the Second Elements of the Ordered Pairs of $G(R)$, mapped to the
Vertical Axis, are called \emph{Ordinates}.

Reduction Relation (\S\ref{sec:reduction_relation})

A Binary Relation $R$ on a Set $A$ is:
\begin{description}
\item[Serial](\S\ref{sec:serial_relation}) if:

  $ \forall a \in A \exists b \in A : xRy $

\item[Reflexive](\S\ref{sec:serial_relation}) if:

  $ xRx = \top $

  Reflexive Implies Transitive and Serial

\item[Irreflexive] (also \emph{Strict}
  \S\ref{sec:irreflexive_relation}) if:

  $ xRx = \bot $

\item[Co-reflexive](\S\ref{sec:coreflexive_relation}) if:

  $ xRy \to a = b $

\item[Transitive](\S\ref{sec:transitive_relation}) if:

  $ xRy \wedge yRc \to xRc $

  Transitive and Irreflexive if and only if Transitive and Asymmetric

\item[Symmetric](\S\ref{sec:symmetric_relation}) if:

  $ xRy \leftrightarrow yRx $

\item[Anti-symmetric](\S\ref{sec:antisymmetric_relation}) if:

  $ xRy \wedge yRx \to a = b $

\item[Asymmetric](\S\ref{sec:antisymmetric_relation}) if both
  Anti-symmetric and Irreflexive:

  $ xRy \to \neg yRx $

\item[Left-total] (\S\ref{sec:left_total}) if:

  $ \forall a \in A \exists b \in A : xRy $

  Multimaps (\S\ref{sec:multimap}) and Functions
  (\S\ref{sec:set_function}) are Left-total

\item[Right-total] (\emph{Surjective} or \emph{Onto}
  \S\ref{sec:surjective_function}) if:

  $ \forall b \in A \exists a \in A : xRy $

\item[Right-unique] (or \emph{Functional} or \emph{Univalent}
  \S\ref{sec:functional_relation}) if:

  $ \forall x \in dom(R), y \in rng(R)
    (xRy \wedge xRz \to y = z) $

\item[Total] (\S\ref{sec:total_relation}) if:

  $ \forall a,b \in A, xRy \vee yRx $

  Total Implies Reflexive (\S\ref{sec:reflexive_relation})

\item[Trichotomous] (\S\ref{sec:trichotomous_relation}) if:

  $ xRy \vee yRx \vee a = b $

  Trichotomous Implies Irreflexive (\S\ref{sec:irreflexive_relation}).
  Trichotomous and Transitive (\S\ref{sec:transitive_relation})
  Implies Asymmetric (\S\ref{sec:asymmetric_relation}). A Transitive
  Trichotomous Relation is a Strict Total Order
  (\S\ref{sec:strict_order}, \S\ref{sec:total_order}).

\item[Right Euclidean] (\S\ref{sec:euclidean_relation}) if:

  $ xRy \wedge xRc \to yRc $

\item[Left Euclidean] (\S\ref{sec:euclidean_relation}) if:

  $ yRx \wedge cRx \to yRc $

\end{description}

From the above Classes of Relations, the following Binary Ordering
Relations (\S\ref{sec:ordering_relation}) are distinguished, listed
here from most general to most restricted:

\begin{description}
\item[Preorder] (\S\ref{sec:preorder}) when Reflexive and Transitive
  (all Partial Orders and Equivalence Relations are Preorders)
\item[Partial Order] (\S\ref{sec:partial_order}) when Preorder and
  Anti-symmetric
\item[Weak order] (\S\ref{sec:weak_order}) when Preorder and Total
\item[Total Order] (\S\ref{sec:total_order}) when a Partial Order and
  Total
\item[Partial Equivalence] (\S\ref{sec:partial_equivalence}) when
  Symmetric and Transitive
\item[Equivalence] (\S\ref{sec:equivalence_relation}) when Reflexive,
  Symmetric, and Transitive
\end{description}

The Set of all Binary Relations on a Set $A$ is denoted
$\mathbf{Rel}(A)$ and is the Powerset (\S\ref{sec:powerset}) of $A
\times A$: $2^{A \times A}$.

A Dependency Relation (\S\ref{sec:dependency_relation}) is a Binary
Relation that is Symmetric and Reflexive.



\subsubsection{Converse}\label{sec:converse}

\emph{Converse} (or \emph{Opposite Relation})



\subsubsection{Reflexive Relation}\label{sec:reflexive_relation}

\subsubsection{Irreflexive Relation}\label{sec:irreflexive_relation}

\subsubsection{Coreflexive Relation}\label{sec:coreflexive_relation}

\subsubsection{Transitive Relation}\label{sec:transitive_relation}

\subsubsection{Symmetric Relation}\label{sec:symmetric_relation}

\fist cf. Symmetry (Algebraic Structure Endomorphism
\S\ref{sec:structure_symmetry})

\fist Mathematical Symmetry (\S\ref{sec:symmetry})



\subsubsection{Antisymmetric Relation}\label{sec:antisymmetric_relation}

Partial Order (\S\ref{sec:partial_order})



\subsubsection{Asymmetric Relation}\label{sec:asymmetric_relation}

\subsubsection{Left-total Relation}\label{sec:left_total}

\subsubsection{Total Relation}\label{sec:total_relation}

\subsubsection{Functional Relation}\label{sec:functional_relation}

Right-unique

A Function (\S\ref{sec:set_function}) is a Left-total
(\S\ref{sec:left_total}) Functional Relation.



\subsubsection{Trichotomous Relation}\label{sec:trichotomous_relation}

\paragraph{Law of Trichotomy}\label{sec:trichotomy_law}\hfill

Trichotomous Relation $<$ on a Set $X$ exactly one of the following
holds for all $x,y \in X$:
\begin{itemize}
\item $x < y$
\item $x = y$
\item $x > y$
\end{itemize}



\subsubsection{Serial Relation}\label{sec:serial_relation}

\subsubsection{Extensional Relation}\label{sec:extensional_relation}

A Binary Relation $R$ is \emph{Extensional} if and only if:
\[
  \forall x,y (x = y \leftrightarrow
    \forall z (R(x,z) \leftrightarrow R(y,z))
\]
The Converse of an Extensional Relation is not necessarily
Extensional.

\subsubsection{Euclidean Relation}\label{sec:euclidean_relation}

\emph{Left-euclidean} \emph{Right-euclidean}



% ------------------------------------------------------------------------------
\subsection{Endorelation}\label{sec:endorelation}
% ------------------------------------------------------------------------------

$X = Y$

\begin{description}
\item [Directed Graph] (\S\ref{sec:directed_graph})

\item [Undirected Graph] (\S\ref{sec:undirected_graph}) Irreflexive,
  Symmetric

\item [Tournament] (\S\ref{sec:tournament}) Irreflexive, Antisymmetric

\item [Dependency] (\S\ref{sec:dependency_relation}) Reflexive,
  Symmetric

\item [Weak Order] (\S\ref{sec:weak_order}) Transitive

\item [Preorder] (\S\ref{sec:preorder}) Reflexive, Transitive

\item [Partial Order] (\S\ref{sec:partial_order}) Reflexive,
  Antisymmetric, Transitive

\item [Strict Partial Order] (\S\ref{sec:strict_order},
  \S\ref{sec:partial_order}) Irreflexive, Antisymmetric, Transitive

\item [Partial Equivalence] (\S\ref{sec:partial_equivalence})
  Symmetric, Transitive

\item [Equivalence Relation] (\S\ref{sec:equivalence_relation})
  Reflexive, Symmetric, Transitive

\end{description}

Any Endorelation may be given as a Directed Graph. An Endorelation is
a Complete Graph (\S\ref{sec:complete_graph}) when $a \neq b
\to xRy$ and implies Symmetry. An Endorelation is a Tournament
when $a \neq b \to xRy \vee yRx$ and Implies Antisymmetry.



% ------------------------------------------------------------------------------
\subsection{Equivalence Relation}\label{sec:equivalence_relation}
% ------------------------------------------------------------------------------

An \emph{Equivalence Relation} on a Set $X$ is a Reflexive, Symmetric,
and Transitive Relation that Partitions (\S\ref{sec:set_partition})
the Set into Disjoint Subsets called \emph{Equivalence Classes}
(\S\ref{sec:equivalence_class}).

cf. Apartness Relation (\S\ref{sec:apartness}) in Constructive Set Theory

The Equivalence Class of an Element $a \in X$ with Equivalence
Relation $\sim$ is the Subset of $X$ defined as:
\[
    [a] = \{x \in X | a \sim x\}
\]
The \emph{Quotient Set} (\S\ref{sec:quotient_set}) is the Set of
Equivalence Classes of a particular Equivalence Relation.

For a Function between Sets $f : S \to T$, one may define an
Equivalence Relation on Elements $a,b \in S$:
\[
    a \sim b \Leftrightarrow f(a) = f(b)
\]
where the Equivalence Classes are the Fibers (\S\ref{sec:fiber}) of
the Elements in $T$ and the Quotient Set is the Image of the
Equivalence Relation viewed as a Function on Elements of $S$ to their
Equivalence Classes.

=

$\bumpeq$

$\sim$

$\simeq$

$\approx$

$\approxeq$

$\cong$

$\equiv$

Inequation



\subsubsection{Equivalence Problem}\label{sec:equivalence_problem}

\emph{Equivalence Problem}: given two objects, determine if they are Equivalent

\fist cf. \emph{Classification Problem} (\S\ref{sec:classification_problem}):
what are the ``objects'' of a given ``type'' up to some Equivalence?; a
Computable Complete Set of Invariants (\S\ref{sec:invariant}) solves the
Classification Problem and the Equivalence Problem



\subsubsection{Equivalence Class}\label{sec:equivalence_class}

(or \emph{Quotient})



\paragraph{Canonical Form}\label{sec:canonical_form}\hfill

a Canonical Form solves a Classification Problem (\S\ref{sec:canonical_form})
and also provides a distinguished (``Canonical'') Element of each Class



\subsubsection{Quotient Set}\label{sec:quotient_set}

The \emph{Quotient Set} is the Set of Equivalence Classes of a
particular Equivalence Relation.



\subsubsection{Partial Equivalence}\label{sec:partial_equivalence}

\subsubsection{Congruence Relation}\label{sec:congruence_relation}

Equivalence Relation on an Algebraic Structure
(\S\ref{sec:algebraic_structure}) compatible with that Structure

there is a Bijection between the Congruence Relations on a Ring and the Ideals
(\S\ref{sec:ring_ideal}) of a Ring



% ------------------------------------------------------------------------------
\subsection{Relation Join}\label{sec:relation_join}
% ------------------------------------------------------------------------------

\emph{Relational Algebra} (\S\ref{sec:relational_algebra})

\emph{Pullback} (\S\ref{sec:pullback})



\subsubsection{Relation Composition}\label{sec:relation_composition}

For two Binary Relations, $R \subseteq X \times Y$ and $S \subseteq Y
\times Z$, the \emph{Relation Composition}, $S \circ R \in X \times Z$
is defined as:
\[
  S \circ R = \{(x,z) \in X \times Z \;|\;
  \exists y \in Y : (x,y) \in R \wedge (y,z) \in S \}
\]

See also Function Composition (\S\ref{sec:function_composition})



% ------------------------------------------------------------------------------
\subsection{Closure}\label{sec:closure}
% ------------------------------------------------------------------------------

A Set, $A$, is \emph{Closed} (has \emph{Closure}) under a Relation,
$L$, if for every $(x,y) \in L$, $y \in dom(L)$:
\[
  (\forall x \in A) (x,y) \in L \to y \in A
\]



\subsubsection{Transitive Closure}\label{sec:transitive_closure}

For a Relation, $R$, on a Set, $X$, the \emph{Transitive Closure}
$R^+$ is a Relation on $X$ such that $R \subseteq R^+$ and $R^+$ is
\emph{Minimal} (the smallest Relation closed under Relation
Composition):
\[
  R^+ = \bigcup_{i \in \{1,2,3,...\}} R^i
\]
If $R$ is Transitive then $R = R^+$.

\emph{Reachability} (\S\ref{sec:dag})

$R_0 \subseteq R \to R_0^+ \subseteq R^+$, i.e. $()^+$ is
Monotone \cite{aczel88}



\paragraph{Transitive Reduction}\label{sec:transitive_reduction}\hfill



\subsubsection{Reflexive Closure}\label{sec:reflexive_closure}

The \emph{Reflexive Closure}, $X$, of a Binary Relation, $R$, on a
Set, $S$, is the smallest Reflexive Relation on $S$ that contains $R$:
\[
  X = R \cup \{(x,x) : x \in S\}
\]
that is, the Union of $R$ with the Equivalence Relation $=$.

The Reflexive Closure of $<$ is $\leq$.



\paragraph{Reflexive Reduction}\label{sec:reflexive_reduction}\hfill

The \emph{Reflexive Reduction} (or \emph{Irreflexive Kernel}) of a
Binary Relation $R$ on a Set $S$ is the smallest Relation $Y$ such
that $Y$ has the same Reflexive Closure as $R$:
\[
  Y = (R\;\setminus=)
\]
The Reflexive Reduction of $x \leq y$ is $x < y$.



\subsubsection{Symmetric Closure}\label{sec:symmetric_closure}

The \emph{Symmetric Closure}, $X$, of a Binary Relation, $R$, on a
Set, $S$, is the Union of $R$ with its Inverse Relation:
\[
  R \cup R^{-1} = X = R \cup \{(x,y) : (y,x) \in R\}
\]



\subsubsection{Finitary Closure}\label{sec:finitary_closure}

\subsubsection{Rewrite Closure}\label{sec:rewrite_closure}

\subsubsection{$P$-closure}\label{sec:p_closure}



% ==============================================================================
\section{Function}\label{sec:set_function}
% ==============================================================================

A \emph{Function} is a Left-total (\S\ref{sec:left_total})
Right-unique (Functional \S\ref{sec:functional_relation}) Relation. A
Function $f$ with Domain $A$ and Codomain $B$ is denoted:
\[
  f : A \to B
\]
The Domain (or \emph{Input}) of $f$ is denoted $dom(f)$ and the
Codomain (or \emph{Ouptut}) as $cod(f)$.

In the case where $img(f) = cod(f)$, $f$ is known as a
\emph{Surjective Function} (\S\ref{sec:surjective_function}).

The Extension (\S\ref{sec:extension}) of a Function may be called a
\emph{Graph} of the Function. Equality of Functions is defined such
that Equal Functions have the same Output for a given Input (equal in
Extension).

An \emph{Empty Function} has the Empty Set as a Domain, defining a
Unique Function for each Set, $A$:
\[
  f_A : \varnothing \to A
\]

\emph{Finitary Function}

\emph{Infinitary Function}

\fist Cf. \emph{Morphism} (\S\ref{sec:morphism})

Functional Predicate (\S\ref{sec:functional_predicate})

Real-valued Function (\S\ref{sec:real_function})

$\to$

Injection, Monomorphism: $\rightarrowtail$

Surjection, Epimorphism: $\twoheadrightarrow$

Bijection: $\rightarrowtail \hspace{-8pt} \twoheadrightarrow$

Isomorphism: $\xrightarrow{\sim}$

Inclusion Map: $\hookrightarrow$

Partial Function: $\nrightarrow$

Multimap: $\multimap$

$\leftrightarrow$

$\nleftrightarrow$

$\rightrightarrows$

$\leftrightarrows$

$\leftrightharpoons$

$\curvearrowright$

$\circlearrowright$

$\dashrightarrow$

$\looparrowright$

$\rightarrowtriangle$

$\rightsquigarrow$

$\to$

$\Rrightarrow$



% ------------------------------------------------------------------------------
\subsection{Independent Variable}\label{sec:independent_variable}
% ------------------------------------------------------------------------------

for a Set Function $f : X \to Y$, a Symbol $x$ representing an Element
of $X$ may be called an \emph{Independent Variable}, i.e. $x$ stands for an
\emph{arbitrary Input}

Functional Dependency %FIXME: general relations ?

\begin{itemize}
  \item Multivariable Calculus (\S\ref{sec:multivariable_calculus}): studies
    Functions of several Independent Variables
  \item Statistical Classification (\S\ref{sec:classification}): Properties of
    Observations are Independent Variables called \emph{Explanatory Variables}
\end{itemize}



% ------------------------------------------------------------------------------
\subsection{Dependent Variable}\label{sec:dependent_variable}
% ------------------------------------------------------------------------------

for a Set Function $f : X \to Y$, a Symbol $y$ representing an Element
of $Y$ may be called an \emph{Dependent Variable}, i.e. $y$ stands for an
\emph{arbitrary Output}

Functional Dependency %FIXME: general relations ?

\begin{itemize}
  \item Vector Calculus (\S\ref{sec:vector_calculus}) studies
    Functions returning several Dependent Variables
\end{itemize}



% ------------------------------------------------------------------------------
\subsection{Image}\label{sec:image}
% ------------------------------------------------------------------------------

The \emph{Image} of the Function is the Subset of the Codomain that
the Function actually Maps to, and the Subset of the Domain that Maps
to it is called the \emph{Inverse Image} or \emph{Preimage}
(\S\ref{sec:preimage}).

Because a Function is Right-unique, each Element of the Preimage Maps
to one Element of the Image. This Property is expressed as:
\[
  (a,b) \in f \wedge (a,c) \in f \to b = c
\]
A Left-total Relation without this Property is known as a
\emph{Multimap} (\S\ref{sec:multimap}).



% ------------------------------------------------------------------------------
\subsection{Preimage}\label{sec:preimage}
% ------------------------------------------------------------------------------

\emph{Preimage} (or \emph{Inverse Image}) of a Subset



\subsubsection{Fiber}\label{sec:fiber}

For a Function $f : A \to B$, the \emph{Fiber} of an Element
$y \in img(f)$ is the Preimage of the Singleton Set $\{y\}$.

Fiber (Topology \S\ref{sec:point_fiber})

Fiber Bundle (\S\ref{sec:fiber_bundle})

a Level Set (\S\ref{sec:level_set}) of a Real Valued Function is a special case
of a Fiber

examples:
\begin{itemize}
  \item $T_x M$ -- the Tangent Space of Smooth Manifold
    (\S\ref{sec:smooth_manifold}) $M$ at $x$, i.e. the Fiber (\S\ref{sec:fiber})
    of the Tangent Bundle (\S\ref{sec:tangent_bundle}) $T M$ over $x$
  \item ... TODO: more?
\end{itemize}



% ------------------------------------------------------------------------------
\subsection{Restriction}\label{sec:function_restriction}
% ------------------------------------------------------------------------------

$f|_A$ smaller Domain $A$

cf. Presheaves (\S\ref{sec:presheaf})



% ------------------------------------------------------------------------------
\subsection{Extension}\label{sec:function_extension}
% ------------------------------------------------------------------------------

$f$ is the Extension of the Restriction $f|_A$



\subsubsection{Overriding Union}\label{sec:overriding_union}

Overriding $f : X \to Y$ by $g : W \to Y$:
\[
  f \oplus g : (X \cup W) \to Y
\]

Union of $g$ and $f|_{X/W}$



% ------------------------------------------------------------------------------
\subsection{Additivity}\label{sec:additivity}
% ------------------------------------------------------------------------------

\emph{Finite Additivity}

cf. Additive Functions (Arithmetic \ref{sec:additive_function})

weaker than \emph{$\sigma$-additivity} (Countable Additivity
\S\ref{sec:sigma_additivity}): $\sigma$-additivity implies Additivity



% ------------------------------------------------------------------------------
\subsection{Identity Function}\label{sec:identity_function}
% ------------------------------------------------------------------------------

$f(x) = x$

$a \mapsto a$



\subsubsection{Inclusion Map}\label{sec:inclusion_map}

An \emph{Inclusion Map} (also \emph{Inclusion Function} or \emph{Insertion Map})
is an Identity Function that Maps Elements of a Subset to those in a Superset:
\[
  \iota : A \hookrightarrow X
\]
where $A \subseteq X$.



% ------------------------------------------------------------------------------
\subsection{Constant Function}\label{sec:constant_function}
% ------------------------------------------------------------------------------

\subsubsection{Locally Constant Function}\label{sec:locally_constant}

\fist de Rham Cohomology (\S\ref{sec:derham_complex}) on Smooth Manifold $M$ --
the Homology Group of Dimension Zero is Isomorphic to the Vector Space of
Locally Constant Functions from $M$ to $\reals$; for a Compact Manifold this is
the Real Vector Space with Dimension equal to the number of Connected Components
of $M$



% ------------------------------------------------------------------------------
\subsection{Injective Function}\label{sec:injective_function}
% ------------------------------------------------------------------------------

An \emph{Injective Function} (or \emph{One-to-one Function} or \emph{Injection})
is one where the Elements of the Codomain are the Images of at most one Elements
of the Domain. A Function that is Non-injective is considered an
\emph{Information Losing Function} because the Inverse is no longer a Function
but it is a \emph{Multimap} (\S\ref{sec:multimap}).

\textbf{Cantor-Schr\"oder-Bernstein Theorem}: \emph{If there exist Injective
  Functions $f : A \to B$ and $g : B \to A$ between Sets $A$ and
  $B$, there exists a Bijective Function $h : A \to B$.}

existence of a Bijection implies that if $|A| \leq |B|$ and $|B| \leq |A|$, then
$|A| = |B|$, i.e. $A$ and $B$ are \emph{Equipotent} (Equinumerous
\S\ref{sec:equipotent}) or of the same \emph{Cardinality}
(\S\ref{sec:cardinality})



% ------------------------------------------------------------------------------
\subsection{Surjective Function}\label{sec:surjective_function}
% ------------------------------------------------------------------------------

A Function $f$ with $img(f) = cod(f)$ is a \emph{Surjective Function}
(or \emph{Surjection}). Such a Function may be said to be \emph{Onto}
$cod(f)$.



% ------------------------------------------------------------------------------
\subsection{Bijective Function}\label{sec:bijective_function}
% ------------------------------------------------------------------------------

A Function that is both Surjective and Injective is a \emph{Bijective Function}
(or \emph{Bijection}). A Function is Bijective if and only if it is also
\emph{Invertible} (\S\ref{sec:inverse_function}).

A Bijective (Set Isomorphism \S\ref{sec:isomorphism}) Transformation (Set
Endomorphism \S\ref{sec:endomorphism}) of a Set is called a \emph{Permutation}
(\S\ref{sec:permutation}). The Group of all Permutations of a Set $X$ is called
the \emph{Symmetric Group} (\S\ref{sec:symmetric_group}), $Sym(X)$, of $X$

Two Sets or Classes for which a Bijection exists are said to be
\emph{Equinumerous} (\emph{Equipotent} \S\ref{sec:equipotent}) and have the same
Cardinality (\S\ref{sec:cardinality}).

\fist \emph{Counting} (\S\ref{sec:counting}) -- a means of establishing a
Bijection (One-to-one correspondence \S\ref{sec:bijective_function}) between the
Subset of Natural Numbers (\S\ref{sec:natural_number}) $\{1,2,\ldots,n\}$ and a
Set of Cardinality $n$; a Set for which no such Bijection exists for any Finite
$n$ is called an \emph{Infinite Set}, otherwise it is a Finite Set;
cf. Countable (\S\ref{sec:countably_infinite}), Uncountable
(\S\ref{sec:uncountably_infinite})

\fist Bijective Proof (Combinatorial Proof \S\ref{sec:bijective_proof})

\textbf{Cantor-Schr\"oder-Bernstein Theorem}: \emph{If there exist Injective
  Functions $f : A \to B$ and $g : B \to A$ between Sets $A$ and
  $B$, there exists a Bijective Function $h : A \to B$.}



\subsubsection{Inverse Function}\label{sec:inverse_function}

$f^{-1}$

to be Invertible, a Function must be Bijective

of Functors: Weak Inverse (\S\ref{sec:weak_inverse})

Locally Invertible (\S\ref{sec:locally_invertible})



\paragraph{Left Inverse}\label{sec:left_inverse}\hfill

or \emph{Pre-inverse}

For a Function $f: X \to Y$, a \emph{Left Inverse} or
\emph{Retraction} of $f$ is a Function $g: Y \to X$ such that
$gf = Id(X)$.



\paragraph{Right Inverse}\label{sec:right_inverse}\hfill

or \emph{Post-inverse}

For a Function $f: X \to Y$, a \emph{Right Inverse} or
\emph{Section} of $f$ is a Function $h: Y \to X$ such that $fh
= Id(Y)$.

Projection (\S\ref{sec:projection})



\paragraph{Involutory Function}\label{sec:involution}\hfill

An \emph{Involution} is a Function thta is its own Inverse:
\[
  f(f(x)) = x
\]

An Identity Map is a trivial Involution.

\fist \emph{Mathematical Duality} (\S\ref{sec:duality}): \emph{Translation} of
``Concepts, Theorems, or Mathematical Structures'' into other Concepts,
Theorems, or Mathematical Structures in a one-to-one fashion, usually by means
of an Involution.



% ------------------------------------------------------------------------------
\subsection{Function Composition}\label{sec:function_composition}
% ------------------------------------------------------------------------------

Given two Functins $f : A \to B$ and $g : B \to C$,
there is a \emph{Composite Function}:
\[
  g \circ f : A \to C
\]
where $(g \circ f)(a) = g(f(a))$ and $a \in A$.

The \emph{Composition Operation} $\circ$ is Associative: $(h \circ g)
\circ f = h \circ (g \circ f)$. Composition of Functions may be
represented with the $\circ$ elided: $gf$.

Identity Element for the Composition Operation is the Identity
Function (\S\ref{sec:identity_function}) on Sets. The Identity
Function for a Set $A$:
\[
  I_A : A \to A
\]
is defined as:
\[
  I_A(a) = a
\]
with the result given the Function $f$ above:
\[
  f \circ I_A = f = I_B \circ f
\]



% ------------------------------------------------------------------------------
\subsection{Pointwise Product}\label{sec:pointwise_product}
% ------------------------------------------------------------------------------

for Functions $f, g : X \to Y$, the \emph{Pointwise Product}
$(f\cdot{g}) : X \to Y$ is defined as:
\[
  (f \cdot g)(x) = f(x) \cdot g(x)
\]

sometimes written $fg$

the Product Rule (\S\ref{sec:product_rule}) of Differentiation: for
Differentiable Functions $f, g : X \to \reals$, the \emph{Derivative}
(\S\ref{sec:derivative}) of their Pointwise Product $fg$ is given by:
\[
  \diffy{(fg)} = (\diffy{f})g + f(\diffy{g})
\]



% ------------------------------------------------------------------------------
\subsection{Kernel}\label{sec:function_kernel}
% ------------------------------------------------------------------------------

The \emph{Kernel} of a Function $f : X \to Y$, $ker(f)$, is an
Equivalence Relation defined as:
\[
  ker(f) = \{ (x,x') \in X \times X : f(x) = f(x') \}
\]
i.e. the Set of all Pairs of Elements that Map to the same Value



% ------------------------------------------------------------------------------
\subsection{Equalizer}\label{sec:function_equalizer}
% ------------------------------------------------------------------------------

Given two Sets $X,Y$ and Functions $f,g : X \to Y$, the
\emph{Equalizer} of $f$ and $g$ is defined as:
\[
  Eq(f,g) = { x \in X | f(x) = g(x) }
\]
i.e. the Set of Arguments at which two (or more) Functions have Equal
Values.


\emph{Coequalizer}



\subsubsection{Difference Kernel}\label{sec:difference_equalizer}

Equalizer of exactly two Functions



% ------------------------------------------------------------------------------
\subsection{Fixed Point}\label{sec:fixed_point}
% ------------------------------------------------------------------------------

A \emph{Fixed Point}, $c$, of a Function $f$, is an Element of the Domain of $f$
that is Mapped to itself by $f$:
\[
  f(c) = c
\]

Fixed-point Combinator (\S\ref{sec:fixedpoint_combinator})

Prefixpoint (\S\ref{sec:prefixpoint})

Least Fixed Point (\S\ref{sec:least_fixedpoint})

Greatest Fixed Point (\S\ref{sec:greatest_fixedpoint})

Postfixpoint (\S\ref{sec:postfixpoint})

\fist Fixed-point Iteration (\S\ref{sec:fixedpoint_iteration}) -- method of
computing Fixed Points of Iterated Functions (\S\ref{sec:iterated_function})

cf. Attractor (\S\ref{sec:attractor_repeller}) of an Iterated Function System
(\S\ref{sec:ifs})



\subsubsection{Least Fixed-point}\label{sec:least_fixedpoint}

Prefixpoint (\S\ref{sec:prefixpoint}) in a Partial Order

The Initial Algebra (\S\ref{sec:initial_algebra}) of an $F$-algebra
(\S\ref{sec:f_algebra}) is a Least Fixed-point given by the Type-level
Fixed-point Combinator:
\[
  Fix (F) = Fx(F (Fix (F)))
\]
where $Fx$ gives the Evaluator of the Initial Algebra.



\subsubsection{Greatest Fixed-point}\label{sec:greatest_fixedpoint}

Postfixpoint (\S\ref{sec:postfixpoint}) in a Partial Order

Terminal Coalgebra (\S\ref{sec:terminal_coalgebra})



\subsubsection{Periodic Point}\label{sec:periodic_point}

A \emph{Periodic Point} is an Element of the Domain of a Function that
is returned to after a finite number of iterations.



\subsubsection{Idempotent Function}\label{sec:idempotent}

A Function, $f$, is \emph{Idempotent} if it maps each Element of
$dom(f)$ to a Fixed Point of $f$:
\[
  f^2 = f
\]



\paragraph{Projection}\label{sec:projection}\hfill

Idempotent mapping of a Set (or Structure) into a Subset (or Sub-structure), or
generally a mapping which has a Right Inverse

\begin{itemize}
  \item Linear Projection (\S\ref{sec:linear_projection}) -- an Idempotent
    Linear Operator
  \item Projection (Category Theory \S\ref{sec:projection_functor})
  \item Subspace Retraction (Topology \S\ref{sec:subspace_retraction})
  \item Covering Projection (Topology \S\ref{sec:covering_map})
\end{itemize}



\subsubsection{Fixed-point Theorem}\label{sec:fixedpoint_theorem}

cf. Topological Degree Theory (\S\ref{sec:degree_theory})



% ------------------------------------------------------------------------------
\subsection{Symmetric Function}\label{sec:symmetric_function}
% ------------------------------------------------------------------------------

a Function that is equivalent under re-ordering of its arguments



% ------------------------------------------------------------------------------
\subsection{Function Space}\label{sec:function_space}
% ------------------------------------------------------------------------------

The \emph{Function Space} of two Sets $A$ and $B$ is the Set of all Functions
from $A$ to $B$ denoted by $B^A$.

When $B$ is a Field (\S\ref{sec:field}), Functions have a Vector
(\S\ref{sec:vector}) structure with two Pointwise Addition Operators and Scalar
Multiplication. %FIXME

Bijective Functions $A \leftrightarrow B$

\begin{itemize}
  \item Sequence Space (\S\ref{sec:sequence_space}) -- Function Space whose
    Elements are Functions from $\nats$ to the Field $K$ of Real or Complex
    Numbers, i.e. all possible Infinite Sequences with Elements in $K$
  \item Hilbert Space (\S\ref{sec:hilbert_space})

    \fist Wave Function (\S\ref{sec:wave_function}) -- Complex-valued
    Probability Amplitude (\S\ref{sec:probability_amplitude}) representative of
    a Quantum State (\S\ref{sec:quantum_state}) as an Element of a Hilbert Space
    that is a Function Space
  \item Lebesgue Space ($L^p$ Space \S\ref{sec:lp_space})
  \item ...
\end{itemize}

\fist Functionals (Functional Analysis \S\ref{sec:functional}) -- mapping from a
Function Space into $\reals$ or sometimes $\comps$

\fist Functional Integration (\S\ref{sec:functional_integration}) -- Integration
over the Domain of a Function Space

\fist Orthogonal Functions (\S\ref{sec:orthogonal_function}) -- belongs to
Function Spaces that are Vector Spaces with a Bilinear Form
(\S\ref{sec:bilinear_form})

\fist Operator Theory (\S\ref{sec:operator_theory}): study of Linear Operators
(\S\ref{sec:linear_operator}) on Function Spaces



\subsubsection{Evaluation Function}\label{sec:evaluation_function}

Given a Function Space $B^A$, the \emph{Evaluation Function} is
defined as:
\[
  eval : B^A \times A \to B
\]

\fist See also Evaluation Strategies ($\lambda$-calculus
\S\ref{sec:evaluation_strategy})



\subsubsection{Basis Function}\label{sec:basis_function}

Approximation Theory (\S\ref{sec:approximation_theory})

cf. Interaction (Multivariate Statistics \S\ref{sec:interaction}), Regression
Analysis (\S\ref{sec:regression_analysis})



\subsubsection{Lebesgue Space}\label{sec:lebesgue_space}

or \emph{$L^p$ Space}

generalization of $p$-norm for Finite-dimensional Vector Spaces



% ------------------------------------------------------------------------------
\subsection{Binary Operation}\label{sec:binary_operation}
% ------------------------------------------------------------------------------

Domains and Codomain Subsets of the same Set

$f : S \times S \to S$



\subsubsection{Commutator}\label{sec:commutator}

Elements $a$ and $b$:
\[
  [a,b] = aba^{-1}b^{-1}
\]
or:
\[
  [a,b] = a^{-1}b^{-1}ab
\]

Group Commutator (\S\ref{sec:group_commutator})



\subsubsection{Zero Element}\label{sec:zero_element}

(or \emph{Absorbing Element} or \emph{Annihilating Element})



% ------------------------------------------------------------------------------
\subsection{Boolean-valued Function}\label{sec:boolean_function}
% ------------------------------------------------------------------------------

Predicate (\S\ref{sec:predicate})

Indicator Function (\S\ref{sec:indicator_function})

Truth-valued Function (\S\ref{sec:truth_function})



% ------------------------------------------------------------------------------
\subsection{Transformation}\label{sec:transformation}
% ------------------------------------------------------------------------------

a Function $f : X \to X$ that maps a Set $X$ to itself, i.e. a Set Endomorphism
(\S\ref{sec:endomorphism})

\fist Full Transformation Monoid (\S\ref{sec:endomorphism_monoid})

a Bijective Transformation (Set Automorphism \S\ref{sec:automorphism}) is called
a \emph{Permutation} (\S\ref{sec:permutation})

cf. a \emph{Geometric Transformation} (\S\ref{sec:geometric_transformation}) is
defined as a Bijection of a Set with a ``Geometric underpinning'' (wiki)

examples:

\begin{itemize}
  \item Linear Transformations (Linear Maps \S\ref{sec:linear_transformation})
  \item Affine Transformations (\S\ref{sec:affine_transformation}) --
    Translations (\S\ref{sec:translation})
  \item Rotations (\S\ref{sec:rotation}), Reflections (\S\ref{sec:reflection}),
  \item Integral Transform (\S\ref{sec:integral_transform})
\end{itemize}



\subsubsection{Invariant}\label{sec:invariant}

(wiki):

an \emph{Invariant} is a Property (\S\ref{sec:property}) held by a Class
of ``objects'' which remains unchanged under Transformations of a certain type

Equality of Invariants of objects is a \emph{necessary condition} for
Equivalence of objects

a ``\emph{Complete Set of Invariants}'' (Classification Problems
\S\ref{sec:classification_problem}) is a Set of Invariants such that Equivalence
of the Invariants in the Set is a \emph{sufficient condition} for Equivalence of
objects

in terms of Group Actions (\S\ref{sec:group_action}): Invariants are Functions
of \emph{Coinvariants} (\S\ref{sec:coinvariant}), i.e. Equivalence Classes or
``Orbits'' (\S\ref{sec:orbit}), and a Complete Set of Invariants Characterizes
the Coinvariants, i.e. it is a Set of Defining Equations for the Coinvariants

cf. Invariant Theory (\S\ref{sec:invariant_theory})



\subsubsection{Coinvariant}\label{sec:coinvariant}

(wiki):

Group Actions (\S\ref{sec:group_action}): \emph{Invariants}
(\S\ref{sec:invariant}) are Functions of Coinvariants, i.e. Equivalence Classes
or ``Orbits'' (\S\ref{sec:orbit}), and a ``\emph{Complete Set of Invariants}''
(\S\ref{sec:classification_problem}) Characterizes the Coinvariants, i.e. it is
a Set of Defining Equations for the Coinvariants



% ------------------------------------------------------------------------------
\subsection{Bundle}\label{sec:bundle}
% ------------------------------------------------------------------------------

generalization of Fiber Bundle (Topology \S\ref{sec:fiber_bundle}) dropping the
condition of a Local Product Structure

a Bundle Object (\S\ref{sec:bundle_object}) can be defined in any Category
$\cat{C}$ as an Epimorphism (\S\ref{sec:epimorphism}) $\pi : E \to B$

e.g. the Moebius Strip is a Bundle but not a Product Manifold

$E \xrightarrow{\pi} M$

if every Preimage of $\pi$ is Homeomorphic to the same Manifold $F$, then $E
\xrightarrow{\pi} M$ is a \emph{Fiber Bundle} with \emph{Typical Fiber} $F$

Physics: a ``Wave Function'' is a Section of the $\comps$-line Bundle over
``Physical Space'' (e.g. $\reals^3$)

Product Bundles $\subset$ Fiber Bundles $\subset$ Bundles

Vector Bundle (\S\ref{sec:vector_bundle})

%FIXME: cleanup



\subsubsection{Cross Section}\label{sec:cross_section}

or \emph{Section} of a Bundle

a Map $\sigma : M \to E$ is a \emph{Section} or \emph{Cross-section} of
the Bundle if $\pi \circ \sigma = 1_M$, i.e. $\sigma$ always maps Points in the
Base Space to the Fiber over that Point

for a Product Bundle, the Sections are just Functions

Physics: a ``Wave Function'' is a Section of the $\comps$-line Bundle over
``Physical Space'' (e.g. $\reals^3$)



\subsubsection{Trivial Bundle}\label{sec:trivial_bundle}

a Bundle or Fiber Bundle is \emph{Trivial} if it is Isomorophic to the Cross
Product of the Base Space and a \emph{Fiber}

i.e. a Fiber Bundle $(E, B, \pi, F)$ is Trivial if the Total Space $E$ is just
the Cross Product of the Base Space and the Fiber, $B \times F$, and $\pi$ is
the natural Projection onto the First Factor, $B \times F \to B$ where
$E = B \times F$

examples of Non-trivial Bundles are Mo\"ebius Strips and Klein Bottles



\subsubsection{Dual Bundle}\label{sec:dual_bundle}

the Cotagent Bundle (\S\ref{sec:cotangent_bundle}) of a Differentiable Manifold
is the Dual Bundle to the Tangent Bundle (\S\ref{sec:tangent_bundle}) of that
Manifold

\fist Dual Vector Bundle (\S\ref{sec:dual_vectorbundle})



\subsubsection{Jet Bundle}\label{sec:jet_bundle}

\url{https://www.physicsforums.com/insights/higher-prequantum-geometry-ii-principle-extremal-action-comonadically/}

\fist Jet Comonad (\S\ref{sec:jet_comonad})



% ------------------------------------------------------------------------------
\subsection{Subadditive Set Function}\label{sec:subadditive_set_function}
% ------------------------------------------------------------------------------

cf. Subadditive Function (Arithmetic \S\ref{sec:subadditive_function})



% ------------------------------------------------------------------------------
\subsection{Supermodular Set Function}\label{sec:supermodular_set_function}
% ------------------------------------------------------------------------------

\fist Supermodular Real-valued Functions (\S\ref{sec:supermodular_function})

as a Utility Function (\S\ref{sec:utility_function}) -- Complementary Goods



% ------------------------------------------------------------------------------
\subsection{Submodular Set Function}\label{sec:submodular_set_function}
% ------------------------------------------------------------------------------

\fist Submodular Real-valued Functions (\S\ref{sec:submodular_function})



% ==============================================================================
\section{Partial Function}\label{sec:partial_function}
% ==============================================================================

A \emph{Partial Function} is a Functional Relation
(\S\ref{sec:functional_relation}) that is not Left-total
(\S\ref{sec:left_total}).

$f : A \nrightarrow B$ or $f : A \rightharpoondown B$




% ------------------------------------------------------------------------------
\subsection{Rice's Theorem}\label{sec:rices_theorem}
% ------------------------------------------------------------------------------

``For any non-trivial Property of Partial Functions, no General and
Effective method can Decide whether an Algorithm computes a Partial
Function with that Property.''



% ==============================================================================
\section{Multimap}\label{sec:multimap}
% ==============================================================================

A \emph{Multimap} (or \emph{Multi-valued Function}) is a Left-total
Relation (\S\ref{sec:left_total}) that is not Right-unique
(\S\ref{sec:functional_relation})

often arises as the Inverse of a Non-injective Function

$A \multimap B$

cf. Analytic Continuation (\S\ref{sec:analytic_continuation})



% ------------------------------------------------------------------------------
\subsection{Branch Point}\label{sec:branch_point}
% ------------------------------------------------------------------------------

%FIXME: move to complex analysis ???

Complex Analysis (\S\ref{sec:complex_analysis})



\subsubsection{Branch Cut}\label{sec:branch_cut}

\subsubsection{Principal Branch}\label{sec:principal_branch}



% ==============================================================================
\section{Axiomatic Set Theory}\label{sec:axiomatic_set_theory}
% ==============================================================================

% ------------------------------------------------------------------------------
\subsection{Axiom of Foundation}\label{sec:foundation_axiom}
% ------------------------------------------------------------------------------

or \emph{Axiom of Regularity}

every Non-empty Set $A$ contains an Element that is Disjoint from $A$:
\[
  \forall x, x \neq \varnothing \to \exists y \in x \mid y
  \cap x = \varnothing
\]

every Set is Isomorphic to a Well-founded Set
(\S\ref{sec:wellfounded_set})

in Non-well-founded Set Theory (\S\ref{sec:non_wellfounded}) the Axiom
of Foundation is replaced by Axioms Implying its Negation



% ------------------------------------------------------------------------------
\subsection{Axiom of Infinity}\label{sec:infinity_axiom}
% ------------------------------------------------------------------------------

% ------------------------------------------------------------------------------
\subsection{Axiom of Finiteness}\label{sec:finiteness_axiom}
% ------------------------------------------------------------------------------

% ------------------------------------------------------------------------------
\subsection{Axiom of Choice}\label{sec:choice_axiom}
% ------------------------------------------------------------------------------

...

\fist to allow for Uncountable Bases (\S\ref{sec:basis}), the Axiom of Choice is
required as it is equivalent to the statement that ``every Vector Space has a
Basis'' (\url{https://math.stackexchange.com/a/884332/239953})



\subsubsection{Tarski's Theorem}\label{sec:tarskis_theorem}

Well-ordering Theorem (\S\ref{sec:wellorder_theorem}): every Set can
be Well-ordered



% ------------------------------------------------------------------------------
\subsection{Axiom of Extensionality}\label{sec:extensionality_axiom}
% ------------------------------------------------------------------------------

\[
  \forall S \forall T
    [S = T \Leftrightarrow \forall R [ R \in S \Leftrightarrow R \in T ]]
\]
cf. \emph{Leibniz Law} (\S\ref{sec:equality}):
\[
  \forall S \forall T
    [S = T \Leftrightarrow \forall R [ S \in R \Leftrightarrow T \in R ]]
\]


% ------------------------------------------------------------------------------
\subsection{Axiom of Countable Choice}\label{sec:countable_choice}
% ------------------------------------------------------------------------------

% ------------------------------------------------------------------------------
\subsection{Zermelo-Fraenkel (ZFC)}\label{sec:zermelo_fraenkel}
% ------------------------------------------------------------------------------

<http://jdh.hamkins.org/wp-content/uploads/2016/06/Pluralism-inspired-mathematics.pdf>:

Thm. \emph{Every Model of ZFC has an Element that is, externally a
  Model of ZFC. Specifically, if $\langle{M,\in^M}\rangle \vDash ZFC$,
then there is $\langle{m,E}\rangle$ in $M$ which when extracted as an
actual Structure, Satisfies ZFC}

\emph{Solovay Model}: assuming the existence of an \emph{Inaccessible Cardinal}
(\S\ref{sec:inaccessible_cardinal}), a Model of ZF Set Theory without the Axiom
of Choice (but the Axiom of Countable Choice holds) in which all Sets of Real
Numbers are Lebesgue Measurable (\S\ref{sec:lebesgue_measure})

if the Axiom of Choice is accepted, according to the \emph{Vitali Theorem},
there are Uncountably many \emph{Vitali Sets} which are not Lebesgue Measurable



% ------------------------------------------------------------------------------
\subsection{Von Neumann-Bernays-G\"odel (NBG)}\label{sec:nbg_set_theory}
% ------------------------------------------------------------------------------

introduces notion of \emph{Class} (\S\ref{sec:class}) as a collection of Sets
defined by a Fomula with Quantifiers Ranging only over Sets

can define Classes ``larger'' than Sets, such as the Class of all Sets and the
Class of all Ordinals



\subsubsection{Morse-Kelley (MK)}\label{sec:mk_set_theory}



% ------------------------------------------------------------------------------
\subsection{Kripke-Platek (KP)}\label{sec:kripke_platek}
% ------------------------------------------------------------------------------

% ------------------------------------------------------------------------------
\subsection{New Foundations (NF)}\label{sec:quine_foundations}
% ------------------------------------------------------------------------------

% ------------------------------------------------------------------------------
\subsection{Non-well-founded Set Theory}\label{sec:non_wellfounded}
% ------------------------------------------------------------------------------

\cite{aczel88}

Foundation Axiom (\S\ref{sec:foundation_axiom}) replaced by Axioms
Implying its Negation (Anti-foundation \S\ref{sec:anti_foundation}).

Quine \emph{New Foundations}

Pure Set (\S\ref{sec:pure_set})

Non-standard Analysis (\S\ref{sec:nonstandard_analysis})

Modelling of Non-terminating Computational Processes
(\S\ref{sec:process}) in Process Calculus
(\S\ref{sec:process_calculus}), e.g. in
\cite{abramsky-gay-nagarajan96}

Bisimulation (\S\ref{sec:bisimulation}): Bisimilar Sets
(\S\ref{sec:bisimilar_object}) are considered Equal leading to a
strengthening of the Extensionality Axiom
(\S\ref{sec:extensionality_axiom}).

Situation Theory (\S\ref{sec:situation_theory})

Universe $\class{V}$

(Assumption) $\class{V} \cong \ords$ Class of Ordinals

$\class{V}$ Greatest Fixed Point (\S\ref{sec:greatest_fixedpoint}) of
$\pow$ is a System (\S\ref{sec:system}) such that for every System
$\class{M}$ there is a Unique System Map $\class{M} \to
\class{V}$

Set Continuous Operators (???): $\pow$, $Id$, $K_\class{A}$ (Constant
???) for Class $\class{A}$

Monotone + Set Based

Fixed Points

$J$ (\S\ref{sec:greatest_fixedpoint})

$I$ (\S\ref{sec:least_fixedpoint})

\emph{Standard Functor}: Set Continuous and Preserves Inclusion Maps,
e.g. $\pow$, $Id$, $K_\class{A}$

\textbf{Thm.}: If $\Phi$ is a Standard Functor then $I_\Phi$ is an
Initial Algebra (\S\ref{sec:initial_algebra})

\textbf{Thm.} Any Final Coalgebra (\S\ref{sec:terminal_coalgebra}) for
$\pow$ is a Model for the Anti-foundation Axiom

assuming AFA, the Largest Fixed Point $\class{V}$ of $\pow$ is a Final
Coalgebra

\textbf{Final Coalgebra Theorem} Any Standard Functor that Preserves
Weak Pullbacks has a Final Coalgebra

\textbf{Special Final Coalgebra Theorem} (Assuming AFA) For Standard
Functor $\Phi$ that is Uniform on Maps (???) then $J_\Phi$ is a Final
Coalgebra

Process Calculus (\S\ref{sec:process_calculus}) applications:

SCCS (\S\ref{sec:sccs}): Synchronization by Abelian Group $Act$ of
Atomic Actions

Abstract Behavior: Bisimulation Relation on a Labelled Transition
System (\S\ref{sec:labelled_transition})

Transition Systems Labelled by Elements of Set $Act$ are Coalgebras
relative to the Standard Functor $\pow(Act \times \cdots)$

when $Act$ is Singleton the Functor is Isomorphic to $\pow$ with Final
Coalgebras as Complete Systems (\S\ref{sec:complete_system}) used to
Model AFA

Model Construction for AFA is a special case of Quotient Construction
for SCCS

Sets in AFA-universe are the Abstract Behaviors for the special case
of SCCS where ther is only one Atomic Action



\subsubsection{Anti-foundation Axiom}\label{sec:anti_foundation}

\cite{aczel88}

AFA (Aczel) -- every Well-founded (\S\ref{sec:wellfounded_graph})
Accessible Pointed Graph (\S\ref{sec:accessible_pointed}) corresponds
to (the Membership Structure of, i.e. is a \emph{Picture} of) a Unique
Set; Maximal Bisimulation Relation (Strong Congruence in SCCS
\S\ref{sec:sccs})

SAFA (Scott) -- Sets are Equal if Isomorphic

FAFA (Finsler) -- Sets are Equal if Isomorphic (different sense)

BAFA (Boffa) -- Axiom of Superuniversality
(\S\ref{sec:superuniversality_axiom}): standard Extensionality as
Equality (Sets are Equal if they have the same Elements)

Extension of the Von Neumann Universe (VN):
\[
  VN \subseteq AFA \subseteq SAFA \subseteq FAFA \subseteq BAFA
\]

AFA$^\sim$ for a suitable Equality Relation $\sim$: Regular
Bisimulation (\S\ref{sec:bisimulation}) Relations

Maximal Bisimulation Relation: AFA

Isomorphism Relations, Regular Bisimulation: SAFA, FAFA

Extensionality: BAFA

\emph{Regular Bisimulation} %FIXME



\textbf{Anti-foundation Axiom} (\S\ref{sec:anti_foundation})
\cite{aczel88}: \emph{Every Graph has a Unique Decoration}

\emph{Decoration}: assignment of a Set to each Node of the Graph such
that the Sets assigned to the Children of that Node are assigned to
the Elements of the Set

\emph{Picture} of a Set: a Decoration of an Accessible Pointed Graph
with the Set assigned to the Point.

\emph{Finite Picture}: Hereditarily Finite Set, e.g. Quine Atom
(\S\ref{sec:quine_atom})

\emph{Mostowski's Collapsing Lemma}: Every Well-founded Graph
(\S\ref{sec:wellfounded_graph}) has a unique Decoration.

\textbf{Thm.} Any Final Coalgebra (\S\ref{sec:terminal_coalgebra}) for
$\pow$ is a Model for the Anti-foundation Axiom

assuming AFA, the Largest Fixed Point $\class{V}$ of $\pow$ is a Final
Coalgebra

\emph{Corollary}: Every Well-founded Accessible Pointed Graph is a
Picture of a unique Set.

\emph{Prop.}: Every Set has a Picture

\emph{Canonical Picture}

An Accessible Pointed Graph is an \emph{Exact Picture} if it has an
Injective Decoration, i.e. the Nodes are assigned distinct Sets by the
Decoration. Equivalently, an Accessible Pointed Graph is an Exact
Picture if it is Isomorphic to a Canonical Picture.

every Picture of a Set can be \emph{Unfolded} into a Tree Picture
(Rooted Tree \S\ref{sec:rooted_tree})

Unfolding the Canonical Picture: \emph{Canonical Tree Picture}

Consequence: Non-well-founded Sets exist and any Non-well-founded
Accessible Pointed Graph will Picture a Non-well-founded Set.

AFA is the Conjunction of:
\begin{itemize}
  \item $AFA_1$: Every Graph has at least one Decoration
  \item $AFA_2$: Every Graph has at most one Decoration
\end{itemize}

Foundation Axiom Implies $AFA_2$ and the Negation of $AFA_1$

For Sets $a,b$, $a \equiv b$ if and only if there is an Accessible
Pointed Graph that is a Picture of both $a$ and $b$

$AFA_2$ is Equivalent to:
\[
  a \equiv b \to a = b \;\text{for all Sets}\; a,b
\]

(Systems \S\ref{sec:system})

$\equiv$ is the (Maximum) Bisimulation (\S\ref{sec:bisimulation}) Relation on
Universe? $\class{V}$; see Systems (\S\ref{sec:system})
%FIXME define V

$AFA_2 \Leftrightarrow \class{V} \;\text{is Strongly Extensional}$

$AFA_2 \Leftrightarrow \;\text{Every Canonical Picture is Strongly
  Extensional}$

$AFA_2 \Leftrightarrow \;\text{Every Exact Picture is Strongly
  Extensional}$

$AFA_1 \Leftrightarrow \;\text{Every Strongly Extensional Accessible
  Pointed Graph is an Exact Picture}$

\textbf{Thm.} -- AFA is equivalent to: An Accessible Pointed Graph is
an Exact Picture if and only if it is Strongly Extensional

Normal Structure Axiom (NSA) (\S\ref{sec:normal_structure_axiom})

Gordeev's Axiom (GA) (\S\ref{sec:gordeevs_axiom})

$AFA_1 \to NSA \to GA$

AFA holds if and only if $\class{V}$ is a Complete System
(\S\ref{sec:complete_system})



\paragraph{Labelled Anti-foundation Axiom}
\label{sec:labelled_antifoundation}\hfill

\cite{aczel88}

Every Labelled Graph has a Unique Labelled Decoration
% FIXME xrefs?

(Consequence of AFA)

\emph{Labelled Graph}: a Graph with an assignment of a Set
$a\downarrow$ of Labels to each Node $a$

\emph{Labelled Decoration}: assignment $d$ of a Set $d a$ to each Node
$a$ such that:
\[
  da = \{db | a \to b\} \cup a \downarrow
\]



\paragraph{Axiom of Superuniversality}
\label{sec:superuniversality_axiom}\hfill

BAFA

Superuniversal System (\S\ref{sec:superuniversal_system})

$BAFA \Leftrightarrow \class{V} \;\text{is Superuniversal}$



\paragraph{Normal Structure Axiom}
\label{sec:normal_structure_axiom}\hfill

Every Kanger Structure is Isomorphic to a Normal one

Kanger Structure

Normal Kanger Structure

$AFA_1 \to NSA \to GA$


\paragraph{Gordeev's Axiom}\label{sec:gordeevs_axiom}\hfill

Every Graph is Isomorphic to a Normal One

Normal Graph

$AFA_1 \to NSA \to GA$



\subsubsection{Pure Set}\label{sec:pure_set}

(or \emph{Hereditary Set})

can only have Sets as Elements (and those Sets are also Pure)
\cite{aczel88}

allowing for Atoms (Urelements \S\ref{sec:urelement}) and Sets built
from them results in an expanded Universe, analagous to the
construction of a Polynomial Ring (\S\ref{sec:polynomial_ring}) from a
Ring by adjoining Indeterminates (\S\ref{sec:indeterminate}) and
adding all the Polynomials in those Indeterminates with Coefficients
taken from the Ring.

for $X$ Class of Atoms, the Sets that may involve Atoms from the Class
$X$ are called \emph{$X$-sets}

Substitution Lemma, Solution Lemma



\paragraph{Quine Atom}\label{sec:quine_atom}\hfill

$Q = \{Q\}$

$Q = \{\{\{\cdots\}\}\}$

Hereditarily Finite (has a Finite Picture):
\[
  \begin{tikzpicture}
    \node (A) {$\bullet$};
    \draw[->,thick,loop] (A) to (A);
  \end{tikzpicture}
\]

Urelement (\S\ref{sec:urelement}): a Set that contains only itself as
an Element

as a Pointed Graph it is the Graph consisting of a single Vertex with
a Loop

Unfolds to the Infinite Tree $\bullet \to \bullet
\to \bullet \to \cdots$

An Accessible Pointed Graph is a Picture of $Q$ if and only if every
Node has a Child. \cite{aczel88}

AFA: Quine Atom exists and is Unique

BAFA: distinct Quine Atoms exist and form a Proper Class

Reflexive Set (???): $x \in x$



\subsubsection{System}\label{sec:system}
\cite{aczel88}

\emph{System} is a Class $\class{M}$ of Nodes with a Class of Edges
consisting of Ordered Pairs of Nodes.

required that for each Node $a$, the Class $a_\class{M} = \{b \in
\class{M} | a \to b\}$ of Children of $a$ is a Set

a Graph is a Small System

Example Large System: Universe $\class{V}$ with $a \to b$
whenever $b \in a$

Systems are Coalgebras (\S\ref{sec:coalgebra}) for the Functor $\pow$

(Assumption) $\class{V} \cong \ords$ Class of Ordinals

$\class{M} a$ -- Accessible Pointed Graph for $a \in \class{M}$ with
Nodes and Edges as the Nodes and Edges of $\class{M}$ that lie on
Paths of $\class{M}$ starting from Node $a$ and the Point of
$\class{M} a$ is $a$

Unique Decoration for each $a \in \class{M}$:
\begin{align*}
  d a &= d_a a \\
      &= \{d_a x | a \to x \in \class{M} a\} \\
      &= \{d x | a \to x \in \class{M}\}
\end{align*}

\emph{Labelled System}

$a \downarrow \class{M}$ -- Set of Labels at $a$ in the Labelled
System $\class{M}$

\textbf{Thm.} (assuming AFA): Each Labelled System has a Unique
Labelled Decoration

\emph{Bisimulation} -- Binary Relation $R$ on System $\class{M}$ is a
Bisimulation on $\class{M}$ if $R \subseteq R^+$ ($R^+$ Transitive
Closure of $R$) where for $a,b \in \class{M}$:
\[
  a R^+ b \Leftrightarrow \forall x \in a_\class{M}
    \exists y \in b_\class{M}, x R y \wedge
    \forall y \in b_\class{M} \exists x \in a_\class{M}, x R y
\]

\emph{Maximum Bisimulation} (or \emph{Largest} or \emph{Weakest
  Bisimulation}) -- $\equiv_\class{M}$ for System $\class{M}$ such
that if $R$ is a Bisimulation on $\class{M}$ then for all $a,b \in
\class{M}$:
\[
  a R b \to a \equiv_\class{M} b
\]
and:
\[
  a \equiv_\class{M} b \Leftrightarrow a R b
\]
for some Small Bisimulation $R$ on $\class{M}$.
%FIXME small bisimulation

\textbf{Thm.}: There is a Unique Maximum Bisimulation
$\equiv_\class{M}$ on each System $\class{M}$

$\equiv$ -- for Sets $a,b$, $a \equiv b$ if and only if there is an
Accessible Pointed Graph that is a Picture of both $a$ and $b$

$\equiv$ is the Maximum Bisimulation on the (Universe?) System
$\class{V}$ %FIXME

\textbf{Prop.}:
\[
  a \equiv b \Leftrightarrow a \equiv_\class{V} b
\]
for all Sets $a,b$

A System $\class{M}$ is \emph{Extensional} if for all $a,b \in
\class{M}$:
\[
  a_\class{M} = b_\class{M} \to a = b
\]
and \emph{Strongly Extensional} if for all $a,b \in \class{M}$:
\[
  a \equiv_\class{M} b \to a = b
\]

Every Extensional Graph has at most one Decoration.

By the Extensionality Axiom (\S\ref{sec:extensionality_axiom}) the
System $\class{V}$ is Extensional.

A \emph{System Map} from System $\class{M}$ to System $\class{M}'$ is
a map $\pi : \class{M} \to \class{M}'$ such that for $a \in
\class{M}$:
\[
  (\pi a)_{\class{M}'} = \{ \pi b | b \in a_\class{M} \}
\]
If $\pi$ is a Bijection then it is a \emph{System Isomorphism}.

A System Map $G \to \class{V}$ from a Graph $G$ is a
Decoration of the Graph.

Systems and System Maps form a Category %FIXME

\textbf{Prop.} For System Maps $\pi_1,\pi_2 : \class{M} \to
\class{M}'$:
\begin{enumerate}
  \item If $R$ is a Bisimulation on $\class{M}$ then:
    \[
      (\pi_1 \times \pi_2) R
        \defeq \{(\pi_1 a_1, \pi_2 a_2) | a_1 R a_2\}
    \]
    is a Bisimulation on $\class{M}'$
  \item If $S$ is a Bisimulation on $\class{M}'$ then:
    \[
      (\pi_1 \times \pi_2)^{-1}S
        \defeq \{(a_1,a_2) \in \class{M} \times \class{M}
        | (\pi_1 a_1) S (\pi_2 a_2)\}
    \]
    is a Bisimulation on $\class{M}$
\end{enumerate}

Interpetation of AFA %FIXME

\textbf{Thm.} $ZFC^- + AFA$ has a Full Model that is Unique up to
Isomorphism

\emph{Co-bisimulation} $R^+ \subseteq R$

There is a Unique Minimal Reflexive Co-bisimulation $\sim_\class{M}$
on a System $\class{M}$

$\class{M}$ is Extensional if and only if:
\[
  a \sim_\class{M} b \to a = b
\]

if $\pi : \class{M} \to \class{M}'$ is an Injective System Map
then for $a,b \in \class{M}$:
\[
  a \sim_\class{M} \Leftrightarrow \pi a \sim_{\class{M}'} \pi b
\]



\paragraph{Complete System}\label{sec:complete_system}\hfill

For a System $\class{M}$, an $\class{M}$-decoration of a Graph $G$ is
a System Map $G \to \class{M}$

A $\class{V}$-decoration of $G$ is just a Decoration of $G$

\emph{Complete System} -- System $\class{M}$ is a Complete System if
every Graph has a Unique $\class{M}$-decoration

AFA holds if and only if $\class{V}$ is a Complete System

the Class of Accessible Pointed Graphs form a System $\class{V}_0$
with Edges $(G a, G b)$ wherever $G$ is a Graph and $a \to b
\in G$

Strongly Extensional Quotient of $\class{V}_0$:
\[
  \pi_c : \class{V}_0 \to \class{V}_c
\]

$\class{V}_c$ is Complete (\S\ref{sec:complete_system})

For each System $\class{M}$ there is a Unique System Map $\class{M}
\to \class{V}_c$

\textbf{Thm.} The following are equivalent for a System $\class{M}$:
\begin{enumerate}
  \item For each System $\class{M}'$ there is a Unique System Map
    $\class{M}' \to \class{M}$
  \item $\class{M}$ is Complete
  \item $\class{M} \cong \class{V}_c$
\end{enumerate}

Every Complete System is Full (\S\ref{sec:full_system})

\textbf{Thm.} Each Complete System is a Model of $ZFC^- + AFA$

\textbf{Thm.} $ZFC^- + AFA$ has a Full Model that is Unique up to
Isomorphism



\paragraph{Full System}\label{sec:full_system}\hfill

A System $\class{M}$ is a \emph{Full System} if for every Set $x
\subseteq \class{M}$ there is a Unique $a \in \class{M}$ such that $x
= a_\class{M}$

$\class{V}$ is a Full System

$\class{V}_{wf}$ of Well-founded Sets is a Full System

$\class{V}_{wf}$ is the smallest Class $\class{M}$ such that
$\class{M} = \pow\class{M}$ and $\class{V}$ is the largest such Class.

Foundation Axiom (\S\ref{sec:foundation_axiom}) is expressed by
$\class{V} = \class{V}_{wf}$

Equivalent statements for a Full System $\class{M}$:
\begin{enumerate}
  \item For each Full System $\class{M}'$ there is a Unique System Map
    $\class{M} \to \class{M}'$
  \item $\class{M}$ is Well-founded
  \item $\class{M} \cong \class{V}_{wf}$
\end{enumerate}

Every Complete System is Full



\paragraph{Superuniversal System}\label{sec:superuniversal_system}
\hfill

Superuniversality Axiom (\S\ref{sec:superuniversality_axiom})

$BAFA \Leftrightarrow \class{V} \;\text{is Superuniversal}$

\textbf{Thm.} Every Superuniversal System is Full

There is a Unique Superuniversal System up to Isomorphism

\emph{Globally Universal}

Every Superuniversal System is Globally Universal



\subsubsection{Hyperset Theory}\label{sec:hyperset_theory}

(Aczel)

Extension of Classical Set Theory

%FIXME merge with non-well-founded set theory?



\paragraph{Hyperset}\label{sec:hyperset}\hfill

\emph{Hyperset} -- Set that is not necessarily Well-founded



% ==============================================================================
\section{Set Algebra}\label{sec:set_algebra}
% =============================================================================

or \emph{Field of Sets} (\emph{Note}: not a ``Field'' in the sense of Field
Theory \S\ref{sec:field})

Representation Theory (\S\ref{sec:representation_theory}) of Boolean Algebras
(\S\ref{sec:boolean_algebra}) -- every Boolean Algebra can be Represented as (is
Isomorphic to) a Field of Sets

if a Set Algebra is Closed under Countable Intersections and Countable Unions,
it is called a \emph{$\sigma$-algebra} (\S\ref{sec:sigma_algebra}), and the
corresponding Field of Sets is called a \emph{Measurable Space}
(\S\ref{sec:measurable_space})

(ncatlab: \url{https://ncatlab.org/nlab/show/sigma-algebra})
given a collection $M$ of Subsets $S \subseteq X$ of a Set $X$, there are
several kinds of structures for $M$:
\begin{itemize}
  \item a \emph{Ring} on $X$ -- $M$ is Closed under Relative Complementation and
    Unions of Finitary Families:
    \begin{enumerate}
      \item $\varnothing \in M$
      \item $S, T \in M \to S \cup T \in M$
      \item $S, T \in M \to S \backslash T \in M$
    \end{enumerate}
  \item a \emph{$\delta$-ring} on $X$ -- $M$ is a Ring Closed under
    Intersections of Countably Infinite Families:
    \begin{enumerate}
      \item $\varnothing \in M$
      \item $S, T \in M \to S \cup T \in M$
      \item $S, T \in M \to S \backslash T \in M$
      \item $S_1, S_2, S_3, \ldots \in M \to \cap_i S_i \in M$
    \end{enumerate}
  \item a \emph{$\sigma$-ring} on $X$ -- $M$ is a Ring Closed under Unions of
    Countably Infinite Families:
    \begin{enumerate}
      \item $\varnothing \in M$
      \item $S, T \in M \to S \cup T \in M$
      \item $S, T \in M \to S \backslash T \in M$
      \item $S_1, S_2, S_3, \ldots \in M \to \cup_i S_i \in M$
    \end{enumerate}
    it follows that a $\sigma$-ring is also a $\delta$-ring
  \item an \emph{Algebra} (or \emph{Field}) on $X$ -- $M$ is a Ring to which $X$
    itself belongs:
    \begin{enumerate}
      \item $\varnothing \in M$
      \item $S, T \in M \to S \cup T \in M$
      \item $S, T \in M \to S \backslash T \in M$
      \item $X \in M$
    \end{enumerate}
    note that (2.) is redundant and can be defined in terms of the Relative
    Complement (3.); it follows that $M$ is Closed under Absolute
    Complementation
  \item a $\sigma$-algebra (\S\ref{sec:sigma_algebra}) on $X$ -- $M$ is both an
    Algebra and a $\sigma$-ring:
    \begin{enumerate}
      \item $\varnothing \in M$
      \item $S, T \in M \to S \cup T \in M$
      \item $S, T \in M \to S \backslash T \in M$
      \item $X \in M$
      \item $S_1, S_2, S_3, \ldots \in M \to \cup_i S_i \in M$
    \end{enumerate}
\end{itemize}

\emph{Stone's Representation Theorem for Boolean Algebras} -- every Boolean
Algebra $B$ is Isomorphic to a certain Field of Sets, viz. the Algebra of Clopen
Subsets (\S\ref{sec:clopen_set}) of its Stone Space (\S\ref{sec:stone_space})
$\xspace{S}(B)$ sending an Element $b \in B$ to the Set of all Ultrafilters
(\S\ref{sec:ultrafilter}) that contain $b$; Category Theoretic statement: there
is a Duality between the Category of Boolean Algebras, $\cat{BA}$, and the
Category of Stone Spaces, additionally implying that each Homomorphism from a
Boolean Algebra $A$ to a Boolean Algebra $B$ corresponds in a ``natural way'' to
a \emph{Continuous Function} (\S\ref{sec:continuous_function}) from $S(B)$ to
$S(A)$, i.e. there is a Contravariant Functor
(\S\ref{sec:contravariant_functor}) that gives an Equivalence between Categories
(a Nontrivial Duality of Categories); special case of Stone Duality
(\S\ref{sec:stone_duality}) for Dualities between Topological Spaces
(\S\ref{sec:topological_space}) and Partially Ordered Sets (\S\ref{sec:poset})

\fist \emph{J\'onsson-Tarski Duality} --
generalization of Stone Representation Theorem to Modal Algebras
(\S\ref{sec:modal_algebra}) and General Frames (\S\ref{sec:general_frame})



% ==============================================================================
\section{Algebraic Set Theory}\label{sec:algebraic_set_theory}
% =============================================================================

% ==============================================================================
\section{Descriptive Set Theory}\label{sec:descriptive_set_theory}
% ==============================================================================

\emph{Boldface Borel Hierarchy} (\S\ref{sec:projective_hierarchy})

Borel Hierarchy (\S\ref{sec:borel_hierarchy})

Polish Spaces (\S\ref{sec:polish_space})



% ------------------------------------------------------------------------------
\subsection{Analytic Set}\label{sec:analytic_set}
% ------------------------------------------------------------------------------

% ------------------------------------------------------------------------------
\subsection{Projective Set}\label{sec:projective_set}
% ------------------------------------------------------------------------------

% ------------------------------------------------------------------------------
\subsection{Meagre Set}\label{sec:meagre_set}
% ------------------------------------------------------------------------------

Subset of a Topological Space that is Negligible (\S\ref{sec:negligible_set})

a Subset $A$ of a Topological Space is \emph{Almost Open}
(\S\ref{sec:almost_open}) if it differs from an Open Set by a Meagre Set



% ------------------------------------------------------------------------------
\subsection{Comeagre Set}\label{sec:comeagre_set}
% ------------------------------------------------------------------------------

Complement of a Meagre Set

the Set of Nowhere-differentiable (\S\ref{sec:nowhere_differentiable})
Real-valued Functions on the Closed Interval $[0,1]$ is Comeagre in the Vector
Space $C([0,1]; \reals)$ of Continuous Real-valued Functions on $[0,1]$ with
the Topology of Uniform Convergence (\S\ref{sec:uniform_convergence})



% ------------------------------------------------------------------------------
\subsection{Pointclass}\label{sec:pointclass}
% ------------------------------------------------------------------------------

% ------------------------------------------------------------------------------
\subsection{Effective Descriptive Set Theory}
\label{sec:effective_descriptive}
% ------------------------------------------------------------------------------

Combination of Descriptive Set Theory with \emph{Recursion Theory}
(Part \ref{part:recursion_theory}).



% ==============================================================================
\section{Constructive Set Theory}\label{sec:constructive_set_theory}
% ==============================================================================

% ------------------------------------------------------------------------------
\subsection{Apartness}\label{sec:apartness}
% ------------------------------------------------------------------------------

An \emph{Apartness Relation} is a Binary Relation $\#$ such that:

\begin{enumerate}
\item $\neg (x\#x)$
\item $x\#y \to y\#x$
\item $x\#y \to (x\#z \vee y\#z)$
\end{enumerate}

In Models of IZF all Sets with Apartness Relations are Subcountable
(\S\ref{sec:subcountable}).



\subsubsection{Tight}\label{sec:tight}

A \emph{Tight Apartness Relation} is an Apartness Relation that
additionally satisfies:
\[
  \neg (x \# y) \to x = y
\]



% ------------------------------------------------------------------------------
\subsection{IZF}\label{sec:izf}
% ------------------------------------------------------------------------------

% ------------------------------------------------------------------------------
\subsection{CZF}\label{sec:czf}
% ------------------------------------------------------------------------------

Aczel86 -- Type-theoretic Interpretation



% ------------------------------------------------------------------------------
\subsection{Light Affine Set Theory}\label{sec:light_affine_set_theory}
% ------------------------------------------------------------------------------

Terui02 - \emph{Light Affine Set Theory: A Naive Set Theory of Polynomial Time}

Light Affine Linear Logic (\S\ref{sec:light_linear_logic})

Characterizing Complexity Classes (\S\ref{sec:complexity_class})



% ==============================================================================
\section{Inner Model Theory}\label{sec:inner_model_theory}
% ==============================================================================
