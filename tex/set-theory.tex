%%%%%%%%%%%%%%%%%%%%%%%%%%%%%%%%%%%%%%%%%%%%%%%%%%%%%%%%%%%%%%%%%%%%%%
%%%%%%%%%%%%%%%%%%%%%%%%%%%%%%%%%%%%%%%%%%%%%%%%%%%%%%%%%%%%%%%%%%%%%%
\part{Set Theory}\label{part:set_theory}
%%%%%%%%%%%%%%%%%%%%%%%%%%%%%%%%%%%%%%%%%%%%%%%%%%%%%%%%%%%%%%%%%%%%%%
%%%%%%%%%%%%%%%%%%%%%%%%%%%%%%%%%%%%%%%%%%%%%%%%%%%%%%%%%%%%%%%%%%%%%%

\emph{Set Theory} is formulated within First-order Logic
(\S\ref{sec:firstorder_logic}) and as such the objects of Set Theory
are \emph{Sets} (\S\ref{sec:set}) and \emph{Relations}
(\S\ref{sec:set_relation}). See Axiomatic Set Theory
(\S\ref{sec:axiomatic_set_theory}) for specific Systems of Set Theory.

Combinatorial Set Theory (Infinitary Combinatorics
\S\ref{sec:infinitary_combinatorics})



% ====================================================================
\section{Set}\label{sec:set}
% ====================================================================

A \emph{Set} is a Well-defined (\S\ref{sec:well_defined}) collection
of distinct objects defined by the Property of \emph{Set Membership}
which can be expressed as a Binary Relation
(\S\ref{sec:binary_relation}) ``$\in$''. An object that satisfies this
Relation for a particular Set is a \emph{Member} (or \emph{Element})
of that Set, e.g. $x \in A$ is True when $x$ is a Member of the Set
$A$.

A Set is allowed to have other Sets as Members and conversely a Set is
allowed to be a Member of other Sets. An object in Set Theory that is
not allowed to have Members is called an \emph{Individual} or
\emph{Urelement} (\S\ref{sec:urelement}) and an object that may have
Members but may not be a Member of another object is called a
\emph{Class} (\S\ref{sec:class}).

A Set $A$ may be given by its Extension (\S\ref{sec:extension}), that
is, listing each member in the Set:
\[
  A = \{x,y,z\}
\]
The Intentionsional (\S\ref{sec:intension}) Definition of a Set $B$
may be given by a Property (\S\ref{sec:property}) or other rule that
specifies its Members:
\[
  B = \{ x : x \in \nats_0 \wedge x < 4 \}
\]
defines the Set $B$ with Extension $\{ 0, 1, 2, 3 \}$.

By the Axiom of Extensionality (\S\ref{sec:extensionality_axiom}) Sets
are uniquely defined by their constituent Elements and therefore Sets
with multiple Members that are Identical are Equal and the order in
which Elements are given is unimportant:
\[
  \{ 2, 3 \} = \{ 3, 2 \} = \{ 2, 3, 3, 2 \}
\]

De Morgan's Laws (\S\ref{sec:de_morgan}) for Sets



% --------------------------------------------------------------------
\subsection{Subset}\label{sec:subset}
% --------------------------------------------------------------------

For two Sets $A$ and $B$, if all the Members of $A$ are also Members
of $B$, then $A$ is a \emph{Subset} of $B$, denoted $A \subseteq B$:
\[
  (\forall x) x \in A \rightarrow x \in B \Rightarrow A \subseteq B
\]
By this definition a Set is a Subset of itself, $A = B \Rightarrow A
\subseteq B$, but a \emph{Proper Subset} may be defined as a Subset
that is not equal to the containing Set, $A \subset B$:
\[
  A \subseteq B \wedge B \nsubseteq A \Rightarrow A \subset B
\]



\subsubsection{Indicator Function}\label{sec:indicator_function}



% -------------------------------------------------------------------
\subsection{Set Union}\label{sec:set_union}
% -------------------------------------------------------------------

\emph{Union}

$A \cup B$



\subsubsection{Disjoint Union}\label{sec:disjoint_union}

Given a Family (\S\ref{sec:family}) of Sets ${A_i : i \in I}$,
the \emph{Disjoint Union} is defined as:
\[
  \bigsqcup_{i \in I} A_i = \bigcup_{i \in I} \{(x,i) | x \in A_i \}
\]

Coproduct (\S\ref{sec:coproduct})



\subsubsection{Infinitary Union}\label{sec:infinitary_union}

\[
  x \in \bigcup S \leftrightarrow \exists y \in S : x \in y
\]



% -------------------------------------------------------------------
\subsection{Set Intersection}\label{sec:set_intersection}
% -------------------------------------------------------------------

\emph{Intersection}

$A \cap B$



% -------------------------------------------------------------------
\subsection{Relative Complement}\label{sec:relative_complement}
% -------------------------------------------------------------------

The \emph{Relative Complement} (or \emph{Difference}) of two Sets $A$
and $B$, denoted $A \setminus B$, is the Set of Elements in $A$ but
not $B$.



\subsubsection{Absolute Complement}\label{sec:absolute_complement}

The \emph{Absolute Complement} (or \emph{Complement}) of a Set $A$ is
the Relative Complement of $A$ with the given Universe
(\S\ref{sec:universe}) $\mathcal{U}$:
\[
  \mathcal{U} \setminus A
\]



% -------------------------------------------------------------------
\subsection{Symmetric Difference}\label{sec:symmetric_difference}
% -------------------------------------------------------------------

The \emph{Symmetric Difference} of two Sets $A$ and $B$, denoted $A
\oplus B$, is the Set of Elements in $A$ or $B$ but not both $A$ and
$B$:
\[
  A \oplus B =
  \{ x : (x \in A \vee x \in B) \wedge x \notin A \cap B \}
\]
or:
\[
  A \oplus B = A \setminus B \cup B \setminus A
\]



% -------------------------------------------------------------------
\subsection{Cartesian Product}\label{sec:cartesian_product}
% -------------------------------------------------------------------

% --------------------------------------------------------------------
\subsection{Partition}\label{sec:set_partition}
% --------------------------------------------------------------------

A \emph{Partition} of a Set $X$ is a Set of Non-empty Subsets of $X$
such that every Element of $X$ is in exactly one Subset, thus $X$ is
the Disjoint Union (\S\ref{sec:disjoint_union}) of the Subsets. A
Partition $P$ of $X$ has the Properties:
\begin{enumerate}
  \item $\emptyset \notin P$
  \item $\bigcup_{A \in P}A = X$
  \item $A,B \in P \wedge A \neq B \Rightarrow A \cap B = \emptyset$
\end{enumerate}



% --------------------------------------------------------------------
\subsection{Powerset}\label{sec:powerset}
% --------------------------------------------------------------------

$\pow{X}$, $2^X$

\subsubsection{Cantor's Theorem}\label{sec:cantors_theorem}



% --------------------------------------------------------------------
\subsection{Cardinality}\label{sec:cardinality}
% --------------------------------------------------------------------

The \emph{Cardinality} (or \emph{Size}) of a Set $A$, denoted $|A|$ or
$card(A)$, is a measure of the number of Elements in the Set. Two Sets
with the same Cardinality are said to be \emph{Equinumerous}.

The unique Set with Cardinality 0 is called the \emph{Empty Set} and
is denoted $\{\}$ or $\varnothing$. A Set with Cardinality 1 is called a
\emph{Singleton Set}.

Cardinal Numbers (\S\ref{sec:cardinal_number})

Bijection (\S\ref{sec:bijective_function}) and Injection
(\S\ref{sec:injective_function})

Cantor's Theorem (\S\ref{sec:cantors_theorem})

The Axiom of Choice (\S\ref{sec:choice_axiom}) Implies the Law of
Trichotomy (\S\ref{sec:trichotomy_law}) for Cardinality which assigns
a given Set $X$ to one of the Classes:

\begin{description}
\item [Finite] $|X| < |\nats|$ (\S\ref{sec:finite_cardinality})
\item [Countably Infinite] $|X| = |\nats| = \aleph_0$
  (\S\ref{sec:countably_infinite})
\item [Uncountably Infinite] $|\nats| < |X|$
  (\S\ref{sec:uncountably_infinite})
\end{description}

Infinitary Combinatorics (\S\ref{sec:infinitary_combinatorics})



\subsubsection{Finite}\label{sec:finite_cardinality}

\subsubsection{Countably Infinite}\label{sec:countably_infinite}

\subsubsection{Uncountably Infinite}\label{sec:uncountably_infinite}

\subsubsection{Subcountable}\label{sec:subcountable}

A Set $X$ is \emph{Subcountable} if there is a Partial Surjection from
$\nats$ onto the $X$, that is, if $X$ is Finite
(\S\ref{sec:finite_cardinality}) or Countably Infinite
(\S\ref{sec:countably_infinite})



\subsubsection{Transfinite}\label{sec:transfinite}

Large Cardinal (\S\ref{sec:large_cardinal})



% --------------------------------------------------------------------
\subsection{Index Set}\label{sec:index_set}
% --------------------------------------------------------------------

An \emph{Index Set} is one that \emph{Indexes} (or \emph{Labels})
Members of another Set. Indexing is a Surjective Function
(\S\ref{sec:surjective_function}) from an Index Set onto a target Set.



\subsubsection{Indexed Family}\label{sec:indexed_family}

An \emph{Indexed Family} of Sets is a Function from an Index Set to
the Class (\S\ref{sec:class}) of all Sets. For Index Set $J$ and
Indexed Set $A$, the Indexed Family may be denoted
\[
  (A_j)_{j \in J}
\]



\paragraph{Cover}\label{sec:cover}
\hfill \\

A \emph{Cover} of a Set $X$ is an Indexed Family of Sets $C = \{ U_i :
i \in I \}$ such that their Union contains $X$:
\[
  X \subseteq \bigcup_{i \in I} U_i
\]



% --------------------------------------------------------------------
\subsection{Transitive Set}\label{sec:transitive_set}
% --------------------------------------------------------------------

A Set, $A$, is \emph{Transitive} if and only if:
\[
  \bigcup A \subseteq A
\]
That is for each non-empty Set $B \in A$:
\[
  B \in A \rightarrow B \subset A
\]
Transitivity for Classes (\S\ref{sec:class}) is defined in the same
way.

For two Transitive Sets, $A$ and $B$, the Set $A \cup B \cup \{A,B\}$
is Transitive.

A Set, $B$, containing no Urelements is Transitive if and only if $A
\subset \pow(X)$



\subsubsection{Admissible Set}\label{sec:admissible_set}

An \emph{Admissible Set}, $A$, is a Transitive Set such that $\langle
A, \in \rangle$ is a Model (\S\ref{sec:model}) of Kripke-Platek Set
Theory (\S\ref{sec:kripke_platek}).

The smallest example of an Admissible Set is the Set of
\emph{Hereditarily Finite Sets}. %FIXME ref hereditarily finite



% --------------------------------------------------------------------
\subsection{Pointed Set}\label{sec:pointed_set}
% --------------------------------------------------------------------

A \emph{Pointed Set} (or \emph{Based Set} or \emph{Rooted Set}) is an
Ordered Pair $(X, x_0)$ where $X$ is a Set and $x_0 \in X$ is the
\emph{Basepoint} of $X$. This defines an Algebraic Structure
(\S\ref{sec:algebraic_structure}) on $X$ with a single Nullary
Function that returns the Basepoint.



% --------------------------------------------------------------------
\subsection{Multiset}\label{sec:multiset}
% --------------------------------------------------------------------

A \emph{Multiset} (or \emph{Bag}) is a 2-tuple $(A,m)$ of an
\emph{Underlying Set}, $A$, together with a \emph{Multiplicity
  Function}, $m : A \rightarrow \nats_{\geq 1}$, mapping Elements
of $A$ to Positive Natural Numbers representing the
\emph{Multiplicity} Elements, that is the number of times an Element
occurs in the Multiset. A Multiset may be denoted with square
brackets:
\[
  [a,a,b]
\]
If the Underlying Set is restricted to a Subset of a given
\emph{Universe} (\S\ref{sec:universe}), $U$, the Multiplicity Function
may be extended to $m_U : U \rightarrow \nats$ where $a \in U, a
\notin A \leftrightarrow m(a)=0$.

\emph{Indicator Function} (\S\ref{sec:indicator_function})



% ====================================================================
\section{Urelement}\label{sec:urelement}
% ====================================================================

An \emph{Urelement} (or \emph{Atom} or \emph{Individual}) is an Object
that may be an Element of a Set, but is not itself a Set.

Urelements are the dual to Proper Classes (\S\ref{sec:proper_class})
as an Urelement cannot have Members while a Proper Class cannot be a
Member.

A Two-sorted First-order Theory with Sets and Urelements has $a \in b$
defined only when $b$ is a Set.

A One-sorted First-order Theory may be defined with an Unary Relation
distinguishing Sets and Urelements, which can be achieved if the Unary
Relation can at least distinguish Urelements from the Empty Set, since
all other Sets have Members. The Axiom of Extensionality must also be
formulated to apply only to Sets and not Urelements.



% ====================================================================
\section{Class}\label{sec:class}
% ====================================================================

A \emph{Class} is any Subset of the \emph{Universe}
(\S\ref{sec:universe}) of discussion.

\begin{description}
  \item [Proper Class] a Class that is not a Set
  \item [Small Class] a Class that is a Set
\end{description}



% --------------------------------------------------------------------
\subsection{Proper Class}\label{sec:proper_class}
% --------------------------------------------------------------------

A \emph{Proper Class} is an Object that cannot be a Member of another
Object. This is the Dual concept of an Urelement which cannot have
another Object as a Member.



% --------------------------------------------------------------------
\subsection{Subclass}\label{sec:subclass}
% --------------------------------------------------------------------



% ====================================================================
\section{Family}\label{sec:family}
% ====================================================================

A \emph{Family} is a collection of Sets that is allowed to be a
Multiset (\S\ref{sec:multiset}) and/or a Small or Proper Class
(\S\ref{sec:class}).



% ====================================================================
\section{Universe}\label{sec:universe}
% ====================================================================

% FIXME differentiate between Universes
A \emph{Universe} is a Set, $U$, with the following Closure Properties
\cite{maclane69}:
\begin{enumerate}
\item $x \in A \in U \rightarrow x \in U$
\item $x \in U \wedge y \in U \rightarrow \{x,y\}, \langle x,y
  \rangle, x \times y \in U$
\item $x \in U \rightarrow \pow(x) \in U \wedge \bigcup x \in U$
\item $\omega = \{0,1,2,\ldots\} \in U$
\item Given a Surjective Function, $f : a \rightarrow b, a \in
  U, b \subset U \rightarrow b \in U$
\end{enumerate}



% --------------------------------------------------------------------
\subsection{Small Set}\label{sec:small_set}
% --------------------------------------------------------------------

A \emph{Small Set} may be said to be a member of a Universe that is
not itself a Universe.



% --------------------------------------------------------------------
\subsection{Cumulative Hierarchy}\label{sec:cumulative_hierarchy}
% --------------------------------------------------------------------

% --------------------------------------------------------------------
\subsection{Superstructure}\label{sec:superstructure}
% --------------------------------------------------------------------

An Universe may be generated over a Set resulting in a
\emph{Superstructure}. The Superstructure over a Set $X$:
\[
  \mathbf{S}X := \bigcup^{\infty}_{i=0}\mathbf{S}_i X
\]
can be defined by Structural Recursion
(\S\ref{sec:recursive_definition}) as follows:
\begin{itemize}

\item $\mathbf{S}_0 X = X$
\item $\mathbf{S}_1 X = X \cup \pow(X)$
\item $\mathbf{S}_{n+1} X =
  \mathbf{S}_n X \cup \pow(\mathbf{S}_n X)$

\end{itemize}
Some Properties of $\mathbf{S}\{\}$ (the Superstructure over the Empty
Set):
\begin{itemize}

\item $\nats \subset \mathbf{S}\{\}$
\item $\nats \notin \mathbf{S}\{\}$ (Elements of $\mathbf{S}\{\}$
  are Finite Sets)
\item $\mathbf{S}\{\}$ contains all of the Hereditarily Finite Sets
%FIXME ref Hereditarily finite sets

\end{itemize}

The Superstructure over $\nats$, $\mathbf{S}\nats$, is
considered the \emph{Universe of Ordinary Mathematics}.



% --------------------------------------------------------------------
\subsection{Von Neumann Universe}\label{sec:vonneumann_universe}
% --------------------------------------------------------------------

% --------------------------------------------------------------------
\subsection{Grothendieck Universe}\label{sec:grothendieck_universe}
% --------------------------------------------------------------------

%FIXME similar to definitions above

% --------------------------------------------------------------------
\subsection{Forcing}\label{sec:forcing}
% --------------------------------------------------------------------



% ====================================================================
\section{Relation}\label{sec:set_relation}
% ====================================================================

Domain (\S\ref{sec:domain})

Predicate (\S\ref{sec:predicate})

The Sets $X_1, X_2, \ldots$ are called the \emph{Domain}
(\S\ref{sec:domain}) of $R$. If any Domain of $R$ is empty, then $R$
is the unique \emph{Empty Relation} $R = \varnothing$.

The \emph{Graph} of a Relation $R$, $G(R)$, defines the Relation in
Extension (\S\ref{sec:extension}) as a Subset of the Cartesian Product
(\S\ref{sec:cartesian_product}) of the Domain:
\[
  G(R) \subseteq X_1 \times X_2 \times \ldots
\]

The \emph{Arity} of a Relation is the Dimension of the Domain in the
Graph of a Relation. %FIXME

The Expression $R x_1 x_2 \ldots$ where $x_i \in X_i$ is True when
$(x_1, x_2, \ldots) \in G(R)$ and False otherwise. As such, any
Relation may defined by a Boolean-valued \emph{Characteristic
  Function} (or \emph{Indicator Function}
\S\ref{sec:indicator_function}, cf. Predicate \S\ref{sec:predicate}):
\[
  f_R : X_1 \times X_2 \times \ldots \rightarrow \{\top,\bot\}
\]


% --------------------------------------------------------------------
\subsection{Finitary Relation}\label{sec:finitary_relation}
% --------------------------------------------------------------------

A Relation with a Finite Arity $n$ is called a \emph{Finitary
  Relation} or \emph{$n$-ary Relation}.

There are only two $0$-ary Relations on the Empty Tuple $()$: one that
is always True and one that is always False.

A $2$-ary Relation is usually called a Binary Relation
(\S\ref{sec:binary_relation}), and a $3$-ary Relation may be called a
Ternary Relation.



\subsubsection{Correspondence}\label{sec:correspondence}

An $n$-ary Relation may be completely specified by an $n + 1$-tuple
called a \emph{Correspondence} (or \emph{Embedded} or \emph{Included
  Relation}):
\[
  (X_1, \ldots, X_n, G(R))
\]



% --------------------------------------------------------------------
\subsection{Binary Relation}\label{sec:binary_relation}
% --------------------------------------------------------------------

A \emph{Binary} (or \emph{Dyadic}) Relation is a $2$-ary Relation.

When the Graph of a Binary Relation is plotted on a Coordinate Plane
(\S\ref{sec:cartesian_coordinate}), the First Elements of the Ordered
Pairs of $G(R)$, mapped to the Horizontal Axis, are \emph{Abscissae},
and the Second Elements of the Ordered Pairs of $G(R)$, mapped to the
Vertical Axis, are called \emph{Ordinates}.

Reduction Relation (\S\ref{sec:reduction_relation})

A Binary Relation $R$ on a Set $A$ is:
\begin{description}
\item[Serial](\S\ref{sec:serial_relation}) if:

  $ \forall a \in A \exists b \in A : xRy $

\item[Reflexive](\S\ref{sec:serial_relation}) if:

  $ xRx = \top $

  Reflexive Implies Transitive and Serial

\item[Irreflexive] (also \emph{Strict}
  \S\ref{sec:irreflexive_relation}) if:

  $ xRx = \bot $

\item[Co-reflexive](\S\ref{sec:coreflexive_relation}) if:

  $ xRy \rightarrow a = b $

\item[Transitive](\S\ref{sec:transitive_relation}) if:

  $ xRy \wedge yRc \rightarrow xRc $

  Transitive and Irreflexive if and only if Transitive and Asymmetric

\item[Symmetric](\S\ref{sec:symmetric_relation}) if:

  $ xRy \leftrightarrow yRx $

\item[Anti-symmetric](\S\ref{sec:antisymmetric_relation}) if:

  $ xRy \wedge yRx \rightarrow a = b $

\item[Asymmetric](\S\ref{sec:antisymmetric_relation}) if both
  Anti-symmetric and Irreflexive:

  $ xRy \rightarrow \neg yRx $

\item[Left-total] (\S\ref{sec:left_total}) if:

  $ \forall a \in A \exists b \in A : xRy $

  Multimaps (\S\ref{sec:multimap}) and Functions
  (\S\ref{sec:set_function}) are Left-total

\item[Right-total] (\emph{Surjective} or \emph{Onto}
  \S\ref{sec:surjective_function}) if:

  $ \forall b \in A \exists a \in A : xRy $

\item[Right-unique] (or \emph{Functional} or \emph{Univalent}
  \S\ref{sec:functional_relation}) if:

  $ \forall x \in dom(R), y \in rng(R)
    (xRy \wedge xRz \rightarrow y = z) $

\item[Total] (\S\ref{sec:total_relation}) if:

  $ \forall a,b \in A, xRy \vee yRx $

  Total Implies Reflexive (\S\ref{sec:reflexive_relation})

\item[Trichotomous] (\S\ref{sec:trichotomous_relation}) if:

  $ xRy \vee yRx \vee a = b $

  Trichotomous Implies Irreflexive (\S\ref{sec:irreflexive_relation}).
  Trichotomous and Transitive (\S\ref{sec:transitive_relation})
  Implies Asymmetric (\S\ref{sec:asymmetric_relation}). A Transitive
  Trichotomous Relation is a Strict Total Order
  (\S\ref{sec:strict_order}, \S\ref{sec:total_order}).

\item[Right Euclidean] (\S\ref{sec:euclidean_relation}) if:

  $ xRy \wedge xRc \rightarrow yRc $

\item[Left Euclidean] (\S\ref{sec:euclidean_relation}) if:

  $ yRx \wedge cRx \rightarrow yRc $

\end{description}

From the above Classes of Relations, the following Binary Ordering
Relations (\S\ref{sec:ordering_relation}) are distinguished, listed
here from most general to most restricted:

\begin{description}
\item[Preorder] (\S\ref{sec:preorder}) when Reflexive and Transitive
  (all Partial Orders and Equivalence Relations are Preorders)
\item[Partial Order] (\S\ref{sec:partial_order}) when Preorder and
  Anti-symmetric
\item[Weak order] (\S\ref{sec:weak_order}) when Preorder and Total
\item[Total Order] (\S\ref{sec:total_order}) when a Partial Order and
  Total
\item[Partial Equivalence] (\S\ref{sec:partial_equivalence}) when
  Symmetric and Transitive
\item[Equivalence] (\S\ref{sec:equivalence_relation}) when Reflexive,
  Symmetric, and Transitive
\end{description}

The Set of all Binary Relations on a Set $A$ is denoted
$\mathbf{Rel}(A)$ and is the Powerset (\S\ref{sec:powerset}) of $A
\times A$: $2^{A \times A}$.

A Dependency Relation (\S\ref{sec:dependency_relation}) is a Binary
Relation that is Symmetric and Reflexive.



\subsubsection{Reflexive Relation}\label{sec:reflexive_relation}

\subsubsection{Irreflexive Relation}\label{sec:irreflexive_relation}

\subsubsection{Coreflexive Relation}\label{sec:coreflexive_relation}

\subsubsection{Transitive Relation}\label{sec:transitive_relation}

\subsubsection{Symmetric Relation}\label{sec:symmetric_relation}

\subsubsection{Antisymmetric Relation}\label{sec:antisymmetric_relation}

\subsubsection{Asymmetric Relation}\label{sec:asymmetric_relation}

\subsubsection{Left-total Relation}\label{sec:left_total}

\subsubsection{Total Relation}\label{sec:total_relation}

\subsubsection{Functional Relation}\label{sec:functional_relation}

Right-unique

A Function (\S\ref{sec:set_function}) is a Left-total
(\S\ref{sec:left_total}) Functional Relation.



\subsubsection{Trichotomous Relation}\label{sec:trichotomous_relation}

\paragraph{Law of Trichotomy}\label{sec:trichotomy_law}
\hfill \\

Trichotomous Relation $<$ on a Set $X$ exactly one of the following
holds for all $x,y \in X$:
\begin{itemize}
\item $x < y$
\item $x = y$
\item $x > y$
\end{itemize}



\subsubsection{Serial Relation}\label{sec:serial_relation}

\subsubsection{Extensional Relation}\label{sec:extensional_relation}

A Binary Relation $R$ is \emph{Extensional} if and only if:
\[
  \forall x,y (x = y \leftrightarrow
    \forall z (R(x,z) \leftrightarrow R(y,z))
\]
The Converse of an Extensional Relation is not necessarily
Extensional.

\subsubsection{Euclidean Relation}\label{sec:euclidean_relation}

\emph{Left-euclidean} \emph{Right-euclidean}



% --------------------------------------------------------------------
\subsection{Endorelation}\label{sec:endorelation}
% --------------------------------------------------------------------

$X = Y$

\begin{description}
\item [Directed Graph] (\S\ref{sec:directed_graph})

\item [Undirected Graph] (\S\ref{sec:undirected_graph}) Irreflexive,
  Symmetric

\item [Tournament] (\S\ref{sec:tournament}) Irreflexive, Antisymmetric

\item [Dependency] (\S\ref{sec:dependency_relation}) Reflexive,
  Symmetric

\item [Weak Order] (\S\ref{sec:weak_order}) Transitive

\item [Preorder] (\S\ref{sec:preorder}) Reflexive, Transitive

\item [Partial Order] (\S\ref{sec:partial_order}) Reflexive,
  Antisymmetric, Transitive

\item [Strict Partial Order] (\S\ref{sec:strict_order},
  \S\ref{sec:partial_order}) Irreflexive, Antisymmetric, Transitive

\item [Partial Equivalence] (\S\ref{sec:partial_equivalence})
  Symmetric, Transitive

\item [Equivalence Relation] (\S\ref{sec:equivalence_relation})
  Reflexive, Symmetric, Transitive

\end{description}

Any Endorelation may be given as a Directed Graph. An Endorelation is
a Complete Graph (\S\ref{sec:complete_graph}) when $a \neq b
\rightarrow xRy$ and implies Symmetry. An Endorelation is a Tournament
when $a \neq b \rightarrow xRy \vee yRx$ and Implies Antisymmetry.



% --------------------------------------------------------------------
\subsection{Equivalence Relation}\label{sec:equivalence_relation}
% --------------------------------------------------------------------

An \emph{Equivalence Relation} on a Set $X$ is a Reflexive, Symmetric,
and Transitive Relation that Partitions (\S\ref{sec:set_partition})
the Set into Disjoint Subsets called \emph{Equivalence Classes}
(\S\ref{sec:equivalence_class}).

The Equivalence Class of an Element $a \in X$ with Equivalence
Relation $\sim$ is the Subset of $X$ defined as:
\[
    [a] = \{x \in X | a \sim x\}
\]
The \emph{Quotient Set} is the Set of Equivalence Classes of a
particular Equivalence Relation.

For a Function between Sets $f : S \rightarrow T$, one may define an
Equivalence Relation on Elements $a,b \in S$:
\[
    a \sim b \Leftrightarrow f(a) = f(b)
\]
where the Equivalence Classes are the Fibers (\S\ref{sec:fiber}) of
the Elements in $T$ and the Quotient Set is the Image of the
Equivalence Relation viewed as a Function on Elements of $S$ to their
Equivalence Classes.



\subsubsection{Equivalence Class}\label{sec:equivalence_class}

(or \emph{Quotient})



\paragraph{Canonical Form}\label{sec:canonical_form}



\subsubsection{Partial Equivalence}\label{sec:partial_equivalence}

\subsubsection{Congruence Relation}\label{sec:congruence_relation}

Equivalence Relation on an Algebraic Structure
(\S\ref{sec:algebraic_structure}) compatible with that Structure



% --------------------------------------------------------------------
\subsection{Relation Join}\label{sec:relation_join}
% --------------------------------------------------------------------

\emph{Relational Algebra} (\S\ref{sec:relational_algebra})

\emph{Pullback} (\S\ref{sec:pullback})



\subsubsection{Relation Composition}\label{sec:relation_composition}

For two Binary Relations, $R \subseteq X \times Y$ and $S \subseteq Y
\times Z$, the \emph{Relation Composition}, $S \circ R \in X \times Z$
is defined as:
\[
  S \circ R = \{(x,z) \in X \times Z \;|\;
  \exists y \in Y : (x,y) \in R \wedge (y,z) \in S \}
\]

See also Function Composition (\S\ref{sec:function_composition})



% --------------------------------------------------------------------
\subsection{Closure}\label{sec:set_closure}
% --------------------------------------------------------------------

A Set, $A$, is \emph{Closed} (has \emph{Closure}) under a Relation,
$L$, if for every $(x,y) \in L$, $y \in dom(L)$:
\[
  (\forall x \in A) (x,y) \in L \Rightarrow y \in A
\]



\subsubsection{Transitive Closure}\label{sec:transitive_closure}

For a Relation, $R$, on a Set, $X$, the \emph{Transitive Closure}
$R^+$ is a Relation on $X$ such that $R \subseteq R^+$ and $R^+$ is
\emph{Minimal} (the smallest Relation closed under Relation
Composition):
\[
  R^+ = \bigcup_{i \in \{1,2,3,...\}} R^i
\]
If $R$ is Transitive then $R = R^+$.

\emph{Reachability} (\S\ref{sec:dag})



\subsubsection{Reflexive Closure}\label{sec:reflexive_closure}

The \emph{Reflexive Closure}, $X$, of a Binary Relation, $R$, on a
Set, $S$, is the smallest Reflexive Relation on $S$ that contains $R$:
\[
  X = R \cup \{(x,x) : x \in S\}
\]
that is, the Union of $R$ with the Equivalence Relation $=$.

The Reflexive Closure of $<$ is $\leq$.



\paragraph{Reflexive Reduction}\label{sec:reflexive_reduction}
\hfill \\

The \emph{Reflexive Reduction} (or \emph{Irreflexive Kernel}) of a
Binary Relation $R$ on a Set $S$ is the smallest Relation $Y$ such
that $Y$ has the same Reflexive Closure as $R$:
\[
  Y = (R\;\setminus=)
\]
The Reflexive Reduction of $x \leq y$ is $x < y$.



\subsubsection{Symmetric Closure}\label{sec:symmetric_closure}

The \emph{Symmetric Closure}, $X$, of a Binary Relation, $R$, on a
Set, $S$, is the Union of $R$ with its Inverse Relation:
\[
  R \cup R^{-1} = X = R \cup \{(x,y) : (y,x) \in R\}
\]



\subsubsection{Finitary Closure}\label{sec:finitary_closure}

\subsubsection{Rewrite Closure}\label{sec:rewrite_closure}

\subsubsection{$P$-closure}\label{sec:p_closure}



% ====================================================================
\section{Function}\label{sec:set_function}
% ====================================================================

A \emph{Function} is a Left-total (\S\ref{sec:left_total})
Right-unique (Functional \S\ref{sec:functional_relation}) Relation. A
Function $f$ with Domain $A$ and Codomain $B$ is denoted:
\[
  f : A \rightarrow B
\]
The Domain (or \emph{Input}) of $f$ is denoted $dom(f)$ and the
Codomain (or \emph{Ouptut}) as $cod(f)$.

In the case where $img(f) = cod(f)$, $f$ is known as a
\emph{Surjective Function} (\S\ref{sec:surjective_function}).

The Extension (\S\ref{sec:extension}) of a Function may be called a
\emph{Graph} of the Function. Equality of Functions is defined such
that Equal Functions have the same Output for a given Input (equal in
Extension).

An \emph{Empty Function} has the Empty Set as a Domain, defining a
Unique Function for each Set, $A$:
\[
  f_A : \varnothing \rightarrow A
\]

\emph{Finitary Function}

\emph{Infinitary Function}

\HandRight\; Cf. \emph{Morphism} (\S\ref{sec:morphism})

Functional Predicate (\S\ref{sec:functional_predicate})

Real-valued Function (\S\ref{sec:real_function})

$\rightarrow$

Injection, Monomorphism: $\rightarrowtail$

Surjection, Epimorphism: $\twoheadrightarrow$

Bijection: $\rightarrowtail \hspace{-8pt} \twoheadrightarrow$

Isomorphism: $\xrightarrow{\sim}$

Inclusion Map: $\hookrightarrow$

Partial Function: $\nrightarrow$

Multimap: $\multimap$

$\leftrightarrow$

$\nleftrightarrow$

$\rightrightarrows$

$\leftrightarrows$

$\leftrightharpoons$

$\curvearrowright$

$\circlearrowleft$

$\dashrightarrow$

$\looparrowright$



% --------------------------------------------------------------------
\subsection{Image}\label{sec:image}
% --------------------------------------------------------------------

The \emph{Image} of the Function is the Subset of the Codomain that
the Function actually Maps to, and the Subset of the Domain that Maps
to it is called the \emph{Inverse Image} or \emph{Preimage}
(\S\ref{sec:preimage}).

Because a Function is Right-unique, each Element of the Preimage Maps
to one Element of the Image. This Property is expressed as:
\[
  (a,b) \in f \wedge (a,c) \in f \rightarrow b = c
\]
A Left-total Relation without this Property is known as a
\emph{Multimap} (\S\ref{sec:multimap}).



% --------------------------------------------------------------------
\subsection{Preimage}\label{sec:preimage}
% --------------------------------------------------------------------

\emph{Preimage} (or \emph{Inverse Image}) of a Subset



\subsubsection{Fiber}\label{sec:fiber}

For a Function $f : A \rightarrow B$, the \emph{Fiber} of an Element
$y \in img(f)$ is the Preimage of the Singleton Set $\{y\}$.

Fiber (Topology) \S\ref{sec:point_fiber}



% --------------------------------------------------------------------
\subsection{Restriction}\label{sec:function_restriction}
% --------------------------------------------------------------------

$f|_A$ smaller Domain $A$



% --------------------------------------------------------------------
\subsection{Extension}\label{sec:function_extension}
% --------------------------------------------------------------------

$f$ is the Extension of the Restriction $f|_A$



\subsubsection{Overriding Union}\label{sec:overriding_union}

Overriding $f : X \rightarrow Y$ by $g : W \rightarrow Y$:
\[
  f \oplus g : (X \cup W) \rightarrow Y
\]

Union of $g$ and $f|_{X/W}$



% --------------------------------------------------------------------
\subsection{Identity Function}\label{sec:identity_function}
% --------------------------------------------------------------------

$f(x) = x$

$a \mapsto a$



\subsubsection{Inclusion Map}\label{sec:inclusion_map}

An \emph{Inclusion Map} (or \emph{Inclusion Function}) is an Identity
Function that Maps Elements of a Subset to those in a Superset:
\[
  \iota : A \hookrightarrow X
\]
where $A \subseteq X$.



% --------------------------------------------------------------------
\subsection{Constant Function}\label{sec:constant_function}
% --------------------------------------------------------------------

% --------------------------------------------------------------------
\subsection{Injective Function}\label{sec:injective_function}
% --------------------------------------------------------------------

An \emph{Injective Function} (or \emph{One-to-one Function} or
\emph{Injection}) is one where the Elements of the Codomain are the
Images of at most one Elements of the Domain. A Function that is
Non-injective is considered an \emph{Information Losing Function}
because the Inverse is no longer a Function but it is a
\emph{Multimap} (\S\ref{sec:multimap}).



% --------------------------------------------------------------------
\subsection{Surjective Function}\label{sec:surjective_function}
% --------------------------------------------------------------------

A Function $f$ with $img(f) = cod(f)$ is a \emph{Surjective Function}
(or \emph{Surjection}). Such a Function may be said to be \emph{Onto}
$cod(f)$.



% --------------------------------------------------------------------
\subsection{Bijective Function}\label{sec:bijective_function}
% --------------------------------------------------------------------

A Function that is both Surjective and Injective is a \emph{Bijective
  Function} (or \emph{Bijection}). A Function is Bijective if and only
if it is also \emph{Invertible} (\S\ref{sec:inverse_function}).



\subsubsection{Involutory Function}\label{sec:involution}

$f(f(x)) = x$

An Identity Map is a trivial Involution.



% --------------------------------------------------------------------
\subsection{Inverse Function}\label{sec:inverse_function}
% --------------------------------------------------------------------

$f^{-1}$



\subsubsection{Left Inverse}\label{sec:left_inverse}

For a Function $f: X \rightarrow Y$, a \emph{Left Inverse} or
\emph{Retraction} of $f$ is a Function $g: Y \rightarrow X$ such that
$gf = Id(X)$.



\subsubsection{Right Inverse}\label{sec:right_inverse}

For a Function $f: X \rightarrow Y$, a \emph{Right Inverse} or
\emph{Section} of $f$ is a Function $h: Y \rightarrow X$ such that $fh
= Id(Y)$.



% --------------------------------------------------------------------
\subsection{Function Composition}\label{sec:function_composition}
% --------------------------------------------------------------------

Given two Functins $f : A \rightarrow B$ and $g : B \rightarrow C$,
there is a \emph{Composite Function}:
\[
  g \circ f : A \rightarrow C
\]
where $(g \circ f)(a) = g(f(a))$ and $a \in A$.

The \emph{Composition Operation} $\circ$ is Associative: $(h \circ g)
\circ f = h \circ (g \circ f)$. Composition of Functions may be
represented with the $\circ$ elided: $gf$.

Identity Element for the Composition Operation is the Identity
Function (\S\ref{sec:identity_function}) on Sets. The Identity
Function for a Set $A$:
\[
  I_A : A \rightarrow A
\]
is defined as:
\[
  I_A(a) = a
\]
with the result given the Function $f$ above:
\[
  f \circ I_A = f = I_B \circ f
\]



% --------------------------------------------------------------------
\subsection{Kernel}\label{sec:function_kernel}
% --------------------------------------------------------------------

The \emph{Kernel} of a Function $f : X \rightarrow Y$, $ker(f)$, is an
Equivalence Relation defined as:
\[
  ker(f) = \{ (x,x') \in X \times X : f(x) = f(x') \}
\]
i.e. the Set of all Pairs of Elements that Map to the same Value



% --------------------------------------------------------------------
\subsection{Equalizer}\label{sec:function_equalizer}
% --------------------------------------------------------------------

Given two Sets $X,Y$ and Functions $f,g : X \rightarrow Y$, the
\emph{Equalizer} of $f$ and $g$ is defined as:
\[
  Eq(f,g) = { x \in X | f(x) = g(x) }
\]
i.e. the Set of Arguments at which two (or more) Functions have Equal
Values.


\emph{Coequalizer}



\subsubsection{Difference Kernel}\label{sec:difference_equalizer}

Equalizer of exactly two Functions



% --------------------------------------------------------------------
\subsection{Fixed Point}\label{sec:fixed_point}
% --------------------------------------------------------------------

A \emph{Fixed Point}, $c$, of a Function $f$, is an Element of the
Domain of $f$ that is Mapped to itself by $f$:
\[
  f(c) = c
\]

Fixed-point Combinator (\S\ref{sec:fixedpoint_combinator})

Prefixpoint (\S\ref{sec:prefixpoint})

Least Fixed Point (\S\ref{sec:least_fixedpoint})

Greatest Fixed Point (\S\ref{sec:greatest_fixedpoint})

Postfixpoint (\S\ref{sec:postfixpoint})



\subsubsection{Periodic Point}\label{sec:periodic_point}

A \emph{Periodic Point} is an Element of the Domain of a Function that
is returned to after a finite number of iterations.



\subsubsection{Idempotent Function}\label{sec:idempotent}

A Function, $f$, is \emph{Idempotent} if it maps each Element of
$dom(f)$ to a Fixed Point of $f$:
\[
  f^2 = f
\]



\subsubsection{Fixed-point Theorem}\label{sec:fixedpoint_theorem}



% --------------------------------------------------------------------
\subsection{Function Space}\label{sec:function_space}
% --------------------------------------------------------------------

The \emph{Function Space} of two Sets $A$ and $B$ is the Set of all
Functions from $A$ to $B$ denoted by $B^A$.

When $B$ is a Field (\S\ref{sec:field}), Functions have a Vector
(\S\ref{sec:vector}) structure with two Pointwise Addition Operators
and Scalar Multiplication. %FIXME

Bijective Functions $A \leftrightarrow B$



\subsubsection{Evaluation Function}\label{sec:evaluation_function}

Given a Function Space $B^A$, the \emph{Evaluation Function} is
defined as:
\[
  eval : B^A \times A \rightarrow B
\]



% --------------------------------------------------------------------
\subsection{Binary Operation}\label{sec:binary_operation}
% --------------------------------------------------------------------

Domains and Codomain Subsets of the same Set

$f : S \times S \rightarrow S$



\subsubsection{Commutator}\label{sec:commutator}

Elements $a$ and $b$:
\[
  [a,b] = aba^{-1}b^{-1}
\]
or:
\[
  [a,b] = a^{-1}b^{-1}ab
\]

Group Commutator (\S\ref{sec:group_commutator})



% --------------------------------------------------------------------
\subsection{Boolean-valued Function}\label{sec:boolean_function}
% --------------------------------------------------------------------

Predicate (\S\ref{sec:predicate})

Indicator Function (\S\ref{sec:indicator_function})

Truth-valued Function (\S\ref{sec:truth_function})



% ====================================================================
\section{Partial Function}\label{sec:partial_function}
% ====================================================================

A \emph{Partial Function} is a Functional Relation
(\S\ref{sec:functional_relation}) that is not Left-total
(\S\ref{sec:left_total}).

$f : A \rightharpoondown B$



% --------------------------------------------------------------------
\subsection{Rice's Theorem}\label{sec:rices_theorem}
% --------------------------------------------------------------------

``For any non-trivial Property of Partial Functions, no General and
Effective method can Decide whether an Algorithm computes a Partial
Function with that Property.''



% ====================================================================
\section{Multimap}\label{sec:multimap}
% ====================================================================

A \emph{Multimap} (or \emph{Multi-valued Function}) is a Left-total
Relation (\S\ref{sec:left_total}) that is not Right-unique
(\S\ref{sec:functional_relation}).

$A \multimap B$



% ====================================================================
\section{Axiomatic Set Theory}\label{sec:axiomatic_set_theory}
% ====================================================================

% --------------------------------------------------------------------
\subsection{Axiom of Choice}\label{sec:choice_axiom}
% --------------------------------------------------------------------

\subsubsection{Tarski's Theorem}\label{sec:tarskis_theorem}

Well-ordering Theorem



% --------------------------------------------------------------------
\subsection{Axiom of Extensionality}\label{sec:extensionality_axiom}
% --------------------------------------------------------------------

\[
  \forall S \forall T
    [S = T \Leftrightarrow \forall R [ R \in S \Leftrightarrow R \in T ]]
\]
cf. \emph{Leibniz Law} (\S\ref{sec:equality}):
\[
  \forall S \forall T
    [S = T \Leftrightarrow \forall R [ S \in R \Leftrightarrow T \in R ]]
\]

% --------------------------------------------------------------------
\subsection{Axiom of Regularity}\label{sec:regularity_axiom}
% --------------------------------------------------------------------

% --------------------------------------------------------------------
\subsection{Zermelo-Fraenkel (ZFC)}\label{sec:zermelo_fraenkel}
% --------------------------------------------------------------------


% --------------------------------------------------------------------
\subsection{Kripke-Platek (KP)}\label{sec:kripke_platek}
% --------------------------------------------------------------------

% --------------------------------------------------------------------
\subsection{New Foundations (NF)}\label{sec:quine_foundations}
% --------------------------------------------------------------------

% --------------------------------------------------------------------
\subsection{Non-well-founded Set Theory}\label{sec:non_wellfounded}
% --------------------------------------------------------------------

Non-standard Analysis (\S\ref{sec:nonstandard_analysis})



% ====================================================================
\section{Set Algebra}\label{sec:set_algebra}
% ===================================================================

% ====================================================================
\section{Algebraic Set Theory}\label{sec:algebraic_set_theory}
% ===================================================================

% ====================================================================
\section{Descriptive Set Theory}\label{sec:descriptive_set_theory}
% ====================================================================

\emph{Boldface Borel Hierarchy} (\S\ref{sec:projective_hierarchy})

% --------------------------------------------------------------------
\subsection{Analytic Set}\label{sec:analytic_set}
% --------------------------------------------------------------------

% --------------------------------------------------------------------
\subsection{Polish Space}\label{sec:polish_space}
% --------------------------------------------------------------------

\subsubsection{Cantor Space}\label{sec:cantor_space}

\emph{Cantor Space}, $2^{\omega}$, is the Set of all Infinite
Sequences of $0$s and $1$s.



\subsubsection{Baire Space}\label{sec:baire_space}

\emph{Baire Space}, $\omega^{\omega}$ or $\mathcal{N}$, is the Set of
all Infinite Sequences of Natural Numbers.



% --------------------------------------------------------------------
\subsection{Pointclass}\label{sec:pointclass}
% --------------------------------------------------------------------

% --------------------------------------------------------------------
\subsection{Effective Descriptive Set Theory}
\label{sec:effective_descriptive}
% --------------------------------------------------------------------

Combination of Descriptive Set Theory with \emph{Recursion Theory}
(Part \ref{part:recursion_theory}).

% --------------------------------------------------------------------
\subsection{Borel Hierarchy}\label{sec:borel_hierarchy}
% --------------------------------------------------------------------



% ====================================================================
\section{Constructive Set Theory}\label{sec:constructive_set_theory}
% ====================================================================

% --------------------------------------------------------------------
\subsection{Apartness}\label{sec:apartness}
% --------------------------------------------------------------------

An \emph{Apartness Relation} is a Binary Relation $\#$ such that:

\begin{enumerate}
\item $\neg (x\#x)$
\item $x\#y \rightarrow y\#x$
\item $x\#y \rightarrow (x\#z \vee y\#z)$
\end{enumerate}

In Models of IZF all Sets with Apartness Relations are Subcountable
(\S\ref{sec:subcountable}).



\subsubsection{Tight}\label{sec:tight}

A \emph{Tight Apartness Relation} is an Apartness Relation that
additionally satisfies:
\[
  \neg (x \# y) \rightarrow x = y
\]



% --------------------------------------------------------------------
\subsection{IZF}\label{sec:izf}
% --------------------------------------------------------------------

% --------------------------------------------------------------------
\subsection{CZF}\label{sec:czf}
% --------------------------------------------------------------------



% ====================================================================
\section{Inner Model Theory}\label{sec:inner_model_theory}
% ====================================================================
