%%%%%%%%%%%%%%%%%%%%%%%%%%%%%%%%%%%%%%%%%%%%%%%%%%%%%%%%%%%%%%%%%%%%%%
%%%%%%%%%%%%%%%%%%%%%%%%%%%%%%%%%%%%%%%%%%%%%%%%%%%%%%%%%%%%%%%%%%%%%%
\part{Set Theory}\label{sec:set_theory}
%%%%%%%%%%%%%%%%%%%%%%%%%%%%%%%%%%%%%%%%%%%%%%%%%%%%%%%%%%%%%%%%%%%%%%
%%%%%%%%%%%%%%%%%%%%%%%%%%%%%%%%%%%%%%%%%%%%%%%%%%%%%%%%%%%%%%%%%%%%%%

\emph{Set Theory} is formulated within First-order Logic
(\S\ref{sec:firstorder_logic}) and as such the objects of Set Theory
are \emph{Sets} (\S\ref{sec:set}) and \emph{Relations}
(\S\ref{sec:relation}). See Axiomatic Set Theory
(\S\ref{sec:axiomatic_set_theory}) for specific Systems of Set Theory.



% ====================================================================
\section{Set}\label{sec:set}
% ====================================================================

A \emph{Set} is a Well-defined (\S\ref{sec:well_defined}) collection
of distinct objects defined by the Property of \emph{Set Membership}
which can be expressed as a Binary Relation
(\S\ref{sec:binary_relation}) ``$\in$''. An object that satisfies this
Relation for a particular Set is a \emph{Member} (or \emph{Element})
of that Set, e.g. $x \in A$ is True when $x$ is a Member of the Set
$A$.

A Set is allowed to have other Sets as Members and conversely a Set is
allowed to be a Member of other Sets. An object in Set Theory that is
not allowed to have Members is called an \emph{Individual} or
\emph{Urelement} (\S\ref{sec:urelement}) and an object that may have
Members but may not be a Member of another object is called a
\emph{Class} (\S\ref{sec:class}).

A Set $A$ may be given by its Extension (\S\ref{sec:extension}), that
is, listing each member in the Set:
\[
  A = \{x,y,z\}
\]
The Intentionsional (\S\ref{sec:intension}) Definition of a Set $B$
may be given by a Property (\S\ref{sec:property}) or other rule that
specifies its Members:
\[
  B = \{ x : x \in \mathbb{N}_0 \wedge x < 4 \}
\]
defines the Set $B$ with Extension $\{ 0, 1, 2, 3 \}$.

By the Axiom of Extensionality (\S\ref{sec:extensionality_axiom}) Sets
are uniquely defined by their constituent Elements and therefore Sets
with multiple Members that are Identical are Equal and the order in
which Elements are given is unimportant:
\[
  \{ 2, 3 \} = \{ 3, 2 \} = \{ 2, 3, 3, 2 \}
\]



% --------------------------------------------------------------------
\subsection{Subset}\label{sec:subset}
% --------------------------------------------------------------------

For two Sets $A$ and $B$, if all the Members of $A$ are also Members
of $B$, then $A$ is a \emph{Subset} of $B$, denoted $A \subseteq B$:
\[
  (\forall x) x \in A \rightarrow x \in B \Rightarrow A \subseteq B
\]
By this definition a Set is a Subset of itself, $A = B \Rightarrow A
\subseteq B$, but a \emph{Proper Subset} may be defined as a Subset
that is not equal to the containing Set, $A \subset B$:
\[
  A \subseteq B \wedge B \nsubseteq A \Rightarrow A \subset B
\]



% --------------------------------------------------------------------
\subsection{Powerset}\label{sec:powerset}
% --------------------------------------------------------------------

\subsubsection{Cantor's Theorem}\label{sec:cantors_theorem}



% --------------------------------------------------------------------
\subsection{Partition}\label{sec:set_partition}
% --------------------------------------------------------------------

A \emph{Partition} of a Set $X$ is a Set of Non-empty Subsets of $X$
such that every Element of $X$ is in exactly one Subset, thus $X$ is
the Disjoint Union (\S\ref{sec:disjoint_union}) of the Subsets. A
Partition $P$ of $X$ has the Properties:
\begin{enumerate}
  \item $\emptyset \notin P$
  \item $\bigcup_{A \in P}A = X$
  \item $A,B \in P \wedge A \neq B \Rightarrow A \cap B = \emptyset$
\end{enumerate}



% --------------------------------------------------------------------
\subsection{Cardinality}\label{sec:cardinality}
% --------------------------------------------------------------------

The \emph{Cardinality} (or \emph{Size}) of a Set $A$, denoted $|A|$ or
$card(A)$, is a measure of the number of Elements in the Set. Two Sets
with the same Cardinality are said to be \emph{Equinumerous}.

The unique Set with Cardinality 0 is called the \emph{Empty Set} and
is denoted $\{\}$ or $\varnothing$. A Set with Cardinality 1 is called a
\emph{Singleton Set}.

Cardinal Numbers (\S\ref{sec:cardinal_number})

Bijection (\S\ref{sec:bijective_function}) and Injection
(\S\ref{sec:injective_function})

Cantor's Theorem (\S\ref{sec:cantors_theorem})

The Axiom of Choice (\S\ref{sec:choice_axiom}) Implies the Law of
Trichotomy (\S\ref{sec:trichotomy_law}) for Cardinality which assigns
a given Set $X$ to one of the Classes:

\begin{description}
\item [Finite] $|X| < |\mathbb{N}|$ (\S\ref{sec:finite})
\item [Countably Infinite] $|X| = |\mathbb{N}| = \aleph_0$
  (\S\ref{sec:countably_infinite})
\item [Uncountably Infinite] $|\mathbb{N}| < |X|$
  (\S\ref{sec:uncountably_infinite})
\end{description}



\subsubsection{Finite}\label{sec:finite}

\subsubsection{Countably Infinite}\label{sec:countably_infinite}

\subsubsection{Uncountably Infinite}\label{sec:uncountably_infinite}

\subsubsection{Subcountable}\label{sec:subcountable}

A Set $X$ is \emph{Subcountable} if there is a Partial Surjection from
$\mathbb{N}$ onto the $X$, that is, if $X$ is Finite
(\S\ref{sec:finite}) or Countably Infinite
(\S\ref{sec:countably_infinite})



% --------------------------------------------------------------------
\subsection{Index Set}\label{sec:index_set}
% --------------------------------------------------------------------

An \emph{Index Set} is one that \emph{Indexes} (or \emph{Labels})
Members of another Set. Indexing is a Surjective Function
(\S\ref{sec:function}) from an Index Set onto a target Set.



\subsubsection{Indexed Family}\label{sec:indexed_family}

An \emph{Indexed Family} of Sets is a Function from an Index Set to
the Class (\S\ref{sec:class}) of all Sets. For Index Set $J$ and
Indexed Set $A$, the Indexed Family may be denoted
\[
  (A_j)_{j \in J}
\]



% --------------------------------------------------------------------
\subsection{Cover}\label{sec:cover}
% --------------------------------------------------------------------

A \emph{Cover} of a Set $X$ is an Indexed Family of Sets $C = \{ U_i :
i \in I \}$ such that their Union contains $X$:
\[
  X \subseteq \bigcup_{i \in I} U_i
\]



% --------------------------------------------------------------------
\subsection{Transitive Set}\label{sec:transitive_set}
% --------------------------------------------------------------------

A Set, $A$, is \emph{Transitive} if and only if:
\[
  \bigcup A \subseteq A
\]
That is for each non-empty Set $B \in A$:
\[
  B \in A \rightarrow B \subset A
\]
Transitivity for Classes (\S\ref{sec:class}) is defined in the same
way.

For two Transitive Sets, $A$ and $B$, the Set $A \cup B \cup \{A,B\}$
is Transitive.

A Set, $B$, containing no Urelements is Transitive if and only if $A
\subset \mathcal{P}(X)$



\subsubsection{Admissible Set}\label{sec:admissible_set}

An \emph{Admissible Set}, $A$, is a Transitive Set such that $\langle
A, \in \rangle$ is a Model (\S\ref{sec:model}) of Kripke-Platek Set
Theory (\S\ref{sec:kripke_platek}).

The smallest example of an Admissible Set is the Set of
\emph{Hereditarily Finite Sets}. %FIXME ref hereditarily finite



% --------------------------------------------------------------------
\subsection{Pointed Set}\label{sec:pointed_set}
% --------------------------------------------------------------------

A \emph{Pointed Set} (or \emph{Based Set} or \emph{Rooted Set}) is an
Ordered Pair $(X, x_0)$ where $X$ is a Set and $x_0 \in X$ is the
\emph{Basepoint} of $X$. This defines an Algebraic Structure
(\S\ref{sec:algebraic_structure}) on $X$ with a single Nullary
Function that returns the Basepoint.



% ===================================================================
\section{Set Operation}\label{sec:set_operation}
% ===================================================================

% -------------------------------------------------------------------
\subsection{Union}\label{sec:union}
% -------------------------------------------------------------------

\emph{Union}



\subsubsection{Disjoint Union}\label{sec:disjoint_union}

Given a Family (\S\ref{sec:family}) of Sets ${A_i : i \in I}$,
the \emph{Disjoint Union} is defined as:
\[
  \bigsqcup_{i \in I} A_i = \bigcup_{i \in I} \{(x,i) | x \in A_i \}
\]



\subsubsection{Infinitary Union}\label{sec:infinitary_union}

\[
  x \in \bigcup S \leftrightarrow \exists y \in S : x \in y
\]



% -------------------------------------------------------------------
\subsection{Intersection}\label{sec:intersection}
% -------------------------------------------------------------------

% -------------------------------------------------------------------
\subsection{Relative Complement}\label{sec:relative_complement}
% -------------------------------------------------------------------

The \emph{Relative Complement} (or \emph{Difference}) of two Sets $A$
and $B$, denoted $A \setminus B$, is the Set of Elements in $A$ but
not $B$.



% -------------------------------------------------------------------
\subsection{Symmetric Difference}\label{sec:symmetric_difference}
% -------------------------------------------------------------------

The \emph{Symmetric Difference} of two Sets $A$ and $B$, denoted $A
\oplus B$, is the Set of Elements in $A$ or $B$ but not both $A$ and
$B$:
\[
  A \oplus B =
  \{ x : (x \in A \vee x \in B) \wedge x \notin A \cap B \}
\]
or:
\[
  A \oplus B = A \setminus B \cup B \setminus A
\]



% -------------------------------------------------------------------
\subsection{Absolute Complement}\label{sec:absolute_complement}
% -------------------------------------------------------------------

The \emph{Absolute Complement} (or \emph{Complement}) of a Set $A$ is
the Relative Complement of $A$ with the given Universe
(\S\ref{sec:universe}) $\mathcal{U}$:
\[
  \mathcal{U} \setminus A
\]



% -------------------------------------------------------------------
\subsection{Cartesian Product}\label{sec:cartesian_product}
% -------------------------------------------------------------------

% -------------------------------------------------------------------
\subsection{de Morgan's Law}\label{sec:de_morgan}
% -------------------------------------------------------------------



% ====================================================================
\section{Multiset}\label{sec:multiset}
% ====================================================================

A \emph{Multiset} (or \emph{Bag}) is a 2-tuple $(A,m)$ of an
\emph{Underlying Set}, $A$, together with a \emph{Multiplicity
  Function}, $m : A \rightarrow \mathbb{N}_{\geq 1}$, mapping Elements
of $A$ to Positive Natural Numbers representing the
\emph{Multiplicity} Elements, that is the number of times an Element
occurs in the Multiset. A Multiset may be denoted with square
brackets:
\[
  [a,a,b]
\]
If the Underlying Set is restricted to a Subset of a given
\emph{Universe}, $U$, the Multiplicity Function may be extended to
$m_U : U \rightarrow \mathbb{N}$ where $a \in U, a \notin A
\leftrightarrow m(a)=0$.

\emph{Indicator Function} (\S\ref{sec:indicator_function})



% ====================================================================
\section{Urelement}\label{sec:urelement}
% ====================================================================

An \emph{Urelement} (or \emph{Atom} or \emph{Individual}) is an Object
that may be an Element of a Set, but is not itself a Set.

Urelements are the dual to Proper Classes (\S\ref{sec:proper_class})
as an Urelement cannot have Members while a Proper Class cannot be a
Member.

A Two-sorted First-order Theory with Sets and Urelements has $a \in b$
defined only when $b$ is a Set.

A One-sorted First-order Theory may be defined with an Unary Relation
distinguishing Sets and Urelements, which can be achieved if the Unary
Relation can at least distinguish Urelements from the Empty Set, since
all other Sets have Members. The Axiom of Extensionality must also be
formulated to apply only to Sets and not Urelements.



% ====================================================================
\section{Class}\label{sec:class}
% ====================================================================

A \emph{Class} is any Subset of the \emph{Universe}
(\S\ref{sec:universe}) of discussion.

\begin{description}
  \item [Proper Class] a Class that is not a Set
  \item [Small Class] a Class that is a Set
\end{description}



% --------------------------------------------------------------------
\subsection{Proper Class}\label{sec:proper_class}
% --------------------------------------------------------------------

A \emph{Proper Class} is an Object that cannot be a Member of another
Object. This is the Dual concept of an Urelement which cannot have
another Object as a Member.



% --------------------------------------------------------------------
\subsection{Subclass}\label{sec:subclass}
% --------------------------------------------------------------------



% ====================================================================
\section{Family}\label{sec:family}
% ====================================================================

A \emph{Family} is a collection of Sets that is allowed to be a
Multiset (\S\ref{sec:multiset}) and/or a Small or Proper Class
(\S\ref{sec:class}).



% ====================================================================
\section{Universe}\label{sec:universe}
% ====================================================================

% FIXME differentiate between Universes
A \emph{Universe} is a Set, $U$, with the following Closure Properties
\cite{maclane69}:
\begin{enumerate}
\item $x \in A \in U \rightarrow x \in U$
\item $x \in U \wedge y \in U \rightarrow \{x,y\}, \langle x,y
  \rangle, x \times y \in U$
\item $x \in U \rightarrow \mathcal{P}(x) \in U \wedge \bigcup x \in U$
\item $\omega = \{0,1,2,\ldots\} \in U$
\item Given a Surjective Function, $f : a \rightarrow b, a \in
  U, b \subset U \rightarrow b \in U$
\end{enumerate}



% --------------------------------------------------------------------
\subsection{Small Set}\label{sec:small_set}
% --------------------------------------------------------------------

A \emph{Small Set} may be said to be a member of a Universe that is
not itself a Universe.



% --------------------------------------------------------------------
\subsection{Cumulative Hierarchy}\label{sec:cumulative_hierarchy}
% --------------------------------------------------------------------

% --------------------------------------------------------------------
\subsection{Superstructure}\label{sec:superstructure}
% --------------------------------------------------------------------

An Universe may be generated over a Set resulting in a
\emph{Superstructure}. The Superstructure over a Set $X$:
\[
  \mathbf{S}X := \bigcup^{\infty}_{i=0}\mathbf{S}_i X
\]
can be defined by Structural Recursion
(\S\ref{sec:recursive_definition}) as follows:
\begin{itemize}

\item $\mathbf{S}_0 X = X$
\item $\mathbf{S}_1 X = X \cup \mathcal{P}(X)$
\item $\mathbf{S}_{n+1} X =
  \mathbf{S}_n X \cup \mathcal{P}(\mathbf{S}_n X)$

\end{itemize}
Some Properties of $\mathbf{S}\{\}$ (the Superstructure over the Empty
Set):
\begin{itemize}

\item $\mathbb{N} \subset \mathbf{S}\{\}$
\item $\mathbb{N} \notin \mathbf{S}\{\}$ (Elements of $\mathbf{S}\{\}$
  are Finite Sets)
\item $\mathbf{S}\{\}$ contains all of the Hereditarily Finite Sets
%FIXME ref Hereditarily finite sets

\end{itemize}

The Superstructure over $\mathbb{N}$, $\mathbf{S}\mathbb{N}$, is
considered the \emph{Universe of Ordinary Mathematics}.



% --------------------------------------------------------------------
\subsection{Von Neumann Universe}\label{sec:vonneumann_universe}
% --------------------------------------------------------------------

% --------------------------------------------------------------------
\subsection{Grothendieck Universe}\label{sec:grothendieck_universe}
% --------------------------------------------------------------------

%FIXME similar to definitions above

% --------------------------------------------------------------------
\subsection{Forcing}\label{sec:forcing}
% --------------------------------------------------------------------



% ====================================================================
\section{Relation}\label{sec:relation}
% ====================================================================

A $n$-ary \emph{Relation} $L$ over Sets $X_1, \ldots, X_n$ is defined
in Extension (\S\ref{sec:extension}) as a Subset of the Cartesian
Product $X_1 \times \ldots \times X_n$ called the \emph{Graph} of $L$:
\[
  G(L) \subseteq X_1 \times \ldots \times X_n
\]
When a Graph is actually plotted on a Coordinate Plane
(\S\ref{sec:cartesian_coordinate}), the First Elements of the Ordered
Pairs of $G(L)$, mapped to the Horizontal Axis, are \emph{Abscissae},
and the Second Elements of the Ordered Pairs of $G(L)$, mapped to the
Vertical Axis, are called \emph{Ordinates}.

The Sets $X_1, \ldots, X_n$ are called the \emph{Domain} of $L$. If
any Domain of $L$ is empty, then $L$ is the unique \emph{Empty
  Relation} $L = \varnothing$.

An $n$-ary Relation may be completely specified by an $n + 1$-tuple
called a \emph{Correspondence} (or \emph{Embedded} or \emph{Included
  Relation}):
\[
  (X_1, \ldots, X_n, G(L))
\]
The Expression $L x_1 \ldots x_n$ where $x_i \in X_i$ is True when
$(x_1, \ldots, x_n) \in G(L)$ and False otherwise. As such, an $n$-ary
Relation may also be defined by a Boolean-valued \emph{Characteristic
  Function} (or \emph{Indicator Function}
\S\ref{sec:indicator_function}, cf. Predicate \S\ref{sec:predicate}):
\[
  f_L : X_1 \times \ldots \times X_n \rightarrow \{\top,\bot\}
\]



% --------------------------------------------------------------------
\subsection{Finitary Relation}\label{sec:finitary_relation}
% --------------------------------------------------------------------

A Relation with a Finite Arity is called a \emph{Finitary Relation}.

There are only two $0$-ary Relations on the Empty Tuple $()$: one that
is always True and one that is always False.

A $2$-ary Relation is usually called a Binary Relation
(\S\ref{sec:binary_relation}), and a $3$-ary Relation may be called a
Ternary Relation.



% --------------------------------------------------------------------
\subsection{Binary Relation}\label{sec:binary_relation}
% --------------------------------------------------------------------

A \emph{Binary} (or \emph{Dyadic}) Relation is a $2$-ary Relation.

A Binary Relation $R$ on a Set $A$ is:
\begin{description}
\item[Serial](\S\ref{sec:serial_relation}) if:
  \[
    \forall a \in A \exists b \in A : xRy
  \]
\item[Reflexive](\S\ref{sec:serial_relation}) if:
  \[
    xRx = \top
  \]
  Reflexive implies Transitive and Serial
\item[Irreflexive] (also \emph{Strict}
  \S\ref{sec:irreflexive_relation}) if:
  \[
    xRx = \bot
  \]
\item[Co-reflexive](\S\ref{sec:coreflexive_relation}) if:
  \[
    xRy \rightarrow a = b
  \]
\item[Transitive](\S\ref{sec:transitive_relation}) if:
  \[
    xRy \wedge yRc \rightarrow xRc
  \]
  Transitive and Irreflexive if and only if Transitive and Asymmetric
\item[Symmetric](\S\ref{sec:symmetric_relation}) if:
  \[
    xRy \leftrightarrow yRx
  \]
\item[Anti-symmetric](\S\ref{sec:antisymmetric_relation}) if:
  \[
    xRy \wedge yRx \rightarrow a = b
  \]
\item[Asymmetric](\S\ref{sec:antisymmetric_relation}) if both
  Anti-symmetric and Irreflexive:
  \[
    xRy \rightarrow \neg yRx
  \]
\item[Left-total] (\S\ref{sec:lefttotal_relation}) if:
  \[
    \forall a \in A \exists b \in A : xRy
  \]
  Multimaps (\S\ref{sec:multimap}) and Functions
  (\S\ref{sec:function}) are Left-total
\item[Right-total] (\emph{Surjective} or \emph{Onto}
  (\S\ref{sec:surjective_function})) if:
  \[
    \forall b \in A \exists a \in A : xRy
  \]
\item[Right-unique] (or \emph{Functional} or \emph{Univalent}
  \S\ref{sec:functional_relation}) if:
  \[
    \forall x \in dom(R), y \in rng(R)
    (xRy \wedge xRz \rightarrow y = z)
  \]
\item[Total] (\S\ref{sec:total_relation}) if:
  \[
    \forall a,b \in A, xRy \vee yRx
  \]
  Total implies Reflexive (\S\ref{sec:reflexive_relation})
\item[Trichotomous] (\S\ref{sec:trichotomous_relation}) if:
  \[
    xRy \vee yRx \vee a = b
  \]
  Trichotomous implies Irreflexive (\S\ref{sec:irreflexive_relation}).
  Trichotomous and Transitive (\S\ref{sec:transitive_relation})
  implies Asymmetric (\S\ref{sec:asymmetric_relation}). A Transitive
  Trichotomous Relation is a Strict Total Order
  (\S\ref{sec:strict_order}, \S\ref{sec:total_order}).
\item[Right Euclidean] (\S\ref{sec:euclidean_relation}) if:
  \[
    xRy \wedge xRc \rightarrow yRc
  \]
\item[Left Euclidean] (\S\ref{sec:euclidean_relation}) if:
  \[
    yRx \wedge cRx \rightarrow yRc
  \]
\end{description}
From the above Classes of Relations, the following Orders
(\S\ref{sec:order_theory}) are distinguished, listed here from most
general to most restricted:
\begin{description}
\item[Preorder] (or Quasiorder) when Reflexive and Transitive (all
  Partial Orders and Equivalence Relations
  (\S\ref{sec:equivalence_relation}) are Preorders)
\item[Partial Order] when Preorder and Anti-symmetric (Poset
  \S\ref{sec:poset})
\item[Total Preorder] (or Weak Order) when Preorder and Total.
\item[Total Order] when a Partial Order and Total.
\item[Partial Equivalence] when Symmetric and Transitive.
\item[Equivalence] when Reflexive, Symmetric, and Transitive
  (\S\ref{sec:equivalence_relation}).
\end{description}

The Set of all Binary Relations on a Set $A$ is denoted
$\mathbf{Rel}(A)$ and is the Power Set of $A \times A$: $2^{A \times
  A}$.

A Dependency Relation is a Binary Relation that is Symmetric and
Reflexive.

Undirected Graphs (\S\ref{sec:undirected_graph}) are Symmetric. Any
Binary Relation may be a Directed Graph (\S\ref{sec:directed_graph}).
A Binary Relation is a Complete Graph (\S\ref{sec:complete_graph})
when $a \neq b \rightarrow xRy$ and implies Symmetry. A Binary
Relation is a Tournament when $a \neq b \rightarrow xRy \vee yRx$ and
implies Anti-symmetry.



\subsubsection{Reflexive Relation}\label{sec:reflexive_relation}

\subsubsection{Irreflexive Relation}\label{sec:irreflexive_relation}

\subsubsection{Coreflexive Relation}\label{sec:coreflexive_relation}

\subsubsection{Transitive Relation}\label{sec:transitive_relation}

\subsubsection{Symmetric Relation}\label{sec:symmetric_relation}

\subsubsection{Antisymmetric Relation}\label{sec:antisymmetric_relation}

\subsubsection{Asymmetric Relation}\label{sec:asymmetric_relation}

\subsubsection{Left-total Relation}\label{sec:lefttotal_relation}

\subsubsection{Total Relation}\label{sec:total_relation}

\subsubsection{Functional Relation}\label{sec:functional_relation}

\subsubsection{Trichotomous Relation}\label{sec:trichotomous_relation}

\subsubsection{Serial Relation}\label{sec:serial_relation}

\subsubsection{Rewrite Relation}\label{sec:rewrite_relation}

\emph{Rewrite Relation}

\emph{Rewrite Preorder}

\emph{Reduction Preorder}

\emph{Rewrite Closure}



\subsubsection{Extensional Relation}\label{sec:extensional_relation}

A Binary Relation $R$ is \emph{Extensional} if and only if:
\[
  \forall x,y (x = y \leftrightarrow
    \forall z (R(x,z) \leftrightarrow R(y,z))
\]
The Converse of an Extensional Relation is not necessarily
Extensional.

\subsubsection{Euclidean Relation}\label{sec:euclidean_relation}

\emph{Left-euclidean} \emph{Right-euclidean}



% --------------------------------------------------------------------
\subsection{Endorelation}\label{sec:endorelation}
% --------------------------------------------------------------------

\begin{description}
\item [Directed Graph] (\S\ref{sec:directed_graph})

\item [Undirected Graph] (\S\ref{sec:undirected_graph}) Irreflexive, Symmetric

\item [Tournament] (\S\ref{sec:tournament}) Irreflexive, Antisymmetric

\item [Dependency] (\S\ref{sec:dependency_relation}) Reflexive, Symmetric

\item [Weak Order] (\S\ref{sec:weak_order}) Transitive

\item [Preorder] (\S\ref{sec:preorder}) Reflexive, Transitive

\item [Partial Order] (\S\ref{sec:partial_order}) Reflexive,
  Antisymmetric, Transitive

\item [Strict Partial Order] (\S\ref{sec:strict_order},
  \S\ref{sec:partial_order}) Irreflexive, Antisymmetric, Transitive

\item [Partial Equivalence] (\S\ref{sec:partial_equivalence})
  Symmetric, Transitive

\item [Equivalence Relation] (\S\ref{sec:equivalence_relation})
  Reflexive, Symmetric, Transitive

\end{description}



% --------------------------------------------------------------------
\subsection{Equivalence Relation}\label{sec:equivalence_relation}
% --------------------------------------------------------------------

An \emph{Equivalence Relation} on a Set $X$ is a Reflexive, Symmetric,
and Transitive Relation that Partitions (\S\ref{sec:set_partition})
the Set into Disjoint Subsets called \emph{Equivalence Classes}
(\S\ref{sec:equivalence_class}).

The Equivalence Class of an Element $a \in X$ with Equivalence
Relation $\sim$ is the Subset of $X$ defined as:
\[
    [a] = \{x \in X | a \sim x\}
\]
The \emph{Quotient Set} is the Set of Equivalence Classes of a
particular Equivalence Relation.

For a Function between Sets $f : S \rightarrow T$, one may define an
Equivalence Relation on Elements $a,b \in S$:
\[
    a \sim b \Leftrightarrow f(a) = f(b)
\]
where the Equivalence Classes are the Fibers (\S\ref{sec:function}) of
the Elements in $T$ and the Quotient Set is the Image of the
Equivalence Relation viewed as a Function on Elements of $S$ to their
Equivalence Classes.



\subsubsection{Equivalence Class}\label{sec:equivalence_class}

(or \emph{Quotient})



\subsubsection{Partial Equivalence}\label{sec:partial_equivalence}

\subsubsection{Congruence Relation}\label{sec:congruence_relation}



% --------------------------------------------------------------------
\subsection{Ordering Relation}\label{sec:ordering_relation}
% --------------------------------------------------------------------

Order Theory (\S\ref{sec:order_theory})



\subsubsection{Upper Bound}\label{sec:upper_bound}

\paragraph{Least Upper Bound}\label{sec:least_upperbound}
\hfill \\

Supremum (\S\ref{sec:glb_lub})



\subsubsection{Lower Bound}\label{sec:lower_bound}

\paragraph{Greatest Lower Bound}\label{sec:greatest_lowerbound}
\hfill \\

Infimum (\S\ref{sec:glb_lub})



\subsubsection{Cofinal}\label{sec:cofinal}

A Subset $B$ is \emph{Cofinal} in a Set $A$ with an Ordering Relation
$\leq$ if:
\[
  \forall a \in A, \exists b \in B : a \leq b
\]



\paragraph{Coinitial}\label{sec:coinitial}
\hfill \\

A \emph{Coinitial Subset} is the Order Theoretic Dual of a Cofinal
Subset. A Subset $B$ is Coinitial in a Set $A$ with an Ordering
Relation $\leq$ if:
\[
  \forall a \in A, \exists b \in B : b \leq a
\]



\subsubsection{Weak Order}\label{sec:weak_order}

\subsubsection{Preorder}\label{sec:preorder}

Reflexive, Transitive


\paragraph{Cartesian Closed Preorder}\label{sec:cartesian_preorder}
\hfill \\

Heyting Algebra (\S\ref{sec:heyting_algebra})



\subsubsection{Directed Set}\label{sec:directed_set}

A \emph{Directed Set} $D$ has a Preorder $\leq$ with the extra
condition that any two Elements of $D$ have an Upper Bound
(\S\ref{sec:upper_bound}):
\[
  \forall x, y \in D, \exists z \in D : x \leq z \wedge y \leq z
\]



\subsubsection{Partial Order}\label{sec:partial_order}

Reflexive, Antisymmetric, Transitive

The Signature (\S\ref{sec:signature}) of Partial Orderings is
$\{\prec\}$.



\paragraph{Strict Order}\label{sec:strict_order}

A \emph{Strict Order} is an Irreflexive Partial Order (e.g. $<$).
There is a one-to-one correspondence between Strict and Non-strict
Partial Orders via the Irreflexive Kernel
(\S\ref{sec:reflexive_reduction}):
\[
  a \leq b \wedge a \neq b \Rightarrow a < b
\]
and conversely the Reflexive Closure (\S\ref{sec:reflexive_closure}):
\[
  a < b \vee a = b \Rightarrow a \leq b
\]



\subsubsection{Total Order}\label{sec:total_order}

A \emph{Total Order} (or \emph{Linear Order}) adds the requirement of
Left-totality. A Totally Ordered Subset of some Partially Ordered Set
is called a \emph{Chain}.



\subsubsection{Complete Partial Order}\label{sec:complete_partialorder}

\subsubsection{Well-founded Relation}\label{sec:well_founded}

\emph{Well-founded}

Every non-empty Subset has a Minimal Element



\paragraph{Noetherian Relation}\label{sec:noetherian_relation}
\hfill \\

\emph{Noetherian} (or \emph{Terminating})



% --------------------------------------------------------------------
\subsection{Tolerance Relation}\label{sec:tolerance_relation}
% --------------------------------------------------------------------

\subsubsection{Dependency Relation}\label{sec:dependency_relation}
\hfill \\

Finite Tolerance Relation



% --------------------------------------------------------------------
\subsection{Relation Join}\label{sec:relation_join}
% --------------------------------------------------------------------

\emph{Relational Algebra} (\S\ref{sec:relational_algebra})

\emph{Pullback} (\S\ref{sec:pullback})



\subsubsection{Relation Composition}\label{sec:relation_composition}

For two Binary Relations, $R \subseteq X \times Y$ and $S \subseteq Y
\times Z$, the \emph{Relation Composition}, $S \circ R \in X \times Z$
is defined as:
\[
  S \circ R = \{(x,z) \in X \times Z \;|\;
  \exists y \in Y : (x,y) \in R \wedge (y,z) \in S \}
\]

See also Function Composition (\S\ref{sec:function_composition})



% --------------------------------------------------------------------
\subsection{Closure}\label{sec:set_closure}
% --------------------------------------------------------------------

A Set, $A$, is \emph{Closed} (has \emph{Closure}) under a Relation,
$L$, if for every $(x,y) \in L$, $y \in dom(L)$:
\[
  (\forall x \in A) (x,y) \in L \Rightarrow y \in A
\]



\subsubsection{Transitive Closure}\label{sec:transitive_closure}

For a Relation, $R$, on a Set, $X$, the \emph{Transitive Closure}
$R^+$ is a Relation on $X$ such that $R \subseteq R^+$ and $R^+$ is
\emph{Minimal} (the smallest Relation closed under Relation
Composition):
\[
  R^+ = \bigcup_{i \in \{1,2,3,...\}} R^i
\]
If $R$ is Transitive then $R = R^+$.

\emph{Reachability} (\S\ref{sec:dag})



\subsubsection{Reflexive Closure}\label{sec:reflexive_closure}

The \emph{Reflexive Closure}, $X$, of a Binary Relation, $R$, on a
Set, $S$, is the smallest Reflexive Relation on $S$ that contains $R$:
\[
  X = R \cup \{(x,x) : x \in S\}
\]
that is, the Union of $R$ with the Equivalence Relation $=$.

The Reflexive Closure of $<$ is $\leq$.



\paragraph{Reflexive Reduction}\label{sec:reflexive_reduction}
\hfill \\

The \emph{Reflexive Reduction} (or \emph{Irreflexive Kernel}) of a
Binary Relation $R$ on a Set $S$ is the smallest Relation $Y$ such
that $Y$ has the same Reflexive Closure as $R$:
\[
  Y = (R\;\setminus=)
\]
The Reflexive Reduction of $x \leq y$ is $x < y$.



\subsubsection{Symmetric Closure}\label{sec:symmetric_closure}

The \emph{Symmetric Closure}, $X$, of a Binary Relation, $R$, on a
Set, $S$, is the Union of $R$ with its Inverse Relation:
\[
  R \cup R^{-1} = X = R \cup \{(x,y) : (y,x) \in R\}
\]



\subsubsection{Finitary Closure}\label{sec:finitary_closure}

\subsubsection{Rewrite Closure}\label{sec:rewrite_closure}

\subsubsection{$P$-closure}\label{sec:p_closure}



% ====================================================================
\section{Function}\label{sec:function}
% ====================================================================

A \emph{Function} is a Left-total (\S\ref{sec:lefttotal_relation})
Right-unique (\S\ref{sec:functional_relation}) Relation. A Function
$f$ with Domain $A$ and Codomain $B$ is denoted:
\[
  f : A \rightarrow B
\]
The Domain (or \emph{Input}) of $f$ is denoted $dom(f)$ and the
Codomain (or \emph{Ouptut}) as $cod(f)$.

In the case where $img(f) = cod(f)$, $f$ is known as a
\emph{Surjective Function} (\S\ref{sec:surjective_function}).

The Extension (\S\ref{sec:extension}) of a Function may be called a
\emph{Graph} of the Function. Equality of Functions is defined such
that Equal Functions have the same Output for a given Input (equal in
Extension).

An \emph{Empty Function} has the Empty Set as a Domain, defining a
Unique Function for each Set, $A$:
\[
  f_A : \varnothing \rightarrow A
\]

\emph{Finitary Function}

\emph{Infinitary Function}

$\rightarrow$

Injection, Monomorphism: $\rightarrowtail$

Surjection, Epimorphism: $\twoheadrightarrow$

Bijection: $\rightarrowtail \hspace{-8pt} \twoheadrightarrow$

Isomorphism: $\xrightarrow{\sim}$

Inclusion Map: $\hookrightarrow$

Partial Function: $\nrightarrow$

Multimap: $\multimap$

$\leftrightarrow$

$\nleftrightarrow$

$\rightrightarrows$

$\leftrightarrows$

$\leftrightharpoons$

$\curvearrowright$

$\circlearrowleft$

$\dashrightarrow$

$\looparrowright$

Real-valued Function (\S\ref{sec:real_function})



% --------------------------------------------------------------------
\subsection{Functional Predicate}\label{sec:functional_predicate}
% --------------------------------------------------------------------

% --------------------------------------------------------------------
\subsection{Image}\label{sec:image}
% --------------------------------------------------------------------

The \emph{Image} of the Function is the Subset of the Codomain that
the Function actually Maps to, and the Subset of the Domain that Maps
to it is called the \emph{Inverse Image} or \emph{Preimage}
(\S\ref{sec:preimage}).

Because a Function is Right-unique, each Element of the Preimage Maps
to one Element of the Image. This Property is expressed as:
\[
  (a,b) \in f \wedge (a,c) \in f \rightarrow b = c
\]
A Left-total Relation without this Property is known as a
\emph{Multimap} (\S\ref{sec:multimap}).



\subsection{Preimage}\label{sec:preimage}

\emph{Preimage} (or \emph{Inverse Image})



\subsubsection{Fiber}\label{sec:fiber}

For a Function $f : A \rightarrow B$, the \emph{Fiber} of an Element
$y \in img(f)$ is the Preimage of the Singleton Set $\{y\}$.

Fiber (Topology) \S\ref{sec:point_fiber}



% --------------------------------------------------------------------
\subsection{Restriction}\label{sec:restriction}
% --------------------------------------------------------------------

% --------------------------------------------------------------------
\subsection{Extension}\label{sec:function_extension}
% --------------------------------------------------------------------

(or \emph{Overriding Union})



% --------------------------------------------------------------------
\subsection{Kernel}\label{sec:kernel}
% --------------------------------------------------------------------

The \emph{Kernel} of a Function $f : X \rightarrow Y$, $ker(f)$, is an
Equivalence Relation defined as:
\[
  ker(f) = \{ (x,x') \in X \times X : f(x) = f(x') \}
\]



% --------------------------------------------------------------------
\subsection{Function Composition}\label{sec:function_composition}
% --------------------------------------------------------------------

Given two Functins $f : A \rightarrow B$ and $g : B \rightarrow C$,
there is a \emph{Composite Function}:
\[
  g \circ f : A \rightarrow C
\]
where $(g \circ f)(a) = g(f(a))$ and $a \in A$.

The \emph{Composition Operation} $\circ$ is Associative: $(h \circ g)
\circ f = h \circ (g \circ f)$. Composition of Functions may be
represented with the $\circ$ elided: $gf$.

Identity Element for the Composition Operation is the Identity
Function (\S\ref{sec:identity_function}) on Sets. The Identity
Function for a Set $A$:
\[
  I_A : A \rightarrow A
\]
is defined as:
\[
  I_A(a) = a
\]
with the result given the Function $f$ above:
\[
  f \circ I_A = f = I_B \circ f
\]



% --------------------------------------------------------------------
\subsection{Fixed Point}\label{sec:fixed_point}
% --------------------------------------------------------------------

A \emph{Fixed Point}, $c$, of a Function $f$, is an Element of the
Domain of $f$ that is Mapped to itself by $f$:
\[
  f(c) = c
\]

Fixed-point Theorems



\subsubsection{Periodic Point}\label{sec:periodic_point}

A \emph{Periodic Point} is an Element of the Domain of a Function that
is returned to after a finite number of iterations.



\subsubsection{Least \& Greatest Fixpoint}
\label{sec:leastgreatest_fixpoint}



% --------------------------------------------------------------------
\subsection{Idempotent Function}\label{sec:idempotent}
% --------------------------------------------------------------------

A Function, $f$, is \emph{Idempotent} if it maps each Element of
$dom(f)$ to a Fixed Point of $f$:
\[
  f^2 = f
\]



% --------------------------------------------------------------------
\subsection{Identity Function}\label{sec:identity_function}
% --------------------------------------------------------------------

\subsubsection{Inclusion Map}\label{sec:inclusion_map}

An \emph{Inclusion Map} (or \emph{Inclusion Function}) is an Identity
Function that Maps Elements of a Subset to those in a Superset:
\[
  \iota : A \hookrightarrow X
\]
where $A \subseteq X$.



% --------------------------------------------------------------------
\subsection{Inverse Function}\label{sec:inverse_function}
% --------------------------------------------------------------------

$f^{-1}$



\subsubsection{Left Inverse}\label{sec:left_inverse}

For a Function $f: X \rightarrow Y$, a \emph{Left Inverse} or
\emph{Retraction} of $f$ is a Function $g: Y \rightarrow X$ such that
$gf = Id(X)$.



\subsubsection{Right Inverse}\label{sec:right_inverse}

For a Function $f: X \rightarrow Y$, a \emph{Right Inverse} or
\emph{Section} of $f$ is a Function $h: Y \rightarrow X$ such that $fh
= Id(Y)$.



% --------------------------------------------------------------------
\subsection{Injective Function}\label{sec:injective_function}
% --------------------------------------------------------------------

An \emph{Injective Function} (or \emph{One-to-one Function} or
\emph{Injection}) is one where the Elements of the Codomain are the
Images of at most one Elements of the Domain. A Function that is
Non-injective is considered an \emph{Information Losing Function}
because the Inverse is no longer a Function but it is a
\emph{Multimap} (\S\ref{sec:multimap}).



% --------------------------------------------------------------------
\subsection{Surjective Function}\label{sec:surjective_function}
% --------------------------------------------------------------------

A Function $f$ with $img(f) = cod(f)$ is a \emph{Surjective Function}
(or \emph{Surjection}). Such a Function may be said to be \emph{Onto}
$cod(f)$.



% --------------------------------------------------------------------
\subsection{Bijective Function}\label{sec:bijective_function}
% --------------------------------------------------------------------

A Function that is both Surjective and Injective is a \emph{Bijective
  Function} (or \emph{Bijection}). A Function is Bijective if and only
if it is also \emph{Invertible} (\S\ref{sec:inverse_function}).



\subsubsection{Involutory Function}\label{sec:involution}

$f(f(x)) = x$

An Identity Map is a trivial Involution.



% --------------------------------------------------------------------
\subsection{Indicator Function}\label{sec:indicator_function}
% --------------------------------------------------------------------

% --------------------------------------------------------------------
\subsection{Monotonic Function}\label{sec:monotonic}
% --------------------------------------------------------------------

A \emph{Monotonic Function} is a Function between Posets
(\S\ref{sec:poset}) where the Ordering of Elements of the Domain
Implies the Ordering of Elements in the Image of those Elements under
the Function.



% --------------------------------------------------------------------
\subsection{Function Space}\label{sec:function_space}
% --------------------------------------------------------------------

The \emph{Function Space} of two Sets $A$ and $B$ is the Set of all
Functions from $A$ to $B$ denoted by $B^A$.

When $B$ is a Field (\S\ref{sec:field}), Functions have a Vector
(\S\ref{sec:vector}) structure with two Pointwise Addition Operators
and Scalar Multiplication. %FIXME



\subsubsection{Evaluation Function}\label{sec:evaluation_function}

Given a Function Space $B^A$, the \emph{Evaluation Function} is
defined as:
\[
  eval : B^A \times A \rightarrow B
\]



% --------------------------------------------------------------------
\subsection{Equalizer}\label{sec:function_equalizer}
% --------------------------------------------------------------------

Given two Sets $X,Y$ and Functions $f,g : X \rightarrow Y$, the
\emph{Equalizer} of $f$ and $g$ is defined as:
\[
  Eq(f,g) = { x \in X | f(x) = g(x) }
\]

\emph{Coequalizer}



% --------------------------------------------------------------------
\subsection{Boolean-valued Function}\label{sec:boolean_function}
% --------------------------------------------------------------------

% --------------------------------------------------------------------
\subsection{Pairing Function}\label{sec:pairing_function}
% --------------------------------------------------------------------

A \emph{Pairing Function} is a Primitive Recursive
(\S\ref{sec:primitive_recursion}) Bijection:
\[
  \pi : \mathbb{N} \times \mathbb{N} \rightarrow \mathbb{N}
\]



\subsubsection{Cantor Pairing Function}\label{sec:cantor_pairing}



% ====================================================================
\section{Partial Function}\label{sec:partial_function}
% ====================================================================

A \emph{Partial Function} is a Functional Relation
(\S\ref{sec:functional_relation}) that is not Left-total
(\S\ref{sec:lefttotal_relation}).

$f : A \rightharpoondown B$



% --------------------------------------------------------------------
\subsection{Rice's Theorem}\label{sec:rices_theorem}
% --------------------------------------------------------------------

``For any non-trivial Property of Partial Functions, no General and
Effective method can Decide whether an Algorithm computes a Partial
Function with that Property.''



% ====================================================================
\section{Multimap}\label{sec:multimap}
% ====================================================================

A \emph{Multimap} (or \emph{Multi-valued Function}) is a Left-total
Relation (\S\ref{sec:lefttotal_relation}) that is not Right-unique
(\S\ref{sec:functional_relation}).



% ====================================================================
\section{Axiomatic Set Theory}\label{sec:axiomatic_set_theory}
% ====================================================================

% --------------------------------------------------------------------
\subsection{Axiom of Choice}\label{sec:choice_axiom}
% --------------------------------------------------------------------

\subsubsection{Tarski's Theorem}\label{sec:tarskis_theorem}

Well-ordering Theorem



% --------------------------------------------------------------------
\subsection{Axiom of Extensionality}\label{sec:extensionality_axiom}
% --------------------------------------------------------------------

\[
  \forall S \forall T
    [S = T \Leftrightarrow \forall R [ R \in S \Leftrightarrow R \in T ]]
\]
cf. \emph{Leibniz Law} (\S\ref{sec:equality}):
\[
  \forall S \forall T
    [S = T \Leftrightarrow \forall R [ S \in R \Leftrightarrow T \in R ]]
\]

% --------------------------------------------------------------------
\subsection{Axiom of Regularity}\label{sec:regularity_axiom}
% --------------------------------------------------------------------

% --------------------------------------------------------------------
\subsection{Zermelo-Fraenkel (ZFC)}\label{sec:zermelo_fraenkel}
% --------------------------------------------------------------------


% --------------------------------------------------------------------
\subsection{Kripke-Platek (KP)}\label{sec:kripke_platek}
% --------------------------------------------------------------------

% --------------------------------------------------------------------
\subsection{New Foundations (NF)}\label{sec:quine_foundations}
% --------------------------------------------------------------------

% --------------------------------------------------------------------
\subsection{Non-well-founded Set Theory}\label{sec:non_wellfounded}
% --------------------------------------------------------------------

Non-standard Analysis (\S\ref{sec:nonstandard_analysis})



% ====================================================================
\section{Set Algebra}\label{sec:set_algebra}
% ===================================================================

% ====================================================================
\section{Algebraic Set Theory}\label{sec:algebraic_set_theory}
% ===================================================================

% ====================================================================
\section{Descriptive Set Theory}\label{sec:descriptive_set_theory}
% ====================================================================

\emph{Boldface Borel Hierarchy} (\S\ref{sec:projective_hierarchy})

% --------------------------------------------------------------------
\subsection{Analytic Set}\label{sec:analytic_set}
% --------------------------------------------------------------------

% --------------------------------------------------------------------
\subsection{Polish Space}\label{sec:polish_space}
% --------------------------------------------------------------------

\subsubsection{Cantor Space}\label{sec:cantor_space}

\emph{Cantor Space}, $2^{\omega}$, is the Set of all Infinite
Sequences of $0$s and $1$s.



\subsubsection{Baire Space}\label{sec:baire_space}

\emph{Baire Space}, $\omega^{\omega}$ or $\mathcal{N}$, is the Set of
all Infinite Sequences of Natural Numbers.



% --------------------------------------------------------------------
\subsection{Pointclass}\label{sec:pointclass}
% --------------------------------------------------------------------

% --------------------------------------------------------------------
\subsection{Effective Descriptive Set Theory}
\label{sec:effective_descriptive}
% --------------------------------------------------------------------

Combination of Descriptive Set Theory with \emph{Recursion Theory}
(Part \ref{part:recursion_theory}).

% --------------------------------------------------------------------
\subsection{Borel Hierarchy}\label{sec:borel_hierarchy}
% --------------------------------------------------------------------



% ====================================================================
\section{Constructive Set Theory}\label{sec:constructive_set_theory}
% ====================================================================

% --------------------------------------------------------------------
\subsection{Apartness}\label{sec:apartness}
% --------------------------------------------------------------------

An \emph{Apartness Relation} is a Binary Relation $\#$ such that:

\begin{enumerate}
\item $\neg (x\#x)$
\item $x\#y \rightarrow y\#x$
\item $x\#y \rightarrow (x\#z \vee y\#z)$
\end{enumerate}

In Models of IZF all Sets with Apartness Relations are Subcountable
(\S\ref{sec:subcountable}).



\subsubsection{Tight}\label{sec:tight}

A \emph{Tight Apartness Relation} is an Apartness Relation that
additionally satisfies:
\[
  \neg (x \# y) \rightarrow x = y
\]



% --------------------------------------------------------------------
\subsection{IZF}\label{sec:izf}
% --------------------------------------------------------------------

% --------------------------------------------------------------------
\subsection{CZF}\label{sec:czf}
% --------------------------------------------------------------------



% ====================================================================
\section{Inner Model Theory}\label{sec:inner_model_theory}
% ====================================================================
