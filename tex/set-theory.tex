%%%%%%%%%%%%%%%%%%%%%%%%%%%%%%%%%%%%%%%%%%%%%%%%%%%%%%%%%%%%%%%%%%%%%%
%%%%%%%%%%%%%%%%%%%%%%%%%%%%%%%%%%%%%%%%%%%%%%%%%%%%%%%%%%%%%%%%%%%%%%
\part{Set Theory}\label{sec:set_theory}
%%%%%%%%%%%%%%%%%%%%%%%%%%%%%%%%%%%%%%%%%%%%%%%%%%%%%%%%%%%%%%%%%%%%%%
%%%%%%%%%%%%%%%%%%%%%%%%%%%%%%%%%%%%%%%%%%%%%%%%%%%%%%%%%%%%%%%%%%%%%%

\emph{Set Theory} is formulated within First-order Logic
(\S\ref{sec:predicate_logic}) and as such the objects of Set Theory
are \emph{Sets} and Propositions. See \emph{Axiomatic Set Theory}
(\S\ref{sec:axiomatic_set_theory}) for specific formulations of Set
theory.



% ====================================================================
\section{Set}\label{sec:set}
% ====================================================================

\emph{Sets} are collections of distinct objects (some of which may
themselves be other Sets), built on the Property of \emph{Membership}
which can be expressed as a \emph{Binary Relation}
(\S\ref{sec:binary_relation}) ``$\in$'' which is the \emph{Set
  Membership Relation}. This Relation Maps 2-tuples of generic objects
and Sets to True or False depending on whether the object is a
\emph{Member} (or \emph{Element}) of the Set, e.g. $x \in A$ is True
when the object $x$ is a Member of the Set $A$.

An Element that is specifically not a Set is called an
\emph{Individual} or \emph{Urelement}.

\emph{Extension}

\emph{Intension}



% --------------------------------------------------------------------
\subsection{Definition}\label{sec:set_definition}
% --------------------------------------------------------------------

\emph{Extensional Definition}

\emph{Intensional}

\emph{Identity} \emph{Determinable} \emph{Determinate}



\subsubsection{Property}\label{sec:set_property}

\emph{Properties} (or \emph{Predicables}) are objects that can be
Predicated (\S\ref{sec:predicate_logic}) of other objects. The
Predicate or \emph{Indicator Function}
(\S\ref{sec:indicator_function}) itself is not the Property, but the
\emph{Extension} of that Property. An Extension of a Property is a
\emph{Set}, but not every Property has an Extension as exemplified by
\emph{Russell's Paradox}.

\emph{Predicativity / Impredicativity} - Weyl

\emph{Vicious Circle Principle}



\paragraph{Axiom of Extensionality}\label{sec:extensionality}
\[
    \forall S \forall T
        [S = T \Leftrightarrow \forall R [ R \in S \Leftrightarrow R \in T ]]
\]
cf. \emph{Leibniz Law} (\S\ref{sec:equality}):
\[
    \forall S \forall T
        [S = T \Leftrightarrow \forall R [ S \in R \Leftrightarrow T \in R ]]
\]



\subsubsection{Recursive Definition}\label{sec:recursive_definition}

\emph{Recursive} = \emph{Inductive}

\begin{enumerate}
    \item Base case
    \item Inductive clause
    \item Extremal clause
\end{enumerate}

\emph{Structural Recursion}



\subsubsection{Circular Definition}\label{sec:circular_definition}

\emph{Homoiconic}



% --------------------------------------------------------------------
\subsection{Cardinality}\label{sec:cardinality}
% --------------------------------------------------------------------

\emph{Countable}

\emph{Uncountable}

\emph{Subcountable}

The unique Set with Cardinality 0 is called the \emph{Empty Set} and
is denoted $\{\}$ or $\varnothing$. A Set with Cardinality 1 is called a
\emph{Singleton Set}.



% --------------------------------------------------------------------
\subsection{Unary Set Operators}
% --------------------------------------------------------------------

% --------------------------------------------------------------------
\subsection{Binary Set Operators}
% --------------------------------------------------------------------

\subsubsection{Union}

\emph{Union}



\subsubsection{Disjoint Union}\label{sec:disjoint_union}

Given a Family (\S\ref{sec:family}) of Sets ${A_i : i \in I}$,
the \emph{Disjoint Union} is defined as:
\[
    \bigsqcup_{i \in I} A_i = \bigcup_{i \in I} {(x,i) : x \in A_i}
\]



\subsubsection{Infinitary Union}

\[
    x \in \bigcup S \leftrightarrow \exists y \in S : x \in y
\]



\subsubsection{Intersection}



\subsubsection{Subset}

\paragraph{Filter}

A \emph{Filter}, $\mathcal{F}$, of a Set, $X$, is a Set of Subsets of
$X$
\[
    \mathcal{F} \subseteq \mathcal{P}(X)
\]
where
\begin{itemize}
\item if $A \in \mathcal{F}$ and $B \supseteq A$, then $B \in
  \mathcal{F}$
\item if $A \in \mathcal{F}$ and $B \in \mathcal{F}$ then $A \cap B
  \in \mathcal{F}$
\item $X \in \mathcal{F}$ and $\emptyset \notin \mathcal{F}$
\end{itemize}
An \emph{Ultrafilter} $\mathcal{U}$ is a Filter such that for all $A
\subseteq X$, either $A \in \mathcal{U}$ or $X - A \in \mathcal{U}$.



\subsubsection{Difference}\label{sec:set_difference}

\emph{Complement}

\emph{Symmetric Difference}



\subsubsection{de Morgan's Law}\label{sec:de_morgan}

\subsubsection{Cartesian Product}\label{sec:set_product}



% --------------------------------------------------------------------
\subsection{Family}\label{sec:family}
% --------------------------------------------------------------------

A \emph{Family} is a collection of Sets.



% --------------------------------------------------------------------
\subsection{Partition}\label{sec:partition}
% --------------------------------------------------------------------

A \emph{Partition} of a Set $X$ is a Set of Non-empty Subsets of $X$
such that every Element of $X$ is in exactly one Subset, thus $X$ is
the Disjoint Union (\S\ref{sec:disjoint_union}) of the Subsets. A
Partition $P$ of $X$ has the Properties:
\begin{enumerate}
  \item $\emptyset \notin P$
  \item $\bigcup_{A \in P}A = X$
  \item $A,B \in P \wedge A \neq B \Rightarrow A \cap B = \emptyset$
\end{enumerate}



% --------------------------------------------------------------------
\subsection{Forcing}\label{sec:forcing}
% --------------------------------------------------------------------

% --------------------------------------------------------------------
\subsection{Equalizers}\label{sec:set_equalizer}
% --------------------------------------------------------------------

Given two Sets $X,Y$ and Functions $f,g : X \rightarrow Y$, the
\emph{Equalizer} of $f$ and $g$ is defined as
\[
    Eq(f,g) = { x \in X | f(x) = g(x) }
\]

\emph{Coequalizer}



% --------------------------------------------------------------------
\subsection{Index Sets}\label{sec:index_set}
% --------------------------------------------------------------------

An \emph{Index Set} is one that \emph{Indexes} (or \emph{Labels})
Members of another Set. Indexing is a Surjective Function
(\S\ref{sec:set_function}) from an Index Set onto a target Set.



\subsubsection{Indexed Family}\label{sec:indexed_family}

An \emph{Indexed Family} of Sets is a Function from an Index Set to
the Class of Sets. For Index Set $J$ and Indexed Set $A$, the Indexed
Family may be denoted
\[
    (A_j)_{j \in J}
\]



% --------------------------------------------------------------------
\subsection{Decidable Set}\label{sec:decidable_set}
% --------------------------------------------------------------------

\emph{Semi-decidable Set}



% --------------------------------------------------------------------
\subsection{Recursive Set}\label{sec:recursive_set}
% --------------------------------------------------------------------

\subsubsection{Recursively Enumerable Set}\label{sec:recursively_enumerable}



% --------------------------------------------------------------------
\subsection{Hyperarithmetic Set}\label{sec:hyperarithmetic_set}
% --------------------------------------------------------------------

\emph{Hyperarithmetic Sets} are the Class of Sets denoted as
$\Delta^1_1$ in the \emph{Analytic Hierarchy}
(\S\ref{sec:analytic_hierarchy}).



% --------------------------------------------------------------------
\subsection{Analytical Set}\label{sec:analytical_set}
% --------------------------------------------------------------------



% ====================================================================
\section{Transitive Set}\label{sec:transitive_set}
% ====================================================================

A Set, $S$, is \emph{Transitive} if and only if
\[
    \bigcup S \subseteq S
\]

For two Transitive Sets, $S$ and $T$, the Set $S \cup T \cup \{S,T\}$
is Transitive.

A Set, $S$, containing no Urelements is Transitive if and only if $S
\subset \mathcal{P}(X)$



% --------------------------------------------------------------------
\subsection{Admissible Set}\label{sec:admissible_set}
% --------------------------------------------------------------------

An \emph{Admissible Set}, $A$, is a Transitive Set such that $\langle
A, \in \rangle$ is a Model (\S\ref{sec:model_theory}) of Kripke-Platek
Set Theory (\S\ref{sec:kripke_platek}).

The smallest example of an Admissible Set is the Set of
\emph{Hereditarily Finite Sets}. %FIXME ref hereditarily finite



% ====================================================================
\section{Multiset}\label{sec:multiset}
% ====================================================================

A \emph{Multiset} (or \emph{Bag}) is a 2-tuple $(A,m)$ of an
\emph{Underlying Set}, $A$, together with a \emph{Multiplicity
  Function} (\S\ref{sec:set_function}), $m : A \rightarrow
\mathbb{N}_{\geq 1}$, mapping Elements of $A$ to Positive Natural
Numbers representing the \emph{Multiplicity} Elements, that is the
number of times an Element occurs in the Multiset. A Multiset may be
denoted with square brackets:
\[
    [a,a,b]
\]
If the Underlying Set is restricted to a Subset of a given
\emph{Universe}, $U$, the Multiplicity Function may be extended to
$m_U : U \rightarrow \mathbb{N}$ where $a \in U, a \notin A
\leftrightarrow m(a)=0$.

\emph{Indicator Function}



% ====================================================================
\section{Class}\label{sec:class}
% ====================================================================

A \emph{Class} is any Subset of the \emph{Universe}
(\S\ref{sec:set_universe}) of discussion.

\begin{description}
    \item [Proper Class] a Class that is not a Set
    \item [Small Class] a Class that is a Set
\end{description}



% ====================================================================
\section{Universe}\label{sec:set_universe}
% ====================================================================

% FIXME differentiate between Universes
A \emph{Universe} is a Set, $U$, with the following Closure Properties
\cite{maclane69}:
\begin{enumerate}
\item $x \in A \in U \rightarrow x \in U$
\item $x \in U \wedge y \in U \rightarrow \{x,y\}, \langle x,y
  \rangle, x \times y \in U$
\item $x \in U \rightarrow \mathcal{P}(x) \in U \wedge \bigcup x \in U$
\item $\omega = \{0,1,2,\ldots\} \in U$
\item Given a Surjective Function, $f : a \rightarrow b, a \in
  U, b \subset U \rightarrow b \in U$
\end{enumerate}
A \emph{Small Set} may be said to be a member of a Universe that is
not itself a Universe.



% --------------------------------------------------------------------
\subsection{Superstructure}\label{sec:superstructure}
% --------------------------------------------------------------------

An Universe may be generated over a Set resulting in a
\emph{Superstructure}. The Superstructure over a Set $X$:
\[
    \mathbf{S}X := \bigcup^{\infty}_{i=0}\mathbf{S}_i X
\]
can be defined by Structural Recursion
(\S\ref{sec:recursive_definition}) as follows:
\begin{itemize}

\item $\mathbf{S}_0 X = X$
\item $\mathbf{S}_1 X = X \cup \mathcal{P}(X)$
\item $\mathbf{S}_{n+1} X =
    \mathbf{S}_n X \cup \mathcal{P}(\mathbf{S}_n X)$

\end{itemize}
Some Properties of $\mathbf{S}\{\}$ (the Superstructure over the Empty
Set):
\begin{itemize}

\item $\mathbb{N} \subset \mathbf{S}\{\}$
\item $\mathbb{N} \notin \mathbf{S}\{\}$ (Elements of $\mathbf{S}\{\}$
  are Finite Sets)
\item $\mathbf{S}\{\}$ contains all of the Hereditarily Finite Sets
%FIXME ref Hereditarily finite sets

\end{itemize}

The Superstructure over $\mathbb{N}$, $\mathbf{S}\mathbb{N}$, is
considered the \emph{Universe of Ordinary Mathematics}.



% --------------------------------------------------------------------
\subsection{Von Neumann Universe}\label{sec:vonneumann_universe}
% --------------------------------------------------------------------

% --------------------------------------------------------------------
\subsection{Grothendieck Universe}\label{sec:grothendieck_universe}
% --------------------------------------------------------------------

%FIXME similar to definitions above



% ====================================================================
\section{Map}\label{sec:set_map}
% ====================================================================

A \emph{Map} (or \emph{Mapping}) is a Relation or \emph{Morphism}
(\S\ref{sec:morphism}) that is either a \emph{Function}
(\S\ref{sec:set_function}) or \emph{Multimap} (\S\ref{sec:multimap}),
depending on whether the Mapping is Determinate or not.

Maps refer to the Extensional Definition of Functions (a Subset of the
Cartesian Product of the Domain and Codomain), while the underlying
Function may or may not be Extensionally Defined.



% ====================================================================
\section{Function}\label{sec:set_function}
% ====================================================================

A \emph{Function} $f$, defined for all Members of Set $A$ and $image(f)
\subseteq B$ is written
\[
    f : A \rightarrow B
\]
The \emph{Input} or \emph{Domain} of $f$ is denoted $dom(f)$ and the
\emph{Ouptut} or \emph{Co-domain} as $cod(f)$. The \emph{Image} of the
Function is the Subset of the Codomain that the Function actually Maps
to, and the Subset of the Domain that Maps to it is called the
\emph{Inverse Image} or \emph{Preimage}. For an Element of a Set $Y$,
$y \in Y$, under a Map $f: X \rightarrow Y$, the \emph{Fiber} of $y$
is the Preimage of the Singleton Set $\{y\}$.

Each Element of the Preimage Maps to one Element of the Image. This
Property is expressed as
\[
    (a,b) \in f \wedge (a,c) \in f \rightarrow b = c
\]
A Relation without this Property is known as a \emph{Multimap}
(\S\ref{sec:multimap}).

In the case where $image(f) = cod(f)$, $f$ is known as a
\emph{Surjective Function} (\S\ref{sec:surjective_function}).

\emph{Function Equality} is defined such that Equal Functions have the
same Output for a given Input. A Function can be seen as a Subset of
the Cartesian Product (\S\ref{sec:set_product}) of the Domain and
Codomain:
\[
    f \subseteq A \times B
\]

Given a second Function $g : B \rightarrow C$, there is a
\emph{Composite Function}
\[
    g \circ f : A \rightarrow C
\]
where $(g \circ f)(a) = g(f(a))$ and $a \in A$. The \emph{Composition
  Operation} $\circ$ is Associative: $(h \circ g) \circ f = h \circ (g
\circ f)$. Composition of Functions may be represented with the
$\circ$ elided: $gf$.

%FIXME ref unit operation?
\emph{Unit Operations} for the Composition Operation are
\emph{Identity Functions} on Sets. The Identity Function for a Set $A$
\[
    I_A : A \rightarrow A
\]
is defined as
\[
    I_A(a) = a
\]
with the result given the function $f$ above
\[
    f \circ I_A = f = I_B \circ f
\]

A \emph{Fixed Point}, $c$, is an Element of the Domain Function that
is Mapped to itself by the Function:
\[
    f(c) = c
\]

A \emph{Periodic Point} is an Element of the Domain of a Function that
is returned to after a finite number of iterations.

An \emph{Empty Function} has the Empty Set as a Domain, defining a
Unique Function for each Set, $A$:
\[
    f_A : \varnothing \rightarrow A
\]

\emph{Finitary Function}

\emph{Infinitary Function}



% --------------------------------------------------------------------
\subsection{Injective Function}\label{sec:injective_function}
% --------------------------------------------------------------------

An \emph{Injective Function} (or \emph{One-to-one Function}) is one
where the Elements of the Codomain are the Images of at most one
Elements of the Domain. A Function that is Non-injective is considered
an \emph{Information Losing Function} because the Inverse is no longer
a Function but it is a \emph{Multimap} (\S\ref{sec:multimap}).



% --------------------------------------------------------------------
\subsection{Surjective Function}\label{sec:surjective_function}
% --------------------------------------------------------------------

A \emph{Surjective Function} is one such that the Image is equal to
the Codomain. Such a Function may be said to be \emph{Onto} but is not
necessarily \emph{Injective}.



% --------------------------------------------------------------------
\subsection{Bijective Function}\label{sec:bijective_function}
% --------------------------------------------------------------------

A \emph{Bijective Function} is one that is both Surjective and
Injective (both One-to-one and Onto). A Function is Bijective if and
only if it is also \emph{Invertible}.



% --------------------------------------------------------------------
\subsection{Inverses}\label{sec:inverse_functions}
% --------------------------------------------------------------------

For a Mapping between sets, $f: X \rightarrow Y$, a \emph{Left
  Inverse} or \emph{Retraction} of $f$ is a Map $g: Y \rightarrow X$
such that $gf = Id(X)$. A \emph{Right Inverse} or \emph{Section} of
$f$ would be a Map $h: Y \rightarrow X$ such that $fh = Id(Y)$.



% --------------------------------------------------------------------
\subsection{Idempotence}\label{sec:idempotence}
% --------------------------------------------------------------------

A Function, $f$, is \emph{Idempotent} if $f^2 = f$.



% --------------------------------------------------------------------
\subsection{Partial Function}\label{sec:partial_function}
% --------------------------------------------------------------------

A \emph{Total Function} is one where the Function is defined for every
Element of the Domain, while a \emph{Partial Function} the Function
may be defined for only a Subset of the Domain called the \emph{Domain
  of Definition}.



% --------------------------------------------------------------------
\subsection{Monotonicity}\label{sec:monotonicity}
% --------------------------------------------------------------------

A \emph{Monotonic Function} is a Function between \emph{Posets}
(\S\ref{sec:order_theory}) where the Ordering of Elements of the
Domain Imply the Ordering of Elements in the Image of those Elements
under the Function.



% --------------------------------------------------------------------
\subsection{Inclusion Map}\label{sec:inclusion_map}
% --------------------------------------------------------------------

An \emph{Inclusion Map} or \emph{Inclusion Function} is an Identity
Function that Maps Elements of a Subset to those in a Superset
\[
    \iota : A \hookrightarrow X
\]
where $A \subseteq X$.



% --------------------------------------------------------------------
\subsection{Pointed Sets}
% --------------------------------------------------------------------

A \emph{Pointed Set} is one that has a \emph{Null Element}
(\S\ref{sec:universal_property}).



% --------------------------------------------------------------------
\subsection{Indicator Functions}\label{sec:indicator_function}
% --------------------------------------------------------------------

% --------------------------------------------------------------------
\subsection{Function Space}\label{sec:function_space}
% --------------------------------------------------------------------

The \emph{Function Space} of two Sets $A$ and $B$ is the Set of all
Functions from $A$ to $B$ denoted by $B^A$.



\subsubsection{Evaluation Function}\label{sec:evaluation_function}

Given a Function Space $B^A$, the \emph{Evaluation Function} is
defined as:
\[
    eval : B^A \times A \rightarrow B
\]



% --------------------------------------------------------------------
\subsection{Pairing Function}\label{sec:pairing_function}
% --------------------------------------------------------------------

A \emph{Pairing Function} is a Primitive Recursive
(\S\ref{sec:primitive_recursion}) Bijection:
\[
    \pi : \mathbb{N} \times \mathbb{N} \rightarrow \mathbb{N}
\]



% ====================================================================
\section{Relations}\label{sec:set_relation}
% ====================================================================

A \emph{Relation} is a Mapping between Truth Values and Tuples of
individuals. The general definition for an $n$-ary Relation $L$ over
Sets $X_1, \ldots, X_n$ is a Subset of the Cartesian Product $X_1
\times \ldots \times X_n$ called the \emph{Graph} of $L$:
\[
    G(L) \subseteq X_1 \times \ldots \times X_n
\]
The Expression $L x_1 \ldots x_n$ where $x_i \in X_i$ is True when
$(x_1, \ldots, x_n) \in G(L)$ and False otherwise. Such an $n$-ary
Relation may be completely specified by an $n + 1$-tuple called a
\emph{Correspondence}:
\[
    (X_1, \ldots, X_n, G(L))
\]
Such an $n$-ary Relation may also be described by a
\emph{Characteristic Function} (or \emph{Indicator Function}): %FIXME
                                %ref indicator function
\[
    f_L : X_1 \times \ldots \times X_n \rightarrow \{\top,\bot\}
\]



% --------------------------------------------------------------------
\subsection{Binary Relation}\label{sec:binary_relation}
% --------------------------------------------------------------------

Also \emph{Dyadic Relation}. Given a \emph{Binary Relation} $R$ on a
Set $X$, $R$ is
\begin{description}
\item[Serial] when
\[
    \forall a \in X \exists b \in X : aRb
\]
\item[Reflexive] when
\[
    aRa = \top
\]
Reflexive implies Transitive and Serial
\item[Irreflexive] (also \emph{Strict}) when
\[
    aRa = \bot
\]
\item[Co-reflexive] when
\[
    aRb \rightarrow a = b
\]
\item[Transitive] when
\[
    aRb \wedge bRc \rightarrow aRc
\]
Transitive and Irreflexive if and only if Transitive and Asymmetric
\item[Symmetric] when
\[
    aRb \leftrightarrow bRa
\]
\item[Anti-symmetric] when
\[
    aRb \wedge bRa \rightarrow a = b
\]
\item[Asymmetric] when both Anti-symmetric and Irreflexive
\[
    aRb \rightarrow \neg bRa
\]
\item[Left-total] when
\[
    \forall a \in X \exists b \in X : aRb
\]
\emph{Functions} (\S\ref{sec:set_function}) are Left-total Relations
\item[Right-total] (Surjective or Onto
  (\S\ref{sec:surjective_function})) when
\[
    \forall b \in X \exists a \in X : aRb
\]
\item[Total] when
\[
    \forall a,b \in X, aRb \vee bRa
\]
Total implies Reflexive.
\item[Trichotomous] when
\[
    aRb \vee bRa \vee a = b
\]
Trichotomous implies Irreflexive.
Trichotomous and Transitive implies Asymmetric.
A Transitive Trichotomous Relation is a Strict Total Order.
\item[Right Euclidean] when
\[
    aRb \wedge aRc \rightarrow bRc
\]
\item[Left Euclidean] when
\[
    bRa \wedge cRa \rightarrow bRc
\]
\end{description}
From the above types of Relations, the following Orders
(\S\ref{sec:order_theory}) are distinguished, listed here from most
general to most restricted:
\begin{description}
\item[Preorder] (or Quasiorder) when Reflexive and Transitive (all
  Partial Orders and Equivalence Relations
  (\S\ref{sec:equivalence_relation}) are Preorders)
\item[Partial Order] when Preorder and Anti-symmetric (Poset
  \S\ref{sec:poset})
\item[Total Preorder] (or Weak Order) when Preorder and Total.
\item[Total Order] when a Partial Order and Total.
\item[Partial Equivalence] when Symmetric and Transitive.
\item[Equivalence] when Reflexive, Symmetric, and Transitive
  (\S\ref{sec:equivalence_relation}).
\end{description}

The Set of all Binary Relations on a Set $X$ is denoted
$\mathbf{Rel}(X)$ and is the Power Set of $X \times X$: $2^{X \times
  X}$.

Functions (\S\ref{sec:set_function}) are Well-defined Binary Relations.

A Dependency Relation is a Binary Relation that is Symmetric and
Reflexive.

Undirected Graphs (\S\ref{sec:undirected_graph}) are Symmetric.
Any Binary Relation may be a Directed Graph
(\S\ref{sec:directed_graph}). A Binary Relation is a Complete Graph
(\S\ref{sec:complete_graph}) when $a \neq b \rightarrow aRb$ and
implies Symmetry. A Binary Relation is a Tournament when $a \neq b
\rightarrow aRb \vee bRa$ and implies Anti-symmetry.



\subsubsection{Equivalence Relation}\label{sec:equivalence_relation}

An \emph{Equivalence Relation} on a Set $X$ is a Reflexive, Symmetric,
and Transitive Relation that Partitions (\S\ref{sec:partition}) the
Set into Disjoint Subsets called \emph{Equivalence Classes}.

The Equivalence Class of an Element $a \in X$ with Equivalence
Relation $\sim$ is the Subset of $X$ defined as:
\[
    [a] = \{x \in X | a \sim x\}
\]
The \emph{Quotient Set} is the Set of Equivalence Classes of a
particular Equivalence Relation.

For a Function between Sets $f : S \rightarrow T$, one may define an
Equivalence Relation on Elements $a,b \in S$:
\[
    a \sim b \Leftrightarrow f(a) = f(b)
\]
where the Equivalence Classes are the Fibers
(\S\ref{sec:set_function}) of the Elements in $T$ and the Quotient Set
is the Image of the Equivalence Relation viewed as a Function on
Elements of $S$ to their Equivalence Classes.



\subsubsection{Congruence Relation}\label{sec:congruence_relation}



\subsubsection{Ordering Relation}\label{sec:ordering_relation}

\emph{Partial Order}

The Signature (\S\ref{sec:signature}) of Partial Orderings is
$\{\leq\}$.

\emph{Total Order}



\subsubsection{Rewrite Relation}\label{sec:rewrite_relation}

\emph{Rewrite Relation}

\emph{Rewrite Preorder}

\emph{Reduction Preorder}

\emph{Rewrite Closure}



\subsubsection{Reflexive Reduction}\label{sec:reflexive_reduction}

The \emph{Irreflexive Kernel} or \emph{Reflexive Reduction} of a
Binary Relation $R$ on a Set $S$ is the smallest Relation $Y$
such that $Y$ has the same Reflexive Closure as $R$:
\[
    Y = (R\;\setminus=)
\]
The Reflexive Reduction of $x \leq y$ is $x < y$.



\subsubsection{Well-founded Relation}\label{sec:well_founded}

\emph{Well-founded}



% --------------------------------------------------------------------
\subsection{Relation Join}\label{sec:relation_join}
% --------------------------------------------------------------------

\emph{Relational Algebra} (\S\ref{sec:relational_algebra})

\emph{Pullback} (\S\ref{sec:category_pullback})



\subsubsection{Relation Composition}\label{sec:relation_composition}

For two Binary Relations, $R \subseteq X \times Y$ and $S \subseteq Y
\times Z$, the \emph{Relation Composition}, $S \circ R \in X \times Z$
is defined as:
\[
    S \circ R = \{(x,z) \in X \times Z \;|\;
    \exists y \in Y : (x,y) \in R \wedge (y,z) \in S \}
\]



\subsubsection{Function Composition}\label{sec:function_composition}



% --------------------------------------------------------------------
\subsection{Closure}\label{sec:set_closure}
% --------------------------------------------------------------------

A Set, $S$, is \emph{Closed} (has \emph{Closure}) under an Operation,
$\cdot$, if for every Input from $S$, the Output of the Operation is
also in $S$:
\[
    \forall x \in S : (x,y) \in \cdot \Rightarrow y \in S
\]

\subsubsection{Transitive Closure}\label{sec:transitive_closure}

For a Relation, $R$, on a Set, $X$, the \emph{Transitive Closure}
$R^+$ is a Relation on $X$ such that $R \subseteq R^+$ and $R^+$ is
\emph{Minimal} (the smallest Relation closed under Relation
Composition):
\[
    R^+ = \bigcup_{i \in \{1,2,3,...\}} R^i
\]
If $R$ is Transitive then $R = R^+$.

\emph{Reachability} (\S\ref{sec:dag})



\subsubsection{Reflexive Closure}\label{sec:reflexive_closure}

The \emph{Reflexive Closure}, $X$, of a Binary Relation, $R$, on a
Set, $S$, is the smallest Reflexive Relation on $S$ that contains $R$:
\[
    X = R \cup \{(x,x) : x \in S\}
\]
that is, the Union of $R$ with the Equivalence Relation $=$.

The Reflexive Closure of $<$ is $\leq$.



\subsubsection{Symmetric Closure}\label{sec:symmetric_closure}

The \emph{Symmetric Closure}, $X$, of a Binary Relation, $R$, on a
Set, $S$, is the Union of $R$ with its Inverse Relation:
\[
    R \cup R^{-1} = X = R \cup \{(x,y) : (y,x) \in R\}
\]



\subsubsection{Finitary Closure}\label{sec:finitary_closure}



\subsubsection{Rewrite Closure}\label{sec:rewrite_closure}



% ====================================================================
\section{Multimap}\label{sec:multimap}
% ====================================================================

A \emph{Multimap} (or \emph{Multi-valued Function}) is a Left-total
Relation.



% ====================================================================
\section{Axiomatic Set Theory}\label{sec:axiomatic_set_theory}
% ====================================================================

% --------------------------------------------------------------------
\subsection{Zermelo-Fraenkel (ZFC)}\label{sec:zermelo_fraenkel}
% --------------------------------------------------------------------

% --------------------------------------------------------------------
\subsection{Kripke-Platek (KP)}\label{sec:kripke_platek}
% --------------------------------------------------------------------

% --------------------------------------------------------------------
\subsection{New Foundations (NF)}\label{sec:quine_foundations}
% --------------------------------------------------------------------



% ====================================================================
\section{Order Theory}\label{sec:order_theory}
% ====================================================================

\emph{Ordering Relations} are Binary Relations on a Set $P$ that are
Reflexive, Antisymmetric, and Transitive
(\S\ref{sec:binary_relation}). A \emph{Partial Order} is where some
pairs of Members are allowed to not be included in the Ordering
Relation. A \emph{Total Order} (or \emph{Linear Order}) adds the
requirement of Left-totality. A Totally Ordered Subset of some
Partially Ordered Set is called a \emph{Chain}.

\emph{Weak Ordering}

\emph{Top Element}

\emph{Lexicographic Order}

A \emph{Strict Order} is an Irreflexive Partial Order (e.g. $<$).
There is a one-to-one correspondence between Strict and Non-strict
Partial Orders via the \emph{Irreflexive Kernel}:
\[
    a \leq b \wedge a \neq b \Rightarrow a < b
\]
and conversely the \emph{Reflexive Closure}:
\[
    a < b \vee a = b \Rightarrow a \leq b
\]



% --------------------------------------------------------------------
\subsection{Proset}\label{sec:proset}
% --------------------------------------------------------------------

A \emph{Proset} is a Set, $P$, with a Preorder
(\S\ref{sec:binary_relation}), $\lesssim$, defined on that Set.



% --------------------------------------------------------------------
\subsection{Poset}\label{sec:poset}
% --------------------------------------------------------------------

A \emph{Poset} is a Set, $P$, with a Partial Order
(\S\ref{sec:binary_relation}), $\leq$, defined on that Set (that is
a Relation which is Reflexive, Antisymmetric, and Transitive).

An Element $x \in P$ is a \emph{Greatest Element} when:
\[
    \forall y \in P, y \leq x
\]
and likewise the \emph{Least element} when:
\[
    \forall y \in P, x \leq y
\]

An Element $x \in P$ is a \emph{Maximal Element} when:
\[
    \forall y \in P, x \nleq y
\]
and a \emph{Minimal Element} when:
\[
    \forall y \in P, y \nleq x
\]

The definition of Posets gives rise to the concept of a Monotonic
Function (\S\ref{sec:monotonicity}), that is a Function, $f$,
between Posets which is ordered in both the Domain and the Codomain:
\[
    x \leq y \Rightarrow f(x) \leq f(y)
\]

A Poset satisfies the \emph{Ascending Chain Condition} if every
strictly ascending sequence of Elements eventually terminates. The
\emph{Descending Chain Condition} is defined likewise for descending
sequences of Elements.



\subsubsection{Totally Ordered Set}\label{sec:total_order}

A \emph{Totally Ordered Set} is a Poset whose Ordering Relation is
Total.

A \emph{Well-order} is a Total Order on a Set $S$ such that every
non-empty Subset of $S$ has a Least Element.



\subsubsection{Upper \& Lower Sets}\label{sec:upper_lower}

An \emph{Upper Set} (or \emph{Upset}) is a Subset, $U$, of a Poset,
$P$, such that:
\[
    x \in U \wedge x \leq y \Rightarrow y \in U
\]
The Dual notion of a \emph{Lower Set} (or \emph{Downset} or
\emph{Decreasing Set}), $L$, is defined likewise:
\[
    y \in L \wedge x \leq y \Rightarrow x \in L
\]
The Upper or Lower Set generated from a Subset $Y \subseteq X$ is
denoted by $\uparrow Y$ or $\downarrow Y$, respectively.

An Upper Set has the Property of being \emph{Upward Closed} and a
Lower Set the Property of being \emph{Downward Closed}.



\subsubsection{Infimum \& Supremum}\label{sec:glb_lub}

The \emph{Infimum} (or \emph{Greatest Lower Bound} (GLB) or
\emph{Meet}) of a Subset of a Poset, $S \subseteq P$, is the Greatest
Element of $P$ that is a Least Element of $S$, denoted $\wedge S$.

The \emph{Supremum} (or \emph{Least Upper Bound} (LUB) or \emph{Join})
of a Subset of a Poset, $S \subseteq P$ is the Least Element of $P$
that is a Greatest Element of $S$, denoted $\vee S$.

Such Elements are Unique but not necessarily existing.

Meet and Join can be defined as Commutative, Associative, and
Idempotent Partial Binary Operations on pairs of Elements of $P$.

The Join/Meet of a Totally Ordered Set is its Maximal/Minimal Element.



\subsubsection{Poset Product}\label{sec:poset_product}

The Product of two Elements of a Poset is their Greatest Lower Bound.

The Coproduct of two Elements of a Poset is their Least Upper Bound.



% --------------------------------------------------------------------
\subsection{Order Isomorphism}\label{sec:order_isomorphism}
% --------------------------------------------------------------------



% --------------------------------------------------------------------
\subsection{Lattice Theory}\label{sec:lattice_theory}
% --------------------------------------------------------------------

\subsubsection{Lattice}\label{sec:lattice}

A \emph{Lattice} is a Poset for which every Element has both a Join
and a Meet. For a Lattice with an underlying Set $L$, the Algebraic
Structure (\S\ref{sec:universal_algebra}) $(L, \vee, \wedge)$ has
these Axiom Identities for all $a,b,c \in L$:
\begin{description}
\item[Commutative laws]
\[
    a \vee b = b \vee a
\] \[
    a \wedge b = b \wedge a
\]
\item[Associative laws]
\[
    a \vee (b \vee c) = (a \vee b) \vee c
\] \[
    a \wedge (b \wedge c) = (a \wedge b) \wedge c
\]
\item[Absorption laws]
\[
    a \vee (a \wedge b) = a
\] \[
    a \wedge (a \vee b) = a
\]
\end{description}
The Signature (\S\ref{sec:signature}) of Lattices is
$\{\wedge, \vee\}$.




\subsubsection{Semilattice}\label{sec:semilattice}

A \emph{Join-semilattice} is a Poset for which all pairs have a Join.

A \emph{Meet-semilattice} is a Poset for which all pairs have a Meet.



\subsubsection{Complete Lattice}\label{sec:complete_lattice}

A \emph{Complete Lattice} is a Lattice for which every Subset has a
Meet and a Join.



\subsubsection{Free Lattice}



% --------------------------------------------------------------------
\subsection{Order Ideal}\label{sec:order_ideal}
% --------------------------------------------------------------------



% --------------------------------------------------------------------
\subsection{Domain Theory}\label{sec:domain_theory}
% --------------------------------------------------------------------



% ====================================================================
\section{Sequence}\label{sec:sequence}
% ====================================================================

A \emph{Sequence} can be defined as a Countable Totally Ordered
Multiset (\S\ref{sec:multiset}), that is, a collection of Elements (or
\emph{Terms}) in a given order where duplicate Elements are allowed.
The number of Elements in a Sequence is referred to as its
\emph{Length}.

A \emph{Finite Sequence} is called a \emph{Tuple} (\S\ref{sec:tuple}).



% --------------------------------------------------------------------
\subsection{Tuple}\label{sec:tuple}
% --------------------------------------------------------------------

String (\S\ref{sec:string})


% --------------------------------------------------------------------
\subsection{Infinite Sequence}\label{sec:infinite_sequence}
% --------------------------------------------------------------------

\emph{Infinite Sequences} may be \emph{Singly Infinite}
(\S\ref{sec:singly_infinite}), having an initial Element but no final
Element, or \emph{Doubly Infinite} (\S\ref{sec:doubly_infinite})
having neither a first nor a last Element.



\subsubsection{Singly Infinite Sequence}\label{sec:singly_infinite}

A \emph{Singly Infinite Sequence} (or \emph{One-sided Infinite
  Sequence}) can be defined as a Function, $s$, with a Countably
Infinite Totally Ordered Set of Indices, $X$, for its Domain and a Set
of Elements, $Y$, for the Codomain:

  $s : X \rightarrow Y$ \\
where:

  $X = \{1,2,\ldots,n\}$

  $Y = \{a_1, a_2,\ldots,a_n\}$

  $s = \{(1,a_1), (2,a_2),\ldots, (n,a_n)\}$ \\
for some Countable $n \geq 0$.

Singly Infinite Sequences may be interpreted as Elements of the
Semigroup Ring of the Natural Numbers, $R[\mathbb{N}]$
(\S\ref{sec:group_ring}).



\subsubsection{Doubly Infinite Sequence}\label{sec:doubly_infinite}

A \emph{Doubly Infinite Sequence} (also \emph{Two-way Infinite} or
\emph{Bi-infinite Sequence}) may be defined as a Function from the Set
of all Integers $\mathbb{Z}$ into a Set, denoted
$(2n)^{\infty}_{n=-\infty}$.

Doubly Infinite Sequences may be interpreted as Elements of the Group
Ring of the Integers, $R[\mathbb{Z}]$ (\S\ref{sec:group_ring}.



% ====================================================================
\section{Descriptive Set Theory}\label{sec:descriptive_set_theory}
% ====================================================================

\emph{Boldface Borel Hierarchy} (\S\ref{sec:projective_hierarchy})

% --------------------------------------------------------------------
\subsection{Analytic Set}\label{sec:analytic_set}
% --------------------------------------------------------------------

% --------------------------------------------------------------------
\subsection{Polish Space}\label{sec:polish_space}
% --------------------------------------------------------------------

\subsubsection{Cantor Space}

\emph{Cantor Space}, $2^{\omega}$, is the Set of all Infinite
Sequences of $0$s and $1$s.



\subsubsection{Baire Space}

\emph{Baire Space}, $\omega^{\omega}$ or $\mathcal{N}$, is the Set of
all Infinite Sequences of Natural Numbers.



% --------------------------------------------------------------------
\subsection{Pointclass}\label{sec:pointclass}
% --------------------------------------------------------------------

% --------------------------------------------------------------------
\subsection{Effective Descriptive Set Theory}
% --------------------------------------------------------------------

Combination of Descriptive Set Theory with \emph{Recursion Theory}
(Part \ref{sec:recursion_theory}).

% --------------------------------------------------------------------
\subsection{Borel Hierarchy}\label{sec:borel_hierarchy}
% --------------------------------------------------------------------



% ====================================================================
\section{Combinatorics}\label{sec:combinatorics}
% ====================================================================

% --------------------------------------------------------------------
\subsection{Permutation}\label{sec:permutation}
% --------------------------------------------------------------------

\subsubsection{Cyclic Permutation}\label{sec:cyclic_permutation}

A Permutation is \emph{Cyclic} if and only if it consists of a single
Non-trivial Cycle (Cycle Length > 1).

\paragraph{Transposition}\label{sec:transposition}

A \emph{Transposition} is a Cyclic Permutation of length 2 which
exchanges two Elements and leaves all other Elements fixed.



% --------------------------------------------------------------------
\subsection{Matroid}\label{sec:matroid}
% --------------------------------------------------------------------

\subsubsection{Antimatroid}\label{sec:antimatroid}



% --------------------------------------------------------------------
\subsection{Ramsey's Theorem}\label{sec:ramseys_theorem}
% --------------------------------------------------------------------



% ====================================================================
\section{Constructive Set Theory}\label{sec:constructive_set_theory}
% ====================================================================
